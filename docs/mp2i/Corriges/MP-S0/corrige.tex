\PassOptionsToPackage{dvipsnames,table}{xcolor}
\documentclass[11pt,a4paper]{article}

\usepackage{DS}

\begin{document}
\input{\detokenize{/home/fenarius/Travail/Cours/cpge-info/latex/Macros.tex}}
\ModeExercice
\PC{MP2I}{Mines-Pont sujet zéro}{2023}
\newcommand{\SPATH}{\FPATH Corrigés/MP-S0/}
\newcommand{\maillon}[3]{
	\begin{tabular}{|p{0.6cm}|p{0.6cm}|}
		\hline
		\rnode{#2}{#1} & \rnode{#3}{\phantom{$e_0$}} \\
		\hline
	\end{tabular}
}

\mq{Ecriture de la fonction d'initialisation}
\inputpartC{sol.c}{}{}{31}{40}

\mq{Ecriture de la fonction de localisation}
\inputpartC{sol.c}{}{}{43}{51}

\mq{Jeu de test pour l'insertion}
\tcor{L'énoncé indique que les valeurs à insérer sont toujours \textit{strictement} comprises entre {\sc int\_min} et {\sc int\_max}. Si on part de la liste de départ donnée dans l'énoncé : \\
\maillon{{$-\infty$}}{v0}{p0} \quad \maillon{-7}{v1}{p1} \quad \maillon{-2}{v2}{p2} \quad \maillon{9}{v3}{p3} \quad \maillon{$+\infty$}{v4}{p4}
\ncline[nodesepB=0.2cm,offsetB=-0.05cm]{*->}{p0}{v1}
\ncline[nodesepB=0.2cm,offsetB=-0.05cm]{*->}{p1}{v2}
\ncline[nodesepB=0.2cm,offsetB=-0.05cm]{*->}{p2}{v3}
\ncline[nodesepB=0.2cm,offsetB=-0.01cm]{*->}{p3}{v4}
\\
On a donc pas besoin de tester l'insertion en toute fin ou en en tout debut les listes, les seuls cas à tester sont  :
\begin{itemize}
    \item L'insertion d'un entier $n$ qui ne figure pas encore dans la liste (valeur de retour {\tt true}). Par, exemple, l'insertion de 5 qui doit donner : \\
    \maillon{{$-\infty$}}{w0}{q0} \quad \maillon{-7}{w1}{q1} \quad \maillon{-2}{w2}{q2} \quad \maillon{5}{w3}{q3} \quad \maillon{9}{w4}{q4} \quad \maillon{$+\infty$}{w5}{q5}
\ncline[nodesepB=0.2cm,offsetB=-0.05cm]{*->}{q0}{w1}
\ncline[nodesepB=0.2cm,offsetB=-0.05cm]{*->}{q1}{w2}
\ncline[nodesepB=0.2cm,offsetB=-0.05cm]{*->}{q2}{w3}
\ncline[nodesepB=0.2cm,offsetB=-0.05cm]{*->}{q3}{w4}
\ncline[nodesepB=0.2cm,offsetB=-0.01cm]{*->}{q4}{w5}
\\
    \item L'insertion d'un entier déjà présent dans la liste (valeur de retour {\tt false}). Par exemple, l'insertion de 9 qui ne modifie pas la liste et renvoie {\tt false}.
\end{itemize}
    }

\mq{Message d'erreur du compilateur et première correction}
\tcor{La fonction {\tt localise} prend en paramètre un pointeur sur un struct {\tt maillon\_t}, ici on passe l'adresse d'un objet de ce type car on utilise l'opérateur {\tt \&} qui renvoie l'adresse du maillon, ce qui cause l'erreur. On propose la première correction suivante :}
\inputpartC{sol.c}{}{}{56}{56}

\mq{Correction de la fonction d'insertion}
\tcor{La fonction {\tt insere} ne prend pas en compte le cas où la valeur est déjà présente dans la liste. On peut aussi éventuellement testé si {\tt malloc} échoue.}
\inputpartC{sol.c}{}{}{54}{67}

\mq{Fonction de suppression d'un maillon}
\inputpartC{sol.c}{}{}{70}{80}

\mq{Complexité des fonctions insere et supprime}
\tcor{Dans les deux cas, on fait appel à {tt localise} qui parcourt dans le pire des cas la totalité des éléments de la liste. Les autres opérations s'effectuent en temps constant. La complexité est donc en $O(n)$ où $n$ désigne le nombre d'éléments de la liste.}

\mq{Modèle mémoire}
\tcor{Les différentes régions sont :
\begin{itemize}
	\item La zone des données  qui contient les données dont la taille est connue à la compilation (notamment les variables globales).
	\item La pile qui contient les paramètres d'appels à une fonction ainsi que les variables locales à celle ci
	\item Le tas qui contient les variables alloués dynamiquement par le programme via {\tt malloc}.
\end{itemize}
}
Dans le cas de ce programme, la valeur entière 717, sera stocké dans la zone de données (car {\tt v} est une variable globale initialisée à cette valeur), dans la pile lors de l'appel à {\tt insere(t, v)}, car les paramètres sont passés par valeur et donc copié. Et enfin dans le tas car l'insertion dans la liste crée un maillon contenant la valeur de V donc 717.
\end{document}
\documentclass[11pt,a4paper]{article}

\usepackage{Act}
\begin{document}
\input{\detokenize{/home/fenarius/Travail/Cours/cpge-info/latex/Macros.tex}}
\ModeExercice
\setboolean{corrige}{false}
\IC{1}{Ligne de commande - Premiers pas en C}
\setcounter{Exercise}{0}

\begin{Exercise}[title={un peu de ligne de commande}]\\
Dans un dossier {\tt A\_ranger} de son répertoire utilisateur, Alice a notamment des fichiers dont les noms se terminent par   {\tt .c} ou {\tt .html}. On suppose qu'on se trouve actuellement dans ce dossier. 
\Question{Ecrire les commandes permettant :}
\subQuestion{De créer deux dossiers : {\tt LangageC} et {\tt Web}}
\reponse{0}{0,5}
\tcor{\mintinline{bash}{mkdir LangageC Web}}
\subQuestion{De déplacer tous les fichiers portant l'extension {\tt .c} dans {\tt LangageC} et tous ceux portant l'extension 
{\tt .html} dans {\tt Web}}
\reponse{1}{1}
\tcor{\mintinline{bash}{mv *.c LangageC} \\ \mintinline{bash}{mv *.html Web}}
\subQuestion{D'afficher les fichiers restants dans le dossier (y compris les fichiers cachés).}
\tcor{\mintinline{bash}{ls -a}}
\reponse{0}{1}
\Question{Grâce à la commande précédente, Alice s'aperçoit qu'elle a dans ce répertoire un fichier {\tt notes.txt}, quelle commande permet d'afficher le contenu de ce fichier dans le terminal ?}
\reponse{0}{0,5}
\tcor{\mintinline{bash}{cat notes.txt}}
\Question{Alice souhaite créer un lien physique vers {\tt notes.txt}, ce lien doit se trouver dans son répertoire personnelle dans un dossier existant nommé {\tt Important} et elle veut le nommer {\tt notes.sav}, quelle commande doit-elle taper ?}
\reponse{0}{1}
\tcor{\mintinline{bash}{ln notes.txt ~/Important/notes.sav}}
\end{Exercise}

\begin{Exercise}
\Question{Ecrire le prototype d'une fonction {\tt verifie}  qui prend en argument un caractère et un entier et renvoie un booléen}
\reponse{0}{1}
\ifcorrige
\begin{langageC}
bool verifie(char c, int n)
\end{langageC}
\fi
\Question{On veut utiliser cette fonction depuis le {\tt main} d'un programme}
\subQuestion{écrire les instructions permettant de déclarer un entier {\tt n} initialisé à  42 et un caractère {\tt c} contenant~{\tt @}}
\reponse{1}{1}
\ifcorrige
\begin{langageC} 
int n = 42;
int c = '@';
\end{langageC}
\fi

\subQuestion{écrire une instruction {\tt printf} permettant d'afficher les valeurs de {\tt n} et de {\tt c} dans le terminal}
\reponse{0}{1}
\ifcorrige
\begin{langageC}
printf("n = %d et c = %c",n,c);
\end{langageC}
\fi
\subQuestion{déclarer un booléen {\tt test} et l'initialiser au résultat de l'appel de {\tt verifie} avec les arguments {\tt c} et {\tt n} définis ci-dessus}
\reponse{0}{1}
\ifcorrige
\begin{langageC}
bool test = verifie(c,n)
\end{langageC}
\fi
\subQuestion{Ecrire une instructions conditionnelle qui affiche \og{}{\tt Ok !}\fg{} dans le terminal si {\tt test} vaut {\tt true} et \og{}{\tt Bug !}\fg{} sinon.}
\reponse{3}{2}
\ifcorrige
\begin{langageC}
if (test)
{
    printf("Ok !\n");
}
else
{
    printf("Bug !\n");
}
\end{langageC}
\fi
\end{Exercise}



\end{document}
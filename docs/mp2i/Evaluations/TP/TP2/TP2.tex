\PassOptionsToPackage{dvipsnames,table}{xcolor}
\documentclass[11pt,a4paper]{article}

\usepackage{Act}

\begin{document}
\input{\detokenize{/home/fenarius/Travail/Cours/cpge-info/latex/Macros.tex}}
\ModeExercice
\setboolean{corrige}{false}
\TPnote{2}{Tri en C}{10}
%Nom de la première activité
\setcounter{Exercise}{0}


\begin{Exercise}[title={Programmer un algorithme de tri}]\\
	Trois algorithmes de tri on été vus :
	\begin{itemize}
		\item le tri par sélection,
		\item le tri par insertion,
		\item le tri à bulles.
	\end{itemize}
	Pour \textit{un seul} de ces algorithmes (celui de \textit{de votre choix}) :
	\Question{Indiquer l'algorithme choisi et donner les étapes de son fonctionnement sur le tableau {\tt \{2, 8, 1, 5\}}}
	\reponse{4}{1}
	\tcor{On propose la correction pour les trois algorithmes (un seul était demandé). On note {\tt n} la taille du tableau.
		\begin{itemize}
			\item \textbf{tri par sélection}\\
			      Le principe est de chercher pour {\tt i=0 \dots i=n-1} le minimum du sous tableau {\tt tab[i] \dots tab[n-1]} et de l'échanger avec l'élément d'indice {\tt i}.\\
			      Sur l'exemple on obtient (en soulignant le minimum de chaque sous tableau)\\
			      \begin{tabular}{l|>{\tt}l}
				      Etapes                 & \multicolumn{1}{c}{Etat du tableau} \\
				      \hline
				      Initialisation         & \{2, 8, \underline{1}, 5\}          \\
				      {\tt i=0} 1er minimum  & \{1, 8, \underline{2}, 5\}          \\
				      {\tt i=1} 2eme minimum & \{1, 2, 8, \underline{5}\}          \\
				      {\tt i=2} 3eme minimum & \{1, 2, 5, 8\}                      \\
			      \end{tabular}
			\item \textbf{tri par insertion}\\
			      Le principe est d'insérer pour {\tt i=1 \dots i=n-1}, tab[i] dans le sous tableau {\tt tab[0] \dots tab[i-1]} en \textit{considérant ce sous tableau déjà trié}.
			      Sur l'exemple on obtient (en soulignant la partie déjà triée)\\
			      \begin{tabular}{l|>{\tt}l}
				      Etapes               & \multicolumn{1}{c}{Etat du tableau} \\
				      \hline
				      Initialisation       & \{\underline{2}, 8, 1, 5\}          \\
				      Insertion de {\tt 8} & \{\underline{2, 8}, 1, 5\}          \\
				      Insertion de {\tt 1} & \{\underline{1, 2, 8}, 5\}          \\
				      Insertion de {\tt 5} & \{\underline{1, 2, 5, 8}\}          \\
			      \end{tabular}
			\item \textbf{tri à bulles}\\
			      Le principe est pour {\tt i=n-1 \dots 0} de parcourir le sous tableau {\tt tab[0], \dots tab[i]} en échangeant un élément avec son voisin de droite s'il lui est supérieur. Sur l'exemple on obtient :\\
                  \begin{tabular}{l|>{\tt}l}
                    Etapes               & \multicolumn{1}{c}{Etat du tableau} \\
                    \hline
                    Initialisation       & \{2, 8, 1, 5\}          \\
                    1er parcours & \{2, 1, 5, 8\}          \\
                    2eme parcours & \{1, 2, 5, 8\}          \\
                    3eme parcours & \{1, 2, 5, 8\}          \\
                \end{tabular}
		\end{itemize}
	}
	\Question{Ecrire son implémentation en C, en supposant \textit{déjà écrite} une fonction de prototype : \\\mintinline{c}{echange(int tab[], int i, int j)} qui prend en argument un tableau {\tt tab}, sa taille {\tt size} et deux indices {\tt i} et {\tt j} et échange les éléments situés aux indices {\tt i} et {\tt j} de {\tt tab} en supposant vérifiées les préconditions sur {\tt i} et{\tt j}.}
	\reponse{17}{7}
    \ifcorrige 
    \corpartC{selection.c}{}{}{12}{38}
    \corpartC{insertion.c}{}{}{12}{29}
    \corpartC{bulles.c}{}{}{12}{33}
    \fi
	\Question{Créer un tableau {\tt test} de 5000 entiers et initialiser ce tableau de façon à ce que {\tt tab[i] = i*i - 564*i + 77760} pour {\tt i=0 \dots 4999}. Trier ce tableau à l'aide de l'implémentation de tri précédente et donner la valeur de {\tt tab[2024]}.}
	\reponse{1}{2}
    \ifcorrige
    \corpartC{selection.c}{}{}{40}{49}
    \fi
    \tcor{On obtient la valeur {\tt 3032800}}
\end{Exercise}

\end{document}
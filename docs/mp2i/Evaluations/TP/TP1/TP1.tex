\PassOptionsToPackage{dvipsnames,table}{xcolor}
\documentclass[11pt,a4paper]{article}

\usepackage{Act}

\begin{document}
\input{\detokenize{/home/fenarius/Travail/Cours/cpge-info/latex/Macros.tex}}
\ModeExercice
\setboolean{corrige}{false}
\TPnote{1}{Langage C}{10}
%Nom de la première activité
\setcounter{Exercise}{0}


\begin{Exercise}[title={Calcul d'une somme}]
	\Question{Ecrire une fonction {\tt somme} en C de prototype \mintinline{c}{int somme(int n)} qui renvoie la somme des entiers divisibles par 3 ou par 7 de $\intN{1}{n}$. Par exemple {\tt somme(10)} doit renvoyer {\tt 25} ({\tt 3 + 6 + 7 + 9}).}
	\reponse{8}{4}
	\ifcorrige
		\inputpartC{somme.c}{}{\small}{3}{14}
	\fi
	\Question{Quelle est la valeur renvoyée par {\tt somme(50000)} ?}
	\reponse{0}{1}
	\tcor{\tt   535710714}
\end{Exercise}

\begin{Exercise}[title={Etendue d'un tableau}]
	\Question{Ecrire une fonction {\tt etendue} qui prend en argument un tableau (supposée non vide) et sa taille et renvoie l'écart maximal entre deux éléments de ce tableau. Par exemple, sur le tableau {\tt int ex[7] = \{1, 5, 3, 0, -1, 4, 8 \}}, la fonction {\tt etendue} doit renvoyer {\tt 9}.}
	\reponse{8}{3}
	\ifcorrige \inputpartC{etendue.c}{}{\small}{3}{19} \fi
	\Question{Créer un tableau  d'entiers {\tt u} de taille 100 et à l'aide d'une boucle, l'initialiser avec les valeurs prises par la suite $(u_n)_{n \in \N}$ de terme général $u_n = n^2 - 133n + 3822$ pour $n=0 \dots 99$. c'est-à-dire que {\tt u[i]} doit contenir la valeur de $u_i$ (pour $i \in \intN{0}{99}$), par exemple {\tt u[0]=3822}.}
	\reponse{4}{1}
	\ifcorrige \inputpartC{etendue.c}{}{\small}{23}{28} \fi
	\Question{Déterminer l'écart maximal entre deux éléments du tableau {\tt u} défini à la question précédente et donner la réponse trouvée par votre programme :}
	\reponse{0}{1}
	\tcor{\tt 4422}
\end{Exercise}

\end{document}
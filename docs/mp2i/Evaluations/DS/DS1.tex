\PassOptionsToPackage{dvipsnames,table}{xcolor}
\documentclass[11pt,a4paper]{article}

\usepackage{Act}

\begin{document}
\input{\detokenize{/home/fenarius/Travail/Cours/cpge-info/latex/Macros.tex}}
\ModeExercice
\DS{1}{Septembre 2023}

\alertbox{\danger}{Consignes}{
    \begin{itemize}
        \item[\textbullet] Les programmes demandés doivent être écrits en C et on suppose que les librairies standards usuelles ({\tt <stdio.h}, {\tt <stdlib.h>}, {\tt <stdbool.h>}) sont déjà importées.
        \item[\textbullet] On pourra toujours librement utilser une fonction demandée à une question précédente même si cette question n'a pas été traitée.
        \item[\textbullet] Les deux parties du sujet sont indépendantes.
        \item[\textbullet] Veillez à présenter vos idées et vos réponses partielles même si vous ne trouvez pas la solution complète à une question.
        \item[\textbullet] La clarté et la lisibilité de la rédaction et des programmes sont des éléments de notation.
    \end{itemize}
}

\DSPart

\begin{Exercise}[title = {puissance}]
\Question{Ecrire une fonction {\tt valeur\_absolue} qui prend en argument un entier $n$ et renvoie sa valeur absolue $|n|$. On rappelle que :
$|n| = \left\{ \begin{array}{rl} -n & \mathrm{\ si\ } n<0 \\ n & \mathrm{\ sinon} \end{array}\right.$}
\Question{Ecrire une fonction {\tt puissance} qui prend en argument un flottant (type {\tt double}) $a$ et un entier $n$ et renvoie 
$a^n$. On rappelle que pour $a \in \R^*$, $n \in \Z$ : \\
$\left\{ \begin{array}{ll}
a^n = \underbrace{a\times \dots \times a}_{n \mathrm{\ facteurs}}  & \mathrm{\ si\ } n>0,\\
a^0 = 1, &  \\
a^n = \dfrac{1}{a^{-n}} & \mathrm{\ si\ } n<0. \\
\end{array}\right.$ \\

D'autre part $0^0=1$, $0^n=0$ si $n>0$ et les puissances négatives de zéro ne sont pas définies.On vérifiera la précondition $n>0$ lorsque $a=0$ à l'aide d'une instruction {\tt assert}.} 
\Question{Tracer le graphe de flot de contrôle de cette fonction.}
\Question{Proposer un jeu de test permettant de couvrir tous les arcs.}


\end{Exercise}

\DSPart

\end{document}
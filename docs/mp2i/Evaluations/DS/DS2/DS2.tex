\PassOptionsToPackage{dvipsnames,table}{xcolor}
\documentclass[11pt,a4paper]{article}

\usepackage{DS}

\begin{document}

\newcommand{\ModeExercice}{
% Traduction des noms pour le package exercise
\renewcommand{\ExerciseName}{Exercice}
\renewcommand{\thesubQuestion}{\theQuestion.\alph{subQuestion}}
\renewcommand{\AnswerName}{Réponses à l'exercise }
}
\newcommand{\Fiche}[2]{\lhead{\textbf{{\sc #1}}}
\rhead{Niveau: \textbf{#2}}
\cfoot{}}
\definecolor{cfond}{gray}{0.4}
\renewcommand{\thealgocf}{}

\newcommand{\ModeActivite}{
% Traduction des noms pour le package exercise
\renewcommand{\ExerciseName}{Activité}
}
% Réglages de la mise en forme des exercices 
\renewcommand{\ExerciseHeaderTitle}{\ExerciseTitle}
\renewcommand{\ExerciseHeaderOrigin}{\ExerciseOrigin}
% Un : sépare le numéro de l'exerice de son titre ... si le titre existe. On utilise Origine pour placer les pictogrammes en fin de ligne
\renewcommand{\ExerciseHeader}{\ding{113} \textbf{\sffamily{\ExerciseName \ \ExerciseHeaderNB}} \ifthenelse{\equal{\ExerciseTitle}{\empty}}{}{:} \textit{\ExerciseHeaderTitle} \hfill \ExerciseHeaderOrigin}
\renewcommand{\ExePartHeader}{\quad {\footnotesize \ding{110}} \textbf{Partie \textbf{\ExePartHeaderNB}} : \ExePartName}


% Mode Concours
\newcommand{\ModeConcours}{
   \newcounter{qconcours}
   \setcounter{qconcours}{1}
   \renewcommand{\ExerciseName}{Exercice}
   \renewcommand{\ExerciseHeaderTitle}{\ExerciseTitle}
\renewcommand{\ExerciseHeaderOrigin}{\ExerciseOrigin}
\renewcommand{\ExerciseHeader}{\ding{113} \textbf{\sffamily{\ExerciseName \ \ExerciseHeaderNB}} \ifthenelse{\equal{\ExerciseTitle}{\empty}}{}{:} \textit{\ExerciseHeaderTitle} \hfill \ExerciseHeaderOrigin}
\renewcommand{\QuestionNB}{\textbf{Q\arabic{qconcours}--}\ \addtocounter{qconcours}{1}}}

\newcommand{\noeud}[1]{\Tr{\fbox{\tt #1}}}
\newcommand{\FPATH}{/home/fenarius/Travail/Cours/cpge-info/docs/mp2i/}
\newcommand{\spath}[2]{\FPATH Evaluations/#1/#1#2}
\definecolor{codebg}{gray}{0.90}
\newcommand{\inputpartOCaml}[5]{\begin{mdframed}[backgroundcolor=codebg] \inputminted[breaklines=true,fontsize=#3,linenos=true,highlightcolor=fluo,tabsize=2,highlightlines={#2},firstline=#4,lastline=#5,firstnumber=1]{OCaml}{#1} \end{mdframed}}
\newcommand{\inputpartPython}[5]{\begin{mdframed}[backgroundcolor=codebg] \inputminted[breaklines=true,fontsize=#3,linenos=true,highlightcolor=fluo,tabsize=2,highlightlines={#2},firstline=#4,lastline=#5,firstnumber=1]{python}{#1} \end{mdframed}}
\newcommand{\inputpartC}[5]{\begin{mdframed}[backgroundcolor=codebg] \inputminted[breaklines=true,fontsize=#3,linenos=true,highlightcolor=fluo,tabsize=2,highlightlines={#2},firstline=#4,lastline=#5,firstnumber=1]{c}{#1} \end{mdframed}}
\newcommand{\inputC}[2]{\begin{mdframed}[backgroundcolor=codebg] \inputminted[breaklines=true,fontsize=#2,linenos=true,highlightcolor=fluo,tabsize=2]{c}{#1} \end{mdframed}}
\newminted[langageC]{c}{linenos=true,escapeinside=``,highlightcolor=fluo,tabsize=2}
\newminted[python]{python}{linenos=true,escapeinside=``,highlightcolor=fluo,tabsize=4}
\BeforeBeginEnvironment{minted}{\begin{mdframed}[backgroundcolor=codebg,skipabove=0cm]}
   \AfterEndEnvironment{minted}{\end{mdframed}}

% Font light medium et bold pour tt :
\newcommand{\ttl}[1]{\ttfamily \fontseries{l}\selectfont #1}
\newcommand{\ttm}[1]{\ttfamily \fontseries{m}\selectfont #1}
\newcommand{\ttb}[1]{\ttfamily \fontseries{b}\selectfont #1}

%QCM de NSI \QNSI{Question}{R1}{R2}{R3}{R4}
\newcommand{\QNSI}[5]{
#1
\begin{enumerate}[label=\alph{enumi})]
\item #2
\item #3
\item #4
\item #5
\end{enumerate}
}



\definecolor{grispale}{gray}{0.95}
\newcommand{\htmlmode}{\lstset{language=html,numbers=left, tabsize=2, frame=single, breaklines=true, keywordstyle=\ttfamily, basicstyle=\small,
   numberstyle=\tiny\ttfamily, framexleftmargin=0mm, backgroundcolor=\color{grispale}, xleftmargin=12mm,showstringspaces=false}}
\newcommand{\pythonmode}{\lstset{language=python,numbers=left, tabsize=4, frame=single, breaklines=true, keywordstyle=\ttfamily, basicstyle=\small,
   numberstyle=\tiny\ttfamily, framexleftmargin=0mm, backgroundcolor=\color{grispale}, xleftmargin=12mm, showstringspaces=false}}
\newcommand{\bashmode}{\lstset{language=bash,numbers=left, tabsize=2, frame=single, breaklines=true, basicstyle=\ttfamily,
   numberstyle=\tiny\ttfamily, framexleftmargin=0mm, backgroundcolor=\color{grispale}, xleftmargin=12mm, showstringspaces=false}}
\newcommand{\exomode}{\lstset{language=python,numbers=left, tabsize=2, frame=single, breaklines=true, basicstyle=\ttfamily,
   numberstyle=\tiny\ttfamily, framexleftmargin=13mm, xleftmargin=12mm, basicstyle=\small, showstringspaces=false}}
   
   
  \lstset{%
        inputencoding=utf8,
        extendedchars=true,
        literate=%
        {é}{{\'{e}}}1
        {è}{{\`{e}}}1
        {ê}{{\^{e}}}1
        {ë}{{\¨{e}}}1
        {É}{{\'{E}}}1
        {Ê}{{\^{E}}}1
        {û}{{\^{u}}}1
        {ù}{{\`{u}}}1
        {ú}{{\'{u}}}1
        {â}{{\^{a}}}1
        {à}{{\`{a}}}1
        {á}{{\'{a}}}1
        {ã}{{\~{a}}}1
        {Á}{{\'{A}}}1
        {Â}{{\^{A}}}1
        {Ã}{{\~{A}}}1
        {ç}{{\c{c}}}1
        {Ç}{{\c{C}}}1
        {õ}{{\~{o}}}1
        {ó}{{\'{o}}}1
        {ô}{{\^{o}}}1
        {Õ}{{\~{O}}}1
        {Ó}{{\'{O}}}1
        {Ô}{{\^{O}}}1
        {î}{{\^{i}}}1
        {Î}{{\^{I}}}1
        {í}{{\'{i}}}1
        {Í}{{\~{Í}}}1
}

%tei pour placer les images
%tei{nom de l’image}{échelle de l’image}{sens}{texte a positionner}
%sens ="1" (droite) ou "2" (gauche)
\newlength{\ltxt}
\newcommand{\tei}[4]{
\setlength{\ltxt}{\linewidth}
\setbox0=\hbox{\includegraphics[scale=#2]{#1}}
\addtolength{\ltxt}{-\wd0}
\addtolength{\ltxt}{-10pt}
\ifthenelse{\equal{#3}{1}}{
\begin{minipage}{\wd0}
\includegraphics[scale=#2]{#1}
\end{minipage}
\hfill
\begin{minipage}{\ltxt}
#4
\end{minipage}
}{
\begin{minipage}{\ltxt}
#4
\end{minipage}
\hfill
\begin{minipage}{\wd0}
\includegraphics[scale=#2]{#1}
\end{minipage}
}
}

%Juxtaposition d'une image pspciture et de texte 
%#1: = code pstricks de l'image
%#2: largeur de l'image
%#3: hauteur de l'image
%#4: Texte à écrire
\newcommand{\ptp}[4]{
\setlength{\ltxt}{\linewidth}
\addtolength{\ltxt}{-#2 cm}
\addtolength{\ltxt}{-0.1 cm}
\begin{minipage}[b][#3 cm][t]{\ltxt}
#4
\end{minipage}\hfill
\begin{minipage}[b][#3 cm][c]{#2 cm}
#1
\end{minipage}\par
}



%Macros pour les graphiques
\psset{linewidth=0.5\pslinewidth,PointSymbol=x}
\setlength{\fboxrule}{0.5pt}
\newcounter{tempangle}

%Marque la longueur du segment d'extrémité  #1 et  #2 avec la valeur #3, #4 est la distance par rapport au segment (en %age de la valeur de celui ci) et #5 l'orientation du marquage : +90 ou -90
\newcommand{\afflong}[5]{
\pstRotation[RotAngle=#4,PointSymbol=none,PointName=none]{#1}{#2}[X] 
\pstHomO[PointSymbol=none,PointName=none,HomCoef=#5]{#1}{X}[Y]
\pstTranslation[PointSymbol=none,PointName=none]{#1}{#2}{Y}[Z]
 \ncline{|<->|,linewidth=0.25\pslinewidth}{Y}{Z} \ncput*[nrot=:U]{\footnotesize{#3}}
}
\newcommand{\afflongb}[3]{
\ncline{|<->|,linewidth=0}{#1}{#2} \naput*[nrot=:U]{\footnotesize{#3}}
}

%Construis le point #4 situé à #2 cm du point #1 avant un angle #3 par rapport à l'horizontale. #5 = liste de paramètre
\newcommand{\lsegment}[5]{\pstGeonode[PointSymbol=none,PointName=none](0,0){O'}(#2,0){I'} \pstTranslation[PointSymbol=none,PointName=none]{O'}{I'}{#1}[J'] \pstRotation[RotAngle=#3,PointSymbol=x,#5]{#1}{J'}[#4]}
\newcommand{\tsegment}[5]{\pstGeonode[PointSymbol=none,PointName=none](0,0){O'}(#2,0){I'} \pstTranslation[PointSymbol=none,PointName=none]{O'}{I'}{#1}[J'] \pstRotation[RotAngle=#3,PointSymbol=x,#5]{#1}{J'}[#4] \pstLineAB{#4}{#1}}

%Construis le point #4 situé à #3 cm du point #1 et faisant un angle de  90° avec la droite (#1,#2) #5 = liste de paramètre
\newcommand{\psegment}[5]{
\pstGeonode[PointSymbol=none,PointName=none](0,0){O'}(#3,0){I'}
 \pstTranslation[PointSymbol=none,PointName=none]{O'}{I'}{#1}[J']
 \pstInterLC[PointSymbol=none,PointName=none]{#1}{#2}{#1}{J'}{M1}{M2} \pstRotation[RotAngle=-90,PointSymbol=x,#5]{#1}{M1}[#4]
  }
  
%Construis le point #4 situé à #3 cm du point #1 et faisant un angle de  #5° avec la droite (#1,#2) #6 = liste de paramètre
\newcommand{\mlogo}[6]{
\pstGeonode[PointSymbol=none,PointName=none](0,0){O'}(#3,0){I'}
 \pstTranslation[PointSymbol=none,PointName=none]{O'}{I'}{#1}[J']
 \pstInterLC[PointSymbol=none,PointName=none]{#1}{#2}{#1}{J'}{M1}{M2} \pstRotation[RotAngle=#5,PointSymbol=x,#6]{#1}{M2}[#4]
  }

% Construis un triangle avec #1=liste des 3 sommets séparés par des virgules, #2=liste des 3 longueurs séparés par des virgules, #3 et #4 : paramètre d'affichage des 2e et 3 points et #5 : inclinaison par rapport à l'horizontale
%autre macro identique mais sans tracer les segments joignant les sommets
\noexpandarg
\newcommand{\Triangleccc}[5]{
\StrBefore{#1}{,}[\pointA]
\StrBetween[1,2]{#1}{,}{,}[\pointB]
\StrBehind[2]{#1}{,}[\pointC]
\StrBefore{#2}{,}[\coteA]
\StrBetween[1,2]{#2}{,}{,}[\coteB]
\StrBehind[2]{#2}{,}[\coteC]
\tsegment{\pointA}{\coteA}{#5}{\pointB}{#3} 
\lsegment{\pointA}{\coteB}{0}{Z1}{PointSymbol=none, PointName=none}
\lsegment{\pointB}{\coteC}{0}{Z2}{PointSymbol=none, PointName=none}
\pstInterCC{\pointA}{Z1}{\pointB}{Z2}{\pointC}{Z3} 
\pstLineAB{\pointA}{\pointC} \pstLineAB{\pointB}{\pointC}
\pstSymO[PointName=\pointC,#4]{C}{C}[C]
}
\noexpandarg
\newcommand{\TrianglecccP}[5]{
\StrBefore{#1}{,}[\pointA]
\StrBetween[1,2]{#1}{,}{,}[\pointB]
\StrBehind[2]{#1}{,}[\pointC]
\StrBefore{#2}{,}[\coteA]
\StrBetween[1,2]{#2}{,}{,}[\coteB]
\StrBehind[2]{#2}{,}[\coteC]
\tsegment{\pointA}{\coteA}{#5}{\pointB}{#3} 
\lsegment{\pointA}{\coteB}{0}{Z1}{PointSymbol=none, PointName=none}
\lsegment{\pointB}{\coteC}{0}{Z2}{PointSymbol=none, PointName=none}
\pstInterCC[PointNameB=none,PointSymbolB=none,#4]{\pointA}{Z1}{\pointB}{Z2}{\pointC}{Z1} 
}


% Construis un triangle avec #1=liste des 3 sommets séparés par des virgules, #2=liste formée de 2 longueurs et d'un angle séparés par des virgules, #3 et #4 : paramètre d'affichage des 2e et 3 points et #5 : inclinaison par rapport à l'horizontale
%autre macro identique mais sans tracer les segments joignant les sommets
\newcommand{\Trianglecca}[5]{
\StrBefore{#1}{,}[\pointA]
\StrBetween[1,2]{#1}{,}{,}[\pointB]
\StrBehind[2]{#1}{,}[\pointC]
\StrBefore{#2}{,}[\coteA]
\StrBetween[1,2]{#2}{,}{,}[\coteB]
\StrBehind[2]{#2}{,}[\angleA]
\tsegment{\pointA}{\coteA}{#5}{\pointB}{#3} 
\setcounter{tempangle}{#5}
\addtocounter{tempangle}{\angleA}
\tsegment{\pointA}{\coteB}{\thetempangle}{\pointC}{#4}
\pstLineAB{\pointB}{\pointC}
}
\newcommand{\TriangleccaP}[5]{
\StrBefore{#1}{,}[\pointA]
\StrBetween[1,2]{#1}{,}{,}[\pointB]
\StrBehind[2]{#1}{,}[\pointC]
\StrBefore{#2}{,}[\coteA]
\StrBetween[1,2]{#2}{,}{,}[\coteB]
\StrBehind[2]{#2}{,}[\angleA]
\lsegment{\pointA}{\coteA}{#5}{\pointB}{#3} 
\setcounter{tempangle}{#5}
\addtocounter{tempangle}{\angleA}
\lsegment{\pointA}{\coteB}{\thetempangle}{\pointC}{#4}
}

% Construis un triangle avec #1=liste des 3 sommets séparés par des virgules, #2=liste formée de 1 longueurs et de deux angle séparés par des virgules, #3 et #4 : paramètre d'affichage des 2e et 3 points et #5 : inclinaison par rapport à l'horizontale
%autre macro identique mais sans tracer les segments joignant les sommets
\newcommand{\Trianglecaa}[5]{
\StrBefore{#1}{,}[\pointA]
\StrBetween[1,2]{#1}{,}{,}[\pointB]
\StrBehind[2]{#1}{,}[\pointC]
\StrBefore{#2}{,}[\coteA]
\StrBetween[1,2]{#2}{,}{,}[\angleA]
\StrBehind[2]{#2}{,}[\angleB]
\tsegment{\pointA}{\coteA}{#5}{\pointB}{#3} 
\setcounter{tempangle}{#5}
\addtocounter{tempangle}{\angleA}
\lsegment{\pointA}{1}{\thetempangle}{Z1}{PointSymbol=none, PointName=none}
\setcounter{tempangle}{#5}
\addtocounter{tempangle}{180}
\addtocounter{tempangle}{-\angleB}
\lsegment{\pointB}{1}{\thetempangle}{Z2}{PointSymbol=none, PointName=none}
\pstInterLL[#4]{\pointA}{Z1}{\pointB}{Z2}{\pointC}
\pstLineAB{\pointA}{\pointC}
\pstLineAB{\pointB}{\pointC}
}
\newcommand{\TrianglecaaP}[5]{
\StrBefore{#1}{,}[\pointA]
\StrBetween[1,2]{#1}{,}{,}[\pointB]
\StrBehind[2]{#1}{,}[\pointC]
\StrBefore{#2}{,}[\coteA]
\StrBetween[1,2]{#2}{,}{,}[\angleA]
\StrBehind[2]{#2}{,}[\angleB]
\lsegment{\pointA}{\coteA}{#5}{\pointB}{#3} 
\setcounter{tempangle}{#5}
\addtocounter{tempangle}{\angleA}
\lsegment{\pointA}{1}{\thetempangle}{Z1}{PointSymbol=none, PointName=none}
\setcounter{tempangle}{#5}
\addtocounter{tempangle}{180}
\addtocounter{tempangle}{-\angleB}
\lsegment{\pointB}{1}{\thetempangle}{Z2}{PointSymbol=none, PointName=none}
\pstInterLL[#4]{\pointA}{Z1}{\pointB}{Z2}{\pointC}
}

%Construction d'un cercle de centre #1 et de rayon #2 (en cm)
\newcommand{\Cercle}[2]{
\lsegment{#1}{#2}{0}{Z1}{PointSymbol=none, PointName=none}
\pstCircleOA{#1}{Z1}
}

%construction d'un parallélogramme #1 = liste des sommets, #2 = liste contenant les longueurs de 2 côtés consécutifs et leurs angles;  #3, #4 et #5 : paramètre d'affichage des sommets #6 inclinaison par rapport à l'horizontale 
% meme macro sans le tracé des segements
\newcommand{\Para}[6]{
\StrBefore{#1}{,}[\pointA]
\StrBetween[1,2]{#1}{,}{,}[\pointB]
\StrBetween[2,3]{#1}{,}{,}[\pointC]
\StrBehind[3]{#1}{,}[\pointD]
\StrBefore{#2}{,}[\longueur]
\StrBetween[1,2]{#2}{,}{,}[\largeur]
\StrBehind[2]{#2}{,}[\angle]
\tsegment{\pointA}{\longueur}{#6}{\pointB}{#3} 
\setcounter{tempangle}{#6}
\addtocounter{tempangle}{\angle}
\tsegment{\pointA}{\largeur}{\thetempangle}{\pointD}{#5}
\pstMiddleAB[PointName=none,PointSymbol=none]{\pointB}{\pointD}{Z1}
\pstSymO[#4]{Z1}{\pointA}[\pointC]
\pstLineAB{\pointB}{\pointC}
\pstLineAB{\pointC}{\pointD}
}
\newcommand{\ParaP}[6]{
\StrBefore{#1}{,}[\pointA]
\StrBetween[1,2]{#1}{,}{,}[\pointB]
\StrBetween[2,3]{#1}{,}{,}[\pointC]
\StrBehind[3]{#1}{,}[\pointD]
\StrBefore{#2}{,}[\longueur]
\StrBetween[1,2]{#2}{,}{,}[\largeur]
\StrBehind[2]{#2}{,}[\angle]
\lsegment{\pointA}{\longueur}{#6}{\pointB}{#3} 
\setcounter{tempangle}{#6}
\addtocounter{tempangle}{\angle}
\lsegment{\pointA}{\largeur}{\thetempangle}{\pointD}{#5}
\pstMiddleAB[PointName=none,PointSymbol=none]{\pointB}{\pointD}{Z1}
\pstSymO[#4]{Z1}{\pointA}[\pointC]
}


%construction d'un cerf-volant #1 = liste des sommets, #2 = liste contenant les longueurs de 2 côtés consécutifs et leurs angles;  #3, #4 et #5 : paramètre d'affichage des sommets #6 inclinaison par rapport à l'horizontale 
% meme macro sans le tracé des segements
\newcommand{\CerfVolant}[6]{
\StrBefore{#1}{,}[\pointA]
\StrBetween[1,2]{#1}{,}{,}[\pointB]
\StrBetween[2,3]{#1}{,}{,}[\pointC]
\StrBehind[3]{#1}{,}[\pointD]
\StrBefore{#2}{,}[\longueur]
\StrBetween[1,2]{#2}{,}{,}[\largeur]
\StrBehind[2]{#2}{,}[\angle]
\tsegment{\pointA}{\longueur}{#6}{\pointB}{#3} 
\setcounter{tempangle}{#6}
\addtocounter{tempangle}{\angle}
\tsegment{\pointA}{\largeur}{\thetempangle}{\pointD}{#5}
\pstOrtSym[#4]{\pointB}{\pointD}{\pointA}[\pointC]
\pstLineAB{\pointB}{\pointC}
\pstLineAB{\pointC}{\pointD}
}

%construction d'un quadrilatère quelconque #1 = liste des sommets, #2 = liste contenant les longueurs des 4 côtés et l'angle entre 2 cotés consécutifs  #3, #4 et #5 : paramètre d'affichage des sommets #6 inclinaison par rapport à l'horizontale 
% meme macro sans le tracé des segements
\newcommand{\Quadri}[6]{
\StrBefore{#1}{,}[\pointA]
\StrBetween[1,2]{#1}{,}{,}[\pointB]
\StrBetween[2,3]{#1}{,}{,}[\pointC]
\StrBehind[3]{#1}{,}[\pointD]
\StrBefore{#2}{,}[\coteA]
\StrBetween[1,2]{#2}{,}{,}[\coteB]
\StrBetween[2,3]{#2}{,}{,}[\coteC]
\StrBetween[3,4]{#2}{,}{,}[\coteD]
\StrBehind[4]{#2}{,}[\angle]
\tsegment{\pointA}{\coteA}{#6}{\pointB}{#3} 
\setcounter{tempangle}{#6}
\addtocounter{tempangle}{\angle}
\tsegment{\pointA}{\coteD}{\thetempangle}{\pointD}{#5}
\lsegment{\pointB}{\coteB}{0}{Z1}{PointSymbol=none, PointName=none}
\lsegment{\pointD}{\coteC}{0}{Z2}{PointSymbol=none, PointName=none}
\pstInterCC[PointNameA=none,PointSymbolA=none,#4]{\pointB}{Z1}{\pointD}{Z2}{Z3}{\pointC} 
\pstLineAB{\pointB}{\pointC}
\pstLineAB{\pointC}{\pointD}
}


% Définition des colonnes centrées ou à droite pour tabularx
\newcolumntype{Y}{>{\centering\arraybackslash}X}
\newcolumntype{Z}{>{\flushright\arraybackslash}X}

%Les pointillés à remplir par les élèves
\newcommand{\po}[1]{\makebox[#1 cm]{\dotfill}}
\newcommand{\lpo}[1][3]{%
\multido{}{#1}{\makebox[\linewidth]{\dotfill}
}}

%Liste des pictogrammes utilisés sur la fiche d'exercice ou d'activités
\newcommand{\bombe}{\faBomb}
\newcommand{\livre}{\faBook}
\newcommand{\calculatrice}{\faCalculator}
\newcommand{\oral}{\faCommentO}
\newcommand{\surfeuille}{\faEdit}
\newcommand{\ordinateur}{\faLaptop}
\newcommand{\ordi}{\faDesktop}
\newcommand{\ciseaux}{\faScissors}
\newcommand{\danger}{\faExclamationTriangle}
\newcommand{\out}{\faSignOut}
\newcommand{\cadeau}{\faGift}
\newcommand{\flash}{\faBolt}
\newcommand{\lumiere}{\faLightbulb}
\newcommand{\compas}{\dsmathematical}
\newcommand{\calcullitteral}{\faTimesCircleO}
\newcommand{\raisonnement}{\faCogs}
\newcommand{\recherche}{\faSearch}
\newcommand{\rappel}{\faHistory}
\newcommand{\video}{\faFilm}
\newcommand{\capacite}{\faPuzzlePiece}
\newcommand{\aide}{\faLifeRing}
\newcommand{\loin}{\faExternalLink}
\newcommand{\groupe}{\faUsers}
\newcommand{\bac}{\faGraduationCap}
\newcommand{\histoire}{\faUniversity}
\newcommand{\coeur}{\faSave}
\newcommand{\os}{\faMicrochip}
\newcommand{\rd}{\faCubes}
\newcommand{\data}{\faColumns}
\newcommand{\web}{\faCode}
\newcommand{\prog}{\faFile}
\newcommand{\algo}{\faCogs}
\newcommand{\important}{\faExclamationCircle}
\newcommand{\maths}{\faTimesCircle}
% Traitement des données en tables
\newcommand{\tables}{\faColumns}
% Types construits
\newcommand{\construits}{\faCubes}
% Type et valeurs de base
\newcommand{\debase}{{\footnotesize \faCube}}
% Systèmes d'exploitation
\newcommand{\linux}{\faLinux}
\newcommand{\sd}{\faProjectDiagram}
\newcommand{\bd}{\faDatabase}

%Les ensembles de nombres
\renewcommand{\N}{\mathbb{N}}
\newcommand{\D}{\mathbb{D}}
\newcommand{\Z}{\mathbb{Z}}
\newcommand{\Q}{\mathbb{Q}}
\newcommand{\R}{\mathbb{R}}
\newcommand{\C}{\mathbb{C}}

%Ecriture des vecteurs
\newcommand{\vect}[1]{\vbox{\halign{##\cr 
  \tiny\rightarrowfill\cr\noalign{\nointerlineskip\vskip1pt} 
  $#1\mskip2mu$\cr}}}


%Compteur activités/exos et question et mise en forme titre et questions
\newcounter{numact}
\setcounter{numact}{1}
\newcounter{numseance}
\setcounter{numseance}{1}
\newcounter{numexo}
\setcounter{numexo}{0}
\newcounter{numprojet}
\setcounter{numprojet}{0}
\newcounter{numquestion}
\newcommand{\espace}[1]{\rule[-1ex]{0pt}{#1 cm}}
\newcommand{\Quest}[3]{
\addtocounter{numquestion}{1}
\begin{tabularx}{\textwidth}{X|m{1cm}|}
\cline{2-2}
\textbf{\sffamily{\alph{numquestion})}} #1 & \dots / #2 \\
\hline 
\multicolumn{2}{|l|}{\espace{#3}} \\
\hline
\end{tabularx}
}
\newcommand{\mq}[1]
{\ding{113} \addtocounter{numquestion}{1}
\textbf{Question \arabic{numquestion}} \\ #1}
\newcommand{\QuestR}[3]{
\addtocounter{numquestion}{1}
\begin{tabularx}{\textwidth}{X|m{1cm}|}
\cline{2-2}
\textbf{\sffamily{\alph{numquestion})}} #1 & \dots / #2 \\
\hline 
\multicolumn{2}{|l|}{\cor{#3}} \\
\hline
\end{tabularx}
}
\newcommand{\Pre}{{\sc nsi} 1\textsuperscript{e}}
\newcommand{\Term}{{\sc nsi} Terminale}
\newcommand{\Sec}{2\textsuperscript{e}}
\newcommand{\Exo}[2]{ \addtocounter{numexo}{1} \ding{113} \textbf{\sffamily{Exercice \thenumexo}} : \textit{#1} \hfill #2  \setcounter{numquestion}{0}}
\newcommand{\Projet}[1]{ \addtocounter{numprojet}{1} \ding{118} \textbf{\sffamily{Projet \thenumprojet}} : \textit{#1}}
\newcommand{\ExoD}[2]{ \addtocounter{numexo}{1} \ding{113} \textbf{\sffamily{Exercice \thenumexo}}  \textit{(#1 pts)} \hfill #2  \setcounter{numquestion}{0}}
\newcommand{\ExoB}[2]{ \addtocounter{numexo}{1} \ding{113} \textbf{\sffamily{Exercice \thenumexo}}  \textit{(Bonus de +#1 pts maximum)} \hfill #2  \setcounter{numquestion}{0}}
\newcommand{\Act}[2]{ \ding{113} \textbf{\sffamily{Activité \thenumact}} : \textit{#1} \hfill #2  \addtocounter{numact}{1} \setcounter{numquestion}{0}}
\newcommand{\Seance}{ \rule{1.5cm}{0.5pt}\raisebox{-3pt}{\framebox[4cm]{\textbf{\sffamily{Séance \thenumseance}}}}\hrulefill  \\
  \addtocounter{numseance}{1}}
\newcommand{\Acti}[2]{ {\footnotesize \ding{117}} \textbf{\sffamily{Activité \thenumact}} : \textit{#1} \hfill #2  \addtocounter{numact}{1} \setcounter{numquestion}{0}}
\newcommand{\titre}[1]{\begin{Large}\textbf{\ding{118}}\end{Large} \begin{large}\textbf{ #1}\end{large} \vspace{0.2cm}}
\newcommand{\QListe}[1][0]{
\ifthenelse{#1=0}
{\begin{enumerate}[partopsep=0pt,topsep=0pt,parsep=0pt,itemsep=0pt,label=\textbf{\sffamily{\arabic*.}},series=question]}
{\begin{enumerate}[resume*=question]}}
\newcommand{\SQListe}[1][0]{
\ifthenelse{#1=0}
{\begin{enumerate}[partopsep=0pt,topsep=0pt,parsep=0pt,itemsep=0pt,label=\textbf{\sffamily{\alph*)}},series=squestion]}
{\begin{enumerate}[resume*=squestion]}}
\newcommand{\SQListeL}[1][0]{
\ifthenelse{#1=0}
{\begin{enumerate*}[partopsep=0pt,topsep=0pt,parsep=0pt,itemsep=0pt,label=\textbf{\sffamily{\alph*)}},series=squestion]}
{\begin{enumerate*}[resume*=squestion]}}
\newcommand{\FinListe}{\end{enumerate}}
\newcommand{\FinListeL}{\end{enumerate*}}

%Mise en forme de la correction
\newboolean{corrige}
\setboolean{corrige}{false}
\newcommand{\scor}[1]{\par \textcolor{blue!75!black}{\small #1}}
\newcommand{\cor}[1]{\par \textcolor{blue!75!black}{#1}}
\newcommand{\br}[1]{\cor{\textbf{#1}}}
\newcommand{\tcor}[1]{
\ifthenelse{\boolean{corrige}}{\begin{tcolorbox}[width=\linewidth,colback={white},colbacktitle=white,coltitle=green!50!black,colframe=green!50!black,boxrule=0.2mm]   
\cor{#1}
\end{tcolorbox}}{}
}
\newcommand{\iscor}[1]{\ifthenelse{\boolean{corrige}}{#1}}
\newcommand{\rc}[1]{\textcolor{OliveGreen}{#1}}

%Référence aux exercices par leur numéro
\newcommand{\refexo}[1]{
\refstepcounter{numexo}
\addtocounter{numexo}{-1}
\label{#1}}

%Séparation entre deux activités
\newcommand{\separateur}{\begin{center}
\rule{1.5cm}{0.5pt}\raisebox{-3pt}{\ding{117}}\rule{1.5cm}{0.5pt}  \vspace{0.2cm}
\end{center}}

%Entête et pied de page
\newcommand{\snt}[1]{\lhead{\textbf{SNT -- La photographie numérique} \rhead{\textit{Lycée Nord}}}}
\newcommand{\Activites}[2]{\lhead{\textbf{{\sc #1}}}
\rhead{Activités -- \textbf{#2}}
\cfoot{}}
\newcommand{\Exos}[2]{\lhead{\textbf{Fiche d'exercices: {\sc #1}}}
\rhead{Niveau: \textbf{#2}}
\cfoot{}}
\newcommand{\TD}[2]{\lhead{\textbf{TD #1} : {\sc #2} }
\rhead{{\sc mp2i -- Lycée Leconte de Lisle}}
\cfoot{}}
\newcommand{\Colles}[2]{\lhead{{\sc mp2i -- }\textbf{Colles d'informatique #1}} 
\rhead{{\sc #2}}
\cfoot{}}
\newcommand{\Devoir}[2]{\lhead{\textbf{Devoir de mathématiques : {\sc #1}}}
\rhead{\textbf{#2}} \setlength{\fboxsep}{8pt}
\begin{center}
%Titre de la fiche
\fbox{\parbox[b][1cm][t]{0.3\textwidth}{Nom : \hfill \po{3} \par \vfill Prénom : \hfill \po{3}} } \hfill 
\fbox{\parbox[b][1cm][t]{0.6\textwidth}{Note : \po{1} / 20} }
\end{center} \cfoot{}}
\newcommand{\TPnote}[3]{\lhead{\textbf{TP noté d'informatique n° #1}}
\rhead{\textbf{#2}} \setlength{\fboxsep}{8pt}
\ifthenelse{\boolean{corrige}}{}
{\begin{center}
\fbox{\parbox[b][1cm][t]{0.3\textwidth}{Nom : \hfill \po{3} \par \vfill Prénom : \hfill \po{3}} } \hfill 
\fbox{\parbox[b][1cm][t]{0.6\textwidth}{Note : \po{1} / #3} }
\end{center}} \cfoot{}}
\newcommand{\IC}[2]{\lhead{\textbf{Interro de cours n° #1}}
\rhead{{\sc mp2i --} \textbf{#2}} \setlength{\fboxsep}{8pt}
\ifthenelse{\boolean{corrige}}{}
{\begin{center}
%Titre de la fiche
\fbox{\parbox[b][1cm][t]{0.3\textwidth}{Nom : \hfill \po{3} \par \vfill Prénom : \hfill \po{3}} } \hfill 
\fbox{\parbox[b][1cm][t]{0.6\textwidth}{Note : \po{1} / 10} }
\end{center}}\cfoot{}}
\newcommand{\DS}[3]{\lhead{{#1} : \textbf{DS d'informatique n° #2}}
\rhead{Lycée Leconte de Lisle -- #3} \setlength{\fboxsep}{8pt}
%Titre de la fiche
\begin{center}
   {\Large \textbf{Devoir surveillé d'informatique}}
\end{center} \cfoot{\thepage/\pageref{LastPage}}}

\newcommand{\CB}[2]{\lhead{{#1} : \textbf{Councours blanc - Informatique}}
\rhead{Lycée Leconte de Lisle -- #2} \setlength{\fboxsep}{8pt}
%Titre de la fiche
\begin{center}
   {\Large \textbf{Concours Blanc - Epreuve d'informatique}}
\end{center} \cfoot{\thepage/\pageref{LastPage}}}

\newcommand{\PC}[3]{\lhead{Concours {#1} -- #2}
\rhead{Lycée Leconte de Lisle} \setlength{\fboxsep}{8pt}
%Titre de la fiche
\begin{center}
   {\Large \textbf{Proposition de corrigé}}
\end{center} \cfoot{\thepage/\pageref{LastPage}}}

\newcounter{numdspart}
\setcounter{numdspart}{1}
\newcommand{\DSPart}{\bigskip
   \hrulefill\raisebox{-3pt}{\framebox[4cm]{\textbf{\textbf{Partie \thenumdspart}}}}\hrulefill
   \addtocounter{numdspart}{1}
   \bigskip}

\newcommand{\Sauvegarde}[1]{
   \begin{tcolorbox}[title=\textcolor{black}{\danger\; Attention},colbacktitle=lightgray]
      {Tous vos programmes doivent être enregistrés dans votre dossier personnel, dans {\tt Evaluations}{\tt \textbackslash}{\tt #1}
      }
   \end{tcolorbox}
}

\newcommand{\alertbox}[3]{
   \begin{tcolorbox}[title=\textcolor{black}{#1\; #2},colbacktitle=lightgray]
      {#3}
   \end{tcolorbox}
}

%Devoir programmation en NSI (pas à rendre sur papier)
\newcommand{\PNSI}[2]{\lhead{\textbf{Devoir de {\sc nsi} : \textsf{ #1}}
}
\rhead{\textbf{#2}} \setlength{\fboxsep}{8pt}
 \cfoot{}
 \begin{center}{\Large \textbf{Evaluation de {\sc nsi}}}\end{center}}


%Devoir de NSI
\newcommand{\DNSI}[2]{\lhead{\textbf{Devoir de {\sc nsi} : \textsf{ #1}}}
\rhead{\textbf{#2}} \setlength{\fboxsep}{8pt}
\begin{center}
%Titre de la fiche
\fbox{\parbox[b][1cm][t]{0.3\textwidth}{Nom : \hfill \po{3} \par \vfill Prénom : \hfill \po{3}} } \hfill 
\fbox{\parbox[b][1cm][t]{0.6\textwidth}{Note : \po{1} / 10} }
\end{center} \cfoot{}}

\newcommand{\DevoirNSI}[2]{\lhead{\textbf{Devoir de {\sc nsi} : {\sc #1}}}
\rhead{\textbf{#2}} \setlength{\fboxsep}{8pt}
\cfoot{}}

%La définition de la commande QCM pour auto-multiple-choice
%En premier argument le sujet du qcm, deuxième argument : la classe, 3e : la durée prévue et #4 : présence ou non de questions avec plusieurs bonnes réponses
\newcommand{\QCM}[4]{
{\large \textbf{\ding{52} QCM : #1}} -- Durée : \textbf{#3 min} \hfill {\large Note : \dots/10} 
\hrule \vspace{0.1cm}\namefield{}
Nom :  \textbf{\textbf{\nom{}}} \qquad \qquad Prénom :  \textbf{\prenom{}}  \hfill Classe: \textbf{#2}
\vspace{0.2cm}
\hrule  
\begin{itemize}[itemsep=0pt]
\item[-] \textit{Une bonne réponse vaut un point, une absence de réponse n'enlève pas de point. }
\item[\danger] \textit{Une mauvaise réponse enlève un point.}
\ifthenelse{#4=1}{\item[-] \textit{Les questions marquées du symbole \multiSymbole{} peuvent avoir plusieurs bonnes réponses possibles.}}{}
\end{itemize}
}
\newcommand{\DevoirC}[2]{
\renewcommand{\footrulewidth}{0.5pt}
\lhead{\textbf{Devoir de mathématiques : {\sc #1}}}
\rhead{\textbf{#2}} \setlength{\fboxsep}{8pt}
\fbox{\parbox[b][0.4cm][t]{0.955\textwidth}{Nom : \po{5} \hfill Prénom : \po{5} \hfill Classe: \textbf{1}\textsuperscript{$\dots$}} } 
\rfoot{\thepage} \cfoot{} \lfoot{Lycée Nord}}
\newcommand{\DevoirInfo}[2]{\lhead{\textbf{Evaluation : {\sc #1}}}
\rhead{\textbf{#2}} \setlength{\fboxsep}{8pt}
 \cfoot{}}
\newcommand{\DM}[2]{\lhead{\textbf{Devoir maison à rendre le #1}} \rhead{\textbf{#2}}}

%Macros permettant l'affichage des touches de la calculatrice
%Touches classiques : #1 = 0 fond blanc pour les nombres et #1= 1gris pour les opérations et entrer, second paramètre=contenu
%Si #2=1 touche arrondi avec fond gris
\newcommand{\TCalc}[2]{
\setlength{\fboxsep}{0.1pt}
\ifthenelse{#1=0}
{\psframebox[fillstyle=solid, fillcolor=white]{\parbox[c][0.25cm][c]{0.6cm}{\centering #2}}}
{\ifthenelse{#1=1}
{\psframebox[fillstyle=solid, fillcolor=lightgray]{\parbox[c][0.25cm][c]{0.6cm}{\centering #2}}}
{\psframebox[framearc=.5,fillstyle=solid, fillcolor=white]{\parbox[c][0.25cm][c]{0.6cm}{\centering #2}}}
}}
\newcommand{\Talpha}{\psdblframebox[fillstyle=solid, fillcolor=white]{\hspace{-0.05cm}\parbox[c][0.25cm][c]{0.65cm}{\centering \scriptsize{alpha}}} \;}
\newcommand{\Tsec}{\psdblframebox[fillstyle=solid, fillcolor=white]{\parbox[c][0.25cm][c]{0.6cm}{\centering \scriptsize 2nde}} \;}
\newcommand{\Tfx}{\psdblframebox[fillstyle=solid, fillcolor=white]{\parbox[c][0.25cm][c]{0.6cm}{\centering \scriptsize $f(x)$}} \;}
\newcommand{\Tvar}{\psframebox[framearc=.5,fillstyle=solid, fillcolor=white]{\hspace{-0.22cm} \parbox[c][0.25cm][c]{0.82cm}{$\scriptscriptstyle{X,T,\theta,n}$}}}
\newcommand{\Tgraphe}{\psdblframebox[fillstyle=solid, fillcolor=white]{\hspace{-0.08cm}\parbox[c][0.25cm][c]{0.68cm}{\centering \tiny{graphe}}} \;}
\newcommand{\Tfen}{\psdblframebox[fillstyle=solid, fillcolor=white]{\hspace{-0.08cm}\parbox[c][0.25cm][c]{0.68cm}{\centering \tiny{fenêtre}}} \;}
\newcommand{\Ttrace}{\psdblframebox[fillstyle=solid, fillcolor=white]{\parbox[c][0.25cm][c]{0.6cm}{\centering \scriptsize{trace}}} \;}

% Macroi pour l'affichage  d'un entier n dans  une base b
\newcommand{\base}[2]{ \overline{#1}^{#2}}
% Intervalle d'entiers
\newcommand{\intN}[2]{\llbracket #1; #2 \rrbracket}
% Cadre avec lignes réponses
\def\gaddtotok#1{\global\tabtok\expandafter{\the\tabtok#1}}
\newtoks\tabtok
\newcommand*\reponse[2]{%
   \ifthenelse{\boolean{corrige}}{}{
	\global\tabtok{\\ \renewcommand{\arraystretch}{1.4}\begin{tabularx}{\linewidth}{|X|p{1cm}|}\hline \dotfill & \cellcolor{gray!30}{\small \dots/#2} \\ \cline{2-2}}%
	\multido{}{#1}{\gaddtotok{ \multicolumn{2}{|>{\hsize=\dimexpr1\hsize+2\tabcolsep+\arrayrulewidth+1cm\relax}X|}{\dotfill}\\ }}%
	\gaddtotok{\hline \end{tabularx}}%
	\the\tabtok
   }}

\newcommand{\PE}[1]{\left \lfloor #1 \right \lfloor}}
\ModeExercice
\DS{MP2I}{2}{Novembre 2024}

\setboolean{corrige}{false}

\newcommand{\maillon}[3]{
	\begin{tabular}{|p{0.2cm}|p{0.2cm}|}
		\hline
		\rnode{#2}{#1} & \rnode{#3}{\phantom{$e_0$}} \\
		\hline
	\end{tabular}
}

\alertbox{\danger}{Consignes}{
	\begin{itemize}
		\item[\textbullet] Les programmes demandés doivent être écrits en C ou en OCaml. Dans le cas du C, on suppose que les librairies standards usuelles ({\tt <stdio.h>}, {\tt <stdlib.h>}, {\tt <stdbool.h>}, {\tt <stdassert.h>}, \dots) sont déjà importées.
		\item[\textbullet] On pourra toujours librement utiliser une fonction demandée à une question précédente même si cette question n'a pas été traitée.
		\item[\textbullet] Veillez à présenter vos idées et vos réponses partielles même si vous ne trouvez pas la solution complète à une question.
		\item[\textbullet] La clarté et la lisibilité de la rédaction et des programmes sont des éléments de notation.
	\end{itemize}
}


\begin{Exercise}[title = {Questions de cours}] \\
    On donne ci-dessous l'algorithme d'exponentiation rapide en version itérative : \\
    \begin{algorithm}[H]
		\DontPrintSemicolon
		\caption{Exponentiation rapide}
		\Entree{$a \in \R, n \in \N$}
		\Sortie{$a^n$}
		\everypar={\footnotesize \textcolor{gray}{\nl}}
		$p \leftarrow 1$\;
		\Tq{$n \neq 0$}{
			\Si{$n$ est impair}
			{$p \leftarrow p\times a$ \;}
			$a \leftarrow a*a$ \;
			$n \leftarrow \lfloor\frac{n}{2}\rfloor$ \;
		}
		\Return $p$
	\end{algorithm}
      \Question{Donner les valeurs successives prises par les variables $a$, $n$ et $p$ si on fait fonctionner cet algorithme avec $a=2$ et $n=13$. On pourra recopier et compléter le tableau suivant et donner les valeurs de $a$ et de $p$ sous la forme de puissance de 2 :\\
      \begin{tabular}{|l|c|c|c|}
        \cline{2-4}
        \multicolumn{1}{l|}{} & $a$ & $n$ & $p$ \\
        \hline
        valeurs initiales & 2 & 13 & 1 \\
        \hline
        après un tour de boucle & \comp{$2^2$} &\comp{6} & \comp{$2$} \\
        \hline
        après deux tours de boucle & \comp{$2^4$} &\comp{3} & \comp{$2$} \\
		\hline
		après trois tours de boucle & \comp{$2^8$} &\comp{1} & \comp{$2^5$} \\
		\hline
        après quatre tours de boucle & \comp{$2^{16}$} &\comp{0} & \comp{$2^{13}$} \\
		\hline
      \end{tabular}
      }
      \Question{Donner une implémentation de cet algorithme en langage C sous la forme d'une fonction {\tt exp\_rapide} de signature \mintinline{c}{double exp_rapide(double a, int n)}. On précisera soigneusement la spécification de cette fonction en commentaire dans le code et on vérifiera les préconditions à l'aide d'instructions {\tt assert}.}
	  \ifcorrige
	  \corpartC{exp_rapide.c}{}{}{4}{19}
	  \fi
      \Question{Prouver que cet algorithme termine.}
	  \tcor{
	 Dans l'algorithme ci-dessus, la quantité $n$ est un variant de boucle, en effet :
	 \begin{enumerate}
		\item $n \in \N$ par précondition.
		\item $n$ reste  positif  par condition d'entrée dans la boucle.
		\item $n$ décroit strictement car $n$ est divisé par 2  lors de chaque passage dans la boucle.
	  \end{enumerate}
	  L'algorithme termine car on a trouvé un variant de boucle.
	  }
      \Question{Prouver que cet algorithme est correct. En notant $a_0$ (resp. $n_0$) la valeur initiale de $a$ (resp. $n$), on pourra prouver l'invariant $p \times a^{n} = a_0^{n_0}$.}
	  \tcor{
	  On note, $a_0$ la valeur initiale de $a$ et $n_0$ la valeur initiale de $n$, montrons que la propriété $I$ : \og{} $p \times a^{n} = a_0^{n_0}$ \fg est un invariant de boucle.
	  \begin{enumerate}
		\item Avant d'entrée dans la boucle $p=1$, $a= a_0$ et $n=n_0$ donc $p \times a^{n} = a_0^{n_0}$ et  $I$ est vérifiée.
		\item On suppose $I$ vérifié à l'entrée de la boucle et on note $a'$ (resp. $n'$, resp. $p'$) les valeurs prises par $a$ (resp. $n$, resp. $p$) au tour de boucle suivant, alors : \\
		\begin{itemize}
            \item Si $n$ est paire, $n'=n/2$, $p'=p$ et $a' = a^2$ donc \\
            $p' \times a'^{n'} = p \times (a^2)^{n/2}$ \\
            $p' \times a'^{n'} = p \times a^{n}$ \\ et puisque $I$ était vraie en entrée de boucle,
            $p' \times a'^{n'} = a_0 ^ {n_0}$\\
            \item Sinon, $n'=(n-1)/2$, $p'=p \times a $ et $a' = a^2$ donc \\
            $p' \times a'^{n'} = p \times a \times (a^2)^{(n-1)/2}$ \\
            $p' \times a'^{n'} = p \times a^{n}$ \\ et puisque $I$ était vraie en entrée de boucle,
            $p' \times a'^{n'} = a_0 ^ {n_0}$\\
        \end{itemize}
	  \end{enumerate}
	  En sortie de boucle, puisque $n=0$, cet invariant donne $p \times a^0 = a_0 ^ {n_0}$ et donc $p =a_0 ^ {n_0}$ et donc l'algorithme est correcte.
	  }
    \Question{Donner une implémentation récursive de l'algorithme d'exponentiation rapide en OCaml sous la forme d'une fonction  {\tt exp\_rapide float -> int -> float}.}
    \ifcorrige
	  \corpartOCaml{exp_rapide.ml}{}{}{1}{4}
	\fi
\end{Exercise}


\begin{Exercise}[title={Pointeurs}]
	\Question{Compléter le tableau suivant, qui donne l'état des variables au fur et à mesure des instructions données dans la première colonne (on a indiqué par \faTimes{} une variable non encore déclarée.)
		\begin{center}
			\def\arraystretch{1.2}
			\setlength\tabcolsep{0.5cm}
			\begin{tabular}{|l|c|c|c|c|}
                \hline
				        instructions            & a  & b  & p & q       \\
				\hline
				{\tt int a = 14;}      & 14 & \faTimes & \faTimes & \faTimes      \\
				{\tt int b = 42;}      & 14  & \comp{42}  & \comp{\faTimes}  &  \comp{\faTimes}           \\
				{\tt int *p = \&a;}      & 14   &  \comp{42}  &   \&a     & \comp{\faTimes}     \\
                {\tt int *q = \&b;}      & 14   &  \comp{42}  &   \&a   & \comp{\&b}     \\
				{\tt *p = *p + *q ;} &  \comp{56} &  \comp{42}  &   \comp{\&a}    & \comp{\&b}     \\
				{\tt *q = *p - *q ;} &  \comp{56} &  \comp{14}  &   \comp{\&a}    & \comp{\&b}     \\
                {\tt *p = *p - *q ;} &  \comp{42} &  \comp{14}  &   \comp{\&a}    & \comp{\&b}     \\
                \hline
			\end{tabular}
		\end{center}}
    \Question{Ecrire une fonction {\tt echange} en C qui prend en argument deux pointeurs vers des entiers, ne renvoie rien et échange les valeurs de ces deux entiers \textit{sans utiliser de variable temporaire}.}
    \ifcorrige
	\corpartC{echange.c}{}{\small}{3}{8}
	\fi
    \Question{Compléter le programme suivant en écrivant l'appel à la fonction {\tt echange} afin d'échanger les valeurs des entiers {\tt n} et {\tt m} 
	\inputpartC{echange.c}{}{\small}{12}{13}}
    \ifcorrige
	\corpartC{echange.c}{}{\small}{12}{14}
	\fi
\end{Exercise}

\begin{Exercise}[title = {Quelques expression en OCaml}]\\
Pour chacune des expressions ci-dessous,  indiquer son type et sa valeur lorsqu'elle s'évalue sans erreur. Sinon indiquer la cause de l'erreur rencontrée.
\Question{\mintinline{ocaml}{let n = 24 mod 7;;}}
\tcor{{\tt n} est un entier (type {\tt int}) valant 3 (reste dans la division euclidienne de 24 par 7.)}
\Question{\mintinline{ocaml}{let perimetre = 4 *. 2.5;;}}
\tcor{L'évaluation donne une erreur car l'opérateur {\tt *.} est la multiplication entre deux opérandes de type {\tt float}, ici l'une des opérandes ({\tt 4}) est un {\tt int}.}
\Question{\mintinline{ocaml}{let v = 2.0**10;;}}
\tcor{En OCaml l'opérateur {\tt **} est l'exponentiation, mais il n'est défini que pour deux opérandes de type {\tt float}, on obtient de nouveau une erreur puisque l'une des opérandes est entière. Pour calculer $2^{10}$, il faudrait écrire \mintinline{ocaml}{let v = 2.0**10.;;} }
\Question{\mintinline{ocaml}{let at = '@' in print_char at;;}}
\tcor{Cette expression s'évalue correctement ({\tt at} est bien de type {\tt char} car entre simple quotes), elle renvoie {\tt ()} de type {\tt unit} car c'est un affichage.}
\Question{\mintinline{ocaml}{let coucou = let message = "Bonjour " + "tout le monde" in print_string message;;}}
\tcor{L'opérateur {\tt +} est l'addition de deux entiers, on obtient donc une erreur, l'opérateur de concaténation entre deux chaines de caractères est {\tt \textasciicircum}.}
\Question{\mintinline{ocaml}{let peri = let cote = 5 in 4*cote;;}}
\tcor{L'expression s'évalue correctement et vaut l'entier {\tt 20}.}
\Question{\mintinline{ocaml}{let k = if 2=1+1 then 'A' else 'B';;}}
\tcor{L'expression s'évalue correctement et vaut {\tt 'A'} (type {\tt char}).}
\Question{\mintinline{ocaml}{let rec fact n = if n=0 then 1 else n* fact (n-1);;}}
\tcor{L'expression s'évalue correctement,  c'est une fonction (type {\tt fun}) {\tt fact : int -> int} (qui calcule la factorielle de {\tt n})}
\end{Exercise}


\begin{Exercise}[title = {Un tableau qui connait sa taille}] \\
    En C, on propose le type structuré suivant afin de représenter un \og{} tableau d'entiers qui connaît sa taille \fg{} :
    \inputpartC{tt.c}{}{}{6}{11}
    La champ {\tt taille} contient la taille du tableau et le champ {\tt valeurs} est un pointeur vers une zone mémoire contenant la liste des valeurs du tableau.
	\Question{Ecrire une fonction {\tt somme} de signature \mintinline{c}{int somme(tableau t)} qui renvoie la somme des valeurs contenus dans {\tt t}.}
	\ifcorrige
	\corpartC{tt.c}{}{}{69}{77}
	\fi
    \Question{On veut écrire une fonction {\tt cree\_tableau} de signature \mintinline{c}{tableau cree_tableau(int val, int taille)} qui renvoie un {\tt tableau} de taille {\tt taille} dont toutes les valeurs sont initialisées à {\tt val}. La solution proposée ci-dessous (appelée {\tt cree\_tableau\_bug}) compile sans erreur et sans avertissement (avec les options {\tt -Wall} et {\tt -Wextra}) mais ne fonctionne pas correctement (on obtient une erreur à l'exécution ou les valeurs présentes dans le tableau ne sont pas égales à {\tt val}).
    \inputpartC{tt.c}{}{}{25}{36}
	Expliquer ce comportement en utilisant vos connaissances sur le modèle mémoire du langage C (on pourra illustrer par un schéma).}
	\tcor{
		Le tableau {\tt tab} déclaré à la ligne 5 est stocké sur la pile car c'est une variable locale à la fonction, aussi {\tt t.valeurs} pointe sur la pile dans une zone mémoire qui sera libéré à la sortie de la fonction car le contexte d'appel de la fonction (et donc les variables locales) est alors désalloué. Pour éviter ce problème, on doit allouer sur le tas à la ligne 5 à l'aide d'une instruction {\tt malloc}. Cela permettra de {\tt t.valeurs} vers un zone mémoire qui sera conservée intacte à la sortie de la fonction.
	}
	\Question{Proposer une version correcte de la fonction {\tt cree\_tableau}.}
	\ifcorrige
	\corpartC{tt.c}{}{}{13}{23}
	\fi
	\Question{Le \textit{crible d'Erastothène} est un algorithme permettant de trouver tous les nombres premiers inférieurs à un entier $N$ donné. Il procède en parcourant la liste des entiers de 2 à la racine carrée de $N$ en supprimant les multiples des nombres non encore éliminés rencontrés. Par exemple pour trouver les nombres premiers inférieurs à 20, on part de la liste  des entiers de 2 à 20 :
	\begin{itemize}
	\item \underline{2}, 3, 4, 5, 6, 7, 8, 9, 10, 11, 12, 13, 14, 15, 16, 17, 18, 19, 20 et on barre les multiples de 2  (excepté 2 lui-même).
	\item 2, \underline{3}, \cancel{4}, 5, \cancel{6}, 7, \cancel{8}, 9, \cancel{10}, 11, \cancel{12}, 13, \cancel{14}, 15, \cancel{16}, 17, \cancel{18}, 19, \cancel{20} on barre les multiples de 3 (excepté 3)
	\item 2, 3, \cancel{4}, \underline{5}, \cancel{6}, 7, \cancel{8}, \cancel{9}, \cancel{10}, 11, \cancel{12}, 13, \cancel{14}, \cancel{15}, \cancel{16}, 17, \cancel{18}, 19, \cancel{20} comme $5 > \sqrt{20}$, l'algorithme s'arrête.
	\end{itemize}
	Les nombres premiers inférieurs à 20 sont les nombres non barrés c'est à dire $\{2, 3, 5, 7, 11, 13, 17, 19\}$.\\
	Afin d'implémenter cet algorithme en langage C, on propose d'utiliser un tableau de booléens {\tt premiers} de taille $N+1$ et de mettre {\tt premiers[i]} à {\tt false} lorsqu'on barre {\tt i}. Donc on initialise le tableau à {\tt true} sauf {\tt premiers[0]} et {\tt premiers[1]} qui valent {\tt false} puis on parcourt ce tableau à l'aide d'un indice {\tt i} (de 2 à $\sqrt{N}$), si {\tt premiers[i]} est à {\tt true} alors on met tous les {\tt premiers[k]} où {\tt k} est un multiple de {\tt i} strictement plus grand que {\tt i} à {\tt false}. Ce parcours s'arrête dès que l'entier {\tt i} est supérieur à $\sqrt{N}$. Ecrire cette implémentation sous la forme d'une fonction {\tt crible} de signature \mintinline{c}{bool *crible(int n)
} qui prend en argument un entier {\tt n} et renvoie un tableau de booléens {\tt premiers} tel que {\tt premiers[i]} vaut {\tt true} si et seulement si {\tt i} est premier. On supposera \textit{déjà écrite} une fonction {\tt sqrt} de signature {\tt int sqrt(int n)} qui renvoie la partie entière de $\sqrt{\mathtt{n}}$.
	}
	\ifcorrige
	\corpartC{tt.c}{}{}{111}{134}
	\fi
	\Question{En utilisant la question précédente, écrire une fonction {\tt nombres\_premiers} de signature \\ \mintinline{c}{tableau nombres_premiers(int n)} qui prend en argument un entier {\tt n} et renvoie un {\tt tableau} contenant les nombres premiers inférieurs ou égaux à {\tt int n}. Par exemple, si {\tt n=20}, cette fonction renvoie un tableau {\tt t}, telle que {\tt t.taille=8} et contenant les valeurs  $\{2, 3, 5, 7, 11, 13, 17, 19\}$.}
	\ifcorrige
	\corpartC{tt.c}{}{}{136}{159}
	\fi
	\Question{Expliquer rapidement pourquoi la fonction {\tt nombres\_premiers} (qui utilise {\tt crible}) doit nécessairement contenir une instruction {\tt free}.}
	\tcor{La fonction {\tt nombres\_premiers} fait appel à {\tt crible} qui alloue sur le tas un tableau de booléens, en dehors de la fonction {\tt nombres\_premiers} on ne dispose pas de référence vers ce tableau, il doit donc être libéré depuis cette fonction (ligne 22 du programme précédent).}
\end{Exercise}

\begin{Exercise}[title = {Implémentation des entiers par représentation binaire}]

On rappelle qu'en C, le type {\tt uint64\_t} (disponible dans {\tt stdint.h} qu'on suppose déjà importée dans la suite de l'exercice) représente des entiers \textit{positifs} (non signés) sur 64 bits. D'autre part on rappelle que le spécificateur de format permettant d'afficher un entier de type {\tt uint64\_t} est {\tt \%lu}.

\Question{A propos du format {\tt uint64\_t}.}
\subQuestion{Donner l'intervalle d'entiers représentable avec ce format.}
\tcor{Les entiers représentables avec ce format sont $\intN{0}{2^{64}-1}$.}
\subQuestion{En compilant puis en exécutant le programme suivant sur un ordinateur (les librairies {\tt <stdio.h>} et {\tt <stdint.h>}  sont supposées importées):
\inputpartC{ex2.c}{}{}{4}{9}

on a obtenu l'affichage suivant dans le terminal : {\tt a= 18446744073709551615}. Expliquer cet affichage, s'agit-il d'un comportement indéfini ?}
\tcor{Comme {\tt a} est un entier non signé initialisé à 0, l'instruction {\tt a = a -1} est un dépassement de capacité. Ce n'est pas un comportement indéfini, sur les entiers non signés les calculs sont faits modulo le plus grand entier représentable plus un et donc ici on obtient donc $2^{64}-1$.}
\Question{Représentation des ensembles. \\On utilise à présent les entiers au format {\tt uint64\_t} afin de représenter des ensembles. A chaque entier écrit en base 2 on associe l'ensemble dont les éléments sont les positions des bits égaux à 1. Par exemple :
\begin{itemize}
	\item L'entier $\base{11001}{2}$(=$\base{25}{10}$) a des bits égaux à 1 aux positions 0,3 et 4 et donc représente l'ensemble $\{0,3,4\}$.
	\item L'entier $\base{10000000}{2}$(=$\base{128}{10}$) a un seul bit égal à 1 en position 7 et donc représente l'ensemble $\{7\}$.
	\item L'ensemble $\{1,5\}$ est représenté par l'entier ayant des bits égaux à 1 en position 1 et 5, c'est à dire $\base{100010}{2}=\base{34}{10}$.
\end{itemize}
}
\subQuestion{Quels sont les ensembles ainsi représentables ?}
\tcor{Les ensembles répresentables sont les parties de $\intN{0}{63}$}
\subQuestion{Donner l'écriture en base 10 de l'entier représentant l'ensemble $\{ 2,7\}$}
\tcor{L'entier représentant $\{ 2,7\}$ est $\base{10000100}{2} = \base{132}{10}$.}
\subQuestion{Quel est l'ensemble codé par l'entier $\base{76}{10}$ ?}
\tcor{ $\base{76}{10} = \base{1001100}{2}$ et donc code l'ensemble $\{ 2, 3, 6 \}$. }
\subQuestion{Donner la caractérisation des ensembles représentés par une puissance exacte de 2 (on ne demande pas de justification).}
\tcor{ Les ensembles représentés par une puissance exacte de 2 sont les singletons.}
\subQuestion{Ecrire une fonction {\tt appartient} de signature \mintinline{c}{bool appartient(uint64_t s, int e)} qui prend en argument un entier {\tt s} (type {\tt uint64\_t}) représentant un ensemble et un entier {\tt e} et renvoie {\tt true} si {\tt e} appartient à l'ensemble représenté par {\tt s} et {\tt false sinon}. Par exemple puisque l'ensemble $\{1,5\}$ est codé par $34$, {\tt appartient(34,1)} doit renvoyer {\tt true} tandis que {\tt appartient(34,2)} doit renvoyer {\tt false}.}
\tcor{On utilise l'algorithme des divisions successives de façon à trouver le bit de rang {\tt e}, on renvoie {\tt true} si ce bit est à 1 et {\tt false sinon}}
\ifcorrige
\corpartC{code_entiers.c}{}{}{12}{20}
\fi
\tcor { Les opérations bit à bit ne sont pas au programme, mais ils permettent ici d'écrire une solution bien plus concise. L'opérateur \mintinline{c}{>>e} décale les bits de s de e rang vers la droite et \mintinline{c}{& 1} permet de récupérer le dernier bit.}
\ifcorrige
\corpartC{code_entiers.c}{}{}{7}{10}
\fi
\subQuestion{Ecrire une fonction {\tt encode} en C de signature \mintinline{c}{uint64_t encode(bool tab[])}, qui prend en argument un tableau {\tt tab} de 64 booléens et renvoie l'entier au format {\tt uint64\_t} qui représente l'ensemble dont les éléments sont les entiers {\tt i} tels que {\tt tab[i]=true}. Par exemple, si {\tt tab} est le tableau de booléens de taille 64 ne contenant que des {\tt false} sauf {\tt tab[3]} et {\tt tab[10]} qui valent {\tt true} alors, {\tt encode(tab)} doit renvoyer l'entier qui représente l'ensemble {\tt \{3, 10\}}.}
\ifcorrige
\corpartC{code_entiers.c}{}{}{22}{35}
\fi
\subQuestion{Ecrire une fonction {\tt decode} de signature \mintinline{c}{bool *decode(uint64_t n)}, qui prend en argument un entier {\tt n} au format {\tt uint64\_t} et renvoie l'ensemble qu'il représente sous la forme d'un tableau {\tt tab} de 64 booléens tels que {\tt tab[i]=true} si et seulement si {\tt i} appartient à l'ensemble représenté par {\tt n}. Par exemple {\tt decode(34)} doit renvoyer un tableau {\tt tab} de booléens dont toutes les valeurs sont {\tt false} sauf {\tt tab[1]} et {\tt tab[5]} qui valent {\tt true}.}
\ifcorrige
\corpartC{code_entiers.c}{}{}{37}{46}
\fi
\end{Exercise}

\end{document}
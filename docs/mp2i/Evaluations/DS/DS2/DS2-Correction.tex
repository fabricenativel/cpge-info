\PassOptionsToPackage{dvipsnames,table}{xcolor}
\documentclass[11pt,a4paper]{article}

\usepackage{DS}

\begin{document}
\newcommand{\ModeExercice}{
% Traduction des noms pour le package exercise
\renewcommand{\ExerciseName}{Exercice}
\renewcommand{\thesubQuestion}{\theQuestion.\alph{subQuestion}}
\renewcommand{\AnswerName}{Réponses à l'exercise }
}
\newcommand{\Fiche}[2]{\lhead{\textbf{{\sc #1}}}
\rhead{Niveau: \textbf{#2}}
\cfoot{}}
\definecolor{cfond}{gray}{0.4}
\renewcommand{\thealgocf}{}

\newcommand{\ModeActivite}{
% Traduction des noms pour le package exercise
\renewcommand{\ExerciseName}{Activité}
}
% Réglages de la mise en forme des exercices 
\renewcommand{\ExerciseHeaderTitle}{\ExerciseTitle}
\renewcommand{\ExerciseHeaderOrigin}{\ExerciseOrigin}
% Un : sépare le numéro de l'exerice de son titre ... si le titre existe. On utilise Origine pour placer les pictogrammes en fin de ligne
\renewcommand{\ExerciseHeader}{\ding{113} \textbf{\sffamily{\ExerciseName \ \ExerciseHeaderNB}} \ifthenelse{\equal{\ExerciseTitle}{\empty}}{}{:} \textit{\ExerciseHeaderTitle} \hfill \ExerciseHeaderOrigin}
\renewcommand{\ExePartHeader}{\quad {\footnotesize \ding{110}} \textbf{Partie \textbf{\ExePartHeaderNB}} : \ExePartName}


% Mode Concours
\newcommand{\ModeConcours}{
   \newcounter{qconcours}
   \setcounter{qconcours}{1}
   \renewcommand{\ExerciseName}{Exercice}
   \renewcommand{\ExerciseHeaderTitle}{\ExerciseTitle}
\renewcommand{\ExerciseHeaderOrigin}{\ExerciseOrigin}
\renewcommand{\ExerciseHeader}{\ding{113} \textbf{\sffamily{\ExerciseName \ \ExerciseHeaderNB}} \ifthenelse{\equal{\ExerciseTitle}{\empty}}{}{:} \textit{\ExerciseHeaderTitle} \hfill \ExerciseHeaderOrigin}
\renewcommand{\QuestionNB}{\textbf{Q\arabic{qconcours}--}\ \addtocounter{qconcours}{1}}}

\newcommand{\noeud}[1]{\Tr{\fbox{\tt #1}}}
\newcommand{\FPATH}{/home/fenarius/Travail/Cours/cpge-info/docs/mp2i/}
\newcommand{\spath}[2]{\FPATH Evaluations/#1/#1#2}
\definecolor{codebg}{gray}{0.90}
\newcommand{\inputpartOCaml}[5]{\begin{mdframed}[backgroundcolor=codebg] \inputminted[breaklines=true,fontsize=#3,linenos=true,highlightcolor=fluo,tabsize=2,highlightlines={#2},firstline=#4,lastline=#5,firstnumber=1]{OCaml}{#1} \end{mdframed}}
\newcommand{\inputpartPython}[5]{\begin{mdframed}[backgroundcolor=codebg] \inputminted[breaklines=true,fontsize=#3,linenos=true,highlightcolor=fluo,tabsize=2,highlightlines={#2},firstline=#4,lastline=#5,firstnumber=1]{python}{#1} \end{mdframed}}
\newcommand{\inputpartC}[5]{\begin{mdframed}[backgroundcolor=codebg] \inputminted[breaklines=true,fontsize=#3,linenos=true,highlightcolor=fluo,tabsize=2,highlightlines={#2},firstline=#4,lastline=#5,firstnumber=1]{c}{#1} \end{mdframed}}
\newcommand{\inputC}[2]{\begin{mdframed}[backgroundcolor=codebg] \inputminted[breaklines=true,fontsize=#2,linenos=true,highlightcolor=fluo,tabsize=2]{c}{#1} \end{mdframed}}
\newminted[langageC]{c}{linenos=true,escapeinside=``,highlightcolor=fluo,tabsize=2}
\newminted[python]{python}{linenos=true,escapeinside=``,highlightcolor=fluo,tabsize=4}
\BeforeBeginEnvironment{minted}{\begin{mdframed}[backgroundcolor=codebg,skipabove=0cm]}
   \AfterEndEnvironment{minted}{\end{mdframed}}

% Font light medium et bold pour tt :
\newcommand{\ttl}[1]{\ttfamily \fontseries{l}\selectfont #1}
\newcommand{\ttm}[1]{\ttfamily \fontseries{m}\selectfont #1}
\newcommand{\ttb}[1]{\ttfamily \fontseries{b}\selectfont #1}

%QCM de NSI \QNSI{Question}{R1}{R2}{R3}{R4}
\newcommand{\QNSI}[5]{
#1
\begin{enumerate}[label=\alph{enumi})]
\item #2
\item #3
\item #4
\item #5
\end{enumerate}
}



\definecolor{grispale}{gray}{0.95}
\newcommand{\htmlmode}{\lstset{language=html,numbers=left, tabsize=2, frame=single, breaklines=true, keywordstyle=\ttfamily, basicstyle=\small,
   numberstyle=\tiny\ttfamily, framexleftmargin=0mm, backgroundcolor=\color{grispale}, xleftmargin=12mm,showstringspaces=false}}
\newcommand{\pythonmode}{\lstset{language=python,numbers=left, tabsize=4, frame=single, breaklines=true, keywordstyle=\ttfamily, basicstyle=\small,
   numberstyle=\tiny\ttfamily, framexleftmargin=0mm, backgroundcolor=\color{grispale}, xleftmargin=12mm, showstringspaces=false}}
\newcommand{\bashmode}{\lstset{language=bash,numbers=left, tabsize=2, frame=single, breaklines=true, basicstyle=\ttfamily,
   numberstyle=\tiny\ttfamily, framexleftmargin=0mm, backgroundcolor=\color{grispale}, xleftmargin=12mm, showstringspaces=false}}
\newcommand{\exomode}{\lstset{language=python,numbers=left, tabsize=2, frame=single, breaklines=true, basicstyle=\ttfamily,
   numberstyle=\tiny\ttfamily, framexleftmargin=13mm, xleftmargin=12mm, basicstyle=\small, showstringspaces=false}}
   
   
  \lstset{%
        inputencoding=utf8,
        extendedchars=true,
        literate=%
        {é}{{\'{e}}}1
        {è}{{\`{e}}}1
        {ê}{{\^{e}}}1
        {ë}{{\¨{e}}}1
        {É}{{\'{E}}}1
        {Ê}{{\^{E}}}1
        {û}{{\^{u}}}1
        {ù}{{\`{u}}}1
        {ú}{{\'{u}}}1
        {â}{{\^{a}}}1
        {à}{{\`{a}}}1
        {á}{{\'{a}}}1
        {ã}{{\~{a}}}1
        {Á}{{\'{A}}}1
        {Â}{{\^{A}}}1
        {Ã}{{\~{A}}}1
        {ç}{{\c{c}}}1
        {Ç}{{\c{C}}}1
        {õ}{{\~{o}}}1
        {ó}{{\'{o}}}1
        {ô}{{\^{o}}}1
        {Õ}{{\~{O}}}1
        {Ó}{{\'{O}}}1
        {Ô}{{\^{O}}}1
        {î}{{\^{i}}}1
        {Î}{{\^{I}}}1
        {í}{{\'{i}}}1
        {Í}{{\~{Í}}}1
}

%tei pour placer les images
%tei{nom de l’image}{échelle de l’image}{sens}{texte a positionner}
%sens ="1" (droite) ou "2" (gauche)
\newlength{\ltxt}
\newcommand{\tei}[4]{
\setlength{\ltxt}{\linewidth}
\setbox0=\hbox{\includegraphics[scale=#2]{#1}}
\addtolength{\ltxt}{-\wd0}
\addtolength{\ltxt}{-10pt}
\ifthenelse{\equal{#3}{1}}{
\begin{minipage}{\wd0}
\includegraphics[scale=#2]{#1}
\end{minipage}
\hfill
\begin{minipage}{\ltxt}
#4
\end{minipage}
}{
\begin{minipage}{\ltxt}
#4
\end{minipage}
\hfill
\begin{minipage}{\wd0}
\includegraphics[scale=#2]{#1}
\end{minipage}
}
}

%Juxtaposition d'une image pspciture et de texte 
%#1: = code pstricks de l'image
%#2: largeur de l'image
%#3: hauteur de l'image
%#4: Texte à écrire
\newcommand{\ptp}[4]{
\setlength{\ltxt}{\linewidth}
\addtolength{\ltxt}{-#2 cm}
\addtolength{\ltxt}{-0.1 cm}
\begin{minipage}[b][#3 cm][t]{\ltxt}
#4
\end{minipage}\hfill
\begin{minipage}[b][#3 cm][c]{#2 cm}
#1
\end{minipage}\par
}



%Macros pour les graphiques
\psset{linewidth=0.5\pslinewidth,PointSymbol=x}
\setlength{\fboxrule}{0.5pt}
\newcounter{tempangle}

%Marque la longueur du segment d'extrémité  #1 et  #2 avec la valeur #3, #4 est la distance par rapport au segment (en %age de la valeur de celui ci) et #5 l'orientation du marquage : +90 ou -90
\newcommand{\afflong}[5]{
\pstRotation[RotAngle=#4,PointSymbol=none,PointName=none]{#1}{#2}[X] 
\pstHomO[PointSymbol=none,PointName=none,HomCoef=#5]{#1}{X}[Y]
\pstTranslation[PointSymbol=none,PointName=none]{#1}{#2}{Y}[Z]
 \ncline{|<->|,linewidth=0.25\pslinewidth}{Y}{Z} \ncput*[nrot=:U]{\footnotesize{#3}}
}
\newcommand{\afflongb}[3]{
\ncline{|<->|,linewidth=0}{#1}{#2} \naput*[nrot=:U]{\footnotesize{#3}}
}

%Construis le point #4 situé à #2 cm du point #1 avant un angle #3 par rapport à l'horizontale. #5 = liste de paramètre
\newcommand{\lsegment}[5]{\pstGeonode[PointSymbol=none,PointName=none](0,0){O'}(#2,0){I'} \pstTranslation[PointSymbol=none,PointName=none]{O'}{I'}{#1}[J'] \pstRotation[RotAngle=#3,PointSymbol=x,#5]{#1}{J'}[#4]}
\newcommand{\tsegment}[5]{\pstGeonode[PointSymbol=none,PointName=none](0,0){O'}(#2,0){I'} \pstTranslation[PointSymbol=none,PointName=none]{O'}{I'}{#1}[J'] \pstRotation[RotAngle=#3,PointSymbol=x,#5]{#1}{J'}[#4] \pstLineAB{#4}{#1}}

%Construis le point #4 situé à #3 cm du point #1 et faisant un angle de  90° avec la droite (#1,#2) #5 = liste de paramètre
\newcommand{\psegment}[5]{
\pstGeonode[PointSymbol=none,PointName=none](0,0){O'}(#3,0){I'}
 \pstTranslation[PointSymbol=none,PointName=none]{O'}{I'}{#1}[J']
 \pstInterLC[PointSymbol=none,PointName=none]{#1}{#2}{#1}{J'}{M1}{M2} \pstRotation[RotAngle=-90,PointSymbol=x,#5]{#1}{M1}[#4]
  }
  
%Construis le point #4 situé à #3 cm du point #1 et faisant un angle de  #5° avec la droite (#1,#2) #6 = liste de paramètre
\newcommand{\mlogo}[6]{
\pstGeonode[PointSymbol=none,PointName=none](0,0){O'}(#3,0){I'}
 \pstTranslation[PointSymbol=none,PointName=none]{O'}{I'}{#1}[J']
 \pstInterLC[PointSymbol=none,PointName=none]{#1}{#2}{#1}{J'}{M1}{M2} \pstRotation[RotAngle=#5,PointSymbol=x,#6]{#1}{M2}[#4]
  }

% Construis un triangle avec #1=liste des 3 sommets séparés par des virgules, #2=liste des 3 longueurs séparés par des virgules, #3 et #4 : paramètre d'affichage des 2e et 3 points et #5 : inclinaison par rapport à l'horizontale
%autre macro identique mais sans tracer les segments joignant les sommets
\noexpandarg
\newcommand{\Triangleccc}[5]{
\StrBefore{#1}{,}[\pointA]
\StrBetween[1,2]{#1}{,}{,}[\pointB]
\StrBehind[2]{#1}{,}[\pointC]
\StrBefore{#2}{,}[\coteA]
\StrBetween[1,2]{#2}{,}{,}[\coteB]
\StrBehind[2]{#2}{,}[\coteC]
\tsegment{\pointA}{\coteA}{#5}{\pointB}{#3} 
\lsegment{\pointA}{\coteB}{0}{Z1}{PointSymbol=none, PointName=none}
\lsegment{\pointB}{\coteC}{0}{Z2}{PointSymbol=none, PointName=none}
\pstInterCC{\pointA}{Z1}{\pointB}{Z2}{\pointC}{Z3} 
\pstLineAB{\pointA}{\pointC} \pstLineAB{\pointB}{\pointC}
\pstSymO[PointName=\pointC,#4]{C}{C}[C]
}
\noexpandarg
\newcommand{\TrianglecccP}[5]{
\StrBefore{#1}{,}[\pointA]
\StrBetween[1,2]{#1}{,}{,}[\pointB]
\StrBehind[2]{#1}{,}[\pointC]
\StrBefore{#2}{,}[\coteA]
\StrBetween[1,2]{#2}{,}{,}[\coteB]
\StrBehind[2]{#2}{,}[\coteC]
\tsegment{\pointA}{\coteA}{#5}{\pointB}{#3} 
\lsegment{\pointA}{\coteB}{0}{Z1}{PointSymbol=none, PointName=none}
\lsegment{\pointB}{\coteC}{0}{Z2}{PointSymbol=none, PointName=none}
\pstInterCC[PointNameB=none,PointSymbolB=none,#4]{\pointA}{Z1}{\pointB}{Z2}{\pointC}{Z1} 
}


% Construis un triangle avec #1=liste des 3 sommets séparés par des virgules, #2=liste formée de 2 longueurs et d'un angle séparés par des virgules, #3 et #4 : paramètre d'affichage des 2e et 3 points et #5 : inclinaison par rapport à l'horizontale
%autre macro identique mais sans tracer les segments joignant les sommets
\newcommand{\Trianglecca}[5]{
\StrBefore{#1}{,}[\pointA]
\StrBetween[1,2]{#1}{,}{,}[\pointB]
\StrBehind[2]{#1}{,}[\pointC]
\StrBefore{#2}{,}[\coteA]
\StrBetween[1,2]{#2}{,}{,}[\coteB]
\StrBehind[2]{#2}{,}[\angleA]
\tsegment{\pointA}{\coteA}{#5}{\pointB}{#3} 
\setcounter{tempangle}{#5}
\addtocounter{tempangle}{\angleA}
\tsegment{\pointA}{\coteB}{\thetempangle}{\pointC}{#4}
\pstLineAB{\pointB}{\pointC}
}
\newcommand{\TriangleccaP}[5]{
\StrBefore{#1}{,}[\pointA]
\StrBetween[1,2]{#1}{,}{,}[\pointB]
\StrBehind[2]{#1}{,}[\pointC]
\StrBefore{#2}{,}[\coteA]
\StrBetween[1,2]{#2}{,}{,}[\coteB]
\StrBehind[2]{#2}{,}[\angleA]
\lsegment{\pointA}{\coteA}{#5}{\pointB}{#3} 
\setcounter{tempangle}{#5}
\addtocounter{tempangle}{\angleA}
\lsegment{\pointA}{\coteB}{\thetempangle}{\pointC}{#4}
}

% Construis un triangle avec #1=liste des 3 sommets séparés par des virgules, #2=liste formée de 1 longueurs et de deux angle séparés par des virgules, #3 et #4 : paramètre d'affichage des 2e et 3 points et #5 : inclinaison par rapport à l'horizontale
%autre macro identique mais sans tracer les segments joignant les sommets
\newcommand{\Trianglecaa}[5]{
\StrBefore{#1}{,}[\pointA]
\StrBetween[1,2]{#1}{,}{,}[\pointB]
\StrBehind[2]{#1}{,}[\pointC]
\StrBefore{#2}{,}[\coteA]
\StrBetween[1,2]{#2}{,}{,}[\angleA]
\StrBehind[2]{#2}{,}[\angleB]
\tsegment{\pointA}{\coteA}{#5}{\pointB}{#3} 
\setcounter{tempangle}{#5}
\addtocounter{tempangle}{\angleA}
\lsegment{\pointA}{1}{\thetempangle}{Z1}{PointSymbol=none, PointName=none}
\setcounter{tempangle}{#5}
\addtocounter{tempangle}{180}
\addtocounter{tempangle}{-\angleB}
\lsegment{\pointB}{1}{\thetempangle}{Z2}{PointSymbol=none, PointName=none}
\pstInterLL[#4]{\pointA}{Z1}{\pointB}{Z2}{\pointC}
\pstLineAB{\pointA}{\pointC}
\pstLineAB{\pointB}{\pointC}
}
\newcommand{\TrianglecaaP}[5]{
\StrBefore{#1}{,}[\pointA]
\StrBetween[1,2]{#1}{,}{,}[\pointB]
\StrBehind[2]{#1}{,}[\pointC]
\StrBefore{#2}{,}[\coteA]
\StrBetween[1,2]{#2}{,}{,}[\angleA]
\StrBehind[2]{#2}{,}[\angleB]
\lsegment{\pointA}{\coteA}{#5}{\pointB}{#3} 
\setcounter{tempangle}{#5}
\addtocounter{tempangle}{\angleA}
\lsegment{\pointA}{1}{\thetempangle}{Z1}{PointSymbol=none, PointName=none}
\setcounter{tempangle}{#5}
\addtocounter{tempangle}{180}
\addtocounter{tempangle}{-\angleB}
\lsegment{\pointB}{1}{\thetempangle}{Z2}{PointSymbol=none, PointName=none}
\pstInterLL[#4]{\pointA}{Z1}{\pointB}{Z2}{\pointC}
}

%Construction d'un cercle de centre #1 et de rayon #2 (en cm)
\newcommand{\Cercle}[2]{
\lsegment{#1}{#2}{0}{Z1}{PointSymbol=none, PointName=none}
\pstCircleOA{#1}{Z1}
}

%construction d'un parallélogramme #1 = liste des sommets, #2 = liste contenant les longueurs de 2 côtés consécutifs et leurs angles;  #3, #4 et #5 : paramètre d'affichage des sommets #6 inclinaison par rapport à l'horizontale 
% meme macro sans le tracé des segements
\newcommand{\Para}[6]{
\StrBefore{#1}{,}[\pointA]
\StrBetween[1,2]{#1}{,}{,}[\pointB]
\StrBetween[2,3]{#1}{,}{,}[\pointC]
\StrBehind[3]{#1}{,}[\pointD]
\StrBefore{#2}{,}[\longueur]
\StrBetween[1,2]{#2}{,}{,}[\largeur]
\StrBehind[2]{#2}{,}[\angle]
\tsegment{\pointA}{\longueur}{#6}{\pointB}{#3} 
\setcounter{tempangle}{#6}
\addtocounter{tempangle}{\angle}
\tsegment{\pointA}{\largeur}{\thetempangle}{\pointD}{#5}
\pstMiddleAB[PointName=none,PointSymbol=none]{\pointB}{\pointD}{Z1}
\pstSymO[#4]{Z1}{\pointA}[\pointC]
\pstLineAB{\pointB}{\pointC}
\pstLineAB{\pointC}{\pointD}
}
\newcommand{\ParaP}[6]{
\StrBefore{#1}{,}[\pointA]
\StrBetween[1,2]{#1}{,}{,}[\pointB]
\StrBetween[2,3]{#1}{,}{,}[\pointC]
\StrBehind[3]{#1}{,}[\pointD]
\StrBefore{#2}{,}[\longueur]
\StrBetween[1,2]{#2}{,}{,}[\largeur]
\StrBehind[2]{#2}{,}[\angle]
\lsegment{\pointA}{\longueur}{#6}{\pointB}{#3} 
\setcounter{tempangle}{#6}
\addtocounter{tempangle}{\angle}
\lsegment{\pointA}{\largeur}{\thetempangle}{\pointD}{#5}
\pstMiddleAB[PointName=none,PointSymbol=none]{\pointB}{\pointD}{Z1}
\pstSymO[#4]{Z1}{\pointA}[\pointC]
}


%construction d'un cerf-volant #1 = liste des sommets, #2 = liste contenant les longueurs de 2 côtés consécutifs et leurs angles;  #3, #4 et #5 : paramètre d'affichage des sommets #6 inclinaison par rapport à l'horizontale 
% meme macro sans le tracé des segements
\newcommand{\CerfVolant}[6]{
\StrBefore{#1}{,}[\pointA]
\StrBetween[1,2]{#1}{,}{,}[\pointB]
\StrBetween[2,3]{#1}{,}{,}[\pointC]
\StrBehind[3]{#1}{,}[\pointD]
\StrBefore{#2}{,}[\longueur]
\StrBetween[1,2]{#2}{,}{,}[\largeur]
\StrBehind[2]{#2}{,}[\angle]
\tsegment{\pointA}{\longueur}{#6}{\pointB}{#3} 
\setcounter{tempangle}{#6}
\addtocounter{tempangle}{\angle}
\tsegment{\pointA}{\largeur}{\thetempangle}{\pointD}{#5}
\pstOrtSym[#4]{\pointB}{\pointD}{\pointA}[\pointC]
\pstLineAB{\pointB}{\pointC}
\pstLineAB{\pointC}{\pointD}
}

%construction d'un quadrilatère quelconque #1 = liste des sommets, #2 = liste contenant les longueurs des 4 côtés et l'angle entre 2 cotés consécutifs  #3, #4 et #5 : paramètre d'affichage des sommets #6 inclinaison par rapport à l'horizontale 
% meme macro sans le tracé des segements
\newcommand{\Quadri}[6]{
\StrBefore{#1}{,}[\pointA]
\StrBetween[1,2]{#1}{,}{,}[\pointB]
\StrBetween[2,3]{#1}{,}{,}[\pointC]
\StrBehind[3]{#1}{,}[\pointD]
\StrBefore{#2}{,}[\coteA]
\StrBetween[1,2]{#2}{,}{,}[\coteB]
\StrBetween[2,3]{#2}{,}{,}[\coteC]
\StrBetween[3,4]{#2}{,}{,}[\coteD]
\StrBehind[4]{#2}{,}[\angle]
\tsegment{\pointA}{\coteA}{#6}{\pointB}{#3} 
\setcounter{tempangle}{#6}
\addtocounter{tempangle}{\angle}
\tsegment{\pointA}{\coteD}{\thetempangle}{\pointD}{#5}
\lsegment{\pointB}{\coteB}{0}{Z1}{PointSymbol=none, PointName=none}
\lsegment{\pointD}{\coteC}{0}{Z2}{PointSymbol=none, PointName=none}
\pstInterCC[PointNameA=none,PointSymbolA=none,#4]{\pointB}{Z1}{\pointD}{Z2}{Z3}{\pointC} 
\pstLineAB{\pointB}{\pointC}
\pstLineAB{\pointC}{\pointD}
}


% Définition des colonnes centrées ou à droite pour tabularx
\newcolumntype{Y}{>{\centering\arraybackslash}X}
\newcolumntype{Z}{>{\flushright\arraybackslash}X}

%Les pointillés à remplir par les élèves
\newcommand{\po}[1]{\makebox[#1 cm]{\dotfill}}
\newcommand{\lpo}[1][3]{%
\multido{}{#1}{\makebox[\linewidth]{\dotfill}
}}

%Liste des pictogrammes utilisés sur la fiche d'exercice ou d'activités
\newcommand{\bombe}{\faBomb}
\newcommand{\livre}{\faBook}
\newcommand{\calculatrice}{\faCalculator}
\newcommand{\oral}{\faCommentO}
\newcommand{\surfeuille}{\faEdit}
\newcommand{\ordinateur}{\faLaptop}
\newcommand{\ordi}{\faDesktop}
\newcommand{\ciseaux}{\faScissors}
\newcommand{\danger}{\faExclamationTriangle}
\newcommand{\out}{\faSignOut}
\newcommand{\cadeau}{\faGift}
\newcommand{\flash}{\faBolt}
\newcommand{\lumiere}{\faLightbulb}
\newcommand{\compas}{\dsmathematical}
\newcommand{\calcullitteral}{\faTimesCircleO}
\newcommand{\raisonnement}{\faCogs}
\newcommand{\recherche}{\faSearch}
\newcommand{\rappel}{\faHistory}
\newcommand{\video}{\faFilm}
\newcommand{\capacite}{\faPuzzlePiece}
\newcommand{\aide}{\faLifeRing}
\newcommand{\loin}{\faExternalLink}
\newcommand{\groupe}{\faUsers}
\newcommand{\bac}{\faGraduationCap}
\newcommand{\histoire}{\faUniversity}
\newcommand{\coeur}{\faSave}
\newcommand{\os}{\faMicrochip}
\newcommand{\rd}{\faCubes}
\newcommand{\data}{\faColumns}
\newcommand{\web}{\faCode}
\newcommand{\prog}{\faFile}
\newcommand{\algo}{\faCogs}
\newcommand{\important}{\faExclamationCircle}
\newcommand{\maths}{\faTimesCircle}
% Traitement des données en tables
\newcommand{\tables}{\faColumns}
% Types construits
\newcommand{\construits}{\faCubes}
% Type et valeurs de base
\newcommand{\debase}{{\footnotesize \faCube}}
% Systèmes d'exploitation
\newcommand{\linux}{\faLinux}
\newcommand{\sd}{\faProjectDiagram}
\newcommand{\bd}{\faDatabase}

%Les ensembles de nombres
\renewcommand{\N}{\mathbb{N}}
\newcommand{\D}{\mathbb{D}}
\newcommand{\Z}{\mathbb{Z}}
\newcommand{\Q}{\mathbb{Q}}
\newcommand{\R}{\mathbb{R}}
\newcommand{\C}{\mathbb{C}}

%Ecriture des vecteurs
\newcommand{\vect}[1]{\vbox{\halign{##\cr 
  \tiny\rightarrowfill\cr\noalign{\nointerlineskip\vskip1pt} 
  $#1\mskip2mu$\cr}}}


%Compteur activités/exos et question et mise en forme titre et questions
\newcounter{numact}
\setcounter{numact}{1}
\newcounter{numseance}
\setcounter{numseance}{1}
\newcounter{numexo}
\setcounter{numexo}{0}
\newcounter{numprojet}
\setcounter{numprojet}{0}
\newcounter{numquestion}
\newcommand{\espace}[1]{\rule[-1ex]{0pt}{#1 cm}}
\newcommand{\Quest}[3]{
\addtocounter{numquestion}{1}
\begin{tabularx}{\textwidth}{X|m{1cm}|}
\cline{2-2}
\textbf{\sffamily{\alph{numquestion})}} #1 & \dots / #2 \\
\hline 
\multicolumn{2}{|l|}{\espace{#3}} \\
\hline
\end{tabularx}
}
\newcommand{\mq}[1]
{\ding{113} \addtocounter{numquestion}{1}
\textbf{Question \arabic{numquestion}} \\ #1}
\newcommand{\QuestR}[3]{
\addtocounter{numquestion}{1}
\begin{tabularx}{\textwidth}{X|m{1cm}|}
\cline{2-2}
\textbf{\sffamily{\alph{numquestion})}} #1 & \dots / #2 \\
\hline 
\multicolumn{2}{|l|}{\cor{#3}} \\
\hline
\end{tabularx}
}
\newcommand{\Pre}{{\sc nsi} 1\textsuperscript{e}}
\newcommand{\Term}{{\sc nsi} Terminale}
\newcommand{\Sec}{2\textsuperscript{e}}
\newcommand{\Exo}[2]{ \addtocounter{numexo}{1} \ding{113} \textbf{\sffamily{Exercice \thenumexo}} : \textit{#1} \hfill #2  \setcounter{numquestion}{0}}
\newcommand{\Projet}[1]{ \addtocounter{numprojet}{1} \ding{118} \textbf{\sffamily{Projet \thenumprojet}} : \textit{#1}}
\newcommand{\ExoD}[2]{ \addtocounter{numexo}{1} \ding{113} \textbf{\sffamily{Exercice \thenumexo}}  \textit{(#1 pts)} \hfill #2  \setcounter{numquestion}{0}}
\newcommand{\ExoB}[2]{ \addtocounter{numexo}{1} \ding{113} \textbf{\sffamily{Exercice \thenumexo}}  \textit{(Bonus de +#1 pts maximum)} \hfill #2  \setcounter{numquestion}{0}}
\newcommand{\Act}[2]{ \ding{113} \textbf{\sffamily{Activité \thenumact}} : \textit{#1} \hfill #2  \addtocounter{numact}{1} \setcounter{numquestion}{0}}
\newcommand{\Seance}{ \rule{1.5cm}{0.5pt}\raisebox{-3pt}{\framebox[4cm]{\textbf{\sffamily{Séance \thenumseance}}}}\hrulefill  \\
  \addtocounter{numseance}{1}}
\newcommand{\Acti}[2]{ {\footnotesize \ding{117}} \textbf{\sffamily{Activité \thenumact}} : \textit{#1} \hfill #2  \addtocounter{numact}{1} \setcounter{numquestion}{0}}
\newcommand{\titre}[1]{\begin{Large}\textbf{\ding{118}}\end{Large} \begin{large}\textbf{ #1}\end{large} \vspace{0.2cm}}
\newcommand{\QListe}[1][0]{
\ifthenelse{#1=0}
{\begin{enumerate}[partopsep=0pt,topsep=0pt,parsep=0pt,itemsep=0pt,label=\textbf{\sffamily{\arabic*.}},series=question]}
{\begin{enumerate}[resume*=question]}}
\newcommand{\SQListe}[1][0]{
\ifthenelse{#1=0}
{\begin{enumerate}[partopsep=0pt,topsep=0pt,parsep=0pt,itemsep=0pt,label=\textbf{\sffamily{\alph*)}},series=squestion]}
{\begin{enumerate}[resume*=squestion]}}
\newcommand{\SQListeL}[1][0]{
\ifthenelse{#1=0}
{\begin{enumerate*}[partopsep=0pt,topsep=0pt,parsep=0pt,itemsep=0pt,label=\textbf{\sffamily{\alph*)}},series=squestion]}
{\begin{enumerate*}[resume*=squestion]}}
\newcommand{\FinListe}{\end{enumerate}}
\newcommand{\FinListeL}{\end{enumerate*}}

%Mise en forme de la correction
\newboolean{corrige}
\setboolean{corrige}{false}
\newcommand{\scor}[1]{\par \textcolor{blue!75!black}{\small #1}}
\newcommand{\cor}[1]{\par \textcolor{blue!75!black}{#1}}
\newcommand{\br}[1]{\cor{\textbf{#1}}}
\newcommand{\tcor}[1]{
\ifthenelse{\boolean{corrige}}{\begin{tcolorbox}[width=\linewidth,colback={white},colbacktitle=white,coltitle=green!50!black,colframe=green!50!black,boxrule=0.2mm]   
\cor{#1}
\end{tcolorbox}}{}
}
\newcommand{\iscor}[1]{\ifthenelse{\boolean{corrige}}{#1}}
\newcommand{\rc}[1]{\textcolor{OliveGreen}{#1}}

%Référence aux exercices par leur numéro
\newcommand{\refexo}[1]{
\refstepcounter{numexo}
\addtocounter{numexo}{-1}
\label{#1}}

%Séparation entre deux activités
\newcommand{\separateur}{\begin{center}
\rule{1.5cm}{0.5pt}\raisebox{-3pt}{\ding{117}}\rule{1.5cm}{0.5pt}  \vspace{0.2cm}
\end{center}}

%Entête et pied de page
\newcommand{\snt}[1]{\lhead{\textbf{SNT -- La photographie numérique} \rhead{\textit{Lycée Nord}}}}
\newcommand{\Activites}[2]{\lhead{\textbf{{\sc #1}}}
\rhead{Activités -- \textbf{#2}}
\cfoot{}}
\newcommand{\Exos}[2]{\lhead{\textbf{Fiche d'exercices: {\sc #1}}}
\rhead{Niveau: \textbf{#2}}
\cfoot{}}
\newcommand{\TD}[2]{\lhead{\textbf{TD #1} : {\sc #2} }
\rhead{{\sc mp2i -- Lycée Leconte de Lisle}}
\cfoot{}}
\newcommand{\Colles}[2]{\lhead{{\sc mp2i -- }\textbf{Colles d'informatique #1}} 
\rhead{{\sc #2}}
\cfoot{}}
\newcommand{\Devoir}[2]{\lhead{\textbf{Devoir de mathématiques : {\sc #1}}}
\rhead{\textbf{#2}} \setlength{\fboxsep}{8pt}
\begin{center}
%Titre de la fiche
\fbox{\parbox[b][1cm][t]{0.3\textwidth}{Nom : \hfill \po{3} \par \vfill Prénom : \hfill \po{3}} } \hfill 
\fbox{\parbox[b][1cm][t]{0.6\textwidth}{Note : \po{1} / 20} }
\end{center} \cfoot{}}
\newcommand{\TPnote}[3]{\lhead{\textbf{TP noté d'informatique n° #1}}
\rhead{\textbf{#2}} \setlength{\fboxsep}{8pt}
\ifthenelse{\boolean{corrige}}{}
{\begin{center}
\fbox{\parbox[b][1cm][t]{0.3\textwidth}{Nom : \hfill \po{3} \par \vfill Prénom : \hfill \po{3}} } \hfill 
\fbox{\parbox[b][1cm][t]{0.6\textwidth}{Note : \po{1} / #3} }
\end{center}} \cfoot{}}
\newcommand{\IC}[2]{\lhead{\textbf{Interro de cours n° #1}}
\rhead{{\sc mp2i --} \textbf{#2}} \setlength{\fboxsep}{8pt}
\ifthenelse{\boolean{corrige}}{}
{\begin{center}
%Titre de la fiche
\fbox{\parbox[b][1cm][t]{0.3\textwidth}{Nom : \hfill \po{3} \par \vfill Prénom : \hfill \po{3}} } \hfill 
\fbox{\parbox[b][1cm][t]{0.6\textwidth}{Note : \po{1} / 10} }
\end{center}}\cfoot{}}
\newcommand{\DS}[3]{\lhead{{#1} : \textbf{DS d'informatique n° #2}}
\rhead{Lycée Leconte de Lisle -- #3} \setlength{\fboxsep}{8pt}
%Titre de la fiche
\begin{center}
   {\Large \textbf{Devoir surveillé d'informatique}}
\end{center} \cfoot{\thepage/\pageref{LastPage}}}

\newcommand{\CB}[2]{\lhead{{#1} : \textbf{Councours blanc - Informatique}}
\rhead{Lycée Leconte de Lisle -- #2} \setlength{\fboxsep}{8pt}
%Titre de la fiche
\begin{center}
   {\Large \textbf{Concours Blanc - Epreuve d'informatique}}
\end{center} \cfoot{\thepage/\pageref{LastPage}}}

\newcommand{\PC}[3]{\lhead{Concours {#1} -- #2}
\rhead{Lycée Leconte de Lisle} \setlength{\fboxsep}{8pt}
%Titre de la fiche
\begin{center}
   {\Large \textbf{Proposition de corrigé}}
\end{center} \cfoot{\thepage/\pageref{LastPage}}}

\newcounter{numdspart}
\setcounter{numdspart}{1}
\newcommand{\DSPart}{\bigskip
   \hrulefill\raisebox{-3pt}{\framebox[4cm]{\textbf{\textbf{Partie \thenumdspart}}}}\hrulefill
   \addtocounter{numdspart}{1}
   \bigskip}

\newcommand{\Sauvegarde}[1]{
   \begin{tcolorbox}[title=\textcolor{black}{\danger\; Attention},colbacktitle=lightgray]
      {Tous vos programmes doivent être enregistrés dans votre dossier personnel, dans {\tt Evaluations}{\tt \textbackslash}{\tt #1}
      }
   \end{tcolorbox}
}

\newcommand{\alertbox}[3]{
   \begin{tcolorbox}[title=\textcolor{black}{#1\; #2},colbacktitle=lightgray]
      {#3}
   \end{tcolorbox}
}

%Devoir programmation en NSI (pas à rendre sur papier)
\newcommand{\PNSI}[2]{\lhead{\textbf{Devoir de {\sc nsi} : \textsf{ #1}}
}
\rhead{\textbf{#2}} \setlength{\fboxsep}{8pt}
 \cfoot{}
 \begin{center}{\Large \textbf{Evaluation de {\sc nsi}}}\end{center}}


%Devoir de NSI
\newcommand{\DNSI}[2]{\lhead{\textbf{Devoir de {\sc nsi} : \textsf{ #1}}}
\rhead{\textbf{#2}} \setlength{\fboxsep}{8pt}
\begin{center}
%Titre de la fiche
\fbox{\parbox[b][1cm][t]{0.3\textwidth}{Nom : \hfill \po{3} \par \vfill Prénom : \hfill \po{3}} } \hfill 
\fbox{\parbox[b][1cm][t]{0.6\textwidth}{Note : \po{1} / 10} }
\end{center} \cfoot{}}

\newcommand{\DevoirNSI}[2]{\lhead{\textbf{Devoir de {\sc nsi} : {\sc #1}}}
\rhead{\textbf{#2}} \setlength{\fboxsep}{8pt}
\cfoot{}}

%La définition de la commande QCM pour auto-multiple-choice
%En premier argument le sujet du qcm, deuxième argument : la classe, 3e : la durée prévue et #4 : présence ou non de questions avec plusieurs bonnes réponses
\newcommand{\QCM}[4]{
{\large \textbf{\ding{52} QCM : #1}} -- Durée : \textbf{#3 min} \hfill {\large Note : \dots/10} 
\hrule \vspace{0.1cm}\namefield{}
Nom :  \textbf{\textbf{\nom{}}} \qquad \qquad Prénom :  \textbf{\prenom{}}  \hfill Classe: \textbf{#2}
\vspace{0.2cm}
\hrule  
\begin{itemize}[itemsep=0pt]
\item[-] \textit{Une bonne réponse vaut un point, une absence de réponse n'enlève pas de point. }
\item[\danger] \textit{Une mauvaise réponse enlève un point.}
\ifthenelse{#4=1}{\item[-] \textit{Les questions marquées du symbole \multiSymbole{} peuvent avoir plusieurs bonnes réponses possibles.}}{}
\end{itemize}
}
\newcommand{\DevoirC}[2]{
\renewcommand{\footrulewidth}{0.5pt}
\lhead{\textbf{Devoir de mathématiques : {\sc #1}}}
\rhead{\textbf{#2}} \setlength{\fboxsep}{8pt}
\fbox{\parbox[b][0.4cm][t]{0.955\textwidth}{Nom : \po{5} \hfill Prénom : \po{5} \hfill Classe: \textbf{1}\textsuperscript{$\dots$}} } 
\rfoot{\thepage} \cfoot{} \lfoot{Lycée Nord}}
\newcommand{\DevoirInfo}[2]{\lhead{\textbf{Evaluation : {\sc #1}}}
\rhead{\textbf{#2}} \setlength{\fboxsep}{8pt}
 \cfoot{}}
\newcommand{\DM}[2]{\lhead{\textbf{Devoir maison à rendre le #1}} \rhead{\textbf{#2}}}

%Macros permettant l'affichage des touches de la calculatrice
%Touches classiques : #1 = 0 fond blanc pour les nombres et #1= 1gris pour les opérations et entrer, second paramètre=contenu
%Si #2=1 touche arrondi avec fond gris
\newcommand{\TCalc}[2]{
\setlength{\fboxsep}{0.1pt}
\ifthenelse{#1=0}
{\psframebox[fillstyle=solid, fillcolor=white]{\parbox[c][0.25cm][c]{0.6cm}{\centering #2}}}
{\ifthenelse{#1=1}
{\psframebox[fillstyle=solid, fillcolor=lightgray]{\parbox[c][0.25cm][c]{0.6cm}{\centering #2}}}
{\psframebox[framearc=.5,fillstyle=solid, fillcolor=white]{\parbox[c][0.25cm][c]{0.6cm}{\centering #2}}}
}}
\newcommand{\Talpha}{\psdblframebox[fillstyle=solid, fillcolor=white]{\hspace{-0.05cm}\parbox[c][0.25cm][c]{0.65cm}{\centering \scriptsize{alpha}}} \;}
\newcommand{\Tsec}{\psdblframebox[fillstyle=solid, fillcolor=white]{\parbox[c][0.25cm][c]{0.6cm}{\centering \scriptsize 2nde}} \;}
\newcommand{\Tfx}{\psdblframebox[fillstyle=solid, fillcolor=white]{\parbox[c][0.25cm][c]{0.6cm}{\centering \scriptsize $f(x)$}} \;}
\newcommand{\Tvar}{\psframebox[framearc=.5,fillstyle=solid, fillcolor=white]{\hspace{-0.22cm} \parbox[c][0.25cm][c]{0.82cm}{$\scriptscriptstyle{X,T,\theta,n}$}}}
\newcommand{\Tgraphe}{\psdblframebox[fillstyle=solid, fillcolor=white]{\hspace{-0.08cm}\parbox[c][0.25cm][c]{0.68cm}{\centering \tiny{graphe}}} \;}
\newcommand{\Tfen}{\psdblframebox[fillstyle=solid, fillcolor=white]{\hspace{-0.08cm}\parbox[c][0.25cm][c]{0.68cm}{\centering \tiny{fenêtre}}} \;}
\newcommand{\Ttrace}{\psdblframebox[fillstyle=solid, fillcolor=white]{\parbox[c][0.25cm][c]{0.6cm}{\centering \scriptsize{trace}}} \;}

% Macroi pour l'affichage  d'un entier n dans  une base b
\newcommand{\base}[2]{ \overline{#1}^{#2}}
% Intervalle d'entiers
\newcommand{\intN}[2]{\llbracket #1; #2 \rrbracket}
% Cadre avec lignes réponses
\def\gaddtotok#1{\global\tabtok\expandafter{\the\tabtok#1}}
\newtoks\tabtok
\newcommand*\reponse[2]{%
   \ifthenelse{\boolean{corrige}}{}{
	\global\tabtok{\\ \renewcommand{\arraystretch}{1.4}\begin{tabularx}{\linewidth}{|X|p{1cm}|}\hline \dotfill & \cellcolor{gray!30}{\small \dots/#2} \\ \cline{2-2}}%
	\multido{}{#1}{\gaddtotok{ \multicolumn{2}{|>{\hsize=\dimexpr1\hsize+2\tabcolsep+\arrayrulewidth+1cm\relax}X|}{\dotfill}\\ }}%
	\gaddtotok{\hline \end{tabularx}}%
	\the\tabtok
   }}

\newcommand{\PE}[1]{\left \lfloor #1 \right \lfloor}}
\ModeExercice
\DS{MP2I}{2}{Novembre 2023}
\newcommand{\SPATH}{\FPATH Evaluations/DS/DS2/}

\alertbox{\danger}{Consignes}{
	\begin{itemize}
		\item[\textbullet] Les programmes demandés doivent être écrits en C et on suppose que les librairies standards usuelles ({\tt <stdio.h>}, {\tt <stdlib.h>}, {\tt <stdbool.h>}) sont déjà importées.
		\item[\textbullet] On pourra toujours librement utiliser une fonction demandée à une question précédente même si cette question n'a pas été traitée.
		\item[\textbullet] Veillez à présenter vos idées et vos réponses partielles même si vous ne trouvez pas la solution complète à une question.
		\item[\textbullet] La clarté et la lisibilité de la rédaction et des programmes sont des éléments de notation.
	\end{itemize}
}


\begin{Exercise}[title = {exponentiation rapide}]\\
On rappelle que pour $a \in \R, n \in \N^{*}$,\\
$\left\{ \begin{array}{lll}
	a^n & = & \left(a^\frac{n}{2}\right)^2, \ \mathrm{si\ } n  \mathrm{\ est\ paire} \\
	a^n & = & \left(a^\frac{n-1}{2}\right)^2\times a, \ \mathrm{sinon\ }\end{array} \right. $\\
et d'autre part on convient que pour tout $a \in \R, a^0=1$.
\Question{Vérifier que pour calculer $a^7$, l'utilisation de ces relations ne demande que 4 multiplications, alors qu'il en faut 7 pour l'algorithme qui consiste à multiplier 1 par $a$ à 7 reprises.}
\tcor{
	Pour calculer $a^7$, on effectue les calculs suivants :\\
	$a^7 = (a^3)^2 \times a$ \\
	$a^7 = \left(a^2\times a\right)^2 \times a$ \\
	Et donc on effectue bien 7, quatre multiplications.
}
\Question{Ecrire une fonction récursive {\tt exp\_rapide} en C qui prend en argument un flottant {\tt a} et un entier {\tt n} et renvoie $a^n$ en utilisant les relations de récurrence rappelées en début d'exercice.}
\inputpartC{\SPATH puissance.c}{}{\small}{21}{29}
\Question{\textbf{Bonus : } écrire cette fonction en OCaml.}
\inputpartOCaml{\SPATH exp_rapide.ml}{}{\small}{1}{4}
\Question{Pour $n \in \N$, donner un ordre de grandeur (en le justifiant) du nombre de multiplication nécessaires pour calculer $a^n$ à l'aide de la fonction {\tt exp\_rapide}.}
\tcor{
	On note $C(n)$ le nombre de multiplications nécessaires au calcul de $a^n$, alors $C(n) \leq C(\lfloor n/2 \rfloor) + 2$ puisque dans le cas pair une seule multiplication supplémentaire est nécessaire et dans le cas impair 2. On en déduit que $C(n) \leq 2 \log_2(n)$.
}
\Question{Montrer que le nombre de multiplications nécessaire pour calculer $a^{15}$ avec cet algorithme est 6.}
\tcor{
	Le calcul de $a^{15}$ s'effectue avec 
	$a^{15} = \left( \left(a^2\times a\right)^2 \times a \right)^2 \times a $. \\ 
	Ce qui fait bien 6 multiplications.
}
\Question{Montrer (en donnant les étapes) qu'on peut calculer $a^{15}$ en faisant seulement $5$ multiplications. Que peut-on en conclure sur l'algorithme d'exponentiation rapide ?}
\tcor{
	On peut calculer $a^{15}$ avec seulement 5 multiplications :\\
	$ b = a^2 \times a$ (2 multiplications) \\
	$ a^{15} = \left( b^2 \right)^2 \times a$ (3 multiplications)
	On en déduit que l'algorithme d'exponentiation ne donne pas toujours le nombre de multiplications optimal.
}
\end{Exercise}

\begin{Exercise}[title = {palindrome}]\\
Un \textbf{palindrome} est un mot qui se lit indifféremment de droite à gauche ou de gauche à droite, par exemple \textit{kayak} est un palindrome, de même que \textit{ressasser}. Par contre, \textit{bobo} ou \textit{élevé} ne sont pas des palindromes.

\Question{Ecrire, \textit{sans utiliser} {\tt strlen}, une fonction {\tt longueur} qui prend en argument une chaine de caractère et renvoie sa longueur. \\
\aide \; On rappelle qu'une chaine de caractères est un tableau se terminant par le caractère spécial \mbox{\tt '\textbackslash{}0'} 
}
\inputpartC{\SPATH palindrome.c}{}{\small}{4}{12}
\Question{Ecrire une fonction iterative {\tt est\_palindrome} qui prend en argument une chaine de caractères et renvoie {\tt true} ou {\tt false} suivant que cette chaine soit ou non un palindrome.}
\inputpartC{\SPATH palindrome.c}{}{\small}{14}{22}
\Question{Ecrire une version récursive de cette fonction.\\
\aide \; On pourra éventuellement écrire une fonction auxiliare {\tt palindrome\_rec} qui prend en argument une chaine de caractères {\tt s} ainsi que deux entiers {\tt debut} et {\tt fin} et qui teste si la chaine de caractères démarrant à l'indice {\tt debut} et se terminant à l'indice {\tt fin} (inclus) est un palindrome.
}
\inputpartC{\SPATH palindrome.c}{}{\small}{24}{32}
\end{Exercise}


\begin{Exercise}[title = {convergence d'une suite}]

On considère la suite: \\
	$\left\{ \begin{array}{ll} u_1= 2 \\ u_2 = 3 \\ u_{n+2} = 15 - \dfrac{54}{u_{n+1}} + \dfrac{40}{u_nu_{n+1}} \text{ pour tout } n \geq 0 \end{array}\right.$

\Question{Ecrire une fonction {\tt calcule} qui prend en argument un entier {\tt n}, ne renvoie rien et affiche dans le terminal les valeurs de $u_k$ pour $k=0, \dots, n$ calculées à l'aide de la formule de récurrence ci-dessus. A titre d'exemple on donne ci dessous le résultat souhaité pour l'appel {\tt calcule(10)}} :
\begin{minted}{c}
u0 = 2.000000
u1 = 3.000000
u2 = 3.666667
u3 = 3.909091
u4 = 3.976744
u5 = 3.994152
u6 = 3.998536
u7 = 3.999634
u8 = 3.999908
u9 = 3.999977
u10 = 3.999994
\end{minted}
\inputpartC{\SPATH suite.c}{}{\small}{4}{18}

\Question{Ecrire un programme principal {\tt main} qui prend en  argument un entier sur la ligne de commande, et appelle la fonction {\tt calcule} avec cet entier. A titre d'exemple, si votre programme est compilé sous le nom {\tt suite.exe} alors  {\tt ./suite.exe 10} doit produire l'affiche précédent.\\
\aide \; On rappelle que la fonction {\tt atoi} permet de convertir une chaine de caractères en entier.
}
\inputpartC{\SPATH suite.c}{}{\small}{20}{23}
\Question{Montrer que le terme général de $(u_n)_{n \in \mathbb{N}}$ est $u_n = \dfrac{4^n+2}{4^{n-1}+2}$.\\
\aide \; Penser à faire un raisonnement par récurrence.}
\tcor{
	Pour tout $n \in \N$, on note $\mathcal{P}(n)$ la propriété $u_n = \dfrac{4^n+2}{4^{n-1}+2}$. Montrons par récurrence que $P(n)$ est vraie pour tout $n \in \N$.\\
	$P(1)$ est vraie, en effet : $\dfrac{4^1+2}{4^{0}+2} =2$ et  $u_1 = 2$\\
	$P(2)$ est vraie, en effet : $\dfrac{4^2+2}{4^{1}+2} =3$ et  $u_2 = 3$\\
	On suppose maintenant qu'il existe un rang $n \in N$ tel que $P(n)$ et $P(n+1)$ soient vraies et on montre qu'alors $P(n+2)$ est vraie. Par définition, \\
	$u_{n+2} = 15 - \dfrac{54}{u_{n+1}} + \dfrac{40}{u_n\,u_{n+1}}$, par hypothèse de récurrence $u_n = \dfrac{4^n+2}{4^{n-1}+2}$ (et de même pour $u_{n+1}$) donc \\
	$u_{n+2} = 15 - 54 \dfrac{4^n+2}{4^{n+1}+2} + 40 \dfrac{4^{n-1}+2}{4^{n+1}+2}$\\
	$u_{n+2} = \dfrac{15(4^{n+1}+2) - 54 (4^{n}+2) + 40 (4^{n-1}+2)}{4^{n+1}+2}$ \\
	$u_{n+2} = \dfrac{60\times 4^n + 30 - 54\times 4^n - 108 + 10 \times 4^n + 80}{4^{n+1}+2}$ \\
	$u_{n+2} = \dfrac{16\times 4^n + 2}{4^{n+1}+2}$ \\
	$u_{n+2} = \dfrac{4^{n+2} + 2}{4^{n+1}+2}$ \\
	Donc $P(n+2)$ est vérifiée.
}
\Question{En déduire la limite de $(u_n)_{n \in \mathbb{N}}$.}
\tcor{ $\displaystyle{\lim_{n \rightarrow +\infty}\dfrac{4^{n} + 2}{4^{n-1}+2} = 4}$}
\Question{On donne ci-dessous les 5 dernière lignes affichées par {\tt ./exercice1.exe 60}. Calculer  la valeur exacte de $u_{60}$ grâce à la formule établie à la question 3 et expliquer rapidement la différence avec la valeur calculée par le programme.
\begin{minted}{c}
u56 = 9.999964
u57 = 9.999986
u58 = 9.999994
u59 = 9.999998
u60 = 9.999999
\end{minted}
}
\tcor{Les erreurs d'arrondi dues à l'arithmétique à virgule flottante ne permettent pas de calculer correctement les termes de cette suite.}
\end{Exercise}

\begin{Exercise}[title = {Structures et pointeurs}]\\
Dans un lycée les salles de cours sont nommées par une lettre suivie d'un numéro (par exemple {\tt R4} ou {\tt S6}). De plus ces salles ont une capacité maximale d'élèves donnée sous la forme d'un nombre entier et sont ou non  équipées d'ordinateurs.

\Question{Donner en C, la définition d'un type structuré {\tt salle} contenant les champs suivants : 
\begin{itemize}
\item {\tt batiment} de type {\tt char}
\item {\tt numero} de type {\tt int}
\item {\tt capacite} de type {\tt int}
\item {\tt ordinateur} de type {\tt bool}
\end{itemize}
\inputpartC{\SPATH salles.c}{}{\small}{5}{11}
Dans toute la suite de l'exercice on suppose ce type défini et nommé {\tt salle}.
}
\Question{Ecrire en C, une fonction {\tt cree\_salle} qui prend en argument un caractère {\tt b}, deux entiers {\tt n} et {\tt c} et un booléen {\tt o} et renvoie une variable de type {\tt salle} telle que {\tt salle.batiment = b, salle.numero = n, salle.capacite = c} et {\tt salle.ordinateur = o}}.
\inputpartC{\SPATH salles.c}{}{\small}{13}{17}

\Question{Ecrire en C, une fonction {\tt modifie\_capacite} qui prend en argument une {\tt salle s} et un entier {\tt nc} qui affecte à cette salle la nouvelle capacité {\tt nc}.}
\inputpartC{\SPATH salles.c}{}{\small}{19}{22}

\Question{On suppose à présent qu'on dispose d'un fichier {\tt salles.txt} contenant 100 lignes, sur chaque ligne on trouve le nom d'une salle suivie de sa capacité puis d'un {\tt 1} si la salle est équipée d'ordinateur et d'un {\tt 0} sinon. Par exemple, voici les deux premières lignes du fichier {\tt salles.txt} : 
\begin{minted}{c}
S6 30 0
R2 16 1
\end{minted}
Ecrire une fonction {\tt lire\_salle} qui ne prend pas d'arguments, lit le fichier {\tt salles.txt} et renvoie un tableau {\tt salles}  contenant les 100 salles du fichier. Par exemple le premier élément du tableau renvoyé {\tt salles[0]} doit être tel que {\tt salles[0].batiment = 'S', salles[0].numero = 6, salles[0].capacite = 30} et {\tt salles[0].ordinateur = false}.\\
\aide \; On rappelle que {\tt fopen} permet d'ouvrir un fichier en donnant son nom et que {\tt fscanf} permet de lire dans un fichier en donnant une chaine de formatage et des pointeurs vers les valeurs lues.}
\inputpartC{\SPATH salles.c}{}{\small}{44}{61}
\Question{Ecrire une fonction {\tt capacite\_batiment} qui prend en argument un tableau de {\tt salles} (ainsi que sa taille) et un caractère {\tt bat} et renvoie la capacité totale des salles dont le batiment est {\tt bat}.}
\inputpartC{\SPATH salles.c}{}{\small}{31}{42}
\end{Exercise}

\begin{Exercise}[title = {Implémentation des entiers par représentation binaire}]

On rappelle qu'en C, le type {\tt uint64\_t} (disponible dans {\tt stdint.h} qu'on suppose déjà importée dans la suite de l'exercice) représente des entiers \textit{non signés} sur 64 bits. D'autre part on rappelle que le spécificateur de format permettant d'afficher un entier de type {\tt uint64\_t} est {\tt \%lu}.

\Question{A propos du format {\tt uint64\_t}.}
\subQuestion{Donner l'intervalle d'entiers représentable avec ce format.}
\tcor{Les entiers représentables avec ce format sont $\intN{0}{2^{64}-1}$.}
\subQuestion{En compilant puis en exécutant le programme suivant sur un ordinateur (les librairies {\tt <stdio.h>} et {\tt <stdint.h>}  sont supposées importées):
\inputpartC{\FPATH/Evaluations/DS/DS2/ex2.c}{}{}{4}{7}

on a obtenu l'affichage suivant dans le terminal : {\tt a= 18446744073709551615}. Expliquer cet affichage, s'agit-il d'un comportement indéfini ?}
\tcor{Comme {\tt a} est un entier non signé initialisé à 0, l'instruction {\tt a = a -1} est un dépassement de capacité. Ce n'est pas un comportement indéfini, sur les entiersnon signés les calculs sont faits modulo le plus grand entier représentable plus un et donc ici on obtient donc $2^{64}-1$.}
\subQuestion{Convertir  $\base{11101011}{2}$ en base 10.}
\tcor {$\base{11101011}{2} = \base{235}{10}$.}
\subQuestion{Convertir $\base{307}{10}$ en base 2.}
\tcor {$\base{307}{10} = \base{100110011}{2}$.}
\Question{Représentation des ensembles. \\On utilise à présent les entiers au format {\tt uint64\_t} afin de représenter des ensembles. A chaque entier écrit en base 2 on associe l'ensemble dont les éléments sont les positions des bits égaux à 1. Par exemple :
\begin{itemize}
	\item L'entier $\base{11001}{2}$(=$\base{25}{10}$) a des bits égaux à 1 aux positions 0,3 et 4 et donc représente l'ensemble $\{0,3,4\}$.
	\item L'entier $\base{10000000}{2}$(=$\base{128}{10}$) a un seul bit égal à 1 en position 7 et donc représente l'ensemble $\{7\}$.
	\item L'ensemble $\{1,5\}$ est représenté par l'entier ayant des bits égaux à 1 en position 1 et 5, c'est à dire $\base{100010}{2}=\base{34}{10}$.
\end{itemize}
}
\subQuestion{Quels sont les ensembles représentables avec ce codage avec des entiers au format {\tt uint64\_t} ?}
\tcor{Les ensembles répresentables sont les parties de $\intN{0}{63}$}
\subQuestion{Donner l'écriture en base 10 de l'entier représentant l'ensemble $\{ 2,7\}$}
\tcor{L'entier représentant $\{ 2,7\}$ est $\base{10000100}{2} = \base{132}{10}$.}
\subQuestion{Quel est l'ensemble codé par l'entier $\base{76}{10}$ ?}
\tcor{ $\base{76}{10} = \base{1001100}{2}$ et donc code l'ensemble $\{ 2, 3, 6 \}$. }
\subQuestion{Donner la caractérisation des ensembles représentés par une puissance exacte de 2.}
\tcor{ Les ensembles représentés par une puissance exacte de 2 sont les singletons.}
\subQuestion{Ecrire une fonction {\tt appartient} qui prend en argument un entier  {\tt s} codant un ensemble et un entier {\tt e} et renvoie {\tt true} si {\tt e} appartient à l'ensemble codé par {\tt s} et {\tt false sinon}. Par exemple puisque l'ensemble $\{1,5\}$ est codé par $34$, {\tt appartient(34,1)} doit renvoyer {\tt true} tandis que {\tt appartient(34,2)} doit renvoyer {\tt false}.}
\tcor{On utilise l'algorithme des divisions successives de façon à trouver le bit de rang {\tt e}, on renvoie {\tt true} si ce bit est à 1 et {\tt false sinon}}
\inputpartC{\SPATH code_entiers.c}{}{\small}{10}{14}
\tcor { Les opérations bit à bit ne sont pas au programme, mais ils permettent ici d'écrire une solution bien plus concise. L'opérateur \mintinline{c}{>>e} décale les bits de s de e rang vers la droite et \mintinline{c}{& 1} permet de récupérer le dernier bit.}
\inputpartC{\SPATH code_entiers.c}{}{\small}{5}{8}
\subQuestion{Ecrire une fonction {\tt encode} en C, qui prend en argument un tableau d'entiers (et  sa taille) et renvoie l'entier qui représente l'ensemble dont les éléments sont ceux du tableau. On supposera que les éléments du tableau sont distincts et tous inférieurs à 63. Par exemple, si {\tt tab} est un tableau de taille 2 tel que {\tt tab[0]=1} et {\tt tab[1]=5} alors, {\tt encode(tab,2)} doit renvoyer $\base{100010}{2}$ c'est à dire $\base{34}{10}$.}
\inputpartC{\SPATH code_entiers.c}{}{\small}{17}{25}
\subQuestion{Ecrire une fonction {\tt decode} en C, qui prend en argument un entier au format {\tt uint64\_t} et renvoie l'ensemble qu'il représente sous la forme d'un tableau. Par exemple {\tt decode(34)} doit renvoyer un tableau {\tt tab} tel que {\tt tab[0]=1} et {\tt tab[1]=5}.}
\inputpartC{\SPATH code_entiers.c}{}{\small}{28}{41}

\end{Exercise}
\end{document}
\PassOptionsToPackage{dvipsnames,table}{xcolor}
\documentclass[11pt,a4paper]{article}

\usepackage{DS}

\begin{document}
\newcommand{\ModeExercice}{
% Traduction des noms pour le package exercise
\renewcommand{\ExerciseName}{Exercice}
\renewcommand{\thesubQuestion}{\theQuestion.\alph{subQuestion}}
\renewcommand{\AnswerName}{Réponses à l'exercise }
}
\newcommand{\Fiche}[2]{\lhead{\textbf{{\sc #1}}}
\rhead{Niveau: \textbf{#2}}
\cfoot{}}
\definecolor{cfond}{gray}{0.4}
\renewcommand{\thealgocf}{}

\newcommand{\ModeActivite}{
% Traduction des noms pour le package exercise
\renewcommand{\ExerciseName}{Activité}
}
% Réglages de la mise en forme des exercices 
\renewcommand{\ExerciseHeaderTitle}{\ExerciseTitle}
\renewcommand{\ExerciseHeaderOrigin}{\ExerciseOrigin}
% Un : sépare le numéro de l'exerice de son titre ... si le titre existe. On utilise Origine pour placer les pictogrammes en fin de ligne
\renewcommand{\ExerciseHeader}{\ding{113} \textbf{\sffamily{\ExerciseName \ \ExerciseHeaderNB}} \ifthenelse{\equal{\ExerciseTitle}{\empty}}{}{:} \textit{\ExerciseHeaderTitle} \hfill \ExerciseHeaderOrigin}
\renewcommand{\ExePartHeader}{\quad {\footnotesize \ding{110}} \textbf{Partie \textbf{\ExePartHeaderNB}} : \ExePartName}


% Mode Concours
\newcommand{\ModeConcours}{
   \newcounter{qconcours}
   \setcounter{qconcours}{1}
   \renewcommand{\ExerciseName}{Exercice}
   \renewcommand{\ExerciseHeaderTitle}{\ExerciseTitle}
\renewcommand{\ExerciseHeaderOrigin}{\ExerciseOrigin}
\renewcommand{\ExerciseHeader}{\ding{113} \textbf{\sffamily{\ExerciseName \ \ExerciseHeaderNB}} \ifthenelse{\equal{\ExerciseTitle}{\empty}}{}{:} \textit{\ExerciseHeaderTitle} \hfill \ExerciseHeaderOrigin}
\renewcommand{\QuestionNB}{\textbf{Q\arabic{qconcours}--}\ \addtocounter{qconcours}{1}}}

\newcommand{\noeud}[1]{\Tr{\fbox{\tt #1}}}
\newcommand{\FPATH}{/home/fenarius/Travail/Cours/cpge-info/docs/mp2i/}
\newcommand{\spath}[2]{\FPATH Evaluations/#1/#1#2}
\definecolor{codebg}{gray}{0.90}
\newcommand{\inputpartOCaml}[5]{\begin{mdframed}[backgroundcolor=codebg] \inputminted[breaklines=true,fontsize=#3,linenos=true,highlightcolor=fluo,tabsize=2,highlightlines={#2},firstline=#4,lastline=#5,firstnumber=1]{OCaml}{#1} \end{mdframed}}
\newcommand{\inputpartPython}[5]{\begin{mdframed}[backgroundcolor=codebg] \inputminted[breaklines=true,fontsize=#3,linenos=true,highlightcolor=fluo,tabsize=2,highlightlines={#2},firstline=#4,lastline=#5,firstnumber=1]{python}{#1} \end{mdframed}}
\newcommand{\inputpartC}[5]{\begin{mdframed}[backgroundcolor=codebg] \inputminted[breaklines=true,fontsize=#3,linenos=true,highlightcolor=fluo,tabsize=2,highlightlines={#2},firstline=#4,lastline=#5,firstnumber=1]{c}{#1} \end{mdframed}}
\newcommand{\inputC}[2]{\begin{mdframed}[backgroundcolor=codebg] \inputminted[breaklines=true,fontsize=#2,linenos=true,highlightcolor=fluo,tabsize=2]{c}{#1} \end{mdframed}}
\newminted[langageC]{c}{linenos=true,escapeinside=``,highlightcolor=fluo,tabsize=2}
\newminted[python]{python}{linenos=true,escapeinside=``,highlightcolor=fluo,tabsize=4}
\BeforeBeginEnvironment{minted}{\begin{mdframed}[backgroundcolor=codebg,skipabove=0cm]}
   \AfterEndEnvironment{minted}{\end{mdframed}}

% Font light medium et bold pour tt :
\newcommand{\ttl}[1]{\ttfamily \fontseries{l}\selectfont #1}
\newcommand{\ttm}[1]{\ttfamily \fontseries{m}\selectfont #1}
\newcommand{\ttb}[1]{\ttfamily \fontseries{b}\selectfont #1}

%QCM de NSI \QNSI{Question}{R1}{R2}{R3}{R4}
\newcommand{\QNSI}[5]{
#1
\begin{enumerate}[label=\alph{enumi})]
\item #2
\item #3
\item #4
\item #5
\end{enumerate}
}



\definecolor{grispale}{gray}{0.95}
\newcommand{\htmlmode}{\lstset{language=html,numbers=left, tabsize=2, frame=single, breaklines=true, keywordstyle=\ttfamily, basicstyle=\small,
   numberstyle=\tiny\ttfamily, framexleftmargin=0mm, backgroundcolor=\color{grispale}, xleftmargin=12mm,showstringspaces=false}}
\newcommand{\pythonmode}{\lstset{language=python,numbers=left, tabsize=4, frame=single, breaklines=true, keywordstyle=\ttfamily, basicstyle=\small,
   numberstyle=\tiny\ttfamily, framexleftmargin=0mm, backgroundcolor=\color{grispale}, xleftmargin=12mm, showstringspaces=false}}
\newcommand{\bashmode}{\lstset{language=bash,numbers=left, tabsize=2, frame=single, breaklines=true, basicstyle=\ttfamily,
   numberstyle=\tiny\ttfamily, framexleftmargin=0mm, backgroundcolor=\color{grispale}, xleftmargin=12mm, showstringspaces=false}}
\newcommand{\exomode}{\lstset{language=python,numbers=left, tabsize=2, frame=single, breaklines=true, basicstyle=\ttfamily,
   numberstyle=\tiny\ttfamily, framexleftmargin=13mm, xleftmargin=12mm, basicstyle=\small, showstringspaces=false}}
   
   
  \lstset{%
        inputencoding=utf8,
        extendedchars=true,
        literate=%
        {é}{{\'{e}}}1
        {è}{{\`{e}}}1
        {ê}{{\^{e}}}1
        {ë}{{\¨{e}}}1
        {É}{{\'{E}}}1
        {Ê}{{\^{E}}}1
        {û}{{\^{u}}}1
        {ù}{{\`{u}}}1
        {ú}{{\'{u}}}1
        {â}{{\^{a}}}1
        {à}{{\`{a}}}1
        {á}{{\'{a}}}1
        {ã}{{\~{a}}}1
        {Á}{{\'{A}}}1
        {Â}{{\^{A}}}1
        {Ã}{{\~{A}}}1
        {ç}{{\c{c}}}1
        {Ç}{{\c{C}}}1
        {õ}{{\~{o}}}1
        {ó}{{\'{o}}}1
        {ô}{{\^{o}}}1
        {Õ}{{\~{O}}}1
        {Ó}{{\'{O}}}1
        {Ô}{{\^{O}}}1
        {î}{{\^{i}}}1
        {Î}{{\^{I}}}1
        {í}{{\'{i}}}1
        {Í}{{\~{Í}}}1
}

%tei pour placer les images
%tei{nom de l’image}{échelle de l’image}{sens}{texte a positionner}
%sens ="1" (droite) ou "2" (gauche)
\newlength{\ltxt}
\newcommand{\tei}[4]{
\setlength{\ltxt}{\linewidth}
\setbox0=\hbox{\includegraphics[scale=#2]{#1}}
\addtolength{\ltxt}{-\wd0}
\addtolength{\ltxt}{-10pt}
\ifthenelse{\equal{#3}{1}}{
\begin{minipage}{\wd0}
\includegraphics[scale=#2]{#1}
\end{minipage}
\hfill
\begin{minipage}{\ltxt}
#4
\end{minipage}
}{
\begin{minipage}{\ltxt}
#4
\end{minipage}
\hfill
\begin{minipage}{\wd0}
\includegraphics[scale=#2]{#1}
\end{minipage}
}
}

%Juxtaposition d'une image pspciture et de texte 
%#1: = code pstricks de l'image
%#2: largeur de l'image
%#3: hauteur de l'image
%#4: Texte à écrire
\newcommand{\ptp}[4]{
\setlength{\ltxt}{\linewidth}
\addtolength{\ltxt}{-#2 cm}
\addtolength{\ltxt}{-0.1 cm}
\begin{minipage}[b][#3 cm][t]{\ltxt}
#4
\end{minipage}\hfill
\begin{minipage}[b][#3 cm][c]{#2 cm}
#1
\end{minipage}\par
}



%Macros pour les graphiques
\psset{linewidth=0.5\pslinewidth,PointSymbol=x}
\setlength{\fboxrule}{0.5pt}
\newcounter{tempangle}

%Marque la longueur du segment d'extrémité  #1 et  #2 avec la valeur #3, #4 est la distance par rapport au segment (en %age de la valeur de celui ci) et #5 l'orientation du marquage : +90 ou -90
\newcommand{\afflong}[5]{
\pstRotation[RotAngle=#4,PointSymbol=none,PointName=none]{#1}{#2}[X] 
\pstHomO[PointSymbol=none,PointName=none,HomCoef=#5]{#1}{X}[Y]
\pstTranslation[PointSymbol=none,PointName=none]{#1}{#2}{Y}[Z]
 \ncline{|<->|,linewidth=0.25\pslinewidth}{Y}{Z} \ncput*[nrot=:U]{\footnotesize{#3}}
}
\newcommand{\afflongb}[3]{
\ncline{|<->|,linewidth=0}{#1}{#2} \naput*[nrot=:U]{\footnotesize{#3}}
}

%Construis le point #4 situé à #2 cm du point #1 avant un angle #3 par rapport à l'horizontale. #5 = liste de paramètre
\newcommand{\lsegment}[5]{\pstGeonode[PointSymbol=none,PointName=none](0,0){O'}(#2,0){I'} \pstTranslation[PointSymbol=none,PointName=none]{O'}{I'}{#1}[J'] \pstRotation[RotAngle=#3,PointSymbol=x,#5]{#1}{J'}[#4]}
\newcommand{\tsegment}[5]{\pstGeonode[PointSymbol=none,PointName=none](0,0){O'}(#2,0){I'} \pstTranslation[PointSymbol=none,PointName=none]{O'}{I'}{#1}[J'] \pstRotation[RotAngle=#3,PointSymbol=x,#5]{#1}{J'}[#4] \pstLineAB{#4}{#1}}

%Construis le point #4 situé à #3 cm du point #1 et faisant un angle de  90° avec la droite (#1,#2) #5 = liste de paramètre
\newcommand{\psegment}[5]{
\pstGeonode[PointSymbol=none,PointName=none](0,0){O'}(#3,0){I'}
 \pstTranslation[PointSymbol=none,PointName=none]{O'}{I'}{#1}[J']
 \pstInterLC[PointSymbol=none,PointName=none]{#1}{#2}{#1}{J'}{M1}{M2} \pstRotation[RotAngle=-90,PointSymbol=x,#5]{#1}{M1}[#4]
  }
  
%Construis le point #4 situé à #3 cm du point #1 et faisant un angle de  #5° avec la droite (#1,#2) #6 = liste de paramètre
\newcommand{\mlogo}[6]{
\pstGeonode[PointSymbol=none,PointName=none](0,0){O'}(#3,0){I'}
 \pstTranslation[PointSymbol=none,PointName=none]{O'}{I'}{#1}[J']
 \pstInterLC[PointSymbol=none,PointName=none]{#1}{#2}{#1}{J'}{M1}{M2} \pstRotation[RotAngle=#5,PointSymbol=x,#6]{#1}{M2}[#4]
  }

% Construis un triangle avec #1=liste des 3 sommets séparés par des virgules, #2=liste des 3 longueurs séparés par des virgules, #3 et #4 : paramètre d'affichage des 2e et 3 points et #5 : inclinaison par rapport à l'horizontale
%autre macro identique mais sans tracer les segments joignant les sommets
\noexpandarg
\newcommand{\Triangleccc}[5]{
\StrBefore{#1}{,}[\pointA]
\StrBetween[1,2]{#1}{,}{,}[\pointB]
\StrBehind[2]{#1}{,}[\pointC]
\StrBefore{#2}{,}[\coteA]
\StrBetween[1,2]{#2}{,}{,}[\coteB]
\StrBehind[2]{#2}{,}[\coteC]
\tsegment{\pointA}{\coteA}{#5}{\pointB}{#3} 
\lsegment{\pointA}{\coteB}{0}{Z1}{PointSymbol=none, PointName=none}
\lsegment{\pointB}{\coteC}{0}{Z2}{PointSymbol=none, PointName=none}
\pstInterCC{\pointA}{Z1}{\pointB}{Z2}{\pointC}{Z3} 
\pstLineAB{\pointA}{\pointC} \pstLineAB{\pointB}{\pointC}
\pstSymO[PointName=\pointC,#4]{C}{C}[C]
}
\noexpandarg
\newcommand{\TrianglecccP}[5]{
\StrBefore{#1}{,}[\pointA]
\StrBetween[1,2]{#1}{,}{,}[\pointB]
\StrBehind[2]{#1}{,}[\pointC]
\StrBefore{#2}{,}[\coteA]
\StrBetween[1,2]{#2}{,}{,}[\coteB]
\StrBehind[2]{#2}{,}[\coteC]
\tsegment{\pointA}{\coteA}{#5}{\pointB}{#3} 
\lsegment{\pointA}{\coteB}{0}{Z1}{PointSymbol=none, PointName=none}
\lsegment{\pointB}{\coteC}{0}{Z2}{PointSymbol=none, PointName=none}
\pstInterCC[PointNameB=none,PointSymbolB=none,#4]{\pointA}{Z1}{\pointB}{Z2}{\pointC}{Z1} 
}


% Construis un triangle avec #1=liste des 3 sommets séparés par des virgules, #2=liste formée de 2 longueurs et d'un angle séparés par des virgules, #3 et #4 : paramètre d'affichage des 2e et 3 points et #5 : inclinaison par rapport à l'horizontale
%autre macro identique mais sans tracer les segments joignant les sommets
\newcommand{\Trianglecca}[5]{
\StrBefore{#1}{,}[\pointA]
\StrBetween[1,2]{#1}{,}{,}[\pointB]
\StrBehind[2]{#1}{,}[\pointC]
\StrBefore{#2}{,}[\coteA]
\StrBetween[1,2]{#2}{,}{,}[\coteB]
\StrBehind[2]{#2}{,}[\angleA]
\tsegment{\pointA}{\coteA}{#5}{\pointB}{#3} 
\setcounter{tempangle}{#5}
\addtocounter{tempangle}{\angleA}
\tsegment{\pointA}{\coteB}{\thetempangle}{\pointC}{#4}
\pstLineAB{\pointB}{\pointC}
}
\newcommand{\TriangleccaP}[5]{
\StrBefore{#1}{,}[\pointA]
\StrBetween[1,2]{#1}{,}{,}[\pointB]
\StrBehind[2]{#1}{,}[\pointC]
\StrBefore{#2}{,}[\coteA]
\StrBetween[1,2]{#2}{,}{,}[\coteB]
\StrBehind[2]{#2}{,}[\angleA]
\lsegment{\pointA}{\coteA}{#5}{\pointB}{#3} 
\setcounter{tempangle}{#5}
\addtocounter{tempangle}{\angleA}
\lsegment{\pointA}{\coteB}{\thetempangle}{\pointC}{#4}
}

% Construis un triangle avec #1=liste des 3 sommets séparés par des virgules, #2=liste formée de 1 longueurs et de deux angle séparés par des virgules, #3 et #4 : paramètre d'affichage des 2e et 3 points et #5 : inclinaison par rapport à l'horizontale
%autre macro identique mais sans tracer les segments joignant les sommets
\newcommand{\Trianglecaa}[5]{
\StrBefore{#1}{,}[\pointA]
\StrBetween[1,2]{#1}{,}{,}[\pointB]
\StrBehind[2]{#1}{,}[\pointC]
\StrBefore{#2}{,}[\coteA]
\StrBetween[1,2]{#2}{,}{,}[\angleA]
\StrBehind[2]{#2}{,}[\angleB]
\tsegment{\pointA}{\coteA}{#5}{\pointB}{#3} 
\setcounter{tempangle}{#5}
\addtocounter{tempangle}{\angleA}
\lsegment{\pointA}{1}{\thetempangle}{Z1}{PointSymbol=none, PointName=none}
\setcounter{tempangle}{#5}
\addtocounter{tempangle}{180}
\addtocounter{tempangle}{-\angleB}
\lsegment{\pointB}{1}{\thetempangle}{Z2}{PointSymbol=none, PointName=none}
\pstInterLL[#4]{\pointA}{Z1}{\pointB}{Z2}{\pointC}
\pstLineAB{\pointA}{\pointC}
\pstLineAB{\pointB}{\pointC}
}
\newcommand{\TrianglecaaP}[5]{
\StrBefore{#1}{,}[\pointA]
\StrBetween[1,2]{#1}{,}{,}[\pointB]
\StrBehind[2]{#1}{,}[\pointC]
\StrBefore{#2}{,}[\coteA]
\StrBetween[1,2]{#2}{,}{,}[\angleA]
\StrBehind[2]{#2}{,}[\angleB]
\lsegment{\pointA}{\coteA}{#5}{\pointB}{#3} 
\setcounter{tempangle}{#5}
\addtocounter{tempangle}{\angleA}
\lsegment{\pointA}{1}{\thetempangle}{Z1}{PointSymbol=none, PointName=none}
\setcounter{tempangle}{#5}
\addtocounter{tempangle}{180}
\addtocounter{tempangle}{-\angleB}
\lsegment{\pointB}{1}{\thetempangle}{Z2}{PointSymbol=none, PointName=none}
\pstInterLL[#4]{\pointA}{Z1}{\pointB}{Z2}{\pointC}
}

%Construction d'un cercle de centre #1 et de rayon #2 (en cm)
\newcommand{\Cercle}[2]{
\lsegment{#1}{#2}{0}{Z1}{PointSymbol=none, PointName=none}
\pstCircleOA{#1}{Z1}
}

%construction d'un parallélogramme #1 = liste des sommets, #2 = liste contenant les longueurs de 2 côtés consécutifs et leurs angles;  #3, #4 et #5 : paramètre d'affichage des sommets #6 inclinaison par rapport à l'horizontale 
% meme macro sans le tracé des segements
\newcommand{\Para}[6]{
\StrBefore{#1}{,}[\pointA]
\StrBetween[1,2]{#1}{,}{,}[\pointB]
\StrBetween[2,3]{#1}{,}{,}[\pointC]
\StrBehind[3]{#1}{,}[\pointD]
\StrBefore{#2}{,}[\longueur]
\StrBetween[1,2]{#2}{,}{,}[\largeur]
\StrBehind[2]{#2}{,}[\angle]
\tsegment{\pointA}{\longueur}{#6}{\pointB}{#3} 
\setcounter{tempangle}{#6}
\addtocounter{tempangle}{\angle}
\tsegment{\pointA}{\largeur}{\thetempangle}{\pointD}{#5}
\pstMiddleAB[PointName=none,PointSymbol=none]{\pointB}{\pointD}{Z1}
\pstSymO[#4]{Z1}{\pointA}[\pointC]
\pstLineAB{\pointB}{\pointC}
\pstLineAB{\pointC}{\pointD}
}
\newcommand{\ParaP}[6]{
\StrBefore{#1}{,}[\pointA]
\StrBetween[1,2]{#1}{,}{,}[\pointB]
\StrBetween[2,3]{#1}{,}{,}[\pointC]
\StrBehind[3]{#1}{,}[\pointD]
\StrBefore{#2}{,}[\longueur]
\StrBetween[1,2]{#2}{,}{,}[\largeur]
\StrBehind[2]{#2}{,}[\angle]
\lsegment{\pointA}{\longueur}{#6}{\pointB}{#3} 
\setcounter{tempangle}{#6}
\addtocounter{tempangle}{\angle}
\lsegment{\pointA}{\largeur}{\thetempangle}{\pointD}{#5}
\pstMiddleAB[PointName=none,PointSymbol=none]{\pointB}{\pointD}{Z1}
\pstSymO[#4]{Z1}{\pointA}[\pointC]
}


%construction d'un cerf-volant #1 = liste des sommets, #2 = liste contenant les longueurs de 2 côtés consécutifs et leurs angles;  #3, #4 et #5 : paramètre d'affichage des sommets #6 inclinaison par rapport à l'horizontale 
% meme macro sans le tracé des segements
\newcommand{\CerfVolant}[6]{
\StrBefore{#1}{,}[\pointA]
\StrBetween[1,2]{#1}{,}{,}[\pointB]
\StrBetween[2,3]{#1}{,}{,}[\pointC]
\StrBehind[3]{#1}{,}[\pointD]
\StrBefore{#2}{,}[\longueur]
\StrBetween[1,2]{#2}{,}{,}[\largeur]
\StrBehind[2]{#2}{,}[\angle]
\tsegment{\pointA}{\longueur}{#6}{\pointB}{#3} 
\setcounter{tempangle}{#6}
\addtocounter{tempangle}{\angle}
\tsegment{\pointA}{\largeur}{\thetempangle}{\pointD}{#5}
\pstOrtSym[#4]{\pointB}{\pointD}{\pointA}[\pointC]
\pstLineAB{\pointB}{\pointC}
\pstLineAB{\pointC}{\pointD}
}

%construction d'un quadrilatère quelconque #1 = liste des sommets, #2 = liste contenant les longueurs des 4 côtés et l'angle entre 2 cotés consécutifs  #3, #4 et #5 : paramètre d'affichage des sommets #6 inclinaison par rapport à l'horizontale 
% meme macro sans le tracé des segements
\newcommand{\Quadri}[6]{
\StrBefore{#1}{,}[\pointA]
\StrBetween[1,2]{#1}{,}{,}[\pointB]
\StrBetween[2,3]{#1}{,}{,}[\pointC]
\StrBehind[3]{#1}{,}[\pointD]
\StrBefore{#2}{,}[\coteA]
\StrBetween[1,2]{#2}{,}{,}[\coteB]
\StrBetween[2,3]{#2}{,}{,}[\coteC]
\StrBetween[3,4]{#2}{,}{,}[\coteD]
\StrBehind[4]{#2}{,}[\angle]
\tsegment{\pointA}{\coteA}{#6}{\pointB}{#3} 
\setcounter{tempangle}{#6}
\addtocounter{tempangle}{\angle}
\tsegment{\pointA}{\coteD}{\thetempangle}{\pointD}{#5}
\lsegment{\pointB}{\coteB}{0}{Z1}{PointSymbol=none, PointName=none}
\lsegment{\pointD}{\coteC}{0}{Z2}{PointSymbol=none, PointName=none}
\pstInterCC[PointNameA=none,PointSymbolA=none,#4]{\pointB}{Z1}{\pointD}{Z2}{Z3}{\pointC} 
\pstLineAB{\pointB}{\pointC}
\pstLineAB{\pointC}{\pointD}
}


% Définition des colonnes centrées ou à droite pour tabularx
\newcolumntype{Y}{>{\centering\arraybackslash}X}
\newcolumntype{Z}{>{\flushright\arraybackslash}X}

%Les pointillés à remplir par les élèves
\newcommand{\po}[1]{\makebox[#1 cm]{\dotfill}}
\newcommand{\lpo}[1][3]{%
\multido{}{#1}{\makebox[\linewidth]{\dotfill}
}}

%Liste des pictogrammes utilisés sur la fiche d'exercice ou d'activités
\newcommand{\bombe}{\faBomb}
\newcommand{\livre}{\faBook}
\newcommand{\calculatrice}{\faCalculator}
\newcommand{\oral}{\faCommentO}
\newcommand{\surfeuille}{\faEdit}
\newcommand{\ordinateur}{\faLaptop}
\newcommand{\ordi}{\faDesktop}
\newcommand{\ciseaux}{\faScissors}
\newcommand{\danger}{\faExclamationTriangle}
\newcommand{\out}{\faSignOut}
\newcommand{\cadeau}{\faGift}
\newcommand{\flash}{\faBolt}
\newcommand{\lumiere}{\faLightbulb}
\newcommand{\compas}{\dsmathematical}
\newcommand{\calcullitteral}{\faTimesCircleO}
\newcommand{\raisonnement}{\faCogs}
\newcommand{\recherche}{\faSearch}
\newcommand{\rappel}{\faHistory}
\newcommand{\video}{\faFilm}
\newcommand{\capacite}{\faPuzzlePiece}
\newcommand{\aide}{\faLifeRing}
\newcommand{\loin}{\faExternalLink}
\newcommand{\groupe}{\faUsers}
\newcommand{\bac}{\faGraduationCap}
\newcommand{\histoire}{\faUniversity}
\newcommand{\coeur}{\faSave}
\newcommand{\os}{\faMicrochip}
\newcommand{\rd}{\faCubes}
\newcommand{\data}{\faColumns}
\newcommand{\web}{\faCode}
\newcommand{\prog}{\faFile}
\newcommand{\algo}{\faCogs}
\newcommand{\important}{\faExclamationCircle}
\newcommand{\maths}{\faTimesCircle}
% Traitement des données en tables
\newcommand{\tables}{\faColumns}
% Types construits
\newcommand{\construits}{\faCubes}
% Type et valeurs de base
\newcommand{\debase}{{\footnotesize \faCube}}
% Systèmes d'exploitation
\newcommand{\linux}{\faLinux}
\newcommand{\sd}{\faProjectDiagram}
\newcommand{\bd}{\faDatabase}

%Les ensembles de nombres
\renewcommand{\N}{\mathbb{N}}
\newcommand{\D}{\mathbb{D}}
\newcommand{\Z}{\mathbb{Z}}
\newcommand{\Q}{\mathbb{Q}}
\newcommand{\R}{\mathbb{R}}
\newcommand{\C}{\mathbb{C}}

%Ecriture des vecteurs
\newcommand{\vect}[1]{\vbox{\halign{##\cr 
  \tiny\rightarrowfill\cr\noalign{\nointerlineskip\vskip1pt} 
  $#1\mskip2mu$\cr}}}


%Compteur activités/exos et question et mise en forme titre et questions
\newcounter{numact}
\setcounter{numact}{1}
\newcounter{numseance}
\setcounter{numseance}{1}
\newcounter{numexo}
\setcounter{numexo}{0}
\newcounter{numprojet}
\setcounter{numprojet}{0}
\newcounter{numquestion}
\newcommand{\espace}[1]{\rule[-1ex]{0pt}{#1 cm}}
\newcommand{\Quest}[3]{
\addtocounter{numquestion}{1}
\begin{tabularx}{\textwidth}{X|m{1cm}|}
\cline{2-2}
\textbf{\sffamily{\alph{numquestion})}} #1 & \dots / #2 \\
\hline 
\multicolumn{2}{|l|}{\espace{#3}} \\
\hline
\end{tabularx}
}
\newcommand{\mq}[1]
{\ding{113} \addtocounter{numquestion}{1}
\textbf{Question \arabic{numquestion}} \\ #1}
\newcommand{\QuestR}[3]{
\addtocounter{numquestion}{1}
\begin{tabularx}{\textwidth}{X|m{1cm}|}
\cline{2-2}
\textbf{\sffamily{\alph{numquestion})}} #1 & \dots / #2 \\
\hline 
\multicolumn{2}{|l|}{\cor{#3}} \\
\hline
\end{tabularx}
}
\newcommand{\Pre}{{\sc nsi} 1\textsuperscript{e}}
\newcommand{\Term}{{\sc nsi} Terminale}
\newcommand{\Sec}{2\textsuperscript{e}}
\newcommand{\Exo}[2]{ \addtocounter{numexo}{1} \ding{113} \textbf{\sffamily{Exercice \thenumexo}} : \textit{#1} \hfill #2  \setcounter{numquestion}{0}}
\newcommand{\Projet}[1]{ \addtocounter{numprojet}{1} \ding{118} \textbf{\sffamily{Projet \thenumprojet}} : \textit{#1}}
\newcommand{\ExoD}[2]{ \addtocounter{numexo}{1} \ding{113} \textbf{\sffamily{Exercice \thenumexo}}  \textit{(#1 pts)} \hfill #2  \setcounter{numquestion}{0}}
\newcommand{\ExoB}[2]{ \addtocounter{numexo}{1} \ding{113} \textbf{\sffamily{Exercice \thenumexo}}  \textit{(Bonus de +#1 pts maximum)} \hfill #2  \setcounter{numquestion}{0}}
\newcommand{\Act}[2]{ \ding{113} \textbf{\sffamily{Activité \thenumact}} : \textit{#1} \hfill #2  \addtocounter{numact}{1} \setcounter{numquestion}{0}}
\newcommand{\Seance}{ \rule{1.5cm}{0.5pt}\raisebox{-3pt}{\framebox[4cm]{\textbf{\sffamily{Séance \thenumseance}}}}\hrulefill  \\
  \addtocounter{numseance}{1}}
\newcommand{\Acti}[2]{ {\footnotesize \ding{117}} \textbf{\sffamily{Activité \thenumact}} : \textit{#1} \hfill #2  \addtocounter{numact}{1} \setcounter{numquestion}{0}}
\newcommand{\titre}[1]{\begin{Large}\textbf{\ding{118}}\end{Large} \begin{large}\textbf{ #1}\end{large} \vspace{0.2cm}}
\newcommand{\QListe}[1][0]{
\ifthenelse{#1=0}
{\begin{enumerate}[partopsep=0pt,topsep=0pt,parsep=0pt,itemsep=0pt,label=\textbf{\sffamily{\arabic*.}},series=question]}
{\begin{enumerate}[resume*=question]}}
\newcommand{\SQListe}[1][0]{
\ifthenelse{#1=0}
{\begin{enumerate}[partopsep=0pt,topsep=0pt,parsep=0pt,itemsep=0pt,label=\textbf{\sffamily{\alph*)}},series=squestion]}
{\begin{enumerate}[resume*=squestion]}}
\newcommand{\SQListeL}[1][0]{
\ifthenelse{#1=0}
{\begin{enumerate*}[partopsep=0pt,topsep=0pt,parsep=0pt,itemsep=0pt,label=\textbf{\sffamily{\alph*)}},series=squestion]}
{\begin{enumerate*}[resume*=squestion]}}
\newcommand{\FinListe}{\end{enumerate}}
\newcommand{\FinListeL}{\end{enumerate*}}

%Mise en forme de la correction
\newboolean{corrige}
\setboolean{corrige}{false}
\newcommand{\scor}[1]{\par \textcolor{blue!75!black}{\small #1}}
\newcommand{\cor}[1]{\par \textcolor{blue!75!black}{#1}}
\newcommand{\br}[1]{\cor{\textbf{#1}}}
\newcommand{\tcor}[1]{
\ifthenelse{\boolean{corrige}}{\begin{tcolorbox}[width=\linewidth,colback={white},colbacktitle=white,coltitle=green!50!black,colframe=green!50!black,boxrule=0.2mm]   
\cor{#1}
\end{tcolorbox}}{}
}
\newcommand{\iscor}[1]{\ifthenelse{\boolean{corrige}}{#1}}
\newcommand{\rc}[1]{\textcolor{OliveGreen}{#1}}

%Référence aux exercices par leur numéro
\newcommand{\refexo}[1]{
\refstepcounter{numexo}
\addtocounter{numexo}{-1}
\label{#1}}

%Séparation entre deux activités
\newcommand{\separateur}{\begin{center}
\rule{1.5cm}{0.5pt}\raisebox{-3pt}{\ding{117}}\rule{1.5cm}{0.5pt}  \vspace{0.2cm}
\end{center}}

%Entête et pied de page
\newcommand{\snt}[1]{\lhead{\textbf{SNT -- La photographie numérique} \rhead{\textit{Lycée Nord}}}}
\newcommand{\Activites}[2]{\lhead{\textbf{{\sc #1}}}
\rhead{Activités -- \textbf{#2}}
\cfoot{}}
\newcommand{\Exos}[2]{\lhead{\textbf{Fiche d'exercices: {\sc #1}}}
\rhead{Niveau: \textbf{#2}}
\cfoot{}}
\newcommand{\TD}[2]{\lhead{\textbf{TD #1} : {\sc #2} }
\rhead{{\sc mp2i -- Lycée Leconte de Lisle}}
\cfoot{}}
\newcommand{\Colles}[2]{\lhead{{\sc mp2i -- }\textbf{Colles d'informatique #1}} 
\rhead{{\sc #2}}
\cfoot{}}
\newcommand{\Devoir}[2]{\lhead{\textbf{Devoir de mathématiques : {\sc #1}}}
\rhead{\textbf{#2}} \setlength{\fboxsep}{8pt}
\begin{center}
%Titre de la fiche
\fbox{\parbox[b][1cm][t]{0.3\textwidth}{Nom : \hfill \po{3} \par \vfill Prénom : \hfill \po{3}} } \hfill 
\fbox{\parbox[b][1cm][t]{0.6\textwidth}{Note : \po{1} / 20} }
\end{center} \cfoot{}}
\newcommand{\TPnote}[3]{\lhead{\textbf{TP noté d'informatique n° #1}}
\rhead{\textbf{#2}} \setlength{\fboxsep}{8pt}
\ifthenelse{\boolean{corrige}}{}
{\begin{center}
\fbox{\parbox[b][1cm][t]{0.3\textwidth}{Nom : \hfill \po{3} \par \vfill Prénom : \hfill \po{3}} } \hfill 
\fbox{\parbox[b][1cm][t]{0.6\textwidth}{Note : \po{1} / #3} }
\end{center}} \cfoot{}}
\newcommand{\IC}[2]{\lhead{\textbf{Interro de cours n° #1}}
\rhead{{\sc mp2i --} \textbf{#2}} \setlength{\fboxsep}{8pt}
\ifthenelse{\boolean{corrige}}{}
{\begin{center}
%Titre de la fiche
\fbox{\parbox[b][1cm][t]{0.3\textwidth}{Nom : \hfill \po{3} \par \vfill Prénom : \hfill \po{3}} } \hfill 
\fbox{\parbox[b][1cm][t]{0.6\textwidth}{Note : \po{1} / 10} }
\end{center}}\cfoot{}}
\newcommand{\DS}[3]{\lhead{{#1} : \textbf{DS d'informatique n° #2}}
\rhead{Lycée Leconte de Lisle -- #3} \setlength{\fboxsep}{8pt}
%Titre de la fiche
\begin{center}
   {\Large \textbf{Devoir surveillé d'informatique}}
\end{center} \cfoot{\thepage/\pageref{LastPage}}}

\newcommand{\CB}[2]{\lhead{{#1} : \textbf{Councours blanc - Informatique}}
\rhead{Lycée Leconte de Lisle -- #2} \setlength{\fboxsep}{8pt}
%Titre de la fiche
\begin{center}
   {\Large \textbf{Concours Blanc - Epreuve d'informatique}}
\end{center} \cfoot{\thepage/\pageref{LastPage}}}

\newcommand{\PC}[3]{\lhead{Concours {#1} -- #2}
\rhead{Lycée Leconte de Lisle} \setlength{\fboxsep}{8pt}
%Titre de la fiche
\begin{center}
   {\Large \textbf{Proposition de corrigé}}
\end{center} \cfoot{\thepage/\pageref{LastPage}}}

\newcounter{numdspart}
\setcounter{numdspart}{1}
\newcommand{\DSPart}{\bigskip
   \hrulefill\raisebox{-3pt}{\framebox[4cm]{\textbf{\textbf{Partie \thenumdspart}}}}\hrulefill
   \addtocounter{numdspart}{1}
   \bigskip}

\newcommand{\Sauvegarde}[1]{
   \begin{tcolorbox}[title=\textcolor{black}{\danger\; Attention},colbacktitle=lightgray]
      {Tous vos programmes doivent être enregistrés dans votre dossier personnel, dans {\tt Evaluations}{\tt \textbackslash}{\tt #1}
      }
   \end{tcolorbox}
}

\newcommand{\alertbox}[3]{
   \begin{tcolorbox}[title=\textcolor{black}{#1\; #2},colbacktitle=lightgray]
      {#3}
   \end{tcolorbox}
}

%Devoir programmation en NSI (pas à rendre sur papier)
\newcommand{\PNSI}[2]{\lhead{\textbf{Devoir de {\sc nsi} : \textsf{ #1}}
}
\rhead{\textbf{#2}} \setlength{\fboxsep}{8pt}
 \cfoot{}
 \begin{center}{\Large \textbf{Evaluation de {\sc nsi}}}\end{center}}


%Devoir de NSI
\newcommand{\DNSI}[2]{\lhead{\textbf{Devoir de {\sc nsi} : \textsf{ #1}}}
\rhead{\textbf{#2}} \setlength{\fboxsep}{8pt}
\begin{center}
%Titre de la fiche
\fbox{\parbox[b][1cm][t]{0.3\textwidth}{Nom : \hfill \po{3} \par \vfill Prénom : \hfill \po{3}} } \hfill 
\fbox{\parbox[b][1cm][t]{0.6\textwidth}{Note : \po{1} / 10} }
\end{center} \cfoot{}}

\newcommand{\DevoirNSI}[2]{\lhead{\textbf{Devoir de {\sc nsi} : {\sc #1}}}
\rhead{\textbf{#2}} \setlength{\fboxsep}{8pt}
\cfoot{}}

%La définition de la commande QCM pour auto-multiple-choice
%En premier argument le sujet du qcm, deuxième argument : la classe, 3e : la durée prévue et #4 : présence ou non de questions avec plusieurs bonnes réponses
\newcommand{\QCM}[4]{
{\large \textbf{\ding{52} QCM : #1}} -- Durée : \textbf{#3 min} \hfill {\large Note : \dots/10} 
\hrule \vspace{0.1cm}\namefield{}
Nom :  \textbf{\textbf{\nom{}}} \qquad \qquad Prénom :  \textbf{\prenom{}}  \hfill Classe: \textbf{#2}
\vspace{0.2cm}
\hrule  
\begin{itemize}[itemsep=0pt]
\item[-] \textit{Une bonne réponse vaut un point, une absence de réponse n'enlève pas de point. }
\item[\danger] \textit{Une mauvaise réponse enlève un point.}
\ifthenelse{#4=1}{\item[-] \textit{Les questions marquées du symbole \multiSymbole{} peuvent avoir plusieurs bonnes réponses possibles.}}{}
\end{itemize}
}
\newcommand{\DevoirC}[2]{
\renewcommand{\footrulewidth}{0.5pt}
\lhead{\textbf{Devoir de mathématiques : {\sc #1}}}
\rhead{\textbf{#2}} \setlength{\fboxsep}{8pt}
\fbox{\parbox[b][0.4cm][t]{0.955\textwidth}{Nom : \po{5} \hfill Prénom : \po{5} \hfill Classe: \textbf{1}\textsuperscript{$\dots$}} } 
\rfoot{\thepage} \cfoot{} \lfoot{Lycée Nord}}
\newcommand{\DevoirInfo}[2]{\lhead{\textbf{Evaluation : {\sc #1}}}
\rhead{\textbf{#2}} \setlength{\fboxsep}{8pt}
 \cfoot{}}
\newcommand{\DM}[2]{\lhead{\textbf{Devoir maison à rendre le #1}} \rhead{\textbf{#2}}}

%Macros permettant l'affichage des touches de la calculatrice
%Touches classiques : #1 = 0 fond blanc pour les nombres et #1= 1gris pour les opérations et entrer, second paramètre=contenu
%Si #2=1 touche arrondi avec fond gris
\newcommand{\TCalc}[2]{
\setlength{\fboxsep}{0.1pt}
\ifthenelse{#1=0}
{\psframebox[fillstyle=solid, fillcolor=white]{\parbox[c][0.25cm][c]{0.6cm}{\centering #2}}}
{\ifthenelse{#1=1}
{\psframebox[fillstyle=solid, fillcolor=lightgray]{\parbox[c][0.25cm][c]{0.6cm}{\centering #2}}}
{\psframebox[framearc=.5,fillstyle=solid, fillcolor=white]{\parbox[c][0.25cm][c]{0.6cm}{\centering #2}}}
}}
\newcommand{\Talpha}{\psdblframebox[fillstyle=solid, fillcolor=white]{\hspace{-0.05cm}\parbox[c][0.25cm][c]{0.65cm}{\centering \scriptsize{alpha}}} \;}
\newcommand{\Tsec}{\psdblframebox[fillstyle=solid, fillcolor=white]{\parbox[c][0.25cm][c]{0.6cm}{\centering \scriptsize 2nde}} \;}
\newcommand{\Tfx}{\psdblframebox[fillstyle=solid, fillcolor=white]{\parbox[c][0.25cm][c]{0.6cm}{\centering \scriptsize $f(x)$}} \;}
\newcommand{\Tvar}{\psframebox[framearc=.5,fillstyle=solid, fillcolor=white]{\hspace{-0.22cm} \parbox[c][0.25cm][c]{0.82cm}{$\scriptscriptstyle{X,T,\theta,n}$}}}
\newcommand{\Tgraphe}{\psdblframebox[fillstyle=solid, fillcolor=white]{\hspace{-0.08cm}\parbox[c][0.25cm][c]{0.68cm}{\centering \tiny{graphe}}} \;}
\newcommand{\Tfen}{\psdblframebox[fillstyle=solid, fillcolor=white]{\hspace{-0.08cm}\parbox[c][0.25cm][c]{0.68cm}{\centering \tiny{fenêtre}}} \;}
\newcommand{\Ttrace}{\psdblframebox[fillstyle=solid, fillcolor=white]{\parbox[c][0.25cm][c]{0.6cm}{\centering \scriptsize{trace}}} \;}

% Macroi pour l'affichage  d'un entier n dans  une base b
\newcommand{\base}[2]{ \overline{#1}^{#2}}
% Intervalle d'entiers
\newcommand{\intN}[2]{\llbracket #1; #2 \rrbracket}
% Cadre avec lignes réponses
\def\gaddtotok#1{\global\tabtok\expandafter{\the\tabtok#1}}
\newtoks\tabtok
\newcommand*\reponse[2]{%
   \ifthenelse{\boolean{corrige}}{}{
	\global\tabtok{\\ \renewcommand{\arraystretch}{1.4}\begin{tabularx}{\linewidth}{|X|p{1cm}|}\hline \dotfill & \cellcolor{gray!30}{\small \dots/#2} \\ \cline{2-2}}%
	\multido{}{#1}{\gaddtotok{ \multicolumn{2}{|>{\hsize=\dimexpr1\hsize+2\tabcolsep+\arrayrulewidth+1cm\relax}X|}{\dotfill}\\ }}%
	\gaddtotok{\hline \end{tabularx}}%
	\the\tabtok
   }}

\newcommand{\PE}[1]{\left \lfloor #1 \right \lfloor}}
\ModeConcours
\CB{MP2I}{Mai 2024}

\setboolean{corrige}{true}

\newcommand{\maillon}[3]{
	\begin{tabular}{|p{0.2cm}|p{0.2cm}|}
		\hline
		\rnode{#2}{#1} & \rnode{#3}{\phantom{$e_0$}} \\
		\hline
	\end{tabular}
}

\alertbox{\danger}{Consignes}{
	\begin{itemize}
		\item[\textbullet] Les programmes demandés doivent être écrits en C ou en OCaml suivant l'exercice. Dans le cas du C, on suppose que les librairies standards usuelles ({\tt <stdio.h>}, {\tt <stdlib.h>}, {\tt <stdbool.h>}) sont déjà importées.
		\item[\textbullet] On pourra toujours librement utiliser une fonction demandée à une question précédente même si cette question n'a pas été traitée.
		\item[\textbullet] Veillez à présenter vos idées et vos réponses partielles même si vous ne trouvez pas la solution complète à une question.
		\item[\textbullet] La clarté et la lisibilité de la rédaction et des programmes sont des éléments de notation.
	\end{itemize}
}

\begin{Exercise}[title = {Base de données et {\sc sql}}, origin = {\bac \; {\sc capes nsi 2021}, épreuve 1} ]\\
On s'intéresse dans cette partie à un site Internet d'échange de supports de cours entre enseignants  de {\sc mp2i/mpi}. Chaque personne désirant proposer ou récupérer du contenu doit commencer par se créer un compte sur ce site et peut ensuite accéder à du contenu ou en proposer.

Ce site repose sur une base de données contenant en particulier une table, nommée {\tt ressources}. Elle possède un enregistrement par document téléversé sur le site. Ses attributs sont :
\begin{itemize}
	\item {\tt id}, un identifiant numérique, unique pour chaque ressource ;
	\item {\tt owner}, le pseudo de la personne ayant créé la ressource ;
	\item {\tt annee}, l'année de publication de la ressource ;
	\item {\tt titre}, une chaine de caractères décrivant la ressource ;
	\item {\tt type}, chaine de caractères pouvant être cours, ds, tp ou td.
\end{itemize}

Voici un extrait de cette table :

\medskip

\begin{center}
	\begin{tabular}{|l|l|c|c|r|}
		\hline
		{\tt id}       & {\tt owner}                          & {\tt annee}    & {\tt titre}                     & {\tt type}     \\
		\hline
		4        & dknuth                         & 2020     & Machine à décalage        & cours    \\
		13       & alovelace                      & 2022     & Intelligence artificielle & td       \\
		$\cdots$ & $\cdots$                       & $\cdots$ & $\cdots$                  & $\cdots$ \\
		\hline
	\end{tabular}
\end{center}


\Question{Écrire une requête SQL permettant de connaitre tous les titres des ressources déposées par \og{}{\tt jclarke}\fg{} classées par année de publication croissante.}
\Question{Écrire une requête SQL permettant de connaitre le nombre total de ressources de type cours présentes sur le site.}
\Question{Que fait la requête suivante : ?
	      \begin{minted}{sql}
SELECT R.owner
FROM Ressources AS R
WHERE R.type = 'td'
GROUP BY R.owner
ORDER BY COUNT(*) DESC
LIMIT 3
\end{minted}
}
\end{Exercise}

\begin{Exercise}[title = {Logique et calcul des propositions}, origin = {\bac \; {\sc ccinp 2018 - mp}}]
	Vous avez été sélectionné(e) pour participer au jeu ``Cherchez les Clés du Paradis (CCP)".
	Le jeu se déroule en trois épreuves, au cours desquelles vous devez collecter des clés vertes.
	A l'issue de chacune d'entre elles, vous passez à l'épreuve suivante en cas de succès et êtes éliminé(e) en cas d'échec.\\

	\ExePart[name = {Première épreuve}]\\
	Jean-Pierre Pendule, le célèbre animateur, vous accueille pour la première épreuve.
	Il vous explique la règle du jeu.
	Devant vous, deux boîtes et sur chacune d'entre elles une inscription.
	Chacune des boîtes contient soit une clé verte, soit une clé rouge.
	Vous devez choisir l'une des boîtes~: si le résultat est une clé rouge, alors vous quittez le jeu, si c'est une clé verte vous êtes qualifié(e) pour l'épreuve suivante.

	Jean-Pierre Pendule dévoile les inscriptions sur chacune des boîtes et vous affirme qu'elles sont soit vraies toutes les deux, soit fausses toutes les deux~:
	\begin{itemize}
		\item sur la boîte $1$, il est écrit \og{}Une au moins des deux boîtes contient une clé verte\fg{};
		\item  sur la boîte $2$, il est écrit \og{}Il y a une clé rouge dans l'autre boîte\fg{}.
	\end{itemize}
	Dans toute cette partie, on note $P_i$ la proposition affirmant qu'il y a une clé verte dans la boîte $i$.
	\Question{Donner une formule de la logique des propositions représentant la phrase écrite sur la boîte 1.}
	\Question{Donner de même une formule de la logique des propositions pour l'inscription de la boîte 2.}
	\Question{Donner une formule représentant l'affirmation de l'animateur. Simplifier cette formule de sorte à n'obtenir qu'une seule occurrence de chaque $P_i$.}
	\Question{Quel choix devez-vous faire pour continuer le jeu à coup sûr~?}\\


	\ExePart[name = {Deuxième épreuve}]\\
	Bravo, vous avez obtenu la première clé verte.
	Jean-Pierre Pendule vous félicite et vous annonce que cette première épreuve n'était qu'une mise en jambe.
	Avec les mêmes règles du jeu, l'animateur vous propose alors deux nouvelles boîtes portant les inscriptions suivantes~:
	\begin{itemize}
		\item sur la boîte $1$, il est écrit \og{}Il y a une clé rouge dans cette boîte, ou bien il y a une clé verte dans la boîte 2\fg{};
		\item  sur la boîte $2$, il est écrit \og{}Il y a une clé verte dans la boîte 1\fg{}.
	\end{itemize}
	\Question{Donner une formule de la logique des propositions pour chaque affirmation.}
	\Question{Sachant qu'encore une fois les deux affirmations sont soit vraies toutes les deux, soit fausses toutes les deux, donner le contenu de chaque boîte. En déduire votre choix pour remporter la deuxième clé verte.}\\


	\ExePart[name = {Troisième épreuve}]\\
	Le suspens est à son comble, vous voici arrivé(e) à la dernière épreuve.
	A votre grande surprise, Jean-Pierre Pendule vous dévoile une troisième boîte et vous explique les règles du jeu.
	Dans une des boîtes se cache la clé qui vous permet de remporter la victoire finale.
	Dans une autre boîte se cache une clé rouge qui vous fait tout perdre.
	La dernière boîte est vide.
	Encore une fois, chacune des boîtes porte une inscription~:
	\begin{itemize}
		\item sur la boîte $1$, il est écrit \og{}La boîte 3 est vide\fg{};
		\item  sur la boîte $2$, il est écrit \og{}La clé rouge est dans la boîte 1\fg{}~;
		\item  sur la boîte $3$, il est écrit \og{}Cette boîte est vide\fg{}.
	\end{itemize}
	L'animateur affirme que l'inscription portée sur la boîte contenant la clé verte est vraie, celle portée par la boîte contenant la clé rouge est fausse.
	L'inscription affichée sur la boîte vide est aussi vraie.
	\Question{Donner une formule de la logique des propositions pour chaque inscription.}
	\Question{Donner une formule de la logique des propositions synthétisant l'information que vous a apportée l'animateur.}
	\Question{En supposant que la clé verte est dans la boîte 2, montrer que l'on aboutit à une incohérence.}
	\Question{Donner alors la composition des trois boîtes.}

\end{Exercise}

\begin{Exercise}[title = {Traversée maximale d'une matrice carrée}]\\
	Les fonctions demandées dans cet exercice doivent être écrites en OCaml. On rappelle que qu'en OCaml, on peut convertir une liste en tableau à l'aide de {\tt Array.of\_list} par exemple {\tt Array.of\_list [6; 7; 42;]} renvoie {\tt [| 6; 7; 42; |]}\smallskip \\
	On considère une matrice à $n$ lignes et $n$ colonnes notée $M$ à coefficients dans $\N$, et dont le coefficient situé à la ligne $i$ et à la colonne $j$ avec $0 \leqslant i \leqslant n-1$ et  $0 \leqslant j \leqslant n-1$ sera noté $M_{i,j}$. On s'intéresse aux chemins depuis la première valeur en haut à gauche ($M_{0,0})$ jusqu'à la dernière en bas à droite ($M_{n-1,n-1}$) qui n'utilisent que les déplacements vers la droite ($\rightarrow$) ou vers le bas ($\downarrow$). Voici deux exemples de tels chemins dans une même matrice $A$ de 4 lignes et 4 colonnes : \\
	\begin{center}
		\begin{tabularx}{0.8\textwidth}{YY}
			$\renewcommand\arraystretch{2}
				\begin{pmatrix}
					\circlenode{a}{2} & 5                 & 4                 & 6                 \\
					\circlenode{b}{8} & 5                 & 9                 & 8                 \\
					\circlenode{c}{5} & \circlenode{d}{7} & \circlenode{e}{4} & 6                 \\
					9                 & 6                 & \circlenode{f}{4} & \circlenode{g}{4} \\
				\end{pmatrix}$
			\ncline{->}{a}{b}
			\ncline{->}{b}{c}
			\ncline{->}{c}{d}
			\ncline{->}{d}{e}
			\ncline{->}{e}{f}
			\ncline{->}{f}{g}                   &
			$    \renewcommand\arraystretch{2}
				\begin{pmatrix}

					\circlenode{a}{2} & \circlenode{b}{5} & \circlenode{c}{4} & 6                 \\
					8                 & 5                 & \circlenode{d}{9} & \circlenode{e}{8} \\
					5                 & 7                 & 4                 & \circlenode{f}{6} \\
					9                 & 6                 & 4                 & \circlenode{g}{4} \\
				\end{pmatrix}$
			\ncline{->}{a}{b}
			\ncline{->}{b}{c}
			\ncline{->}{c}{d}
			\ncline{->}{d}{e}
			\ncline{->}{e}{f}
			\ncline{->}{f}{g}
			\\
			Un exemple $C_1$ de chemin dans $A$ & Un exemple $C_2$ de chemin dans $A$ \\
		\end{tabularx} \end{center}
	On appelle \textit{somme d'un chemin} dans une matrice $M$, la somme des coefficients $M_{i,j}$ rencontrés en suivant ce chemin. Dans les exemples ci-dessus, le chemin $C_1$ a une somme de $34$ et le chemin $C_2$ a une somme de $38$. Le but du problème est d'étudier divers algorithmes qui cherchent à déterminer un chemin dont la somme est maximale.\\
	\ExePart[name = {Un premier algorithme naïf}]\\
	On propose l'algorithme suivant afin de résoudre ce problème  : si les deux directions bas ou droite sont possibles on choisit celle qui mène vers le coefficient de matrice le plus élevé, en cas d'égalité, on choisit (arbitrairement) la direction bas. Dans l'exemple ci-dessus à partir de $M_{0,0}=2$ on peut aller vers à droite vers $M_{0,1} = 5$ ou en bas vers $M_{1,0}=8$ et on choisit donc d'aller en bas puisque $8>5$. Une fois arrivé en $M_{1,0}$ les deux directions possibles mènent vers des coefficients identiques (5 des deux côtés) et donc, on va vers le bas. Enfin, lorsqu'une seule direction est possible (parce qu'on a atteint la dernière colonne ou la dernière ligne), c'est cette direction qui est choisie.
	\Question{Poursuivre le déroulement de cet algorithme sur la matrice $A$ donnée en exemple ci-dessus, dessiner le chemin obtenu et donner sa somme.}
	\Question{Dans quelle famille d'algorithme peut-on classer cet algorithme ? Justifier.}
	\Question{Donner la complexité de cet algorithme.}
	\Question{Montrer en exhibant un exemple dans le cas $n=4$ que cet algorithme ne fournit par toujours un chemin de somme maximale.}
	\Question{Montrer que pour une matrice de taille $n$, un chemin de $M_{0,0}$ à $M_{n-1,n-1}$ est constitué de $2n-2$ directions.}
	\Question{On définit le type :
	\inputpartOCaml{crossmax.ml}{}{}{2}{2}
	Ecrire une fonction {\tt choix : int array array -> int -> int -> direction} qui prend en argument une matrice d'entiers {\tt M}, une ligne {\tt lig} et une colonne  {\tt col} et renvoie {\tt Bas} ou {\tt Droite} suivant le choix effectué par l'algorithme décrit ci-dessus à partir de la position située à la ligne {\tt lig} à la colonne {\tt col} de la matrice {\tt M}.
	}
	\Question{Ecrire une fonction {\tt naïf int array array -> direction array} qui prend en entrée une matrice d'entiers {\tt M} et renvoie le chemin fournit par l'algorithme naïf décrit ci-dessus.}
	\Question{Ecrire une fonction {\tt somme int array array -> direction array -> int} qui prend en entrée une matrice d'entiers {\tt M}, un chemin dans cette matrice et renvoie la somme de ce chemin.}
	\ExePart[name = {Résolution exacte par force brute}] \\
	On propose l'algorithme par force brute qui consiste à énumérer tous les chemins possibles de $M_{0,0}$ vers $M_{n-1,n-1}$ à calculer leur somme puis à renvoyer la somme maximale atteignable. Afin de décrire un chemin on le fait correspondre à un tableau de $n-1$ entiers compris entre $1$ et $2n-2$, qui indique la position des {\tt Bas} dans le chemin. Par exemple pour $n=3$, le chemin {\tt [| Bas; Droite; Bas; Droite |]} sera représenté par {\tt [| 1; 3 |]} car les {\tt Bas} se trouvent en première et troisième position du tableau. De la même façon, le chemin $C_1$ donné en exemple en début d'énoncé dans le cas $n=4$ est {\tt [| Bas; Bas; Droite; Droite; Bas; Droite |]} et sera donc représenté par {\tt [| 1; 2; 5 |]}.
    \Question{A quel chemin correspond le tableau d'entiers {\tt [|3; 5; 6|]} ? }
    \Question{Ecrire  une fonction {\tt cree\_chemin int array -> direction array} qui prend en argument un tableau d'entiers  et renvoie le chemin  correspondant. Par exemple, {\tt cree\_chemin [| 4; 1 |]} renvoie le tableau {\tt [| Bas; Droite; Droite; Bas |]}.}
	\Question{Ecrire en OCaml une fonction {\tt entiers : int -> int list}} qui prend en entrée un entier {\tt n} et renvoie la liste des entiers de {\tt n} à {\tt 1}. Par exemple {\tt entiers 4} renvoie la liste {\tt [4; 3; 2; 1]}.
	\Question{On souhaite maintenant énumérer les tableaux d'entiers représentant les chemins. Ecrire en OCaml une fonction {\tt chemins :  int -> int array list} qui prend en argument un entier {\tt n}  et renvoie la liste des tableaux d'entiers représentant les chemins possibles. Par exemple {\tt chemins 2} renvoie {\tt [[| 4; 3 |]; [| 4; 2 |]; [| 4; 1 |]; [| 3; 2 |]; [| 3; 1 |]; [| 2; 1 |]]}.}
	\Question{En déduire une fonction {\tt force\_brute : int array array -> int} qui prend en entrée une matrice d'entiers et renvoie la somme maximale atteignable en traversant cette matrice.}
    \Question{Montrer que pour une matrice  $M$ de $n$ lignes et $n$ colonnes, il existe $\binom{2n-2}{n-1}$ chemins possibles de $M_{0,0}$ vers $M_{n-1,n-1}$.}
	\Question{En déduire la complexité de l'algorithme par force brute en admettant que le calcul de la somme d'un chemin donné est en $O(n)$ où $n$ est la longueur de ce chemin. On pourra utiliser la formule de Stirling : $n! \sim \left(\frac{n}{e}\right)^n\sqrt{2 \pi n}$.}
	\ExePart[name = {Résolution par programmation dynamique}] \\
	Etant donné une matrice carré $M$ de taille $n$ on note $S$ la matrice dont le coefficient situé ligne $i$ colonne $j$ est la somme maximale des chemins  de $M_{0,0}$ vers $M_{i,j}$, on a donc $S_{0,0}=M_{0,0}$.
	\Question{Ecrire la relation de recurrence permettant de calculer les $S_{0,k}$ pour $1 \leqslant k \leqslant n-1$.}
	\Question{Ecrire la relation de recurrence permettant de calculer les $S_{k,0}$ pour $1 \leqslant k \leqslant n-1$.}
	\Question{Montrer que pour $i$ et $j$ non nuls, $S_{i,j} = \max \left\{ S_{i-1,j}, S_{i,j-1} \right\} + M_{i,j}$.} 
    \Question{Ecrire une fonction \textit{récursive} qui calcule de façon descendante les coefficients de la matrice $S$.}
    \Question{Faire un schéma des appels récursifs de la fonction précédente, que peut-on en conclure ?}
	\Question{Ecrire une fonction OCaml \textit{itérative} {\tt somme\_maxi int array array -> int array array} qui prend en argument une matrice $M$ et calcule de façon ascendante les coefficients de la matrice $S$.}
	\Question{Déterminer la complexité de cette fonction en fonction $n$.}
\end{Exercise}

\begin{Exercise}[title = {Liste chaînée circulaire}]\\
    Le langage utilisé dans cette partie est le langage C. \smallskip\\ 
	On veut implémenter une structure de données de liste chainée circulaire avec un pointeur sur la queue telle que représentée ci-dessous :
	\begin{center}
		\begin{tabular}{ccllllc}
			                                           &                         &                         &                         &                         & \rnode{liste}{{\footnotesize queue}} & \\
			                                           &                         &                         &                         &                         &                                      & \\
			\rnode{head}{\raisebox{-2pt}{\phantom{Y}}} & \maillon{$e_0$}{v0}{p0} & \maillon{$e_1$}{v1}{p1} & \maillon{$e_2$}{v2}{p2} & \maillon{$e_3$}{v3}{p3} & \maillon{$e_4$}{v4}{p4}              & \\
		\end{tabular}\\
		\ncline[nodesepB=0.25cm,nodesepA=-0.25cm]{*->}{p0}{v1}
		\ncline[nodesepB=0.25cm,nodesepA=-0.25cm]{*->}{p1}{v2}
		\ncline[nodesepB=0.25cm,nodesepA=-0.25cm]{*->}{p2}{v3}
		\ncline[nodesepB=0.25cm,nodesepA=-0.25cm]{*->}{p3}{v4}
		\ncline[nodesepB=0.25cm,nodesepA=0.05cm]{*->}{liste}{v4}
		\ncbar[nodesepB=0.2cm,nodesepA=-0.2cm,angleA=-90,angleB=-90]{*->}{p4}{v0}
	\end{center} \bigskip
	La queue pointe toujours vers \textit{le dernier élément inséré} ainsi, après l'ajout d'un nouvel élément $e_5$, la structure de donnée ci-dessus devient :
	\begin{center}
		\begin{tabular}{ccllllcl}
			                                           &                         &                         &                         &                         &                         & \hspace{-0.6cm}\rnode{liste}{{\footnotesize queue}} & \\
			                                           &                         &                         &                         &                         &                         &                                                        \\
			\rnode{head}{\raisebox{-2pt}{\phantom{Y}}} & \maillon{$e_0$}{v0}{p0} & \maillon{$e_1$}{v1}{p1} & \maillon{$e_2$}{v2}{p2} & \maillon{$e_3$}{v3}{p3} & \maillon{$e_4$}{v4}{p4} & \maillon{$e_5$}{v5}{p5}                                \\
		\end{tabular}\\
		\ncline[nodesepB=0.25cm,nodesepA=-0.25cm]{*->}{p0}{v1}
		\ncline[nodesepB=0.25cm,nodesepA=-0.25cm]{*->}{p1}{v2}
		\ncline[nodesepB=0.25cm,nodesepA=-0.25cm]{*->}{p2}{v3}
		\ncline[nodesepB=0.25cm,nodesepA=-0.25cm]{*->}{p3}{v4}
		\ncline[nodesepB=0.25cm,nodesepA=-0.25cm]{*->}{p4}{v5}
		\ncline[nodesepB=0.25cm,nodesepA=0.05cm]{*->}{liste}{v5}
		\ncbar[nodesepB=0.2cm,nodesepA=-0.2cm,angleA=-90,angleB=-90]{*->}{p5}{v0}
	\end{center} \bigskip
	Lorsqu'on retire un élément de cette structure de données, on retire le maillon qui suit le dernier élément inséré. Par conséquent, si on retire un élément de la structure de donnée ci-dessus, c'est le maillon contenant $e_0$ qui est retiré et on obtient :
	\begin{center}
		\begin{tabular}{ccllllcl}
			                                           &  &                         &                         &                         &                         & \hspace{-0.6cm}\rnode{liste}{{\footnotesize queue}} & \\
			                                           &  &                         &                         &                         &                         &                                                        \\
			\rnode{head}{\raisebox{-2pt}{\phantom{Y}}} &  & \maillon{$e_1$}{v1}{p1} & \maillon{$e_2$}{v2}{p2} & \maillon{$e_3$}{v3}{p3} & \maillon{$e_4$}{v4}{p4} & \maillon{$e_5$}{v5}{p5}                                \\
		\end{tabular}\\
		\ncline[nodesepB=0.25cm,nodesepA=-0.25cm]{*->}{p1}{v2}
		\ncline[nodesepB=0.25cm,nodesepA=-0.25cm]{*->}{p2}{v3}
		\ncline[nodesepB=0.25cm,nodesepA=-0.25cm]{*->}{p3}{v4}
		\ncline[nodesepB=0.25cm,nodesepA=-0.25cm]{*->}{p4}{v5}
		\ncline[nodesepB=0.25cm,nodesepA=0.05cm]{*->}{liste}{v5}
		\ncbar[nodesepB=0.2cm,nodesepA=-0.2cm,angleA=-90,angleB=-90]{*->}{p5}{v1}
	\end{center} \bigskip
	Afin d'implémenter cette structure de données, on propose d'utiliser les types suivants
	\inputpartC{file.c}{}{}{5}{11}
	La liste chainée circulaire vide est alors représentée par le pointeur {\sc null}.
    \Question{En partant d'une liste chainée circulaire initialement vide, donner les étapes de son évolution après les opérations suivantes  : \begin{enumerate}
    \item {\tt ajouter 12}
    \item {\tt ajouter  6}
    \item {\tt ajouter  7}
    \item {\tt retirer }
    \item {\tt ajouter 42}
    \item {\tt retirer} 
    \end{enumerate}
    On précisera la valeur des entiers renvoyés par la fonction {\tt Retirer}.
    }
	\Question{Pour ajouter un élément dans une liste circulaire, on propose la fonction ci-dessous :
		\inputpartC{file.c}{}{}{35}{42}
		Indiquer, en justifiant votre réponse, quelle ligne de cette fonction est problématique si la liste circulaire {\tt lc} est vide.
	}
	\Question{Corriger la fonction {\tt ajouter\_errone} afin de traiter convenablement le cas où la liste circulaire {\tt lc} est vide.}
	\Question{Ecrire la fonction {\tt retirer} de signature \mintinline{c}{int retirer(liste_circulaire *lc)} qui prend en argument une liste circulaire \textit{supposée non vide} et renvoie la valeur du maillon situé après le pointeur de queue en le retirant de la liste circulaire.}
	\Question{Quelle structure de données connue a-t-on implémenté ici ? Justifier et proposer des noms plus appropriés pour les fonctions {\tt ajouter} et {\tt retirer}.}
	\Question{Ecrire une fonction {\tt longueur} de signature \mintinline{c}{int longueur(liste_circulaire *lc)} qui renvoie le nombre d'éléments d'une liste chaînée circulaire.}
	\Question{Donner, en les justifiant, les complexités des opérations {\tt retirer}, {\tt ajouter} et {\tt longueur}.}
\end{Exercise}

\end{document}
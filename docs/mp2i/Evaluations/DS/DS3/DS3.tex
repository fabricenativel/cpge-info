\PassOptionsToPackage{dvipsnames,table}{xcolor}
\documentclass[11pt,a4paper]{article}

\usepackage{DS}

\begin{document}

\newcommand{\ModeExercice}{
% Traduction des noms pour le package exercise
\renewcommand{\ExerciseName}{Exercice}
\renewcommand{\thesubQuestion}{\theQuestion.\alph{subQuestion}}
\renewcommand{\AnswerName}{Réponses à l'exercise }
}
\newcommand{\Fiche}[2]{\lhead{\textbf{{\sc #1}}}
\rhead{Niveau: \textbf{#2}}
\cfoot{}}
\definecolor{cfond}{gray}{0.4}
\renewcommand{\thealgocf}{}

\newcommand{\ModeActivite}{
% Traduction des noms pour le package exercise
\renewcommand{\ExerciseName}{Activité}
}
% Réglages de la mise en forme des exercices 
\renewcommand{\ExerciseHeaderTitle}{\ExerciseTitle}
\renewcommand{\ExerciseHeaderOrigin}{\ExerciseOrigin}
% Un : sépare le numéro de l'exerice de son titre ... si le titre existe. On utilise Origine pour placer les pictogrammes en fin de ligne
\renewcommand{\ExerciseHeader}{\ding{113} \textbf{\sffamily{\ExerciseName \ \ExerciseHeaderNB}} \ifthenelse{\equal{\ExerciseTitle}{\empty}}{}{:} \textit{\ExerciseHeaderTitle} \hfill \ExerciseHeaderOrigin}
\renewcommand{\ExePartHeader}{\quad {\footnotesize \ding{110}} \textbf{Partie \textbf{\ExePartHeaderNB}} : \ExePartName}


% Mode Concours
\newcommand{\ModeConcours}{
   \newcounter{qconcours}
   \setcounter{qconcours}{1}
   \renewcommand{\ExerciseName}{Exercice}
   \renewcommand{\ExerciseHeaderTitle}{\ExerciseTitle}
\renewcommand{\ExerciseHeaderOrigin}{\ExerciseOrigin}
\renewcommand{\ExerciseHeader}{\ding{113} \textbf{\sffamily{\ExerciseName \ \ExerciseHeaderNB}} \ifthenelse{\equal{\ExerciseTitle}{\empty}}{}{:} \textit{\ExerciseHeaderTitle} \hfill \ExerciseHeaderOrigin}
\renewcommand{\QuestionNB}{\textbf{Q\arabic{qconcours}--}\ \addtocounter{qconcours}{1}}}

\newcommand{\noeud}[1]{\Tr{\fbox{\tt #1}}}
\newcommand{\FPATH}{/home/fenarius/Travail/Cours/cpge-info/docs/mp2i/}
\newcommand{\spath}[2]{\FPATH Evaluations/#1/#1#2}
\definecolor{codebg}{gray}{0.90}
\newcommand{\inputpartOCaml}[5]{\begin{mdframed}[backgroundcolor=codebg] \inputminted[breaklines=true,fontsize=#3,linenos=true,highlightcolor=fluo,tabsize=2,highlightlines={#2},firstline=#4,lastline=#5,firstnumber=1]{OCaml}{#1} \end{mdframed}}
\newcommand{\inputpartPython}[5]{\begin{mdframed}[backgroundcolor=codebg] \inputminted[breaklines=true,fontsize=#3,linenos=true,highlightcolor=fluo,tabsize=2,highlightlines={#2},firstline=#4,lastline=#5,firstnumber=1]{python}{#1} \end{mdframed}}
\newcommand{\inputpartC}[5]{\begin{mdframed}[backgroundcolor=codebg] \inputminted[breaklines=true,fontsize=#3,linenos=true,highlightcolor=fluo,tabsize=2,highlightlines={#2},firstline=#4,lastline=#5,firstnumber=1]{c}{#1} \end{mdframed}}
\newcommand{\inputC}[2]{\begin{mdframed}[backgroundcolor=codebg] \inputminted[breaklines=true,fontsize=#2,linenos=true,highlightcolor=fluo,tabsize=2]{c}{#1} \end{mdframed}}
\newminted[langageC]{c}{linenos=true,escapeinside=``,highlightcolor=fluo,tabsize=2}
\newminted[python]{python}{linenos=true,escapeinside=``,highlightcolor=fluo,tabsize=4}
\BeforeBeginEnvironment{minted}{\begin{mdframed}[backgroundcolor=codebg,skipabove=0cm]}
   \AfterEndEnvironment{minted}{\end{mdframed}}

% Font light medium et bold pour tt :
\newcommand{\ttl}[1]{\ttfamily \fontseries{l}\selectfont #1}
\newcommand{\ttm}[1]{\ttfamily \fontseries{m}\selectfont #1}
\newcommand{\ttb}[1]{\ttfamily \fontseries{b}\selectfont #1}

%QCM de NSI \QNSI{Question}{R1}{R2}{R3}{R4}
\newcommand{\QNSI}[5]{
#1
\begin{enumerate}[label=\alph{enumi})]
\item #2
\item #3
\item #4
\item #5
\end{enumerate}
}



\definecolor{grispale}{gray}{0.95}
\newcommand{\htmlmode}{\lstset{language=html,numbers=left, tabsize=2, frame=single, breaklines=true, keywordstyle=\ttfamily, basicstyle=\small,
   numberstyle=\tiny\ttfamily, framexleftmargin=0mm, backgroundcolor=\color{grispale}, xleftmargin=12mm,showstringspaces=false}}
\newcommand{\pythonmode}{\lstset{language=python,numbers=left, tabsize=4, frame=single, breaklines=true, keywordstyle=\ttfamily, basicstyle=\small,
   numberstyle=\tiny\ttfamily, framexleftmargin=0mm, backgroundcolor=\color{grispale}, xleftmargin=12mm, showstringspaces=false}}
\newcommand{\bashmode}{\lstset{language=bash,numbers=left, tabsize=2, frame=single, breaklines=true, basicstyle=\ttfamily,
   numberstyle=\tiny\ttfamily, framexleftmargin=0mm, backgroundcolor=\color{grispale}, xleftmargin=12mm, showstringspaces=false}}
\newcommand{\exomode}{\lstset{language=python,numbers=left, tabsize=2, frame=single, breaklines=true, basicstyle=\ttfamily,
   numberstyle=\tiny\ttfamily, framexleftmargin=13mm, xleftmargin=12mm, basicstyle=\small, showstringspaces=false}}
   
   
  \lstset{%
        inputencoding=utf8,
        extendedchars=true,
        literate=%
        {é}{{\'{e}}}1
        {è}{{\`{e}}}1
        {ê}{{\^{e}}}1
        {ë}{{\¨{e}}}1
        {É}{{\'{E}}}1
        {Ê}{{\^{E}}}1
        {û}{{\^{u}}}1
        {ù}{{\`{u}}}1
        {ú}{{\'{u}}}1
        {â}{{\^{a}}}1
        {à}{{\`{a}}}1
        {á}{{\'{a}}}1
        {ã}{{\~{a}}}1
        {Á}{{\'{A}}}1
        {Â}{{\^{A}}}1
        {Ã}{{\~{A}}}1
        {ç}{{\c{c}}}1
        {Ç}{{\c{C}}}1
        {õ}{{\~{o}}}1
        {ó}{{\'{o}}}1
        {ô}{{\^{o}}}1
        {Õ}{{\~{O}}}1
        {Ó}{{\'{O}}}1
        {Ô}{{\^{O}}}1
        {î}{{\^{i}}}1
        {Î}{{\^{I}}}1
        {í}{{\'{i}}}1
        {Í}{{\~{Í}}}1
}

%tei pour placer les images
%tei{nom de l’image}{échelle de l’image}{sens}{texte a positionner}
%sens ="1" (droite) ou "2" (gauche)
\newlength{\ltxt}
\newcommand{\tei}[4]{
\setlength{\ltxt}{\linewidth}
\setbox0=\hbox{\includegraphics[scale=#2]{#1}}
\addtolength{\ltxt}{-\wd0}
\addtolength{\ltxt}{-10pt}
\ifthenelse{\equal{#3}{1}}{
\begin{minipage}{\wd0}
\includegraphics[scale=#2]{#1}
\end{minipage}
\hfill
\begin{minipage}{\ltxt}
#4
\end{minipage}
}{
\begin{minipage}{\ltxt}
#4
\end{minipage}
\hfill
\begin{minipage}{\wd0}
\includegraphics[scale=#2]{#1}
\end{minipage}
}
}

%Juxtaposition d'une image pspciture et de texte 
%#1: = code pstricks de l'image
%#2: largeur de l'image
%#3: hauteur de l'image
%#4: Texte à écrire
\newcommand{\ptp}[4]{
\setlength{\ltxt}{\linewidth}
\addtolength{\ltxt}{-#2 cm}
\addtolength{\ltxt}{-0.1 cm}
\begin{minipage}[b][#3 cm][t]{\ltxt}
#4
\end{minipage}\hfill
\begin{minipage}[b][#3 cm][c]{#2 cm}
#1
\end{minipage}\par
}



%Macros pour les graphiques
\psset{linewidth=0.5\pslinewidth,PointSymbol=x}
\setlength{\fboxrule}{0.5pt}
\newcounter{tempangle}

%Marque la longueur du segment d'extrémité  #1 et  #2 avec la valeur #3, #4 est la distance par rapport au segment (en %age de la valeur de celui ci) et #5 l'orientation du marquage : +90 ou -90
\newcommand{\afflong}[5]{
\pstRotation[RotAngle=#4,PointSymbol=none,PointName=none]{#1}{#2}[X] 
\pstHomO[PointSymbol=none,PointName=none,HomCoef=#5]{#1}{X}[Y]
\pstTranslation[PointSymbol=none,PointName=none]{#1}{#2}{Y}[Z]
 \ncline{|<->|,linewidth=0.25\pslinewidth}{Y}{Z} \ncput*[nrot=:U]{\footnotesize{#3}}
}
\newcommand{\afflongb}[3]{
\ncline{|<->|,linewidth=0}{#1}{#2} \naput*[nrot=:U]{\footnotesize{#3}}
}

%Construis le point #4 situé à #2 cm du point #1 avant un angle #3 par rapport à l'horizontale. #5 = liste de paramètre
\newcommand{\lsegment}[5]{\pstGeonode[PointSymbol=none,PointName=none](0,0){O'}(#2,0){I'} \pstTranslation[PointSymbol=none,PointName=none]{O'}{I'}{#1}[J'] \pstRotation[RotAngle=#3,PointSymbol=x,#5]{#1}{J'}[#4]}
\newcommand{\tsegment}[5]{\pstGeonode[PointSymbol=none,PointName=none](0,0){O'}(#2,0){I'} \pstTranslation[PointSymbol=none,PointName=none]{O'}{I'}{#1}[J'] \pstRotation[RotAngle=#3,PointSymbol=x,#5]{#1}{J'}[#4] \pstLineAB{#4}{#1}}

%Construis le point #4 situé à #3 cm du point #1 et faisant un angle de  90° avec la droite (#1,#2) #5 = liste de paramètre
\newcommand{\psegment}[5]{
\pstGeonode[PointSymbol=none,PointName=none](0,0){O'}(#3,0){I'}
 \pstTranslation[PointSymbol=none,PointName=none]{O'}{I'}{#1}[J']
 \pstInterLC[PointSymbol=none,PointName=none]{#1}{#2}{#1}{J'}{M1}{M2} \pstRotation[RotAngle=-90,PointSymbol=x,#5]{#1}{M1}[#4]
  }
  
%Construis le point #4 situé à #3 cm du point #1 et faisant un angle de  #5° avec la droite (#1,#2) #6 = liste de paramètre
\newcommand{\mlogo}[6]{
\pstGeonode[PointSymbol=none,PointName=none](0,0){O'}(#3,0){I'}
 \pstTranslation[PointSymbol=none,PointName=none]{O'}{I'}{#1}[J']
 \pstInterLC[PointSymbol=none,PointName=none]{#1}{#2}{#1}{J'}{M1}{M2} \pstRotation[RotAngle=#5,PointSymbol=x,#6]{#1}{M2}[#4]
  }

% Construis un triangle avec #1=liste des 3 sommets séparés par des virgules, #2=liste des 3 longueurs séparés par des virgules, #3 et #4 : paramètre d'affichage des 2e et 3 points et #5 : inclinaison par rapport à l'horizontale
%autre macro identique mais sans tracer les segments joignant les sommets
\noexpandarg
\newcommand{\Triangleccc}[5]{
\StrBefore{#1}{,}[\pointA]
\StrBetween[1,2]{#1}{,}{,}[\pointB]
\StrBehind[2]{#1}{,}[\pointC]
\StrBefore{#2}{,}[\coteA]
\StrBetween[1,2]{#2}{,}{,}[\coteB]
\StrBehind[2]{#2}{,}[\coteC]
\tsegment{\pointA}{\coteA}{#5}{\pointB}{#3} 
\lsegment{\pointA}{\coteB}{0}{Z1}{PointSymbol=none, PointName=none}
\lsegment{\pointB}{\coteC}{0}{Z2}{PointSymbol=none, PointName=none}
\pstInterCC{\pointA}{Z1}{\pointB}{Z2}{\pointC}{Z3} 
\pstLineAB{\pointA}{\pointC} \pstLineAB{\pointB}{\pointC}
\pstSymO[PointName=\pointC,#4]{C}{C}[C]
}
\noexpandarg
\newcommand{\TrianglecccP}[5]{
\StrBefore{#1}{,}[\pointA]
\StrBetween[1,2]{#1}{,}{,}[\pointB]
\StrBehind[2]{#1}{,}[\pointC]
\StrBefore{#2}{,}[\coteA]
\StrBetween[1,2]{#2}{,}{,}[\coteB]
\StrBehind[2]{#2}{,}[\coteC]
\tsegment{\pointA}{\coteA}{#5}{\pointB}{#3} 
\lsegment{\pointA}{\coteB}{0}{Z1}{PointSymbol=none, PointName=none}
\lsegment{\pointB}{\coteC}{0}{Z2}{PointSymbol=none, PointName=none}
\pstInterCC[PointNameB=none,PointSymbolB=none,#4]{\pointA}{Z1}{\pointB}{Z2}{\pointC}{Z1} 
}


% Construis un triangle avec #1=liste des 3 sommets séparés par des virgules, #2=liste formée de 2 longueurs et d'un angle séparés par des virgules, #3 et #4 : paramètre d'affichage des 2e et 3 points et #5 : inclinaison par rapport à l'horizontale
%autre macro identique mais sans tracer les segments joignant les sommets
\newcommand{\Trianglecca}[5]{
\StrBefore{#1}{,}[\pointA]
\StrBetween[1,2]{#1}{,}{,}[\pointB]
\StrBehind[2]{#1}{,}[\pointC]
\StrBefore{#2}{,}[\coteA]
\StrBetween[1,2]{#2}{,}{,}[\coteB]
\StrBehind[2]{#2}{,}[\angleA]
\tsegment{\pointA}{\coteA}{#5}{\pointB}{#3} 
\setcounter{tempangle}{#5}
\addtocounter{tempangle}{\angleA}
\tsegment{\pointA}{\coteB}{\thetempangle}{\pointC}{#4}
\pstLineAB{\pointB}{\pointC}
}
\newcommand{\TriangleccaP}[5]{
\StrBefore{#1}{,}[\pointA]
\StrBetween[1,2]{#1}{,}{,}[\pointB]
\StrBehind[2]{#1}{,}[\pointC]
\StrBefore{#2}{,}[\coteA]
\StrBetween[1,2]{#2}{,}{,}[\coteB]
\StrBehind[2]{#2}{,}[\angleA]
\lsegment{\pointA}{\coteA}{#5}{\pointB}{#3} 
\setcounter{tempangle}{#5}
\addtocounter{tempangle}{\angleA}
\lsegment{\pointA}{\coteB}{\thetempangle}{\pointC}{#4}
}

% Construis un triangle avec #1=liste des 3 sommets séparés par des virgules, #2=liste formée de 1 longueurs et de deux angle séparés par des virgules, #3 et #4 : paramètre d'affichage des 2e et 3 points et #5 : inclinaison par rapport à l'horizontale
%autre macro identique mais sans tracer les segments joignant les sommets
\newcommand{\Trianglecaa}[5]{
\StrBefore{#1}{,}[\pointA]
\StrBetween[1,2]{#1}{,}{,}[\pointB]
\StrBehind[2]{#1}{,}[\pointC]
\StrBefore{#2}{,}[\coteA]
\StrBetween[1,2]{#2}{,}{,}[\angleA]
\StrBehind[2]{#2}{,}[\angleB]
\tsegment{\pointA}{\coteA}{#5}{\pointB}{#3} 
\setcounter{tempangle}{#5}
\addtocounter{tempangle}{\angleA}
\lsegment{\pointA}{1}{\thetempangle}{Z1}{PointSymbol=none, PointName=none}
\setcounter{tempangle}{#5}
\addtocounter{tempangle}{180}
\addtocounter{tempangle}{-\angleB}
\lsegment{\pointB}{1}{\thetempangle}{Z2}{PointSymbol=none, PointName=none}
\pstInterLL[#4]{\pointA}{Z1}{\pointB}{Z2}{\pointC}
\pstLineAB{\pointA}{\pointC}
\pstLineAB{\pointB}{\pointC}
}
\newcommand{\TrianglecaaP}[5]{
\StrBefore{#1}{,}[\pointA]
\StrBetween[1,2]{#1}{,}{,}[\pointB]
\StrBehind[2]{#1}{,}[\pointC]
\StrBefore{#2}{,}[\coteA]
\StrBetween[1,2]{#2}{,}{,}[\angleA]
\StrBehind[2]{#2}{,}[\angleB]
\lsegment{\pointA}{\coteA}{#5}{\pointB}{#3} 
\setcounter{tempangle}{#5}
\addtocounter{tempangle}{\angleA}
\lsegment{\pointA}{1}{\thetempangle}{Z1}{PointSymbol=none, PointName=none}
\setcounter{tempangle}{#5}
\addtocounter{tempangle}{180}
\addtocounter{tempangle}{-\angleB}
\lsegment{\pointB}{1}{\thetempangle}{Z2}{PointSymbol=none, PointName=none}
\pstInterLL[#4]{\pointA}{Z1}{\pointB}{Z2}{\pointC}
}

%Construction d'un cercle de centre #1 et de rayon #2 (en cm)
\newcommand{\Cercle}[2]{
\lsegment{#1}{#2}{0}{Z1}{PointSymbol=none, PointName=none}
\pstCircleOA{#1}{Z1}
}

%construction d'un parallélogramme #1 = liste des sommets, #2 = liste contenant les longueurs de 2 côtés consécutifs et leurs angles;  #3, #4 et #5 : paramètre d'affichage des sommets #6 inclinaison par rapport à l'horizontale 
% meme macro sans le tracé des segements
\newcommand{\Para}[6]{
\StrBefore{#1}{,}[\pointA]
\StrBetween[1,2]{#1}{,}{,}[\pointB]
\StrBetween[2,3]{#1}{,}{,}[\pointC]
\StrBehind[3]{#1}{,}[\pointD]
\StrBefore{#2}{,}[\longueur]
\StrBetween[1,2]{#2}{,}{,}[\largeur]
\StrBehind[2]{#2}{,}[\angle]
\tsegment{\pointA}{\longueur}{#6}{\pointB}{#3} 
\setcounter{tempangle}{#6}
\addtocounter{tempangle}{\angle}
\tsegment{\pointA}{\largeur}{\thetempangle}{\pointD}{#5}
\pstMiddleAB[PointName=none,PointSymbol=none]{\pointB}{\pointD}{Z1}
\pstSymO[#4]{Z1}{\pointA}[\pointC]
\pstLineAB{\pointB}{\pointC}
\pstLineAB{\pointC}{\pointD}
}
\newcommand{\ParaP}[6]{
\StrBefore{#1}{,}[\pointA]
\StrBetween[1,2]{#1}{,}{,}[\pointB]
\StrBetween[2,3]{#1}{,}{,}[\pointC]
\StrBehind[3]{#1}{,}[\pointD]
\StrBefore{#2}{,}[\longueur]
\StrBetween[1,2]{#2}{,}{,}[\largeur]
\StrBehind[2]{#2}{,}[\angle]
\lsegment{\pointA}{\longueur}{#6}{\pointB}{#3} 
\setcounter{tempangle}{#6}
\addtocounter{tempangle}{\angle}
\lsegment{\pointA}{\largeur}{\thetempangle}{\pointD}{#5}
\pstMiddleAB[PointName=none,PointSymbol=none]{\pointB}{\pointD}{Z1}
\pstSymO[#4]{Z1}{\pointA}[\pointC]
}


%construction d'un cerf-volant #1 = liste des sommets, #2 = liste contenant les longueurs de 2 côtés consécutifs et leurs angles;  #3, #4 et #5 : paramètre d'affichage des sommets #6 inclinaison par rapport à l'horizontale 
% meme macro sans le tracé des segements
\newcommand{\CerfVolant}[6]{
\StrBefore{#1}{,}[\pointA]
\StrBetween[1,2]{#1}{,}{,}[\pointB]
\StrBetween[2,3]{#1}{,}{,}[\pointC]
\StrBehind[3]{#1}{,}[\pointD]
\StrBefore{#2}{,}[\longueur]
\StrBetween[1,2]{#2}{,}{,}[\largeur]
\StrBehind[2]{#2}{,}[\angle]
\tsegment{\pointA}{\longueur}{#6}{\pointB}{#3} 
\setcounter{tempangle}{#6}
\addtocounter{tempangle}{\angle}
\tsegment{\pointA}{\largeur}{\thetempangle}{\pointD}{#5}
\pstOrtSym[#4]{\pointB}{\pointD}{\pointA}[\pointC]
\pstLineAB{\pointB}{\pointC}
\pstLineAB{\pointC}{\pointD}
}

%construction d'un quadrilatère quelconque #1 = liste des sommets, #2 = liste contenant les longueurs des 4 côtés et l'angle entre 2 cotés consécutifs  #3, #4 et #5 : paramètre d'affichage des sommets #6 inclinaison par rapport à l'horizontale 
% meme macro sans le tracé des segements
\newcommand{\Quadri}[6]{
\StrBefore{#1}{,}[\pointA]
\StrBetween[1,2]{#1}{,}{,}[\pointB]
\StrBetween[2,3]{#1}{,}{,}[\pointC]
\StrBehind[3]{#1}{,}[\pointD]
\StrBefore{#2}{,}[\coteA]
\StrBetween[1,2]{#2}{,}{,}[\coteB]
\StrBetween[2,3]{#2}{,}{,}[\coteC]
\StrBetween[3,4]{#2}{,}{,}[\coteD]
\StrBehind[4]{#2}{,}[\angle]
\tsegment{\pointA}{\coteA}{#6}{\pointB}{#3} 
\setcounter{tempangle}{#6}
\addtocounter{tempangle}{\angle}
\tsegment{\pointA}{\coteD}{\thetempangle}{\pointD}{#5}
\lsegment{\pointB}{\coteB}{0}{Z1}{PointSymbol=none, PointName=none}
\lsegment{\pointD}{\coteC}{0}{Z2}{PointSymbol=none, PointName=none}
\pstInterCC[PointNameA=none,PointSymbolA=none,#4]{\pointB}{Z1}{\pointD}{Z2}{Z3}{\pointC} 
\pstLineAB{\pointB}{\pointC}
\pstLineAB{\pointC}{\pointD}
}


% Définition des colonnes centrées ou à droite pour tabularx
\newcolumntype{Y}{>{\centering\arraybackslash}X}
\newcolumntype{Z}{>{\flushright\arraybackslash}X}

%Les pointillés à remplir par les élèves
\newcommand{\po}[1]{\makebox[#1 cm]{\dotfill}}
\newcommand{\lpo}[1][3]{%
\multido{}{#1}{\makebox[\linewidth]{\dotfill}
}}

%Liste des pictogrammes utilisés sur la fiche d'exercice ou d'activités
\newcommand{\bombe}{\faBomb}
\newcommand{\livre}{\faBook}
\newcommand{\calculatrice}{\faCalculator}
\newcommand{\oral}{\faCommentO}
\newcommand{\surfeuille}{\faEdit}
\newcommand{\ordinateur}{\faLaptop}
\newcommand{\ordi}{\faDesktop}
\newcommand{\ciseaux}{\faScissors}
\newcommand{\danger}{\faExclamationTriangle}
\newcommand{\out}{\faSignOut}
\newcommand{\cadeau}{\faGift}
\newcommand{\flash}{\faBolt}
\newcommand{\lumiere}{\faLightbulb}
\newcommand{\compas}{\dsmathematical}
\newcommand{\calcullitteral}{\faTimesCircleO}
\newcommand{\raisonnement}{\faCogs}
\newcommand{\recherche}{\faSearch}
\newcommand{\rappel}{\faHistory}
\newcommand{\video}{\faFilm}
\newcommand{\capacite}{\faPuzzlePiece}
\newcommand{\aide}{\faLifeRing}
\newcommand{\loin}{\faExternalLink}
\newcommand{\groupe}{\faUsers}
\newcommand{\bac}{\faGraduationCap}
\newcommand{\histoire}{\faUniversity}
\newcommand{\coeur}{\faSave}
\newcommand{\os}{\faMicrochip}
\newcommand{\rd}{\faCubes}
\newcommand{\data}{\faColumns}
\newcommand{\web}{\faCode}
\newcommand{\prog}{\faFile}
\newcommand{\algo}{\faCogs}
\newcommand{\important}{\faExclamationCircle}
\newcommand{\maths}{\faTimesCircle}
% Traitement des données en tables
\newcommand{\tables}{\faColumns}
% Types construits
\newcommand{\construits}{\faCubes}
% Type et valeurs de base
\newcommand{\debase}{{\footnotesize \faCube}}
% Systèmes d'exploitation
\newcommand{\linux}{\faLinux}
\newcommand{\sd}{\faProjectDiagram}
\newcommand{\bd}{\faDatabase}

%Les ensembles de nombres
\renewcommand{\N}{\mathbb{N}}
\newcommand{\D}{\mathbb{D}}
\newcommand{\Z}{\mathbb{Z}}
\newcommand{\Q}{\mathbb{Q}}
\newcommand{\R}{\mathbb{R}}
\newcommand{\C}{\mathbb{C}}

%Ecriture des vecteurs
\newcommand{\vect}[1]{\vbox{\halign{##\cr 
  \tiny\rightarrowfill\cr\noalign{\nointerlineskip\vskip1pt} 
  $#1\mskip2mu$\cr}}}


%Compteur activités/exos et question et mise en forme titre et questions
\newcounter{numact}
\setcounter{numact}{1}
\newcounter{numseance}
\setcounter{numseance}{1}
\newcounter{numexo}
\setcounter{numexo}{0}
\newcounter{numprojet}
\setcounter{numprojet}{0}
\newcounter{numquestion}
\newcommand{\espace}[1]{\rule[-1ex]{0pt}{#1 cm}}
\newcommand{\Quest}[3]{
\addtocounter{numquestion}{1}
\begin{tabularx}{\textwidth}{X|m{1cm}|}
\cline{2-2}
\textbf{\sffamily{\alph{numquestion})}} #1 & \dots / #2 \\
\hline 
\multicolumn{2}{|l|}{\espace{#3}} \\
\hline
\end{tabularx}
}
\newcommand{\mq}[1]
{\ding{113} \addtocounter{numquestion}{1}
\textbf{Question \arabic{numquestion}} \\ #1}
\newcommand{\QuestR}[3]{
\addtocounter{numquestion}{1}
\begin{tabularx}{\textwidth}{X|m{1cm}|}
\cline{2-2}
\textbf{\sffamily{\alph{numquestion})}} #1 & \dots / #2 \\
\hline 
\multicolumn{2}{|l|}{\cor{#3}} \\
\hline
\end{tabularx}
}
\newcommand{\Pre}{{\sc nsi} 1\textsuperscript{e}}
\newcommand{\Term}{{\sc nsi} Terminale}
\newcommand{\Sec}{2\textsuperscript{e}}
\newcommand{\Exo}[2]{ \addtocounter{numexo}{1} \ding{113} \textbf{\sffamily{Exercice \thenumexo}} : \textit{#1} \hfill #2  \setcounter{numquestion}{0}}
\newcommand{\Projet}[1]{ \addtocounter{numprojet}{1} \ding{118} \textbf{\sffamily{Projet \thenumprojet}} : \textit{#1}}
\newcommand{\ExoD}[2]{ \addtocounter{numexo}{1} \ding{113} \textbf{\sffamily{Exercice \thenumexo}}  \textit{(#1 pts)} \hfill #2  \setcounter{numquestion}{0}}
\newcommand{\ExoB}[2]{ \addtocounter{numexo}{1} \ding{113} \textbf{\sffamily{Exercice \thenumexo}}  \textit{(Bonus de +#1 pts maximum)} \hfill #2  \setcounter{numquestion}{0}}
\newcommand{\Act}[2]{ \ding{113} \textbf{\sffamily{Activité \thenumact}} : \textit{#1} \hfill #2  \addtocounter{numact}{1} \setcounter{numquestion}{0}}
\newcommand{\Seance}{ \rule{1.5cm}{0.5pt}\raisebox{-3pt}{\framebox[4cm]{\textbf{\sffamily{Séance \thenumseance}}}}\hrulefill  \\
  \addtocounter{numseance}{1}}
\newcommand{\Acti}[2]{ {\footnotesize \ding{117}} \textbf{\sffamily{Activité \thenumact}} : \textit{#1} \hfill #2  \addtocounter{numact}{1} \setcounter{numquestion}{0}}
\newcommand{\titre}[1]{\begin{Large}\textbf{\ding{118}}\end{Large} \begin{large}\textbf{ #1}\end{large} \vspace{0.2cm}}
\newcommand{\QListe}[1][0]{
\ifthenelse{#1=0}
{\begin{enumerate}[partopsep=0pt,topsep=0pt,parsep=0pt,itemsep=0pt,label=\textbf{\sffamily{\arabic*.}},series=question]}
{\begin{enumerate}[resume*=question]}}
\newcommand{\SQListe}[1][0]{
\ifthenelse{#1=0}
{\begin{enumerate}[partopsep=0pt,topsep=0pt,parsep=0pt,itemsep=0pt,label=\textbf{\sffamily{\alph*)}},series=squestion]}
{\begin{enumerate}[resume*=squestion]}}
\newcommand{\SQListeL}[1][0]{
\ifthenelse{#1=0}
{\begin{enumerate*}[partopsep=0pt,topsep=0pt,parsep=0pt,itemsep=0pt,label=\textbf{\sffamily{\alph*)}},series=squestion]}
{\begin{enumerate*}[resume*=squestion]}}
\newcommand{\FinListe}{\end{enumerate}}
\newcommand{\FinListeL}{\end{enumerate*}}

%Mise en forme de la correction
\newboolean{corrige}
\setboolean{corrige}{false}
\newcommand{\scor}[1]{\par \textcolor{blue!75!black}{\small #1}}
\newcommand{\cor}[1]{\par \textcolor{blue!75!black}{#1}}
\newcommand{\br}[1]{\cor{\textbf{#1}}}
\newcommand{\tcor}[1]{
\ifthenelse{\boolean{corrige}}{\begin{tcolorbox}[width=\linewidth,colback={white},colbacktitle=white,coltitle=green!50!black,colframe=green!50!black,boxrule=0.2mm]   
\cor{#1}
\end{tcolorbox}}{}
}
\newcommand{\iscor}[1]{\ifthenelse{\boolean{corrige}}{#1}}
\newcommand{\rc}[1]{\textcolor{OliveGreen}{#1}}

%Référence aux exercices par leur numéro
\newcommand{\refexo}[1]{
\refstepcounter{numexo}
\addtocounter{numexo}{-1}
\label{#1}}

%Séparation entre deux activités
\newcommand{\separateur}{\begin{center}
\rule{1.5cm}{0.5pt}\raisebox{-3pt}{\ding{117}}\rule{1.5cm}{0.5pt}  \vspace{0.2cm}
\end{center}}

%Entête et pied de page
\newcommand{\snt}[1]{\lhead{\textbf{SNT -- La photographie numérique} \rhead{\textit{Lycée Nord}}}}
\newcommand{\Activites}[2]{\lhead{\textbf{{\sc #1}}}
\rhead{Activités -- \textbf{#2}}
\cfoot{}}
\newcommand{\Exos}[2]{\lhead{\textbf{Fiche d'exercices: {\sc #1}}}
\rhead{Niveau: \textbf{#2}}
\cfoot{}}
\newcommand{\TD}[2]{\lhead{\textbf{TD #1} : {\sc #2} }
\rhead{{\sc mp2i -- Lycée Leconte de Lisle}}
\cfoot{}}
\newcommand{\Colles}[2]{\lhead{{\sc mp2i -- }\textbf{Colles d'informatique #1}} 
\rhead{{\sc #2}}
\cfoot{}}
\newcommand{\Devoir}[2]{\lhead{\textbf{Devoir de mathématiques : {\sc #1}}}
\rhead{\textbf{#2}} \setlength{\fboxsep}{8pt}
\begin{center}
%Titre de la fiche
\fbox{\parbox[b][1cm][t]{0.3\textwidth}{Nom : \hfill \po{3} \par \vfill Prénom : \hfill \po{3}} } \hfill 
\fbox{\parbox[b][1cm][t]{0.6\textwidth}{Note : \po{1} / 20} }
\end{center} \cfoot{}}
\newcommand{\TPnote}[3]{\lhead{\textbf{TP noté d'informatique n° #1}}
\rhead{\textbf{#2}} \setlength{\fboxsep}{8pt}
\ifthenelse{\boolean{corrige}}{}
{\begin{center}
\fbox{\parbox[b][1cm][t]{0.3\textwidth}{Nom : \hfill \po{3} \par \vfill Prénom : \hfill \po{3}} } \hfill 
\fbox{\parbox[b][1cm][t]{0.6\textwidth}{Note : \po{1} / #3} }
\end{center}} \cfoot{}}
\newcommand{\IC}[2]{\lhead{\textbf{Interro de cours n° #1}}
\rhead{{\sc mp2i --} \textbf{#2}} \setlength{\fboxsep}{8pt}
\ifthenelse{\boolean{corrige}}{}
{\begin{center}
%Titre de la fiche
\fbox{\parbox[b][1cm][t]{0.3\textwidth}{Nom : \hfill \po{3} \par \vfill Prénom : \hfill \po{3}} } \hfill 
\fbox{\parbox[b][1cm][t]{0.6\textwidth}{Note : \po{1} / 10} }
\end{center}}\cfoot{}}
\newcommand{\DS}[3]{\lhead{{#1} : \textbf{DS d'informatique n° #2}}
\rhead{Lycée Leconte de Lisle -- #3} \setlength{\fboxsep}{8pt}
%Titre de la fiche
\begin{center}
   {\Large \textbf{Devoir surveillé d'informatique}}
\end{center} \cfoot{\thepage/\pageref{LastPage}}}

\newcommand{\CB}[2]{\lhead{{#1} : \textbf{Councours blanc - Informatique}}
\rhead{Lycée Leconte de Lisle -- #2} \setlength{\fboxsep}{8pt}
%Titre de la fiche
\begin{center}
   {\Large \textbf{Concours Blanc - Epreuve d'informatique}}
\end{center} \cfoot{\thepage/\pageref{LastPage}}}

\newcommand{\PC}[3]{\lhead{Concours {#1} -- #2}
\rhead{Lycée Leconte de Lisle} \setlength{\fboxsep}{8pt}
%Titre de la fiche
\begin{center}
   {\Large \textbf{Proposition de corrigé}}
\end{center} \cfoot{\thepage/\pageref{LastPage}}}

\newcounter{numdspart}
\setcounter{numdspart}{1}
\newcommand{\DSPart}{\bigskip
   \hrulefill\raisebox{-3pt}{\framebox[4cm]{\textbf{\textbf{Partie \thenumdspart}}}}\hrulefill
   \addtocounter{numdspart}{1}
   \bigskip}

\newcommand{\Sauvegarde}[1]{
   \begin{tcolorbox}[title=\textcolor{black}{\danger\; Attention},colbacktitle=lightgray]
      {Tous vos programmes doivent être enregistrés dans votre dossier personnel, dans {\tt Evaluations}{\tt \textbackslash}{\tt #1}
      }
   \end{tcolorbox}
}

\newcommand{\alertbox}[3]{
   \begin{tcolorbox}[title=\textcolor{black}{#1\; #2},colbacktitle=lightgray]
      {#3}
   \end{tcolorbox}
}

%Devoir programmation en NSI (pas à rendre sur papier)
\newcommand{\PNSI}[2]{\lhead{\textbf{Devoir de {\sc nsi} : \textsf{ #1}}
}
\rhead{\textbf{#2}} \setlength{\fboxsep}{8pt}
 \cfoot{}
 \begin{center}{\Large \textbf{Evaluation de {\sc nsi}}}\end{center}}


%Devoir de NSI
\newcommand{\DNSI}[2]{\lhead{\textbf{Devoir de {\sc nsi} : \textsf{ #1}}}
\rhead{\textbf{#2}} \setlength{\fboxsep}{8pt}
\begin{center}
%Titre de la fiche
\fbox{\parbox[b][1cm][t]{0.3\textwidth}{Nom : \hfill \po{3} \par \vfill Prénom : \hfill \po{3}} } \hfill 
\fbox{\parbox[b][1cm][t]{0.6\textwidth}{Note : \po{1} / 10} }
\end{center} \cfoot{}}

\newcommand{\DevoirNSI}[2]{\lhead{\textbf{Devoir de {\sc nsi} : {\sc #1}}}
\rhead{\textbf{#2}} \setlength{\fboxsep}{8pt}
\cfoot{}}

%La définition de la commande QCM pour auto-multiple-choice
%En premier argument le sujet du qcm, deuxième argument : la classe, 3e : la durée prévue et #4 : présence ou non de questions avec plusieurs bonnes réponses
\newcommand{\QCM}[4]{
{\large \textbf{\ding{52} QCM : #1}} -- Durée : \textbf{#3 min} \hfill {\large Note : \dots/10} 
\hrule \vspace{0.1cm}\namefield{}
Nom :  \textbf{\textbf{\nom{}}} \qquad \qquad Prénom :  \textbf{\prenom{}}  \hfill Classe: \textbf{#2}
\vspace{0.2cm}
\hrule  
\begin{itemize}[itemsep=0pt]
\item[-] \textit{Une bonne réponse vaut un point, une absence de réponse n'enlève pas de point. }
\item[\danger] \textit{Une mauvaise réponse enlève un point.}
\ifthenelse{#4=1}{\item[-] \textit{Les questions marquées du symbole \multiSymbole{} peuvent avoir plusieurs bonnes réponses possibles.}}{}
\end{itemize}
}
\newcommand{\DevoirC}[2]{
\renewcommand{\footrulewidth}{0.5pt}
\lhead{\textbf{Devoir de mathématiques : {\sc #1}}}
\rhead{\textbf{#2}} \setlength{\fboxsep}{8pt}
\fbox{\parbox[b][0.4cm][t]{0.955\textwidth}{Nom : \po{5} \hfill Prénom : \po{5} \hfill Classe: \textbf{1}\textsuperscript{$\dots$}} } 
\rfoot{\thepage} \cfoot{} \lfoot{Lycée Nord}}
\newcommand{\DevoirInfo}[2]{\lhead{\textbf{Evaluation : {\sc #1}}}
\rhead{\textbf{#2}} \setlength{\fboxsep}{8pt}
 \cfoot{}}
\newcommand{\DM}[2]{\lhead{\textbf{Devoir maison à rendre le #1}} \rhead{\textbf{#2}}}

%Macros permettant l'affichage des touches de la calculatrice
%Touches classiques : #1 = 0 fond blanc pour les nombres et #1= 1gris pour les opérations et entrer, second paramètre=contenu
%Si #2=1 touche arrondi avec fond gris
\newcommand{\TCalc}[2]{
\setlength{\fboxsep}{0.1pt}
\ifthenelse{#1=0}
{\psframebox[fillstyle=solid, fillcolor=white]{\parbox[c][0.25cm][c]{0.6cm}{\centering #2}}}
{\ifthenelse{#1=1}
{\psframebox[fillstyle=solid, fillcolor=lightgray]{\parbox[c][0.25cm][c]{0.6cm}{\centering #2}}}
{\psframebox[framearc=.5,fillstyle=solid, fillcolor=white]{\parbox[c][0.25cm][c]{0.6cm}{\centering #2}}}
}}
\newcommand{\Talpha}{\psdblframebox[fillstyle=solid, fillcolor=white]{\hspace{-0.05cm}\parbox[c][0.25cm][c]{0.65cm}{\centering \scriptsize{alpha}}} \;}
\newcommand{\Tsec}{\psdblframebox[fillstyle=solid, fillcolor=white]{\parbox[c][0.25cm][c]{0.6cm}{\centering \scriptsize 2nde}} \;}
\newcommand{\Tfx}{\psdblframebox[fillstyle=solid, fillcolor=white]{\parbox[c][0.25cm][c]{0.6cm}{\centering \scriptsize $f(x)$}} \;}
\newcommand{\Tvar}{\psframebox[framearc=.5,fillstyle=solid, fillcolor=white]{\hspace{-0.22cm} \parbox[c][0.25cm][c]{0.82cm}{$\scriptscriptstyle{X,T,\theta,n}$}}}
\newcommand{\Tgraphe}{\psdblframebox[fillstyle=solid, fillcolor=white]{\hspace{-0.08cm}\parbox[c][0.25cm][c]{0.68cm}{\centering \tiny{graphe}}} \;}
\newcommand{\Tfen}{\psdblframebox[fillstyle=solid, fillcolor=white]{\hspace{-0.08cm}\parbox[c][0.25cm][c]{0.68cm}{\centering \tiny{fenêtre}}} \;}
\newcommand{\Ttrace}{\psdblframebox[fillstyle=solid, fillcolor=white]{\parbox[c][0.25cm][c]{0.6cm}{\centering \scriptsize{trace}}} \;}

% Macroi pour l'affichage  d'un entier n dans  une base b
\newcommand{\base}[2]{ \overline{#1}^{#2}}
% Intervalle d'entiers
\newcommand{\intN}[2]{\llbracket #1; #2 \rrbracket}
% Cadre avec lignes réponses
\def\gaddtotok#1{\global\tabtok\expandafter{\the\tabtok#1}}
\newtoks\tabtok
\newcommand*\reponse[2]{%
   \ifthenelse{\boolean{corrige}}{}{
	\global\tabtok{\\ \renewcommand{\arraystretch}{1.4}\begin{tabularx}{\linewidth}{|X|p{1cm}|}\hline \dotfill & \cellcolor{gray!30}{\small \dots/#2} \\ \cline{2-2}}%
	\multido{}{#1}{\gaddtotok{ \multicolumn{2}{|>{\hsize=\dimexpr1\hsize+2\tabcolsep+\arrayrulewidth+1cm\relax}X|}{\dotfill}\\ }}%
	\gaddtotok{\hline \end{tabularx}}%
	\the\tabtok
   }}

\newcommand{\PE}[1]{\left \lfloor #1 \right \lfloor}}
\ModeExercice
\DS{MP2I}{3}{Décembre 2024}

\setboolean{corrige}{false}

\newcommand{\maillon}[3]{
	\begin{tabular}{|p{0.2cm}|p{0.2cm}|}
		\hline
		\rnode{#2}{#1} & \rnode{#3}{\phantom{$e_0$}} \\
		\hline
	\end{tabular}
}

\alertbox{\danger}{Consignes}{
	\begin{itemize}
		\item[\textbullet] Les programmes demandés doivent être écrits en C ou en OCaml. Dans le cas du C, on suppose que les librairies standards usuelles ({\tt <stdio.h>}, {\tt <stdlib.h>}, {\tt <stdbool.h>}, {\tt <stdassert.h>}, \dots) sont déjà importées.
		\item[\textbullet] On pourra toujours librement utiliser une fonction demandée à une question précédente même si cette question n'a pas été traitée.
		\item[\textbullet] Veillez à présenter vos idées et vos réponses partielles même si vous ne trouvez pas la solution complète à une question.
		\item[\textbullet] La clarté et la lisibilité de la rédaction et des programmes sont des éléments de notation.
	\end{itemize}
}


\begin{Exercise}[title={Analyse d'un algorithme}]
	\begin{quote}
		\og{} \textit{Le problème du drapeau hollandais est un problème de programmation, présenté par Edsger Dijkstra, qui consiste à réorganiser une collection d'éléments identifiés par leur couleur, sachant que seules trois couleurs sont présentes (par exemple, rouge, blanc, bleu, dans le cas du drapeau des Pays-Bas).
			Étant donné un nombre quelconque de balles rouges, blanches et bleues alignées dans n'importe quel ordre, le problème est à les réarranger dans le bon ordre : les bleues d'abord, puis les blanches, puis les rouges.}\fg{}
		\begin{flushright}
			(Wikipedia)
		\end{flushright}
	\end{quote}
	On suppose déjà écrite la fonction {\tt echange} de prototype \mintinline{c}{void echange(int tab[], int i, int j)} qui échange les éléments d'indice {\tt i} et {\tt j} dans le tableau {\tt tab} et on considère dans la suite que cet échange s'effectue en temps constant. On donne ci-dessous une implémentation de  l'algorithme du drapeau hollandais en langage C permettant de réarranger les éléments d'un tableau ne contenant les trois valeurs entières 1, 2 et 3  :
	\inputpartC{hollandais.c}{}{}{22}{45}
	\Question{Etude de l'algorithme du drapeau hollandais.}
	\subQuestion{Faire fonctionner cet algorithme sur le tableau {\tt tab = \{1, 3, 2, 2, 3, 1\}}, et donner le contenu de {\tt tab} ainsi que celui des variables {\tt i1}, {\tt i2}, {\tt i3} lors du déroulement de l'algorithme en recopiant et complétant le tableau suivant : \smallskip \\
		\renewcommand{\arraystretch}{1.5}
		\begin{tabular}{|l|>{\tt }c|>{\tt}c|>{\tt}c|>{\tt}c|}
			\cline{2-5}
			\multicolumn{1}{c|}{} & tab                                                                                                                                       & i1             & i2             & i3             \\
			\hline
			Initialisation        & \{\rnode{d0}{1}, 3, 2, 2, 3, \rnode{f0}{1}\}                                                                                              & 0              & 5              & 5              \\
			\hline
			Etape 1               & \comp{\cor{\{1, \rnode{d1}{3}, 2, 2, 3, \rnode{f1}{1}\}} \ncbar[angle=-90,arm=0.1cm,nodesep=0.05cm,linewidth=1pt,linecolor=gray]{d1}{f1}} & \comp{\cor{1}} & \comp{\cor{5}} & \comp{\cor{5}} \\
			\hline
			Etape 2               & \comp{\cor{\{1, \rnode{d2}{1}, 2, 2, \rnode{f2}{3}, 3\}} \ncbar[angle=-90,arm=0.1cm,nodesep=0.05cm,linewidth=1pt,linecolor=gray]{d2}{f2}} & \comp{\cor{1}} & \comp{\cor{4}} & \comp{\cor{4}} \\
			\hline
			Etape 3               & \comp{\cor{\{1, 1, \rnode{d3}{2}, 2, \rnode{f3}{3}, 3\}} \ncbar[angle=-90,arm=0.1cm,nodesep=0.05cm,linewidth=1pt,linecolor=gray]{d3}{f3}} & \comp{\cor{2}} & \comp{\cor{4}} & \comp{\cor{4}} \\
			\hline
			Etape 4               & \comp{\cor{\{1, 1, \rnode{d4}{3}, \rnode{f4}{2}, 2, 3\}} \ncbar[angle=-90,arm=0.1cm,nodesep=0.05cm,linewidth=1pt,linecolor=gray]{d4}{f4}} & \comp{\cor{2}} & \comp{\cor{3}} & \comp{\cor{4}} \\
			\hline
			Etape 5               & \comp{\cor{\{1, 1, \underline{2}, 2, 3, 3\}}}                                                                                             & \comp{\cor{2}} & \comp{\cor{2}} & \comp{\cor{3}} \\
			\hline
			Etape 5               & \comp{\cor{\{1, 1, 2, 2, 3, 3\}}}                                                                                                         & \comp{\cor{2}} & \comp{\cor{1}} & \comp{\cor{3}} \\
			\hline
		\end{tabular}\smallskip}
	\subQuestion{Prouver la terminaison de cet algorithme.}
	\tcor{Montrons que {\tt i2-i1} est un variant :
		\begin{itemize}
			\item[(H1)] {\tt i2-i1} est entier comme différence de deux entiers
			\item[(H2)] {\tt i2-i1} est positif avant d'entrer dans la boucle et reste positif par condition d'entrée dans la boucle {\tt while}
			\item[(H3)] {\tt i2-i1} décroit à chaque tour de boucle car soit {\tt i1} augmente (si {\tt tab[i1]==1}) soit {\tt i2} diminue (si {\tt tab[i1]==2} ou {\tt tab[i1==3]}).
		\end{itemize}
		Donc {\tt i2-i1} est bien un variant et donc la fonction termine.
	}
	\subQuestion{Prouver la correction de cet algorithme.\\
		\aide \; \textit{Indication : on pourra noter $n$ la taille du tableau et  :
			\begin{itemize}
				\item $P_1$ la tranche du tableau {\tt tab} comprise entre les indices 0 et {\tt i1} (exclu)
				\item $P_2$ la portion du tableau {\tt tab} comprise entre les indices {\tt i2} et {\tt i3} (exclu)
				\item $P_3$ la portion du tableau {\tt tab} comprise entre les indices {\tt i3} et $n-1$ (exclu)
			\end{itemize}
			Et prouver l'invariant suivant : \og{} $P_1$ ne contient que des 1, $P_2$ que des 2 et $P_3$ que des 3 \fg{}.
		}}
	\tcor{Avec les notations proposées pour $P_1$, P$_2$ et $P_3$, montrons  l'invariant = \og{}$P_1$ ne contient que des 1, $P_2$ que des 2 et $P_3$ que des 3 \fg{}
		\begin{itemize}
			\item Initialisation: avant d'entrer dans la boucle, $i_1=0$, $i_2=n-1$, $i_3=n-1$ donc $P_1$, $P_2$ et $P_3$ sont vides et donc $I$ est vraie.
			\item Conservation : On suppose $I$ vérifié et on montre qu'il reste vraie au tour de boucle suivant, on distingue alors 3 cas suivant la première valeur non encore triée c'est à dire {\tt tab[i1]} :
			      \begin{itemize}
				      \item Si {\tt tab[i1]} vaut 1, alors {\tt i1} augmente de 1 donc un 1 est ajouté à $P_1$ qui puisque $I$ est supposé vérifié en entrant dans la boucle ne contenait que des 1. Ni $P_2$ ni $P_3$ ne sont modifiés et donc dans ce cas $I$ rest vrai.
				      \item Si {\tt tab[i1]} vaut 2, alors ce 2 est placé à l'indice {\tt i2} qui est décrémenté. Cela revient donc à ajouter un 2 à $P2$ qui par hypothèse  ne contenait que des 2. Ni $P_1$, ni $P_3$ ne sont modifiés et donc $I$ reste vrai.
				      \item Sinon {\tt tab[i1]} vaut nécessairement 3, alors ce 3 est placé à l'indice {\tt i3}, La partie $P_2$ est décalée d'un rang à gauche (elle garde la même longueur), $i_2$ et $i_3$ sont décrémentés. Un 3 étant ajouté à $P_3$ qui par hypothèse ne contenait que des 3, l'invariant est préservé.
			      \end{itemize}
			      On en conclut la fonction est totalement correcte (elle termine et elle est correcte)
		\end{itemize}
	}
	\subQuestion{Donner, en la justifiant brièvement, la complexité de l'algorithme du drapeau hollandais.}
	\tcor{En notant $n$ la taille du tableau, la boucle {\tt while} s'exécute $n$ fois et comme elle ne contient que des opérations élémentaires, la complexité de l'algorithme est un $O(n)$.}
	\Question{Comparaison avec le tri par insertion}
	\subQuestion{Rappeler  la complexité de l'algorithme du tri par insertion (on ne demande pas de justification).}
	\tcor{Le tri par insertion a une complexité quadratique.}
	\subQuestion{On a mesuré qu'en utilisant l'algorithme du tri par insertion un ordinateur trie une liste de dix million d'éléments en 5 secondes. Quel est le temps prévisible approximatif pour trier une liste contenant un milliard d'éléments ?}
	\tcor{Le nombre d'élements est multiplié par 100 et l'algorithme ayant une complexité quadratique, le temps prévisible sera multiplé par $100^2 = \numprint{10000}$. Donc l'exécution prendra environ \numprint{50000} secondes (environ 14 heures). }
	\subQuestion{On a mesuré qu'en utilisant l'algorithme du drapeau hollandais un ordinateur trie une liste de dix million d'éléments en 0.2 secondes. Quel est le temps prévisible approximatif pour un trier une liste contenant un milliard d'éléments ?}
	\tcor{Le nombre d'élements est multiplié par 100 et l'algorithme ayant une complexité linéaire, le temps prévisible sera multiplé lui aussi par 100. Donc l'exécution prendra environ \numprint{20} secondes.}


\end{Exercise}

\begin{Exercise}[title = {Manipulation de listes en OCaml}]\\
	On considère la fonction {\tt est\_triee} suivante :
	\inputpartOCaml{lst.ml}{}{}{1}{6}
	\Question{Proposer au moins un test permettant de montrer que cette fonction ne répond pas à sa spécification donnée en commentaire dans le code.}
	\tcor{La fonction teste si les deux premiers éléments sont dans l'ordre croisant puisse passe au reste de la liste \textit{à partir du 3e élément}, on ne vérifie donc pas que le 2e et le 3e élement sont bien dans l'ordre croissant. Le test suivant permettrait de faire apparaitre le problème : {\tt est\_triee [3; 4; 1; 2]} }
	\Question{Corriger le code de cette fonction afin qu'elle soit conforme à sa spécification (on pourra simplement indiquer le numéro de la ligne à modifier et donner son nouveau contenu).}
	\tcor{On doit relancer la récursivité sur la suite de la liste à partir du 2e élément dont sur {\tt i::t}, ce qui donne :}
	\ifcorrige
	\corpartOCaml{lst.ml}{}{}{8}{13}
	\fi
	\Question{Ecrire une fonction {\tt fusion: int list -> int list -> int list} qui prend en argument deux listes \textit{supposées triées} et renvoie la fusion de ces deux listes (avec répétition éventuelle des éléments). Par exemples :
		\begin{itemize}
			\item {\tt fusion [4; 8; 9] [0; 2; 3; 10]} renvoie {\tt [0; 2; 3; 4; 8; 9; 10]}
			\item {\tt fusion [4; 4; 5; 7] [4; 6]} renvoie {\tt [4; 4; 4; 5; 6; 7]}
			\item {\tt fusion [3] [1; 2; 2; 4; 6]} renvoie {\tt [1; 2; 2; 3; 4; 6]}
		\end{itemize}
	}
	\ifcorrige
	\corpartOCaml{lst.ml}{}{}{15}{19}
	\fi
	\Question{Prouver que la fonction {\tt fusion} termine.}
	\tcor{On note $n_1$ la longueur et {\tt l1} et $n_2$ celle de {\tt l2}, $n_1+n_2$ est bien un entier positif comme somme de deux entiers positifs (des longueurs de listes). A chaque appel récursif, $n_1$ décrémente $n2$ ne change pas ou c'est l'inverse, dans les deux cas, $n_1 + n_2$ décroît strictement. Par conséquent $n_1+n_2$ est un entier positif qui décroît strictement à chaque appel récursif donc cette fonction termine.}
\end{Exercise}


\begin{Exercise}[title = {Liste chaînée circulaire}]\\
	Le langage utilisé dans cet exercice est le langage C. \smallskip\\
	On veut implémenter une structure de données de liste chainée circulaire avec un pointeur sur la queue telle que représentée ci-dessous :
	\begin{center}
		\begin{tabular}{ccllllc}
			                                           &                         &                         &                         &                         & \rnode{liste}{{\footnotesize queue}} & \\
			                                           &                         &                         &                         &                         &                                      & \\
			\rnode{head}{\raisebox{-2pt}{\phantom{Y}}} & \maillon{$e_0$}{v0}{p0} & \maillon{$e_1$}{v1}{p1} & \maillon{$e_2$}{v2}{p2} & \maillon{$e_3$}{v3}{p3} & \maillon{$e_4$}{v4}{p4}              & \\
		\end{tabular}\\
		\ncline[nodesepB=0.25cm,nodesepA=-0.25cm]{*->}{p0}{v1}
		\ncline[nodesepB=0.25cm,nodesepA=-0.25cm]{*->}{p1}{v2}
		\ncline[nodesepB=0.25cm,nodesepA=-0.25cm]{*->}{p2}{v3}
		\ncline[nodesepB=0.25cm,nodesepA=-0.25cm]{*->}{p3}{v4}
		\ncline[nodesepB=0.25cm,nodesepA=0.05cm]{*->}{liste}{v4}
		\ncbar[nodesepB=0.2cm,nodesepA=-0.2cm,angleA=-90,angleB=-90]{*->}{p4}{v0}
	\end{center} \bigskip
	La queue pointe toujours vers \textit{le dernier élément inséré} ainsi, après l'ajout d'un nouvel élément $e_5$, la structure de données ci-dessus devient :
	\begin{center}
		\begin{tabular}{ccllllcl}
			                                           &                         &                         &                         &                         &                         & \hspace{-0.6cm}\rnode{liste}{{\footnotesize queue}} & \\
			                                           &                         &                         &                         &                         &                         &                                                       \\
			\rnode{head}{\raisebox{-2pt}{\phantom{Y}}} & \maillon{$e_0$}{v0}{p0} & \maillon{$e_1$}{v1}{p1} & \maillon{$e_2$}{v2}{p2} & \maillon{$e_3$}{v3}{p3} & \maillon{$e_4$}{v4}{p4} & \maillon{$e_5$}{v5}{p5}                               \\
		\end{tabular}\\
		\ncline[nodesepB=0.25cm,nodesepA=-0.25cm]{*->}{p0}{v1}
		\ncline[nodesepB=0.25cm,nodesepA=-0.25cm]{*->}{p1}{v2}
		\ncline[nodesepB=0.25cm,nodesepA=-0.25cm]{*->}{p2}{v3}
		\ncline[nodesepB=0.25cm,nodesepA=-0.25cm]{*->}{p3}{v4}
		\ncline[nodesepB=0.25cm,nodesepA=-0.25cm]{*->}{p4}{v5}
		\ncline[nodesepB=0.25cm,nodesepA=0.05cm]{*->}{liste}{v5}
		\ncbar[nodesepB=0.2cm,nodesepA=-0.2cm,angleA=-90,angleB=-90]{*->}{p5}{v0}
	\end{center} \bigskip
	Lorsqu'on retire un élément de cette structure de données, on retire le maillon qui \textit{suit le pointeur de queue}. Par conséquent, si on retire un élément de la structure de donnée ci-dessus, c'est le maillon contenant $e_0$ qui est retiré et on obtient :
	\begin{center}
		\begin{tabular}{ccllllcl}
			                                           &  &                         &                         &                         &                         & \hspace{-0.6cm}\rnode{liste}{{\footnotesize queue}} & \\
			                                           &  &                         &                         &                         &                         &                                                       \\
			\rnode{head}{\raisebox{-2pt}{\phantom{Y}}} &  & \maillon{$e_1$}{v1}{p1} & \maillon{$e_2$}{v2}{p2} & \maillon{$e_3$}{v3}{p3} & \maillon{$e_4$}{v4}{p4} & \maillon{$e_5$}{v5}{p5}                               \\
		\end{tabular}\\
		\ncline[nodesepB=0.25cm,nodesepA=-0.25cm]{*->}{p1}{v2}
		\ncline[nodesepB=0.25cm,nodesepA=-0.25cm]{*->}{p2}{v3}
		\ncline[nodesepB=0.25cm,nodesepA=-0.25cm]{*->}{p3}{v4}
		\ncline[nodesepB=0.25cm,nodesepA=-0.25cm]{*->}{p4}{v5}
		\ncline[nodesepB=0.25cm,nodesepA=0.05cm]{*->}{liste}{v5}
		\ncbar[nodesepB=0.2cm,nodesepA=-0.2cm,angleA=-90,angleB=-90]{*->}{p5}{v1}
	\end{center} \bigskip
	Afin d'implémenter cette structure de données, on propose d'utiliser les types suivants
	\inputpartC{file.c}{}{}{5}{11}
	La liste chainée circulaire vide est alors représentée par le pointeur {\sc null}.
	\Question{En partant d'une liste chainée circulaire initialement vide, donner les étapes de son évolution après les opérations suivantes  : \begin{enumerate}
			\item {\tt ajouter 12}
			\item {\tt ajouter  6}
			\item {\tt ajouter  7}
			\item {\tt retirer }
			\item {\tt ajouter 42}
			\item {\tt retirer}
		\end{enumerate}
		On précisera la valeur des entiers renvoyés par la fonction {\tt retirer}.
	}
	\tcor{
		\begin{enumerate}
			\item {\tt ajouter 12} : {\tt \underline{12}}
			\item {\tt ajouter  6} : {\tt 12} $\rightarrow$ {\tt \underline{6}}
			\item {\tt ajouter  7} : {\tt 12} $\rightarrow$ {\tt 6} $ \rightarrow$ {\tt \underline{7}}
			\item {\tt retirer } : {\tt 6} $\rightarrow$ {\tt \underline{7}}, l'opération renvoie 12
			\item {\tt ajouter 42}: {\tt 6} $\rightarrow$ {\tt 7} $\rightarrow${\tt \underline{42}}
			\item {\tt retirer} : {\tt 7} $\rightarrow${\tt \underline{42}}, l'opération renvoie 6
		\end{enumerate}
	}
	\Question{Ecrire la fonction {\tt ajouter} de signature \mintinline{c}{void ajouter(liste_circulaire *lc, int v)} qui modifie la liste circulaire donnée en argument en lui ajoutant un nouveau maillon contenant la valeur {\tt v}.\\
	{\small \aide \;} Indication : on fera attention à traiter le cas particulier d'une liste circulaire initialement vide.
	}
	\ifcorrige
	\corpartC{file.c}{}{}{18}{32}
	\fi
	\Question{Ecrire la fonction {\tt retirer} de signature \mintinline{c}{int retirer(liste_circulaire *lc)} qui prend en argument une liste circulaire \textit{supposée non vide} et renvoie la valeur du maillon situé après le pointeur de queue en le retirant de la liste circulaire.}
	\ifcorrige
	\corpartC{file.c}{}{}{43}{57}
	\fi
	\Question{Quelle structure de données connue a-t-on implémenté ici ? Justifier et proposer des noms plus appropriés pour les fonctions {\tt ajouter} et {\tt retirer}.}
	\tcor{Les premiers éléments ajoutés à la structure sont aussi les premiers à être retirés, il s'agit donc d'une struture de données de type {\sc fifo}, c'est à  dire une file. L'opération ajouter correspond à enfiler et l'opération retirer à défiler.}
	\Question{Ecrire une fonction {\tt longueur} de signature \mintinline{c}{int longueur(liste_circulaire lc)} qui renvoie le nombre d'éléments d'une liste chaînée circulaire.}
	\ifcorrige
	\corpartC{file.c}{}{}{59}{76}
	\fi
	\Question{Donner, en les justifiant, les complexités des opérations {\tt retirer}, {\tt ajouter} et {\tt longueur}.}
	\tcor{Retirer et ajouter sont en $O(1)$ car on ne fait que des opérations élémentaires et longueur est en $O(n)$ où $n$ est la longueur de la liste car celle ci doit être parcourue en entier (on détecte un retour vers le pointeur initial).}
\end{Exercise}

\begin{Exercise}[title={Plus petit entier manquant}]\\
	Le langage utilisé dans cet exercice est le langage Ocaml. \smallskip\\
	Etant donné une liste d'entiers \textit{naturels}, on cherche à déterminer le plus petit entier manquant dans cette liste. Par exemples :
	\begin{itemize}
		\item si la liste est {\tt [6; 4; 2; 0; 1; 2; 5]} alors le plus petit entier manquant est 3
		\item si la liste est {\tt [0; 2; 1; 1; 3; 2; 1]} alors le plus petit entier manquant est 4
		\item si la liste est {\tt [1; 3; 7;]} alors le plus petit entier manquant est 0
		\item si la liste est vide alors le plus petit entier manquant est 0.
	\end{itemize}
	Dans toute la suite de l'exercice, on notera $N$ la longueur de la liste donnée en argument et $M$ le maximum de ses éléments.
	\Question{Dans cette partie, on propose de répondre au problème en testant successivement la présence de chaque entier dans la liste.}
	\subQuestion{Ecrire une fonction {\tt est\_dans 'a -> 'a list -> bool} qui prend en entrée un élément {\tt elt} et une liste {\tt lst} et renvoie {\tt true} lorsque {\tt elt} est dans {\tt lst} et {\tt false} sinon.}
	\ifcorrige
	\corpartOCaml{manquant.ml}{}{}{1}{6}
	\fi
	\subQuestion{Ecrire une fonction {\tt manquant: intlist -> int} qui renvoie le plus petit entier manquant dans la liste donnée en argument en testant successivement la présence des entiers $0, 1, 2, \dots$ et en s'arrêtant dès que l'un deux n'est pas trouvé.\\
	{\small \aide \;} Indication : on pourra utiliser une fonction auxiliaire récursive.}
	\subQuestion{Donner en la justifiant la complexité de la fonction {\tt manquant}.}
	\tcor{La complexité de {\tt est\_dans} est un $O(N)$ et on appelle cette fonction au plus $N$ fois, donc la fonction {\tt manquant} a une complexité quadratique $\mathcal{O}(N^2)$}
	\Question{Dans cette partie, on propose de trier au préalable la liste d'entiers. On suppose \textit{déjà écrite} une fonction {\tt trie int list -> int list} qui prend en argument une liste d'entiers et renvoie cette liste triée.}
	\subQuestion{Ecrire une fonction {\tt manquant\_trie : int list -> int} qui prend en argument une liste triée d'entiers et renvoie le plus petit entier manquant dans cette liste.}
	\ifcorrige
	\corpartOCaml{manquant.ml}{}{}{19}{26}
	\fi
	\subQuestion{Citer au moins un algorithme permettant de trier une liste avec une complexité en $\mathcal{O}(N \log N)$ où $N$ est la longueur de la liste.}
	\tcor{Le tri fusion a une complexité linéarithmique (on pouvait aussi citer le tri rapide).}
	\subQuestion{On suppose que la fonction {\tt trie} a une complexité linéarithmique, donner alors en la justifiant la complexité de la méthode consistant à trier au préalable la liste avec la fonction {\tt trie} puis à utiliser la fonction {\tt manquant\_trie}.}
	\tcor{Le tri a une complexité linéarithmique par hypothèse, on appelle ensuite la fonction {\tt manquant\_trie} qui a une complexité linéaire en la longueur $N$ de la liste (la fonction {\tt aux} effectue au plus $N$ appels récursifs et ne contient que des opérations élémentaires). Comme $\mathcal{O}(N \log N) + \mathcal{O}{N}$ est un $\mathcal{O}(N \log N)$ la fonction {\tt manquant\_trie} a une complexité linéarithmique.}
	\Question{Proposer un algorithme permettant de résoudre le problème de la recherche du plus petit entier manquant en complexité $\mathcal{O}(M)$ où $M$ est le maximum des éléments de la liste. On \textit{ne demande pas de coder} cet algorithme, mais simplement d'en décrire le fonctionnement (ou de l'écrire en pseudo-langage) et d'en justifier la complexité.\\
	{\small \aide \;} Indication : on pourra utiliser un tableau de booléens de taille $M$.}
	\tcor{On propose l'algorithme suivant :
	\begin{enumerate}
	\item on détermine le maximum $M$ des éléments de la liste, pour cela un parcours des éléments de la liste suffit, la complexité est donc en $\mathcal{O}(N)$ où $N$ est la longueur de la liste.
	\item on crée un tableau de {\tt tab} de $M+1$ booléens qu'on initialise à {\tt Faux}, il faut pour cela parcourir la totalité du tableau, cette étape est donc en $\mathcal{O}(M)$
	\item on parcourt de nouveau la liste, et pour chaque élément {\tt x} trouvé on affecte {\tt tab[x]} à {\tt Vrai}. Cette nouvelle étape est en $\mathcal{O}(N)$.
	\item on parcourt le tableau par indice croissant et on renvoie l'indice du premier {\tt Faux} rencontré, cette étape est donc en $\mathcal{O}(M)$.
	\end{enumerate}
	}
\end{Exercise}

\end{document}
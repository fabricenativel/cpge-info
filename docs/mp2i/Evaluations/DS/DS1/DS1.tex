\PassOptionsToPackage{dvipsnames,table}{xcolor}
\documentclass[11pt,a4paper]{article}

\usepackage{DS}

\begin{document}
\input{\detokenize{/home/fenarius/Travail/Cours/cpge-info/latex/Macros.tex}}
\ModeExercice
\DS{MP2I}{1}{Septembre 2023}

\alertbox{\danger}{Consignes}{
	\begin{itemize}
		\item[\textbullet] Les programmes demandés doivent être écrits en C et on suppose que les librairies standards usuelles ({\tt <stdio.h>}, {\tt <stdlib.h>}, {\tt <stdbool.h>}) sont déjà importées.
		\item[\textbullet] On pourra toujours librement utiliser une fonction demandée à une question précédente même si cette question n'a pas été traitée.
		\item[\textbullet] Veillez à présenter vos idées et vos réponses partielles même si vous ne trouvez pas la solution complète à une question.
		\item[\textbullet] La clarté et la lisibilité de la rédaction et des programmes sont des éléments de notation.
	\end{itemize}
}
 
\begin{Exercise}[title = {chercher les erreurs}]
	\Question{Les fonctions ci-dessous contiennent une ou plusieurs erreurs et/ou ne respectent pas leurs spécifications (données en commentaire). Dans chaque cas, expliquer les erreurs commises et les corriger.}

	\subQuestion{Fonction {\tt harmonique}
		\inputpartC{\FPATH/Evaluations/DS/DS1/DS1_ex1.c}{}{}{5}{13}
	}

	\subQuestion{Fonction {\tt tous\_egaux}
		\inputpartC{\FPATH/Evaluations/DS/DS1/DS1_ex1.c}{}{}{15}{29}
	}

	\subQuestion{Fonction {\tt cree\_tab\_entiers}
		\inputpartC{\FPATH/Evaluations/DS/DS1/DS1_ex1.c}{}{}{31}{38}
	}

	\subQuestion{Fonction {\tt syracuse}
		\inputpartC{\FPATH/Evaluations/DS/DS1/DS1_ex1.c}{}{}{40}{49}
	}

	\Question{Les programmes suivants, produisent une erreur à l'exécution. Expliquer l'origine du problème et apporter les corrections nécessaires}
	\subQuestion{Affichage d'une adresse
		\inputpartC{\FPATH/Evaluations/DS/DS1/DS1_ex1.c}{}{}{51}{55}
	}

	\subQuestion{Calcul de la somme de deux entiers
		\inputpartC{\FPATH/Evaluations/DS/DS1/DS1_ex1.c}{}{}{57}{64}
	}
\end{Exercise}

\begin{Exercise}[title={Pointeurs}]
	\Question{Compléter le tableau suivant, qui donne l'état des variables au fur et à mesure des instructions données dans la première colonne (on a indiqué par \faTimes{} une variable non encore déclaré.)
		\begin{center}
			\def\arraystretch{1.2}
			\setlength\tabcolsep{0.5cm}
			\begin{tabular}{|l|c|c|c|c|}
                \hline
				        instructions            & a  & b  & p & q       \\
				\hline
				{\tt int a = 14;}      & 14 & \faTimes & \faTimes & \faTimes      \\
				{\tt int b = 42;}      & 14  & \dots  & \dots  &  \dots           \\
				{\tt int *p = \&a;}      & 14   &  \dots  &   \&a     & \dots     \\
                {\tt int *q = \&b;}      & 14   &  \dots  &   \&a     & \dots     \\
				{\tt *p = *p + *q ;} &  \dots  & \dots    &  \dots & \dots        \\
				{\tt *q = *p - *q ;} &  \dots  & \dots    &  \dots & \dots        \\
                {\tt *p = *p - *q ;} &  \dots  & \dots    &  \dots & \dots        \\
                \hline
			\end{tabular}
		\end{center}}
    \Question{Ecrire une fonction en C qui prend en argument deux pointeurs vers des entiers, ne renvoie rien et échange les valeurs de ces deux entiers \textit{sans utiliser de variable temporaire}.}
    \Question{Compléter le programme suivant en écrivant l'appel à la fonction {\tt echange} afin d'échanger les valeurs des entiers {\tt n} et {\tt m} 
    \inputpartC{\FPATH/Evaluations/DS/DS1/echange.c}{}{}{12}{13}}
\end{Exercise}

\begin{Exercise}[title = {puissance}]
	\Question{Ecrire une fonction {\tt valeur\_absolue} qui prend en argument un entier $n$ et renvoie sa valeur absolue $|n|$. On rappelle que :
		$|n| = \left\{ \begin{array}{rl} -n & \mathrm{\ si\ } n<0 \\ n & \mathrm{\ sinon} \end{array}\right.$}
	\Question{Ecrire une fonction {\tt puissance} qui prend en argument un flottant (type {\tt double}) $a$ et un entier $n$ et renvoie
		$a^n$. On rappelle que pour $a \in \R^*$, $n \in \Z$ : \\
		$\left\{ \begin{array}{ll}
				a^n = \underbrace{a\times \dots \times a}_{n \mathrm{\ facteurs}} & \mathrm{\ si\ } n>0, \\
				a^0 = 1,                                                          &                      \\
				a^n = \dfrac{1}{a^{-n}}                                           & \mathrm{\ si\ } n<0. \\
			\end{array}\right.$ \\

		D'autre part $0^0=1$, $0^n=0$ si $n>0$ et les puissances négatives de zéro ne sont pas définies. On vérifiera la précondition $n>0$ lorsque $a=0$ à l'aide d'une instruction {\tt assert}.}
	\Question{Tracer le graphe de flot de contrôle de cette fonction.}
	\Question{Proposer un jeu de test permettant de couvrir tous les arcs.}
\end{Exercise}


\begin{Exercise}[title={Annagrammes}]

	Deux mots \textit{de même longueur} sont anagrammes l'un de l'autre lorsque l'un est formé en réarrangeant les lettres de l'autre. Par exemple :
	\begin{itemize}
		\item \textit{niche} et \textit{chien} sont des anagrammes.
		\item \textit{epele} et \textit{pelle}, ne sont pas des anagrammes, en effet bien qu'ils soient formés avec les mêmes lettres, la lettre \textit{l} ne figure qu'à un seul exemplaire dans \textit{epele} et il en faut deux pour écrire \textit{pelle}.
	\end{itemize}
	Le but de l'exercice est d'écrire une fonction en C qui renvoie {\tt true} si les deux chaines données en argument sont des anagrammes et {\tt false} sinon. On suppose  que les chaines sont constitués uniquement de lettres majuscules.

	\Question{Ecrire une fonction {\tt tableau\_egaux} qui prend en argument deux tableaux ainsi que leur taille et renvoie {\tt true} lorsque ces deux tableaux ont un contenu et une longueur identique et {\tt false} sinon.}

	\Question{Ecrire une fonction {\tt nb\_lettres} qui prend en argument une chaine de caractères {\tt chaine} et renvoie un tableau d'entiers {\tt tab} de longueur 26 de sorte que {\tt tab[i-1]} contienne le nombre de fois où la ième lettre de l'alphabet apparait dans {\tt chaine}. Par exemple {tab[0]}, doit contenir le nombre de fois où la lettre \textit{a} apparait dans le mot.}

	\Question{En supposant les deux fonctions précédentes correctement écrites, on propose le code suivant pour la fonction {\tt anagrammes} qui teste si les deux chaines données en argument sont des anagrammes :
		\inputpartC{\FPATH/Evaluations/DS/DS1/anagrammes.c}{}{}{35}{38}
		Cette fonction renvoie le résultat attendu mais pose un problème, lequel ? Expliquer et proposer une correction.
	}


\end{Exercise}


\begin{Exercise}[title = {tri à bulles}] \\
	Le tri à bulles est un algorithme de tri qui parcourt le tableau à l'aide d'un indice $i$ de la fin vers le début. Pour chacun de ces indices $i$, on parcourt la partie du tableau allant de l'indice 0 à l'indice $i-1$ et si les deux éléments consécutifs situés aux indices $j$ et $j+1$ ne sont pas dans l'ordre croissant, on les échange. Par exemple sur le tableau {\tt \{7, 2, 9, 5, 3\}}
	\begin{itemize}
		\item[\textbullet] $i = 4$ (on rappelle que $i$ parcourt de la fin vers le début)
			\begin{itemize}
				\item $j=0$, donc on échange {\tt tab[0]} et {\tt tab[1]} car ils ne sont pas dans l'ordre croisant  : {\tt \{2, 7, 9, 5, 3\}}
				\item $j=1$, pas d'échange (7 et 9 sont en ordre croissant)
				\item $j=2$, échange car 9 et 5 ne sont pas dans l'ordre croissant : {\tt \{2, 7, 5, 9, 3\}}
				\item $j=3$, échange et on obtient {\tt \{2, 7, 5, 3, 9\}}
			\end{itemize}
		\item[\textbullet] $i=3$, cette fois on parcourt avec $j=0$ jusqu'à $2$, en effet à l'étape précédente le plus grand élément du tableau se retrouve forcement en dernière position. On obtient en fin de parcours : {\tt \{2, 5, 3, 7, 9\}}
		\item[\textbullet] $i=2$, à la fin de cette itération on obtient {\tt \{2, 3, 5, 7, 9\}}
	\end{itemize}

	\Question{Faire fonctionner cet algorithme à la main sur le tableau {\tt \{11, 2, 5, 13, 8, 4\}} et donner l'état du tableau à la fin de chaque itération de l'indice $j$ pour $j$ variant de $0$ à $i$ en recopiant et complétant le tableau suivant : \\
		\begin{tabular}{|l|p{12cm}|}
			\hline
			$i$ & valeurs contenu dans le tableau à la fin de l'itération d'indice $i$. \\
			\hline
			5   & \dotfill                                                              \\
			\hline
			4   & \dotfill                                                              \\
			\hline
			3   & \dotfill                                                              \\
			\hline
			2   & \dotfill                                                              \\
			\hline
			1   & \dotfill                                                              \\
			\hline
		\end{tabular}
	}
	\Question{Donner un exemple de tableau de longueur 5, \textit{non trié initialement} qui sera entièrement trié après le premier tour de boucle de l'indice $j$ (c'est à dire pour $i=4$).}
	\Question{Donner un exemple de tableau qui  ne sera trié qu'à la fin de toutes les itérations de l'indice~$i$.}
	\Question{On veut maintenant écrire cet algorithme en C, en effectuant le tri dans une copie du tableau. On suppose déjà écrite la fonction {\tt echange} qui prend en argument un tableau et deux indices ne renvoie rien et échange les éléments situés à ces deux indices.}
	\subQuestion{Ecrire un fonction {\tt copie\_tab} qui prend en argument un tableau ainsi que sa taille et renvoie un pointeur vers une copie de ce tableau}
	\subQuestion{Ecrire une fonction {\tt tri\_bulles} qui prend en argument un tableau {\tt tab} ainsi que sa taille, ne modifie pas ce tableau et renvoie un pointeur vers un tableau contenant les éléments du tableau {\tt tab} triés dans l'ordre croissant.}
	\subQuestion{Afin de tester cette fonction on a écrit le programme principal suivant :
		\inputpartC{\FPATH/Evaluations/DS/DS1/tri_bulles.c}{}{}{53}{60}
		Quelle instruction est manquante dans ce programme ? Quelle option de compilation signalerait le problème lors de l'exécution ?
	}
\end{Exercise}

\end{document}
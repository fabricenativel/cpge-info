\PassOptionsToPackage{dvipsnames,table}{xcolor}
\documentclass[11pt,a4paper]{article}

\usepackage{DS}

\begin{document}
\newcommand{\ModeExercice}{
% Traduction des noms pour le package exercise
\renewcommand{\ExerciseName}{Exercice}
\renewcommand{\thesubQuestion}{\theQuestion.\alph{subQuestion}}
\renewcommand{\AnswerName}{Réponses à l'exercise }
}
\newcommand{\Fiche}[2]{\lhead{\textbf{{\sc #1}}}
\rhead{Niveau: \textbf{#2}}
\cfoot{}}
\definecolor{cfond}{gray}{0.4}
\renewcommand{\thealgocf}{}

\newcommand{\ModeActivite}{
% Traduction des noms pour le package exercise
\renewcommand{\ExerciseName}{Activité}
}
% Réglages de la mise en forme des exercices 
\renewcommand{\ExerciseHeaderTitle}{\ExerciseTitle}
\renewcommand{\ExerciseHeaderOrigin}{\ExerciseOrigin}
% Un : sépare le numéro de l'exerice de son titre ... si le titre existe. On utilise Origine pour placer les pictogrammes en fin de ligne
\renewcommand{\ExerciseHeader}{\ding{113} \textbf{\sffamily{\ExerciseName \ \ExerciseHeaderNB}} \ifthenelse{\equal{\ExerciseTitle}{\empty}}{}{:} \textit{\ExerciseHeaderTitle} \hfill \ExerciseHeaderOrigin}
\renewcommand{\ExePartHeader}{\quad {\footnotesize \ding{110}} \textbf{Partie \textbf{\ExePartHeaderNB}} : \ExePartName}


% Mode Concours
\newcommand{\ModeConcours}{
   \newcounter{qconcours}
   \setcounter{qconcours}{1}
   \renewcommand{\ExerciseName}{Exercice}
   \renewcommand{\ExerciseHeaderTitle}{\ExerciseTitle}
\renewcommand{\ExerciseHeaderOrigin}{\ExerciseOrigin}
\renewcommand{\ExerciseHeader}{\ding{113} \textbf{\sffamily{\ExerciseName \ \ExerciseHeaderNB}} \ifthenelse{\equal{\ExerciseTitle}{\empty}}{}{:} \textit{\ExerciseHeaderTitle} \hfill \ExerciseHeaderOrigin}
\renewcommand{\QuestionNB}{\textbf{Q\arabic{qconcours}--}\ \addtocounter{qconcours}{1}}}

\newcommand{\noeud}[1]{\Tr{\fbox{\tt #1}}}
\newcommand{\FPATH}{/home/fenarius/Travail/Cours/cpge-info/docs/mp2i/}
\newcommand{\spath}[2]{\FPATH Evaluations/#1/#1#2}
\definecolor{codebg}{gray}{0.90}
\newcommand{\inputpartOCaml}[5]{\begin{mdframed}[backgroundcolor=codebg] \inputminted[breaklines=true,fontsize=#3,linenos=true,highlightcolor=fluo,tabsize=2,highlightlines={#2},firstline=#4,lastline=#5,firstnumber=1]{OCaml}{#1} \end{mdframed}}
\newcommand{\inputpartPython}[5]{\begin{mdframed}[backgroundcolor=codebg] \inputminted[breaklines=true,fontsize=#3,linenos=true,highlightcolor=fluo,tabsize=2,highlightlines={#2},firstline=#4,lastline=#5,firstnumber=1]{python}{#1} \end{mdframed}}
\newcommand{\inputpartC}[5]{\begin{mdframed}[backgroundcolor=codebg] \inputminted[breaklines=true,fontsize=#3,linenos=true,highlightcolor=fluo,tabsize=2,highlightlines={#2},firstline=#4,lastline=#5,firstnumber=1]{c}{#1} \end{mdframed}}
\newcommand{\inputC}[2]{\begin{mdframed}[backgroundcolor=codebg] \inputminted[breaklines=true,fontsize=#2,linenos=true,highlightcolor=fluo,tabsize=2]{c}{#1} \end{mdframed}}
\newminted[langageC]{c}{linenos=true,escapeinside=``,highlightcolor=fluo,tabsize=2}
\newminted[python]{python}{linenos=true,escapeinside=``,highlightcolor=fluo,tabsize=4}
\BeforeBeginEnvironment{minted}{\begin{mdframed}[backgroundcolor=codebg,skipabove=0cm]}
   \AfterEndEnvironment{minted}{\end{mdframed}}

% Font light medium et bold pour tt :
\newcommand{\ttl}[1]{\ttfamily \fontseries{l}\selectfont #1}
\newcommand{\ttm}[1]{\ttfamily \fontseries{m}\selectfont #1}
\newcommand{\ttb}[1]{\ttfamily \fontseries{b}\selectfont #1}

%QCM de NSI \QNSI{Question}{R1}{R2}{R3}{R4}
\newcommand{\QNSI}[5]{
#1
\begin{enumerate}[label=\alph{enumi})]
\item #2
\item #3
\item #4
\item #5
\end{enumerate}
}



\definecolor{grispale}{gray}{0.95}
\newcommand{\htmlmode}{\lstset{language=html,numbers=left, tabsize=2, frame=single, breaklines=true, keywordstyle=\ttfamily, basicstyle=\small,
   numberstyle=\tiny\ttfamily, framexleftmargin=0mm, backgroundcolor=\color{grispale}, xleftmargin=12mm,showstringspaces=false}}
\newcommand{\pythonmode}{\lstset{language=python,numbers=left, tabsize=4, frame=single, breaklines=true, keywordstyle=\ttfamily, basicstyle=\small,
   numberstyle=\tiny\ttfamily, framexleftmargin=0mm, backgroundcolor=\color{grispale}, xleftmargin=12mm, showstringspaces=false}}
\newcommand{\bashmode}{\lstset{language=bash,numbers=left, tabsize=2, frame=single, breaklines=true, basicstyle=\ttfamily,
   numberstyle=\tiny\ttfamily, framexleftmargin=0mm, backgroundcolor=\color{grispale}, xleftmargin=12mm, showstringspaces=false}}
\newcommand{\exomode}{\lstset{language=python,numbers=left, tabsize=2, frame=single, breaklines=true, basicstyle=\ttfamily,
   numberstyle=\tiny\ttfamily, framexleftmargin=13mm, xleftmargin=12mm, basicstyle=\small, showstringspaces=false}}
   
   
  \lstset{%
        inputencoding=utf8,
        extendedchars=true,
        literate=%
        {é}{{\'{e}}}1
        {è}{{\`{e}}}1
        {ê}{{\^{e}}}1
        {ë}{{\¨{e}}}1
        {É}{{\'{E}}}1
        {Ê}{{\^{E}}}1
        {û}{{\^{u}}}1
        {ù}{{\`{u}}}1
        {ú}{{\'{u}}}1
        {â}{{\^{a}}}1
        {à}{{\`{a}}}1
        {á}{{\'{a}}}1
        {ã}{{\~{a}}}1
        {Á}{{\'{A}}}1
        {Â}{{\^{A}}}1
        {Ã}{{\~{A}}}1
        {ç}{{\c{c}}}1
        {Ç}{{\c{C}}}1
        {õ}{{\~{o}}}1
        {ó}{{\'{o}}}1
        {ô}{{\^{o}}}1
        {Õ}{{\~{O}}}1
        {Ó}{{\'{O}}}1
        {Ô}{{\^{O}}}1
        {î}{{\^{i}}}1
        {Î}{{\^{I}}}1
        {í}{{\'{i}}}1
        {Í}{{\~{Í}}}1
}

%tei pour placer les images
%tei{nom de l’image}{échelle de l’image}{sens}{texte a positionner}
%sens ="1" (droite) ou "2" (gauche)
\newlength{\ltxt}
\newcommand{\tei}[4]{
\setlength{\ltxt}{\linewidth}
\setbox0=\hbox{\includegraphics[scale=#2]{#1}}
\addtolength{\ltxt}{-\wd0}
\addtolength{\ltxt}{-10pt}
\ifthenelse{\equal{#3}{1}}{
\begin{minipage}{\wd0}
\includegraphics[scale=#2]{#1}
\end{minipage}
\hfill
\begin{minipage}{\ltxt}
#4
\end{minipage}
}{
\begin{minipage}{\ltxt}
#4
\end{minipage}
\hfill
\begin{minipage}{\wd0}
\includegraphics[scale=#2]{#1}
\end{minipage}
}
}

%Juxtaposition d'une image pspciture et de texte 
%#1: = code pstricks de l'image
%#2: largeur de l'image
%#3: hauteur de l'image
%#4: Texte à écrire
\newcommand{\ptp}[4]{
\setlength{\ltxt}{\linewidth}
\addtolength{\ltxt}{-#2 cm}
\addtolength{\ltxt}{-0.1 cm}
\begin{minipage}[b][#3 cm][t]{\ltxt}
#4
\end{minipage}\hfill
\begin{minipage}[b][#3 cm][c]{#2 cm}
#1
\end{minipage}\par
}



%Macros pour les graphiques
\psset{linewidth=0.5\pslinewidth,PointSymbol=x}
\setlength{\fboxrule}{0.5pt}
\newcounter{tempangle}

%Marque la longueur du segment d'extrémité  #1 et  #2 avec la valeur #3, #4 est la distance par rapport au segment (en %age de la valeur de celui ci) et #5 l'orientation du marquage : +90 ou -90
\newcommand{\afflong}[5]{
\pstRotation[RotAngle=#4,PointSymbol=none,PointName=none]{#1}{#2}[X] 
\pstHomO[PointSymbol=none,PointName=none,HomCoef=#5]{#1}{X}[Y]
\pstTranslation[PointSymbol=none,PointName=none]{#1}{#2}{Y}[Z]
 \ncline{|<->|,linewidth=0.25\pslinewidth}{Y}{Z} \ncput*[nrot=:U]{\footnotesize{#3}}
}
\newcommand{\afflongb}[3]{
\ncline{|<->|,linewidth=0}{#1}{#2} \naput*[nrot=:U]{\footnotesize{#3}}
}

%Construis le point #4 situé à #2 cm du point #1 avant un angle #3 par rapport à l'horizontale. #5 = liste de paramètre
\newcommand{\lsegment}[5]{\pstGeonode[PointSymbol=none,PointName=none](0,0){O'}(#2,0){I'} \pstTranslation[PointSymbol=none,PointName=none]{O'}{I'}{#1}[J'] \pstRotation[RotAngle=#3,PointSymbol=x,#5]{#1}{J'}[#4]}
\newcommand{\tsegment}[5]{\pstGeonode[PointSymbol=none,PointName=none](0,0){O'}(#2,0){I'} \pstTranslation[PointSymbol=none,PointName=none]{O'}{I'}{#1}[J'] \pstRotation[RotAngle=#3,PointSymbol=x,#5]{#1}{J'}[#4] \pstLineAB{#4}{#1}}

%Construis le point #4 situé à #3 cm du point #1 et faisant un angle de  90° avec la droite (#1,#2) #5 = liste de paramètre
\newcommand{\psegment}[5]{
\pstGeonode[PointSymbol=none,PointName=none](0,0){O'}(#3,0){I'}
 \pstTranslation[PointSymbol=none,PointName=none]{O'}{I'}{#1}[J']
 \pstInterLC[PointSymbol=none,PointName=none]{#1}{#2}{#1}{J'}{M1}{M2} \pstRotation[RotAngle=-90,PointSymbol=x,#5]{#1}{M1}[#4]
  }
  
%Construis le point #4 situé à #3 cm du point #1 et faisant un angle de  #5° avec la droite (#1,#2) #6 = liste de paramètre
\newcommand{\mlogo}[6]{
\pstGeonode[PointSymbol=none,PointName=none](0,0){O'}(#3,0){I'}
 \pstTranslation[PointSymbol=none,PointName=none]{O'}{I'}{#1}[J']
 \pstInterLC[PointSymbol=none,PointName=none]{#1}{#2}{#1}{J'}{M1}{M2} \pstRotation[RotAngle=#5,PointSymbol=x,#6]{#1}{M2}[#4]
  }

% Construis un triangle avec #1=liste des 3 sommets séparés par des virgules, #2=liste des 3 longueurs séparés par des virgules, #3 et #4 : paramètre d'affichage des 2e et 3 points et #5 : inclinaison par rapport à l'horizontale
%autre macro identique mais sans tracer les segments joignant les sommets
\noexpandarg
\newcommand{\Triangleccc}[5]{
\StrBefore{#1}{,}[\pointA]
\StrBetween[1,2]{#1}{,}{,}[\pointB]
\StrBehind[2]{#1}{,}[\pointC]
\StrBefore{#2}{,}[\coteA]
\StrBetween[1,2]{#2}{,}{,}[\coteB]
\StrBehind[2]{#2}{,}[\coteC]
\tsegment{\pointA}{\coteA}{#5}{\pointB}{#3} 
\lsegment{\pointA}{\coteB}{0}{Z1}{PointSymbol=none, PointName=none}
\lsegment{\pointB}{\coteC}{0}{Z2}{PointSymbol=none, PointName=none}
\pstInterCC{\pointA}{Z1}{\pointB}{Z2}{\pointC}{Z3} 
\pstLineAB{\pointA}{\pointC} \pstLineAB{\pointB}{\pointC}
\pstSymO[PointName=\pointC,#4]{C}{C}[C]
}
\noexpandarg
\newcommand{\TrianglecccP}[5]{
\StrBefore{#1}{,}[\pointA]
\StrBetween[1,2]{#1}{,}{,}[\pointB]
\StrBehind[2]{#1}{,}[\pointC]
\StrBefore{#2}{,}[\coteA]
\StrBetween[1,2]{#2}{,}{,}[\coteB]
\StrBehind[2]{#2}{,}[\coteC]
\tsegment{\pointA}{\coteA}{#5}{\pointB}{#3} 
\lsegment{\pointA}{\coteB}{0}{Z1}{PointSymbol=none, PointName=none}
\lsegment{\pointB}{\coteC}{0}{Z2}{PointSymbol=none, PointName=none}
\pstInterCC[PointNameB=none,PointSymbolB=none,#4]{\pointA}{Z1}{\pointB}{Z2}{\pointC}{Z1} 
}


% Construis un triangle avec #1=liste des 3 sommets séparés par des virgules, #2=liste formée de 2 longueurs et d'un angle séparés par des virgules, #3 et #4 : paramètre d'affichage des 2e et 3 points et #5 : inclinaison par rapport à l'horizontale
%autre macro identique mais sans tracer les segments joignant les sommets
\newcommand{\Trianglecca}[5]{
\StrBefore{#1}{,}[\pointA]
\StrBetween[1,2]{#1}{,}{,}[\pointB]
\StrBehind[2]{#1}{,}[\pointC]
\StrBefore{#2}{,}[\coteA]
\StrBetween[1,2]{#2}{,}{,}[\coteB]
\StrBehind[2]{#2}{,}[\angleA]
\tsegment{\pointA}{\coteA}{#5}{\pointB}{#3} 
\setcounter{tempangle}{#5}
\addtocounter{tempangle}{\angleA}
\tsegment{\pointA}{\coteB}{\thetempangle}{\pointC}{#4}
\pstLineAB{\pointB}{\pointC}
}
\newcommand{\TriangleccaP}[5]{
\StrBefore{#1}{,}[\pointA]
\StrBetween[1,2]{#1}{,}{,}[\pointB]
\StrBehind[2]{#1}{,}[\pointC]
\StrBefore{#2}{,}[\coteA]
\StrBetween[1,2]{#2}{,}{,}[\coteB]
\StrBehind[2]{#2}{,}[\angleA]
\lsegment{\pointA}{\coteA}{#5}{\pointB}{#3} 
\setcounter{tempangle}{#5}
\addtocounter{tempangle}{\angleA}
\lsegment{\pointA}{\coteB}{\thetempangle}{\pointC}{#4}
}

% Construis un triangle avec #1=liste des 3 sommets séparés par des virgules, #2=liste formée de 1 longueurs et de deux angle séparés par des virgules, #3 et #4 : paramètre d'affichage des 2e et 3 points et #5 : inclinaison par rapport à l'horizontale
%autre macro identique mais sans tracer les segments joignant les sommets
\newcommand{\Trianglecaa}[5]{
\StrBefore{#1}{,}[\pointA]
\StrBetween[1,2]{#1}{,}{,}[\pointB]
\StrBehind[2]{#1}{,}[\pointC]
\StrBefore{#2}{,}[\coteA]
\StrBetween[1,2]{#2}{,}{,}[\angleA]
\StrBehind[2]{#2}{,}[\angleB]
\tsegment{\pointA}{\coteA}{#5}{\pointB}{#3} 
\setcounter{tempangle}{#5}
\addtocounter{tempangle}{\angleA}
\lsegment{\pointA}{1}{\thetempangle}{Z1}{PointSymbol=none, PointName=none}
\setcounter{tempangle}{#5}
\addtocounter{tempangle}{180}
\addtocounter{tempangle}{-\angleB}
\lsegment{\pointB}{1}{\thetempangle}{Z2}{PointSymbol=none, PointName=none}
\pstInterLL[#4]{\pointA}{Z1}{\pointB}{Z2}{\pointC}
\pstLineAB{\pointA}{\pointC}
\pstLineAB{\pointB}{\pointC}
}
\newcommand{\TrianglecaaP}[5]{
\StrBefore{#1}{,}[\pointA]
\StrBetween[1,2]{#1}{,}{,}[\pointB]
\StrBehind[2]{#1}{,}[\pointC]
\StrBefore{#2}{,}[\coteA]
\StrBetween[1,2]{#2}{,}{,}[\angleA]
\StrBehind[2]{#2}{,}[\angleB]
\lsegment{\pointA}{\coteA}{#5}{\pointB}{#3} 
\setcounter{tempangle}{#5}
\addtocounter{tempangle}{\angleA}
\lsegment{\pointA}{1}{\thetempangle}{Z1}{PointSymbol=none, PointName=none}
\setcounter{tempangle}{#5}
\addtocounter{tempangle}{180}
\addtocounter{tempangle}{-\angleB}
\lsegment{\pointB}{1}{\thetempangle}{Z2}{PointSymbol=none, PointName=none}
\pstInterLL[#4]{\pointA}{Z1}{\pointB}{Z2}{\pointC}
}

%Construction d'un cercle de centre #1 et de rayon #2 (en cm)
\newcommand{\Cercle}[2]{
\lsegment{#1}{#2}{0}{Z1}{PointSymbol=none, PointName=none}
\pstCircleOA{#1}{Z1}
}

%construction d'un parallélogramme #1 = liste des sommets, #2 = liste contenant les longueurs de 2 côtés consécutifs et leurs angles;  #3, #4 et #5 : paramètre d'affichage des sommets #6 inclinaison par rapport à l'horizontale 
% meme macro sans le tracé des segements
\newcommand{\Para}[6]{
\StrBefore{#1}{,}[\pointA]
\StrBetween[1,2]{#1}{,}{,}[\pointB]
\StrBetween[2,3]{#1}{,}{,}[\pointC]
\StrBehind[3]{#1}{,}[\pointD]
\StrBefore{#2}{,}[\longueur]
\StrBetween[1,2]{#2}{,}{,}[\largeur]
\StrBehind[2]{#2}{,}[\angle]
\tsegment{\pointA}{\longueur}{#6}{\pointB}{#3} 
\setcounter{tempangle}{#6}
\addtocounter{tempangle}{\angle}
\tsegment{\pointA}{\largeur}{\thetempangle}{\pointD}{#5}
\pstMiddleAB[PointName=none,PointSymbol=none]{\pointB}{\pointD}{Z1}
\pstSymO[#4]{Z1}{\pointA}[\pointC]
\pstLineAB{\pointB}{\pointC}
\pstLineAB{\pointC}{\pointD}
}
\newcommand{\ParaP}[6]{
\StrBefore{#1}{,}[\pointA]
\StrBetween[1,2]{#1}{,}{,}[\pointB]
\StrBetween[2,3]{#1}{,}{,}[\pointC]
\StrBehind[3]{#1}{,}[\pointD]
\StrBefore{#2}{,}[\longueur]
\StrBetween[1,2]{#2}{,}{,}[\largeur]
\StrBehind[2]{#2}{,}[\angle]
\lsegment{\pointA}{\longueur}{#6}{\pointB}{#3} 
\setcounter{tempangle}{#6}
\addtocounter{tempangle}{\angle}
\lsegment{\pointA}{\largeur}{\thetempangle}{\pointD}{#5}
\pstMiddleAB[PointName=none,PointSymbol=none]{\pointB}{\pointD}{Z1}
\pstSymO[#4]{Z1}{\pointA}[\pointC]
}


%construction d'un cerf-volant #1 = liste des sommets, #2 = liste contenant les longueurs de 2 côtés consécutifs et leurs angles;  #3, #4 et #5 : paramètre d'affichage des sommets #6 inclinaison par rapport à l'horizontale 
% meme macro sans le tracé des segements
\newcommand{\CerfVolant}[6]{
\StrBefore{#1}{,}[\pointA]
\StrBetween[1,2]{#1}{,}{,}[\pointB]
\StrBetween[2,3]{#1}{,}{,}[\pointC]
\StrBehind[3]{#1}{,}[\pointD]
\StrBefore{#2}{,}[\longueur]
\StrBetween[1,2]{#2}{,}{,}[\largeur]
\StrBehind[2]{#2}{,}[\angle]
\tsegment{\pointA}{\longueur}{#6}{\pointB}{#3} 
\setcounter{tempangle}{#6}
\addtocounter{tempangle}{\angle}
\tsegment{\pointA}{\largeur}{\thetempangle}{\pointD}{#5}
\pstOrtSym[#4]{\pointB}{\pointD}{\pointA}[\pointC]
\pstLineAB{\pointB}{\pointC}
\pstLineAB{\pointC}{\pointD}
}

%construction d'un quadrilatère quelconque #1 = liste des sommets, #2 = liste contenant les longueurs des 4 côtés et l'angle entre 2 cotés consécutifs  #3, #4 et #5 : paramètre d'affichage des sommets #6 inclinaison par rapport à l'horizontale 
% meme macro sans le tracé des segements
\newcommand{\Quadri}[6]{
\StrBefore{#1}{,}[\pointA]
\StrBetween[1,2]{#1}{,}{,}[\pointB]
\StrBetween[2,3]{#1}{,}{,}[\pointC]
\StrBehind[3]{#1}{,}[\pointD]
\StrBefore{#2}{,}[\coteA]
\StrBetween[1,2]{#2}{,}{,}[\coteB]
\StrBetween[2,3]{#2}{,}{,}[\coteC]
\StrBetween[3,4]{#2}{,}{,}[\coteD]
\StrBehind[4]{#2}{,}[\angle]
\tsegment{\pointA}{\coteA}{#6}{\pointB}{#3} 
\setcounter{tempangle}{#6}
\addtocounter{tempangle}{\angle}
\tsegment{\pointA}{\coteD}{\thetempangle}{\pointD}{#5}
\lsegment{\pointB}{\coteB}{0}{Z1}{PointSymbol=none, PointName=none}
\lsegment{\pointD}{\coteC}{0}{Z2}{PointSymbol=none, PointName=none}
\pstInterCC[PointNameA=none,PointSymbolA=none,#4]{\pointB}{Z1}{\pointD}{Z2}{Z3}{\pointC} 
\pstLineAB{\pointB}{\pointC}
\pstLineAB{\pointC}{\pointD}
}


% Définition des colonnes centrées ou à droite pour tabularx
\newcolumntype{Y}{>{\centering\arraybackslash}X}
\newcolumntype{Z}{>{\flushright\arraybackslash}X}

%Les pointillés à remplir par les élèves
\newcommand{\po}[1]{\makebox[#1 cm]{\dotfill}}
\newcommand{\lpo}[1][3]{%
\multido{}{#1}{\makebox[\linewidth]{\dotfill}
}}

%Liste des pictogrammes utilisés sur la fiche d'exercice ou d'activités
\newcommand{\bombe}{\faBomb}
\newcommand{\livre}{\faBook}
\newcommand{\calculatrice}{\faCalculator}
\newcommand{\oral}{\faCommentO}
\newcommand{\surfeuille}{\faEdit}
\newcommand{\ordinateur}{\faLaptop}
\newcommand{\ordi}{\faDesktop}
\newcommand{\ciseaux}{\faScissors}
\newcommand{\danger}{\faExclamationTriangle}
\newcommand{\out}{\faSignOut}
\newcommand{\cadeau}{\faGift}
\newcommand{\flash}{\faBolt}
\newcommand{\lumiere}{\faLightbulb}
\newcommand{\compas}{\dsmathematical}
\newcommand{\calcullitteral}{\faTimesCircleO}
\newcommand{\raisonnement}{\faCogs}
\newcommand{\recherche}{\faSearch}
\newcommand{\rappel}{\faHistory}
\newcommand{\video}{\faFilm}
\newcommand{\capacite}{\faPuzzlePiece}
\newcommand{\aide}{\faLifeRing}
\newcommand{\loin}{\faExternalLink}
\newcommand{\groupe}{\faUsers}
\newcommand{\bac}{\faGraduationCap}
\newcommand{\histoire}{\faUniversity}
\newcommand{\coeur}{\faSave}
\newcommand{\os}{\faMicrochip}
\newcommand{\rd}{\faCubes}
\newcommand{\data}{\faColumns}
\newcommand{\web}{\faCode}
\newcommand{\prog}{\faFile}
\newcommand{\algo}{\faCogs}
\newcommand{\important}{\faExclamationCircle}
\newcommand{\maths}{\faTimesCircle}
% Traitement des données en tables
\newcommand{\tables}{\faColumns}
% Types construits
\newcommand{\construits}{\faCubes}
% Type et valeurs de base
\newcommand{\debase}{{\footnotesize \faCube}}
% Systèmes d'exploitation
\newcommand{\linux}{\faLinux}
\newcommand{\sd}{\faProjectDiagram}
\newcommand{\bd}{\faDatabase}

%Les ensembles de nombres
\renewcommand{\N}{\mathbb{N}}
\newcommand{\D}{\mathbb{D}}
\newcommand{\Z}{\mathbb{Z}}
\newcommand{\Q}{\mathbb{Q}}
\newcommand{\R}{\mathbb{R}}
\newcommand{\C}{\mathbb{C}}

%Ecriture des vecteurs
\newcommand{\vect}[1]{\vbox{\halign{##\cr 
  \tiny\rightarrowfill\cr\noalign{\nointerlineskip\vskip1pt} 
  $#1\mskip2mu$\cr}}}


%Compteur activités/exos et question et mise en forme titre et questions
\newcounter{numact}
\setcounter{numact}{1}
\newcounter{numseance}
\setcounter{numseance}{1}
\newcounter{numexo}
\setcounter{numexo}{0}
\newcounter{numprojet}
\setcounter{numprojet}{0}
\newcounter{numquestion}
\newcommand{\espace}[1]{\rule[-1ex]{0pt}{#1 cm}}
\newcommand{\Quest}[3]{
\addtocounter{numquestion}{1}
\begin{tabularx}{\textwidth}{X|m{1cm}|}
\cline{2-2}
\textbf{\sffamily{\alph{numquestion})}} #1 & \dots / #2 \\
\hline 
\multicolumn{2}{|l|}{\espace{#3}} \\
\hline
\end{tabularx}
}
\newcommand{\mq}[1]
{\ding{113} \addtocounter{numquestion}{1}
\textbf{Question \arabic{numquestion}} \\ #1}
\newcommand{\QuestR}[3]{
\addtocounter{numquestion}{1}
\begin{tabularx}{\textwidth}{X|m{1cm}|}
\cline{2-2}
\textbf{\sffamily{\alph{numquestion})}} #1 & \dots / #2 \\
\hline 
\multicolumn{2}{|l|}{\cor{#3}} \\
\hline
\end{tabularx}
}
\newcommand{\Pre}{{\sc nsi} 1\textsuperscript{e}}
\newcommand{\Term}{{\sc nsi} Terminale}
\newcommand{\Sec}{2\textsuperscript{e}}
\newcommand{\Exo}[2]{ \addtocounter{numexo}{1} \ding{113} \textbf{\sffamily{Exercice \thenumexo}} : \textit{#1} \hfill #2  \setcounter{numquestion}{0}}
\newcommand{\Projet}[1]{ \addtocounter{numprojet}{1} \ding{118} \textbf{\sffamily{Projet \thenumprojet}} : \textit{#1}}
\newcommand{\ExoD}[2]{ \addtocounter{numexo}{1} \ding{113} \textbf{\sffamily{Exercice \thenumexo}}  \textit{(#1 pts)} \hfill #2  \setcounter{numquestion}{0}}
\newcommand{\ExoB}[2]{ \addtocounter{numexo}{1} \ding{113} \textbf{\sffamily{Exercice \thenumexo}}  \textit{(Bonus de +#1 pts maximum)} \hfill #2  \setcounter{numquestion}{0}}
\newcommand{\Act}[2]{ \ding{113} \textbf{\sffamily{Activité \thenumact}} : \textit{#1} \hfill #2  \addtocounter{numact}{1} \setcounter{numquestion}{0}}
\newcommand{\Seance}{ \rule{1.5cm}{0.5pt}\raisebox{-3pt}{\framebox[4cm]{\textbf{\sffamily{Séance \thenumseance}}}}\hrulefill  \\
  \addtocounter{numseance}{1}}
\newcommand{\Acti}[2]{ {\footnotesize \ding{117}} \textbf{\sffamily{Activité \thenumact}} : \textit{#1} \hfill #2  \addtocounter{numact}{1} \setcounter{numquestion}{0}}
\newcommand{\titre}[1]{\begin{Large}\textbf{\ding{118}}\end{Large} \begin{large}\textbf{ #1}\end{large} \vspace{0.2cm}}
\newcommand{\QListe}[1][0]{
\ifthenelse{#1=0}
{\begin{enumerate}[partopsep=0pt,topsep=0pt,parsep=0pt,itemsep=0pt,label=\textbf{\sffamily{\arabic*.}},series=question]}
{\begin{enumerate}[resume*=question]}}
\newcommand{\SQListe}[1][0]{
\ifthenelse{#1=0}
{\begin{enumerate}[partopsep=0pt,topsep=0pt,parsep=0pt,itemsep=0pt,label=\textbf{\sffamily{\alph*)}},series=squestion]}
{\begin{enumerate}[resume*=squestion]}}
\newcommand{\SQListeL}[1][0]{
\ifthenelse{#1=0}
{\begin{enumerate*}[partopsep=0pt,topsep=0pt,parsep=0pt,itemsep=0pt,label=\textbf{\sffamily{\alph*)}},series=squestion]}
{\begin{enumerate*}[resume*=squestion]}}
\newcommand{\FinListe}{\end{enumerate}}
\newcommand{\FinListeL}{\end{enumerate*}}

%Mise en forme de la correction
\newboolean{corrige}
\setboolean{corrige}{false}
\newcommand{\scor}[1]{\par \textcolor{blue!75!black}{\small #1}}
\newcommand{\cor}[1]{\par \textcolor{blue!75!black}{#1}}
\newcommand{\br}[1]{\cor{\textbf{#1}}}
\newcommand{\tcor}[1]{
\ifthenelse{\boolean{corrige}}{\begin{tcolorbox}[width=\linewidth,colback={white},colbacktitle=white,coltitle=green!50!black,colframe=green!50!black,boxrule=0.2mm]   
\cor{#1}
\end{tcolorbox}}{}
}
\newcommand{\iscor}[1]{\ifthenelse{\boolean{corrige}}{#1}}
\newcommand{\rc}[1]{\textcolor{OliveGreen}{#1}}

%Référence aux exercices par leur numéro
\newcommand{\refexo}[1]{
\refstepcounter{numexo}
\addtocounter{numexo}{-1}
\label{#1}}

%Séparation entre deux activités
\newcommand{\separateur}{\begin{center}
\rule{1.5cm}{0.5pt}\raisebox{-3pt}{\ding{117}}\rule{1.5cm}{0.5pt}  \vspace{0.2cm}
\end{center}}

%Entête et pied de page
\newcommand{\snt}[1]{\lhead{\textbf{SNT -- La photographie numérique} \rhead{\textit{Lycée Nord}}}}
\newcommand{\Activites}[2]{\lhead{\textbf{{\sc #1}}}
\rhead{Activités -- \textbf{#2}}
\cfoot{}}
\newcommand{\Exos}[2]{\lhead{\textbf{Fiche d'exercices: {\sc #1}}}
\rhead{Niveau: \textbf{#2}}
\cfoot{}}
\newcommand{\TD}[2]{\lhead{\textbf{TD #1} : {\sc #2} }
\rhead{{\sc mp2i -- Lycée Leconte de Lisle}}
\cfoot{}}
\newcommand{\Colles}[2]{\lhead{{\sc mp2i -- }\textbf{Colles d'informatique #1}} 
\rhead{{\sc #2}}
\cfoot{}}
\newcommand{\Devoir}[2]{\lhead{\textbf{Devoir de mathématiques : {\sc #1}}}
\rhead{\textbf{#2}} \setlength{\fboxsep}{8pt}
\begin{center}
%Titre de la fiche
\fbox{\parbox[b][1cm][t]{0.3\textwidth}{Nom : \hfill \po{3} \par \vfill Prénom : \hfill \po{3}} } \hfill 
\fbox{\parbox[b][1cm][t]{0.6\textwidth}{Note : \po{1} / 20} }
\end{center} \cfoot{}}
\newcommand{\TPnote}[3]{\lhead{\textbf{TP noté d'informatique n° #1}}
\rhead{\textbf{#2}} \setlength{\fboxsep}{8pt}
\ifthenelse{\boolean{corrige}}{}
{\begin{center}
\fbox{\parbox[b][1cm][t]{0.3\textwidth}{Nom : \hfill \po{3} \par \vfill Prénom : \hfill \po{3}} } \hfill 
\fbox{\parbox[b][1cm][t]{0.6\textwidth}{Note : \po{1} / #3} }
\end{center}} \cfoot{}}
\newcommand{\IC}[2]{\lhead{\textbf{Interro de cours n° #1}}
\rhead{{\sc mp2i --} \textbf{#2}} \setlength{\fboxsep}{8pt}
\ifthenelse{\boolean{corrige}}{}
{\begin{center}
%Titre de la fiche
\fbox{\parbox[b][1cm][t]{0.3\textwidth}{Nom : \hfill \po{3} \par \vfill Prénom : \hfill \po{3}} } \hfill 
\fbox{\parbox[b][1cm][t]{0.6\textwidth}{Note : \po{1} / 10} }
\end{center}}\cfoot{}}
\newcommand{\DS}[3]{\lhead{{#1} : \textbf{DS d'informatique n° #2}}
\rhead{Lycée Leconte de Lisle -- #3} \setlength{\fboxsep}{8pt}
%Titre de la fiche
\begin{center}
   {\Large \textbf{Devoir surveillé d'informatique}}
\end{center} \cfoot{\thepage/\pageref{LastPage}}}

\newcommand{\CB}[2]{\lhead{{#1} : \textbf{Councours blanc - Informatique}}
\rhead{Lycée Leconte de Lisle -- #2} \setlength{\fboxsep}{8pt}
%Titre de la fiche
\begin{center}
   {\Large \textbf{Concours Blanc - Epreuve d'informatique}}
\end{center} \cfoot{\thepage/\pageref{LastPage}}}

\newcommand{\PC}[3]{\lhead{Concours {#1} -- #2}
\rhead{Lycée Leconte de Lisle} \setlength{\fboxsep}{8pt}
%Titre de la fiche
\begin{center}
   {\Large \textbf{Proposition de corrigé}}
\end{center} \cfoot{\thepage/\pageref{LastPage}}}

\newcounter{numdspart}
\setcounter{numdspart}{1}
\newcommand{\DSPart}{\bigskip
   \hrulefill\raisebox{-3pt}{\framebox[4cm]{\textbf{\textbf{Partie \thenumdspart}}}}\hrulefill
   \addtocounter{numdspart}{1}
   \bigskip}

\newcommand{\Sauvegarde}[1]{
   \begin{tcolorbox}[title=\textcolor{black}{\danger\; Attention},colbacktitle=lightgray]
      {Tous vos programmes doivent être enregistrés dans votre dossier personnel, dans {\tt Evaluations}{\tt \textbackslash}{\tt #1}
      }
   \end{tcolorbox}
}

\newcommand{\alertbox}[3]{
   \begin{tcolorbox}[title=\textcolor{black}{#1\; #2},colbacktitle=lightgray]
      {#3}
   \end{tcolorbox}
}

%Devoir programmation en NSI (pas à rendre sur papier)
\newcommand{\PNSI}[2]{\lhead{\textbf{Devoir de {\sc nsi} : \textsf{ #1}}
}
\rhead{\textbf{#2}} \setlength{\fboxsep}{8pt}
 \cfoot{}
 \begin{center}{\Large \textbf{Evaluation de {\sc nsi}}}\end{center}}


%Devoir de NSI
\newcommand{\DNSI}[2]{\lhead{\textbf{Devoir de {\sc nsi} : \textsf{ #1}}}
\rhead{\textbf{#2}} \setlength{\fboxsep}{8pt}
\begin{center}
%Titre de la fiche
\fbox{\parbox[b][1cm][t]{0.3\textwidth}{Nom : \hfill \po{3} \par \vfill Prénom : \hfill \po{3}} } \hfill 
\fbox{\parbox[b][1cm][t]{0.6\textwidth}{Note : \po{1} / 10} }
\end{center} \cfoot{}}

\newcommand{\DevoirNSI}[2]{\lhead{\textbf{Devoir de {\sc nsi} : {\sc #1}}}
\rhead{\textbf{#2}} \setlength{\fboxsep}{8pt}
\cfoot{}}

%La définition de la commande QCM pour auto-multiple-choice
%En premier argument le sujet du qcm, deuxième argument : la classe, 3e : la durée prévue et #4 : présence ou non de questions avec plusieurs bonnes réponses
\newcommand{\QCM}[4]{
{\large \textbf{\ding{52} QCM : #1}} -- Durée : \textbf{#3 min} \hfill {\large Note : \dots/10} 
\hrule \vspace{0.1cm}\namefield{}
Nom :  \textbf{\textbf{\nom{}}} \qquad \qquad Prénom :  \textbf{\prenom{}}  \hfill Classe: \textbf{#2}
\vspace{0.2cm}
\hrule  
\begin{itemize}[itemsep=0pt]
\item[-] \textit{Une bonne réponse vaut un point, une absence de réponse n'enlève pas de point. }
\item[\danger] \textit{Une mauvaise réponse enlève un point.}
\ifthenelse{#4=1}{\item[-] \textit{Les questions marquées du symbole \multiSymbole{} peuvent avoir plusieurs bonnes réponses possibles.}}{}
\end{itemize}
}
\newcommand{\DevoirC}[2]{
\renewcommand{\footrulewidth}{0.5pt}
\lhead{\textbf{Devoir de mathématiques : {\sc #1}}}
\rhead{\textbf{#2}} \setlength{\fboxsep}{8pt}
\fbox{\parbox[b][0.4cm][t]{0.955\textwidth}{Nom : \po{5} \hfill Prénom : \po{5} \hfill Classe: \textbf{1}\textsuperscript{$\dots$}} } 
\rfoot{\thepage} \cfoot{} \lfoot{Lycée Nord}}
\newcommand{\DevoirInfo}[2]{\lhead{\textbf{Evaluation : {\sc #1}}}
\rhead{\textbf{#2}} \setlength{\fboxsep}{8pt}
 \cfoot{}}
\newcommand{\DM}[2]{\lhead{\textbf{Devoir maison à rendre le #1}} \rhead{\textbf{#2}}}

%Macros permettant l'affichage des touches de la calculatrice
%Touches classiques : #1 = 0 fond blanc pour les nombres et #1= 1gris pour les opérations et entrer, second paramètre=contenu
%Si #2=1 touche arrondi avec fond gris
\newcommand{\TCalc}[2]{
\setlength{\fboxsep}{0.1pt}
\ifthenelse{#1=0}
{\psframebox[fillstyle=solid, fillcolor=white]{\parbox[c][0.25cm][c]{0.6cm}{\centering #2}}}
{\ifthenelse{#1=1}
{\psframebox[fillstyle=solid, fillcolor=lightgray]{\parbox[c][0.25cm][c]{0.6cm}{\centering #2}}}
{\psframebox[framearc=.5,fillstyle=solid, fillcolor=white]{\parbox[c][0.25cm][c]{0.6cm}{\centering #2}}}
}}
\newcommand{\Talpha}{\psdblframebox[fillstyle=solid, fillcolor=white]{\hspace{-0.05cm}\parbox[c][0.25cm][c]{0.65cm}{\centering \scriptsize{alpha}}} \;}
\newcommand{\Tsec}{\psdblframebox[fillstyle=solid, fillcolor=white]{\parbox[c][0.25cm][c]{0.6cm}{\centering \scriptsize 2nde}} \;}
\newcommand{\Tfx}{\psdblframebox[fillstyle=solid, fillcolor=white]{\parbox[c][0.25cm][c]{0.6cm}{\centering \scriptsize $f(x)$}} \;}
\newcommand{\Tvar}{\psframebox[framearc=.5,fillstyle=solid, fillcolor=white]{\hspace{-0.22cm} \parbox[c][0.25cm][c]{0.82cm}{$\scriptscriptstyle{X,T,\theta,n}$}}}
\newcommand{\Tgraphe}{\psdblframebox[fillstyle=solid, fillcolor=white]{\hspace{-0.08cm}\parbox[c][0.25cm][c]{0.68cm}{\centering \tiny{graphe}}} \;}
\newcommand{\Tfen}{\psdblframebox[fillstyle=solid, fillcolor=white]{\hspace{-0.08cm}\parbox[c][0.25cm][c]{0.68cm}{\centering \tiny{fenêtre}}} \;}
\newcommand{\Ttrace}{\psdblframebox[fillstyle=solid, fillcolor=white]{\parbox[c][0.25cm][c]{0.6cm}{\centering \scriptsize{trace}}} \;}

% Macroi pour l'affichage  d'un entier n dans  une base b
\newcommand{\base}[2]{ \overline{#1}^{#2}}
% Intervalle d'entiers
\newcommand{\intN}[2]{\llbracket #1; #2 \rrbracket}
% Cadre avec lignes réponses
\def\gaddtotok#1{\global\tabtok\expandafter{\the\tabtok#1}}
\newtoks\tabtok
\newcommand*\reponse[2]{%
   \ifthenelse{\boolean{corrige}}{}{
	\global\tabtok{\\ \renewcommand{\arraystretch}{1.4}\begin{tabularx}{\linewidth}{|X|p{1cm}|}\hline \dotfill & \cellcolor{gray!30}{\small \dots/#2} \\ \cline{2-2}}%
	\multido{}{#1}{\gaddtotok{ \multicolumn{2}{|>{\hsize=\dimexpr1\hsize+2\tabcolsep+\arrayrulewidth+1cm\relax}X|}{\dotfill}\\ }}%
	\gaddtotok{\hline \end{tabularx}}%
	\the\tabtok
   }}

\newcommand{\PE}[1]{\left \lfloor #1 \right \lfloor}}
\ModeExercice
\DS{MP2I}{1}{Septembre 2023}

\alertbox{\danger}{Consignes}{
    \begin{itemize}
        \item[\textbullet] Les programmes demandés doivent être écrits en C et on suppose que les librairies standards usuelles ({\tt <stdio.h>}, {\tt <stdlib.h>}, {\tt <stdbool.h>}) sont déjà importées.
        \item[\textbullet] On pourra toujours librement utiliser une fonction demandée à une question précédente même si cette question n'a pas été traitée.
        \item[\textbullet] Veillez à présenter vos idées et vos réponses partielles même si vous ne trouvez pas la solution complète à une question.
        \item[\textbullet] La clarté et la lisibilité de la rédaction et des programmes sont des éléments de notation.
    \end{itemize}
}

\begin{Exercise}[title = {chercher les erreurs}]
\Question{Les fonctions ci-dessous contiennent une ou plusieurs erreurs et/ou ne respectent pas leur spécification (donnée en commentaire). Dans chaque cas, expliquer les erreurs commises et les corriger de façon à ce que la fonction obéisse à sa spécification.}

\subQuestion{Fonction {\tt harmonique}
\inputpartC{\FPATH Evaluations/DS/DS1/DS1_ex1.c}{}{\small}{5}{13}
}

\tcor{
    \textbullet\  A la ligne 6, on ne doit pas commencer à l'indice 0 (sinon on effectue une division par zéro) ni s'arrêter à l'indice ($n-1$) (on ne respecte pas la specification). La ligne correcte est donc \mintinline{c}{for (int i=1; i<n+1; i = i + 1)} \\
    \textbullet\  A la ligne 8, {\tt 1/i} est une division entière car les deux opérandes sont des entiers. Le résultat est donc 0 (même si ensuite on a une conversion implicite de type vers {\tt double} de façon à additionner avec {\tt sh}). On doit donc pour forcer la division décimale en changeant cette ligne en \mintinline{c}{sh = sh + 1.0/i;} ou encore avec une conversion explicite : \mintinline{c}{sh = sh + 1/(double)i;}
}

\subQuestion{Fonction {\tt tous\_egaux}
\inputpartC{\FPATH/Evaluations/DS/DS1/DS1_ex1.c}{}{\small}{15}{29}
}

\tcor{
    \textbullet\  la ligne 7 provoque un accès en dehors des limites du tableau car $i$ a pour valeur maximale {\tt size-1}. On doit donc modifier la lignes 5 en \mintinline{c}{for (int i=0; i<size-1; i = i + 1)}  \\
    \textbullet\  Le bloc du {\tt else} ligne 11 à 14 renvoie {\tt true} dès que les deux premiers éléments sont égaux. Il ne faut le faire qu'après avoir parcouru tout le tableau. Le {\tt if} ne devrait donc pas avoir de bloc {\tt else}, et le {\tt return true} doit être positioné après le bloc de la boucle {\tt for}
}


\subQuestion{Fonction {\tt cree\_tab\_entiers}
\inputpartC{\FPATH/Evaluations/DS/DS1/DS1_ex1.c}{}{\small}{31}{38}
}

\tcor{
    \textbullet\  le tableau {\tt tab\_entiers} de la ligne 4 est alloué sur la pile car c'est une variable locale à la fonction. On doit l'allouer sur le tas de façon à conserver cette zone mémoire en quittant la fonction. On doit donc remplacer cette ligne par \mintinline{c}{int *tab_entiers = malloc(sizeof(int)*size);}  \\
    \textbullet\  {\tt tab\_entiers} est un tableau, donc un pointeur vers l'adresse de son premier élément. On doit donc simplement écrire \mintinline{c}{return tab_entiers;}
}

\subQuestion{Fonction {\tt syracuse}
\inputpartC{\FPATH/Evaluations/DS/DS1/DS1_ex1.c}{}{\small}{40}{49}}
\tcor{
    \textbullet\  Les variables en C sont passés par valeur, donc cette fonction ne modifie pas {\tt n}. On doit si on veut respecter la spécification passer un pointeur vers {\tt n}. La ligne 2 est donc \mintinline{c}{void syracuse(int *n)} et on doit remplacer {\tt n} par {\tt *n} dans le reste de la fonction. L'appel à cette fonction si {\tt n} est déclaré comme un entier se fera avec \mintinline{c}{syracuse(&n)} de façon à passer l'adresse de {\tt n}. \\
    \textbullet\  En C, le test d'égalité est {\tt ==}, la ligne 4 est donc \mintinline{c}{if (n}{\tt \%}\mintinline{c}{2 == 0)}.
}

\Question{Les programmes suivants, produisent une erreur à l'exécution. Expliquer l'origine du problème et apporter les corrections nécessaires}
\subQuestion{Affichage d'une adresse
\inputpartC{\FPATH/Evaluations/DS/DS1/DS1_ex1.c}{}{\small}{51}{55}
}
\tcor{
    La variable {\tt p} est un pointeur \textit{non initialisé} vers un entier, aucune zone mémoire n'a été réservé pour sa valeur. On doit donc lorsqu'on déclare cette variable lui alloué un emplacement mémoire avec {\tt malloc}. La ligne 4 est donc \mintinline{c}{int *p = malloc(sizeof(int));}
}
\subQuestion{Calcul de la somme de deux entiers 
\inputpartC{\FPATH/Evaluations/DS/DS1/DS1_ex1.c}{}{\small}{57}{64}
}
\tcor{
    La fonction {\tt scanf} va modifier la valeur de la variable donnée en argument (ici {\tt a} puis {\tt b}), donc elle prend en argument des pointeurs vers ces valeurs (sinon elle ne pourrait pas les modifier), la ligne 6 est donc \tt{scanf("\%d", \&a);}}

\end{Exercise}


\begin{Exercise}[title={Pointeurs}]
	\Question{Compléter le tableau suivant, qui donne l'état des variables au fur et à mesure des instructions données dans la première colonne (on a indiqué par \faTimes{} une variable non encore déclaré.)
		\begin{center}
			\def\arraystretch{1.2}
			\setlength\tabcolsep{0.5cm}
			\begin{tabular}{|l|c|c|c|c|}
                \hline
				        instructions            & a  & b  & p & q       \\
				\hline
				{\tt int a = 14;}      & 14 & \faTimes & \faTimes & \faTimes      \\
				{\tt int b = 42;}      & 14  & \cor{42}  & \cor{\faTimes}  &  \cor{\faTimes}           \\
				{\tt int *p = \&a;}      & 14   &  \cor{42}  &   \cor{\&a}     & \cor{\faTimes}     \\
                {\tt int *q = \&b;}      & 14   &  \cor{42}  &   \cor{\&a}    & \cor{\&b}     \\
				{\tt *p = *p + *q ;} &  \cor{56} &  \cor{42}  &   \cor{\&a}    & \cor{\&b}     \\
				{\tt *q = *p - *q ;} &  \cor{56} &  \cor{14}  &   \cor{\&a}    & \cor{\&b}     \\
                {\tt *p = *p - *q ;} &  \cor{42} &  \cor{14}  &   \cor{\&a}    & \cor{\&b}     \\
                \hline
			\end{tabular}
		\end{center}}
    \Question{Ecrire une fonction en C qui prend en argument deux pointeurs vers des entiers, ne renvoie rien et échange les valeurs de ces deux entiers \textit{sans utiliser de variable temporaire}.}
    \inputpartC{\FPATH/Evaluations/DS/DS1/echange.c}{}{\small}{3}{8}
    \Question{Compléter le programme suivant en écrivant l'appel à la fonction {\tt echange} afin d'échanger les valeurs des entiers {\tt n} et {\tt m} 
    \inputpartC{\FPATH/Evaluations/DS/DS1/echange.c}{}{\small}{14}{15}}
\end{Exercise}

\begin{Exercise}[title = {puissance}]
\Question{Ecrire une fonction {\tt valeur\_absolue} qui prend en argument un entier $n$ et renvoie sa valeur absolue $|n|$. On rappelle que :
$|n| = \left\{ \begin{array}{rl} -n & \mathrm{\ si\ } n<0 \\ n & \mathrm{\ sinon} \end{array}\right.$}
\inputpartC{\FPATH/Evaluations/DS/DS1/puissance.c}{}{\small}{6}{11}
\Question{Ecrire une fonction {\tt puissance} qui prend en argument un flottant (type {\tt double}) $a$ et un entier $n$ et renvoie 
$a^n$. On rappelle que pour $a \in \R^*$, $n \in \Z$ : \\
$\left\{ \begin{array}{ll}
a^n = \underbrace{a\times \dots \times a}_{n \mathrm{\ facteurs}}  & \mathrm{\ si\ } n>0,\\
a^0 = 1, &  \\
a^n = \dfrac{1}{a^{-n}} & \mathrm{\ si\ } n<0. \\
\end{array}\right.$ \\

D'autre part $0^0=1$, $0^n=0$ si $n>0$ et les puissances négatives de zéro ne sont pas définies.On vérifiera la précondition $n>0$ lorsque $a=0$ à l'aide d'une instruction {\tt assert}.} 
\inputpartC{\FPATH/Evaluations/DS/DS1/puissance.c}{}{\small}{13}{43}
\Question{Tracer le graphe de flot de contrôle de cette fonction.}
\tcor{\vspace{6cm}
    \rput(4,-0.2){\circlenode[linecolor=red]{E}{\textcolor{red}{E}}}
    \rput(4,-1){\rnode{I}{\psframebox{\tt p=1.0}}}
    \ncline{->}{E}{I}
    \rput(4,-2){\ovalnode{T1}{{\tt a==0}}}
    \ncline{->}{I}{T1}
    \rput(6,-2){\ovalnode{T2}{{\tt n==0}}}
    \rput(9,-2){\rnode{R1}{\psframebox{\tt return 1.0}}} 
    \ncline{->}{T2}{R1} \naput[labelsep=1pt]{\small \textcolor{OliveGreen}{V}}
    \rput(9,-3){\rnode{R2}{\psframebox{\tt return 0.0}}} 
    \ncline{->}{T2}{R2} \nbput[labelsep=1pt]{\small \textcolor{BrickRed}{F}}
    \ncline{->}{T1}{T2} \naput[labelsep=1pt]{\small \textcolor{OliveGreen}{V}}
    \rput(4,-3){\rnode{I2}{\psframebox{\tt i=0}}}
    \ncline{->}{T1}{I2} \naput[labelsep=1pt]{\small \textcolor{BrickRed}{F}}
    \rput(4,-4){\ovalnode{T3}{{\tt i>=|n|}}}
    \ncline{->}{I2}{T3}
    \rput(6,-4){\ovalnode{T4}{{\tt n<0}}}
    \ncline{->}{T3}{T4} \naput[labelsep=1pt]{\small \textcolor{OliveGreen}{V}}
    \rput(9,-4){\rnode{R3}{\psframebox{\tt return 1/p}}} 
    \rput(9,-5){\rnode{R4}{\psframebox{\tt return p}}} 
    \ncline{->}{T4}{R3} \naput[labelsep=1pt]{\small \textcolor{OliveGreen}{V}}
    \ncline{->}{T4}{R4} \naput[labelsep=1pt]{\small \textcolor{BrickRed}{F}}
    \rput(4,-5.5){\rnode{M}{\psframebox{\begin{tabular}{>{\tt}l}  p = p * a \\ i = i +1 \end{tabular}}}}
    \ncline{->}{T3}{M} \naput[labelsep=1pt]{\small \textcolor{BrickRed}{F}}
    \rput(13,-3.5){\circlenode[linecolor=red]{S}{\textcolor{red}{S}}}
    \ncline{->}{R1}{S}
    \ncline{->}{R2}{S}
    \ncline{->}{R3}{S}
    \ncline{->}{R4}{S}
    \nccurve[angle=180]{->}{M}{T3}
}
\Question{Proposer un jeu de test permettant de couvrir tous les arcs.}
\tcor{Le jeu de test suivant permet de couvrir tous les arcs :
\begin{itemize}
    \item[\textbullet] $a=0$ et $n=0$
    \item[\textbullet] $a=0$ et $n=1$
    \item[\textbullet] $a=2$ et $n=3$
    \item[\textbullet] $a=2$ et $n=-3$
\end{itemize}}
\end{Exercise}


\begin{Exercise}[title={Annagrammes}]

Deux mots \textit{de même longueur} sont anagrammes l'un de l'autre lorsque l'un est formé en réarrangeant les lettres de l'autre. Par exemple :
\begin{itemize}
    \item \textit{niche} et \textit{chien} sont des anagrammes.
    \item \textit{epele} et \textit{pelle}, ne sont pas des anagrammes, en effet bien qu'ils soient formés avec les mêmes lettres, la lettre \textit{l} ne figure qu'à un seul exemplaire dans \textit{epele} et il en faut deux pour écrire \textit{pelle}.
\end{itemize}
Le but de l'exercice est d'écrire une fonction en C qui renvoie {\tt true} si les deux chaines données en argument sont des anagrammes et {\tt false} sinon. On suppose par facilité que les chaines sont constitués uniquement de lettres majuscules.

\Question{Ecrire une fonction {\tt tableau\_egaux} qui prend en argument deux tableaux ainsi que leur taille et renvoie {\tt true} lorsque ces deux tableaux ont un contenu et une longueur identique et {\tt false} sinon.}
\inputpartC{\FPATH/Evaluations/DS/DS1/anagrammes.c}{}{\small}{5}{15}

\Question{Ecrire une fonction {\tt nb\_lettres} qui prend en argument une chaine de caractères {\tt chaine} et renvoie un tableau d'entiers {\tt tab} de longueur 26 de sorte que {\tt tab[i-1]} contienne le nombre de fois où la ième lettre de l'alphabet apparait dans {\tt chaine}. Par exemple {tab[0]}, doit contenir le nombre de fois où la lettre \textit{a} apparait dans le mot.}
\inputpartC{\FPATH/Evaluations/DS/DS1/anagrammes.c}{}{\small}{17}{33}

\Question{En supposant les deux fonctions précédentes correctement écrites, on propose le code suivant pour la fonction {\tt anagrammes} qui teste si les deux chaines données en argument sont des anagrammes :
\inputpartC{\FPATH/Evaluations/DS/DS1/anagrammes.c}{}{\small}{35}{38}
Cette fonction renvoie le résultat attendu mais pose un problème, lequel ? Expliquer et proposer une correction.
}
\tcor{
    La fonction {\tt nb\_lettres} renvoie un pointeur vers une zone mémoire allouée sur le tas. Dans la fonction {\tt anagrammes}, on ne conserve pas de référence sur cette zone (elle est directement utilisée sur la fonction {\tt tab\_egaux}). Donc, cette zone mémoire n'est pas libérable par un {\tt free} puisqu'on n'a pas de référence vers elle. Chaque appel à la fonction {\tt anagrammes} entraine donc une fuite mémoire. On peut proposer la fonction  suivante comme correction, où on garde des références vers les tableaux crées par {\tt nb\_lettres} de façon à libérer l'espace mémoire alloué.
}
\inputpartC{\FPATH/Evaluations/DS/DS1/anagrammes.c}{}{\small}{40}{57}


\end{Exercise}


\begin{Exercise}[title = {tri à bulles}] \\
Le tri à bulles est un algorithme de tri qui parcourt le tableau à l'aide d'un indice $i$ de la fin vers le début. Pour chacun de ces indices $i$, on parcourt la partie du tableau allant de l'indice 0 à l'indice $i-1$ et si les deux éléments consécutifs situés aux indices $j$ et $j+1$ ne sont pas dans l'ordre croissant, on les échange. Par exemple sur le tableau {\tt \{7, 2, 9, 5, 3\}}
\begin{itemize}
    \item[\textbullet] $i = 4$ (on rappelle que $i$ parcourt de la fin vers le début)
    \begin{itemize}
        \item $j=0$, donc on échange {\tt tab[0]} et {\tt tab[1]} car ils ne sont pas dans l'ordre croisant  : {\tt \{2, 7, 9, 5, 3\}}
        \item $j=1$, pas d'échange (7 et 9 sont en ordre croissant)
        \item $j=2$, échange car 9 et 5 ne sont pas dans l'ordre croissant : {\tt \{2, 7, 5, 9, 3\}}
        \item $j=3$, échange et on obtient {\tt \{2, 7, 5, 3, 9\}}
    \end{itemize}
    \item[\textbullet] $i=3$, cette fois on parcourt avec $j=0$ jusqu'à $2$, en effet à l'étape précédente le plus grand élément du tableau se retrouve forcement en dernière position. On obtient en fin de parcours : {\tt \{2, 5, 3, 7, 9\}}
    \item[\textbullet] $i=2$, à la fin de cette itération on obtient {\tt \{2, 3, 5, 7, 9\}}
\end{itemize}

\Question{Faire fonctionner cet algorithme à la main sur le tableau {\tt \{11, 2, 5, 13, 8, 4\}} et donner l'état du tableau à la fin de chaque itération de l'indice $j$ pour $j$ variant de $0$ à $i$ en recopiant et complétant le tableau suivant : \\
\begin{tabular}{|l|p{12cm}|}
    \hline
    $i$ & valeurs contenu dans le tableau à la fin de l'itération d'indice $i$.\\
    \hline
    5  & \cor{\tt \{2,5,11,8,4,13\}} \\
    \hline
    4  & \cor{\tt \{2,5,8,4,11,13\}} \\
    \hline
    3  & \cor{\tt \{2,5,4,8,11,13\}} \\
    \hline
    2  & \cor{\tt \{2,4,5,8,11,13\}} \\
        \hline
    1  & \cor{\tt \{2,4,5,8,11,13\}} \\
    \hline
\end{tabular}
    }
\Question{Donner un exemple de tableau de longueur 5, \textit{non trié initialement} qui sera entièrement trié après le premier tour de boucle de l'indice $j$ (c'est à dire pour $i=4$).}
\tcor{Si les éléments sont déjà triés à l'exception de l'élément maximal, la seule itération de $j$ pour $j=0$ jusqu'à $i=4$, suffit à ramener cet élément en fin tableau. Donc par exemple le tableau {\tt \{ 2, 99, 3, 4, 5 \} } sera entièrement trié dès que $i=3$..}
\Question{Donner un exemple de tableau qui  ne sera trié qu'à la fin de toutes les itérations de l'indice~$i$.}
\tcor{Si le tableau a son élément minimal tout à la fin, alors cet élément se déplace vers le début du tableau une fois par itération de l'indice $i$, donc le tableau ne sera trié qu'à la toute fin de l'algorithme. On peut donner l'exemple du tableau {\tt \{ 50, 49, 48, 47, 1 \} }}
\Question{On veut maintenant écrire cet algorithme en C, en effectuant le tri dans une copie du tableau. On suppose déjà écrite la fonction {\tt echange} qui prend en argument un tableau et deux indices ne renvoie rien et échange les éléments situés à ces deux indices.}
\subQuestion{Ecrire un fonction {\tt copie\_tab} qui prend en argument un tableau ainsi que sa taille et renvoie un pointeur vers une copie de ce tableau}
\inputpartC{\FPATH/Evaluations/DS/DS1/tri_bulles.c}{}{\small}{25}{33}

\subQuestion{Ecrire une fonction {\tt tri\_bulles} qui prend en argument un tableau {\tt tab} ainsi que sa taille, ne modifie pas ce tableau et renvoie un pointeur vers un tableau contenant les éléments du tableau {\tt tab} triés dans l'ordre croissant.}
\inputpartC{\FPATH/Evaluations/DS/DS1/tri_bulles.c}{}{\small}{35}{51}

\subQuestion{Afin de tester cette fonction on a écrit le programme principal suivant :
\inputpartC{\FPATH/Evaluations/DS/DS1/tri_bulles.c}{}{\small}{53}{60}
Quelle instruction est manquante dans ce programme ? Quelle option de compilation signalerait le problème lors de l'exécution ?
}
\tcor{On doit libérer la mémoire allouée, l'instruction manquante est donc \mintinline{c}{free(tab_trie);} qu'on peut placer dès qu'on a plus besoin du tableau trié (donc ici après la boucle {\tt for} d'affichage). Le problème serait signalé lors de l'exécution si on on compile avec l'option {\tt fsanitize = address}.
}
\end{Exercise}

\end{document}
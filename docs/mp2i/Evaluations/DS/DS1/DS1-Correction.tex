\PassOptionsToPackage{dvipsnames,table}{xcolor}
\documentclass[11pt,a4paper]{article}

\usepackage{DS}

\begin{document}
\input{\detokenize{/home/fenarius/Travail/Cours/cpge-info/latex/Macros.tex}}
\ModeExercice
\DS{MP2I}{1}{Septembre 2023}

\alertbox{\danger}{Consignes}{
    \begin{itemize}
        \item[\textbullet] Les programmes demandés doivent être écrits en C et on suppose que les librairies standards usuelles ({\tt <stdio.h>}, {\tt <stdlib.h>}, {\tt <stdbool.h>}) sont déjà importées.
        \item[\textbullet] On pourra toujours librement utiliser une fonction demandée à une question précédente même si cette question n'a pas été traitée.
        \item[\textbullet] Veillez à présenter vos idées et vos réponses partielles même si vous ne trouvez pas la solution complète à une question.
        \item[\textbullet] La clarté et la lisibilité de la rédaction et des programmes sont des éléments de notation.
    \end{itemize}
}

\begin{Exercise}[title = {chercher les erreurs}]
\Question{Les fonctions ci-dessous contiennent une ou plusieurs erreurs et/ou ne respectent pas leur spécification (donnée en commentaire). Dans chaque cas, expliquer les erreurs commises et les corriger de façon à ce que la fonction obéisse à sa spécification.}

\subQuestion{Fonction {\tt harmonique}
\inputpartC{\FPATH Evaluations/DS/DS1/DS1_ex1.c}{}{\small}{5}{15}
}

\tcor{
    \textbullet\  A la ligne 6, on ne doit pas commencer à l'indice 0 (sinon on effectue une division par zéro) ni s'arrêter à l'indice ($n-1$) (on ne respecte pas la specification). La ligne correcte est donc \mintinline{c}{for (int i=1; i<n+1; i = i + 1)} \\
    \textbullet\  A la ligne 8, {\tt 1/i} est une division entière car les deux opérandes sont des entiers. Le résultat est donc 0 (même si ensuite on a une conversion implicite de type vers {\tt double} de façon à additionner avec {\tt sh}). On doit donc pour forcer la division décimale en changeant cette ligne en \mintinline{c}{sh = sh + 1.0/i;} ou encore avec une conversion explicite : \mintinline{c}{sh = sh + 1/(double)i;}
}

\subQuestion{Fonction {\tt tous\_egaux}
\inputpartC{\FPATH/Evaluations/DS/DS1/DS1_ex1.c}{}{\small}{17}{32}
}

\tcor{
    \textbullet\  la ligne 7 provoque un accès en dehors des limites du tableau car $i$ a pour valeur maximale {\tt size-1}. On doit donc modifier la lignes 5 en \mintinline{c}{for (int i=0; i<size-1; i = i + 1)}  \\
    \textbullet\  Le bloc du {\tt else} ligne 11 à 14 renvoie {\tt true} dès que les deux premiers éléments sont égaux. Il ne faut le faire qu'après avoir parcouru tout le tableau. Le {\tt if} ne devrait donc pas avoir de bloc {\tt else}, et le {\tt return true} doit être positioné après le bloc de la boucle {\tt for}
}


\subQuestion{Fonction {\tt cree\_tab\_entiers}
\inputpartC{\FPATH/Evaluations/DS/DS1/DS1_ex1.c}{}{\small}{34}{43}
}

\tcor{
    \textbullet\  le tableau {\tt tab\_entiers} de la ligne 4 est alloué sur la pile car c'est une variable locale à la fonction. On doit l'allouer sur le tas de façon à conserver cette zone mémoire en quittant la fonction. On doit donc remplacer cette ligne par \mintinline{c}{int *tab_entiers = malloc(sizeof(int)*size);}  \\
    \textbullet\  {\tt tab\_entiers} est un tableau, donc un pointeur vers l'adresse de son premier élément. On doit donc simplement écrire \mintinline{c}{return tab_entiers;}
}

\subQuestion{Fonction {\tt syracuse}
\inputpartC{\FPATH/Evaluations/DS/DS1/DS1_ex1.c}{}{\small}{45}{56}}
\tcor{
    \textbullet\  Les variables en C sont passés par valeur, donc cette fonction ne modifie pas {\tt n}. On doit si on veut respecter la spécification passer un pointeur vers {\tt n}. La ligne 2 est donc \mintinline{c}{void syracuse(int *n)} et on doit remplacer {\tt n} par {\tt *n} dans le reste de la fonction. L'appel à cette fonction si {\tt n} est déclaré comme un entier se fera avec \mintinline{c}{syracuse(&n)} de façon à passer l'adresse de {\tt n}. \\
    \textbullet\  En C, le test d'égalité est {\tt ==}, la ligne 4 est donc \mintinline{c}{if (n}{\tt \%}\mintinline{c}{2 == 0)}.
}

\Question{Les programmes suivants, produisent une erreur à l'exécution. Expliquer l'origine du problème et apporter les corrections nécessaires}
\subQuestion{Affichage d'une adresse
\inputpartC{\FPATH/Evaluations/DS/DS1/DS1_ex1.c}{}{\small}{58}{64}
}
\tcor{
    La variable {\tt p} est un pointeur \textit{non initialisé} vers un entier, aucune zone mémoire n'a été réservé pour sa valeur. On doit donc lorsqu'on déclare cette variable lui alloué un emplacement mémoire avec {\tt malloc}. La ligne 4 est donc \mintinline{c}{int *p = malloc(sizeof(int));}
}
\subQuestion{Calcul de la somme de deux entiers 
\inputpartC{\FPATH/Evaluations/DS/DS1/DS1_ex1.c}{}{\small}{55}{64}
}
\tcor{
    La fonction {\tt scanf} va modifier la valeur de la variable donnée en argument (ici {\tt a} puis {\tt b}), donc elle prend en argument des pointeurs vers ces valeurs (sinon elle ne pourrait pas les modifier), la ligne 6 est donc \tt{scanf("\%d", \&a);}}

\end{Exercise}


\begin{Exercise}[title={Pointeurs}]
	\Question{Compléter le tableau suivant, qui donne l'état des variables au fur et à mesure des instructions données dans la première colonne (on a indiqué par \faTimes{} une variable non encore déclaré.)
		\begin{center}
			\def\arraystretch{1.2}
			\setlength\tabcolsep{0.5cm}
			\begin{tabular}{|l|c|c|c|c|}
                \hline
				        instructions            & a  & b  & p & q       \\
				\hline
				{\tt int a = 14;}      & 14 & \faTimes & \faTimes & \faTimes      \\
				{\tt int b = 42;}      & 14  & \cor{42}  & \cor{\faTimes}  &  \cor{\faTimes}           \\
				{\tt int *p = \&a;}      & 14   &  \cor{42}  &   \cor{\&a}     & \cor{\faTimes}     \\
                {\tt int *q = \&b;}      & 14   &  \cor{42}  &   \cor{\&a}    & \cor{\&b}     \\
				{\tt *p = *p + *q ;} &  \cor{56} &  \cor{42}  &   \cor{\&a}    & \cor{\&b}     \\
				{\tt *q = *p - *q ;} &  \cor{56} &  \cor{14}  &   \cor{\&a}    & \cor{\&b}     \\
                {\tt *p = *p - *q ;} &  \cor{42} &  \cor{14}  &   \cor{\&a}    & \cor{\&b}     \\
                \hline
			\end{tabular}
		\end{center}}
    \Question{Ecrire une fonction en C qui prend en argument deux pointeurs vers des entiers, ne renvoie rien et échange les valeurs de ces deux entiers \textit{sans utiliser de variable temporaire}.}
    \inputpartC{\FPATH/Evaluations/DS/DS1/echange.c}{}{\small}{3}{8}
    \Question{Compléter le programme suivant en écrivant l'appel à la fonction {\tt echange} afin d'échanger les valeurs des entiers {\tt n} et {\tt m} 
    \inputpartC{\FPATH/Evaluations/DS/DS1/echange.c}{}{\small}{14}{15}}
\end{Exercise}

\begin{Exercise}[title = {puissance}]
\Question{Ecrire une fonction {\tt valeur\_absolue} qui prend en argument un entier $n$ et renvoie sa valeur absolue $|n|$. On rappelle que :
$|n| = \left\{ \begin{array}{rl} -n & \mathrm{\ si\ } n<0 \\ n & \mathrm{\ sinon} \end{array}\right.$}
\inputpartC{\FPATH/Evaluations/DS/DS1/puissance.c}{}{\small}{7}{11}
\Question{Ecrire une fonction {\tt puissance} qui prend en argument un flottant (type {\tt double}) $a$ et un entier $n$ et renvoie 
$a^n$. On rappelle que pour $a \in \R^*$, $n \in \Z$ : \\
$\left\{ \begin{array}{ll}
a^n = \underbrace{a\times \dots \times a}_{n \mathrm{\ facteurs}}  & \mathrm{\ si\ } n>0,\\
a^0 = 1, &  \\
a^n = \dfrac{1}{a^{-n}} & \mathrm{\ si\ } n<0. \\
\end{array}\right.$ \\

D'autre part $0^0=1$, $0^n=0$ si $n>0$ et les puissances négatives de zéro ne sont pas définies.On vérifiera la précondition $n>0$ lorsque $a=0$ à l'aide d'une instruction {\tt assert}.} 
\inputpartC{\FPATH/Evaluations/DS/DS1/puissance.c}{}{\small}{13}{43}
\Question{Tracer le graphe de flot de contrôle de cette fonction.}
\tcor{\vspace{6cm}
    \rput(4,-0.2){\circlenode[linecolor=red]{E}{\textcolor{red}{E}}}
    \rput(4,-1){\rnode{I}{\psframebox{\tt p=1.0}}}
    \ncline{->}{E}{I}
    \rput(4,-2){\ovalnode{T1}{{\tt a==0}}}
    \ncline{->}{I}{T1}
    \rput(6,-2){\ovalnode{T2}{{\tt n==0}}}
    \rput(9,-2){\rnode{R1}{\psframebox{\tt return 1.0}}} 
    \ncline{->}{T2}{R1} \naput[labelsep=1pt]{\small \textcolor{OliveGreen}{V}}
    \rput(9,-3){\rnode{R2}{\psframebox{\tt return 0.0}}} 
    \ncline{->}{T2}{R2} \nbput[labelsep=1pt]{\small \textcolor{BrickRed}{F}}
    \ncline{->}{T1}{T2} \naput[labelsep=1pt]{\small \textcolor{OliveGreen}{V}}
    \rput(4,-3){\rnode{I2}{\psframebox{\tt i=0}}}
    \ncline{->}{T1}{I2} \naput[labelsep=1pt]{\small \textcolor{BrickRed}{F}}
    \rput(4,-4){\ovalnode{T3}{{\tt i>=|n|}}}
    \ncline{->}{I2}{T3}
    \rput(6,-4){\ovalnode{T4}{{\tt n<0}}}
    \ncline{->}{T3}{T4} \naput[labelsep=1pt]{\small \textcolor{OliveGreen}{V}}
    \rput(9,-4){\rnode{R3}{\psframebox{\tt return 1/p}}} 
    \rput(9,-5){\rnode{R4}{\psframebox{\tt return p}}} 
    \ncline{->}{T4}{R3} \naput[labelsep=1pt]{\small \textcolor{OliveGreen}{V}}
    \ncline{->}{T4}{R4} \naput[labelsep=1pt]{\small \textcolor{BrickRed}{F}}
    \rput(4,-5.5){\rnode{M}{\psframebox{\begin{tabular}{>{\tt}l}  p = p * a \\ i = i +1 \end{tabular}}}}
    \ncline{->}{T3}{M} \naput[labelsep=1pt]{\small \textcolor{BrickRed}{F}}
    \rput(13,-3.5){\circlenode[linecolor=red]{S}{\textcolor{red}{S}}}
    \ncline{->}{R1}{S}
    \ncline{->}{R2}{S}
    \ncline{->}{R3}{S}
    \ncline{->}{R4}{S}
    \nccurve[angle=180]{->}{M}{T3}
}
\Question{Proposer un jeu de test permettant de couvrir tous les arcs.}
\tcor{Le jeu de test suivant permet de couvrir tous les arcs :
\begin{itemize}
    \item[\textbullet] $a=0$ et $n=0$
    \item[\textbullet] $a=0$ et $n=1$
    \item[\textbullet] $a=2$ et $n=3$
    \item[\textbullet] $a=2$ et $n=-3$
\end{itemize}}
\end{Exercise}


\begin{Exercise}[title={Annagrammes}]

Deux mots \textit{de même longueur} sont anagrammes l'un de l'autre lorsque l'un est formé en réarrangeant les lettres de l'autre. Par exemple :
\begin{itemize}
    \item \textit{niche} et \textit{chien} sont des anagrammes.
    \item \textit{epele} et \textit{pelle}, ne sont pas des anagrammes, en effet bien qu'ils soient formés avec les mêmes lettres, la lettre \textit{l} ne figure qu'à un seul exemplaire dans \textit{epele} et il en faut deux pour écrire \textit{pelle}.
\end{itemize}
Le but de l'exercice est d'écrire une fonction en C qui renvoie {\tt true} si les deux chaines données en argument sont des anagrammes et {\tt false} sinon. On suppose par facilité que les chaines sont constitués uniquement de lettres majuscules.

\Question{Ecrire une fonction {\tt tableau\_egaux} qui prend en argument deux tableaux ainsi que leur taille et renvoie {\tt true} lorsque ces deux tableaux ont un contenu et une longueur identique et {\tt false} sinon.}
\inputpartC{\FPATH/Evaluations/DS/DS1/anagrammes.c}{}{\small}{5}{15}

\Question{Ecrire une fonction {\tt nb\_lettres} qui prend en argument une chaine de caractères {\tt chaine} et renvoie un tableau d'entiers {\tt tab} de longueur 26 de sorte que {\tt tab[i-1]} contienne le nombre de fois où la ième lettre de l'alphabet apparait dans {\tt chaine}. Par exemple {tab[0]}, doit contenir le nombre de fois où la lettre \textit{a} apparait dans le mot.}
\inputpartC{\FPATH/Evaluations/DS/DS1/anagrammes.c}{}{\small}{17}{33}

\Question{En supposant les deux fonctions précédentes correctement écrites, on propose le code suivant pour la fonction {\tt anagrammes} qui teste si les deux chaines données en argument sont des anagrammes :
\inputpartC{\FPATH/Evaluations/DS/DS1/anagrammes.c}{}{\small}{35}{38}
Cette fonction renvoie le résultat attendu mais pose un problème, lequel ? Expliquer et proposer une correction.
}
\tcor{
    La fonction {\tt nb\_lettres} renvoie un pointeur vers une zone mémoire allouée sur le tas. Dans la fonction {\tt anagrammes}, on ne conserve pas de référence sur cette zone (elle est directement utilisée sur la fonction {\tt tab\_egaux}). Donc, cette zone mémoire n'est pas libérable par un {\tt free} puisqu'on n'a pas de référence vers elle. Chaque appel à la fonction {\tt anagrammes} entraine donc une fuite mémoire. On peut proposer la fonction  suivante comme correction, où on garde des références vers les tableaux crées par {\tt nb\_lettres} de façon à libérer l'espace mémoire alloué.
}
\inputpartC{\FPATH/Evaluations/DS/DS1/anagrammes.c}{}{\small}{40}{57}


\end{Exercise}


\begin{Exercise}[title = {tri à bulles}] \\
Le tri à bulles est un algorithme de tri qui parcourt le tableau à l'aide d'un indice $i$ de la fin vers le début. Pour chacun de ces indices $i$, on parcourt la partie du tableau allant de l'indice 0 à l'indice $i-1$ et si les deux éléments consécutifs situés aux indices $j$ et $j+1$ ne sont pas dans l'ordre croissant, on les échange. Par exemple sur le tableau {\tt \{7, 2, 9, 5, 3\}}
\begin{itemize}
    \item[\textbullet] $i = 4$ (on rappelle que $i$ parcourt de la fin vers le début)
    \begin{itemize}
        \item $j=0$, donc on échange {\tt tab[0]} et {\tt tab[1]} car ils ne sont pas dans l'ordre croisant  : {\tt \{2, 7, 9, 5, 3\}}
        \item $j=1$, pas d'échange (7 et 9 sont en ordre croissant)
        \item $j=2$, échange car 9 et 5 ne sont pas dans l'ordre croissant : {\tt \{2, 7, 5, 9, 3\}}
        \item $j=3$, échange et on obtient {\tt \{2, 7, 5, 3, 9\}}
    \end{itemize}
    \item[\textbullet] $i=3$, cette fois on parcourt avec $j=0$ jusqu'à $2$, en effet à l'étape précédente le plus grand élément du tableau se retrouve forcement en dernière position. On obtient en fin de parcours : {\tt \{2, 5, 3, 7, 9\}}
    \item[\textbullet] $i=2$, à la fin de cette itération on obtient {\tt \{2, 3, 5, 7, 9\}}
\end{itemize}

\Question{Faire fonctionner cet algorithme à la main sur le tableau {\tt \{11, 2, 5, 13, 8, 4\}} et donner l'état du tableau à la fin de chaque itération de l'indice $j$ pour $j$ variant de $0$ à $i$ en recopiant et complétant le tableau suivant : \\
\begin{tabular}{|l|p{12cm}|}
    \hline
    $i$ & valeurs contenu dans le tableau à la fin de l'itération d'indice $i$.\\
    \hline
    5  & \cor{\tt \{2,5,11,8,4,13\}} \\
    \hline
    4  & \cor{\tt \{2,5,8,4,11,13\}} \\
    \hline
    3  & \cor{\tt \{2,5,4,8,11,13\}} \\
    \hline
    2  & \cor{\tt \{2,4,5,8,11,13\}} \\
        \hline
    1  & \cor{\tt \{2,4,5,8,11,13\}} \\
    \hline
\end{tabular}
    }
\Question{Donner un exemple de tableau de longueur 5, \textit{non trié initialement} qui sera entièrement trié après le premier tour de boucle de l'indice $j$ (c'est à dire pour $i=4$).}
\tcor{Si les éléments sont déjà triés à l'exception de l'élément maximal, la seule itération de $j$ pour $j=0$ jusqu'à $i=4$, suffit à ramener cet élément en fin tableau. Donc par exemple le tableau {\tt \{ 2, 99, 3, 4, 5 \} } sera entièrement trié dès que $i=3$..}
\Question{Donner un exemple de tableau qui  ne sera trié qu'à la fin de toutes les itérations de l'indice~$i$.}
\tcor{Si le tableau a son élément minimal tout à la fin, alors cet élément se déplace vers le début du tableau une fois par itération de l'indice $i$, donc le tableau ne sera trié qu'à la toute fin de l'algorithme. On peut donner l'exemple du tableau {\tt \{ 50, 49, 48, 47, 1 \} }}
\Question{On veut maintenant écrire cet algorithme en C, en effectuant le tri dans une copie du tableau. On suppose déjà écrite la fonction {\tt echange} qui prend en argument un tableau et deux indices ne renvoie rien et échange les éléments situés à ces deux indices.}
\subQuestion{Ecrire un fonction {\tt copie\_tab} qui prend en argument un tableau ainsi que sa taille et renvoie un pointeur vers une copie de ce tableau}
\inputpartC{\FPATH/Evaluations/DS/DS1/tri_bulles.c}{}{\small}{25}{33}

\subQuestion{Ecrire une fonction {\tt tri\_bulles} qui prend en argument un tableau {\tt tab} ainsi que sa taille, ne modifie pas ce tableau et renvoie un pointeur vers un tableau contenant les éléments du tableau {\tt tab} triés dans l'ordre croissant.}
\inputpartC{\FPATH/Evaluations/DS/DS1/tri_bulles.c}{}{\small}{35}{51}

\subQuestion{Afin de tester cette fonction on a écrit le programme principal suivant :
\inputpartC{\FPATH/Evaluations/DS/DS1/tri_bulles.c}{}{\small}{53}{60}
Quelle instruction est manquante dans ce programme ? Quelle option de compilation signalerait le problème lors de l'exécution ?
}
\tcor{On doit libérer la mémoire allouée, l'instruction manquante est donc \mintinline{c}{free(tab_trie);} qu'on peut placer dès qu'on a plus besoin du tableau trié (donc ici après la boucle {\tt for} d'affichage). Le problème serait signalé lors de l'exécution si on on compile avec l'option {\tt fsanitize = address}.
}
\end{Exercise}

\end{document}
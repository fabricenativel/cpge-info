\PassOptionsToPackage{dvipsnames,table}{xcolor}
\documentclass[11pt,a4paper]{article}

\usepackage{DS}

\begin{document}

\newcommand{\ModeExercice}{
% Traduction des noms pour le package exercise
\renewcommand{\ExerciseName}{Exercice}
\renewcommand{\thesubQuestion}{\theQuestion.\alph{subQuestion}}
\renewcommand{\AnswerName}{Réponses à l'exercise }
}
\newcommand{\Fiche}[2]{\lhead{\textbf{{\sc #1}}}
\rhead{Niveau: \textbf{#2}}
\cfoot{}}
\definecolor{cfond}{gray}{0.4}
\renewcommand{\thealgocf}{}

\newcommand{\ModeActivite}{
% Traduction des noms pour le package exercise
\renewcommand{\ExerciseName}{Activité}
}
% Réglages de la mise en forme des exercices 
\renewcommand{\ExerciseHeaderTitle}{\ExerciseTitle}
\renewcommand{\ExerciseHeaderOrigin}{\ExerciseOrigin}
% Un : sépare le numéro de l'exerice de son titre ... si le titre existe. On utilise Origine pour placer les pictogrammes en fin de ligne
\renewcommand{\ExerciseHeader}{\ding{113} \textbf{\sffamily{\ExerciseName \ \ExerciseHeaderNB}} \ifthenelse{\equal{\ExerciseTitle}{\empty}}{}{:} \textit{\ExerciseHeaderTitle} \hfill \ExerciseHeaderOrigin}
\renewcommand{\ExePartHeader}{\quad {\footnotesize \ding{110}} \textbf{Partie \textbf{\ExePartHeaderNB}} : \ExePartName}


% Mode Concours
\newcommand{\ModeConcours}{
   \newcounter{qconcours}
   \setcounter{qconcours}{1}
   \renewcommand{\ExerciseName}{Exercice}
   \renewcommand{\ExerciseHeaderTitle}{\ExerciseTitle}
\renewcommand{\ExerciseHeaderOrigin}{\ExerciseOrigin}
\renewcommand{\ExerciseHeader}{\ding{113} \textbf{\sffamily{\ExerciseName \ \ExerciseHeaderNB}} \ifthenelse{\equal{\ExerciseTitle}{\empty}}{}{:} \textit{\ExerciseHeaderTitle} \hfill \ExerciseHeaderOrigin}
\renewcommand{\QuestionNB}{\textbf{Q\arabic{qconcours}--}\ \addtocounter{qconcours}{1}}}

\newcommand{\noeud}[1]{\Tr{\fbox{\tt #1}}}
\newcommand{\FPATH}{/home/fenarius/Travail/Cours/cpge-info/docs/mp2i/}
\newcommand{\spath}[2]{\FPATH Evaluations/#1/#1#2}
\definecolor{codebg}{gray}{0.90}
\newcommand{\inputpartOCaml}[5]{\begin{mdframed}[backgroundcolor=codebg] \inputminted[breaklines=true,fontsize=#3,linenos=true,highlightcolor=fluo,tabsize=2,highlightlines={#2},firstline=#4,lastline=#5,firstnumber=1]{OCaml}{#1} \end{mdframed}}
\newcommand{\inputpartPython}[5]{\begin{mdframed}[backgroundcolor=codebg] \inputminted[breaklines=true,fontsize=#3,linenos=true,highlightcolor=fluo,tabsize=2,highlightlines={#2},firstline=#4,lastline=#5,firstnumber=1]{python}{#1} \end{mdframed}}
\newcommand{\inputpartC}[5]{\begin{mdframed}[backgroundcolor=codebg] \inputminted[breaklines=true,fontsize=#3,linenos=true,highlightcolor=fluo,tabsize=2,highlightlines={#2},firstline=#4,lastline=#5,firstnumber=1]{c}{#1} \end{mdframed}}
\newcommand{\inputC}[2]{\begin{mdframed}[backgroundcolor=codebg] \inputminted[breaklines=true,fontsize=#2,linenos=true,highlightcolor=fluo,tabsize=2]{c}{#1} \end{mdframed}}
\newminted[langageC]{c}{linenos=true,escapeinside=``,highlightcolor=fluo,tabsize=2}
\newminted[python]{python}{linenos=true,escapeinside=``,highlightcolor=fluo,tabsize=4}
\BeforeBeginEnvironment{minted}{\begin{mdframed}[backgroundcolor=codebg,skipabove=0cm]}
   \AfterEndEnvironment{minted}{\end{mdframed}}

% Font light medium et bold pour tt :
\newcommand{\ttl}[1]{\ttfamily \fontseries{l}\selectfont #1}
\newcommand{\ttm}[1]{\ttfamily \fontseries{m}\selectfont #1}
\newcommand{\ttb}[1]{\ttfamily \fontseries{b}\selectfont #1}

%QCM de NSI \QNSI{Question}{R1}{R2}{R3}{R4}
\newcommand{\QNSI}[5]{
#1
\begin{enumerate}[label=\alph{enumi})]
\item #2
\item #3
\item #4
\item #5
\end{enumerate}
}



\definecolor{grispale}{gray}{0.95}
\newcommand{\htmlmode}{\lstset{language=html,numbers=left, tabsize=2, frame=single, breaklines=true, keywordstyle=\ttfamily, basicstyle=\small,
   numberstyle=\tiny\ttfamily, framexleftmargin=0mm, backgroundcolor=\color{grispale}, xleftmargin=12mm,showstringspaces=false}}
\newcommand{\pythonmode}{\lstset{language=python,numbers=left, tabsize=4, frame=single, breaklines=true, keywordstyle=\ttfamily, basicstyle=\small,
   numberstyle=\tiny\ttfamily, framexleftmargin=0mm, backgroundcolor=\color{grispale}, xleftmargin=12mm, showstringspaces=false}}
\newcommand{\bashmode}{\lstset{language=bash,numbers=left, tabsize=2, frame=single, breaklines=true, basicstyle=\ttfamily,
   numberstyle=\tiny\ttfamily, framexleftmargin=0mm, backgroundcolor=\color{grispale}, xleftmargin=12mm, showstringspaces=false}}
\newcommand{\exomode}{\lstset{language=python,numbers=left, tabsize=2, frame=single, breaklines=true, basicstyle=\ttfamily,
   numberstyle=\tiny\ttfamily, framexleftmargin=13mm, xleftmargin=12mm, basicstyle=\small, showstringspaces=false}}
   
   
  \lstset{%
        inputencoding=utf8,
        extendedchars=true,
        literate=%
        {é}{{\'{e}}}1
        {è}{{\`{e}}}1
        {ê}{{\^{e}}}1
        {ë}{{\¨{e}}}1
        {É}{{\'{E}}}1
        {Ê}{{\^{E}}}1
        {û}{{\^{u}}}1
        {ù}{{\`{u}}}1
        {ú}{{\'{u}}}1
        {â}{{\^{a}}}1
        {à}{{\`{a}}}1
        {á}{{\'{a}}}1
        {ã}{{\~{a}}}1
        {Á}{{\'{A}}}1
        {Â}{{\^{A}}}1
        {Ã}{{\~{A}}}1
        {ç}{{\c{c}}}1
        {Ç}{{\c{C}}}1
        {õ}{{\~{o}}}1
        {ó}{{\'{o}}}1
        {ô}{{\^{o}}}1
        {Õ}{{\~{O}}}1
        {Ó}{{\'{O}}}1
        {Ô}{{\^{O}}}1
        {î}{{\^{i}}}1
        {Î}{{\^{I}}}1
        {í}{{\'{i}}}1
        {Í}{{\~{Í}}}1
}

%tei pour placer les images
%tei{nom de l’image}{échelle de l’image}{sens}{texte a positionner}
%sens ="1" (droite) ou "2" (gauche)
\newlength{\ltxt}
\newcommand{\tei}[4]{
\setlength{\ltxt}{\linewidth}
\setbox0=\hbox{\includegraphics[scale=#2]{#1}}
\addtolength{\ltxt}{-\wd0}
\addtolength{\ltxt}{-10pt}
\ifthenelse{\equal{#3}{1}}{
\begin{minipage}{\wd0}
\includegraphics[scale=#2]{#1}
\end{minipage}
\hfill
\begin{minipage}{\ltxt}
#4
\end{minipage}
}{
\begin{minipage}{\ltxt}
#4
\end{minipage}
\hfill
\begin{minipage}{\wd0}
\includegraphics[scale=#2]{#1}
\end{minipage}
}
}

%Juxtaposition d'une image pspciture et de texte 
%#1: = code pstricks de l'image
%#2: largeur de l'image
%#3: hauteur de l'image
%#4: Texte à écrire
\newcommand{\ptp}[4]{
\setlength{\ltxt}{\linewidth}
\addtolength{\ltxt}{-#2 cm}
\addtolength{\ltxt}{-0.1 cm}
\begin{minipage}[b][#3 cm][t]{\ltxt}
#4
\end{minipage}\hfill
\begin{minipage}[b][#3 cm][c]{#2 cm}
#1
\end{minipage}\par
}



%Macros pour les graphiques
\psset{linewidth=0.5\pslinewidth,PointSymbol=x}
\setlength{\fboxrule}{0.5pt}
\newcounter{tempangle}

%Marque la longueur du segment d'extrémité  #1 et  #2 avec la valeur #3, #4 est la distance par rapport au segment (en %age de la valeur de celui ci) et #5 l'orientation du marquage : +90 ou -90
\newcommand{\afflong}[5]{
\pstRotation[RotAngle=#4,PointSymbol=none,PointName=none]{#1}{#2}[X] 
\pstHomO[PointSymbol=none,PointName=none,HomCoef=#5]{#1}{X}[Y]
\pstTranslation[PointSymbol=none,PointName=none]{#1}{#2}{Y}[Z]
 \ncline{|<->|,linewidth=0.25\pslinewidth}{Y}{Z} \ncput*[nrot=:U]{\footnotesize{#3}}
}
\newcommand{\afflongb}[3]{
\ncline{|<->|,linewidth=0}{#1}{#2} \naput*[nrot=:U]{\footnotesize{#3}}
}

%Construis le point #4 situé à #2 cm du point #1 avant un angle #3 par rapport à l'horizontale. #5 = liste de paramètre
\newcommand{\lsegment}[5]{\pstGeonode[PointSymbol=none,PointName=none](0,0){O'}(#2,0){I'} \pstTranslation[PointSymbol=none,PointName=none]{O'}{I'}{#1}[J'] \pstRotation[RotAngle=#3,PointSymbol=x,#5]{#1}{J'}[#4]}
\newcommand{\tsegment}[5]{\pstGeonode[PointSymbol=none,PointName=none](0,0){O'}(#2,0){I'} \pstTranslation[PointSymbol=none,PointName=none]{O'}{I'}{#1}[J'] \pstRotation[RotAngle=#3,PointSymbol=x,#5]{#1}{J'}[#4] \pstLineAB{#4}{#1}}

%Construis le point #4 situé à #3 cm du point #1 et faisant un angle de  90° avec la droite (#1,#2) #5 = liste de paramètre
\newcommand{\psegment}[5]{
\pstGeonode[PointSymbol=none,PointName=none](0,0){O'}(#3,0){I'}
 \pstTranslation[PointSymbol=none,PointName=none]{O'}{I'}{#1}[J']
 \pstInterLC[PointSymbol=none,PointName=none]{#1}{#2}{#1}{J'}{M1}{M2} \pstRotation[RotAngle=-90,PointSymbol=x,#5]{#1}{M1}[#4]
  }
  
%Construis le point #4 situé à #3 cm du point #1 et faisant un angle de  #5° avec la droite (#1,#2) #6 = liste de paramètre
\newcommand{\mlogo}[6]{
\pstGeonode[PointSymbol=none,PointName=none](0,0){O'}(#3,0){I'}
 \pstTranslation[PointSymbol=none,PointName=none]{O'}{I'}{#1}[J']
 \pstInterLC[PointSymbol=none,PointName=none]{#1}{#2}{#1}{J'}{M1}{M2} \pstRotation[RotAngle=#5,PointSymbol=x,#6]{#1}{M2}[#4]
  }

% Construis un triangle avec #1=liste des 3 sommets séparés par des virgules, #2=liste des 3 longueurs séparés par des virgules, #3 et #4 : paramètre d'affichage des 2e et 3 points et #5 : inclinaison par rapport à l'horizontale
%autre macro identique mais sans tracer les segments joignant les sommets
\noexpandarg
\newcommand{\Triangleccc}[5]{
\StrBefore{#1}{,}[\pointA]
\StrBetween[1,2]{#1}{,}{,}[\pointB]
\StrBehind[2]{#1}{,}[\pointC]
\StrBefore{#2}{,}[\coteA]
\StrBetween[1,2]{#2}{,}{,}[\coteB]
\StrBehind[2]{#2}{,}[\coteC]
\tsegment{\pointA}{\coteA}{#5}{\pointB}{#3} 
\lsegment{\pointA}{\coteB}{0}{Z1}{PointSymbol=none, PointName=none}
\lsegment{\pointB}{\coteC}{0}{Z2}{PointSymbol=none, PointName=none}
\pstInterCC{\pointA}{Z1}{\pointB}{Z2}{\pointC}{Z3} 
\pstLineAB{\pointA}{\pointC} \pstLineAB{\pointB}{\pointC}
\pstSymO[PointName=\pointC,#4]{C}{C}[C]
}
\noexpandarg
\newcommand{\TrianglecccP}[5]{
\StrBefore{#1}{,}[\pointA]
\StrBetween[1,2]{#1}{,}{,}[\pointB]
\StrBehind[2]{#1}{,}[\pointC]
\StrBefore{#2}{,}[\coteA]
\StrBetween[1,2]{#2}{,}{,}[\coteB]
\StrBehind[2]{#2}{,}[\coteC]
\tsegment{\pointA}{\coteA}{#5}{\pointB}{#3} 
\lsegment{\pointA}{\coteB}{0}{Z1}{PointSymbol=none, PointName=none}
\lsegment{\pointB}{\coteC}{0}{Z2}{PointSymbol=none, PointName=none}
\pstInterCC[PointNameB=none,PointSymbolB=none,#4]{\pointA}{Z1}{\pointB}{Z2}{\pointC}{Z1} 
}


% Construis un triangle avec #1=liste des 3 sommets séparés par des virgules, #2=liste formée de 2 longueurs et d'un angle séparés par des virgules, #3 et #4 : paramètre d'affichage des 2e et 3 points et #5 : inclinaison par rapport à l'horizontale
%autre macro identique mais sans tracer les segments joignant les sommets
\newcommand{\Trianglecca}[5]{
\StrBefore{#1}{,}[\pointA]
\StrBetween[1,2]{#1}{,}{,}[\pointB]
\StrBehind[2]{#1}{,}[\pointC]
\StrBefore{#2}{,}[\coteA]
\StrBetween[1,2]{#2}{,}{,}[\coteB]
\StrBehind[2]{#2}{,}[\angleA]
\tsegment{\pointA}{\coteA}{#5}{\pointB}{#3} 
\setcounter{tempangle}{#5}
\addtocounter{tempangle}{\angleA}
\tsegment{\pointA}{\coteB}{\thetempangle}{\pointC}{#4}
\pstLineAB{\pointB}{\pointC}
}
\newcommand{\TriangleccaP}[5]{
\StrBefore{#1}{,}[\pointA]
\StrBetween[1,2]{#1}{,}{,}[\pointB]
\StrBehind[2]{#1}{,}[\pointC]
\StrBefore{#2}{,}[\coteA]
\StrBetween[1,2]{#2}{,}{,}[\coteB]
\StrBehind[2]{#2}{,}[\angleA]
\lsegment{\pointA}{\coteA}{#5}{\pointB}{#3} 
\setcounter{tempangle}{#5}
\addtocounter{tempangle}{\angleA}
\lsegment{\pointA}{\coteB}{\thetempangle}{\pointC}{#4}
}

% Construis un triangle avec #1=liste des 3 sommets séparés par des virgules, #2=liste formée de 1 longueurs et de deux angle séparés par des virgules, #3 et #4 : paramètre d'affichage des 2e et 3 points et #5 : inclinaison par rapport à l'horizontale
%autre macro identique mais sans tracer les segments joignant les sommets
\newcommand{\Trianglecaa}[5]{
\StrBefore{#1}{,}[\pointA]
\StrBetween[1,2]{#1}{,}{,}[\pointB]
\StrBehind[2]{#1}{,}[\pointC]
\StrBefore{#2}{,}[\coteA]
\StrBetween[1,2]{#2}{,}{,}[\angleA]
\StrBehind[2]{#2}{,}[\angleB]
\tsegment{\pointA}{\coteA}{#5}{\pointB}{#3} 
\setcounter{tempangle}{#5}
\addtocounter{tempangle}{\angleA}
\lsegment{\pointA}{1}{\thetempangle}{Z1}{PointSymbol=none, PointName=none}
\setcounter{tempangle}{#5}
\addtocounter{tempangle}{180}
\addtocounter{tempangle}{-\angleB}
\lsegment{\pointB}{1}{\thetempangle}{Z2}{PointSymbol=none, PointName=none}
\pstInterLL[#4]{\pointA}{Z1}{\pointB}{Z2}{\pointC}
\pstLineAB{\pointA}{\pointC}
\pstLineAB{\pointB}{\pointC}
}
\newcommand{\TrianglecaaP}[5]{
\StrBefore{#1}{,}[\pointA]
\StrBetween[1,2]{#1}{,}{,}[\pointB]
\StrBehind[2]{#1}{,}[\pointC]
\StrBefore{#2}{,}[\coteA]
\StrBetween[1,2]{#2}{,}{,}[\angleA]
\StrBehind[2]{#2}{,}[\angleB]
\lsegment{\pointA}{\coteA}{#5}{\pointB}{#3} 
\setcounter{tempangle}{#5}
\addtocounter{tempangle}{\angleA}
\lsegment{\pointA}{1}{\thetempangle}{Z1}{PointSymbol=none, PointName=none}
\setcounter{tempangle}{#5}
\addtocounter{tempangle}{180}
\addtocounter{tempangle}{-\angleB}
\lsegment{\pointB}{1}{\thetempangle}{Z2}{PointSymbol=none, PointName=none}
\pstInterLL[#4]{\pointA}{Z1}{\pointB}{Z2}{\pointC}
}

%Construction d'un cercle de centre #1 et de rayon #2 (en cm)
\newcommand{\Cercle}[2]{
\lsegment{#1}{#2}{0}{Z1}{PointSymbol=none, PointName=none}
\pstCircleOA{#1}{Z1}
}

%construction d'un parallélogramme #1 = liste des sommets, #2 = liste contenant les longueurs de 2 côtés consécutifs et leurs angles;  #3, #4 et #5 : paramètre d'affichage des sommets #6 inclinaison par rapport à l'horizontale 
% meme macro sans le tracé des segements
\newcommand{\Para}[6]{
\StrBefore{#1}{,}[\pointA]
\StrBetween[1,2]{#1}{,}{,}[\pointB]
\StrBetween[2,3]{#1}{,}{,}[\pointC]
\StrBehind[3]{#1}{,}[\pointD]
\StrBefore{#2}{,}[\longueur]
\StrBetween[1,2]{#2}{,}{,}[\largeur]
\StrBehind[2]{#2}{,}[\angle]
\tsegment{\pointA}{\longueur}{#6}{\pointB}{#3} 
\setcounter{tempangle}{#6}
\addtocounter{tempangle}{\angle}
\tsegment{\pointA}{\largeur}{\thetempangle}{\pointD}{#5}
\pstMiddleAB[PointName=none,PointSymbol=none]{\pointB}{\pointD}{Z1}
\pstSymO[#4]{Z1}{\pointA}[\pointC]
\pstLineAB{\pointB}{\pointC}
\pstLineAB{\pointC}{\pointD}
}
\newcommand{\ParaP}[6]{
\StrBefore{#1}{,}[\pointA]
\StrBetween[1,2]{#1}{,}{,}[\pointB]
\StrBetween[2,3]{#1}{,}{,}[\pointC]
\StrBehind[3]{#1}{,}[\pointD]
\StrBefore{#2}{,}[\longueur]
\StrBetween[1,2]{#2}{,}{,}[\largeur]
\StrBehind[2]{#2}{,}[\angle]
\lsegment{\pointA}{\longueur}{#6}{\pointB}{#3} 
\setcounter{tempangle}{#6}
\addtocounter{tempangle}{\angle}
\lsegment{\pointA}{\largeur}{\thetempangle}{\pointD}{#5}
\pstMiddleAB[PointName=none,PointSymbol=none]{\pointB}{\pointD}{Z1}
\pstSymO[#4]{Z1}{\pointA}[\pointC]
}


%construction d'un cerf-volant #1 = liste des sommets, #2 = liste contenant les longueurs de 2 côtés consécutifs et leurs angles;  #3, #4 et #5 : paramètre d'affichage des sommets #6 inclinaison par rapport à l'horizontale 
% meme macro sans le tracé des segements
\newcommand{\CerfVolant}[6]{
\StrBefore{#1}{,}[\pointA]
\StrBetween[1,2]{#1}{,}{,}[\pointB]
\StrBetween[2,3]{#1}{,}{,}[\pointC]
\StrBehind[3]{#1}{,}[\pointD]
\StrBefore{#2}{,}[\longueur]
\StrBetween[1,2]{#2}{,}{,}[\largeur]
\StrBehind[2]{#2}{,}[\angle]
\tsegment{\pointA}{\longueur}{#6}{\pointB}{#3} 
\setcounter{tempangle}{#6}
\addtocounter{tempangle}{\angle}
\tsegment{\pointA}{\largeur}{\thetempangle}{\pointD}{#5}
\pstOrtSym[#4]{\pointB}{\pointD}{\pointA}[\pointC]
\pstLineAB{\pointB}{\pointC}
\pstLineAB{\pointC}{\pointD}
}

%construction d'un quadrilatère quelconque #1 = liste des sommets, #2 = liste contenant les longueurs des 4 côtés et l'angle entre 2 cotés consécutifs  #3, #4 et #5 : paramètre d'affichage des sommets #6 inclinaison par rapport à l'horizontale 
% meme macro sans le tracé des segements
\newcommand{\Quadri}[6]{
\StrBefore{#1}{,}[\pointA]
\StrBetween[1,2]{#1}{,}{,}[\pointB]
\StrBetween[2,3]{#1}{,}{,}[\pointC]
\StrBehind[3]{#1}{,}[\pointD]
\StrBefore{#2}{,}[\coteA]
\StrBetween[1,2]{#2}{,}{,}[\coteB]
\StrBetween[2,3]{#2}{,}{,}[\coteC]
\StrBetween[3,4]{#2}{,}{,}[\coteD]
\StrBehind[4]{#2}{,}[\angle]
\tsegment{\pointA}{\coteA}{#6}{\pointB}{#3} 
\setcounter{tempangle}{#6}
\addtocounter{tempangle}{\angle}
\tsegment{\pointA}{\coteD}{\thetempangle}{\pointD}{#5}
\lsegment{\pointB}{\coteB}{0}{Z1}{PointSymbol=none, PointName=none}
\lsegment{\pointD}{\coteC}{0}{Z2}{PointSymbol=none, PointName=none}
\pstInterCC[PointNameA=none,PointSymbolA=none,#4]{\pointB}{Z1}{\pointD}{Z2}{Z3}{\pointC} 
\pstLineAB{\pointB}{\pointC}
\pstLineAB{\pointC}{\pointD}
}


% Définition des colonnes centrées ou à droite pour tabularx
\newcolumntype{Y}{>{\centering\arraybackslash}X}
\newcolumntype{Z}{>{\flushright\arraybackslash}X}

%Les pointillés à remplir par les élèves
\newcommand{\po}[1]{\makebox[#1 cm]{\dotfill}}
\newcommand{\lpo}[1][3]{%
\multido{}{#1}{\makebox[\linewidth]{\dotfill}
}}

%Liste des pictogrammes utilisés sur la fiche d'exercice ou d'activités
\newcommand{\bombe}{\faBomb}
\newcommand{\livre}{\faBook}
\newcommand{\calculatrice}{\faCalculator}
\newcommand{\oral}{\faCommentO}
\newcommand{\surfeuille}{\faEdit}
\newcommand{\ordinateur}{\faLaptop}
\newcommand{\ordi}{\faDesktop}
\newcommand{\ciseaux}{\faScissors}
\newcommand{\danger}{\faExclamationTriangle}
\newcommand{\out}{\faSignOut}
\newcommand{\cadeau}{\faGift}
\newcommand{\flash}{\faBolt}
\newcommand{\lumiere}{\faLightbulb}
\newcommand{\compas}{\dsmathematical}
\newcommand{\calcullitteral}{\faTimesCircleO}
\newcommand{\raisonnement}{\faCogs}
\newcommand{\recherche}{\faSearch}
\newcommand{\rappel}{\faHistory}
\newcommand{\video}{\faFilm}
\newcommand{\capacite}{\faPuzzlePiece}
\newcommand{\aide}{\faLifeRing}
\newcommand{\loin}{\faExternalLink}
\newcommand{\groupe}{\faUsers}
\newcommand{\bac}{\faGraduationCap}
\newcommand{\histoire}{\faUniversity}
\newcommand{\coeur}{\faSave}
\newcommand{\os}{\faMicrochip}
\newcommand{\rd}{\faCubes}
\newcommand{\data}{\faColumns}
\newcommand{\web}{\faCode}
\newcommand{\prog}{\faFile}
\newcommand{\algo}{\faCogs}
\newcommand{\important}{\faExclamationCircle}
\newcommand{\maths}{\faTimesCircle}
% Traitement des données en tables
\newcommand{\tables}{\faColumns}
% Types construits
\newcommand{\construits}{\faCubes}
% Type et valeurs de base
\newcommand{\debase}{{\footnotesize \faCube}}
% Systèmes d'exploitation
\newcommand{\linux}{\faLinux}
\newcommand{\sd}{\faProjectDiagram}
\newcommand{\bd}{\faDatabase}

%Les ensembles de nombres
\renewcommand{\N}{\mathbb{N}}
\newcommand{\D}{\mathbb{D}}
\newcommand{\Z}{\mathbb{Z}}
\newcommand{\Q}{\mathbb{Q}}
\newcommand{\R}{\mathbb{R}}
\newcommand{\C}{\mathbb{C}}

%Ecriture des vecteurs
\newcommand{\vect}[1]{\vbox{\halign{##\cr 
  \tiny\rightarrowfill\cr\noalign{\nointerlineskip\vskip1pt} 
  $#1\mskip2mu$\cr}}}


%Compteur activités/exos et question et mise en forme titre et questions
\newcounter{numact}
\setcounter{numact}{1}
\newcounter{numseance}
\setcounter{numseance}{1}
\newcounter{numexo}
\setcounter{numexo}{0}
\newcounter{numprojet}
\setcounter{numprojet}{0}
\newcounter{numquestion}
\newcommand{\espace}[1]{\rule[-1ex]{0pt}{#1 cm}}
\newcommand{\Quest}[3]{
\addtocounter{numquestion}{1}
\begin{tabularx}{\textwidth}{X|m{1cm}|}
\cline{2-2}
\textbf{\sffamily{\alph{numquestion})}} #1 & \dots / #2 \\
\hline 
\multicolumn{2}{|l|}{\espace{#3}} \\
\hline
\end{tabularx}
}
\newcommand{\mq}[1]
{\ding{113} \addtocounter{numquestion}{1}
\textbf{Question \arabic{numquestion}} \\ #1}
\newcommand{\QuestR}[3]{
\addtocounter{numquestion}{1}
\begin{tabularx}{\textwidth}{X|m{1cm}|}
\cline{2-2}
\textbf{\sffamily{\alph{numquestion})}} #1 & \dots / #2 \\
\hline 
\multicolumn{2}{|l|}{\cor{#3}} \\
\hline
\end{tabularx}
}
\newcommand{\Pre}{{\sc nsi} 1\textsuperscript{e}}
\newcommand{\Term}{{\sc nsi} Terminale}
\newcommand{\Sec}{2\textsuperscript{e}}
\newcommand{\Exo}[2]{ \addtocounter{numexo}{1} \ding{113} \textbf{\sffamily{Exercice \thenumexo}} : \textit{#1} \hfill #2  \setcounter{numquestion}{0}}
\newcommand{\Projet}[1]{ \addtocounter{numprojet}{1} \ding{118} \textbf{\sffamily{Projet \thenumprojet}} : \textit{#1}}
\newcommand{\ExoD}[2]{ \addtocounter{numexo}{1} \ding{113} \textbf{\sffamily{Exercice \thenumexo}}  \textit{(#1 pts)} \hfill #2  \setcounter{numquestion}{0}}
\newcommand{\ExoB}[2]{ \addtocounter{numexo}{1} \ding{113} \textbf{\sffamily{Exercice \thenumexo}}  \textit{(Bonus de +#1 pts maximum)} \hfill #2  \setcounter{numquestion}{0}}
\newcommand{\Act}[2]{ \ding{113} \textbf{\sffamily{Activité \thenumact}} : \textit{#1} \hfill #2  \addtocounter{numact}{1} \setcounter{numquestion}{0}}
\newcommand{\Seance}{ \rule{1.5cm}{0.5pt}\raisebox{-3pt}{\framebox[4cm]{\textbf{\sffamily{Séance \thenumseance}}}}\hrulefill  \\
  \addtocounter{numseance}{1}}
\newcommand{\Acti}[2]{ {\footnotesize \ding{117}} \textbf{\sffamily{Activité \thenumact}} : \textit{#1} \hfill #2  \addtocounter{numact}{1} \setcounter{numquestion}{0}}
\newcommand{\titre}[1]{\begin{Large}\textbf{\ding{118}}\end{Large} \begin{large}\textbf{ #1}\end{large} \vspace{0.2cm}}
\newcommand{\QListe}[1][0]{
\ifthenelse{#1=0}
{\begin{enumerate}[partopsep=0pt,topsep=0pt,parsep=0pt,itemsep=0pt,label=\textbf{\sffamily{\arabic*.}},series=question]}
{\begin{enumerate}[resume*=question]}}
\newcommand{\SQListe}[1][0]{
\ifthenelse{#1=0}
{\begin{enumerate}[partopsep=0pt,topsep=0pt,parsep=0pt,itemsep=0pt,label=\textbf{\sffamily{\alph*)}},series=squestion]}
{\begin{enumerate}[resume*=squestion]}}
\newcommand{\SQListeL}[1][0]{
\ifthenelse{#1=0}
{\begin{enumerate*}[partopsep=0pt,topsep=0pt,parsep=0pt,itemsep=0pt,label=\textbf{\sffamily{\alph*)}},series=squestion]}
{\begin{enumerate*}[resume*=squestion]}}
\newcommand{\FinListe}{\end{enumerate}}
\newcommand{\FinListeL}{\end{enumerate*}}

%Mise en forme de la correction
\newboolean{corrige}
\setboolean{corrige}{false}
\newcommand{\scor}[1]{\par \textcolor{blue!75!black}{\small #1}}
\newcommand{\cor}[1]{\par \textcolor{blue!75!black}{#1}}
\newcommand{\br}[1]{\cor{\textbf{#1}}}
\newcommand{\tcor}[1]{
\ifthenelse{\boolean{corrige}}{\begin{tcolorbox}[width=\linewidth,colback={white},colbacktitle=white,coltitle=green!50!black,colframe=green!50!black,boxrule=0.2mm]   
\cor{#1}
\end{tcolorbox}}{}
}
\newcommand{\iscor}[1]{\ifthenelse{\boolean{corrige}}{#1}}
\newcommand{\rc}[1]{\textcolor{OliveGreen}{#1}}

%Référence aux exercices par leur numéro
\newcommand{\refexo}[1]{
\refstepcounter{numexo}
\addtocounter{numexo}{-1}
\label{#1}}

%Séparation entre deux activités
\newcommand{\separateur}{\begin{center}
\rule{1.5cm}{0.5pt}\raisebox{-3pt}{\ding{117}}\rule{1.5cm}{0.5pt}  \vspace{0.2cm}
\end{center}}

%Entête et pied de page
\newcommand{\snt}[1]{\lhead{\textbf{SNT -- La photographie numérique} \rhead{\textit{Lycée Nord}}}}
\newcommand{\Activites}[2]{\lhead{\textbf{{\sc #1}}}
\rhead{Activités -- \textbf{#2}}
\cfoot{}}
\newcommand{\Exos}[2]{\lhead{\textbf{Fiche d'exercices: {\sc #1}}}
\rhead{Niveau: \textbf{#2}}
\cfoot{}}
\newcommand{\TD}[2]{\lhead{\textbf{TD #1} : {\sc #2} }
\rhead{{\sc mp2i -- Lycée Leconte de Lisle}}
\cfoot{}}
\newcommand{\Colles}[2]{\lhead{{\sc mp2i -- }\textbf{Colles d'informatique #1}} 
\rhead{{\sc #2}}
\cfoot{}}
\newcommand{\Devoir}[2]{\lhead{\textbf{Devoir de mathématiques : {\sc #1}}}
\rhead{\textbf{#2}} \setlength{\fboxsep}{8pt}
\begin{center}
%Titre de la fiche
\fbox{\parbox[b][1cm][t]{0.3\textwidth}{Nom : \hfill \po{3} \par \vfill Prénom : \hfill \po{3}} } \hfill 
\fbox{\parbox[b][1cm][t]{0.6\textwidth}{Note : \po{1} / 20} }
\end{center} \cfoot{}}
\newcommand{\TPnote}[3]{\lhead{\textbf{TP noté d'informatique n° #1}}
\rhead{\textbf{#2}} \setlength{\fboxsep}{8pt}
\ifthenelse{\boolean{corrige}}{}
{\begin{center}
\fbox{\parbox[b][1cm][t]{0.3\textwidth}{Nom : \hfill \po{3} \par \vfill Prénom : \hfill \po{3}} } \hfill 
\fbox{\parbox[b][1cm][t]{0.6\textwidth}{Note : \po{1} / #3} }
\end{center}} \cfoot{}}
\newcommand{\IC}[2]{\lhead{\textbf{Interro de cours n° #1}}
\rhead{{\sc mp2i --} \textbf{#2}} \setlength{\fboxsep}{8pt}
\ifthenelse{\boolean{corrige}}{}
{\begin{center}
%Titre de la fiche
\fbox{\parbox[b][1cm][t]{0.3\textwidth}{Nom : \hfill \po{3} \par \vfill Prénom : \hfill \po{3}} } \hfill 
\fbox{\parbox[b][1cm][t]{0.6\textwidth}{Note : \po{1} / 10} }
\end{center}}\cfoot{}}
\newcommand{\DS}[3]{\lhead{{#1} : \textbf{DS d'informatique n° #2}}
\rhead{Lycée Leconte de Lisle -- #3} \setlength{\fboxsep}{8pt}
%Titre de la fiche
\begin{center}
   {\Large \textbf{Devoir surveillé d'informatique}}
\end{center} \cfoot{\thepage/\pageref{LastPage}}}

\newcommand{\CB}[2]{\lhead{{#1} : \textbf{Councours blanc - Informatique}}
\rhead{Lycée Leconte de Lisle -- #2} \setlength{\fboxsep}{8pt}
%Titre de la fiche
\begin{center}
   {\Large \textbf{Concours Blanc - Epreuve d'informatique}}
\end{center} \cfoot{\thepage/\pageref{LastPage}}}

\newcommand{\PC}[3]{\lhead{Concours {#1} -- #2}
\rhead{Lycée Leconte de Lisle} \setlength{\fboxsep}{8pt}
%Titre de la fiche
\begin{center}
   {\Large \textbf{Proposition de corrigé}}
\end{center} \cfoot{\thepage/\pageref{LastPage}}}

\newcounter{numdspart}
\setcounter{numdspart}{1}
\newcommand{\DSPart}{\bigskip
   \hrulefill\raisebox{-3pt}{\framebox[4cm]{\textbf{\textbf{Partie \thenumdspart}}}}\hrulefill
   \addtocounter{numdspart}{1}
   \bigskip}

\newcommand{\Sauvegarde}[1]{
   \begin{tcolorbox}[title=\textcolor{black}{\danger\; Attention},colbacktitle=lightgray]
      {Tous vos programmes doivent être enregistrés dans votre dossier personnel, dans {\tt Evaluations}{\tt \textbackslash}{\tt #1}
      }
   \end{tcolorbox}
}

\newcommand{\alertbox}[3]{
   \begin{tcolorbox}[title=\textcolor{black}{#1\; #2},colbacktitle=lightgray]
      {#3}
   \end{tcolorbox}
}

%Devoir programmation en NSI (pas à rendre sur papier)
\newcommand{\PNSI}[2]{\lhead{\textbf{Devoir de {\sc nsi} : \textsf{ #1}}
}
\rhead{\textbf{#2}} \setlength{\fboxsep}{8pt}
 \cfoot{}
 \begin{center}{\Large \textbf{Evaluation de {\sc nsi}}}\end{center}}


%Devoir de NSI
\newcommand{\DNSI}[2]{\lhead{\textbf{Devoir de {\sc nsi} : \textsf{ #1}}}
\rhead{\textbf{#2}} \setlength{\fboxsep}{8pt}
\begin{center}
%Titre de la fiche
\fbox{\parbox[b][1cm][t]{0.3\textwidth}{Nom : \hfill \po{3} \par \vfill Prénom : \hfill \po{3}} } \hfill 
\fbox{\parbox[b][1cm][t]{0.6\textwidth}{Note : \po{1} / 10} }
\end{center} \cfoot{}}

\newcommand{\DevoirNSI}[2]{\lhead{\textbf{Devoir de {\sc nsi} : {\sc #1}}}
\rhead{\textbf{#2}} \setlength{\fboxsep}{8pt}
\cfoot{}}

%La définition de la commande QCM pour auto-multiple-choice
%En premier argument le sujet du qcm, deuxième argument : la classe, 3e : la durée prévue et #4 : présence ou non de questions avec plusieurs bonnes réponses
\newcommand{\QCM}[4]{
{\large \textbf{\ding{52} QCM : #1}} -- Durée : \textbf{#3 min} \hfill {\large Note : \dots/10} 
\hrule \vspace{0.1cm}\namefield{}
Nom :  \textbf{\textbf{\nom{}}} \qquad \qquad Prénom :  \textbf{\prenom{}}  \hfill Classe: \textbf{#2}
\vspace{0.2cm}
\hrule  
\begin{itemize}[itemsep=0pt]
\item[-] \textit{Une bonne réponse vaut un point, une absence de réponse n'enlève pas de point. }
\item[\danger] \textit{Une mauvaise réponse enlève un point.}
\ifthenelse{#4=1}{\item[-] \textit{Les questions marquées du symbole \multiSymbole{} peuvent avoir plusieurs bonnes réponses possibles.}}{}
\end{itemize}
}
\newcommand{\DevoirC}[2]{
\renewcommand{\footrulewidth}{0.5pt}
\lhead{\textbf{Devoir de mathématiques : {\sc #1}}}
\rhead{\textbf{#2}} \setlength{\fboxsep}{8pt}
\fbox{\parbox[b][0.4cm][t]{0.955\textwidth}{Nom : \po{5} \hfill Prénom : \po{5} \hfill Classe: \textbf{1}\textsuperscript{$\dots$}} } 
\rfoot{\thepage} \cfoot{} \lfoot{Lycée Nord}}
\newcommand{\DevoirInfo}[2]{\lhead{\textbf{Evaluation : {\sc #1}}}
\rhead{\textbf{#2}} \setlength{\fboxsep}{8pt}
 \cfoot{}}
\newcommand{\DM}[2]{\lhead{\textbf{Devoir maison à rendre le #1}} \rhead{\textbf{#2}}}

%Macros permettant l'affichage des touches de la calculatrice
%Touches classiques : #1 = 0 fond blanc pour les nombres et #1= 1gris pour les opérations et entrer, second paramètre=contenu
%Si #2=1 touche arrondi avec fond gris
\newcommand{\TCalc}[2]{
\setlength{\fboxsep}{0.1pt}
\ifthenelse{#1=0}
{\psframebox[fillstyle=solid, fillcolor=white]{\parbox[c][0.25cm][c]{0.6cm}{\centering #2}}}
{\ifthenelse{#1=1}
{\psframebox[fillstyle=solid, fillcolor=lightgray]{\parbox[c][0.25cm][c]{0.6cm}{\centering #2}}}
{\psframebox[framearc=.5,fillstyle=solid, fillcolor=white]{\parbox[c][0.25cm][c]{0.6cm}{\centering #2}}}
}}
\newcommand{\Talpha}{\psdblframebox[fillstyle=solid, fillcolor=white]{\hspace{-0.05cm}\parbox[c][0.25cm][c]{0.65cm}{\centering \scriptsize{alpha}}} \;}
\newcommand{\Tsec}{\psdblframebox[fillstyle=solid, fillcolor=white]{\parbox[c][0.25cm][c]{0.6cm}{\centering \scriptsize 2nde}} \;}
\newcommand{\Tfx}{\psdblframebox[fillstyle=solid, fillcolor=white]{\parbox[c][0.25cm][c]{0.6cm}{\centering \scriptsize $f(x)$}} \;}
\newcommand{\Tvar}{\psframebox[framearc=.5,fillstyle=solid, fillcolor=white]{\hspace{-0.22cm} \parbox[c][0.25cm][c]{0.82cm}{$\scriptscriptstyle{X,T,\theta,n}$}}}
\newcommand{\Tgraphe}{\psdblframebox[fillstyle=solid, fillcolor=white]{\hspace{-0.08cm}\parbox[c][0.25cm][c]{0.68cm}{\centering \tiny{graphe}}} \;}
\newcommand{\Tfen}{\psdblframebox[fillstyle=solid, fillcolor=white]{\hspace{-0.08cm}\parbox[c][0.25cm][c]{0.68cm}{\centering \tiny{fenêtre}}} \;}
\newcommand{\Ttrace}{\psdblframebox[fillstyle=solid, fillcolor=white]{\parbox[c][0.25cm][c]{0.6cm}{\centering \scriptsize{trace}}} \;}

% Macroi pour l'affichage  d'un entier n dans  une base b
\newcommand{\base}[2]{ \overline{#1}^{#2}}
% Intervalle d'entiers
\newcommand{\intN}[2]{\llbracket #1; #2 \rrbracket}
% Cadre avec lignes réponses
\def\gaddtotok#1{\global\tabtok\expandafter{\the\tabtok#1}}
\newtoks\tabtok
\newcommand*\reponse[2]{%
   \ifthenelse{\boolean{corrige}}{}{
	\global\tabtok{\\ \renewcommand{\arraystretch}{1.4}\begin{tabularx}{\linewidth}{|X|p{1cm}|}\hline \dotfill & \cellcolor{gray!30}{\small \dots/#2} \\ \cline{2-2}}%
	\multido{}{#1}{\gaddtotok{ \multicolumn{2}{|>{\hsize=\dimexpr1\hsize+2\tabcolsep+\arrayrulewidth+1cm\relax}X|}{\dotfill}\\ }}%
	\gaddtotok{\hline \end{tabularx}}%
	\the\tabtok
   }}

\newcommand{\PE}[1]{\left \lfloor #1 \right \lfloor}}
\ModeConcours
\DS{MP2I}{5}{Avril 2025}

\setboolean{corrige}{false}
\newcommand{\SPATH}{/home/fenarius/Travail/Cours/cpge-info/docs/mp2i/files/C12}

\newcommand{\maillon}[3]{
	\begin{tabular}{|p{0.2cm}|p{0.2cm}|}
		\hline
		\rnode{#2}{#1} & \rnode{#3}{\phantom{$e_0$}} \\
		\hline
	\end{tabular}
}

\alertbox{\danger}{Consignes}{
	\begin{itemize}
		\item[\textbullet] Les programmes demandés doivent être écrits en C ou en OCaml. Dans le cas du C, on suppose que les librairies standards usuelles ({\tt <stdio.h>}, {\tt <stdlib.h>}, {\tt <stdbool.h>}, {\tt <stdassert.h>}, \dots) sont déjà importées.
		\item[\textbullet] On pourra toujours librement utiliser une fonction demandée à une question précédente même si cette question n'a pas été traitée.
		\item[\textbullet] Veillez à présenter vos idées et vos réponses partielles même si vous ne trouvez pas la solution complète à une question.
		\item[\textbullet] La clarté et la lisibilité de la rédaction et des programmes sont des éléments de notation.
	\end{itemize}
}

\psset{arrows=->,treesep=0.8cm,levelsep=0.8cm, radius=0.3cm}
\begin{Exercise}[title={Questions de cours : terminaison}]\\
	On considère la fonction d'Ackerman $a : \N \times \N \mapsto \N$ définie par par : \\
	$\left\{
		\begin{array}{lll}
			a(0, m) & = & m + 1                                             \\
			a(n, 0) & = & a(n-1,1) \text{ si } n>0                          \\
			a(n,m)  & = & a(n-1, a(n, m-1)) \text{ si } n>0 \text{ et } m>0 \\
		\end{array}
		\right.$
	\Question{Calculer $a(1,2)$}
  \tcor{$\begin{array}{lcl}
    a(1,2) & = & a(0, a(1, 1)) \\
           & = & a(0, a(0, a(1, 0))) \\
           & = & a(0, a(0, a(0, 1))) \\
           & = & a(0, a(0, 2)) \\
           & = & a(0, 3) \\
           & = & 4
  \end{array}$}
	\Question{Ecrire en OCaml une fonction {\tt ack: int -> int -> int} qui prend en argument deux entiers positifs $n$ et $m$ et renvoie $a(n,m)$.}
  \ifcorrige
  \corpartOCaml{ack.ml}{}{}{1}{5}
  \fi
	\Question{Prouver la terminaison de la fonction {\tt ack} on précisera soigneusement le variant et la relation d'ordre bien fondée utilisée.}
  \tcor{On considère l'ordre lexicographique sur $N^2$ noté $\preccurlyeq_L$, montrons que $(n,m)$ est un variant .
  \begin{itemize}
    \item si $m=0$ alors on effectue un appel récursif avec $(n-1,1)$ et comme $(n-1,1) \preccurlyeq_L (n,0)$ la quantité $(n,m)$ décroit strictement.
    \item sinon, on effectue un premier appel récursif avec $(n, m-1)$ et comme $(n, m-1) \preccurlyeq_L (n,m)$ la quantité $(n,m)$ décroit strictement. Le second appel récursif s'effectue avec $(n-1, a(n,m-1))$ qui est strictement inférieur à $(n,m)$. 
  \end{itemize}
  Dans tous les cas $(n,m)$ décroit strictement à chaque appel récursif, c'est donc un variant et puisque $(N^2,\preccurlyeq)$ est un ensemble bien fondé, la fonction {\tt ack} termine.}
\end{Exercise}

\begin{Exercise}[title={Questions de cours : preuve par induction structurelle}]
	\Question{Donner la définition inductive des arbres binaires sur un ensemble d'étiquettes $E$.}
  \tcor{
    On définit inductivement l'ensemble des arbres binaires sur un ensemble d'étiquettes $E$ avec :
    \begin{itemize}
    \item L'ensemble d'axiomes $X_0 = \{\varnothing\}$ où $\varnothing$ est l'arbre vide.
    \item La règle d'inférence d'arité 2 : $ r : (g,d) \rightarrow (g,e,d)$ où $e \in E$.
    \end{itemize}
  }
	\Question{Ecrire en OCaml un type {\tt arbrebin} représentant un arbre binaire en utilisant un type paramétré {\tt 'a} pour l'ensemble des étiquettes.}
  \ifcorrige
  \corpartOCaml{ack.ml}{}{}{7}{9}
  \fi
	\Question{Rappeler la définition de la hauteur et de la taille d'un arbre binaire.}
  \tcor{
    \begin{itemize}
			\item[\textbullet] Le nombre de noeuds d'un arbre binaire $A$, noté $n(A)$, se définit récursivement par :\\
				$ \left\{
					\begin{array}{llll}
						n(A) & = & 0               & \text{si $A$ est vide}  \\
						n(A) & = & 1 + n(g) + n(d) & \text{si $A = r(g,d)$} \\
					\end{array}
					\right.
				$
			\item[\textbullet] La hauteur d'un arbre binaire $A$, noté $h(A)$, se définit récursivement par : \\
				$\left\{
					\begin{array}{llll}
						h(A) & = & -1                  & \text{si $A$ est vide}  \\
						h(A) & = & 1 + \max(h(g),h(d)) & \text{si $A = r(g,d)$} \\
					\end{array}
					\right.
				$
		\end{itemize}
  }
  \Question{Prouver \textit{par induction structurelle} que la taille d'un arbre binaire de hauteur $h$ est inférieure ou égale à $2^{h+1}-1$.}
  \tcor{Soit $A$ un arbre binaire de hauteur $n$ et de taille $h$ on note $\mathcal{P}(A)$ la propriété $n \leqslant 2^{h+1}-1$. Montrons par induction structurelle que $P$ est vraie pour tous les arbres binaires.
  \begin{itemize}
    \item $\mathcal{P}$ est vraie pour tout axiome, en effet il y a un seul axiome, l'arbre vide qui par définition est de taille 0 et de hauteur $-1$ et on a bien $2^{-1 + 1} -1 \geqslant 0 $.
    \item Supposons que $\mathcal{P}$ est vraie pour deux arbres $g$ et $d$ et montrons qu'alors $\mathcal{P}(r(g,d))$ est vraie, c'est à dire la conservation de $\mathcal{P}$ par application de l'unique règle d'inférence. \\
    $\begin{array}{lcll}
      n & = & n(g) + n(d) + 1 & \text{par définition de la taille de }r(g,d)   \\
      n & \leqslant & 2^{h(g) + 1} -1 + 2^{h(g) + 1} -1 + 1  & \text{car $P(g)$ et $P(d)$ sont vraies} \\
      n & \leqslant & 2^{\max(h(g),h(d)) + 1} -1 + 2^{\max(h(g),h(d)) + 1} -1 + 1  &  \\
      n & \leqslant & 2^{\max(h(g),h(d)) + 2} - 1 &  \\
      n & \leqslant & 2^{h+1} -1
    \end{array}$
    Donc par induction structurelle,  $\mathcal{P}$ est vraie pour tout arbre binaire.
  \end{itemize}

  }
\end{Exercise}


\begin{Exercise}[title = {Recherche des $k$ premiers maximums d'une liste}] \\
	Les fonctions demandées dans cet exercice doivent être écrites en langage OCaml.\smallskip \\
	On s'intéresse au problème de la recherche des $k$ premiers maximums d'une liste de $n$ entiers. Dans toute la suite de l'exercice on supposera que la liste est \textit{non vide} : $n>0$ et qu'on extrait moins de maximums qu'il n'y a d'éléments dans la liste c'est à dire que $k \leqslant n$. On cherche donc à écrire une fonction {\tt kmax : int list -> int -> int list}  qui renvoie la liste des {\tt k} premiers maximums de la liste donnée en argument. Par exemples :
	\begin{itemize}
		\item {\tt kmax [2; 5; 1; 8; 3; 0; 4] 2} renvoie {\tt [8, 5]}
		\item {\tt kmax [7; 8; 8; 1; 6; 3; 2; 9] 3} renvoie {\tt [9; 8; 8]}
		\item {\tt kmax [1; 0; 1; 2; 4]} 4 renvoie {\tt [4; 2; 1; 1]}
	\end{itemize}
	Les trois parties sont indépendantes et dans chacune d'elle on propose un algorithme différent afin d'écrire la fonction {\tt kmax}.\medskip

	\ExePart[name = {Résolution par recherche successive des maximums}]
	\Question{Ecrire une fonction {\tt max\_reste : int list -> int * int list} qui prend en argument une liste  et renvoie le couple composé du maximum de cette liste et de cette liste privée d'un de ses maximums. Par exemple {\tt max\_reste [2; 6; 4; 6; 5]} renvoie {\tt (6, [2; 4; 6; 5])}. On pourra procéder par correspondance de motif et traiter le cas de la liste vide par un {\tt failwith}}.
  \ifcorrige
  \corpartOCaml{kmax.ml}{}{}{3}{7}
  \fi
	\Question{Donner en la justifiant brièvement la complexité de la fonction {\tt max\_reste}.}
  \tcor{Le fonction {\tt max\_reste} a une complexité linéaire en fonction de la taille $n$ de la liste car elle effectue $n-1$ appels récursifs et chacun de ces appels n'effectue que des opérations élémentaires.}
	\Question{Ecrire une première version de la fonction {\tt kmax} qui procède par extraction successive des {\tt k} premiers maximums en utilisant la fonction {\tt max\_reste}. On procèdera de façon récursive sans utiliser les aspects impératifs de OCaml.}
  \ifcorrige
  \corpartOCaml{kmax.ml}{}{}{9}{14}
  \fi
	\Question{Quelle est la complexité de cette version de la fonction {\tt kmax} en fonction du nombre $k$ de maximums à extraire et de la longueur $n$ de la liste?}\medskip 
  \tcor{La fonction {\tt kmax} effectue un appel récursif à {\tt max\_reste} pour chacun des $k$ maximums à extraire et comme {\tt max\_reste} a une complexité en $\mathcal{O}(n)$, cette version de {\tt kmax} a une complexité en $\mathcal{O}(kn)$.}
	\ExePart[name = {Résolution par un tri}]
	\Question{Ecrire une fonction {\tt kpremiers int list -> int -> int list} qui prend en argument une liste {\tt lst} et un entier {\tt k} et renvoie la liste composée des {\tt k} premiers éléments de {\tt lst}. Par exemple {\tt kpremiers [2; 7; 1; 8; 5] 3} renvoie la liste {\tt [2; 7; 1]}. On procédera de façon récursive sans utiliser les apsects impératifs de OCaml.}
  \ifcorrige
  \corpartOCaml{kmax.ml}{}{}{16}{20}
  \fi
	\Question{On propose d'écrire la fonction {\tt kmax} en triant la liste  par ordre décroissant puis en prenant ses $k$ premiers éléments. En supposant que l'algorithme de tri utilisé a une complexité en $\mathcal{O}(n\log n)$, donner la complexité de ce nouvel algorithme (on ne demande \textit{pas} de le programmer).}\medskip
  \tcor{La fonction {\tt kmax} effectue un appel à la fonction de tri qui a une complexité en $\mathcal{O}(n\log n)$ et ensuite elle effectue un appel à la fonction {\tt kpremiers} qui a une complexité en $\mathcal{O}(k)$. Donc la complexité de cette version de {\tt kmax} est en $\mathcal{O}(n\log n + k)$ et comme $k <n$ la complexité est en $\mathcal{O}(n\log n)$.}
	\ExePart[name = {Résolution en utilisant un tas}]\\
	Dans la suite on suppose que la structure de données de tas d'entiers (type {\tt int}), est \textit{déjà implémentée} par un type {\tt tas} sur lequel on dispose des fonctions suivantes :
	\begin{itemize}
		\item {\tt cree\_tas : int -> tas} qui prend en argument un entier {\tt cap} et renvoie un tas binaire vide de capacité maximale {\tt cap}.
		\item {\tt donne\_taille : tas -> int} qui prend en argument un tas et renvoie sa taille (le nombre d'éléments actuellement stocké dans le tas).
		\item {\tt insere : int -> tas -> unit} qui insère une nouvelle valeur dans le tas. Cette fonction échoue lorsque le tas est plein.
		\item {\tt donne\_min : tas -> int} qui renvoie la valeur minimale contenu dans le tas  \textit{sans modifier le tas}. Cette fonction échoue lorsque le tas est vide.
		\item {\tt extrait\_min : tas -> int} qui renvoie, \textit{en le supprimant du tas} le minimum du tas. Cette fonction échoue lorsque le tas est vide.
	\end{itemize}
	\Question{Rappeler en les justifiant, les complexités des opérations {\tt insere} et {\tt extrait\_min} en fonction de la taille du tas notée {\tt k} si on suppose que l'implémentation de la structure de tas est réalisée grâce à un tableau.}\medskip
	\tcor{Un tas est un arbre binaire complet et à chaque étape d'une insertion ou d'une extraction, on remonte (ou on descend) d'un niveau dans cet arbre, la complexité de ces opérations est donc en $O(h)$ où $h$ est la  hauteur de l'arbre or l'arbre étant complet $\mathcal{O}(h)=\mathcal{O}(\log k)$ où $k$ est la taille de l'arbre. En conclusion, {\tt insere} et {\tt extrait\_min} ont une complexité en $\mathcal{O}(k)$}

	Afin d'extraire les {\tt k} premiers éléments d'une liste de taille {\tt n}, on propose créer un tas de taille {\tt k} puis  de parcourir récursivement la liste, pour chaque élément rencontré :
	\begin{itemize}
		\item si le tas n'est pas plein  on y insère l'élément.
		\item sinon,  on compare l'élément avec le minimum du tas, s'il est plus grand on extrait le minimum du tas et on insère l'élément dans le tas.
	\end{itemize}
	Par exemple, si on veut extraire les 3 premiers maximums de la liste {\tt [4; 6; 2; 8; 3; 7; 1; 9; 5]}, après l'insertion des trois premiers éléments, le tas est : \\
	\pstree[arrows=->,treesep=0.8cm,levelsep=1cm]{\Tcircle{2}}{\Tcircle{6} \Tcircle{4}} \\
	A l'étape suivante, 8 étant plus grand que 2 (le minimum du tas), on extrait 2 du tas et on y insère 8 ce qui donne : \\
	\pstree[arrows=->,treesep=0.8cm,levelsep=1cm]{\Tcircle{4}}{\Tcircle{6} \Tn{}} \quad \pstree[arrows=->,treesep=0.8cm,levelsep=1cm]{\Tcircle{4}}{\Tcircle{6} \Tcircle{8}}  \\
	\psset{arrows=->,treesep=0.8cm,levelsep=1cm}
	\Question{Poursuivre le déroulement de cet algorithme en faisant figurer comme ci-dessus les étapes de l'évolution du tas.}
	\tcor{
		\begin{itemize}
			\item On traite 3, comme $3<4$ pas de modification \pstree{\Tcircle{4}}{\Tcircle{6} \Tcircle{8}}
			\item On traite 7, comme $7>4$, on extrait 4 et on insère 7 \pstree{\Tcircle{6}}{\Tcircle{8} \Tcircle{7}}
			\item On traite 1, comme $1<6$, pas de modification \pstree{\Tcircle{6}}{\Tcircle{8} \Tcircle{7}}
			\item On traite 9, comme $9>6$, on extrait 6 et on insère 9 \pstree{\Tcircle{7}}{\Tcircle{8} \Tcircle{9}}
			\item On traite 5, comme $5<7$, pas de modification \pstree{\Tcircle{7}}{\Tcircle{8} \Tcircle{9}}
		\end{itemize}}
	\Question{Donner une implémentation de la fonction {\tt kmax} utilisant ce nouvel algorithme. On rappelle qu'on pourra utiliser les fonctions de manipulation de la structure de {\tt tas} données en début de partie. Comme précédemment, on procédera par récurrence sans utiliser les aspects impératifs de OCaml.}
  \ifcorrige
  \corpartOCaml{kmax.ml}{}{}{82}{95}
  \fi
	\Question{Donner en la justifiant la complexité de ce nouvel algorithme en fonction de $k$ et $n$.}
	\tcor{Les opérations d'insertion et d'extraction dans le tas sont toutes en $O(\log k)$ car le tas est de taille $k$. On effectue ces opérations au plus $n$  fois (une fois pour chaque élément du tableau) et donc la complexité de ce nouvel algorithme est $O(n \log k)$}
\end{Exercise}


\begin{Exercise}[title={Saut de valeur maximale}, origin={\bac {\sc capes nsi 2023}}]\\
	Les fonctions demandées dans cet exercice sont à écrire en langage C. \smallskip \\
	Dans un tableau de flottants (type {\tt double} du langage C) {\tt tab} de taille {\tt n}, on appelle \textit{saut} un couple $(i,j)$ avec $0 \leq i \leq j <$ {\tt n  } et la \textit{valeur} d'un saut est la valeur {\tt tab[j]-tab[i]}. Le but de l'exercice est de rechercher la valeur maximale d'un saut dans un tableau. Par exemple, dans le tableau {\tt \{ 2.0, 0.2, 3.0, 5.3, 2.0\}}, la valeur maximale d'un saut est {\tt 5.3 - 0.2 = 5.1}, cette valeur est obtenue en considérant le saut {\tt (1,3)}.\medskip

  \ExePart[name = {Questions préliminaires et résolution naïve}]
	\Question{Ecrire une fonction de signature \mintinline{c}{int valeur(int tab[], int i, int j, int n)} qui prend en argument un tableau {\tt tab} de taile {\tt n} ainsi que deux indices {\tt i} et {\tt j} et renvoie la valeur du saut {\tt (i,j)}. On vérifiera les préconditions sur {\tt i} et {\tt j} à l'aide d'instructions {\tt assert}. Par exemple  si le tableau {\tt tab} est {\tt \{2.0; 0.2; 3.0; 5.3; 2.0\}} alors {\tt valeur(tab, 2, 0, 5)} renvoie {\tt 1.0} (car {\tt tab[2]-tab[0] = 1.0}).}
  \ifcorrige
  \corpartC{sautmax.c}{}{}{7}{11}
  \fi
	\Question{Donner un exemple de tableau avec exactement deux sauts de valeur maximale et préciser ces sauts.}
	\tcor{La liste {\tt [2, 6, 1, 5]} possède deux sauts de valeurs maximale : {\tt (0,1)} et {\tt (2,3)} (ces deux sauts ont une valeur de 4)}
	\Question{À l'aide d'un contre-exemple, montrer qu'on ne peut pas se contenter de chercher le minimum et le maximum d'un tableau pour trouver un saut de valeur maximale.}
	\tcor{Dans la liste {\tt [2, 6, 1, 5]} le minimum est à l'indice 2 (c'est 1) et le maximum à l'indice 1 (c'est 3) et comme le minimum est après le maximum ce n'est pas le saut maximal.}
	\Question{Écrire une fonction {\tt sautmax\_naif} qui renvoie un saut de valeur maximale dans un tableau de taille {\tt n} en testant tous les couples $(i,j)$ tels que $0 \leq i \leq j <$ {\tt n}.\label{naif}}
  \ifcorrige
  \corpartC{sautmax.c}{}{}{7}{11}
  \fi
	\Question{Quelle est la complexité de la fonction {\tt sautmax\_naif} en fonction de la taille $n$ du tableau ?}\medskip
  \tcor{Il y a deux boucles {\tt for} imbriquées et les deux sont exécutés au plus {\tt n fois}, les opérations à l'intérieur de ces boucles sont toutes des opérations élémentaires, donc la complexité est en $\mathcal{O}(n^2)$}
  \ExePart[name = {Résolution avec une méthode diviser pour régner}]\\
	On propose maintenant d'utiliser une méthode diviser pour régner afin de calculer la valeur maximale d'un saut. On note $n$ la taille du tableau {\tt t} et $p = \lfloor \frac{n}{2} \rfloor$ (où $\lfloor \; \rfloor$ désigne la partie entière). On souhaite calculer :
	\begin{itemize}
		\item $(i_g,j_g)$ un saut de valeur maximale lorsque $j_g < p$ (c'est à dire un saut maximal dans la moitié gauche)
		\item $(i_d,j_d)$ un saut de valeur maximale lorsque $id \geqslant p$ (c'est à dire un saut maximal dans la moitié gauche)
		\item $(i_m,j_m)$ un saut de valeur maximal lorsque $i_m < p < j_m$ (c'est à dire un saut maximal dont le premier indice est dans la moitié gauche et le second dans la moitié droite)
	\end{itemize}

	\Question{Justifier qu'un saut de valeur maximale du tableau {\tt t} est nécessairement un des trois ci-dessus. On pourra faire un schéma pour illustrer le raisonnement.}
	\tcor{Un saut de valeur maximal $(i,j)$ du tableau {\tt t} est tel que $i \leqslant j$ trois situations sont donc possibles en fonction de la position relative de $i,j$ et $p$ :
		\begin{itemize}
			\item $i\leqslant j<p$ et donc le saut maximal $(i,j)$ se situe dans la moité gauche du tableau.\smallskip \\
      \psline{|-|}(0,0)(10,0)
      \uput[d](0,0){$0$}
      \uput[d](10,0){$n$}
      \uput[d](5,0.65){$p$}
      \psdot(5,0)
      % Draw and label i and j in the left half
      \uput[d](2,0.7){$i$}
      \psdot(2,0)
      \uput[d](4,0.7){$j$}
      \psdot(4,0)
			\item $p \leqslant i \leqslant j$ et donc le saut maximal $(i,j)$ se situe dans la moité droite du tableau.\smallskip \\
      \psline{|-|}(0,0)(10,0)
      \uput[d](0,0){$0$}
      \uput[d](10,0){$n$}
      \uput[d](5,0.65){$p$}
      \psdot(5,0)
      % Draw and label i and j in the left half
      \uput[d](7,0.7){$i$}
      \psdot(7,0)
      \uput[d](9.5,0.7){$j$}
      \psdot(9.5,0)
			\item $i < p < j$ c'est à dire que le saut \og{} traverse \fg{} le milieu du tableau.\smallskip \\
			\psline{|-|}(0,0)(10,0)
      \uput[d](0,0){$0$}
      \uput[d](10,0){$n$}
      \uput[d](5,0.65){$p$}
      \psdot(5,0)
      % Draw and label i and j in the left half
      \uput[d](3,0.7){$i$}
      \psdot(3,0)
      \uput[d](6.5,0.7){$j$}
      \psdot(6.5,0)
		\end{itemize}
		Ce qui correspond bien aux trois situations décrites ci-dessus.
	}
	\Question{Justifier que $i_m$ est nécessairement l'indice d'une valeur minimale dans la moitié gauche de {\tt t} (on admettra que de même $j_m$ est nécessairement l'indice d'une valeur maximale dans la moitié droite de {\tt t}).}
	\tcor{On raisonne par l'absurde, si tel n'était pas le cas, on aurait une valeur d'indice $q$ dans la moitié gauche strictement inférieur à $t[i_m]$ et donc le saut $(q,j_m)$ aurait une valeur supérieure au saut $(i_m,j_m)$ ce qui contredit que $(i_m,j_m)$ est un saut de valeur maximale.}
  \Question{Ecrire une fonction de signature \mintinline{c}{int min(int tab[], int a, int b)} qui prend en argument un tableau {\tt tab}, ainsi que deux entiers {\tt a} et {\tt b} (avec {\tt a<=b}) et renvoie l'indice d'un minimum de {\tt tab} entre les deux indices {\tt a} (inclus) et {\tt b} (exclu).\\On supposera dans la suite \textit{déjà écrite} une fonction qui {\tt max} qui prend les mêmes arguments et renvoie l'indice d'un maximum du sous tableau {\tt \{tab[a],...,tab[b-1]\}} }
  \ifcorrige
  \corpartC{sautmax.c}{}{}{45}{57}
  \fi
	\Question{Ecrire une fonction de signature \mintinline{c}{int sautmax_dpr(int tab[], int a, int b)} qui prend en argument un tableau {\tt tab} ainsi que deux entiers {\tt a} et {\tt b} (avec {\tt a<=b}) et renvoie la valeur d'un saut maximale dans {\tt tab} entre les deux indices {\tt a} (inclus) et {\tt b} (exclu).\textit{Cette fonction doit être récursive et utiliser la méthode diviser pour régner}. On pourra supposer déjà écrite une fonction {\tt max3} qui renvoie le maximum des trois {\tt double} donnés en argument.
	}
  \ifcorrige
  \corpartC{sautmax.c}{}{}{73}{88}
  \fi
  \Question{Déterminer la complexité de la fonction {\tt sautmax\_dpr}.}
  \tcor{
    On obtient l'équation de compexité $C(2n) = 2 C(n) + \mathcal{O}(n)$ en effet on résout deux sous problèmes de taille deux fois plus petite et on doit ensuite calculer un minimum et un maximum d'une liste de taille $n$ et ces opérations sont en $\mathcal{O}(n)$. On suppose sans perdre de généralité que $n=2^k$ et on note $u_k = C(2^k)/2^k$, en divisant l'équation de complexité par $2^{k+1}$ on obtient alors : \\
    $u_{k+1} \leqslant u_k + \dfrac{M2^k}{2^{k+1}}$ donc,
    $u_{k+1} \leqslant u_k + M\times \dfrac{1}{2}$ et en sommant pour $i=0$ à $k$ on obtient : \\
    $u_{k} \leqslant u_0 + M' k$\\
    $C(n) \leqslant n \, \left( C(1) + M' \log n \right)$\\
    Et donc $C(n) \in \mathcal{O}(n \log n)$.
  }
  \ExePart[name = {Résolution par programmation dynamique}]\\
    On cherche maintenant à résoudre ce problème par programmation dynamique, et on adopte les notations suivantes :
    \begin{itemize}
    \item $t$ est un tableau de taille $n$ contenant des flottants 
    \item $t[i]$ est l'élément d'indice $i$ ($0\leqslant i < n$) de $t$,
    \item pour $0<k \leqslant n$, $t_k$ est le sous tableau $t[0], \dots, t[k-1]$
    \item $m_k$ est l'indice d'un minimum de $t_k$ pour $0<k<n$.
    \item $(i_k,j_k)$ est un saut de valeur maximale dans $t_k$ pour $0<k<n$.
    \end{itemize}
    \Question{Donner les valeurs de $i_1$, $j_1$ et $m_1$}
    \tcor{Comme le tableau ne contient qu'un seul élément (celui d'indice 0), $i_1 = 0, j_1=0$ et $m_1=0$.}
    \Question{Donner la relation de récurrence liant $m_{k+1}$, $m_k$ et $t[k+1]$.}
    \tcor{Si $t[k+1]<m_k$ alors $m_{k+1} = t[k+1]$ sinon $m_{k+1} = m_k$.}
    \Question{Justifier que la relation suivante est correcte :\\
    $(i_{k+1},j_{k+1}) = \left\{ \begin{array}{ll} (i_k,j_k) \text{ si } t[k]-t[m_k]<t[j_k]-t[i_k] \\ (m_k,k) \text{ sinon} \end{array}\right.$
    }
    \tcor{Le saut de valeur maximal ayant $k$ comme second indice est $(m_k, k)$ et vaut $t[k]-t[m_k]$. En effet le second indice étant fixé on l'obtient en prenant le miminum du sous tableau $t_k$. On doit donc comparer ce nouveau saut avec le le saut maximal du sous tableau $t_k$. Soit il est plus petit c'est à dire $t[k]-t[m_k] < t[j_k] - t[i_k]$ et donc $(i_{k+1},j_{k+1}) = (i_k,j_k)$ ou bien il est plus grand et donc $(i_{k+1},j_{k+1}) = (m_k,k)$.}
    \Question{Ecrire une fonction de signature \mintinline{c}{int sautmax_dyn(int tab[], int n)} qui prend en argument un tableau et sa taille et renvoie la valeur maximale d'un saut de ce tableau. On procèdera \textit{de façon ascendante} en utilisant les relations de récurrences de la question précédente et en calculant successivement les valeurs maximales de saut dans les sous tableaux $t_1$ puis $t_2$,  $\dots$ jusqu'à $t_n$.}
    \ifcorrige
    \corpartC{sautmax.c}{}{}{90}{111}
    \fi
    \Question{Déterminer la complexité de la fonction {\tt sautmax\_dyn}}
    \tcor{La fonction parcourt le tableau à l'aide d'une boucle {\tt for} qui ne contient que des opérations élémentaires, la complexité est donc linéaire en fonction de la taille du tableau.}
\end{Exercise}


\end{document}
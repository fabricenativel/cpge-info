\PassOptionsToPackage{dvipsnames,table}{xcolor}
\documentclass[11pt,a4paper]{article}

\usepackage{DS}

\begin{document}
\input{\detokenize{/home/fenarius/Travail/Cours/cpge-info/latex/Macros.tex}}
\newcommand{\SPATH}{/home/fenarius/Travail/Cours/cpge-info/docs/mp2i/files/C12/}
\ModeExercice
\DM{Lundi 18 mars 2023}{MP2I}

\begin{Exercise}[title = {Arbre binaire de recherche}] \\
    Cette exercice est à traiter en OCaml en utilisant le type  :
    \inputpartOCaml{\SPATH/arbres_binaires_int.ml}{}{}{2}{2}
    \Question{}
    On considère l'arbre binaire suivant :
    \begin{center}
    \pstree[arrows=->,treesep=1cm,levelsep=1cm]{\TCircle{29}}
        {\pstree{\TCircle{24}}
        {\pstree{\TCircle{14}}
        { \Tn{} 
        \pstree{\TCircle{16}}
        { \Tn{} 
        \TCircle{20} 
        }}\TCircle{28} 
        }\pstree{\TCircle{31}}
        { \TCircle{30}
        \pstree{\TCircle{35}}
         { \Tn{} \TCircle{42}}
          
          }}
    \end{center}
    \Question{Justifier (en utilisant la définition) qu'il s'agit bien d'un arbre binaire de recherche ({\sc abr}).}
    \Question{On suppose \textit{seulement dans cette question}  que le fils gauche du noeud 16 porte l'étiquette $x \in \N$. Rappeler la caractérisation d'un {\sc abr} par son parcours infixe et utiliser cette caractérisation  pour déterminer les valeurs possibles de $x$.}
    \Question{Rappeler l'algorithme d'insertion d'un élément dans un arbre binaire de recherche}
    \Question{Détailler l'insertion des valeurs 17 et 33 dans cette arbre et dessiner l'arbre obtenu après ces insertions.}
    \Question{Donner une implémentation en OCaml de l'algorithme d'insertion sous la forme d'une fonction {\tt insere} de signature {\tt abr -> int -> abr}}
    \Question{On note $m$ la valeur minimale d'un {\sc abr}. Justifier que l'arbre obtenu en remplaçant le noeud d'étiquette $m$ par son sous arbre droit,  est un {\sc abr}.}
    \Question{Ecrire une fonction {\tt extraire\_min} en OCaml de signaure {\tt abr -> int * abr}, qui renvoie un couple composé du minimum d'un arbre binaire de recherche et de cet arbre privé du noeud de valeur minimale. On gère le cas de l'arbre vide avec {\tt failwith}}
    \Question{Donner la complexité de {\tt extraire\_min}.}
    \Question{Afin de supprimer une valeur dans un {\sc abr}, on propose de remplacer  cette valeur par la plus petite valeur présente dans son sous arbre droit en y supprimant cette valeur grâce à la fonction précédente. 
    Détailler le fonctionnement de cette méthode sur l'arbre suivant d'où on veut supprimer la valeur 2 : 
    \begin{center}
    \pstree[arrows=->,treesep=1cm,levelsep=1cm]{\TCircle{9}}
{\pstree{\TCircle{2}}
{\TCircle{0} 
\pstree{\TCircle{6}}
{\TCircle{5} 
\pstree{\TCircle{8}}
{ 
\TCircle{7} 
 \Tn{} }}}\pstree{\TCircle{12}}
{\TCircle{10} 
\TCircle{13} 
}} 
    
    \end{center}
    \Question{Donner une implémentation en OCaml sous la forme d'une fonction {\tt supprime} de signature {\tt abr -> int -> abr}}
    
    }
\end{Exercise} 


\begin{Exercise}[title = {Problème du sac à dos en force brute}] \\
    Voir l'énoncé en ligne : chapitre 13 exercice 1
\end{Exercise}

\end{document}
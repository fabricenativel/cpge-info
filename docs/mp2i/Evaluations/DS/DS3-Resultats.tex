
\documentclass[11pt,a4paper]{article}
\usepackage{Act}
\begin{document}
\input{\detokenize{/home/fenarius/Travail/Cours/cpge-info/latex/Macros.tex}}
\RDS{3}{08/11/2025}

\begin{tcolorbox}[enhanced,width=\textwidth,center upper,fontupper=\bfseries,drop shadow southwest,sharp corners]
{\sc \large Astruc} Alexandre
\end{tcolorbox}
\medskip
\begin{tabularx}{\textwidth}{p{5cm}X}
	\alertbox{\faAward}{Note}{
		\begin{itemize}[leftmargin=0pt]
			\item[\textbullet] Note : \textbf{\large 10.3}
			\item[\textbullet] Rang : \textbf{7}
			\item[\textbullet] Traité : 97 \%
		\end{itemize}
	} &
	\alertbox{\faChartLine}{Statistiques des notes}{
        \psset{xunit=1cm, yunit=1cm,fillstyle=solid}
		\begin{pspicture}(0,-0.1)(16,1.45)
		   \savedata{\data}[10.3 16.4 19.2 8.0 8.1 4.8 6.6 0.9 7.5 14.8 16.4 13.6 19.7 7.0 9.0]
		   \rput{-90}(0,0.9){\psBoxplot[barwidth=1.1cm,yunit=0.5,fillcolor=gray,linewidth=1pt]{\data}}
		   \psaxes[yAxis=false,dx=1cm,Dx=2,labelsep=1pt,linecolor=gray,xlabelFontSize=\scriptstyle](0,0)(10.1,4)
		   \psdot[dotsize=8pt,dotstyle=diamond,linecolor=black,fillstyle=solid,fillcolor=white,linewidth=1pt](5.15,0.85)
           \psdot[dotsize=6pt,dotstyle=x,linecolor=black,linewidth=3pt](5.409999999999999,0.85)
		   \end{pspicture}
	} \\
    
\end{tabularx}\\
\begin{tabularx}{\textwidth}{X}
\alertbox{\faComment}{Commentaire}
{
	C’est correct dans l’ensemble mais il faut revoir le tri par sélection et le programmer complètement. Tu as confondu minimum et indice du miminum !
Les capacités sont les manipulations de listes chainées sont aussi à consolider
}
\end{tabularx}
\medskip
    \ding{113} \textbf{\sffamily{Résultats par thème}} \medskip \\
    \renewcommand{\arraystretch}{1.2}
    \begin{tabular}{|l|r|r|}
    \cline{2-3}
    \multicolumn{1}{l|}{} & \multicolumn{1}{|c|}{Points} & \multicolumn{1}{|c|}{Traitées} \\
    \hline
    {Comprendre un algorithme} & 90\% \;{\small (09/10)} & 100\% \;{\small (2/2)} \\ \hline {Base d'OCaml (fonctionnel)} & 40\% \;{\small (34/85)} & 100\% \;{\small (11/11)} \\ \hline {Programmation de base en C} & 24\% \;{\small (06/25)} & 75\% \;{\small (3/4)} \\ \hline {Types structurés en C et pointeurs} & 28\% \;{\small (13/45)} & 100\% \;{\small (4/4)} \\ \hline {Structure de données séquentielles} & 100\% \;{\small (05/5)} & 100\% \;{\small (1/1)} \\ \hline {Complexité d'un algorithme} & 86\% \;{\small (52/60)} & 100\% \;{\small (8/8)} \\ \hline \end{tabular} \\\\\medskip \\
    \ding{113} \textbf{\sffamily{Résultats par exercice}} \medskip \\
    \renewcommand{\arraystretch}{1.2}
    \begin{tabular}{|l|r|r|}
    \cline{2-3}
    \multicolumn{1}{l|}{} & \multicolumn{1}{|c|}{Points} & \multicolumn{1}{|c|}{Traitées} \\
    \hline
    Exercice {1} & 62\% \;{\small (47/75)} & 100\% \;{\small (10/10)} \\ \hline Exercice {2} & 51\% \;{\small (31/60)} & 100\% \;{\small (6/6)} \\ \hline Exercice {3} & 32\% \;{\small (13/40)} & 100\% \;{\small (6/6)} \\ \hline Exercice {4} & 50\% \;{\small (28/55)} & 87\% \;{\small (7/8)} \\ \hline \end{tabular} \\\\
    \vspace{0.5cm}\\
    \ding{113} \textbf{\sffamily{Historique des notes}} \medskip \\
    \psset{xunit=1.4cm, yunit=0.2cm}
    \begin{pspicture}(-1,-1)(12,22)

% --- Axe X : numéros de devoir ---
\psaxes[
    Dx=1,
    Dy=2,
    Ox=0,
    Oy=0,
    labels=all,
    ticks=all,
](0,0)(0,0)(7,20)


\listplot[plotstyle=line,showpoints=true,linecolor=black,linewidth=0.7pt, dotstyle=diamond*, dotsize=0.2]{%
    1 11.7
2 11.9
3 10.3
}

% Minimum de la classe
\listplot[plotstyle=line,showpoints=true,linecolor=gray,linewidth=0.7pt, dotstyle=|, dotangle=90, dotsize=0.15, linestyle=dotted]{%
    1 5.8
2 4.6
3 0.9
}

% Maximum de la classe
\listplot[plotstyle=line,showpoints=true,linecolor=gray,linewidth=0.7pt, dotstyle=|, dotangle=90, dotsize=0.15, linestyle=dotted]{%
    1 18.9
2 18.4
3 19.7
}

% Moyenne de la classe
\listplot[plotstyle=line,showpoints=true,linecolor=gray,linewidth=0.7pt, dotstyle=x, dotsize = 0.15, linestyle=dashed]{%
    1 12.4466666666667
2 13.1933333333333
3 10.82
}
\end{pspicture}
\pagebreak
\begin{tcolorbox}[enhanced,width=\textwidth,center upper,fontupper=\bfseries,drop shadow southwest,sharp corners]
{\sc \large Berfeuil} Rohan
\end{tcolorbox}
\medskip
\begin{tabularx}{\textwidth}{p{5cm}X}
	\alertbox{\faAward}{Note}{
		\begin{itemize}[leftmargin=0pt]
			\item[\textbullet] Note : \textbf{\large 16.4}
			\item[\textbullet] Rang : \textbf{3}
			\item[\textbullet] Traité : 93 \%
		\end{itemize}
	} &
	\alertbox{\faChartLine}{Statistiques des notes}{
        \psset{xunit=1cm, yunit=1cm,fillstyle=solid}
		\begin{pspicture}(0,-0.1)(16,1.45)
		   \savedata{\data}[10.3 16.4 19.2 8.0 8.1 4.8 6.6 0.9 7.5 14.8 16.4 13.6 19.7 7.0 9.0]
		   \rput{-90}(0,0.9){\psBoxplot[barwidth=1.1cm,yunit=0.5,fillcolor=gray,linewidth=1pt]{\data}}
		   \psaxes[yAxis=false,dx=1cm,Dx=2,labelsep=1pt,linecolor=gray,xlabelFontSize=\scriptstyle](0,0)(10.1,4)
		   \psdot[dotsize=8pt,dotstyle=diamond,linecolor=black,fillstyle=solid,fillcolor=white,linewidth=1pt](8.2,0.85)
           \psdot[dotsize=6pt,dotstyle=x,linecolor=black,linewidth=3pt](5.409999999999999,0.85)
		   \end{pspicture}
	} \\
    
\end{tabularx}\\
\begin{tabularx}{\textwidth}{X}
\alertbox{\faComment}{Commentaire}
{
	Bon travail, tu es en net progrès c’est bien. La manipulation de listes en Ocaml semble bien assimilée. Par contre tu dois encore travailler la manipulation de listes chainées en C (écris des programmes) !
}
\end{tabularx}
\medskip
    \ding{113} \textbf{\sffamily{Résultats par thème}} \medskip \\
    \renewcommand{\arraystretch}{1.2}
    \begin{tabular}{|l|r|r|}
    \cline{2-3}
    \multicolumn{1}{l|}{} & \multicolumn{1}{|c|}{Points} & \multicolumn{1}{|c|}{Traitées} \\
    \hline
    {Comprendre un algorithme} & 100\% \;{\small (10/10)} & 100\% \;{\small (2/2)} \\ \hline {Base d'OCaml (fonctionnel)} & 71\% \;{\small (61/85)} & 81\% \;{\small (9/11)} \\ \hline {Programmation de base en C} & 96\% \;{\small (24/25)} & 100\% \;{\small (4/4)} \\ \hline {Types structurés en C et pointeurs} & 64\% \;{\small (29/45)} & 100\% \;{\small (4/4)} \\ \hline {Structure de données séquentielles} & 100\% \;{\small (05/5)} & 100\% \;{\small (1/1)} \\ \hline {Complexité d'un algorithme} & 100\% \;{\small (60/60)} & 100\% \;{\small (8/8)} \\ \hline \end{tabular} \\\\\medskip \\
    \ding{113} \textbf{\sffamily{Résultats par exercice}} \medskip \\
    \renewcommand{\arraystretch}{1.2}
    \begin{tabular}{|l|r|r|}
    \cline{2-3}
    \multicolumn{1}{l|}{} & \multicolumn{1}{|c|}{Points} & \multicolumn{1}{|c|}{Traitées} \\
    \hline
    Exercice {1} & 98\% \;{\small (74/75)} & 100\% \;{\small (10/10)} \\ \hline Exercice {2} & 75\% \;{\small (45/60)} & 100\% \;{\small (6/6)} \\ \hline Exercice {3} & 55\% \;{\small (22/40)} & 66\% \;{\small (4/6)} \\ \hline Exercice {4} & 87\% \;{\small (48/55)} & 100\% \;{\small (8/8)} \\ \hline \end{tabular} \\\\
    \vspace{0.5cm}\\
    \ding{113} \textbf{\sffamily{Historique des notes}} \medskip \\
    \psset{xunit=1.4cm, yunit=0.2cm}
    \begin{pspicture}(-1,-1)(12,22)

% --- Axe X : numéros de devoir ---
\psaxes[
    Dx=1,
    Dy=2,
    Ox=0,
    Oy=0,
    labels=all,
    ticks=all,
](0,0)(0,0)(7,20)


\listplot[plotstyle=line,showpoints=true,linecolor=black,linewidth=0.7pt, dotstyle=diamond*, dotsize=0.2]{%
    1 11.2
2 15.4
3 16.4
}

% Minimum de la classe
\listplot[plotstyle=line,showpoints=true,linecolor=gray,linewidth=0.7pt, dotstyle=|, dotangle=90, dotsize=0.15, linestyle=dotted]{%
    1 5.8
2 4.6
3 0.9
}

% Maximum de la classe
\listplot[plotstyle=line,showpoints=true,linecolor=gray,linewidth=0.7pt, dotstyle=|, dotangle=90, dotsize=0.15, linestyle=dotted]{%
    1 18.9
2 18.4
3 19.7
}

% Moyenne de la classe
\listplot[plotstyle=line,showpoints=true,linecolor=gray,linewidth=0.7pt, dotstyle=x, dotsize = 0.15, linestyle=dashed]{%
    1 12.4466666666667
2 13.1933333333333
3 10.82
}
\end{pspicture}
\pagebreak
\begin{tcolorbox}[enhanced,width=\textwidth,center upper,fontupper=\bfseries,drop shadow southwest,sharp corners]
{\sc \large Body} Timothée
\end{tcolorbox}
\medskip
\begin{tabularx}{\textwidth}{p{5cm}X}
	\alertbox{\faAward}{Note}{
		\begin{itemize}[leftmargin=0pt]
			\item[\textbullet] Note : \textbf{\large 19.2}
			\item[\textbullet] Rang : \textbf{2}
			\item[\textbullet] Traité : 100 \%
		\end{itemize}
	} &
	\alertbox{\faChartLine}{Statistiques des notes}{
        \psset{xunit=1cm, yunit=1cm,fillstyle=solid}
		\begin{pspicture}(0,-0.1)(16,1.45)
		   \savedata{\data}[10.3 16.4 19.2 8.0 8.1 4.8 6.6 0.9 7.5 14.8 16.4 13.6 19.7 7.0 9.0]
		   \rput{-90}(0,0.9){\psBoxplot[barwidth=1.1cm,yunit=0.5,fillcolor=gray,linewidth=1pt]{\data}}
		   \psaxes[yAxis=false,dx=1cm,Dx=2,labelsep=1pt,linecolor=gray,xlabelFontSize=\scriptstyle](0,0)(10.1,4)
		   \psdot[dotsize=8pt,dotstyle=diamond,linecolor=black,fillstyle=solid,fillcolor=white,linewidth=1pt](9.6,0.85)
           \psdot[dotsize=6pt,dotstyle=x,linecolor=black,linewidth=3pt](5.409999999999999,0.85)
		   \end{pspicture}
	} \\
    
\end{tabularx}\\
\begin{tabularx}{\textwidth}{X}
\alertbox{\faComment}{Commentaire}
{
	Excellent travail, dans la fonction qui renvoie la longueur d’une liste, on ne peut pas utiliser la fonction retirier.
}
\end{tabularx}
\medskip
    \ding{113} \textbf{\sffamily{Résultats par thème}} \medskip \\
    \renewcommand{\arraystretch}{1.2}
    \begin{tabular}{|l|r|r|}
    \cline{2-3}
    \multicolumn{1}{l|}{} & \multicolumn{1}{|c|}{Points} & \multicolumn{1}{|c|}{Traitées} \\
    \hline
    {Comprendre un algorithme} & 100\% \;{\small (10/10)} & 100\% \;{\small (2/2)} \\ \hline {Base d'OCaml (fonctionnel)} & 100\% \;{\small (85/85)} & 100\% \;{\small (11/11)} \\ \hline {Programmation de base en C} & 100\% \;{\small (25/25)} & 100\% \;{\small (4/4)} \\ \hline {Types structurés en C et pointeurs} & 80\% \;{\small (36/45)} & 100\% \;{\small (4/4)} \\ \hline {Structure de données séquentielles} & 100\% \;{\small (05/5)} & 100\% \;{\small (1/1)} \\ \hline {Complexité d'un algorithme} & 100\% \;{\small (60/60)} & 100\% \;{\small (8/8)} \\ \hline \end{tabular} \\\\\medskip \\
    \ding{113} \textbf{\sffamily{Résultats par exercice}} \medskip \\
    \renewcommand{\arraystretch}{1.2}
    \begin{tabular}{|l|r|r|}
    \cline{2-3}
    \multicolumn{1}{l|}{} & \multicolumn{1}{|c|}{Points} & \multicolumn{1}{|c|}{Traitées} \\
    \hline
    Exercice {1} & 100\% \;{\small (75/75)} & 100\% \;{\small (10/10)} \\ \hline Exercice {2} & 85\% \;{\small (51/60)} & 100\% \;{\small (6/6)} \\ \hline Exercice {3} & 100\% \;{\small (40/40)} & 100\% \;{\small (6/6)} \\ \hline Exercice {4} & 100\% \;{\small (55/55)} & 100\% \;{\small (8/8)} \\ \hline \end{tabular} \\\\
    \vspace{0.5cm}\\
    \ding{113} \textbf{\sffamily{Historique des notes}} \medskip \\
    \psset{xunit=1.4cm, yunit=0.2cm}
    \begin{pspicture}(-1,-1)(12,22)

% --- Axe X : numéros de devoir ---
\psaxes[
    Dx=1,
    Dy=2,
    Ox=0,
    Oy=0,
    labels=all,
    ticks=all,
](0,0)(0,0)(7,20)


\listplot[plotstyle=line,showpoints=true,linecolor=black,linewidth=0.7pt, dotstyle=diamond*, dotsize=0.2]{%
    1 18.9
2 18.4
3 19.2
}

% Minimum de la classe
\listplot[plotstyle=line,showpoints=true,linecolor=gray,linewidth=0.7pt, dotstyle=|, dotangle=90, dotsize=0.15, linestyle=dotted]{%
    1 5.8
2 4.6
3 0.9
}

% Maximum de la classe
\listplot[plotstyle=line,showpoints=true,linecolor=gray,linewidth=0.7pt, dotstyle=|, dotangle=90, dotsize=0.15, linestyle=dotted]{%
    1 18.9
2 18.4
3 19.7
}

% Moyenne de la classe
\listplot[plotstyle=line,showpoints=true,linecolor=gray,linewidth=0.7pt, dotstyle=x, dotsize = 0.15, linestyle=dashed]{%
    1 12.4466666666667
2 13.1933333333333
3 10.82
}
\end{pspicture}
\pagebreak
\begin{tcolorbox}[enhanced,width=\textwidth,center upper,fontupper=\bfseries,drop shadow southwest,sharp corners]
{\sc \large Boucher} Mathis
\end{tcolorbox}
\medskip
\begin{tabularx}{\textwidth}{p{5cm}X}
	\alertbox{\faAward}{Note}{
		\begin{itemize}[leftmargin=0pt]
			\item[\textbullet] Note : \textbf{\large 8.0}
			\item[\textbullet] Rang : \textbf{10}
			\item[\textbullet] Traité : 83 \%
		\end{itemize}
	} &
	\alertbox{\faChartLine}{Statistiques des notes}{
        \psset{xunit=1cm, yunit=1cm,fillstyle=solid}
		\begin{pspicture}(0,-0.1)(16,1.45)
		   \savedata{\data}[10.3 16.4 19.2 8.0 8.1 4.8 6.6 0.9 7.5 14.8 16.4 13.6 19.7 7.0 9.0]
		   \rput{-90}(0,0.9){\psBoxplot[barwidth=1.1cm,yunit=0.5,fillcolor=gray,linewidth=1pt]{\data}}
		   \psaxes[yAxis=false,dx=1cm,Dx=2,labelsep=1pt,linecolor=gray,xlabelFontSize=\scriptstyle](0,0)(10.1,4)
		   \psdot[dotsize=8pt,dotstyle=diamond,linecolor=black,fillstyle=solid,fillcolor=white,linewidth=1pt](4.0,0.85)
           \psdot[dotsize=6pt,dotstyle=x,linecolor=black,linewidth=3pt](5.409999999999999,0.85)
		   \end{pspicture}
	} \\
    
\end{tabularx}\\
\begin{tabularx}{\textwidth}{X}
\alertbox{\faComment}{Commentaire}
{
	L’ensemble est trop juste car tu ne maitrises pas les exercices où on demande de manipuler des listes en Ocaml. Il faut revoir le cours et reprendre les exercices.
}
\end{tabularx}
\medskip
    \ding{113} \textbf{\sffamily{Résultats par thème}} \medskip \\
    \renewcommand{\arraystretch}{1.2}
    \begin{tabular}{|l|r|r|}
    \cline{2-3}
    \multicolumn{1}{l|}{} & \multicolumn{1}{|c|}{Points} & \multicolumn{1}{|c|}{Traitées} \\
    \hline
    {Comprendre un algorithme} & 100\% \;{\small (10/10)} & 100\% \;{\small (2/2)} \\ \hline {Base d'OCaml (fonctionnel)} & 29\% \;{\small (25/85)} & 81\% \;{\small (9/11)} \\ \hline {Programmation de base en C} & 40\% \;{\small (10/25)} & 75\% \;{\small (3/4)} \\ \hline {Types structurés en C et pointeurs} & 11\% \;{\small (05/45)} & 50\% \;{\small (2/4)} \\ \hline {Structure de données séquentielles} & 60\% \;{\small (03/5)} & 100\% \;{\small (1/1)} \\ \hline {Complexité d'un algorithme} & 65\% \;{\small (39/60)} & 100\% \;{\small (8/8)} \\ \hline \end{tabular} \\\\\medskip \\
    \ding{113} \textbf{\sffamily{Résultats par exercice}} \medskip \\
    \renewcommand{\arraystretch}{1.2}
    \begin{tabular}{|l|r|r|}
    \cline{2-3}
    \multicolumn{1}{l|}{} & \multicolumn{1}{|c|}{Points} & \multicolumn{1}{|c|}{Traitées} \\
    \hline
    Exercice {1} & 48\% \;{\small (36/75)} & 100\% \;{\small (10/10)} \\ \hline Exercice {2} & 20\% \;{\small (12/60)} & 66\% \;{\small (4/6)} \\ \hline Exercice {3} & 50\% \;{\small (20/40)} & 66\% \;{\small (4/6)} \\ \hline Exercice {4} & 43\% \;{\small (24/55)} & 87\% \;{\small (7/8)} \\ \hline \end{tabular} \\\\
    \vspace{0.5cm}\\
    \ding{113} \textbf{\sffamily{Historique des notes}} \medskip \\
    \psset{xunit=1.4cm, yunit=0.2cm}
    \begin{pspicture}(-1,-1)(12,22)

% --- Axe X : numéros de devoir ---
\psaxes[
    Dx=1,
    Dy=2,
    Ox=0,
    Oy=0,
    labels=all,
    ticks=all,
](0,0)(0,0)(7,20)


\listplot[plotstyle=line,showpoints=true,linecolor=black,linewidth=0.7pt, dotstyle=diamond*, dotsize=0.2]{%
    1 8.4
2 11.8
3 8
}

% Minimum de la classe
\listplot[plotstyle=line,showpoints=true,linecolor=gray,linewidth=0.7pt, dotstyle=|, dotangle=90, dotsize=0.15, linestyle=dotted]{%
    1 5.8
2 4.6
3 0.9
}

% Maximum de la classe
\listplot[plotstyle=line,showpoints=true,linecolor=gray,linewidth=0.7pt, dotstyle=|, dotangle=90, dotsize=0.15, linestyle=dotted]{%
    1 18.9
2 18.4
3 19.7
}

% Moyenne de la classe
\listplot[plotstyle=line,showpoints=true,linecolor=gray,linewidth=0.7pt, dotstyle=x, dotsize = 0.15, linestyle=dashed]{%
    1 12.4466666666667
2 13.1933333333333
3 10.82
}
\end{pspicture}
\pagebreak
\begin{tcolorbox}[enhanced,width=\textwidth,center upper,fontupper=\bfseries,drop shadow southwest,sharp corners]
{\sc \large Chane-lock} Maxime
\end{tcolorbox}
\medskip
\begin{tabularx}{\textwidth}{p{5cm}X}
	\alertbox{\faAward}{Note}{
		\begin{itemize}[leftmargin=0pt]
			\item[\textbullet] Note : \textbf{\large 8.1}
			\item[\textbullet] Rang : \textbf{9}
			\item[\textbullet] Traité : 50 \%
		\end{itemize}
	} &
	\alertbox{\faChartLine}{Statistiques des notes}{
        \psset{xunit=1cm, yunit=1cm,fillstyle=solid}
		\begin{pspicture}(0,-0.1)(16,1.45)
		   \savedata{\data}[10.3 16.4 19.2 8.0 8.1 4.8 6.6 0.9 7.5 14.8 16.4 13.6 19.7 7.0 9.0]
		   \rput{-90}(0,0.9){\psBoxplot[barwidth=1.1cm,yunit=0.5,fillcolor=gray,linewidth=1pt]{\data}}
		   \psaxes[yAxis=false,dx=1cm,Dx=2,labelsep=1pt,linecolor=gray,xlabelFontSize=\scriptstyle](0,0)(10.1,4)
		   \psdot[dotsize=8pt,dotstyle=diamond,linecolor=black,fillstyle=solid,fillcolor=white,linewidth=1pt](4.05,0.85)
           \psdot[dotsize=6pt,dotstyle=x,linecolor=black,linewidth=3pt](5.409999999999999,0.85)
		   \end{pspicture}
	} \\
    
\end{tabularx}\\
\begin{tabularx}{\textwidth}{X}
\alertbox{\faComment}{Commentaire}
{
	Il y a de bonnes choses mais tu n’as pas traité suffisamment de question pour atteindre la moyenne. Il faut mieux gérer le temps !
}
\end{tabularx}
\medskip
    \ding{113} \textbf{\sffamily{Résultats par thème}} \medskip \\
    \renewcommand{\arraystretch}{1.2}
    \begin{tabular}{|l|r|r|}
    \cline{2-3}
    \multicolumn{1}{l|}{} & \multicolumn{1}{|c|}{Points} & \multicolumn{1}{|c|}{Traitées} \\
    \hline
    {Comprendre un algorithme} & 90\% \;{\small (09/10)} & 100\% \;{\small (2/2)} \\ \hline {Base d'OCaml (fonctionnel)} & 23\% \;{\small (20/85)} & 18\% \;{\small (2/11)} \\ \hline {Programmation de base en C} & 40\% \;{\small (10/25)} & 50\% \;{\small (2/4)} \\ \hline {Types structurés en C et pointeurs} & 48\% \;{\small (22/45)} & 75\% \;{\small (3/4)} \\ \hline {Structure de données séquentielles} & 100\% \;{\small (05/5)} & 100\% \;{\small (1/1)} \\ \hline {Complexité d'un algorithme} & 45\% \;{\small (27/60)} & 62\% \;{\small (5/8)} \\ \hline \end{tabular} \\\\\medskip \\
    \ding{113} \textbf{\sffamily{Résultats par exercice}} \medskip \\
    \renewcommand{\arraystretch}{1.2}
    \begin{tabular}{|l|r|r|}
    \cline{2-3}
    \multicolumn{1}{l|}{} & \multicolumn{1}{|c|}{Points} & \multicolumn{1}{|c|}{Traitées} \\
    \hline
    Exercice {1} & 81\% \;{\small (61/75)} & 90\% \;{\small (9/10)} \\ \hline Exercice {2} & 53\% \;{\small (32/60)} & 100\% \;{\small (6/6)} \\ \hline Exercice {3} & 0\% \;{\small (00/40)} & 0\% \;{\small (0/6)} \\ \hline Exercice {4} & 0\% \;{\small (00/55)} & 0\% \;{\small (0/8)} \\ \hline \end{tabular} \\\\
    \vspace{0.5cm}\\
    \ding{113} \textbf{\sffamily{Historique des notes}} \medskip \\
    \psset{xunit=1.4cm, yunit=0.2cm}
    \begin{pspicture}(-1,-1)(12,22)

% --- Axe X : numéros de devoir ---
\psaxes[
    Dx=1,
    Dy=2,
    Ox=0,
    Oy=0,
    labels=all,
    ticks=all,
](0,0)(0,0)(7,20)


\listplot[plotstyle=line,showpoints=true,linecolor=black,linewidth=0.7pt, dotstyle=diamond*, dotsize=0.2]{%
    1 8.4
2 9
3 8.1
}

% Minimum de la classe
\listplot[plotstyle=line,showpoints=true,linecolor=gray,linewidth=0.7pt, dotstyle=|, dotangle=90, dotsize=0.15, linestyle=dotted]{%
    1 5.8
2 4.6
3 0.9
}

% Maximum de la classe
\listplot[plotstyle=line,showpoints=true,linecolor=gray,linewidth=0.7pt, dotstyle=|, dotangle=90, dotsize=0.15, linestyle=dotted]{%
    1 18.9
2 18.4
3 19.7
}

% Moyenne de la classe
\listplot[plotstyle=line,showpoints=true,linecolor=gray,linewidth=0.7pt, dotstyle=x, dotsize = 0.15, linestyle=dashed]{%
    1 12.4466666666667
2 13.1933333333333
3 10.82
}
\end{pspicture}
\pagebreak
\begin{tcolorbox}[enhanced,width=\textwidth,center upper,fontupper=\bfseries,drop shadow southwest,sharp corners]
{\sc \large Courounadin-Mouny} Maxence
\end{tcolorbox}
\medskip
\begin{tabularx}{\textwidth}{p{5cm}X}
	\alertbox{\faAward}{Note}{
		\begin{itemize}[leftmargin=0pt]
			\item[\textbullet] Note : \textbf{\large 4.8}
			\item[\textbullet] Rang : \textbf{14}
			\item[\textbullet] Traité : 60 \%
		\end{itemize}
	} &
	\alertbox{\faChartLine}{Statistiques des notes}{
        \psset{xunit=1cm, yunit=1cm,fillstyle=solid}
		\begin{pspicture}(0,-0.1)(16,1.45)
		   \savedata{\data}[10.3 16.4 19.2 8.0 8.1 4.8 6.6 0.9 7.5 14.8 16.4 13.6 19.7 7.0 9.0]
		   \rput{-90}(0,0.9){\psBoxplot[barwidth=1.1cm,yunit=0.5,fillcolor=gray,linewidth=1pt]{\data}}
		   \psaxes[yAxis=false,dx=1cm,Dx=2,labelsep=1pt,linecolor=gray,xlabelFontSize=\scriptstyle](0,0)(10.1,4)
		   \psdot[dotsize=8pt,dotstyle=diamond,linecolor=black,fillstyle=solid,fillcolor=white,linewidth=1pt](2.4,0.85)
           \psdot[dotsize=6pt,dotstyle=x,linecolor=black,linewidth=3pt](5.409999999999999,0.85)
		   \end{pspicture}
	} \\
    
\end{tabularx}\\
\begin{tabularx}{\textwidth}{X}
\alertbox{\faComment}{Commentaire}
{
	C’est très insuffisant, il faut revoir en priorité la manipulation des listes en Ocaml et les exercices associées. Pour l’exercice sur les listes chainées tu n’as pas bien compris l’énoncé. C’était une structure de file.
}
\end{tabularx}
\medskip
    \ding{113} \textbf{\sffamily{Résultats par thème}} \medskip \\
    \renewcommand{\arraystretch}{1.2}
    \begin{tabular}{|l|r|r|}
    \cline{2-3}
    \multicolumn{1}{l|}{} & \multicolumn{1}{|c|}{Points} & \multicolumn{1}{|c|}{Traitées} \\
    \hline
    {Comprendre un algorithme} & 40\% \;{\small (04/10)} & 100\% \;{\small (2/2)} \\ \hline {Base d'OCaml (fonctionnel)} & 8\% \;{\small (07/85)} & 54\% \;{\small (6/11)} \\ \hline {Programmation de base en C} & 8\% \;{\small (02/25)} & 50\% \;{\small (2/4)} \\ \hline {Types structurés en C et pointeurs} & 42\% \;{\small (19/45)} & 75\% \;{\small (3/4)} \\ \hline {Structure de données séquentielles} & 0\% \;{\small (00/5)} & 100\% \;{\small (1/1)} \\ \hline {Complexité d'un algorithme} & 38\% \;{\small (23/60)} & 50\% \;{\small (4/8)} \\ \hline \end{tabular} \\\\\medskip \\
    \ding{113} \textbf{\sffamily{Résultats par exercice}} \medskip \\
    \renewcommand{\arraystretch}{1.2}
    \begin{tabular}{|l|r|r|}
    \cline{2-3}
    \multicolumn{1}{l|}{} & \multicolumn{1}{|c|}{Points} & \multicolumn{1}{|c|}{Traitées} \\
    \hline
    Exercice {1} & 25\% \;{\small (19/75)} & 80\% \;{\small (8/10)} \\ \hline Exercice {2} & 48\% \;{\small (29/60)} & 100\% \;{\small (6/6)} \\ \hline Exercice {3} & 12\% \;{\small (05/40)} & 50\% \;{\small (3/6)} \\ \hline Exercice {4} & 3\% \;{\small (02/55)} & 12\% \;{\small (1/8)} \\ \hline \end{tabular} \\\\
    \vspace{0.5cm}\\
    \ding{113} \textbf{\sffamily{Historique des notes}} \medskip \\
    \psset{xunit=1.4cm, yunit=0.2cm}
    \begin{pspicture}(-1,-1)(12,22)

% --- Axe X : numéros de devoir ---
\psaxes[
    Dx=1,
    Dy=2,
    Ox=0,
    Oy=0,
    labels=all,
    ticks=all,
](0,0)(0,0)(7,20)


\listplot[plotstyle=line,showpoints=true,linecolor=black,linewidth=0.7pt, dotstyle=diamond*, dotsize=0.2]{%
    1 10.9
2 11.1
3 4.8
}

% Minimum de la classe
\listplot[plotstyle=line,showpoints=true,linecolor=gray,linewidth=0.7pt, dotstyle=|, dotangle=90, dotsize=0.15, linestyle=dotted]{%
    1 5.8
2 4.6
3 0.9
}

% Maximum de la classe
\listplot[plotstyle=line,showpoints=true,linecolor=gray,linewidth=0.7pt, dotstyle=|, dotangle=90, dotsize=0.15, linestyle=dotted]{%
    1 18.9
2 18.4
3 19.7
}

% Moyenne de la classe
\listplot[plotstyle=line,showpoints=true,linecolor=gray,linewidth=0.7pt, dotstyle=x, dotsize = 0.15, linestyle=dashed]{%
    1 12.4466666666667
2 13.1933333333333
3 10.82
}
\end{pspicture}
\pagebreak
\begin{tcolorbox}[enhanced,width=\textwidth,center upper,fontupper=\bfseries,drop shadow southwest,sharp corners]
{\sc \large Dominguez} Raphaël
\end{tcolorbox}
\medskip
\begin{tabularx}{\textwidth}{p{5cm}X}
	\alertbox{\faAward}{Note}{
		\begin{itemize}[leftmargin=0pt]
			\item[\textbullet] Note : \textbf{\large 6.6}
			\item[\textbullet] Rang : \textbf{13}
			\item[\textbullet] Traité : 53 \%
		\end{itemize}
	} &
	\alertbox{\faChartLine}{Statistiques des notes}{
        \psset{xunit=1cm, yunit=1cm,fillstyle=solid}
		\begin{pspicture}(0,-0.1)(16,1.45)
		   \savedata{\data}[10.3 16.4 19.2 8.0 8.1 4.8 6.6 0.9 7.5 14.8 16.4 13.6 19.7 7.0 9.0]
		   \rput{-90}(0,0.9){\psBoxplot[barwidth=1.1cm,yunit=0.5,fillcolor=gray,linewidth=1pt]{\data}}
		   \psaxes[yAxis=false,dx=1cm,Dx=2,labelsep=1pt,linecolor=gray,xlabelFontSize=\scriptstyle](0,0)(10.1,4)
		   \psdot[dotsize=8pt,dotstyle=diamond,linecolor=black,fillstyle=solid,fillcolor=white,linewidth=1pt](3.3,0.85)
           \psdot[dotsize=6pt,dotstyle=x,linecolor=black,linewidth=3pt](5.409999999999999,0.85)
		   \end{pspicture}
	} \\
    
\end{tabularx}\\
\begin{tabularx}{\textwidth}{X}
\alertbox{\faComment}{Commentaire}
{
	Tu n’as pas traité suffisamment de question, tu manques d’automatismes en programmation (probablement par manque de pratique). C’est dommage, les idées et le raisonnement sont bons !
}
\end{tabularx}
\medskip
    \ding{113} \textbf{\sffamily{Résultats par thème}} \medskip \\
    \renewcommand{\arraystretch}{1.2}
    \begin{tabular}{|l|r|r|}
    \cline{2-3}
    \multicolumn{1}{l|}{} & \multicolumn{1}{|c|}{Points} & \multicolumn{1}{|c|}{Traitées} \\
    \hline
    {Comprendre un algorithme} & 100\% \;{\small (10/10)} & 100\% \;{\small (2/2)} \\ \hline {Base d'OCaml (fonctionnel)} & 21\% \;{\small (18/85)} & 45\% \;{\small (5/11)} \\ \hline {Programmation de base en C} & 40\% \;{\small (10/25)} & 50\% \;{\small (2/4)} \\ \hline {Types structurés en C et pointeurs} & 6\% \;{\small (03/45)} & 50\% \;{\small (2/4)} \\ \hline {Structure de données séquentielles} & 100\% \;{\small (05/5)} & 100\% \;{\small (1/1)} \\ \hline {Complexité d'un algorithme} & 50\% \;{\small (30/60)} & 50\% \;{\small (4/8)} \\ \hline \end{tabular} \\\\\medskip \\
    \ding{113} \textbf{\sffamily{Résultats par exercice}} \medskip \\
    \renewcommand{\arraystretch}{1.2}
    \begin{tabular}{|l|r|r|}
    \cline{2-3}
    \multicolumn{1}{l|}{} & \multicolumn{1}{|c|}{Points} & \multicolumn{1}{|c|}{Traitées} \\
    \hline
    Exercice {1} & 70\% \;{\small (53/75)} & 90\% \;{\small (9/10)} \\ \hline Exercice {2} & 21\% \;{\small (13/60)} & 66\% \;{\small (4/6)} \\ \hline Exercice {3} & 25\% \;{\small (10/40)} & 50\% \;{\small (3/6)} \\ \hline Exercice {4} & 0\% \;{\small (00/55)} & 0\% \;{\small (0/8)} \\ \hline \end{tabular} \\\\
    \vspace{0.5cm}\\
    \ding{113} \textbf{\sffamily{Historique des notes}} \medskip \\
    \psset{xunit=1.4cm, yunit=0.2cm}
    \begin{pspicture}(-1,-1)(12,22)

% --- Axe X : numéros de devoir ---
\psaxes[
    Dx=1,
    Dy=2,
    Ox=0,
    Oy=0,
    labels=all,
    ticks=all,
](0,0)(0,0)(7,20)


\listplot[plotstyle=line,showpoints=true,linecolor=black,linewidth=0.7pt, dotstyle=diamond*, dotsize=0.2]{%
    1 15.7
2 16.1
3 6.6
}

% Minimum de la classe
\listplot[plotstyle=line,showpoints=true,linecolor=gray,linewidth=0.7pt, dotstyle=|, dotangle=90, dotsize=0.15, linestyle=dotted]{%
    1 5.8
2 4.6
3 0.9
}

% Maximum de la classe
\listplot[plotstyle=line,showpoints=true,linecolor=gray,linewidth=0.7pt, dotstyle=|, dotangle=90, dotsize=0.15, linestyle=dotted]{%
    1 18.9
2 18.4
3 19.7
}

% Moyenne de la classe
\listplot[plotstyle=line,showpoints=true,linecolor=gray,linewidth=0.7pt, dotstyle=x, dotsize = 0.15, linestyle=dashed]{%
    1 12.4466666666667
2 13.1933333333333
3 10.82
}
\end{pspicture}
\pagebreak
\begin{tcolorbox}[enhanced,width=\textwidth,center upper,fontupper=\bfseries,drop shadow southwest,sharp corners]
{\sc \large Garbal} Alizée
\end{tcolorbox}
\medskip
\begin{tabularx}{\textwidth}{p{5cm}X}
	\alertbox{\faAward}{Note}{
		\begin{itemize}[leftmargin=0pt]
			\item[\textbullet] Note : \textbf{\large 0.9}
			\item[\textbullet] Rang : \textbf{15}
			\item[\textbullet] Traité : 37 \%
		\end{itemize}
	} &
	\alertbox{\faChartLine}{Statistiques des notes}{
        \psset{xunit=1cm, yunit=1cm,fillstyle=solid}
		\begin{pspicture}(0,-0.1)(16,1.45)
		   \savedata{\data}[10.3 16.4 19.2 8.0 8.1 4.8 6.6 0.9 7.5 14.8 16.4 13.6 19.7 7.0 9.0]
		   \rput{-90}(0,0.9){\psBoxplot[barwidth=1.1cm,yunit=0.5,fillcolor=gray,linewidth=1pt]{\data}}
		   \psaxes[yAxis=false,dx=1cm,Dx=2,labelsep=1pt,linecolor=gray,xlabelFontSize=\scriptstyle](0,0)(10.1,4)
		   \psdot[dotsize=8pt,dotstyle=diamond,linecolor=black,fillstyle=solid,fillcolor=white,linewidth=1pt](0.45,0.85)
           \psdot[dotsize=6pt,dotstyle=x,linecolor=black,linewidth=3pt](5.409999999999999,0.85)
		   \end{pspicture}
	} \\
    
\end{tabularx}\\
\begin{tabularx}{\textwidth}{X}
\alertbox{\faComment}{Commentaire}
{
	Les lacunes et les difficultés s’accumulent, il devient de plus en plus difficile de rattraper le retard !
}
\end{tabularx}
\medskip
    \ding{113} \textbf{\sffamily{Résultats par thème}} \medskip \\
    \renewcommand{\arraystretch}{1.2}
    \begin{tabular}{|l|r|r|}
    \cline{2-3}
    \multicolumn{1}{l|}{} & \multicolumn{1}{|c|}{Points} & \multicolumn{1}{|c|}{Traitées} \\
    \hline
    {Comprendre un algorithme} & 40\% \;{\small (04/10)} & 100\% \;{\small (2/2)} \\ \hline {Base d'OCaml (fonctionnel)} & 4\% \;{\small (04/85)} & 36\% \;{\small (4/11)} \\ \hline {Programmation de base en C} & 4\% \;{\small (01/25)} & 50\% \;{\small (2/4)} \\ \hline {Types structurés en C et pointeurs} & 0\% \;{\small (00/45)} & 25\% \;{\small (1/4)} \\ \hline {Structure de données séquentielles} & 0\% \;{\small (00/5)} & 0\% \;{\small (0/1)} \\ \hline {Complexité d'un algorithme} & 1\% \;{\small (01/60)} & 25\% \;{\small (2/8)} \\ \hline \end{tabular} \\\\\medskip \\
    \ding{113} \textbf{\sffamily{Résultats par exercice}} \medskip \\
    \renewcommand{\arraystretch}{1.2}
    \begin{tabular}{|l|r|r|}
    \cline{2-3}
    \multicolumn{1}{l|}{} & \multicolumn{1}{|c|}{Points} & \multicolumn{1}{|c|}{Traitées} \\
    \hline
    Exercice {1} & 10\% \;{\small (08/75)} & 80\% \;{\small (8/10)} \\ \hline Exercice {2} & 0\% \;{\small (00/60)} & 33\% \;{\small (2/6)} \\ \hline Exercice {3} & 5\% \;{\small (02/40)} & 16\% \;{\small (1/6)} \\ \hline Exercice {4} & 0\% \;{\small (00/55)} & 0\% \;{\small (0/8)} \\ \hline \end{tabular} \\\\
    \vspace{0.5cm}\\
    \ding{113} \textbf{\sffamily{Historique des notes}} \medskip \\
    \psset{xunit=1.4cm, yunit=0.2cm}
    \begin{pspicture}(-1,-1)(12,22)

% --- Axe X : numéros de devoir ---
\psaxes[
    Dx=1,
    Dy=2,
    Ox=0,
    Oy=0,
    labels=all,
    ticks=all,
](0,0)(0,0)(7,20)


\listplot[plotstyle=line,showpoints=true,linecolor=black,linewidth=0.7pt, dotstyle=diamond*, dotsize=0.2]{%
    1 5.8
2 4.6
3 0.9
}

% Minimum de la classe
\listplot[plotstyle=line,showpoints=true,linecolor=gray,linewidth=0.7pt, dotstyle=|, dotangle=90, dotsize=0.15, linestyle=dotted]{%
    1 5.8
2 4.6
3 0.9
}

% Maximum de la classe
\listplot[plotstyle=line,showpoints=true,linecolor=gray,linewidth=0.7pt, dotstyle=|, dotangle=90, dotsize=0.15, linestyle=dotted]{%
    1 18.9
2 18.4
3 19.7
}

% Moyenne de la classe
\listplot[plotstyle=line,showpoints=true,linecolor=gray,linewidth=0.7pt, dotstyle=x, dotsize = 0.15, linestyle=dashed]{%
    1 12.4466666666667
2 13.1933333333333
3 10.82
}
\end{pspicture}
\pagebreak
\begin{tcolorbox}[enhanced,width=\textwidth,center upper,fontupper=\bfseries,drop shadow southwest,sharp corners]
{\sc \large Hoarau} Alessandro
\end{tcolorbox}
\medskip
\begin{tabularx}{\textwidth}{p{5cm}X}
	\alertbox{\faAward}{Note}{
		\begin{itemize}[leftmargin=0pt]
			\item[\textbullet] Note : \textbf{\large 7.5}
			\item[\textbullet] Rang : \textbf{11}
			\item[\textbullet] Traité : 60 \%
		\end{itemize}
	} &
	\alertbox{\faChartLine}{Statistiques des notes}{
        \psset{xunit=1cm, yunit=1cm,fillstyle=solid}
		\begin{pspicture}(0,-0.1)(16,1.45)
		   \savedata{\data}[10.3 16.4 19.2 8.0 8.1 4.8 6.6 0.9 7.5 14.8 16.4 13.6 19.7 7.0 9.0]
		   \rput{-90}(0,0.9){\psBoxplot[barwidth=1.1cm,yunit=0.5,fillcolor=gray,linewidth=1pt]{\data}}
		   \psaxes[yAxis=false,dx=1cm,Dx=2,labelsep=1pt,linecolor=gray,xlabelFontSize=\scriptstyle](0,0)(10.1,4)
		   \psdot[dotsize=8pt,dotstyle=diamond,linecolor=black,fillstyle=solid,fillcolor=white,linewidth=1pt](3.75,0.85)
           \psdot[dotsize=6pt,dotstyle=x,linecolor=black,linewidth=3pt](5.409999999999999,0.85)
		   \end{pspicture}
	} \\
    
\end{tabularx}\\
\begin{tabularx}{\textwidth}{X}
\alertbox{\faComment}{Commentaire}
{
	C’est en progrès mais encore en dessous de la moyenne, tu ne traites pas suffisamment de question, il faut gagner en rapidité.La manipulation des listes en Ocaml doit aussi être retravaillée.
}
\end{tabularx}
\medskip
    \ding{113} \textbf{\sffamily{Résultats par thème}} \medskip \\
    \renewcommand{\arraystretch}{1.2}
    \begin{tabular}{|l|r|r|}
    \cline{2-3}
    \multicolumn{1}{l|}{} & \multicolumn{1}{|c|}{Points} & \multicolumn{1}{|c|}{Traitées} \\
    \hline
    {Comprendre un algorithme} & 90\% \;{\small (09/10)} & 100\% \;{\small (2/2)} \\ \hline {Base d'OCaml (fonctionnel)} & 20\% \;{\small (17/85)} & 63\% \;{\small (7/11)} \\ \hline {Programmation de base en C} & 40\% \;{\small (10/25)} & 50\% \;{\small (2/4)} \\ \hline {Types structurés en C et pointeurs} & 55\% \;{\small (25/45)} & 75\% \;{\small (3/4)} \\ \hline {Structure de données séquentielles} & 100\% \;{\small (05/5)} & 100\% \;{\small (1/1)} \\ \hline {Complexité d'un algorithme} & 33\% \;{\small (20/60)} & 37\% \;{\small (3/8)} \\ \hline \end{tabular} \\\\\medskip \\
    \ding{113} \textbf{\sffamily{Résultats par exercice}} \medskip \\
    \renewcommand{\arraystretch}{1.2}
    \begin{tabular}{|l|r|r|}
    \cline{2-3}
    \multicolumn{1}{l|}{} & \multicolumn{1}{|c|}{Points} & \multicolumn{1}{|c|}{Traitées} \\
    \hline
    Exercice {1} & 34\% \;{\small (26/75)} & 80\% \;{\small (8/10)} \\ \hline Exercice {2} & 75\% \;{\small (45/60)} & 100\% \;{\small (6/6)} \\ \hline Exercice {3} & 37\% \;{\small (15/40)} & 66\% \;{\small (4/6)} \\ \hline Exercice {4} & 0\% \;{\small (00/55)} & 0\% \;{\small (0/8)} \\ \hline \end{tabular} \\\\
    \vspace{0.5cm}\\
    \ding{113} \textbf{\sffamily{Historique des notes}} \medskip \\
    \psset{xunit=1.4cm, yunit=0.2cm}
    \begin{pspicture}(-1,-1)(12,22)

% --- Axe X : numéros de devoir ---
\psaxes[
    Dx=1,
    Dy=2,
    Ox=0,
    Oy=0,
    labels=all,
    ticks=all,
](0,0)(0,0)(7,20)


\listplot[plotstyle=line,showpoints=true,linecolor=black,linewidth=0.7pt, dotstyle=diamond*, dotsize=0.2]{%
    1 8
2 8.2
3 7.5
}

% Minimum de la classe
\listplot[plotstyle=line,showpoints=true,linecolor=gray,linewidth=0.7pt, dotstyle=|, dotangle=90, dotsize=0.15, linestyle=dotted]{%
    1 5.8
2 4.6
3 0.9
}

% Maximum de la classe
\listplot[plotstyle=line,showpoints=true,linecolor=gray,linewidth=0.7pt, dotstyle=|, dotangle=90, dotsize=0.15, linestyle=dotted]{%
    1 18.9
2 18.4
3 19.7
}

% Moyenne de la classe
\listplot[plotstyle=line,showpoints=true,linecolor=gray,linewidth=0.7pt, dotstyle=x, dotsize = 0.15, linestyle=dashed]{%
    1 12.4466666666667
2 13.1933333333333
3 10.82
}
\end{pspicture}
\pagebreak
\begin{tcolorbox}[enhanced,width=\textwidth,center upper,fontupper=\bfseries,drop shadow southwest,sharp corners]
{\sc \large Mahomed Issop} Jérémy
\end{tcolorbox}
\medskip
\begin{tabularx}{\textwidth}{p{5cm}X}
	\alertbox{\faAward}{Note}{
		\begin{itemize}[leftmargin=0pt]
			\item[\textbullet] Note : \textbf{\large 14.8}
			\item[\textbullet] Rang : \textbf{5}
			\item[\textbullet] Traité : 87 \%
		\end{itemize}
	} &
	\alertbox{\faChartLine}{Statistiques des notes}{
        \psset{xunit=1cm, yunit=1cm,fillstyle=solid}
		\begin{pspicture}(0,-0.1)(16,1.45)
		   \savedata{\data}[10.3 16.4 19.2 8.0 8.1 4.8 6.6 0.9 7.5 14.8 16.4 13.6 19.7 7.0 9.0]
		   \rput{-90}(0,0.9){\psBoxplot[barwidth=1.1cm,yunit=0.5,fillcolor=gray,linewidth=1pt]{\data}}
		   \psaxes[yAxis=false,dx=1cm,Dx=2,labelsep=1pt,linecolor=gray,xlabelFontSize=\scriptstyle](0,0)(10.1,4)
		   \psdot[dotsize=8pt,dotstyle=diamond,linecolor=black,fillstyle=solid,fillcolor=white,linewidth=1pt](7.4,0.85)
           \psdot[dotsize=6pt,dotstyle=x,linecolor=black,linewidth=3pt](5.409999999999999,0.85)
		   \end{pspicture}
	} \\
    
\end{tabularx}\\
\begin{tabularx}{\textwidth}{X}
\alertbox{\faComment}{Commentaire}
{
	Très bon travail dans l’ensemble. Encore des petites erreurs sur les listes chainées (je te conseille de refaire et de coder en les testant les fonctions demandées). Attention à ta fonction de fusion de deux listes triées !
}
\end{tabularx}
\medskip
    \ding{113} \textbf{\sffamily{Résultats par thème}} \medskip \\
    \renewcommand{\arraystretch}{1.2}
    \begin{tabular}{|l|r|r|}
    \cline{2-3}
    \multicolumn{1}{l|}{} & \multicolumn{1}{|c|}{Points} & \multicolumn{1}{|c|}{Traitées} \\
    \hline
    {Comprendre un algorithme} & 90\% \;{\small (09/10)} & 100\% \;{\small (2/2)} \\ \hline {Base d'OCaml (fonctionnel)} & 69\% \;{\small (59/85)} & 72\% \;{\small (8/11)} \\ \hline {Programmation de base en C} & 60\% \;{\small (15/25)} & 100\% \;{\small (4/4)} \\ \hline {Types structurés en C et pointeurs} & 75\% \;{\small (34/45)} & 100\% \;{\small (4/4)} \\ \hline {Structure de données séquentielles} & 100\% \;{\small (05/5)} & 100\% \;{\small (1/1)} \\ \hline {Complexité d'un algorithme} & 80\% \;{\small (48/60)} & 87\% \;{\small (7/8)} \\ \hline \end{tabular} \\\\\medskip \\
    \ding{113} \textbf{\sffamily{Résultats par exercice}} \medskip \\
    \renewcommand{\arraystretch}{1.2}
    \begin{tabular}{|l|r|r|}
    \cline{2-3}
    \multicolumn{1}{l|}{} & \multicolumn{1}{|c|}{Points} & \multicolumn{1}{|c|}{Traitées} \\
    \hline
    Exercice {1} & 72\% \;{\small (54/75)} & 90\% \;{\small (9/10)} \\ \hline Exercice {2} & 81\% \;{\small (49/60)} & 100\% \;{\small (6/6)} \\ \hline Exercice {3} & 50\% \;{\small (20/40)} & 50\% \;{\small (3/6)} \\ \hline Exercice {4} & 85\% \;{\small (47/55)} & 100\% \;{\small (8/8)} \\ \hline \end{tabular} \\\\
    \vspace{0.5cm}\\
    \ding{113} \textbf{\sffamily{Historique des notes}} \medskip \\
    \psset{xunit=1.4cm, yunit=0.2cm}
    \begin{pspicture}(-1,-1)(12,22)

% --- Axe X : numéros de devoir ---
\psaxes[
    Dx=1,
    Dy=2,
    Ox=0,
    Oy=0,
    labels=all,
    ticks=all,
](0,0)(0,0)(7,20)


\listplot[plotstyle=line,showpoints=true,linecolor=black,linewidth=0.7pt, dotstyle=diamond*, dotsize=0.2]{%
    1 13.5
2 15
3 14.8
}

% Minimum de la classe
\listplot[plotstyle=line,showpoints=true,linecolor=gray,linewidth=0.7pt, dotstyle=|, dotangle=90, dotsize=0.15, linestyle=dotted]{%
    1 5.8
2 4.6
3 0.9
}

% Maximum de la classe
\listplot[plotstyle=line,showpoints=true,linecolor=gray,linewidth=0.7pt, dotstyle=|, dotangle=90, dotsize=0.15, linestyle=dotted]{%
    1 18.9
2 18.4
3 19.7
}

% Moyenne de la classe
\listplot[plotstyle=line,showpoints=true,linecolor=gray,linewidth=0.7pt, dotstyle=x, dotsize = 0.15, linestyle=dashed]{%
    1 12.4466666666667
2 13.1933333333333
3 10.82
}
\end{pspicture}
\pagebreak
\begin{tcolorbox}[enhanced,width=\textwidth,center upper,fontupper=\bfseries,drop shadow southwest,sharp corners]
{\sc \large Mamodhoussen} Djavad
\end{tcolorbox}
\medskip
\begin{tabularx}{\textwidth}{p{5cm}X}
	\alertbox{\faAward}{Note}{
		\begin{itemize}[leftmargin=0pt]
			\item[\textbullet] Note : \textbf{\large 16.4}
			\item[\textbullet] Rang : \textbf{4}
			\item[\textbullet] Traité : 100 \%
		\end{itemize}
	} &
	\alertbox{\faChartLine}{Statistiques des notes}{
        \psset{xunit=1cm, yunit=1cm,fillstyle=solid}
		\begin{pspicture}(0,-0.1)(16,1.45)
		   \savedata{\data}[10.3 16.4 19.2 8.0 8.1 4.8 6.6 0.9 7.5 14.8 16.4 13.6 19.7 7.0 9.0]
		   \rput{-90}(0,0.9){\psBoxplot[barwidth=1.1cm,yunit=0.5,fillcolor=gray,linewidth=1pt]{\data}}
		   \psaxes[yAxis=false,dx=1cm,Dx=2,labelsep=1pt,linecolor=gray,xlabelFontSize=\scriptstyle](0,0)(10.1,4)
		   \psdot[dotsize=8pt,dotstyle=diamond,linecolor=black,fillstyle=solid,fillcolor=white,linewidth=1pt](8.2,0.85)
           \psdot[dotsize=6pt,dotstyle=x,linecolor=black,linewidth=3pt](5.409999999999999,0.85)
		   \end{pspicture}
	} \\
    
\end{tabularx}\\
\begin{tabularx}{\textwidth}{X}
\alertbox{\faComment}{Commentaire}
{
	Très bon travail, il y a encore des ajustements à faire sur la manipulation des listes de Ocaml. Tu as tendance à vouloir utiliser de l’impératif.
}
\end{tabularx}
\medskip
    \ding{113} \textbf{\sffamily{Résultats par thème}} \medskip \\
    \renewcommand{\arraystretch}{1.2}
    \begin{tabular}{|l|r|r|}
    \cline{2-3}
    \multicolumn{1}{l|}{} & \multicolumn{1}{|c|}{Points} & \multicolumn{1}{|c|}{Traitées} \\
    \hline
    {Comprendre un algorithme} & 100\% \;{\small (10/10)} & 100\% \;{\small (2/2)} \\ \hline {Base d'OCaml (fonctionnel)} & 75\% \;{\small (64/85)} & 100\% \;{\small (11/11)} \\ \hline {Programmation de base en C} & 68\% \;{\small (17/25)} & 100\% \;{\small (4/4)} \\ \hline {Types structurés en C et pointeurs} & 80\% \;{\small (36/45)} & 100\% \;{\small (4/4)} \\ \hline {Structure de données séquentielles} & 100\% \;{\small (05/5)} & 100\% \;{\small (1/1)} \\ \hline {Complexité d'un algorithme} & 95\% \;{\small (57/60)} & 100\% \;{\small (8/8)} \\ \hline \end{tabular} \\\\\medskip \\
    \ding{113} \textbf{\sffamily{Résultats par exercice}} \medskip \\
    \renewcommand{\arraystretch}{1.2}
    \begin{tabular}{|l|r|r|}
    \cline{2-3}
    \multicolumn{1}{l|}{} & \multicolumn{1}{|c|}{Points} & \multicolumn{1}{|c|}{Traitées} \\
    \hline
    Exercice {1} & 100\% \;{\small (75/75)} & 100\% \;{\small (10/10)} \\ \hline Exercice {2} & 85\% \;{\small (51/60)} & 100\% \;{\small (6/6)} \\ \hline Exercice {3} & 57\% \;{\small (23/40)} & 100\% \;{\small (6/6)} \\ \hline Exercice {4} & 72\% \;{\small (40/55)} & 100\% \;{\small (8/8)} \\ \hline \end{tabular} \\\\
    \vspace{0.5cm}\\
    \ding{113} \textbf{\sffamily{Historique des notes}} \medskip \\
    \psset{xunit=1.4cm, yunit=0.2cm}
    \begin{pspicture}(-1,-1)(12,22)

% --- Axe X : numéros de devoir ---
\psaxes[
    Dx=1,
    Dy=2,
    Ox=0,
    Oy=0,
    labels=all,
    ticks=all,
](0,0)(0,0)(7,20)


\listplot[plotstyle=line,showpoints=true,linecolor=black,linewidth=0.7pt, dotstyle=diamond*, dotsize=0.2]{%
    1 17.8
2 18
3 16.4
}

% Minimum de la classe
\listplot[plotstyle=line,showpoints=true,linecolor=gray,linewidth=0.7pt, dotstyle=|, dotangle=90, dotsize=0.15, linestyle=dotted]{%
    1 5.8
2 4.6
3 0.9
}

% Maximum de la classe
\listplot[plotstyle=line,showpoints=true,linecolor=gray,linewidth=0.7pt, dotstyle=|, dotangle=90, dotsize=0.15, linestyle=dotted]{%
    1 18.9
2 18.4
3 19.7
}

% Moyenne de la classe
\listplot[plotstyle=line,showpoints=true,linecolor=gray,linewidth=0.7pt, dotstyle=x, dotsize = 0.15, linestyle=dashed]{%
    1 12.4466666666667
2 13.1933333333333
3 10.82
}
\end{pspicture}
\pagebreak
\begin{tcolorbox}[enhanced,width=\textwidth,center upper,fontupper=\bfseries,drop shadow southwest,sharp corners]
{\sc \large Morel} Lucas
\end{tcolorbox}
\medskip
\begin{tabularx}{\textwidth}{p{5cm}X}
	\alertbox{\faAward}{Note}{
		\begin{itemize}[leftmargin=0pt]
			\item[\textbullet] Note : \textbf{\large 13.6}
			\item[\textbullet] Rang : \textbf{6}
			\item[\textbullet] Traité : 90 \%
		\end{itemize}
	} &
	\alertbox{\faChartLine}{Statistiques des notes}{
        \psset{xunit=1cm, yunit=1cm,fillstyle=solid}
		\begin{pspicture}(0,-0.1)(16,1.45)
		   \savedata{\data}[10.3 16.4 19.2 8.0 8.1 4.8 6.6 0.9 7.5 14.8 16.4 13.6 19.7 7.0 9.0]
		   \rput{-90}(0,0.9){\psBoxplot[barwidth=1.1cm,yunit=0.5,fillcolor=gray,linewidth=1pt]{\data}}
		   \psaxes[yAxis=false,dx=1cm,Dx=2,labelsep=1pt,linecolor=gray,xlabelFontSize=\scriptstyle](0,0)(10.1,4)
		   \psdot[dotsize=8pt,dotstyle=diamond,linecolor=black,fillstyle=solid,fillcolor=white,linewidth=1pt](6.8,0.85)
           \psdot[dotsize=6pt,dotstyle=x,linecolor=black,linewidth=3pt](5.409999999999999,0.85)
		   \end{pspicture}
	} \\
    
\end{tabularx}\\
\begin{tabularx}{\textwidth}{X}
\alertbox{\faComment}{Commentaire}
{
	Bon travail dans l’ensemble. C’est encore un peu confus (mais il y a les idées) sur les listes chainées. Je te conseille de refaire l’exercice 2 en codant et en testant vraiment les fonctions (le corrigé est en ligne). 
}
\end{tabularx}
\medskip
    \ding{113} \textbf{\sffamily{Résultats par thème}} \medskip \\
    \renewcommand{\arraystretch}{1.2}
    \begin{tabular}{|l|r|r|}
    \cline{2-3}
    \multicolumn{1}{l|}{} & \multicolumn{1}{|c|}{Points} & \multicolumn{1}{|c|}{Traitées} \\
    \hline
    {Comprendre un algorithme} & 100\% \;{\small (10/10)} & 100\% \;{\small (2/2)} \\ \hline {Base d'OCaml (fonctionnel)} & 75\% \;{\small (64/85)} & 81\% \;{\small (9/11)} \\ \hline {Programmation de base en C} & 60\% \;{\small (15/25)} & 75\% \;{\small (3/4)} \\ \hline {Types structurés en C et pointeurs} & 48\% \;{\small (22/45)} & 100\% \;{\small (4/4)} \\ \hline {Structure de données séquentielles} & 80\% \;{\small (04/5)} & 100\% \;{\small (1/1)} \\ \hline {Complexité d'un algorithme} & 68\% \;{\small (41/60)} & 100\% \;{\small (8/8)} \\ \hline \end{tabular} \\\\\medskip \\
    \ding{113} \textbf{\sffamily{Résultats par exercice}} \medskip \\
    \renewcommand{\arraystretch}{1.2}
    \begin{tabular}{|l|r|r|}
    \cline{2-3}
    \multicolumn{1}{l|}{} & \multicolumn{1}{|c|}{Points} & \multicolumn{1}{|c|}{Traitées} \\
    \hline
    Exercice {1} & 85\% \;{\small (64/75)} & 100\% \;{\small (10/10)} \\ \hline Exercice {2} & 50\% \;{\small (30/60)} & 100\% \;{\small (6/6)} \\ \hline Exercice {3} & 62\% \;{\small (25/40)} & 66\% \;{\small (4/6)} \\ \hline Exercice {4} & 67\% \;{\small (37/55)} & 87\% \;{\small (7/8)} \\ \hline \end{tabular} \\\\
    \vspace{0.5cm}\\
    \ding{113} \textbf{\sffamily{Historique des notes}} \medskip \\
    \psset{xunit=1.4cm, yunit=0.2cm}
    \begin{pspicture}(-1,-1)(12,22)

% --- Axe X : numéros de devoir ---
\psaxes[
    Dx=1,
    Dy=2,
    Ox=0,
    Oy=0,
    labels=all,
    ticks=all,
](0,0)(0,0)(7,20)


\listplot[plotstyle=line,showpoints=true,linecolor=black,linewidth=0.7pt, dotstyle=diamond*, dotsize=0.2]{%
    1 16.7
2 15.5
3 13.6
}

% Minimum de la classe
\listplot[plotstyle=line,showpoints=true,linecolor=gray,linewidth=0.7pt, dotstyle=|, dotangle=90, dotsize=0.15, linestyle=dotted]{%
    1 5.8
2 4.6
3 0.9
}

% Maximum de la classe
\listplot[plotstyle=line,showpoints=true,linecolor=gray,linewidth=0.7pt, dotstyle=|, dotangle=90, dotsize=0.15, linestyle=dotted]{%
    1 18.9
2 18.4
3 19.7
}

% Moyenne de la classe
\listplot[plotstyle=line,showpoints=true,linecolor=gray,linewidth=0.7pt, dotstyle=x, dotsize = 0.15, linestyle=dashed]{%
    1 12.4466666666667
2 13.1933333333333
3 10.82
}
\end{pspicture}
\pagebreak
\begin{tcolorbox}[enhanced,width=\textwidth,center upper,fontupper=\bfseries,drop shadow southwest,sharp corners]
{\sc \large Randriamiarivola Korodo} Lionel
\end{tcolorbox}
\medskip
\begin{tabularx}{\textwidth}{p{5cm}X}
	\alertbox{\faAward}{Note}{
		\begin{itemize}[leftmargin=0pt]
			\item[\textbullet] Note : \textbf{\large 19.7}
			\item[\textbullet] Rang : \textbf{1}
			\item[\textbullet] Traité : 100 \%
		\end{itemize}
	} &
	\alertbox{\faChartLine}{Statistiques des notes}{
        \psset{xunit=1cm, yunit=1cm,fillstyle=solid}
		\begin{pspicture}(0,-0.1)(16,1.45)
		   \savedata{\data}[10.3 16.4 19.2 8.0 8.1 4.8 6.6 0.9 7.5 14.8 16.4 13.6 19.7 7.0 9.0]
		   \rput{-90}(0,0.9){\psBoxplot[barwidth=1.1cm,yunit=0.5,fillcolor=gray,linewidth=1pt]{\data}}
		   \psaxes[yAxis=false,dx=1cm,Dx=2,labelsep=1pt,linecolor=gray,xlabelFontSize=\scriptstyle](0,0)(10.1,4)
		   \psdot[dotsize=8pt,dotstyle=diamond,linecolor=black,fillstyle=solid,fillcolor=white,linewidth=1pt](9.85,0.85)
           \psdot[dotsize=6pt,dotstyle=x,linecolor=black,linewidth=3pt](5.409999999999999,0.85)
		   \end{pspicture}
	} \\
    
\end{tabularx}\\
\begin{tabularx}{\textwidth}{X}
\alertbox{\faComment}{Commentaire}
{
	Excellent travail ! L’exercice sur la liste chainée circulaire est particulièrement très bien traité alors que cela n’avait pas été vu en cous, bravo ! 
}
\end{tabularx}
\medskip
    \ding{113} \textbf{\sffamily{Résultats par thème}} \medskip \\
    \renewcommand{\arraystretch}{1.2}
    \begin{tabular}{|l|r|r|}
    \cline{2-3}
    \multicolumn{1}{l|}{} & \multicolumn{1}{|c|}{Points} & \multicolumn{1}{|c|}{Traitées} \\
    \hline
    {Comprendre un algorithme} & 100\% \;{\small (10/10)} & 100\% \;{\small (2/2)} \\ \hline {Base d'OCaml (fonctionnel)} & 100\% \;{\small (85/85)} & 100\% \;{\small (11/11)} \\ \hline {Programmation de base en C} & 100\% \;{\small (25/25)} & 100\% \;{\small (4/4)} \\ \hline {Types structurés en C et pointeurs} & 93\% \;{\small (42/45)} & 100\% \;{\small (4/4)} \\ \hline {Structure de données séquentielles} & 100\% \;{\small (05/5)} & 100\% \;{\small (1/1)} \\ \hline {Complexité d'un algorithme} & 100\% \;{\small (60/60)} & 100\% \;{\small (8/8)} \\ \hline \end{tabular} \\\\\medskip \\
    \ding{113} \textbf{\sffamily{Résultats par exercice}} \medskip \\
    \renewcommand{\arraystretch}{1.2}
    \begin{tabular}{|l|r|r|}
    \cline{2-3}
    \multicolumn{1}{l|}{} & \multicolumn{1}{|c|}{Points} & \multicolumn{1}{|c|}{Traitées} \\
    \hline
    Exercice {1} & 100\% \;{\small (75/75)} & 100\% \;{\small (10/10)} \\ \hline Exercice {2} & 95\% \;{\small (57/60)} & 100\% \;{\small (6/6)} \\ \hline Exercice {3} & 100\% \;{\small (40/40)} & 100\% \;{\small (6/6)} \\ \hline Exercice {4} & 100\% \;{\small (55/55)} & 100\% \;{\small (8/8)} \\ \hline \end{tabular} \\\\
    \vspace{0.5cm}\\
    \ding{113} \textbf{\sffamily{Historique des notes}} \medskip \\
    \psset{xunit=1.4cm, yunit=0.2cm}
    \begin{pspicture}(-1,-1)(12,22)

% --- Axe X : numéros de devoir ---
\psaxes[
    Dx=1,
    Dy=2,
    Ox=0,
    Oy=0,
    labels=all,
    ticks=all,
](0,0)(0,0)(7,20)


\listplot[plotstyle=line,showpoints=true,linecolor=black,linewidth=0.7pt, dotstyle=diamond*, dotsize=0.2]{%
    1 18.6
2 17.5
3 19.7
}

% Minimum de la classe
\listplot[plotstyle=line,showpoints=true,linecolor=gray,linewidth=0.7pt, dotstyle=|, dotangle=90, dotsize=0.15, linestyle=dotted]{%
    1 5.8
2 4.6
3 0.9
}

% Maximum de la classe
\listplot[plotstyle=line,showpoints=true,linecolor=gray,linewidth=0.7pt, dotstyle=|, dotangle=90, dotsize=0.15, linestyle=dotted]{%
    1 18.9
2 18.4
3 19.7
}

% Moyenne de la classe
\listplot[plotstyle=line,showpoints=true,linecolor=gray,linewidth=0.7pt, dotstyle=x, dotsize = 0.15, linestyle=dashed]{%
    1 12.4466666666667
2 13.1933333333333
3 10.82
}
\end{pspicture}
\pagebreak
\begin{tcolorbox}[enhanced,width=\textwidth,center upper,fontupper=\bfseries,drop shadow southwest,sharp corners]
{\sc \large Rasolofotsara} Ando
\end{tcolorbox}
\medskip
\begin{tabularx}{\textwidth}{p{5cm}X}
	\alertbox{\faAward}{Note}{
		\begin{itemize}[leftmargin=0pt]
			\item[\textbullet] Note : \textbf{\large 7.0}
			\item[\textbullet] Rang : \textbf{12}
			\item[\textbullet] Traité : 70 \%
		\end{itemize}
	} &
	\alertbox{\faChartLine}{Statistiques des notes}{
        \psset{xunit=1cm, yunit=1cm,fillstyle=solid}
		\begin{pspicture}(0,-0.1)(16,1.45)
		   \savedata{\data}[10.3 16.4 19.2 8.0 8.1 4.8 6.6 0.9 7.5 14.8 16.4 13.6 19.7 7.0 9.0]
		   \rput{-90}(0,0.9){\psBoxplot[barwidth=1.1cm,yunit=0.5,fillcolor=gray,linewidth=1pt]{\data}}
		   \psaxes[yAxis=false,dx=1cm,Dx=2,labelsep=1pt,linecolor=gray,xlabelFontSize=\scriptstyle](0,0)(10.1,4)
		   \psdot[dotsize=8pt,dotstyle=diamond,linecolor=black,fillstyle=solid,fillcolor=white,linewidth=1pt](3.5,0.85)
           \psdot[dotsize=6pt,dotstyle=x,linecolor=black,linewidth=3pt](5.409999999999999,0.85)
		   \end{pspicture}
	} \\
    
\end{tabularx}\\
\begin{tabularx}{\textwidth}{X}
\alertbox{\faComment}{Commentaire}
{
	La manipulation des listes en Ocaml doit vraiment être retravaillées. Commence par les exercices simples du cours et des TP.
}
\end{tabularx}
\medskip
    \ding{113} \textbf{\sffamily{Résultats par thème}} \medskip \\
    \renewcommand{\arraystretch}{1.2}
    \begin{tabular}{|l|r|r|}
    \cline{2-3}
    \multicolumn{1}{l|}{} & \multicolumn{1}{|c|}{Points} & \multicolumn{1}{|c|}{Traitées} \\
    \hline
    {Comprendre un algorithme} & 100\% \;{\small (10/10)} & 100\% \;{\small (2/2)} \\ \hline {Base d'OCaml (fonctionnel)} & 12\% \;{\small (11/85)} & 54\% \;{\small (6/11)} \\ \hline {Programmation de base en C} & 56\% \;{\small (14/25)} & 75\% \;{\small (3/4)} \\ \hline {Types structurés en C et pointeurs} & 24\% \;{\small (11/45)} & 100\% \;{\small (4/4)} \\ \hline {Structure de données séquentielles} & 100\% \;{\small (05/5)} & 100\% \;{\small (1/1)} \\ \hline {Complexité d'un algorithme} & 50\% \;{\small (30/60)} & 62\% \;{\small (5/8)} \\ \hline \end{tabular} \\\\\medskip \\
    \ding{113} \textbf{\sffamily{Résultats par exercice}} \medskip \\
    \renewcommand{\arraystretch}{1.2}
    \begin{tabular}{|l|r|r|}
    \cline{2-3}
    \multicolumn{1}{l|}{} & \multicolumn{1}{|c|}{Points} & \multicolumn{1}{|c|}{Traitées} \\
    \hline
    Exercice {1} & 54\% \;{\small (41/75)} & 90\% \;{\small (9/10)} \\ \hline Exercice {2} & 26\% \;{\small (16/60)} & 83\% \;{\small (5/6)} \\ \hline Exercice {3} & 20\% \;{\small (08/40)} & 50\% \;{\small (3/6)} \\ \hline Exercice {4} & 29\% \;{\small (16/55)} & 50\% \;{\small (4/8)} \\ \hline \end{tabular} \\\\
    \vspace{0.5cm}\\
    \ding{113} \textbf{\sffamily{Historique des notes}} \medskip \\
    \psset{xunit=1.4cm, yunit=0.2cm}
    \begin{pspicture}(-1,-1)(12,22)

% --- Axe X : numéros de devoir ---
\psaxes[
    Dx=1,
    Dy=2,
    Ox=0,
    Oy=0,
    labels=all,
    ticks=all,
](0,0)(0,0)(7,20)


\listplot[plotstyle=line,showpoints=true,linecolor=black,linewidth=0.7pt, dotstyle=diamond*, dotsize=0.2]{%
    1 9.5
2 10
3 7
}

% Minimum de la classe
\listplot[plotstyle=line,showpoints=true,linecolor=gray,linewidth=0.7pt, dotstyle=|, dotangle=90, dotsize=0.15, linestyle=dotted]{%
    1 5.8
2 4.6
3 0.9
}

% Maximum de la classe
\listplot[plotstyle=line,showpoints=true,linecolor=gray,linewidth=0.7pt, dotstyle=|, dotangle=90, dotsize=0.15, linestyle=dotted]{%
    1 18.9
2 18.4
3 19.7
}

% Moyenne de la classe
\listplot[plotstyle=line,showpoints=true,linecolor=gray,linewidth=0.7pt, dotstyle=x, dotsize = 0.15, linestyle=dashed]{%
    1 12.4466666666667
2 13.1933333333333
3 10.82
}
\end{pspicture}
\pagebreak
\begin{tcolorbox}[enhanced,width=\textwidth,center upper,fontupper=\bfseries,drop shadow southwest,sharp corners]
{\sc \large Silotia} Donovan
\end{tcolorbox}
\medskip
\begin{tabularx}{\textwidth}{p{5cm}X}
	\alertbox{\faAward}{Note}{
		\begin{itemize}[leftmargin=0pt]
			\item[\textbullet] Note : \textbf{\large 9.0}
			\item[\textbullet] Rang : \textbf{8}
			\item[\textbullet] Traité : 77 \%
		\end{itemize}
	} &
	\alertbox{\faChartLine}{Statistiques des notes}{
        \psset{xunit=1cm, yunit=1cm,fillstyle=solid}
		\begin{pspicture}(0,-0.1)(16,1.45)
		   \savedata{\data}[10.3 16.4 19.2 8.0 8.1 4.8 6.6 0.9 7.5 14.8 16.4 13.6 19.7 7.0 9.0]
		   \rput{-90}(0,0.9){\psBoxplot[barwidth=1.1cm,yunit=0.5,fillcolor=gray,linewidth=1pt]{\data}}
		   \psaxes[yAxis=false,dx=1cm,Dx=2,labelsep=1pt,linecolor=gray,xlabelFontSize=\scriptstyle](0,0)(10.1,4)
		   \psdot[dotsize=8pt,dotstyle=diamond,linecolor=black,fillstyle=solid,fillcolor=white,linewidth=1pt](4.5,0.85)
           \psdot[dotsize=6pt,dotstyle=x,linecolor=black,linewidth=3pt](5.409999999999999,0.85)
		   \end{pspicture}
	} \\
    
\end{tabularx}\\
\begin{tabularx}{\textwidth}{X}
\alertbox{\faComment}{Commentaire}
{
	C’est bien tu t’approches de la moyenne. Il faut progresser dans la manipulation des listes en Ocaml. Tu raisonnes trop en impératif en essayant de modifier des listes (qui sont non mutables).
}
\end{tabularx}
\medskip
    \ding{113} \textbf{\sffamily{Résultats par thème}} \medskip \\
    \renewcommand{\arraystretch}{1.2}
    \begin{tabular}{|l|r|r|}
    \cline{2-3}
    \multicolumn{1}{l|}{} & \multicolumn{1}{|c|}{Points} & \multicolumn{1}{|c|}{Traitées} \\
    \hline
    {Comprendre un algorithme} & 100\% \;{\small (10/10)} & 100\% \;{\small (2/2)} \\ \hline {Base d'OCaml (fonctionnel)} & 44\% \;{\small (38/85)} & 100\% \;{\small (11/11)} \\ \hline {Programmation de base en C} & 40\% \;{\small (10/25)} & 50\% \;{\small (2/4)} \\ \hline {Types structurés en C et pointeurs} & 13\% \;{\small (06/45)} & 25\% \;{\small (1/4)} \\ \hline {Structure de données séquentielles} & 100\% \;{\small (05/5)} & 100\% \;{\small (1/1)} \\ \hline {Complexité d'un algorithme} & 56\% \;{\small (34/60)} & 75\% \;{\small (6/8)} \\ \hline \end{tabular} \\\\\medskip \\
    \ding{113} \textbf{\sffamily{Résultats par exercice}} \medskip \\
    \renewcommand{\arraystretch}{1.2}
    \begin{tabular}{|l|r|r|}
    \cline{2-3}
    \multicolumn{1}{l|}{} & \multicolumn{1}{|c|}{Points} & \multicolumn{1}{|c|}{Traitées} \\
    \hline
    Exercice {1} & 81\% \;{\small (61/75)} & 100\% \;{\small (10/10)} \\ \hline Exercice {2} & 26\% \;{\small (16/60)} & 50\% \;{\small (3/6)} \\ \hline Exercice {3} & 37\% \;{\small (15/40)} & 100\% \;{\small (6/6)} \\ \hline Exercice {4} & 20\% \;{\small (11/55)} & 50\% \;{\small (4/8)} \\ \hline \end{tabular} \\\\
    \vspace{0.5cm}\\
    \ding{113} \textbf{\sffamily{Historique des notes}} \medskip \\
    \psset{xunit=1.4cm, yunit=0.2cm}
    \begin{pspicture}(-1,-1)(12,22)

% --- Axe X : numéros de devoir ---
\psaxes[
    Dx=1,
    Dy=2,
    Ox=0,
    Oy=0,
    labels=all,
    ticks=all,
](0,0)(0,0)(7,20)


\listplot[plotstyle=line,showpoints=true,linecolor=black,linewidth=0.7pt, dotstyle=diamond*, dotsize=0.2]{%
    1 11.6
2 15.4
3 9
}

% Minimum de la classe
\listplot[plotstyle=line,showpoints=true,linecolor=gray,linewidth=0.7pt, dotstyle=|, dotangle=90, dotsize=0.15, linestyle=dotted]{%
    1 5.8
2 4.6
3 0.9
}

% Maximum de la classe
\listplot[plotstyle=line,showpoints=true,linecolor=gray,linewidth=0.7pt, dotstyle=|, dotangle=90, dotsize=0.15, linestyle=dotted]{%
    1 18.9
2 18.4
3 19.7
}

% Moyenne de la classe
\listplot[plotstyle=line,showpoints=true,linecolor=gray,linewidth=0.7pt, dotstyle=x, dotsize = 0.15, linestyle=dashed]{%
    1 12.4466666666667
2 13.1933333333333
3 10.82
}
\end{pspicture}
\pagebreak\end{document}
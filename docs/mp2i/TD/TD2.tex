\documentclass[11pt,a4paper]{article}

\usepackage{Act}

\begin{document}
\input{\detokenize{/home/fenarius/Travail/Cours/cpge-info/latex/Macros.tex}}
\ModeExercice
\TD{2}{Validation et tests}

\begin{Exercise}[title={Spécifications}]\\
Proposer un nom, une spécification, des préconditions et un jeu de tests pour les fonctions suivantes :
\Question{\
\begin{langageC}
bool fonction1(int a, int b, int c)
	{return (a==b) || (b==c) || (a==c);}
\end{langageC}
}
\Question{\ 
\begin{langageC}
float fonction2(float x, float y)
	{return 1/(x*x+y*y);}
\end{langageC}
}
\Question{\
\begin{langageC}
float fonction3(float a, float b) {
	if (a<b)
	{return a;}
	else 
	{return b;}
	}
\end{langageC}
}
\end{Exercise}

\begin{Exercise}[title={Fonction mystère}]\\
On considère la fonction {\tt mystere} suivante :
	\begin{langageC}
bool mystere(int n) {
	int d=2;
	while (d*d<=n)
	{	if (n%d==0)
			{return false;}
		d=d+1;}
	return true;
}
\end{langageC}
\Question{Nommer cette fonction et en donner une spécification.}
\Question{Tracer son graphe de flot de contrôle.}
\Question{Proposer un jeu de tests permettant de couvrir tous les arcs.}
\end{Exercise}

\begin{Exercise}[title={nombre de jours dans un mois}]
	\Question{Ecrire une fonction {\tt nb\_jours} qui prend en argument un entier {\tt mois} et un entier {\tt annee} et qui renvoie le nombre de jours de ce mois. Par exemple, {\tt nb\_jours(5,1970)} doit renvoyer le nombres de jours du mois de mai 1970. On pourra utiliser sans la réécrire la fonction {\tt bissextile} vue en cours.}
	\Question{Proposer des préconditions pour cette fonction.}
	\Question{Proposer un jeu de tests pour cette fonction.}
\end{Exercise}

\begin{Exercise}[title={Triangles}]
\Question{Ecrire une fonction {\tt triangle} qui prend en argument trois entiers et renvoie :
\begin{itemize}
	\item 0 si les trois entiers ne sont pas les côtés d'un triangle
	\item 1 si les trois entiers sont les côtés d'un triangle scalène
	\item 2 si les trois entiers sont les côtés d'un triangle isocèle non rectangle
	\item 3 si les trois entiers sont les côtés d'un triangle équilatéral
	\item 4 si les trois entiers sont les côtés d'un triangle rectangle
\end{itemize}
\Question{Tracer le graphe de flot de contrôle de cette fonction.}
\Question{Proposer un jeu de tests pour cette fonction.}
}
\end{Exercise}

\begin{Exercise}[title = {Compte à rebours}]\\
	On considère la fonction C suivante :
	\inputpartC{rebours.c}{}{}{4}{13}
	\Question{Cette fonction est-elle conforme à sa spécification ?}
	\Question{$n$ est-il un variant de boucle ? Sinon quelle propriété est manquante ?}
\end{Exercise}

\begin{Exercise}[title = {Maximum des éléments d'un tableauœ}]
	\Question{Ecrire en C une fonction qui renvoie le maximum des éléments d'un tableau non vide d'entiers. Par exemple pour le tableau {\tt \{2, 5, 4, 9, 7, 6\}}, la fonction renvoie {\tt 9}.}
	\Question{Prouver la correction de cette fonction.}
\end{Exercise}



\begin{Exercise}[title = {Somme des éléments pairs}]
	\Question{Ecrire en C une fonction qui renvoie la somme des éléments pairs d'un tableau. Par exemple pour le tableau {\tt \{2, 5, 4, 9, 7, 6\}}, la fonction renvoie {\tt 2 + 4 + 6 = 12}.}
	\Question{Prouver la correction de cette fonction.}
\end{Exercise}




\begin{Exercise}[title = {Etude d'un algorithme}]\\
	On considère l'algorithme suivant : \\
	\SetAlFnt{\small}
	\setlength{\algomargin}{8pt}
	\begin{algorithm}[H]
		\DontPrintSemicolon
		\caption{Quotient dans la division euclidienne}
		\Entree{$a \in \N, b \in \N^*$}
		\Sortie{$q \in \N$, quotient de $a$ par $b$}
		\everypar={\footnotesize \textcolor{gray}{\nl}}
		$q \leftarrow 0$\;
		$r \leftarrow a$\;
		\Tq{$r \geqslant 0$}{
		$q \leftarrow q+1$ \;
		$r \leftarrow r-b$ \;
		}
		\Return $q - 1$
	  \end{algorithm}

\Question{Proposer une implémentation de cet algorithme en langage C, sous la forme d'une fonction dont on précisera soigneusement la spécification.}
\Question{Etudier la terminaison de cet algorithme.}
\Question{Etudier la correction de cet algorithme.}
\end{Exercise}

\begin{Exercise}[title = {Multiplication à la russe}]\\
	On considère l'algorithme suivant : \\
	\SetAlFnt{\small}
	\setlength{\algomargin}{8pt}
	\begin{algorithm}[H]
		\DontPrintSemicolon
		\caption{Multiplication à la russe}
		\Entree{$a \in \N, b \in \N$}
		\Sortie{$ab$}
		\everypar={\footnotesize \textcolor{gray}{\nl}}
		$m \leftarrow 0$\;
		\Tq{$a >0 $ }{
			\Si{$a \mod 2 =1$}
			{ $m \leftarrow m + b$ \;
			  $a \leftarrow a - 1$\;
			}
			\Sinon
			{$a \leftarrow a / 2$\;
			 $b \leftarrow b + b$ \;
			}
		}
		\Return $m$
	  \end{algorithm}
	  \Question{Faire fonctionner cet algorithme à la main pour $a = 7$ et $b=5$.}
	  \Question{Donner une implémentation en C de cette algorithme sous la forme d'une fonction dont on précisera soigneusement la spécification}
	  \Question{Prouver la terminaison de cet algorithme. \\
	 	{\small \aide \;} $a$ est un variant de boucle. 
	  }
	  \Question{Prouver la correction de cet algorithme. \\
	 	{\small \aide\;} $m + ab$ vaut toujours le produit des valeurs initales de $a$ et de $b$.} 

\end{Exercise}

\begin{Exercise}[title = {Tri par sélection}] \\
	L'algorithme du tri par sélection d'un tableau $t$ de longueur $n$, consiste, pour chaque entier $i$ de $0$ à $n-1$ à :
	\begin{itemize}
	\item rechercher le plus petit élément du sous tableau $\{t[i], \dots, t[n-1]\}$.
	\item l'échanger avec celui situé à l'indice $i$.
	\end{itemize}
	\Question{Ecrire cet algorithme en pseudo langage}
	\Question{En donner une implémentation en langage C en spécifiant soigneusement les fonctions utilisées.}
	\Question{Proposer un jeu de tests}
	\Question{Prouver la correction totale de cet algorithme.}
\end{Exercise}

\end{document}
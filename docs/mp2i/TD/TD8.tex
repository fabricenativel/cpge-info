\documentclass[11pt,a4paper]{article}

\usepackage{Act}

\begin{document}
\input{\detokenize{/home/fenarius/Travail/Cours/cpge-info/latex/Macros.tex}}
\ModeExercice
\TD{7}{Structure de données linéaires}

\setcounter{Exercise}{0}

\begin{Exercise}[title={Inversion au sommet}]\\
    On suppose qu'on dispose d'une structure de données de type pile dotée de son interface habituelle c'est à dire {\tt empiler}, {\tt dépiler} et {\tt est\_vide}.
    Proposer une suite d'opérations permettant d'inverser, lorsqu'ils existent, les deux éléments situés au sommet de cette pile.
    Si la pile contient moins de deux éléments, elle doit rester en l'état.
\end{Exercise}



\begin{Exercise}[title={Un exemple de complexité amortie}]\\
On prend l'exemple de la structure de données implémentant le type de \textit{list} de Python grâce à un tableau dynamique en C (voir TP). On considère un tableau donc la taille initiale est 1 et sur lequel on effectue $n$ opérations {\tt append}. Le but de l'exercice est de montrer qu'on obtiendra alors un nombre d'opérations en $O(n)$. On dira alors que le {\tt append} a une \textbf{complexité amortie} en $O(1)$.

\Question{Montrer que le coût du redimensionnement d'un tableau de taille $n$ est un $O(n).$} 
\Question{Montrer que $n$ opérations {\tt append} vont nécessiter $\lfloor\log_2(n)\rfloor$ redimensionnement de tableau}.
\Question{Donner la taille des tableaux lors de chaque redimensionnement.}
\Question{En déduire le coût des redimensionnements.}
\Question{Montrer que le coût total est un $O(n)$.}
\end{Exercise}

\begin{Exercise}[title={Implémentation d'une file avec deux piles}]
    On reprend l'implémentation d'une file avec deux piles (voir TP), le but de l'exercice est d'établir la complexité de {\tt défiler}
    \Question{Donner les complexités dans le pire et le meilleur des cas}
    \Question{On suppose à présent qu'en tout on a enfilé et défilé $n$ éléments. Montrer que le nombre total d'opérations nécessaire est un $O(n)$.}
\end{Exercise}
    

\end{document}
\documentclass[11pt,a4paper]{article}

\usepackage{Act}

\begin{document}
\input{\detokenize{/home/fenarius/Travail/Cours/cpge-info/latex/Macros.tex}}
\ModeExercice
\TD{1}{Langage C}

\setcounter{Exercise}{0}
\begin{Exercise}[title={Pointeurs}]\\
    On considère le programme suivant :
    \begin{langageC*}{fontsize=\small}
int a = 4;
int b = 1;
int c = 2;
int* p;
int* q;
p = &a;
q = &c;
*p = *q + 1;
p = q;
q = &b;
*p = *p - *q;
*q = *q + 1;
*p *= *q;
    \end{langageC*}
Compléter le tableau suivant, donner l'état des variables au cours de l'exécution du programme :
\begin{center}
    \def\arraystretch{1.2}
    \setlength\tabcolsep{0.5cm}
    \begin{tabular}{l|c|c|c|c|c}
    & a & b & c & p & q \\
    \hline
    initialisation & 4 & 1 & 2 & ? & ?\\
    {\tt p = \&a;}  & 4 & 1 & 2 & \&a & ?\\
    {\tt q = \&c;}  &  &  &  &  \\
    {\tt *p = *q + 1;}  &  &  &  &  \\
    {\tt p = q;}  &  &  &  &  \\
    {\tt q = \&b;}  &  &  &  &  \\
    {\tt *p = *p - *q;}  &  &  &  &  \\
    {\tt *q = *q + 1;}  &  &  &  &  \\
    {\tt *p *= *q;}  &  &  &  &  \\
    \end{tabular}
\end{center}
\end{Exercise}

\begin{Exercise}[title={{\tt printf} et {\tt scanf}}]
    \Question{Ecrire l'instruction permettant d'afficher une variable {\tt n} de type entier avec {\tt printf}}
    \Question{Ecrire l'instruction permettant de saisir au clavier une variable {\tt n} de type entier avec {\tt scanf}}
    \Question{Expliquer la différence entre le mode de passage de {\tt n} dans ces deux fonctions}
\end{Exercise}

\begin{Exercise}[title={{\tt Pointeurs}}] \\
On considère le programme suivant :
\inputC{/home/fenarius/Travail/Cours/cpge-info/docs/mp2i/files/C3/pointeur.c}{\small}
\Question{Ce programme est-il correct ?}
\Question{Proposer une correction.}
\end{Exercise}

\begin{Exercise}[title={Incrémenter une variable}]\\
La fonction suivante doit incrémenter la variable {\tt n} donnée en argument :
\begin{langageC*}{fontsize=\small}
void incremente(int x) {
    x = x + 1;
    }    
\end{langageC*}
\Question{Commenter}
\Question{Proposer une correction.}
\end{Exercise}

\end{document}
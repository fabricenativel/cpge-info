\documentclass[11pt,a4paper]{article}

\usepackage{Act}

\begin{document}
\newcommand{\ModeExercice}{
% Traduction des noms pour le package exercise
\renewcommand{\ExerciseName}{Exercice}
\renewcommand{\thesubQuestion}{\theQuestion.\alph{subQuestion}}
\renewcommand{\AnswerName}{Réponses à l'exercise }
}
\newcommand{\Fiche}[2]{\lhead{\textbf{{\sc #1}}}
\rhead{Niveau: \textbf{#2}}
\cfoot{}}
\definecolor{cfond}{gray}{0.4}
\renewcommand{\thealgocf}{}

\newcommand{\ModeActivite}{
% Traduction des noms pour le package exercise
\renewcommand{\ExerciseName}{Activité}
}
% Réglages de la mise en forme des exercices 
\renewcommand{\ExerciseHeaderTitle}{\ExerciseTitle}
\renewcommand{\ExerciseHeaderOrigin}{\ExerciseOrigin}
% Un : sépare le numéro de l'exerice de son titre ... si le titre existe. On utilise Origine pour placer les pictogrammes en fin de ligne
\renewcommand{\ExerciseHeader}{\ding{113} \textbf{\sffamily{\ExerciseName \ \ExerciseHeaderNB}} \ifthenelse{\equal{\ExerciseTitle}{\empty}}{}{:} \textit{\ExerciseHeaderTitle} \hfill \ExerciseHeaderOrigin}
\renewcommand{\ExePartHeader}{\quad {\footnotesize \ding{110}} \textbf{Partie \textbf{\ExePartHeaderNB}} : \ExePartName}


% Mode Concours
\newcommand{\ModeConcours}{
   \newcounter{qconcours}
   \setcounter{qconcours}{1}
   \renewcommand{\ExerciseName}{Exercice}
   \renewcommand{\ExerciseHeaderTitle}{\ExerciseTitle}
\renewcommand{\ExerciseHeaderOrigin}{\ExerciseOrigin}
\renewcommand{\ExerciseHeader}{\ding{113} \textbf{\sffamily{\ExerciseName \ \ExerciseHeaderNB}} \ifthenelse{\equal{\ExerciseTitle}{\empty}}{}{:} \textit{\ExerciseHeaderTitle} \hfill \ExerciseHeaderOrigin}
\renewcommand{\QuestionNB}{\textbf{Q\arabic{qconcours}--}\ \addtocounter{qconcours}{1}}}

\newcommand{\noeud}[1]{\Tr{\fbox{\tt #1}}}
\newcommand{\FPATH}{/home/fenarius/Travail/Cours/cpge-info/docs/mp2i/}
\newcommand{\spath}[2]{\FPATH Evaluations/#1/#1#2}
\definecolor{codebg}{gray}{0.90}
\newcommand{\inputpartOCaml}[5]{\begin{mdframed}[backgroundcolor=codebg] \inputminted[breaklines=true,fontsize=#3,linenos=true,highlightcolor=fluo,tabsize=2,highlightlines={#2},firstline=#4,lastline=#5,firstnumber=1]{OCaml}{#1} \end{mdframed}}
\newcommand{\inputpartPython}[5]{\begin{mdframed}[backgroundcolor=codebg] \inputminted[breaklines=true,fontsize=#3,linenos=true,highlightcolor=fluo,tabsize=2,highlightlines={#2},firstline=#4,lastline=#5,firstnumber=1]{python}{#1} \end{mdframed}}
\newcommand{\inputpartC}[5]{\begin{mdframed}[backgroundcolor=codebg] \inputminted[breaklines=true,fontsize=#3,linenos=true,highlightcolor=fluo,tabsize=2,highlightlines={#2},firstline=#4,lastline=#5,firstnumber=1]{c}{#1} \end{mdframed}}
\newcommand{\inputC}[2]{\begin{mdframed}[backgroundcolor=codebg] \inputminted[breaklines=true,fontsize=#2,linenos=true,highlightcolor=fluo,tabsize=2]{c}{#1} \end{mdframed}}
\newminted[langageC]{c}{linenos=true,escapeinside=``,highlightcolor=fluo,tabsize=2}
\newminted[python]{python}{linenos=true,escapeinside=``,highlightcolor=fluo,tabsize=4}
\BeforeBeginEnvironment{minted}{\begin{mdframed}[backgroundcolor=codebg,skipabove=0cm]}
   \AfterEndEnvironment{minted}{\end{mdframed}}

% Font light medium et bold pour tt :
\newcommand{\ttl}[1]{\ttfamily \fontseries{l}\selectfont #1}
\newcommand{\ttm}[1]{\ttfamily \fontseries{m}\selectfont #1}
\newcommand{\ttb}[1]{\ttfamily \fontseries{b}\selectfont #1}

%QCM de NSI \QNSI{Question}{R1}{R2}{R3}{R4}
\newcommand{\QNSI}[5]{
#1
\begin{enumerate}[label=\alph{enumi})]
\item #2
\item #3
\item #4
\item #5
\end{enumerate}
}



\definecolor{grispale}{gray}{0.95}
\newcommand{\htmlmode}{\lstset{language=html,numbers=left, tabsize=2, frame=single, breaklines=true, keywordstyle=\ttfamily, basicstyle=\small,
   numberstyle=\tiny\ttfamily, framexleftmargin=0mm, backgroundcolor=\color{grispale}, xleftmargin=12mm,showstringspaces=false}}
\newcommand{\pythonmode}{\lstset{language=python,numbers=left, tabsize=4, frame=single, breaklines=true, keywordstyle=\ttfamily, basicstyle=\small,
   numberstyle=\tiny\ttfamily, framexleftmargin=0mm, backgroundcolor=\color{grispale}, xleftmargin=12mm, showstringspaces=false}}
\newcommand{\bashmode}{\lstset{language=bash,numbers=left, tabsize=2, frame=single, breaklines=true, basicstyle=\ttfamily,
   numberstyle=\tiny\ttfamily, framexleftmargin=0mm, backgroundcolor=\color{grispale}, xleftmargin=12mm, showstringspaces=false}}
\newcommand{\exomode}{\lstset{language=python,numbers=left, tabsize=2, frame=single, breaklines=true, basicstyle=\ttfamily,
   numberstyle=\tiny\ttfamily, framexleftmargin=13mm, xleftmargin=12mm, basicstyle=\small, showstringspaces=false}}
   
   
  \lstset{%
        inputencoding=utf8,
        extendedchars=true,
        literate=%
        {é}{{\'{e}}}1
        {è}{{\`{e}}}1
        {ê}{{\^{e}}}1
        {ë}{{\¨{e}}}1
        {É}{{\'{E}}}1
        {Ê}{{\^{E}}}1
        {û}{{\^{u}}}1
        {ù}{{\`{u}}}1
        {ú}{{\'{u}}}1
        {â}{{\^{a}}}1
        {à}{{\`{a}}}1
        {á}{{\'{a}}}1
        {ã}{{\~{a}}}1
        {Á}{{\'{A}}}1
        {Â}{{\^{A}}}1
        {Ã}{{\~{A}}}1
        {ç}{{\c{c}}}1
        {Ç}{{\c{C}}}1
        {õ}{{\~{o}}}1
        {ó}{{\'{o}}}1
        {ô}{{\^{o}}}1
        {Õ}{{\~{O}}}1
        {Ó}{{\'{O}}}1
        {Ô}{{\^{O}}}1
        {î}{{\^{i}}}1
        {Î}{{\^{I}}}1
        {í}{{\'{i}}}1
        {Í}{{\~{Í}}}1
}

%tei pour placer les images
%tei{nom de l’image}{échelle de l’image}{sens}{texte a positionner}
%sens ="1" (droite) ou "2" (gauche)
\newlength{\ltxt}
\newcommand{\tei}[4]{
\setlength{\ltxt}{\linewidth}
\setbox0=\hbox{\includegraphics[scale=#2]{#1}}
\addtolength{\ltxt}{-\wd0}
\addtolength{\ltxt}{-10pt}
\ifthenelse{\equal{#3}{1}}{
\begin{minipage}{\wd0}
\includegraphics[scale=#2]{#1}
\end{minipage}
\hfill
\begin{minipage}{\ltxt}
#4
\end{minipage}
}{
\begin{minipage}{\ltxt}
#4
\end{minipage}
\hfill
\begin{minipage}{\wd0}
\includegraphics[scale=#2]{#1}
\end{minipage}
}
}

%Juxtaposition d'une image pspciture et de texte 
%#1: = code pstricks de l'image
%#2: largeur de l'image
%#3: hauteur de l'image
%#4: Texte à écrire
\newcommand{\ptp}[4]{
\setlength{\ltxt}{\linewidth}
\addtolength{\ltxt}{-#2 cm}
\addtolength{\ltxt}{-0.1 cm}
\begin{minipage}[b][#3 cm][t]{\ltxt}
#4
\end{minipage}\hfill
\begin{minipage}[b][#3 cm][c]{#2 cm}
#1
\end{minipage}\par
}



%Macros pour les graphiques
\psset{linewidth=0.5\pslinewidth,PointSymbol=x}
\setlength{\fboxrule}{0.5pt}
\newcounter{tempangle}

%Marque la longueur du segment d'extrémité  #1 et  #2 avec la valeur #3, #4 est la distance par rapport au segment (en %age de la valeur de celui ci) et #5 l'orientation du marquage : +90 ou -90
\newcommand{\afflong}[5]{
\pstRotation[RotAngle=#4,PointSymbol=none,PointName=none]{#1}{#2}[X] 
\pstHomO[PointSymbol=none,PointName=none,HomCoef=#5]{#1}{X}[Y]
\pstTranslation[PointSymbol=none,PointName=none]{#1}{#2}{Y}[Z]
 \ncline{|<->|,linewidth=0.25\pslinewidth}{Y}{Z} \ncput*[nrot=:U]{\footnotesize{#3}}
}
\newcommand{\afflongb}[3]{
\ncline{|<->|,linewidth=0}{#1}{#2} \naput*[nrot=:U]{\footnotesize{#3}}
}

%Construis le point #4 situé à #2 cm du point #1 avant un angle #3 par rapport à l'horizontale. #5 = liste de paramètre
\newcommand{\lsegment}[5]{\pstGeonode[PointSymbol=none,PointName=none](0,0){O'}(#2,0){I'} \pstTranslation[PointSymbol=none,PointName=none]{O'}{I'}{#1}[J'] \pstRotation[RotAngle=#3,PointSymbol=x,#5]{#1}{J'}[#4]}
\newcommand{\tsegment}[5]{\pstGeonode[PointSymbol=none,PointName=none](0,0){O'}(#2,0){I'} \pstTranslation[PointSymbol=none,PointName=none]{O'}{I'}{#1}[J'] \pstRotation[RotAngle=#3,PointSymbol=x,#5]{#1}{J'}[#4] \pstLineAB{#4}{#1}}

%Construis le point #4 situé à #3 cm du point #1 et faisant un angle de  90° avec la droite (#1,#2) #5 = liste de paramètre
\newcommand{\psegment}[5]{
\pstGeonode[PointSymbol=none,PointName=none](0,0){O'}(#3,0){I'}
 \pstTranslation[PointSymbol=none,PointName=none]{O'}{I'}{#1}[J']
 \pstInterLC[PointSymbol=none,PointName=none]{#1}{#2}{#1}{J'}{M1}{M2} \pstRotation[RotAngle=-90,PointSymbol=x,#5]{#1}{M1}[#4]
  }
  
%Construis le point #4 situé à #3 cm du point #1 et faisant un angle de  #5° avec la droite (#1,#2) #6 = liste de paramètre
\newcommand{\mlogo}[6]{
\pstGeonode[PointSymbol=none,PointName=none](0,0){O'}(#3,0){I'}
 \pstTranslation[PointSymbol=none,PointName=none]{O'}{I'}{#1}[J']
 \pstInterLC[PointSymbol=none,PointName=none]{#1}{#2}{#1}{J'}{M1}{M2} \pstRotation[RotAngle=#5,PointSymbol=x,#6]{#1}{M2}[#4]
  }

% Construis un triangle avec #1=liste des 3 sommets séparés par des virgules, #2=liste des 3 longueurs séparés par des virgules, #3 et #4 : paramètre d'affichage des 2e et 3 points et #5 : inclinaison par rapport à l'horizontale
%autre macro identique mais sans tracer les segments joignant les sommets
\noexpandarg
\newcommand{\Triangleccc}[5]{
\StrBefore{#1}{,}[\pointA]
\StrBetween[1,2]{#1}{,}{,}[\pointB]
\StrBehind[2]{#1}{,}[\pointC]
\StrBefore{#2}{,}[\coteA]
\StrBetween[1,2]{#2}{,}{,}[\coteB]
\StrBehind[2]{#2}{,}[\coteC]
\tsegment{\pointA}{\coteA}{#5}{\pointB}{#3} 
\lsegment{\pointA}{\coteB}{0}{Z1}{PointSymbol=none, PointName=none}
\lsegment{\pointB}{\coteC}{0}{Z2}{PointSymbol=none, PointName=none}
\pstInterCC{\pointA}{Z1}{\pointB}{Z2}{\pointC}{Z3} 
\pstLineAB{\pointA}{\pointC} \pstLineAB{\pointB}{\pointC}
\pstSymO[PointName=\pointC,#4]{C}{C}[C]
}
\noexpandarg
\newcommand{\TrianglecccP}[5]{
\StrBefore{#1}{,}[\pointA]
\StrBetween[1,2]{#1}{,}{,}[\pointB]
\StrBehind[2]{#1}{,}[\pointC]
\StrBefore{#2}{,}[\coteA]
\StrBetween[1,2]{#2}{,}{,}[\coteB]
\StrBehind[2]{#2}{,}[\coteC]
\tsegment{\pointA}{\coteA}{#5}{\pointB}{#3} 
\lsegment{\pointA}{\coteB}{0}{Z1}{PointSymbol=none, PointName=none}
\lsegment{\pointB}{\coteC}{0}{Z2}{PointSymbol=none, PointName=none}
\pstInterCC[PointNameB=none,PointSymbolB=none,#4]{\pointA}{Z1}{\pointB}{Z2}{\pointC}{Z1} 
}


% Construis un triangle avec #1=liste des 3 sommets séparés par des virgules, #2=liste formée de 2 longueurs et d'un angle séparés par des virgules, #3 et #4 : paramètre d'affichage des 2e et 3 points et #5 : inclinaison par rapport à l'horizontale
%autre macro identique mais sans tracer les segments joignant les sommets
\newcommand{\Trianglecca}[5]{
\StrBefore{#1}{,}[\pointA]
\StrBetween[1,2]{#1}{,}{,}[\pointB]
\StrBehind[2]{#1}{,}[\pointC]
\StrBefore{#2}{,}[\coteA]
\StrBetween[1,2]{#2}{,}{,}[\coteB]
\StrBehind[2]{#2}{,}[\angleA]
\tsegment{\pointA}{\coteA}{#5}{\pointB}{#3} 
\setcounter{tempangle}{#5}
\addtocounter{tempangle}{\angleA}
\tsegment{\pointA}{\coteB}{\thetempangle}{\pointC}{#4}
\pstLineAB{\pointB}{\pointC}
}
\newcommand{\TriangleccaP}[5]{
\StrBefore{#1}{,}[\pointA]
\StrBetween[1,2]{#1}{,}{,}[\pointB]
\StrBehind[2]{#1}{,}[\pointC]
\StrBefore{#2}{,}[\coteA]
\StrBetween[1,2]{#2}{,}{,}[\coteB]
\StrBehind[2]{#2}{,}[\angleA]
\lsegment{\pointA}{\coteA}{#5}{\pointB}{#3} 
\setcounter{tempangle}{#5}
\addtocounter{tempangle}{\angleA}
\lsegment{\pointA}{\coteB}{\thetempangle}{\pointC}{#4}
}

% Construis un triangle avec #1=liste des 3 sommets séparés par des virgules, #2=liste formée de 1 longueurs et de deux angle séparés par des virgules, #3 et #4 : paramètre d'affichage des 2e et 3 points et #5 : inclinaison par rapport à l'horizontale
%autre macro identique mais sans tracer les segments joignant les sommets
\newcommand{\Trianglecaa}[5]{
\StrBefore{#1}{,}[\pointA]
\StrBetween[1,2]{#1}{,}{,}[\pointB]
\StrBehind[2]{#1}{,}[\pointC]
\StrBefore{#2}{,}[\coteA]
\StrBetween[1,2]{#2}{,}{,}[\angleA]
\StrBehind[2]{#2}{,}[\angleB]
\tsegment{\pointA}{\coteA}{#5}{\pointB}{#3} 
\setcounter{tempangle}{#5}
\addtocounter{tempangle}{\angleA}
\lsegment{\pointA}{1}{\thetempangle}{Z1}{PointSymbol=none, PointName=none}
\setcounter{tempangle}{#5}
\addtocounter{tempangle}{180}
\addtocounter{tempangle}{-\angleB}
\lsegment{\pointB}{1}{\thetempangle}{Z2}{PointSymbol=none, PointName=none}
\pstInterLL[#4]{\pointA}{Z1}{\pointB}{Z2}{\pointC}
\pstLineAB{\pointA}{\pointC}
\pstLineAB{\pointB}{\pointC}
}
\newcommand{\TrianglecaaP}[5]{
\StrBefore{#1}{,}[\pointA]
\StrBetween[1,2]{#1}{,}{,}[\pointB]
\StrBehind[2]{#1}{,}[\pointC]
\StrBefore{#2}{,}[\coteA]
\StrBetween[1,2]{#2}{,}{,}[\angleA]
\StrBehind[2]{#2}{,}[\angleB]
\lsegment{\pointA}{\coteA}{#5}{\pointB}{#3} 
\setcounter{tempangle}{#5}
\addtocounter{tempangle}{\angleA}
\lsegment{\pointA}{1}{\thetempangle}{Z1}{PointSymbol=none, PointName=none}
\setcounter{tempangle}{#5}
\addtocounter{tempangle}{180}
\addtocounter{tempangle}{-\angleB}
\lsegment{\pointB}{1}{\thetempangle}{Z2}{PointSymbol=none, PointName=none}
\pstInterLL[#4]{\pointA}{Z1}{\pointB}{Z2}{\pointC}
}

%Construction d'un cercle de centre #1 et de rayon #2 (en cm)
\newcommand{\Cercle}[2]{
\lsegment{#1}{#2}{0}{Z1}{PointSymbol=none, PointName=none}
\pstCircleOA{#1}{Z1}
}

%construction d'un parallélogramme #1 = liste des sommets, #2 = liste contenant les longueurs de 2 côtés consécutifs et leurs angles;  #3, #4 et #5 : paramètre d'affichage des sommets #6 inclinaison par rapport à l'horizontale 
% meme macro sans le tracé des segements
\newcommand{\Para}[6]{
\StrBefore{#1}{,}[\pointA]
\StrBetween[1,2]{#1}{,}{,}[\pointB]
\StrBetween[2,3]{#1}{,}{,}[\pointC]
\StrBehind[3]{#1}{,}[\pointD]
\StrBefore{#2}{,}[\longueur]
\StrBetween[1,2]{#2}{,}{,}[\largeur]
\StrBehind[2]{#2}{,}[\angle]
\tsegment{\pointA}{\longueur}{#6}{\pointB}{#3} 
\setcounter{tempangle}{#6}
\addtocounter{tempangle}{\angle}
\tsegment{\pointA}{\largeur}{\thetempangle}{\pointD}{#5}
\pstMiddleAB[PointName=none,PointSymbol=none]{\pointB}{\pointD}{Z1}
\pstSymO[#4]{Z1}{\pointA}[\pointC]
\pstLineAB{\pointB}{\pointC}
\pstLineAB{\pointC}{\pointD}
}
\newcommand{\ParaP}[6]{
\StrBefore{#1}{,}[\pointA]
\StrBetween[1,2]{#1}{,}{,}[\pointB]
\StrBetween[2,3]{#1}{,}{,}[\pointC]
\StrBehind[3]{#1}{,}[\pointD]
\StrBefore{#2}{,}[\longueur]
\StrBetween[1,2]{#2}{,}{,}[\largeur]
\StrBehind[2]{#2}{,}[\angle]
\lsegment{\pointA}{\longueur}{#6}{\pointB}{#3} 
\setcounter{tempangle}{#6}
\addtocounter{tempangle}{\angle}
\lsegment{\pointA}{\largeur}{\thetempangle}{\pointD}{#5}
\pstMiddleAB[PointName=none,PointSymbol=none]{\pointB}{\pointD}{Z1}
\pstSymO[#4]{Z1}{\pointA}[\pointC]
}


%construction d'un cerf-volant #1 = liste des sommets, #2 = liste contenant les longueurs de 2 côtés consécutifs et leurs angles;  #3, #4 et #5 : paramètre d'affichage des sommets #6 inclinaison par rapport à l'horizontale 
% meme macro sans le tracé des segements
\newcommand{\CerfVolant}[6]{
\StrBefore{#1}{,}[\pointA]
\StrBetween[1,2]{#1}{,}{,}[\pointB]
\StrBetween[2,3]{#1}{,}{,}[\pointC]
\StrBehind[3]{#1}{,}[\pointD]
\StrBefore{#2}{,}[\longueur]
\StrBetween[1,2]{#2}{,}{,}[\largeur]
\StrBehind[2]{#2}{,}[\angle]
\tsegment{\pointA}{\longueur}{#6}{\pointB}{#3} 
\setcounter{tempangle}{#6}
\addtocounter{tempangle}{\angle}
\tsegment{\pointA}{\largeur}{\thetempangle}{\pointD}{#5}
\pstOrtSym[#4]{\pointB}{\pointD}{\pointA}[\pointC]
\pstLineAB{\pointB}{\pointC}
\pstLineAB{\pointC}{\pointD}
}

%construction d'un quadrilatère quelconque #1 = liste des sommets, #2 = liste contenant les longueurs des 4 côtés et l'angle entre 2 cotés consécutifs  #3, #4 et #5 : paramètre d'affichage des sommets #6 inclinaison par rapport à l'horizontale 
% meme macro sans le tracé des segements
\newcommand{\Quadri}[6]{
\StrBefore{#1}{,}[\pointA]
\StrBetween[1,2]{#1}{,}{,}[\pointB]
\StrBetween[2,3]{#1}{,}{,}[\pointC]
\StrBehind[3]{#1}{,}[\pointD]
\StrBefore{#2}{,}[\coteA]
\StrBetween[1,2]{#2}{,}{,}[\coteB]
\StrBetween[2,3]{#2}{,}{,}[\coteC]
\StrBetween[3,4]{#2}{,}{,}[\coteD]
\StrBehind[4]{#2}{,}[\angle]
\tsegment{\pointA}{\coteA}{#6}{\pointB}{#3} 
\setcounter{tempangle}{#6}
\addtocounter{tempangle}{\angle}
\tsegment{\pointA}{\coteD}{\thetempangle}{\pointD}{#5}
\lsegment{\pointB}{\coteB}{0}{Z1}{PointSymbol=none, PointName=none}
\lsegment{\pointD}{\coteC}{0}{Z2}{PointSymbol=none, PointName=none}
\pstInterCC[PointNameA=none,PointSymbolA=none,#4]{\pointB}{Z1}{\pointD}{Z2}{Z3}{\pointC} 
\pstLineAB{\pointB}{\pointC}
\pstLineAB{\pointC}{\pointD}
}


% Définition des colonnes centrées ou à droite pour tabularx
\newcolumntype{Y}{>{\centering\arraybackslash}X}
\newcolumntype{Z}{>{\flushright\arraybackslash}X}

%Les pointillés à remplir par les élèves
\newcommand{\po}[1]{\makebox[#1 cm]{\dotfill}}
\newcommand{\lpo}[1][3]{%
\multido{}{#1}{\makebox[\linewidth]{\dotfill}
}}

%Liste des pictogrammes utilisés sur la fiche d'exercice ou d'activités
\newcommand{\bombe}{\faBomb}
\newcommand{\livre}{\faBook}
\newcommand{\calculatrice}{\faCalculator}
\newcommand{\oral}{\faCommentO}
\newcommand{\surfeuille}{\faEdit}
\newcommand{\ordinateur}{\faLaptop}
\newcommand{\ordi}{\faDesktop}
\newcommand{\ciseaux}{\faScissors}
\newcommand{\danger}{\faExclamationTriangle}
\newcommand{\out}{\faSignOut}
\newcommand{\cadeau}{\faGift}
\newcommand{\flash}{\faBolt}
\newcommand{\lumiere}{\faLightbulb}
\newcommand{\compas}{\dsmathematical}
\newcommand{\calcullitteral}{\faTimesCircleO}
\newcommand{\raisonnement}{\faCogs}
\newcommand{\recherche}{\faSearch}
\newcommand{\rappel}{\faHistory}
\newcommand{\video}{\faFilm}
\newcommand{\capacite}{\faPuzzlePiece}
\newcommand{\aide}{\faLifeRing}
\newcommand{\loin}{\faExternalLink}
\newcommand{\groupe}{\faUsers}
\newcommand{\bac}{\faGraduationCap}
\newcommand{\histoire}{\faUniversity}
\newcommand{\coeur}{\faSave}
\newcommand{\os}{\faMicrochip}
\newcommand{\rd}{\faCubes}
\newcommand{\data}{\faColumns}
\newcommand{\web}{\faCode}
\newcommand{\prog}{\faFile}
\newcommand{\algo}{\faCogs}
\newcommand{\important}{\faExclamationCircle}
\newcommand{\maths}{\faTimesCircle}
% Traitement des données en tables
\newcommand{\tables}{\faColumns}
% Types construits
\newcommand{\construits}{\faCubes}
% Type et valeurs de base
\newcommand{\debase}{{\footnotesize \faCube}}
% Systèmes d'exploitation
\newcommand{\linux}{\faLinux}
\newcommand{\sd}{\faProjectDiagram}
\newcommand{\bd}{\faDatabase}

%Les ensembles de nombres
\renewcommand{\N}{\mathbb{N}}
\newcommand{\D}{\mathbb{D}}
\newcommand{\Z}{\mathbb{Z}}
\newcommand{\Q}{\mathbb{Q}}
\newcommand{\R}{\mathbb{R}}
\newcommand{\C}{\mathbb{C}}

%Ecriture des vecteurs
\newcommand{\vect}[1]{\vbox{\halign{##\cr 
  \tiny\rightarrowfill\cr\noalign{\nointerlineskip\vskip1pt} 
  $#1\mskip2mu$\cr}}}


%Compteur activités/exos et question et mise en forme titre et questions
\newcounter{numact}
\setcounter{numact}{1}
\newcounter{numseance}
\setcounter{numseance}{1}
\newcounter{numexo}
\setcounter{numexo}{0}
\newcounter{numprojet}
\setcounter{numprojet}{0}
\newcounter{numquestion}
\newcommand{\espace}[1]{\rule[-1ex]{0pt}{#1 cm}}
\newcommand{\Quest}[3]{
\addtocounter{numquestion}{1}
\begin{tabularx}{\textwidth}{X|m{1cm}|}
\cline{2-2}
\textbf{\sffamily{\alph{numquestion})}} #1 & \dots / #2 \\
\hline 
\multicolumn{2}{|l|}{\espace{#3}} \\
\hline
\end{tabularx}
}
\newcommand{\mq}[1]
{\ding{113} \addtocounter{numquestion}{1}
\textbf{Question \arabic{numquestion}} \\ #1}
\newcommand{\QuestR}[3]{
\addtocounter{numquestion}{1}
\begin{tabularx}{\textwidth}{X|m{1cm}|}
\cline{2-2}
\textbf{\sffamily{\alph{numquestion})}} #1 & \dots / #2 \\
\hline 
\multicolumn{2}{|l|}{\cor{#3}} \\
\hline
\end{tabularx}
}
\newcommand{\Pre}{{\sc nsi} 1\textsuperscript{e}}
\newcommand{\Term}{{\sc nsi} Terminale}
\newcommand{\Sec}{2\textsuperscript{e}}
\newcommand{\Exo}[2]{ \addtocounter{numexo}{1} \ding{113} \textbf{\sffamily{Exercice \thenumexo}} : \textit{#1} \hfill #2  \setcounter{numquestion}{0}}
\newcommand{\Projet}[1]{ \addtocounter{numprojet}{1} \ding{118} \textbf{\sffamily{Projet \thenumprojet}} : \textit{#1}}
\newcommand{\ExoD}[2]{ \addtocounter{numexo}{1} \ding{113} \textbf{\sffamily{Exercice \thenumexo}}  \textit{(#1 pts)} \hfill #2  \setcounter{numquestion}{0}}
\newcommand{\ExoB}[2]{ \addtocounter{numexo}{1} \ding{113} \textbf{\sffamily{Exercice \thenumexo}}  \textit{(Bonus de +#1 pts maximum)} \hfill #2  \setcounter{numquestion}{0}}
\newcommand{\Act}[2]{ \ding{113} \textbf{\sffamily{Activité \thenumact}} : \textit{#1} \hfill #2  \addtocounter{numact}{1} \setcounter{numquestion}{0}}
\newcommand{\Seance}{ \rule{1.5cm}{0.5pt}\raisebox{-3pt}{\framebox[4cm]{\textbf{\sffamily{Séance \thenumseance}}}}\hrulefill  \\
  \addtocounter{numseance}{1}}
\newcommand{\Acti}[2]{ {\footnotesize \ding{117}} \textbf{\sffamily{Activité \thenumact}} : \textit{#1} \hfill #2  \addtocounter{numact}{1} \setcounter{numquestion}{0}}
\newcommand{\titre}[1]{\begin{Large}\textbf{\ding{118}}\end{Large} \begin{large}\textbf{ #1}\end{large} \vspace{0.2cm}}
\newcommand{\QListe}[1][0]{
\ifthenelse{#1=0}
{\begin{enumerate}[partopsep=0pt,topsep=0pt,parsep=0pt,itemsep=0pt,label=\textbf{\sffamily{\arabic*.}},series=question]}
{\begin{enumerate}[resume*=question]}}
\newcommand{\SQListe}[1][0]{
\ifthenelse{#1=0}
{\begin{enumerate}[partopsep=0pt,topsep=0pt,parsep=0pt,itemsep=0pt,label=\textbf{\sffamily{\alph*)}},series=squestion]}
{\begin{enumerate}[resume*=squestion]}}
\newcommand{\SQListeL}[1][0]{
\ifthenelse{#1=0}
{\begin{enumerate*}[partopsep=0pt,topsep=0pt,parsep=0pt,itemsep=0pt,label=\textbf{\sffamily{\alph*)}},series=squestion]}
{\begin{enumerate*}[resume*=squestion]}}
\newcommand{\FinListe}{\end{enumerate}}
\newcommand{\FinListeL}{\end{enumerate*}}

%Mise en forme de la correction
\newboolean{corrige}
\setboolean{corrige}{false}
\newcommand{\scor}[1]{\par \textcolor{blue!75!black}{\small #1}}
\newcommand{\cor}[1]{\par \textcolor{blue!75!black}{#1}}
\newcommand{\br}[1]{\cor{\textbf{#1}}}
\newcommand{\tcor}[1]{
\ifthenelse{\boolean{corrige}}{\begin{tcolorbox}[width=\linewidth,colback={white},colbacktitle=white,coltitle=green!50!black,colframe=green!50!black,boxrule=0.2mm]   
\cor{#1}
\end{tcolorbox}}{}
}
\newcommand{\iscor}[1]{\ifthenelse{\boolean{corrige}}{#1}}
\newcommand{\rc}[1]{\textcolor{OliveGreen}{#1}}

%Référence aux exercices par leur numéro
\newcommand{\refexo}[1]{
\refstepcounter{numexo}
\addtocounter{numexo}{-1}
\label{#1}}

%Séparation entre deux activités
\newcommand{\separateur}{\begin{center}
\rule{1.5cm}{0.5pt}\raisebox{-3pt}{\ding{117}}\rule{1.5cm}{0.5pt}  \vspace{0.2cm}
\end{center}}

%Entête et pied de page
\newcommand{\snt}[1]{\lhead{\textbf{SNT -- La photographie numérique} \rhead{\textit{Lycée Nord}}}}
\newcommand{\Activites}[2]{\lhead{\textbf{{\sc #1}}}
\rhead{Activités -- \textbf{#2}}
\cfoot{}}
\newcommand{\Exos}[2]{\lhead{\textbf{Fiche d'exercices: {\sc #1}}}
\rhead{Niveau: \textbf{#2}}
\cfoot{}}
\newcommand{\TD}[2]{\lhead{\textbf{TD #1} : {\sc #2} }
\rhead{{\sc mp2i -- Lycée Leconte de Lisle}}
\cfoot{}}
\newcommand{\Colles}[2]{\lhead{{\sc mp2i -- }\textbf{Colles d'informatique #1}} 
\rhead{{\sc #2}}
\cfoot{}}
\newcommand{\Devoir}[2]{\lhead{\textbf{Devoir de mathématiques : {\sc #1}}}
\rhead{\textbf{#2}} \setlength{\fboxsep}{8pt}
\begin{center}
%Titre de la fiche
\fbox{\parbox[b][1cm][t]{0.3\textwidth}{Nom : \hfill \po{3} \par \vfill Prénom : \hfill \po{3}} } \hfill 
\fbox{\parbox[b][1cm][t]{0.6\textwidth}{Note : \po{1} / 20} }
\end{center} \cfoot{}}
\newcommand{\TPnote}[3]{\lhead{\textbf{TP noté d'informatique n° #1}}
\rhead{\textbf{#2}} \setlength{\fboxsep}{8pt}
\ifthenelse{\boolean{corrige}}{}
{\begin{center}
\fbox{\parbox[b][1cm][t]{0.3\textwidth}{Nom : \hfill \po{3} \par \vfill Prénom : \hfill \po{3}} } \hfill 
\fbox{\parbox[b][1cm][t]{0.6\textwidth}{Note : \po{1} / #3} }
\end{center}} \cfoot{}}
\newcommand{\IC}[2]{\lhead{\textbf{Interro de cours n° #1}}
\rhead{{\sc mp2i --} \textbf{#2}} \setlength{\fboxsep}{8pt}
\ifthenelse{\boolean{corrige}}{}
{\begin{center}
%Titre de la fiche
\fbox{\parbox[b][1cm][t]{0.3\textwidth}{Nom : \hfill \po{3} \par \vfill Prénom : \hfill \po{3}} } \hfill 
\fbox{\parbox[b][1cm][t]{0.6\textwidth}{Note : \po{1} / 10} }
\end{center}}\cfoot{}}
\newcommand{\DS}[3]{\lhead{{#1} : \textbf{DS d'informatique n° #2}}
\rhead{Lycée Leconte de Lisle -- #3} \setlength{\fboxsep}{8pt}
%Titre de la fiche
\begin{center}
   {\Large \textbf{Devoir surveillé d'informatique}}
\end{center} \cfoot{\thepage/\pageref{LastPage}}}

\newcommand{\CB}[2]{\lhead{{#1} : \textbf{Councours blanc - Informatique}}
\rhead{Lycée Leconte de Lisle -- #2} \setlength{\fboxsep}{8pt}
%Titre de la fiche
\begin{center}
   {\Large \textbf{Concours Blanc - Epreuve d'informatique}}
\end{center} \cfoot{\thepage/\pageref{LastPage}}}

\newcommand{\PC}[3]{\lhead{Concours {#1} -- #2}
\rhead{Lycée Leconte de Lisle} \setlength{\fboxsep}{8pt}
%Titre de la fiche
\begin{center}
   {\Large \textbf{Proposition de corrigé}}
\end{center} \cfoot{\thepage/\pageref{LastPage}}}

\newcounter{numdspart}
\setcounter{numdspart}{1}
\newcommand{\DSPart}{\bigskip
   \hrulefill\raisebox{-3pt}{\framebox[4cm]{\textbf{\textbf{Partie \thenumdspart}}}}\hrulefill
   \addtocounter{numdspart}{1}
   \bigskip}

\newcommand{\Sauvegarde}[1]{
   \begin{tcolorbox}[title=\textcolor{black}{\danger\; Attention},colbacktitle=lightgray]
      {Tous vos programmes doivent être enregistrés dans votre dossier personnel, dans {\tt Evaluations}{\tt \textbackslash}{\tt #1}
      }
   \end{tcolorbox}
}

\newcommand{\alertbox}[3]{
   \begin{tcolorbox}[title=\textcolor{black}{#1\; #2},colbacktitle=lightgray]
      {#3}
   \end{tcolorbox}
}

%Devoir programmation en NSI (pas à rendre sur papier)
\newcommand{\PNSI}[2]{\lhead{\textbf{Devoir de {\sc nsi} : \textsf{ #1}}
}
\rhead{\textbf{#2}} \setlength{\fboxsep}{8pt}
 \cfoot{}
 \begin{center}{\Large \textbf{Evaluation de {\sc nsi}}}\end{center}}


%Devoir de NSI
\newcommand{\DNSI}[2]{\lhead{\textbf{Devoir de {\sc nsi} : \textsf{ #1}}}
\rhead{\textbf{#2}} \setlength{\fboxsep}{8pt}
\begin{center}
%Titre de la fiche
\fbox{\parbox[b][1cm][t]{0.3\textwidth}{Nom : \hfill \po{3} \par \vfill Prénom : \hfill \po{3}} } \hfill 
\fbox{\parbox[b][1cm][t]{0.6\textwidth}{Note : \po{1} / 10} }
\end{center} \cfoot{}}

\newcommand{\DevoirNSI}[2]{\lhead{\textbf{Devoir de {\sc nsi} : {\sc #1}}}
\rhead{\textbf{#2}} \setlength{\fboxsep}{8pt}
\cfoot{}}

%La définition de la commande QCM pour auto-multiple-choice
%En premier argument le sujet du qcm, deuxième argument : la classe, 3e : la durée prévue et #4 : présence ou non de questions avec plusieurs bonnes réponses
\newcommand{\QCM}[4]{
{\large \textbf{\ding{52} QCM : #1}} -- Durée : \textbf{#3 min} \hfill {\large Note : \dots/10} 
\hrule \vspace{0.1cm}\namefield{}
Nom :  \textbf{\textbf{\nom{}}} \qquad \qquad Prénom :  \textbf{\prenom{}}  \hfill Classe: \textbf{#2}
\vspace{0.2cm}
\hrule  
\begin{itemize}[itemsep=0pt]
\item[-] \textit{Une bonne réponse vaut un point, une absence de réponse n'enlève pas de point. }
\item[\danger] \textit{Une mauvaise réponse enlève un point.}
\ifthenelse{#4=1}{\item[-] \textit{Les questions marquées du symbole \multiSymbole{} peuvent avoir plusieurs bonnes réponses possibles.}}{}
\end{itemize}
}
\newcommand{\DevoirC}[2]{
\renewcommand{\footrulewidth}{0.5pt}
\lhead{\textbf{Devoir de mathématiques : {\sc #1}}}
\rhead{\textbf{#2}} \setlength{\fboxsep}{8pt}
\fbox{\parbox[b][0.4cm][t]{0.955\textwidth}{Nom : \po{5} \hfill Prénom : \po{5} \hfill Classe: \textbf{1}\textsuperscript{$\dots$}} } 
\rfoot{\thepage} \cfoot{} \lfoot{Lycée Nord}}
\newcommand{\DevoirInfo}[2]{\lhead{\textbf{Evaluation : {\sc #1}}}
\rhead{\textbf{#2}} \setlength{\fboxsep}{8pt}
 \cfoot{}}
\newcommand{\DM}[2]{\lhead{\textbf{Devoir maison à rendre le #1}} \rhead{\textbf{#2}}}

%Macros permettant l'affichage des touches de la calculatrice
%Touches classiques : #1 = 0 fond blanc pour les nombres et #1= 1gris pour les opérations et entrer, second paramètre=contenu
%Si #2=1 touche arrondi avec fond gris
\newcommand{\TCalc}[2]{
\setlength{\fboxsep}{0.1pt}
\ifthenelse{#1=0}
{\psframebox[fillstyle=solid, fillcolor=white]{\parbox[c][0.25cm][c]{0.6cm}{\centering #2}}}
{\ifthenelse{#1=1}
{\psframebox[fillstyle=solid, fillcolor=lightgray]{\parbox[c][0.25cm][c]{0.6cm}{\centering #2}}}
{\psframebox[framearc=.5,fillstyle=solid, fillcolor=white]{\parbox[c][0.25cm][c]{0.6cm}{\centering #2}}}
}}
\newcommand{\Talpha}{\psdblframebox[fillstyle=solid, fillcolor=white]{\hspace{-0.05cm}\parbox[c][0.25cm][c]{0.65cm}{\centering \scriptsize{alpha}}} \;}
\newcommand{\Tsec}{\psdblframebox[fillstyle=solid, fillcolor=white]{\parbox[c][0.25cm][c]{0.6cm}{\centering \scriptsize 2nde}} \;}
\newcommand{\Tfx}{\psdblframebox[fillstyle=solid, fillcolor=white]{\parbox[c][0.25cm][c]{0.6cm}{\centering \scriptsize $f(x)$}} \;}
\newcommand{\Tvar}{\psframebox[framearc=.5,fillstyle=solid, fillcolor=white]{\hspace{-0.22cm} \parbox[c][0.25cm][c]{0.82cm}{$\scriptscriptstyle{X,T,\theta,n}$}}}
\newcommand{\Tgraphe}{\psdblframebox[fillstyle=solid, fillcolor=white]{\hspace{-0.08cm}\parbox[c][0.25cm][c]{0.68cm}{\centering \tiny{graphe}}} \;}
\newcommand{\Tfen}{\psdblframebox[fillstyle=solid, fillcolor=white]{\hspace{-0.08cm}\parbox[c][0.25cm][c]{0.68cm}{\centering \tiny{fenêtre}}} \;}
\newcommand{\Ttrace}{\psdblframebox[fillstyle=solid, fillcolor=white]{\parbox[c][0.25cm][c]{0.6cm}{\centering \scriptsize{trace}}} \;}

% Macroi pour l'affichage  d'un entier n dans  une base b
\newcommand{\base}[2]{ \overline{#1}^{#2}}
% Intervalle d'entiers
\newcommand{\intN}[2]{\llbracket #1; #2 \rrbracket}
% Cadre avec lignes réponses
\def\gaddtotok#1{\global\tabtok\expandafter{\the\tabtok#1}}
\newtoks\tabtok
\newcommand*\reponse[2]{%
   \ifthenelse{\boolean{corrige}}{}{
	\global\tabtok{\\ \renewcommand{\arraystretch}{1.4}\begin{tabularx}{\linewidth}{|X|p{1cm}|}\hline \dotfill & \cellcolor{gray!30}{\small \dots/#2} \\ \cline{2-2}}%
	\multido{}{#1}{\gaddtotok{ \multicolumn{2}{|>{\hsize=\dimexpr1\hsize+2\tabcolsep+\arrayrulewidth+1cm\relax}X|}{\dotfill}\\ }}%
	\gaddtotok{\hline \end{tabularx}}%
	\the\tabtok
   }}

\newcommand{\PE}[1]{\left \lfloor #1 \right \lfloor}}
\ModeExercice
\TD{1}{Langage C}

\setcounter{Exercise}{0}
\begin{Exercise}[title={Pointeurs}]\\
    On considère le programme suivant :
    \begin{langageC*}{fontsize=\small}
int a = 4;
int b = 1;
int c = 2;
int* p;
int* q;
p = &a;
q = &c;
*p = *q + 1;
p = q;
q = &b;
*p = *p - *q;
*q = *q + 1;
*p *= *q;
    \end{langageC*}
Compléter le tableau suivant, donner l'état des variables au cours de l'exécution du programme :
\begin{center}
    \def\arraystretch{1.2}
    \setlength\tabcolsep{0.5cm}
    \begin{tabular}{l|c|c|c|c|c}
    & a & b & c & p & q \\
    \hline
    initialisation & 4 & 1 & 2 & ? & ?\\
    {\tt p = \&a;}  & 4 & 1 & 2 & \&a & ?\\
    {\tt q = \&c;}  &  &  &  &  \\
    {\tt *p = *q + 1;}  &  &  &  &  \\
    {\tt p = q;}  &  &  &  &  \\
    {\tt q = \&b;}  &  &  &  &  \\
    {\tt *p = *p - *q;}  &  &  &  &  \\
    {\tt *q = *q + 1;}  &  &  &  &  \\
    {\tt *p *= *q;}  &  &  &  &  \\
    \end{tabular}
\end{center}
\end{Exercise}

\begin{Exercise}[title={{\tt printf} et {\tt scanf}}]
    \Question{Ecrire l'instruction permettant d'afficher une variable {\tt n} de type entier avec {\tt printf}}
    \Question{Ecrire l'instruction permettant de saisir au clavier une variable {\tt n} de type entier avec {\tt scanf}}
    \Question{Expliquer la différence entre le mode de passage de {\tt n} dans ces deux fonctions}
\end{Exercise}

\begin{Exercise}[title={{\tt Pointeurs}}] \\
On considère le programme suivant :
\inputC{/home/fenarius/Travail/Cours/cpge-info/docs/mp2i/files/C3/pointeur.c}{\small}
\Question{Ce programme est-il correct ?}
\Question{Proposer une correction.}
\end{Exercise}

\begin{Exercise}[title={Incrémenter une variable}]\\
La fonction suivante doit incrémenter la variable {\tt n} donnée en argument :
\begin{langageC*}{fontsize=\small}
void incremente(int x) {
    x = x + 1;
    }    
\end{langageC*}
\Question{Commenter}
\Question{Proposer une correction.}
\end{Exercise}

\end{document}
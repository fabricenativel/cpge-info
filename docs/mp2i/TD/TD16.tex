\documentclass[11pt,a4paper]{article}

\usepackage{Act}

\begin{document}
\input{\detokenize{/home/fenarius/Travail/Cours/cpge-info/latex/Macros.tex}}
\ModeExercice
\TD{16}{Logique}
\newcommand{\SPATH}{/home/fenarius/Travail/Cours/cpge-info/docs/mp2i/files/C14/}

\newcommand{\non}{\neg}
\newcommand{\et}{\wedge}
\newcommand{\ou}{\vee}
\newcommand{\imp}{\to}
\newcommand{\eq}{\leftrightarrow}
\newcommand{\lnode}[1]{\TCircle{$#1$}}
\psset{treesep=0.5cm,levelsep=0.8cm}

\setcounter{Exercise}{0}


\begin{Exercise}[title = {formules logiques}] \\
    Les expressions suivantes sont-elles des fes formules logique sur l'ensemble de propositions $V = \{p, q, r\}$ ? (on ne s'autorise pas dans cet exercice les simplfications d'écriture)
    \Question{ $(p) \ou \non q$}
    \Question{ $((p \et q) \ou (\top \et r))$}
    \Question{ $ ( p \ou r) \et (\non q))$}
    \Question{ $ (\non (p \ou q)) \et \non r)$}
    \Question{ $ ((s \et t) \ou (p \et q))$}
\end{Exercise}

\begin{Exercise}[title = {Représentation arborescente}]
    \Question{Représenter les arbres syntaxiques des formules logiques suivantes }
    \subQuestion{ $(p \ou q) \imp (r \et s)$}
    \subQuestion{ $ (\non p \ou \non q) \eq (r \imp s)$}
    \Question{Ecrire les formules formules logiques dont les arbres sont syntaxique sont}
    \subQuestion{ 
    \pstree{\lnode{\et}}{    
    \pstree{\lnode{\ou}}{\lnode{p} \pstree{\lnode{\non}}{\lnode{q}}} 
    \pstree{\lnode{\imp}}{ \pstree{\lnode{\non}}{\lnode{r}} \lnode{s} } }
    }
    \subQuestion{
        \pstree{\lnode{\eq}}{
            \pstree{\lnode{\et}}{
                \lnode{r}
                \pstree{\lnode{\non}}{\lnode{q}}
            }
            \pstree{\lnode{\ou}}{
                \pstree{\lnode{\imp}}{\lnode{p} \lnode{q}}
                \pstree{\lnode{\eq}}{\lnode{r} \lnode{s}}
                }
        }
    }
\end{Exercise}
\end{document}
\documentclass[11pt,a4paper]{article}

\usepackage{Act}

\begin{document}
\input{\detokenize{/home/fenarius/Travail/Cours/cpge-info/latex/Macros.tex}}
\ModeExercice
\TD{21}{Compléments sur les arbres}
\newcommand{\SPATH}{/home/fenarius/Travail/Cours/cpge-info/docs/mp2i/files/C21/}
\newcommand{\CR}[1]{\TCircle[radius=0.25]{#1}}
\newcommand{\CN}[1]{\Tdia[fillstyle=solid,fillcolor=gray!20]{#1}}
\newcommand{\Tri}[1]{\Ttri[linestyle=dashed]{#1}}
\setcounter{Exercise}{0}
\psset{arrows=->,treesep=0.8cm,levelsep=0.8cm, radius=0.3cm}

\begin{Exercise}[title = {Equilibrage d'un arbre rouge-noir}]

\Question{Pour insérer un noeud dans un arbre rouge-noir, on commence par utiliser l'algorithme d'insertion usuel dans un {\sc abr} et on attribut au nouveau noeud la couleur \textit{rouge}. Quel est alors le seul conflit possible ? (on appellera un tel conflit un \textit{conflit rouge-rouge}).}
\Question{Si le conflit rouge-rouge se situe à la racine, donner une méthode simple pour le résoudre.}
\Question{Si le conflit n'est pas situé à la racine, justifier qu'on se trouve dans l'un des quatres cas suivants où les noeuds rouges sont représentés dans un cercle et les noeuds noirs dans un losange grisé:}
\begin{center}
\begin{tabularx}{0.8\textwidth}{Y|Y}
    \pstree[arrows=->,treesep=1cm,levelsep=1cm]{\CN{$z$}}{
				\pstree{\CR{$y$}}{
					\pstree{\CR{$x$}}
					{
						\Tri{$t_1$}
						\Tri{$t_2$}
					}
                    \Tri{$t_3$}
				}
				\Tri{$t_4$}
			} 
    
    & 
    \pstree[arrows=->,treesep=1cm,levelsep=1cm]{\CN{$x$}}{
        \Tri{$t_1$}			
    \pstree{\CR{$y$}}{
                    \Tri{$t_2$}
					\pstree{\CR{$z$}}
					{
						\Tri{$t_3$}
						\Tri{$t_4$}
					}
				}
				
			} 
    \\
    \hline 
    & \\
    \pstree[arrows=->,treesep=1cm,levelsep=1cm]{\CN{$z$}}{
				\pstree{\CR{$x$}}{
					\Tri{$t_1$}
					\pstree{\CR{$y$}}
					{
						\Tri{$t_2$}
						\Tri{$t_3$}
					}
				}
				\Tri{$t_4$}
			} & 
            \pstree[arrows=->,treesep=1cm,levelsep=1cm]{\CN{$x$}}{
                \Tri{$t_1$}	
            \pstree{\CR{$z$}}{
                \pstree{\CR{$y$}}
                {
                    \Tri{$t_2$}
                    \Tri{$t_3$}
                }		
            \Tri{$t_4$}
					
				}
				
			}
            \\
   
\end{tabularx}
\end{center}
\Question{Montrer qu'en effectuant une ou plusieurs rotations, ces arbres se ramènent à }
\begin{center}
\pstree[arrows=->,treesep=1cm,levelsep=1cm]{\CN{$y$}}{
            \pstree{\CR{$x$}}{
                {
                    \Tri{$t_1$}
                    \Tri{$t_2$}
                }}	
                \pstree{\CR{$z$}}{
                    {
                        \Tri{$t_3$}
                        \Tri{$t_4$}
                    }		
					
				}
				
			}
\end{center}
\end{Exercise}


\end{document}
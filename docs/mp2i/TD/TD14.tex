\documentclass[11pt,a4paper]{article}

\usepackage{Act}

\begin{document}
\input{\detokenize{/home/fenarius/Travail/Cours/cpge-info/latex/Macros.tex}}
\ModeExercice
\TD{14}{Induction}
\newcommand{\SPATH}{/home/fenarius/Travail/Cours/cpge-info/docs/mp2i/files/C14/}


\setcounter{Exercise}{0}


\begin{Exercise}[title = {Relation divise}] \\
    Sur $\N$, on définit la relation binaire $|$ par $x|y$ si et seulement si $x$ divise $y$.
    \Question{Montrer que $(\N,|)$ est un ensemble ordonnée}
    \Question{L'ordre est-il total ? Justifier }
\end{Exercise}

\begin{Exercise}[title = {Ordre inverse}] \\
    Soit $(E, \preccurlyeq)$ un ensemble ordonné, on définit la relation $ \succcurlyeq$ par $x \succcurlyeq y$ si et seulement si $y \preccurlyeq x$.
    \Question{Montrer que $\succcurlyeq$ est une relation d'ordre sur $E$. On l'appelle l'ordre inverse de $\preccurlyeq$.}
    \Question{Si $\preccurlyeq$ est un ordre bien fondé, son ordre inverse $\succcurlyeq$ l'est-il aussi ?}
\end{Exercise}


\end{document}
\documentclass[11pt,a4paper]{article}

\usepackage{Act}

\begin{document}
\input{\detokenize{/home/fenarius/Travail/Cours/cpge-info/latex/Macros.tex}}
\ModeExercice
\TD{19}{Algorithme des textes}
\newcommand{\SPATH}{/home/fenarius/Travail/Cours/cpge-info/docs/mp2i/files/C18/}


\psset{treesep=0.5cm,levelsep=0.8cm}

\setcounter{Exercise}{0}

\begin{Exercise}[title = complexité des algorithmes de recherche]\\
Dans tout cet exercice, on exprime la complexité en nombre de comparaisons effectués par l'algorithme. Et note $t$ la longueur du texte et $m$ la longueur du motif.
\Question{Montrer que la recherche naïve effectue entre $t$ et $t\times m$ comparaisons et donner des exemples dans laquelle ces deux bornes sont atteintes.}
\Question{Déterminer de même les bornes du nombre de comparaisons effectuées par l'algorithme de Boyer-Moore-Hoorspool et donner un exemple où elles sont atteintes.}
\end{Exercise}

\begin{Exercise}[title = Algorithme naïf avec motif à caractère unique]
    \Question{Montrer qu'il est possible d'améliorer l'algorithme de recherche naïf si on suppose que tous les caractères du motifs apparaissent une seule fois dans le motif. \\
    \aide \; Faire par exemple la recherche de {\tt abcd} dans le texte {\tt abceababccabcdabb} et observer ce qu'il se passe lorsqu'on trouve un début de correspondance.}
    \Question{Donner une implémentation en C ou en OCaml de ce nouvel algorithme.}
\end{Exercise}

\begin{Exercise}[title = Dérouler l'algorithme de Boyer-Moore-Hoorspool]
\Question{Donner la table de décalage du motif {\tt tele}}
\Question{Donner les étapes du déroulement de l'algorithme de Boyer-Moore-Hoorspool pour recherche toutes les occurrences de ce motif dans le texte : {\tt le telephone ou la television} }
\end{Exercise}

\begin{Exercise}[title = Rabin-Karp à deux motifs]
\Question{Ecrire une fonction {\tt hash  : string -> int} qui effectue la moyenne des code {\sc ascii} de la chaine donnée en argument en leur affectant pour coefficient leur position dans la chaine (le coefficient du premier caractère est 1).}
\Question{Ecrire l'implémentation de l'algorithme de Rabin-Karp avec cette fonction de hachage afin de rechercher un motif dans un texte.}
\Question{Adapter votre algorithme de façon à pourvoir rechercher les occurences de deux motifs {\tt m1} et {\tt m2} qu'on suppose de même longueur.}
\Question{Proposer une nouvelle modification si les motifs sont de taille différentes.}
\end{Exercise}

\begin{Exercise}[title = dérouler l'algorithme de Huffman]\\
On considère la phrase : \textit{\og{} comprendre cet algorithme \fg{}}.
\Question{Faire une tableau indiquant pour chaque caractère son nombre d'occurrence dans la phrase.}
\Question{Dérouler l'algorithme de Huffman afin de construire l'arbre de codage préfixe.}
\Question{Donner les codes de chaque caractère.}
\Question{Déterminer le taux de compression de l'algorithme sur cet exemple.}
\end{Exercise}

\newcommand{\tv}{\textvisiblespace}
\newcommand{\dn}[1]{\TCircle{\textcolor{blue}{\tt #1}}}
\begin{Exercise}[title = Quelques arbres de Huffmann]
    \Question{Donner l'arbre de Huffman obtenu pour un texte contenant 10 caractères ayant tous le même nombre d'occurences $n$.}
    \Question{Donner l'arbre de Huffman obtenu pour un texte contenant 7 caractères  dont les nombres d'occurences sont 1, 2, 4, 8, 16, 32 et 64.}
    \Question{On donne l'arbre de Huffmann suivant :
    \begin{center}
        \pstree{\Tdot}{
        \pstree{\Tdot}{\pstree{\Tdot}{\pstree{\Tdot}{{\dn{a} \dn{f}}} \dn{c}} \pstree{\Tdot}{\dn{i} \dn{\tv}}}
        \pstree{\Tdot}{\dn{e} \pstree{\Tdot}{\pstree{\Tdot}{\dn{r} \pstree{\Tdot}{\dn{x} \dn{p}}} \pstree{\Tdot}{\dn{l} \dn{t}}}}
        }\vspace{0.7cm}
    \end{center}
    Donner les codes  des caractères présents dans l'arbre.
    }
    \Question{Donner le texte dont la compression est : \\{\tt 11101001111011101111010111101110110101011000010100011001100010000001010111010}}
    \Question{Calculer la taille initiale du texte et la taille du texte compressé.}
\end{Exercise}

\begin{Exercise}[title = Séquence génétique]\\
    On rappelle que le principe de la compression {\sc lzw} est d'attribuer un code aux préfixes rencontrés lors de la lecture du texte à compresser de façon à disposer d'un code compact si ce code se présente à nouveau. Initialement, seuls les codes de l'alphabet (usuellement les caractères {\sc ascii}) sont présents dans la table de codage. \\
    Ici, on s'intéresse à des chaines de caractères représentant des séquences génétique codés sur l'alphabet {\tt \{A, C, G, T\}}, par souci de simplicité on attribue initialement les codes {\tt A} $\rightarrow 0$, {\tt C} $\rightarrow 1$, {\tt G} $\rightarrow 2$  et {\tt T} $\rightarrow 3$. Et on souhaite compresser la séquence {\tt ATGTGATGTCCT}.
    Le début de l'algorithme va donc consister à attribuer un nouveau code au premier préfixe non encore codé qui apparaît lors de la lecture du texte. Et donc, ici, on attribue le code {\tt 4} au prefixe {\tt AT} et on emmettra le code de {\tt A} c'est à dire 0.
    \Question{Poursuivre le déroulement de cet algorithme en complétant le tableau suivant : \\
    \begin{tabular}{|l|l|l|}
        \hline
        Position dans le texte & Code émis & Nouveau préfixe ajouté \\
        \hline
        {\tt \underline{A}TGTGATGTCCT} & 0 & {\tt AT } $\rightarrow 4$ \\
    \end{tabular}\\
    }
    \tcor{
        \begin{tabular}{|l|l|l|}
        \hline
        Position dans le texte & Code émis & Nouveau préfixe ajouté \\
        \hline
        {\tt \underline{A}TGTGATGTCCT} & 0 & {\tt AT } $\rightarrow 4$ \\
        \hline
        {\tt A\underline{T}GTGATGTCCT} & 3 & {\tt TG } $\rightarrow 5$ \\
        \hline
        {\tt AT\underline{G}TGATGTCCT} & 2 & {\tt GT } $\rightarrow 6$ \\
         \hline
        {\tt ATG\underline{TG}ATGTCCT} & 5 & {\tt TGA } $\rightarrow 7$ \\
         \hline
        {\tt ATGTG\underline{AT}GTCCT} & 4 & {\tt ATT } $\rightarrow 8$ \\
        \hline
        {\tt ATGTGAT\underline{GT}CCT} & 6 & {\tt GTT } $\rightarrow 9$ \\
        \hline
        {\tt ATGTGATGT\underline{C}CT} & 1 & {\tt CC } $\rightarrow 10$ \\
        \hline
        {\tt ATGTGATGTC\underline{C}T} & 1 & {\tt CT } $\rightarrow 11$ \\
        \hline
        {\tt ATGTGATGTCC\underline{T}} & 3 &   \\
        \hline
    \end{tabular}\\
    On obtient donc la suite de code : {\tt [0; 3; 2; 5; 4; 6; 1; 1; 3]}}
    \Question{Quel est le taux de compression obtenu (la taille d'un code est un octet) ?}
    \tcor{12 sur 9}
    \Question{Décompresser le texte $T$ codé par suite de codes {\tt [3; 1; 4; 6; 5; 2; 0; 2 ]} sur ce même alphabet en expliquant comment est reconstruit le dictionnaire de décompression.}
    \tcor{$T = $ {\tt TCTCTCTCTGAG}}
\end{Exercise}

\begin{Exercise}[title = Algorithme {\sc lzw}]
    \Question{Décrire le fonction de l'algorithme {\sc lzw} sur une chaine qui contient $n$ caractères identiques.}
    \Question{On considère la compression d'un texte contenant $5$ caractères différents. Donner une chaîne de caractères de longueur 20 pour laquelle l'algorithme {\sc lzw} ne réduit pas la taille du résultat par rapport à l'entrée.}
\end{Exercise}

\begin{Exercise}[title = diminution de la taille]\\
    Peut-on écrire un proramme qui prend en entrée un fichier et renvoie une version compressée (sans perte d'information) de taille \textit{strictement} inférieure de ce fichier ?
\end{Exercise}

\end{document}
\documentclass[11pt,a4paper]{article}

\usepackage{Act}

\begin{document}
\input{\detokenize{/home/fenarius/Travail/Cours/cpge-info/latex/Macros.tex}}
\ModeExercice
\TD{1}{Langage C}

\setcounter{Exercise}{0}
\begin{Exercise}[title={Déclaration}]
	\Question{Ecrire les instructions permettant de :}
		\subQuestion{Déclarer une variable {\tt n} de type entier.}
		\subQuestion{Déclarer une variable {\tt x} de type flottant initialisée à 1.}
		\subQuestion{Déclarer une variable {\tt test} de type booléen, quelle librairie est nécessaire ?}
		\subQuestion{Déclarer un tableau de 5 entiers initialisés aux valeurs \{1, 4, 9, 16, 25\}}
	\Question{Ecrire les signatures des fonctions suivantes :}
		\subQuestion{{\tt divisible\_par} qui prend en argument deux entiers $n$ et $p$ et renvoie un booléen.}
		\subQuestion{{\tt somme} qui prend en argument un tableau de flottant et renvoie un flottant.}
		\subQuestion{{\tt carre} qui prend en argument un entier  renvoie un entier.}
		\subQuestion{{\tt affiche} qui prend en argument un booléen  et ne renvoie rien.}
\end{Exercise}

\begin{Exercise}[title={Portée}] \\
On considère le programme C suivant :
\begin{langageC}
#include <stdio.h>

const float pi = 3.1415;
int k = 1;

int main() {
	float s = 0;
	int k = 1;
	while (pi * pi / 6 - s > 0.25) {
		float v;
		v = 1.0 / (k * k);
		s += v;
		k = k + 1;
	}
	return 0;
	}

\end{langageC}
\Question{Pour chacune des variables du programme, indiquer si elle est globale ou locale et donner sa portée.}
\Question{Déterminer la valeur de chacune des variables existantes juste avant l'instruction {\tt return} de la ligne 15. \\
\aide \; On peut utiliser une calculatrice !}
\end{Exercise}
	
\begin{Exercise}[title={Conversion}] \\
Déterminer le type et la valeur des expressions suivantes. Indiquer lorsqu'une conversion implicite ou explicite a eu lieu.

\Question{\tt !(5<7)}
\Question{\tt 3 + 0.14}
\Question{\tt (int)7.5 + (int)12.3}
\Question{\tt 7.0 /2}
\Question{\tt (true || false) \&\& (false || true)}
\Question{\tt (int) 19.6 \% 4}
\end{Exercise}

\begin{Exercise}[title={Analyser un programme}] \\
On considère le programme suivant :
\inputC{/home/fenarius/Travail/Cours/cpge-info/docs/mp2i/files/C1/echange.c}{}
\Question{Quel sera le résultat de l'exécution de ce programme ? Pourquoi ?}
\Question{Quel sera l'affichage produit si on déplace l'affichage des variables {\tt a} et {\tt b} dans la fonction {\tt echange} ? Pourquoi ?}
\end{Exercise}

\begin{Exercise}[title={Programmes à commenter}]\\
Que penser des programmes suivants (erreurs, avertissements, comportements indéfinis, \dots ?)
\begin{itemize}
	\item[\textbullet] Programme A :
	\begin{langageC}
		#include <stdio.h>
		int main()
		int tab[5] = {42}
		for (i=0;i<=5;i++){
			printf("%d \n",tab[i]);
		}
	\end{langageC}
	\item[\textbullet] Programme B :
	\begin{langageC}
		#include <stdio.h>
		int main(){
		int tab[5] = {42}
		int i = 0;
		while (true)
			{
				printf("%d \n",tab[i]);
				i = i + 1
				if (i==5) {break;}
			}
		}
	\end{langageC}
	\item[\textbullet] Programme C :
	\begin{langageC}
		#include <stdio.h>
		int programme() {
			float s=0;
			for int (i=0;i<10000;i++)
			{
				s = s + 1/i;
			}
			print("somme =%f\n",%s);
		}
	\end{langageC}
\end{itemize}
\end{Exercise}
\end{document}

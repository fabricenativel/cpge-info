\documentclass[11pt,a4paper]{article}

\usepackage{Act}

\begin{document}
\input{\detokenize{/home/fenarius/Travail/Cours/cpge-info/latex/Macros.tex}}
\ModeExercice
\TD{17}{Graphes}
\newcommand{\SPATH}{/home/fenarius/Travail/Cours/cpge-info/docs/mp2i/files/C14/}


\psset{treesep=0.5cm,levelsep=0.8cm}

\setcounter{Exercise}{0}


\begin{Exercise}[title = {Définition et représentation d'un graphe non orienté}] \\
    On note :
    $S = \{a, b, c, d, e ,f , g\}$ et $A = \{ ab, ac, bc, ef, gf, ed, ce, bg \}$
    \Question{Représenter le graphe non orienté $G = (S,A)$}
    \Question{Donner le degré de chaque sommet.}
    \Question{Donner la représentation de $G$ sous forme de matrice d'adjacence.}
    \Question{Donner la représentation de $G$ sous forme de listes d'adjacence.}
\end{Exercise}

\begin{Exercise}[title = {Définition et représentation d'un graphe  orienté}] \\
    On note :
    $S = \{a, b, c, d, e ,f , g\}$ et $A = \{ ab, ac, bc, ef, gf, ed, ce, bg \}$
    \Question{Représenter le graphe orienté $G = (S,A)$}
    \Question{Donner les degrés entrant et sortants de chaque sommet.}
    \Question{Donner la représentation de $G$ sous forme de matrice d'adjacence.}
    \Question{Donner la représentation de $G$ sous forme de listes d'adjacence.}
\end{Exercise}

\begin{Exercise}[title = {Représentation d'un graphe}]\\
    On considère le graphe suivant : \medskip\\
    \begin{pspicture}(0,-2.2)(5,2.2)
    
		\rput(3,2){\circlenode{x0}{$x_0$}}
		\rput(1,0){\circlenode{x1}{$x_1$}}
		\rput(3,0){\circlenode{x2}{$x_2$}}
		\rput(5,0){\circlenode{x3}{$x_3$}}
		\rput(7,0){\circlenode{x4}{$x_4$}}
	   \rput(3,-2){\circlenode{x5}{$x_5$}}
       \ncarc{-}{x1}{x0}
       \ncarc{-}{x0}{x4}
       \ncarc{-}{x5}{x1}
       \ncarc{-}{x4}{x5}
       \ncline{-}{x1}{x2}
       \ncline{-}{x0}{x2}
       \ncline{-}{x5}{x2}
       \ncline{-}{x3}{x2}
       \ncline{-}{x3}{x4}
    \end{pspicture}
    \Question{Donner sa représentation sous forme de matrice d'adjacence.}
    \Question{Donner sa représentation sous forme de listes d'adjacence.}
    \Question{Quel est le sommet de plus haut degré ? Donner la liste de ses voisins.}
\end{Exercise}

\begin{Exercise}[title = {Représentation d'un graphe orienté}]\\
    On considère le graphe $G$ représenté ci-dessous : \medskip\\
    \begin{pspicture}(0,-2.2)(5,2.2)
        \psset{arrowsize=0.15}
		\rput(3,2){\circlenode{x0}{$x_0$}}
		\rput(1,0){\circlenode{x1}{$x_1$}}
		\rput(3,0){\circlenode{x2}{$x_2$}}
		\rput(5,0){\circlenode{x3}{$x_3$}}
		\rput(7,0){\circlenode{x4}{$x_4$}}
	   \rput(3,-2){\circlenode{x5}{$x_5$}}
       \ncarc{->}{x1}{x0}
       \ncarc{->}{x0}{x4}
       \ncarc{->}{x5}{x1}
       \ncarc{->}{x4}{x5}
       \ncline{<-}{x1}{x2}
       \ncline{->}{x0}{x2}
       \ncline{->}{x5}{x2}
       \ncline{->}{x3}{x2}
       \ncline{<-}{x3}{x4}
    \end{pspicture}
    \Question{Donner sa représentation sous forme de matrice d'adjacence.}
    \Question{Donner sa représentation sous forme de listes d'adjacence.}
    \Question{Donner les degrés entrants et sortants de chaque sommet.}
    \Question{Donner ${\cal V_+}(x_0)$ et ${\cal V_-}(x_1)$}
    \Question{Dessiner le graphe dont la matrice d'adjacence est la transposée de celle de ce graphe.}
    \
\end{Exercise}

\begin{Exercise}[title = {Graphe régulier, graphe complet}]\\
    Les graphes considérés dans cet exercice sont non orientés. On dit qu'un graphe  $G = (S,A)$ est \textit{régulier} lorsque tous ses sommets ont le même degré. Et on dit qu'un graphe est \textit{complet} lorsque qu'il y a une arête entre tous les paires de sommets
    \Question{Dessiner un graphe non orienté régulier de taille 6 dont les sommets sont de degré 3}
    \Question{Dessiner un graphe complet de taille 5}
    \Question{Déterminer le nombre d'arêtes du graphe complet à $n$ sommets}
    \Question{Un graphe complet est-il régulier ?}
    \Question{Peut-on construire un graphe régulier de taille 5 dont tous les sommets sont de degré 3 ?}
    \Question{A quelle condition portant sur $n$ et $k$ peut-on construire un graphe régulier de taille $n$ dont tous les sommets sont de degré $k$ ?}
\end{Exercise}

\begin{Exercise}[title = {Sommet isolé}]\\
    On dit qu'un sommet d'un graphe non orienté $G=(S,A)$ est \textit{isolé} lorsque son degré est nul.
    \Question{Montrer qu'un graphe ne peut avoir simultanément un sommet isolé et un sommet de degré $|S|-1$}
    \Question{En déduire qu'un graphe a au moins deux sommets de même degré.}
\end{Exercise}

\begin{Exercise}[title = {Parité}]\\
    Soit $G = (S,A)$ un graphe non orienté, on note $d(x)$ le degré d'un sommet $x \in A$.
    \Question{Montrer que $\displaystyle{\sum_{x \in A} d(x) = 2|A|}$}
    \Question{En déduire que $G$ a forcément un nombre pair de sommets de degré impair}
\end{Exercise}

\begin{Exercise}[title = {Un peu de dénombrement}]
\Question{Montrer qu'il y a $2^{\frac{n(n-1)}{2}}$ graphes non orientés à $n$ sommets.}
\Question{Déterminer le nombre de graphes orientés à $n$ sommets.}
\end{Exercise}

\begin{Exercise}[title = {Matrice d'adjacence et nombre de chemins}] \\
Soit $G=(S,A)$ et $M$ sa matrice d'adjacence, le but de l'exercice est de calculer le nombre de chemins de longueur $k$ entre deux sommets $i$ et $j$ d'un graphe qu'on notera $c_{i,j,k}$.
\Question{Montrer que $c_{i,j,1} = M_{i,j}$}
\Question{Montrer que pour tout $k \in \N$, $c_{i,j,k} = M^k_{i,j}$} (on pourra raisonner par récurrence)
\Question{En supposant qu'on calcule $M^k$ avec l'algorithme d'exponentiation rapide, donner la complexité de cette méthode pour calculer les $c_{i,j,k}$}
\end{Exercise}

\end{document}
\documentclass[11pt,a4paper]{article}

\usepackage{Act}

\begin{document}
\input{\detokenize{/home/fenarius/Travail/Cours/cpge-info/latex/Macros.tex}}
\ModeExercice
\TD{17}{Graphes}
\newcommand{\SPATH}{/home/fenarius/Travail/Cours/cpge-info/docs/mp2i/files/C14/}


\psset{treesep=0.5cm,levelsep=0.8cm}

\setcounter{Exercise}{0}


\begin{Exercise}[title = {Définition et représentation d'un graphe non orienté}] \\
    On note :
    $S = \{a, b, c, d, e ,f , g\}$ et $A = \{ ab, ac, bc, ef, gf, ed, ce, bg \}$
    \Question{Représenter le graphe $G = (S,A)$}
    \Question{Donner le degré de chaque sommet.}
    \Question{Donner la représentation de $G$ sous forme de matrice d'adjacence.}
    \Question{Donner la représentation de $G$ sous forme de listes d'adjacence.}
\end{Exercise}

\end{document}
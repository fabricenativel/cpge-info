\documentclass[11pt,a4paper]{article}

\usepackage{Act}

\begin{document}
\input{\detokenize{/home/fenarius/Travail/Cours/cpge-info/latex/Macros.tex}}
\ModeExercice
\TD{17}{Graphes}
\newcommand{\SPATH}{/home/fenarius/Travail/Cours/cpge-info/docs/mp2i/files/C14/}


\psset{treesep=0.5cm,levelsep=0.8cm}

\setcounter{Exercise}{0}


\begin{Exercise}[title = {Définition et représentation d'un graphe non orienté}] \\
    On note :
    $S = \{a, b, c, d, e ,f , g\}$ et $A = \{ ab, ac, bc, ef, gf, ed, ce, bg \}$
    \Question{Représenter le graphe non orienté $G = (S,A)$}
    \Question{Donner le degré de chaque sommet.}
    \Question{Donner la représentation de $G$ sous forme de matrice d'adjacence.}
    \Question{Donner la représentation de $G$ sous forme de listes d'adjacence.}
\end{Exercise}

\begin{Exercise}[title = {Définition et représentation d'un graphe  orienté}] \\
    On note :
    $S = \{a, b, c, d, e ,f , g\}$ et $A = \{ ab, ac, bc, ef, gf, ed, ce, bg \}$
    \Question{Représenter le graphe orienté $G = (S,A)$}
    \Question{Donner les degrés entrant et sortants de chaque sommet.}
    \Question{Donner la représentation de $G$ sous forme de matrice d'adjacence.}
    \Question{Donner la représentation de $G$ sous forme de listes d'adjacence.}
\end{Exercise}

\begin{Exercise}[title = {Graphe régulier, graphe complet}]\\
    Les graphes considérés dans cet exercice sont non orientés. On dit qu'un graphe  $G = (S,A)$ est \textit{régulier} lorsque tous ses sommets ont le même degré. Et on dit qu'un graphe est \textit{complet} lorsque qu'il y a une arête entre tous les couples de sommets
    \Question{Dessiner un graphe non orienté régulier de taille 6 dont les sommets sont de degré 3}
    \Question{Dessiner un graphe complet de taille 5}
    \Question{Déterminer le nombre d'arête du graphe complet à $n$ sommets}
    \Question{Un graphe complet est-il régulier ?}
    \Question{Peut-on construire un graphe régulier de taille 5 dont tous les sommets sont de degré 3 ?}
    \Question{A quelle condition portant sur $n$ et $k$ peut-on construire un graphe régulier de taille $n$ dont tous les sommets sont de degré $k$ ?}
\end{Exercise}

\begin{Exercise}[title = {Parité}]\\
    Soit $G = (S,A)$ un graphe non orienté, on note $d(x)$ le degré d'un sommet $x \in A$.
    \Question{Montrer que $\displaystyle{\sum_{x \in A} d(x) = 2|A|}$}
    \Question{En déduire que $G$ a forcément un nombre pair de sommets de degré impair}
\end{Exercise}

\end{document}
\documentclass[11pt,a4paper]{article}

\usepackage{Act}

\begin{document}
\input{\detokenize{/home/fenarius/Travail/Cours/cpge-info/latex/Macros.tex}}
\ModeExercice
\TD{7}{Terminaison, correction, complexité}

\setcounter{Exercise}{0}
\begin{Exercise}[title={Notation $O$}]
\Question{Déterminer un $O$ des suites de terme général :}
\subQuestion{$2023 n^2$}
\subQuestion{$n^2 + 10^9 n$}
\subQuestion{$3 n + 7 \log n$}
\subQuestion{$2^{n+7} + n^{10}$}
\subQuestion{$\sqrt{19n^2+3}$}
\Question{Montrer que si $u_n = O(v_n)$ et $v_n = O(w_n)$ alors $u_n = O(w_n)$.}
\Question{Montrer que $O(u_n + v_n) = O(\max(u_n,v_n)$).}
\Question{Montrer que si $u_n = O(a_n)$ et $v_n = O(b_n)$ alors $u_nv_n = O(a_nb_n)$.}
\Question{Déterminer un $O$ (le \og{} meilleur \fg{} possible) des expressions suivantes :}
\subQuestion{$O(n^4)+O(n^2)$}
\subQuestion{$O(n^5)+O(n^5)$}
\subQuestion{$O(n^3)+O(\log(n))$}
\subQuestion{$O(n^4)\times O(n^3)$}
\subQuestion{$O(n^4)\times O(\sqrt{n})$}
\end{Exercise}

\begin{Exercise}[title={Vérification du tri}]
    \Question{Ecrire un algorithme permettant de vérifier qu'un tableau est trié par ordre croissant.}
    \Question{Prouver que votre algorithme est correct.}
    \Question{Déterminer sa complexité.}
\end{Exercise}

\begin{Exercise}[title={multiplier en additionnant}]
    \inputpartC{add_mult.c}{}{}{3}{8}
\Question{En supposant $p>0$ montrer la terminaison.}
\Question{Prouver que cette fonction renvoie $p \times n$.}
\Question{Déterminer sa complexité.}
\end{Exercise}

\begin{Exercise}[title={exponentiation rapide}]\\
On rappelle la fonction d'exponentiation rapide dans sa version récursive :
\inputpartC{exp_rapide.c}{}{}{5}{13}
\Question{Prouver que cet algorithme termine.}
\Question{Prouver qu'il est correct.\\
\aide \; En notant $n_0$ la valeur initiale de $n$, on pourra considérer l'invariant suivant : {\tt res * cp\textsuperscript{n}}=$a^{n_0}$} 
\Question{Donner sa complexité.} 
\end{Exercise}

\end{document}
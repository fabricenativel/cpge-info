\documentclass[11pt,a4paper]{article}

\usepackage{Act}

\begin{document}
\input{\detokenize{/home/fenarius/Travail/Cours/cpge-info/latex/Macros.tex}}
\ModeExercice
\TD{12}{Arbres binaires}
\newcommand{\SPATH}{/home/fenarius/Travail/Cours/cpge-info/docs/mp2i/files/C12/}
\newcommand{\CN}[1]{\TCircle[radius=0.25]{#1}}
\setcounter{Exercise}{0}

\begin{Exercise}[title = {Représentation d'arbres binaires}]
    \Question{Dessiner tous les arbres binaires ayant 3 noeuds.}
    \Question{Dessiner tous les arbres binaires ayant 4 noeuds.}
    \Question{Dessiner un arbre binaire ayant 8 noeuds et de hauteur maximale (resp. minimale).}
\end{Exercise}

\begin{Exercise}[title = {Représentation en C}]\\
    On rappelle qu'on a défini en C, un arbre binaire (avec des étiquettes entières) par :
    \inputpartOCaml{\SPATH/arbres_binaires.c}{}{}{4}{11}
\Question{Rappeler la définition de la hauteur d'un arbre binaire et écrire une fonction de prototype \\
\mintinline{c}{int hauteur(ab arbrebinaire)}} qui renvoie la hauteur de l'arbre donné en argument.
\Question{On rappelle que dans cette implémentation, l'espace nécessaire au stockage des noeuds est alloué dynamiquement à l'aide d'instructions {\tt malloc}. Ecrire une fonction de prototype \\
\mintinline{c}{void libere(ab* arbrebinaire)} qui détruit l'arbre binaire donné en paramètre, en libérant l'espace alloué par ses noeuds. A la fin de l'appel {\tt ab} vaut {\tt NULL}.} 
\end{Exercise}


\begin{Exercise}[title = {Représentation en OCaml}]\\
    \label{abcaml}
    On rappelle qu'on a défini en OCaml un arbre binaire (avec des étiquettes entières) par :
    \inputpartOCaml{\SPATH/arbres_binaires_int.ml}{}{}{1}{3}
  \Question{Dessiner l'arbre représenté par :
  \inputpartOCaml{\SPATH/arbres_binaires_int.ml}{}{}{46}{54}
  }
  \Question{Donner sa taille et sa hauteur}
  \Question{S'agit-il d'un arbre binaire de recherche ? Justifier}
  \Question{Donner la représentation en OCaml de l'arbre :\\
  \pstree[arrows=->,treesep=0.8cm,levelsep=1cm]{\CN{$7$}}
					{
						\pstree{\CN{$2$}}{
							\Tn{}
							\CN{$5$}
						}
						\pstree{\CN{$7$}}{
							\CN{$1$}
							\Tn{}
						}
					}
  }
\end{Exercise}

\begin{Exercise}[title = {Arbre binaire aléatoire}]\\
    Dans le langage de votre choix, écrire une fonction {\tt arbre\_aleatoire} qui prend en argument un entier $n$ et renvoie un arbre binaire aléatoire ayant $n$ noeuds. Les étiquettes sont les entiers de 1 à $n$.
\end{Exercise}


\begin{Exercise}[title = {Un peu de dénombrement}]\\
    On note $T_n$ le nombre d'arbres binaires à $n$ noeuds.
    \Question{Donner $T_0$ et déterminer une relation de récurrence liant les $(T_k)_{0 \leqslant k \leqslant n}$ \\
    {\small \aide}\ Utiliser la définition par récurrence des arbres binaires.}
    \Question{Vérifier que $T_5 = 42$.}
    \Question{Le nombre de Catalan d'indice $n$ est défini par : $$ C_n = \dfrac{1}{n+1} \binom{2n}{n}$$
    Prouver que $T_n = C_n$.
    }
\end{Exercise}

\begin{Exercise}[title = {Parcours d'un arbre binaire}] \vspace{0.2cm}\\ 
    \psset{arrows=->,treesep=0.8cm,levelsep=0.8cm, radius=0.3cm}
    \begin{tabularx}{\textwidth}{Y|Y|Y}
        \pstree{\TCircle{\tt 29}}
        {\pstree{\TCircle{\tt 24}}
        {\pstree{\TCircle{\tt 14}}
        { \Tn{} 
        \pstree{\TCircle{\tt 16}}
        { \Tn{} 
        \TCircle{\tt 20} 
        }}\TCircle{\tt 28} 
        }\pstree{\TCircle{\tt 31}}
        { \Tn{} 
        \TCircle{\tt 30} 
         
          }} &
          \pstree{\TCircle{22}}
{\pstree{\TCircle{20}}
{\pstree{\TCircle{14}}
{ \Tn{} 
\TCircle{18} 
}\TCircle{21} 
}\pstree{\TCircle{31}}
{ \Tn{} 
\pstree{\TCircle{24}}
{ 
\TCircle{27} 
\Tn{} 
} 
 \Tn{} }}
          & 
          \pstree{\TCircle{26}}
{\pstree{\TCircle{16}}
{\pstree{\TCircle{12}}
{ \Tn{} 
\TCircle{15} 
}\pstree{\TCircle{27}}
{\TCircle{17} 
\TCircle{24} 
}}\TCircle{30} 
}
          \\
          $T_1$ & $T_2$ & $T_2$ \\
    \end{tabularx}
\Question{Pour chacun des trois arbres binaires ci-dessus, donner l'ordre des noeuds lors d'un parcours prefixe, infixe et suffixe.}
\Question{Lequel de ces arbres binaires est un {\sc abr} ? Justifier}
\end{Exercise}

\begin{Exercise}[title = {Un peu de complexité}]\\
On considère la fonction OCaml suivante qui prend en argument un arbre binaire tel que défini par le type de l'exercice \ref{abcaml}
\inputpartOCaml{\SPATH/arbres_binaires_int.ml}{}{}{45}{48}
\Question{Ecrire une spécification et donner un nom plus approprié à la fonction {\tt mystère}.}
\Question{Rappeler la complexité de l'opérateur {\tt @} et en déduire celle de la fonction {\tt mystere}}
\Question{Proposer une version de cette fonction ayant une complexité linéaire en fonction du nombre de noeuds de l'arbre.\\
{\small \aide\;} Utiliser une fonction auxiliaire avec un accumulateur.
}
\end{Exercise}


\end{document}
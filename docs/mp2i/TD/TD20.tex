\documentclass[11pt,a4paper]{article}

\usepackage{Act}

\begin{document}
\input{\detokenize{/home/fenarius/Travail/Cours/cpge-info/latex/Macros.tex}}
\ModeExercice
\TD{20}{Algorithmes des graphes}
\newcommand{\SPATH}{/home/fenarius/Travail/Cours/cpge-info/docs/mp2i/files/C20/}


\psset{treesep=0.5cm,levelsep=0.8cm}

\setcounter{Exercise}{0}


\begin{Exercise}[title = {Parcours d'un graphe non orienté}]\\
    On considère le graphe $G$ représenté ci-dessous : \medskip \\
    \begin{pspicture}(0,-2.2)(5,2.2)
        \psset{arrowsize=0.15}
		\rput(3,2){\circlenode{x0}{$x_0$}}
		\rput(1,0){\circlenode{x1}{$x_1$}}
		\rput(3,0){\circlenode{x2}{$x_2$}}
		\rput(5,0){\circlenode{x3}{$x_3$}}
		\rput(7,0){\circlenode{x4}{$x_4$}}
	   \rput(3,-2){\circlenode{x5}{$x_5$}}
       \ncarc{-}{x0}{x4}
       \ncarc{-}{x4}{x5}
       \ncline{-}{x1}{x2}
       \ncline{-}{x0}{x2}
       \ncline{-}{x5}{x2}
       \ncline{-}{x3}{x2}
       \ncline{-}{x3}{x4}
    \end{pspicture}
    \Question{Donner le résultat d'un parcours en largeur de ce graphe en démarrant du sommet $x_1$.}
    \Question{Même question en démarrant du sommet $x_2$.}
    \Question{Donner le résultat d'un parcours en profondeur de ce graphe en démarrant du sommet $x_1$.}
    \Question{Même question en démarrant du sommet $x_2$.}
\end{Exercise}
\end{document}
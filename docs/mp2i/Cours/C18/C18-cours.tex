\PassOptionsToPackage{dvipsnames,table}{xcolor}
\documentclass[10pt]{beamer}
\usepackage{Cours}

\begin{document}


\input{\detokenize{/home/fenarius/Travail/Cours/cpge-info/latex/MacrosCours.tex}}

% Numéro et titre de chapitre
\setcounter{numchap}{18}
\newcommand{\Ctitle}{\cnum {Modèle Entité-Association}}
\newcommand{\SPATH}{/home/fenarius/Travail/Cours/cpge-info/docs/mp2i/files/C\thenumchap/}


\makess{Modèle entité-association}
\begin{frame}{\Ctitle}{\stitle}
	\begin{alertblock}{Définitions}
		\begin{itemize}
			\item<1-> Une \textcolor{blue}{entité} est une modélisation d'un objet concret ou abstrait à propos duquel on souhaite conserver des informations. Une entité doit pouvoir être identifiée de façon unique via un \textcolor{blue}{identifiant d'identité}.
				\onslide<2->\textcolor{gray}{Par exemple un livre (identifié par son {\sc isbn}), une facture (identifié par son code), un client (identifié par son email), un anniversaire (identifié par une personne et une date), une transaction commerciale (identifié par un code) \dots} \\
			\item<3-> Une entité possède un ou plusieurs \textcolor{blue}{attributs}.
				\onslide<4->\textcolor{gray}{Par exemple, l'entité \textit{film} peut avoir les attributs date, titre, année, \dots}
			\item<5-> Une \textcolor{blue}{instance} d'une entité est un objet en particulier.
				\onslide<6-> \textcolor{gray}{Par exemple, \textit{Forrest Gump} est une instance de l'entité \textit{Film}.}
			\item<7->  Une \textcolor{blue}{association} est un lien entre plusieurs entités. Le degré d'une association est le nombre d'entités intervenant dans l'association.
				\onslide<8->\textcolor{gray}{Par exemple, l'association \textit{écrit} de degré 2, relie l'entité \textit{auteur}  à l'entité \textit{livre}}
		\end{itemize}
	\end{alertblock}
\end{frame}

\makess{Modèle entité-association}
\begin{frame}{\Ctitle}{\stitle}
	\begin{alertblock}{Définitions}
		\begin{itemize}
			\item<2-> Pour les associations de degré 2 (binaire), on précise de chaque côté d'une association le nombre d'entités concernées. C'est la \textcolor{blue}{cardinalité} de l'association qui se résume à trois types principaux :
				\begin{itemize}
					\item<3-> \textcolor{blue}{1--1} association directe et exclusive entre deux entités (\textit{one to one}).
						\onslide<4->\textcolor{gray}{Par exemple, un \textit{lycée} a un \textit{proviseur}, un \textit{pays} a une seule \textit{capitale}.}
					\item<5-> \textcolor{blue}{1--*} (aussi noté \textcolor{blue}{1--1..N})  association d'une instance de la première entité à un ensemble d'instances de la seconde (\textit{one to many}).
						\onslide<6->\textcolor{gray}{Par exemple, un \textit{propriétaire} peut avoir plusieurs \textit{voitures}, un \textit{client} peut avoir plusieurs \textit{numéro de téléphone}.}
					\item<7-> \textcolor{blue}{*--*} (aussi noté \textcolor{blue}{1..N--1..N}) association d'un ensemble d'instances à un autre ensemble d'instance.
						\onslide<8->\textcolor{gray}{Par exemple, un \textit{livre} peut avoir plusieurs \textit{auteurs} et un \textit{auteur} peut écrire plusieurs \textit{livres}.}
				\end{itemize}
			\item<9-> Les associations de types \textcolor{blue}{*--*} peuvent être séparées entre deux associations de type \textcolor{blue}{1--*} à l'aide d'une nouvelle entité. \\
				\onslide<10->\textcolor{gray}{\small Par exemple, en créant une entité \textit{attribution}, un \textit{livre}  a plusieurs \textit{attributions} (car il a été écrit par plusieurs \textit{auteurs}) et un auteur à plusieurs \textit{attributions} (car il a écrit plusieurs livres)}
		\end{itemize}
	\end{alertblock}
\end{frame}


\begin{frame}{\Ctitle}{\stitle}
	\begin{exampleblock}{Exemples}
		\begin{itemize}
			\item<1-> Un exemple d'association \textit{one to one} :
				\begin{center}
					\renewcommand{\arraystretch}{1.1}
					\begin{tabular}{|p{2cm}|}
						\hline
						\multicolumn{1}{|c|}{\cellcolor{lightgray}{\small \textbf{Pays}}} \\
						\hline
						\cellcolor{white}\underline{\textbf {\footnotesize nom}}          \\
						\hline
						\cellcolor{white}{\footnotesize region}                           \\
						\hline
						\cellcolor{white}{\footnotesize population}                       \\
						\hline
						\cellcolor{white}{\footnotesize surface}                          \\
						\hline
					\end{tabular} \rnode{D}{$1$} \quad \quad \rnode{A}{$1$}
					\ncline[offset=-0.2cm,nodesep=-0.3cm,linewidth=1pt]{D}{A}
					\begin{tabular}{|p{2cm}|}
						\hline
						\multicolumn{1}{|c|}{\cellcolor{lightgray}{\small \textbf{Capitale}}} \\
						\hline
						\cellcolor{white}\underline{\textbf {\footnotesize nomcap}}              \\
						\hline
						\cellcolor{white}{\footnotesize longitude}                            \\
						\hline
						\cellcolor{white}{\footnotesize latitude}                             \\
						\hline
					\end{tabular}
				\end{center}
			\item<2-> Un exemple d'association \textit{one to many} :
				\begin{center}
					\renewcommand{\arraystretch}{1.1}
					\begin{tabular}{|p{2cm}|}
						\hline
						\multicolumn{1}{|c|}{\cellcolor{lightgray}{\small \textbf{Client}}} \\
						\hline
						\cellcolor{white}\underline{\textbf {\footnotesize num}}            \\
						\hline
						\cellcolor{white}{\footnotesize nom}                                \\
						\hline
						\cellcolor{white}{\small prenom}                                    \\
						\hline
						\cellcolor{white}{\footnotesize email}                              \\
						\hline
					\end{tabular} \rnode{D}{$1$} \quad \quad \rnode{A}{\phantom{1}$*$}
					\ncline[offset=-0.2cm,nodesepA=-0.3cm,nodesepB=-0.5cm,linewidth=1pt]{D}{A}
					\begin{tabular}{|p{2cm}|}
						\hline
						\multicolumn{1}{|c|}{\cellcolor{lightgray}{\small \textbf{Commande}}} \\
						\hline
						\cellcolor{white}\underline{\textbf {\footnotesize id}}               \\
						\hline
						\cellcolor{white}{\footnotesize prix}                                 \\
						\hline
						\cellcolor{white}{\footnotesize articles}                             \\
						\hline
						\cellcolor{white}{\footnotesize dates}                                \\
						\hline
					\end{tabular}
				\end{center}
		\end{itemize}
	\end{exampleblock}
\end{frame}


\begin{frame}{\Ctitle}{\stitle}
	\begin{exampleblock}{Exemples}
		\begin{itemize}
			\item<2-> Un exemple d'association \textit{many to many}:
				\begin{center}
					\renewcommand{\arraystretch}{1.1}
					\begin{tabular}{|p{2cm}|}
						\hline
						\multicolumn{1}{|c|}{\cellcolor{lightgray}{\small \textbf{Etudiant}}} \\
						\hline
						\cellcolor{white}\underline{\textbf {\footnotesize ine}}              \\
						\hline
						\cellcolor{white}{\footnotesize nom}                                  \\
						\hline
						\cellcolor{white}{\footnotesize prenom}                               \\
						\hline
						\cellcolor{white}{\footnotesize adresse}                              \\
						\hline
					\end{tabular} \rnode{D}{$*$} \quad \quad \rnode{A}{$*$}
					\ncline[offset=-0.22cm,nodesep=-0.3cm,linewidth=1pt]{D}{A}
					\begin{tabular}{|p{2cm}|}
						\hline
						\multicolumn{1}{|c|}{\cellcolor{lightgray}{\small \textbf{Cours}}} \\
						\hline
						\cellcolor{white}\underline{\textbf {\footnotesize id}}            \\
						\hline
						\cellcolor{white}{\footnotesize matière}                           \\
						\hline
						\cellcolor{white}{\footnotesize durée}                             \\
						\hline
						\cellcolor{white}{\footnotesize description}                       \\
						\hline
					\end{tabular}
				\end{center}
			\item<3-> Sa transformation en deux associations \textit{one to many} à l'aide d'une table de liaison
				\begin{center}
					\renewcommand{\arraystretch}{1.1}
					\begin{tabular}{|p{2cm}|}
						\hline
						\multicolumn{1}{|c|}{\cellcolor{lightgray}{\small \textbf{Etudiant}}} \\
						\hline
						\cellcolor{white}\underline{\textbf {\footnotesize ine}}              \\
						\hline
						\cellcolor{white}{\footnotesize nom}                                  \\
						\hline
						\cellcolor{white}{\footnotesize prenom}                               \\
						\hline
						\cellcolor{white}{\footnotesize adresse}                              \\
						\hline
					\end{tabular} \rnode{D}{$1$} \quad \quad \rnode{E}{\phantom{1}$*$}
					\ncline[offset=-0.22cm,nodesepA=-0.3cm,nodesepB=-0.45cm,linewidth=1pt]{D}{E}
					\begin{tabular}{|p{2cm}|}
						\hline
						\multicolumn{1}{|c|}{\cellcolor{lightgray}{\small \textbf{Inscription}}} \\
						\hline
						\cellcolor{white}\underline{\textbf {\footnotesize Etudiant}}            \\
						\hline
						\cellcolor{white}\underline{\textbf {\footnotesize Cours}}               \\
						\hline
					\end{tabular}
					\rnode{F}{\phantom{1}$*$} \quad \quad \rnode{A}{$1$}
					\ncline[offset=-0.22cm,nodesepA=-0.45cm,nodesepB=-0.3cm,linewidth=1pt]{F}{A}
					\begin{tabular}{|p{2cm}|}
						\hline
						\multicolumn{1}{|c|}{\cellcolor{lightgray}{\small \textbf{Cours}}} \\
						\hline
						\cellcolor{white}\underline{\textbf {\footnotesize id}}            \\
						\hline
						\cellcolor{white}{\footnotesize matière}                           \\
						\hline
						\cellcolor{white}{\footnotesize durée}                             \\
						\hline
						\cellcolor{white}{\footnotesize description}                       \\
						\hline
					\end{tabular}
				\end{center}
		\end{itemize}
	\end{exampleblock}
\end{frame}

\makess{Du modèle EA au modèle relationnel}
\begin{frame}{\Ctitle}{\stitle}
	\begin{block}{Méthode}
		Pour passer du modèle entité association au modèle relationnel :
		\begin{itemize}
			\item<1-> Une entité devient une relation (c'est à dire une table)
			\item<2-> L'identifiant d'identité devient la clé primaire de cette table
			\item<3-> On transforme les associations suivant les cas de figure
		\end{itemize}
	\end{block}
\end{frame}

\begin{frame}{\Ctitle}{\stitle}
	\begin{block}{Cas des associations \textit{one to one} : fusion}
		\onslide<1->{Deux entités associées en \textit{one to one} peuvent fusionner dans la même relation. \\}
		\onslide<2->\textcolor{gray}{Par exemple, les entités \textit{pays} et \textit{capitale} peuvent fusionner dans une seule table \textit{pays} en ajoutant dans cette table les attributs des capitales.}
		\onslide<3->
		\begin{center}
			\renewcommand{\arraystretch}{1.1}
			\begin{tabular}{|p{2cm}|}
				\hline
				\multicolumn{1}{|c|}{\cellcolor{lightgray}{\small \textbf{Pays}}} \\
				\hline
				\cellcolor{white}\underline{\textbf {\footnotesize nom}}          \\
				\hline
				\cellcolor{white}{\footnotesize region}                           \\
				\hline
				\cellcolor{white}{\footnotesize population}                       \\
				\hline
				\cellcolor{white}{\footnotesize surface}                          \\
				\hline
			\end{tabular} \rnode{D}{$1$} \quad \quad \rnode{A}{$1$}
			\ncline[offset=-0.2cm,nodesep=-0.3cm,linewidth=1pt]{D}{A}
			\begin{tabular}{|p{2cm}|}
				\hline
				\multicolumn{1}{|c|}{\cellcolor{lightgray}{\small \textbf{Capitale}}} \\
				\hline
				\cellcolor{white}\underline{\textbf {\footnotesize nomcap}}              \\
				\hline
				\cellcolor{white}{\footnotesize longitude}                            \\
				\hline
				\cellcolor{white}{\footnotesize latitude}                             \\
				\hline
			\end{tabular} \rnode{X}{} \quad \quad \rnode{Y}{}
			\ncline[doubleline=true,doublesep=2pt,doublecolor=blue,linecolor=blue,linewidth=1pt,arrowsize=10pt,arrowinset=0.2,arrowlength=1.2]{->}{X}{Y}
			\begin{tabular}{|p{2cm}>{\footnotesize \sc}r|}
				\hline
				\multicolumn{2}{|c|}{\cellcolor{lightgray}{\small \textbf{Pays}}}                   \\
				\hline
				\cellcolor{white}\underline{\textbf {\footnotesize nom}} & \cellcolor{white}{text}  \\
				\hline
				\cellcolor{white}{\footnotesize region}                  & \cellcolor{white}{text}  \\
				\hline
				\cellcolor{white}{\footnotesize population}              & \cellcolor{white}{int}   \\
				\hline
				\cellcolor{white}{\footnotesize surface}                 & \cellcolor{white}{float} \\
				\hline
				\cellcolor{white}{\footnotesize nomcap}                & \cellcolor{white}{text}  \\
				\hline
				\cellcolor{white}{\footnotesize longitude}               & \cellcolor{white}{float} \\
				\hline
				\cellcolor{white}{\footnotesize latitude}                & \cellcolor{white}{float} \\
				\hline
			\end{tabular}
		\end{center}
		\onslide<4->{
			On obtient alors le schéma relationnel suivant : \smallskip
			\begin{mdframed}
				\textbf{Pays} (\underline{nom}, region, population, surface, capitale, longitude, latitude)
			\end{mdframed}}
	\end{block}
\end{frame}


\begin{frame}{\Ctitle}{\stitle}
	\begin{block}{Cas des associations \textit{one to one} : clé étrangère}
		on peut aussi choisir de garder les deux entités séparées et donc dans deux relations différentes, on introduit alors le concept de \textcolor{blue}{clé étrangère} c'est à dire la clé primaire d'une autre table qui indique dans l'une des tables la référence vers l'autre\\
		\onslide<2->{\begin{center}
				\renewcommand{\arraystretch}{1.1}
				\begin{tabular}{|l|}
					\hline
					\multicolumn{1}{|c|}{\cellcolor{lightgray}{\small \textbf{Pays}}} \\
					\hline
					\cellcolor{white}\underline{\textbf {\footnotesize nom}}          \\
					\hline
					\cellcolor{white}{\footnotesize region}                           \\
					\hline
					\cellcolor{white}{\footnotesize population}                       \\
					\hline
					\cellcolor{white}{\footnotesize surface}                          \\
					\hline
				\end{tabular} \rnode{D}{$1$} \  \rnode{A}{$1$}
				\ncline[offset=-0.2cm,nodesep=-0.3cm,linewidth=1pt]{D}{A}
				\begin{tabular}{|l|}
					\hline
					\multicolumn{1}{|c|}{\cellcolor{lightgray}{\small \textbf{Capitale}}} \\
					\hline
					\cellcolor{white}\underline{\textbf {\footnotesize nom}}              \\
					\hline
					\cellcolor{white}{\footnotesize longitude}                            \\
					\hline
					\cellcolor{white}{\footnotesize latitude}                             \\
					\hline
				\end{tabular} \rnode{X}{} \quad \quad \rnode{Y}{}
				\ncline[doubleline=true,doublesep=2pt,doublecolor=blue,linecolor=blue,linewidth=1pt,arrowsize=10pt,arrowinset=0.2,arrowlength=1.2]{->}{X}{Y}
				\begin{tabular}{|l>{\footnotesize \sc}r|}
					\hline
					\multicolumn{2}{|c|}{\cellcolor{lightgray}{\small \textbf{Pays}}}                                      \\
					\hline
					\cellcolor{white}\underline{\textbf {\footnotesize nom}}        & \cellcolor{white}{text}              \\
					\hline
					\cellcolor{white}{\footnotesize region}                         & \cellcolor{white}{text}              \\
					\hline
					\cellcolor{white}{\footnotesize population}                     & \cellcolor{white}{int}               \\
					\hline
					\cellcolor{white}{\footnotesize surface}                        & \cellcolor{white}{float}             \\
					\hline
					\cellcolor{white}{\footnotesize \textcolor{Sepia}{\# capitale}} & \cellcolor{white}{\rnode{DCE}{text}} \\
					\hline
				\end{tabular} \quad \quad
				\begin{tabular}{|l>{\footnotesize \sc}r|}
					\hline
					\multicolumn{2}{|c|}{\cellcolor{lightgray}{\small \textbf{Capitale}}}                            \\
					\hline
					\cellcolor{white}\underline{\textbf {\footnotesize \rnode{ACE}{nom}}} & \cellcolor{white}{text}  \\
					\hline
					\cellcolor{white}{\footnotesize longitude}                            & \cellcolor{white}{float} \\
					\hline
					\cellcolor{white}{\footnotesize latitude}                             & \cellcolor{white}{float} \\
					\hline
				\end{tabular}
				\ncangle[nodesepA=0.1cm,angleA=0,angleB=180]{*->}{DCE}{ACE}
			\end{center}}
		\onslide<3->{
			On obtient alors le schéma relationnel suivant : \smallskip
			\begin{mdframed}
				\textbf{Pays} (\underline{nom}, region, population, surface, \#capitale) \\
				\textbf{Capitale} (\underline{nom}, longitude, latitude)
			\end{mdframed}}
		\onslide<4->{\textcolor{BrickRed}{\small \danger\;} \textcolor{BrickRed}{Intégrité référentielle} : un pays doit avoir une capitale !}
	\end{block}
\end{frame}

\begin{frame}{\Ctitle}{\stitle}
	\begin{block}{Cas des associations \textit{one to many}}
		On utilise là aussi la \textcolor{blue}{clé étrangère} de façon à ce qu'un élément du côté \textit{"many"} de l'association soit associé à un unique élément du côté \textit{"one"}. \\
		\onslide<2->{\begin{center}
				\renewcommand{\arraystretch}{1.1}
				\begin{tabular}{|l|}
					\hline
					\multicolumn{1}{|c|}{\cellcolor{lightgray}{\small \textbf{Client}}} \\
					\hline
					\cellcolor{white}\underline{\textbf {\footnotesize num}}            \\
					\hline
					\cellcolor{white}{\footnotesize nom}                                \\
					\hline
					\cellcolor{white}{\small prenom}                                    \\
					\hline
					\cellcolor{white}{\footnotesize email}                              \\
					\hline
				\end{tabular} \rnode{D}{$1$} \  \rnode{A}{\phantom{1}$*$}
				\ncline[offset=-0.2cm,nodesepA=-0.3cm,nodesepB=-0.5cm,linewidth=1pt]{D}{A}
				\begin{tabular}{|l|}
					\hline
					\multicolumn{1}{|c|}{\cellcolor{lightgray}{\small \textbf{Commande}}} \\
					\hline
					\cellcolor{white}\underline{\textbf {\footnotesize id}}               \\
					\hline
					\cellcolor{white}{\footnotesize prix}                                 \\
					\hline
					\cellcolor{white}{\footnotesize articles}                             \\
					\hline
					\cellcolor{white}{\footnotesize dates}                                \\
					\hline
				\end{tabular}
				\rnode{X}{} \quad \quad \rnode{Y}{}
				\ncline[doubleline=true,doublesep=2pt,doublecolor=blue,linecolor=blue,linewidth=1pt,arrowsize=10pt,arrowinset=0.2,arrowlength=1.2]{->}{X}{Y}
				\begin{tabular}{|l>{\footnotesize \sc}r|}
					\hline
					\multicolumn{2}{|c|}{\cellcolor{lightgray}{\small \textbf{Client}}}                             \\
					\hline
					\cellcolor{white}\underline{\textbf {\footnotesize \rnode{ACE}{num}}} & \cellcolor{white}{int}  \\
					\hline
					\cellcolor{white}{\footnotesize nom}                                  & \cellcolor{white}{text} \\
					\hline
					\cellcolor{white}{\footnotesize prenom}                               & \cellcolor{white}{text} \\
					\hline
					\cellcolor{white}{\footnotesize email}                                & \cellcolor{white}{text} \\
					\hline
				\end{tabular} \quad \quad
				\begin{tabular}{|l>{\footnotesize \sc}r|}
					\hline
					\multicolumn{2}{|c|}{\cellcolor{lightgray}{\small \textbf{Commande}}}                                     \\
					\hline
					\cellcolor{white}\underline{\textbf {\footnotesize id}}                        & \cellcolor{white}{int}   \\
					\hline
					\cellcolor{white}{\footnotesize \rnode{DCE}{\textcolor{Sepia}{\# id\_client}}} & \cellcolor{white}{int}   \\
					\hline
					\cellcolor{white}{\footnotesize prix}                                          & \cellcolor{white}{float} \\
					\hline
					\cellcolor{white}{\footnotesize articles}                                      & \cellcolor{white}{text}  \\
					\hline
					\cellcolor{white}{\footnotesize date}                                          & \cellcolor{white}{float} \\
					\hline
				\end{tabular}
				\ncangle[nodesepA=0.1cm,angleA=180,angleB=0,nodesepB=1.5cm]{*->}{DCE}{ACE}
			\end{center}}
		\onslide<4->{
			On obtient alors le schéma relationnel suivant : \smallskip
			\begin{mdframed}
				\textbf{Client} (\underline{num}, nom, prenom, email) \\
				\textbf{Commande} (\underline{id}, \# id\_client, prix, articles, date)
			\end{mdframed}}
	\end{block}
\end{frame}

\begin{frame}{\Ctitle}{\stitle}
	\begin{block}{Cas des associations \textit{many to many}}
		On crée trois tables : une pour chacune des entités et la table de liaison, celle-ci a pour clé primaire l'union des clés primaires des deux entités et est en liaison avec celles-ci en utilisant des clés étrangères.
		\onslide<2->{\begin{center}
				\renewcommand{\arraystretch}{1.1}
				\begin{tabular}{|p{2cm}|}
					\hline
					\multicolumn{1}{|c|}{\cellcolor{lightgray}{\small \textbf{Etudiant}}} \\
					\hline
					\cellcolor{white}\underline{\textbf {\footnotesize ine}}              \\
					\hline
					\cellcolor{white}{\footnotesize nom}                                  \\
					\hline
					\cellcolor{white}{\footnotesize prenom}                               \\
					\hline
					\cellcolor{white}{\footnotesize adresse}                              \\
					\hline
				\end{tabular} \rnode{D}{$1$} \quad  \rnode{E}{\phantom{1}$*$}
				\ncline[offset=-0.22cm,nodesepA=-0.3cm,nodesepB=-0.45cm,linewidth=1pt]{D}{E}
				\begin{tabular}{|p{2cm}|}
					\hline
					\multicolumn{1}{|c|}{\cellcolor{lightgray}{\small \textbf{Inscription}}} \\
					\hline
					\cellcolor{white}\underline{\textbf {\footnotesize Etudiant}}            \\
					\hline
					\cellcolor{white}\underline{\textbf {\footnotesize Cours}}               \\
					\hline
				\end{tabular}
				\rnode{F}{\phantom{1}$*$} \quad  \rnode{A}{$1$}
				\ncline[offset=-0.22cm,nodesepA=-0.45cm,nodesepB=-0.3cm,linewidth=1pt]{F}{A}
				\begin{tabular}{|p{2cm}|}
					\hline
					\multicolumn{1}{|c|}{\cellcolor{lightgray}{\small \textbf{Cours}}} \\
					\hline
					\cellcolor{white}\underline{\textbf {\footnotesize id}}            \\
					\hline
					\cellcolor{white}{\footnotesize matière}                           \\
					\hline
					\cellcolor{white}{\footnotesize durée}                             \\
					\hline
					\cellcolor{white}{\footnotesize description}                       \\
					\hline
				\end{tabular}
			\end{center} \vspace{0.4cm}}
		\onslide<3->{\begin{center}
				\begin{tabular}{|l>{\footnotesize \sc}r|}
					\hline
					\multicolumn{2}{|c|}{\cellcolor{lightgray}{\small \textbf{Etudiant}}}                          \\
					\hline
					\cellcolor{white}\underline{\textbf {\footnotesize \rnode{AE}{ine}}} & \cellcolor{white}{int}  \\
					\hline
					\cellcolor{white}{\footnotesize nom}                                 & \cellcolor{white}{text} \\
					\hline
					\cellcolor{white}{\footnotesize prenom}                              & \cellcolor{white}{text} \\
					\hline
					\cellcolor{white}{\footnotesize adresse}                             & \cellcolor{white}{text} \\
					\hline
				\end{tabular} \quad \quad \quad
				\begin{tabular}{|l>{\footnotesize \sc}r|}
					\hline
					\multicolumn{2}{|c|}{\cellcolor{lightgray}{\small \textbf{Inscription}}}                                                 \\
					\hline
					\cellcolor{white}\underline{\textbf {\footnotesize \textcolor{Sepia}{\rnode{DE}{\# Etudiant}}}} & \cellcolor{white}{int} \\
					\hline
					\cellcolor{white}\underline{\textbf {\footnotesize \textcolor{Sepia}{\rnode{DC}{\# Cours}}}}    & \cellcolor{white}{int} \\
					\hline
				\end{tabular} \quad \quad \quad
				\begin{tabular}{|l>{\footnotesize \sc}r|}
					\hline
					\multicolumn{2}{|c|}{\cellcolor{lightgray}{\small \textbf{Cours}}}                            \\
					\hline
					\cellcolor{white}\underline{\textbf {\footnotesize \rnode{AC}{id}}} & \cellcolor{white}{int}  \\
					\hline
					\cellcolor{white}{\footnotesize matière}                            & \cellcolor{white}{text} \\
					\hline
					\cellcolor{white}{\footnotesize durée}                              & \cellcolor{white}{int}  \\
					\hline
					\cellcolor{white}{\footnotesize description}                        & \cellcolor{white}{text} \\
					\hline
				\end{tabular}
				\ncangle[nodesepA=0.1cm,angleA=180,angleB=0,nodesepB=1.8cm]{*->}{DE}{AE}
				\ncangle[angleA=0,angleB=180,nodesepA=1.3cm,nodesepB=0.1cm]{*->}{DC}{AC}
			\end{center}}
	\end{block}
\end{frame}


\begin{frame}{\Ctitle}{\stitle}
	\begin{block}{Schéma relationnel}
		\begin{center}
			\begin{tabular}{|l>{\footnotesize \sc}r|}
				\hline
				\multicolumn{2}{|c|}{\cellcolor{lightgray}{\small \textbf{Etudiant}}}                          \\
				\hline
				\cellcolor{white}\underline{\textbf {\footnotesize \rnode{AE}{ine}}} & \cellcolor{white}{int}  \\
				\hline
				\cellcolor{white}{\footnotesize nom}                                 & \cellcolor{white}{text} \\
				\hline
				\cellcolor{white}{\footnotesize prenom}                              & \cellcolor{white}{text} \\
				\hline
				\cellcolor{white}{\footnotesize adresse}                             & \cellcolor{white}{text} \\
				\hline
			\end{tabular} \quad \quad \quad
			\begin{tabular}{|l>{\footnotesize \sc}r|}
				\hline
				\multicolumn{2}{|c|}{\cellcolor{lightgray}{\small \textbf{Inscription}}}                                                 \\
				\hline
				\cellcolor{white}\underline{\textbf {\footnotesize \textcolor{Sepia}{\rnode{DE}{\# Etudiant}}}} & \cellcolor{white}{int} \\
				\hline
				\cellcolor{white}\underline{\textbf {\footnotesize \textcolor{Sepia}{\rnode{DC}{\# Cours}}}}    & \cellcolor{white}{int} \\
				\hline
			\end{tabular} \quad \quad \quad
			\begin{tabular}{|l>{\footnotesize \sc}r|}
				\hline
				\multicolumn{2}{|c|}{\cellcolor{lightgray}{\small \textbf{Cours}}}                            \\
				\hline
				\cellcolor{white}\underline{\textbf {\footnotesize \rnode{AC}{id}}} & \cellcolor{white}{int}  \\
				\hline
				\cellcolor{white}{\footnotesize matière}                            & \cellcolor{white}{text} \\
				\hline
				\cellcolor{white}{\footnotesize durée}                              & \cellcolor{white}{int}  \\
				\hline
				\cellcolor{white}{\footnotesize description}                        & \cellcolor{white}{text} \\
				\hline
			\end{tabular}
			\ncangle[nodesepA=0.1cm,angleA=180,angleB=0,nodesepB=1.8cm]{*->}{DE}{AE}
			\ncangle[angleA=0,angleB=180,nodesepA=1.3cm,nodesepB=0.1cm]{*->}{DC}{AC}
		\end{center}
		\onslide<2->{
			On obtient alors le schéma relationnel suivant : \smallskip
			\begin{mdframed}
				\textbf{Etudiant} (\underline{ine}, nom, prenom, adresse) \\
				\textbf{Cours} (\underline{id}, matière, durée, description) \\
				\textbf{Inscription} (\underline{\# Etudiant}, \underline{\# cours})
			\end{mdframed}}
	\end{block}
\end{frame}

\begin{frame}{\Ctitle}{\stitle}
	\begin{exampleblock}{Exemple}
		\onslide<2->{On souhaite créer une base de données permettant de gérer les notes obtenus par des élèves dans des matières.
		\begin{itemize}
			\item[-]<3-> Les élèves ont les attributs suivants : nom, prénom, date de naissance, et identifiant unique.
			\item[-]<4-> Les matières ont les attributs suivants : nom (unique), horaire, coefficient
			\item[-]<5-> Chaque élève peut avoir plusieurs notes par matière.
		\end{itemize}
		\begin{itemize}
			\item<6-> Expliquer pourquoi un schéma relationnel d'une seule table notes n'est pas satisfaisant.
			\item<8-> Proposer un schéma relationnel constitué de 3 tables issu du modèle entité-association.
				}
		\end{itemize}
	\end{exampleblock}
\end{frame}


\end{document}  
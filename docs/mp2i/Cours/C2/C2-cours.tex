\PassOptionsToPackage{dvipsnames,table}{xcolor}
\documentclass[10pt,french]{beamer}
\usepackage{Cours}

\begin{document}

\input{\detokenize{/home/fenarius/Travail/Cours/cpge-info/latex//MacrosCours.tex}}

% Numéro et titre de chapitre
\setcounter{numchap}{2}
\newcommand{\Ctitle}{\cnum {Discipline de programmation}}
\newcommand{\SPATH}{/home/fenarius/Travail/Cours/cpge-info/docs/mp2i/files/C\thenumchap/}

\makess{Algorithme, programme, paradigme}
\begin{frame}{\Ctitle}{\stitle}
	\begin{block}{Définitions}
		\begin{itemize}
			\item<1-> Un \textcolor{blue}{algorithme} est une suite d'instructions et d'opérations permettant de résoudre un problème. \\
			\onslide<2->{\textcolor{gray}{Par exemple pour résoudre le problème du tri d'une liste de valeurs, on peut utiliser l'algorithme du tri par insertion. Cela consiste à prendre une à une chaque valeur et à l'insérer au bon emplacement dans une liste initialement vide.}}
			\item<3-> Un \textcolor{blue}{programme} est la traduction d'un algorithme dans un langage informatique. On dit qu'un programme est la mise en oeuvre d'un algorithme.\\
			\onslide<4->{\textcolor{gray}{L'algorithme du tri par insertion peut être écrit en Python, en C, \dots}}
			\item<5-> Un \textcolor{blue}{paradigme} de programmation est une façon de d'approcher un problème et d'en concevoir et  modéliser une solution.\\
			\onslide<6->{\textcolor{gray}{Le langage C est une illustration du paradigme de programmation impérative. Le langage OCaml nous permettra d'illustrer le paradigme fonctionnel}}
		\end{itemize}
	\end{block}
\end{frame}

\makess{Spécification}
\begin{frame}{\Ctitle}{\stitle}
	\begin{exampleblock}{Exemple introductif}
		On considère le programme C suivant :
		\inputpartC{/home/fenarius/Travail/Cours/cpge-info/docs/mp2i/files/C2/get_max.c}{}{\small}{3}{10}
	\begin{enumerate}
		\item<1-> Quel est le résultat renvoyé si le tableau fourni en argument contient dans cet ordre les valeurs : {\tt 12, 18, 11, 9, 10} ?
		\item<2-> Même question avec le tableau  {\tt 12, 18, 11, 18, 10}
		\item<3-> Même question avec le tableau {\tt -12, -15, -7}
		\item<4-> Même question si le tableau est vide (c'est-à-dire que {\tt size =0})
	\end{enumerate}
	\end{exampleblock}
\end{frame}

\begin{frame}{\Ctitle}{\stitle}
	\begin{exampleblock}{Correction}
		\begin{enumerate}
			\item<1-> \textcolor{OliveGreen} {La fonction renvoie 1 (indice de l'élément maximal du tableau)}
			\item<2-> \textcolor{OliveGreen}{la fonction renvoie 3, c'est l'indice de la dernière occurence du maximum des éléments du tablea }
			\item<3-> \textcolor{OliveGreen}{C'est un comportement indéfini, la variable {\tt imax} n'est pas initialisée.}
			\item<4-> \textcolor{OliveGreen}{Une nouvelle fois, le comportement est indéfini car on renvoie la variable {\tt imax} qui n'a jamais été initialisée.}
		\end{enumerate}
	\end{exampleblock}
	\begin{alertblock}{Définition}
		\onslide<5->{Ecrire la \textcolor{blue}{spécification} d'une fonction c'est donné une description formelle et détaillée de ses caractéristiques. En particulier :}
		\begin{itemize}
			\item<6-> les entrées admissibles : types et valeurs possibles des arguments (\textcolor{blue}{préconditions}), 
			\item<7-> ce que renvoie la fonction  et les  les effets de bords (\textit{side effects}) éventuels : modification des arguments, affichage, \dots . Ce sont les (\textcolor{blue}{postconditions}).
		\end{itemize}
	\end{alertblock}
\end{frame}

\begin{frame}{\Ctitle}{\stitle}
\begin{block}{Remarques}
\begin{itemize}
\item<1-> La spécification est souvent donnée en commentaire dans le code source. \\
\onslide<2-> \textcolor{gray}{Les commentaires en C s'écrivent entre \kw{/*} et \kw{*/} et en OCaml entre \kw{(*} et \kw{*)}}
\item<3-> La vérification des préconditions peut s'effectuer à l'aide d'instructions \kw{assert}. Elles sont de la forme \kw{assert (condition)} en C, et nécessitent d'importer \kw{assert.h}. Si la condition échoue le programme s'arrête et affiche une erreur, c'est ce qu'on appelle de la programmation défensive (anticipation des erreurs).
\end{itemize}
\end{block}
\begin{exampleblock}{Exemples}
\onslide<4->{Ecrire en C, une fonction qui :}
\begin{itemize}
	\item<5-> Accepte en argument un tablau \textit{non vide} d'entiers.
	\item<6-> Renvoie l'indice de la première occurence du maximum des éléments de ce tableau.
\end{itemize}
\end{exampleblock}
\end{frame}


\begin{frame}{\Ctitle}{\stitle}
	\begin{exampleblock}{Correction}
		\inputpartC{/home/fenarius/Travail/Cours/cpge-info/docs/mp2i/files/C2/get_max_ok.c}{}{\small}{2}{14}
	\end{exampleblock}
\end{frame}

\makess{Validation, test}
\begin{frame}{\Ctitle}{\stitle}
	\begin{block}{Jeu de tests}
		\onslide<1->Le comportement correct d'une fonction peut être "validé" (mais pas prouvé), par l'utilisation d'un \textcolor{blue}{jeu de test}. c'est-à-dire un ensemble de couple d'entrées du programme et de sorties attendues. \\
		\onslide<2->\textcolor{gray}{Dans la fonction précédente, on pourrait tester (entre autres) des cas limites (\textit{edge cases}), comme par exemple un tableau à un seul élément, ou vide ou des situations ou le maximum se trouve en première ou dernière position du tableau.}
	\end{block}
	\begin{exampleblock}{Exemple}
		\begin{enumerate}
		\item<3-> Ecrire une fonction {\tt contient\_double} qui prend en argument un tableau d'entiers et renvoie {\tt true} si ce tableau contient deux éléments consécutifs égaux.
		\item<4-> Proposer un jeu de tests pour cette fonction.
		\end{enumerate}
	\end{exampleblock}
\end{frame}


\begin{frame}{\Ctitle}{\stitle}
	\begin{exampleblock}{Correction}
		\begin{enumerate}
		\item<1-> \ \\ \inputpartC{/home/fenarius/Travail/Cours/cpge-info/docs/mp2i/files/C2/contient_double.c}{}{\small}{4}{8}
		\item<2-> On pourrait tester les cas suivants :
		\begin{itemize}
			\item<3-> Tableau vide ou à un seul élément
			\item<4-> Même élément mais non consécutifs
			\item<5-> Elément présent en plus de deux exemplaires consécutifs
			\item<6-> Présence du double en tout début ou toute fin de tableau
		\end{itemize}
		\end{enumerate}
	\end{exampleblock}
\end{frame}

\makess{Graphe de flot de contôle}
\begin{frame}{\Ctitle}{\stitle}
	\begin{exampleblock}{Exemple introductif}
		Ecrire la fonction {\tt duree\_vol} qui prend en argument un entier positif {\tt n} et renvoie le nombre d'itérations nécessaires avant d'atteindre 1 en prenant cet entier comme valeur initiale dans la suite de syracuse. On rappelle que : \\
		$s_{n+1} = \left\{ \begin{array}{ll} \dfrac{s_n}{2} & \mathrm{\ si\ } n \mathrm{\ est \ paire} \\ 3s_n+1 & \mathrm{\ sinon} \end{array}\right.$
		\onslide<2->\inputpartC{/home/fenarius/Travail/Cours/cpge-info/docs/mp2i/files/C2/syracuse.c}{}{\small}{3}{11}
	\end{exampleblock}
\end{frame}



\begin{frame}{\Ctitle}{\stitle}
	\begin{exampleblock}{Représentation des exécutions possibles}
		\vspace{6.2cm}
		\rput(5,6){\circlenode[linecolor=red]{E}{\textcolor{red}{E}}}
		\rput(5,5){\rnode{I}{\psframebox{\tt dvol=0}}}
		\rput(5,4){\ovalnode[linecolor=blue]{W}{\kw{n==1}}}
		\rput(7,4){\circlenode[linecolor=red]{S}{\textcolor{red}{S}}}
		\ncline{->}{W}{S} \naput[labelsep=1pt]{\small \textcolor{OliveGreen}{V}}
		\ncline{->}{E}{I}
		\ncline{->}{I}{W}
		\rput(5,3){\ovalnode[linecolor=blue]{T}{\kw {n\%2==0}}}
		\rput(5,1){\rnode{D}{\psframebox{\tt dvol++}}}
		\ncline{->}{W}{T} \naput[labelsep=1pt]{\small \textcolor{OrangeRed}{F}}
		\rput(3,2){\rnode{PA}{\psframebox{\tt n=n/2}}}
		\rput(7,2){\rnode{IM}{\psframebox{\tt n=3n+1}}}
		\ncline{->}{T}{PA} \naput[labelsep=1pt]{\small \textcolor{OliveGreen}{V}}
		\ncline{->}{T}{IM} \nbput[labelsep=1pt]{\small \textcolor{OrangeRed}{F}}
		\ncline{->}{PA}{D}
		\ncline{<-}{D}{IM}
		\ncbar[angle=180,arm=2.5cm]{->}{D}{W}
	\end{exampleblock}
\end{frame}

\begin{frame}{\Ctitle}{\stitle}
	\begin{block}{Définitions}
		Le \textcolor{blue}{graphe de flot de contrôle} représente les exécutions possibles d'un programme.
		\begin{itemize}
			\item<1-> Les sommets \kw{E} et \kw{S} représentent l'entrée et la sortie
			\item<2-> Les autres noeuds sont les blocs d'instructions
			\item<3-> On dessine un arc entre deux noeuds {\tt A} et {\tt B} lorsque l'exécution de {\tt B} peut suivre celle de {\tt A}
			\item<4-> Les noeuds quittant les instructions conditionnelle sont étiquetés par {\tt V} ou {\tt F}.
			\item<5-> On dira qu'un chemin dans le graphe est \textcolor{blue}{faisable} lorsqu'il existe des valeurs d'entrées pour lesquels l'exécution passe par tous les noeuds de ce chemin.
		\end{itemize}
	\end{block}
	\begin{alertblock}{Définition}
		\onslide<6->{On dit qu'un jeu de test couvre tous les sommets (resp. tous les arcs) lorsque son exécution permet de passer par tous les noeuds (resp. tous les arcs)}
	\end{alertblock}
\end{frame}

\begin{frame}{\Ctitle}{\stitle}
	\begin{exampleblock}{Exemple}
		\begin{enumerate}
			\item<1-> Montrer que Le jeu de test \kw{\{n=1, n=8\}} ne couvre ni tous les arcs, ni tous les sommets. \\
			\onslide<2->\textcolor{OliveGreen}{On ne passe jamais par le noeud représentant l'instruction {\tt n=3*n+1}, en effet pour {\tt n=1} on atteint directement la sortie {\tt S} et pour {\tt n=8}, tous les valeurs de la suite sont paires.}
			\item<3-> Proposer un jeu de test permettant de couvrir tous les arcs. \\
			\onslide<4->\textcolor{OliveGreen}{On peut ajouter par exemple le test \kw{n=5}}
		\end{enumerate}
	\end{exampleblock}
	\begin{exampleblock}{Exercice}
		\begin{enumerate}
			\item<5-> Ecrire une fonction prenant en argument un flottant {\tt x} et un entier positif ou nul {\tt n} et qui renvoie {\tt x} puissance {\tt n}.
			\item<6-> Tracer son graphe de flot de contrôle.
			\item<7-> Proposer un jeu de test.
		\end{enumerate}
	\end{exampleblock}
\end{frame}



\begin{frame}{\Ctitle}{\stitle}
	\begin{exampleblock}{Correction} 
		\begin{tabularx}{\textwidth}{XX}
		\begin{minipage}{0.6\textwidth}
			\vspace{0.5cm}
			\inputpartC{/home/fenarius/Travail/Cours/cpge-info/docs/mp2i/files/C2/puissance.c}{}{\small}{5}{10}
			\vspace{0.8cm}
		\end{minipage} &
	\end{tabularx}
	\end{exampleblock}
	\end{frame}


	\begin{frame}{\Ctitle}{\stitle}
		\begin{exampleblock}{Correction} 
			\begin{tabularx}{\textwidth}{XX}
			\begin{minipage}{0.6\textwidth}
				\vspace{0.5cm}
				\inputpartC{/home/fenarius/Travail/Cours/cpge-info/docs/mp2i/files/C2/puissance.c}{}{\small}{5}{10}
				\vspace{0.8cm}
			\end{minipage} &
			\rput(3,1.8){\circlenode[linecolor=red]{E}{\textcolor{red}{E}}}
			\rput(3,0.8){\rnode{I}{\psframebox{\begin{tabular}{>{\tt}l}  xn=1.0 \\ i=1 \end{tabular}}}}
			\rput(3,-0.2){\ovalnode[linecolor=blue]{W}{\kw{i<=n}}}
			\rput(5,-0.2){\circlenode[linecolor=red]{S}{\textcolor{red}{S}}}
			\rput(3,-1.4){\rnode{D}{\psframebox{\begin{tabular}{>{\tt}l}  xn=xn*x \\ i=i+1 \end{tabular}}}}
			\ncline{->}{E}{I}
			\ncline{->}{I}{W}
			\ncline{->}{W}{S} \naput[labelsep=1pt]{\small \textcolor{OrangeRed}{F}}
			\ncline{->}{W}{D} \naput[labelsep=1pt]{\small \textcolor{OliveGreen}{V}}
			\ncbar[angle=180]{->}{D}{W}
		\end{tabularx}
		\end{exampleblock}
		\end{frame}

\begin{frame}{\Ctitle}{\stitle}
\begin{exampleblock}{Correction} 
	\begin{tabularx}{\textwidth}{XX}
	\begin{minipage}{0.6\textwidth}
		\vspace{0.5cm}
		\inputpartC{/home/fenarius/Travail/Cours/cpge-info/docs/mp2i/files/C2/puissance.c}{}{\small}{5}{10}
		\vspace{0.8cm}
	\end{minipage} &
	\rput(3,1.8){\circlenode[linecolor=red]{E}{\textcolor{red}{E}}}
	\rput(3,0.8){\rnode{I}{\psframebox{\begin{tabular}{>{\tt}l}  xn=1.0 \\ i=1 \end{tabular}}}}
	\rput(3,-0.2){\ovalnode[linecolor=blue]{W}{\kw{i<=n}}}
	\rput(5,-0.2){\circlenode[linecolor=red]{S}{\textcolor{red}{S}}}
	\rput(3,-1.4){\rnode{D}{\psframebox{\begin{tabular}{>{\tt}l}  xn=xn*x \\ i=i+1 \end{tabular}}}}
	\ncline{->}{E}{I}
	\ncline{->}{I}{W}
	\ncline{->}{W}{S} \naput[labelsep=1pt]{\small \textcolor{OrangeRed}{F}}
	\ncline{->}{W}{D} \naput[labelsep=1pt]{\small \textcolor{OliveGreen}{V}}
	\ncbar[angle=180]{->}{D}{W}
\end{tabularx}
On peut proposer le jeu de test suivant :
\begin{itemize}
  \item<1-> \kw{x=2.0, n=0}
  \item<2-> \kw{x=2.0, n=3}
\end{itemize}
\end{exampleblock}
\end{frame}

\makess{Test exhaustif d'une conditionnelle}
\begin{frame}{\Ctitle}{\stitle}
	\begin{block}{Remarque}
		Lorsqu'une instruction conditionnelle comporte des conjonctions ou des disjonctions, on peut formuler des tests qui prévoient toutes les possibilités de la satisfaire.
	\end{block} 
	\begin{exampleblock}{Exemple}
		Une année est bissextile si elle est divisible par 4 mais pas par 100 ou s'il est divisible par 400.
		\begin{enumerate}
			\item<1-> Ecrire une fonction bissextile qui prend en argument un entier positif {\tt annee} et renvoie \kw{true} si l'année est bissextile et \kw{false} sinon.
			\item<2-> Proposer un jeu de test pour cette fonction qui prévoit toutes les possibilités de satisfaire la condition d'être bissextile.
		\end{enumerate}
	\end{exampleblock}
\end{frame}

\begin{frame}[fragile]{\Ctitle}{\stitle}
\begin{exampleblock}{Correction}
	\begin{enumerate}
	\item<1-> \ \\
	\begin{langageC}
bool bissextile(int annee) {
	return ((annee%4==0 && annee%100!=0) || (annee%400==0))
}
	\end{langageC}
	\item<2-> On peut proposer les tests suivants :
	\begin{itemize}
		\item<3-> \kw{n=2004} qui permet de valider la première condition
		\item<4-> \kw{n=2000} qui permet de valider la deuxième
	\end{itemize}
\end{enumerate}
\end{exampleblock}
\end{frame}

\makess{Terminaison d'un algorithme}
\begin{frame}[fragile]{\Ctitle}{\stitle}
	\begin{block}{Exemple introductif}
	\SetAlFnt{\small}
	\setlength{\algomargin}{8pt}
	\begin{algorithm}[H]
		\DontPrintSemicolon
		\caption{Multiplier sans utiliser {\tt *}}
		\Entree{$n \in \N, m \in \N$}
		\Sortie{$nm$}
		\everypar={\footnotesize \textcolor{gray}{\nl}}
		$r \leftarrow 0$\;
		\Tq{$m >0$}{
		$m \leftarrow m-1$ \;
		$r \leftarrow r+n$ \;
		}
		\Return $r$
	  \end{algorithm}
	  \begin{enumerate}
	  \item<2-> Donner une implémentation en C de cet algorithme.
	  \item<3-> Montrer que dans la boucle "tant que":
	  \begin{itemize}
		\item<3-> $m$ ne prend que des valeurs entières
		\item<4-> $m$ prend des valeurs \textit{strictement} positives
		\item<5-> les valeurs prises par $m$ sont \textit{strictement} décroissantes.
	  \end{itemize}
	  \end{enumerate}
	\end{block}
\end{frame}

\begin{frame}[fragile]{\Ctitle}{\stitle}
	\begin{block}{Définitions}
    \begin{itemize}
        \item<2-> On dit qu'un algorithme \textcolor{blue}{termine} lorsqu'il renvoie un résultat en un nombre fini d'étapes quels que soient les valeurs des entrées.
        \item<3-> Un \textcolor{blue}{variant de boucle} est une quantité :
            \begin{enumerate}
            \item<4-> à valeurs entières,
            \item<5-> \textit{strictement} positives,
            \item<6-> qui décroît \textit{strictement} à chaque passage dans la boucle.
            \end{enumerate}
		\end{itemize}
	\end{block}
	\onslide<7->
	{\begin{alertblock}{\textcolor{yellow}{\important \;} Propriété}
		Si une boucle admet un variant, alors cette boucle termine.
	\end{alertblock}}
	\onslide<8->
	{
		\begin{exampleblock}{Exemple}
			$m$ est un variant de boucle de l'algorithme de multiplication ci-dessus et donc cet algorithme termine.
		\end{exampleblock}
	}
\end{frame}

\begin{frame}[fragile]{\Ctitle}{\stitle}
	\begin{exampleblock}{Exemple}
	\SetAlFnt{\small}
	\setlength{\algomargin}{8pt}
	\begin{algorithm}[H]
		\DontPrintSemicolon
		\caption{Nombre de chiffres en base 10}
		\Entree{$n \in \N$}
		\Sortie{$p$ nombre de chiffres de $n$ dans son écriture en base 10.}
		\everypar={\footnotesize \textcolor{gray}{\nl}}
		\Si{$n=0$}{\Return 1}
		$p \leftarrow 0$\;
		\Tq{$n >0$}{
		$p \leftarrow p+1$ \;
		$n \leftarrow \lfloor \frac{n}{10} \rfloor $ \;
		}
		\Return $p$
	  \end{algorithm}
	  \begin{enumerate}
	  \item<2-> Donner une implémentation en C de cet algorithme.
	  \item<3-> Prouver la terminaison de cet algorithme.
	  \end{enumerate}
	\end{exampleblock}
\end{frame}

\begin{frame}{\Ctitle}{\stitle}
	\begin{exampleblock}{Implémentation en C}
		\inputpartC{\SPATH/nbchiffres.c}{}{\small}{4}{18}
	\end{exampleblock}
\end{frame}

\begin{frame}{\Ctitle}{\stitle}
	\begin{exampleblock}{Preuve de terminaison}
		Montrons que $n$ est un variant de la boucle {\tt tant que} de l'algorithme :
		\begin{enumerate}
		\item<1-> $n$ ne prend que des valeurs entières, en effet, $n \in \N$  en entrée et $\PE{\dfrac{n}{10}}$ est entier.
		\item<2-> $n$ est positif strictement par condition d'entrée dans la boucle. 
		\item<3-> $n$ décroît strictement à chaque passage dans la boucle car comme $n > 0$, $\PE{\dfrac{n}{10}} < n$
		\end{enumerate}
	\end{exampleblock}
\end{frame}

\begin{frame}{\Ctitle}{\stitle}
	\begin{exampleblock}{Exercice}
		\begin{enumerate}
		\item<1-> Ecrire un algorithme qui prend en entrée un entier $n>1$ et renvoie le premier diviseur strictement supérieur à 1 de cet entier. Par exemple pour $n=7$ l'algorithme renvoie 7 et pour $n=15$, l'algorithme renvoie $3$.
		\item<2-> Donner une implémentation en C de cet algorithme sous la forme d'une fonction dont on précisera soigneusement la spécification.
		\item<3-> Prouver la terminaison de cet algorithme.
		\end{enumerate}
	\end{exampleblock}
\end{frame}

\makess{Correction d'un algorithme}
\begin{frame}[fragile]{\Ctitle}{\stitle}
	\begin{block}{Exemple introductif}
	\SetAlFnt{\small}
	\setlength{\algomargin}{8pt}
	\begin{algorithm}[H]
		\DontPrintSemicolon
		\caption{Multiplier sans utiliser {\tt *}}
		\Entree{$n \in \N, m \in \N$}
		\Sortie{$nm$}
		\everypar={\footnotesize \textcolor{gray}{\nl}}
		$r \leftarrow 0$\;
		\Tq{$m >0$}{
		$m \leftarrow m-1$ \;
		$r \leftarrow r+n$ \;
		}
		\Return $r$
	  \end{algorithm}
	  On note $k$ le nombre de passage dans la boucle {\tt tant que} et considère la propriété suivante suivante noté $\mathcal{P}(k)$ : \og{} après $k$ tour de boucles, $r = kn$ \fg{}. Montrer que cette propriété est vraie :
	  \begin{enumerate}
	  \item<2-> avant d'entrée dans la boucle
	  \item<3-> qu'elle reste vraie à chaque tour de boucle
	  \end{enumerate}
	  \onslide<4->{Que peut-on en conclure ?}
	\end{block}
\end{frame}

\begin{frame}[fragile]{\Ctitle}{\stitle}
	\begin{block}{Définitions}
    \begin{itemize}
        \item<1-> Un \textcolor{blue}{invariant de boucle} est une propriété qui :
            \begin{itemize}
            \item<2-> est vraie à l'entrée dans la boucle (\textcolor{blue}{initialisation}),
            \item<3-> reste vraie à chaque itération si elle l'était à l'itération précédente (\textcolor{blue}{conservation}).
            \end{itemize}
		\item<4-> Un algorithme est dit \textcolor{blue}{partiellement correct} lorsqu'il renvoie la réponse attendue quand il se termine.
		\item<5-> Un algorithme est dit \textcolor{blue}{totalement correct} lorsqu'il est partiellement correcte et que sa terminaison est prouvée.
		\end{itemize}
	\end{block}
	\onslide<7->
	{\begin{alertblock}{\textcolor{yellow}{\important \;} Prouver la correction d'un algorithme}
		On utilise un invariant de boucle qui en sortie de boucle fournit une propriété permettant de montrer la correction.\\
		\textcolor{gray}{La méthode est similaire à une récurrence mathématique.}
	\end{alertblock}}
\end{frame}


\begin{frame}[fragile]{\Ctitle}{\stitle}
	\begin{exampleblock}{Exemple}
	\SetAlFnt{\small}
	\setlength{\algomargin}{8pt}
	\begin{algorithm}[H]
		\DontPrintSemicolon
		\caption{Nombre de chiffres en base 10}
		\Entree{$n \in \N$}
		\Sortie{$p$ nombre de chiffres de $n$ dans son écriture en base 10.}
		\everypar={\footnotesize \textcolor{gray}{\nl}}
		\Si{$n=0$}{\Return 1}
		$p \leftarrow 0$\;
		\Tq{$n >0$}{
		$p \leftarrow p+1$ \;
		$n \leftarrow \lfloor \frac{n}{10} \rfloor $ \;
		}
		\Return $p$
	  \end{algorithm}
	  Prouver que cet algorithme est correct.
	\end{exampleblock}
\end{frame}

\begin{frame}[fragile]{\Ctitle}{\stitle}
\begin{exampleblock}{Correction}
	\textcolor{gray}{Idée : à chaque tour de boucle, on incrémente $p$ et $n$ perd un chiffre. La somme de $p$ et du nombre de chiffres de $n$ est donc constante. C'est l'invariant de boucle ! \\ \medskip}
	\onslide<2->{L'algorithme est correct pour $n=0$, montrons qu'il l'est aussi pour $n>0$, on note $n_k$ (resp. $p_k$) la valeur de $n$ (resp $p$) après $k$ tour de boucles, $c(m)$ le nombre de chiffre d'un entiers $m>0$ et on pose $c(0)=0$. On considère la propriété $P(k)$ : \og{} $c(n) = p_k + c(n_k)$ \fg{}.}
	\begin{itemize}
	\item<3-> initialisation : $P(0)$ est vraie puisque $p_0=0$ et $n_0=n$.
	\item<4-> conservation : Soit $k \in \N$ tel que $P(k)$ est vraie, au tour de boucle suivant : \\
	\onslide<5->{$p_{k+1} + c(n_{k+1}) = p_k + 1 + c\left(\PE{\dfrac{n_k}{10}}\right)$ \\}
	\onslide<6->{$p_{k+1} + c(n_{k+1}) = p_k + 1 + c\left(n_k\right) - 1$ \\}
	\onslide<7->{$p_{k+1} + c(n_{k+1}) = c(n)$, par $P(k)$ est vraie.}
	\end{itemize}
	\onslide<8->{En sortie de boucle, on a  $n_k=0$ et donc $p = c(n)$ et l'algorithme est correct.}
\end{exampleblock}
\end{frame}

\begin{frame}[fragile]{\Ctitle}{\stitle}
	\begin{exampleblock}{Exemple}
	\SetAlFnt{\small}
	\setlength{\algomargin}{8pt}
	\begin{algorithm}[H]
		\DontPrintSemicolon
		\caption{Premier diviseur}
		\Entree{$n \in \N, n > 1$}
		\Sortie{$d$ premier diviseur de $n$ strictement supérieur à $1$.}
		\everypar={\footnotesize \textcolor{gray}{\nl}}
		$d \leftarrow 2$\;
		\Tq{$n \mod d \neq 0$}{
		$d \leftarrow d+1$ \;
		}
		\Return $d$
	  \end{algorithm}
	  Prouver que cet algorithme est correct.
	\end{exampleblock}
\end{frame}

\end{document}


\PassOptionsToPackage{dvipsnames,table}{xcolor}
\documentclass[10pt]{beamer}
\usepackage{Cours}

\begin{document}

\input{\detokenize{/home/fenarius/Travail/Cours/cpge-info/latex/MacrosCours.tex}}

% Numéro et titre de chapitre
\setcounter{numchap}{9}
\newcommand{\Ctitle}{\cnum {OCaml : aspects fonctiimpératifsonnels}}
\newcommand{\SPATH}{/home/fenarius/Travail/Cours/cpge-info/docs/mp2i/files/C\thenumchap/}

% Définition d'une structure de données
\makess{Variables mutables}
\begin{frame}{\Ctitle}{\stitle}
	\begin{block}{Introduction}
        \begin{itemize}
            \item<1-> Pour le moment, nous nous sommes limités aux aspects fonctionnels d'OCaml et donc aux variables et aux structures de données \textcolor{blue}{non mutables}.
            \item<2-> Cependant, OCaml est un langage de programmation multi-paradigme et la programmation impérative (et donc les variables mutables) peuvent être manipulées en OCaml.
        \end{itemize}
    \end{block}
    \begin{block}{Champ mutable d'un enregistrement}
        On peut déclarer en OCaml un enregistrement ayant des champs mutables grâce au mot-clé \mintinline{ocaml}{mutable}. Par exemple,
        \inputpartOCaml{\SPATH mutable.ml}{}{}{1}{1}
        \textcolor{BrickRed}{\small \danger} Pour modifier la valeur du champ on utilise \textcolor{blue}{\tt <-}. Le symbole {\tt =} est réservé à la comparaison.
    \end{block}
\end{frame}

\begin{frame}{\Ctitle}{\stitle}
    \begin{block}{Références}
        
    \end{block}
\end{frame}





\end{document}
\PassOptionsToPackage{dvipsnames,table}{xcolor}
\documentclass[10pt]{beamer}
\usepackage{Cours}

\begin{document}

\input{\detokenize{/home/fenarius/Travail/Cours/cpge-info/latex/MacrosCours.tex}}

% Numéro et titre de chapitre
\setcounter{numchap}{9}
\newcommand{\Ctitle}{\cnum {OCaml : aspects impératifs}}
\newcommand{\SPATH}{/home/fenarius/Travail/Cours/cpge-info/docs/mp2i/files/C\thenumchap/}

% Définition d'une structure de données
\makess{Variables mutables}
\begin{frame}{\Ctitle}{\stitle}
	\begin{block}{Introduction}
        \begin{itemize}
            \item<1-> Pour le moment, nous nous sommes limités aux aspects fonctionnels d'OCaml et donc aux variables et aux structures de données \textcolor{blue}{non mutables}.
            \item<2-> Cependant, OCaml est un langage de programmation multi-paradigme et la programmation impérative (et donc les variables mutables) peuvent être manipulées en OCaml.
        \end{itemize}
    \end{block}
    
    \begin{block}{Champ mutable d'un enregistrement}
        \onslide<3->{On peut déclarer en OCaml un enregistrement ayant des champs mutables grâce au mot-clé \mintinline{ocaml}{mutable}.}
        \onslide<4->{ Par exemple,
        \inputpartOCaml{\SPATH mutable.ml}{}{}{1}{1}}
        \onslide<5->{\textcolor{BrickRed}{\small \danger} Pour modifier la valeur du champ on utilise \textcolor{blue}{\tt <-}. Le symbole {\tt =} est réservé à la comparaison.}
    \end{block}
\end{frame}

\begin{frame}{\Ctitle}{\stitle}
    \begin{exampleblock}{Exemple}
        \inputpartOCaml{mutable.ml}{\small}{}{1}{12}
    \end{exampleblock}
\end{frame}

\begin{frame}{\Ctitle}{\stitle}
    \begin{block}{Les références}
        \begin{itemize}
            \item<1-> Le type \mintinline{ocaml}{ref}  est prédéfini dans OCaml et correspond exactement à ce que nous venons de faire (sauf que le champ mutable s'appelle {\tt contents})\\
            \onslide<2-> Par exemple \mintinline{ocaml}{let a = {contents = 5};} crée une variable ayant son champ mutable {\tt contents} qui vaut 5.
            \item<3-> Afin d'alléger la syntaxe, on peut écrire directement \mintinline{ocaml}{let a = ref 5;;}. \\
            \onslide<4->\textcolor{gray}{On retiendra que l'effet reste le même : on a crée une variable {\tt a} ayant un champ mutable entier qui contient 5.}
            \item<5-> Si on veut accéder à la valeur du champ mutable, une syntaxe "adoucie" est aussi disponible avec \textcolor{BrickRed}{\tt !}. On écrira par exemple  \mintinline{ocaml}{print_int !a} pour afficher le contenu du champ mutable de {\tt a}. \\
            \onslide<6->\textcolor{gray}{On retiendra que c'est identique à \mintinline{ocaml}{print_int a.contents}}
            \item<7-> Pour modifier la valeur de {\tt a}, une syntaxe plus simple  est aussi disponible avec \textcolor{BrickRed}{\tt :=}. On écrira par exemple \mintinline{ocaml}{a := 5}\\
            \onslide<8-> Par exemple, \mintinline{ocaml}{ a:= !a +1} permet d'incrémenter de 1 la valeur (du champ mutable) de {\tt a}.
        \end{itemize}
    \end{block}
\end{frame}


\begin{frame}{\Ctitle}{\stitle}
    \begin{exampleblock}{Exemple}
        Créer une reférence {\tt a} vers 42 et une référence {\tt b} vers 2023. Echanger le contenu de ces deux variables en utilisant une troisième référence {\tt temp}
        \onslide<2->
        \inputpartOCaml{echange.ml}{}{}{1}{5}
    \end{exampleblock}
\end{frame}

\makess{Boucles}
\begin{frame}[fragile]{\Ctitle}{\stitle}
	\begin{block}{Boucle  {\tt for}}
    Il y a deux versions :
    \begin{itemize}
    \item<1-> Indice croissant :
    \begin{OCaml*}{fontsize=\small}
for indice = debut to fin do
    expression
done
    \end{OCaml*}
    \item<2-> Indice décroissant :
    \begin{OCaml*}{fontsize=\small}
for indice = debut downto fin do
    expression
done
    \end{OCaml*}
    \end{itemize}
    \onslide<3->{\textcolor{BrickRed}{\small \danger\;} Attention :}
    \begin{itemize}
        \item<4-> {\tt expression} est de type \kw{unit} (ex : affectation, affichage, ...)
        \item<5-> Les \textcolor{blue}{deux} bornes de la boucle sont incluses
        \item<6-> Pas de \kw{break} ou de \kw{continue} et {\tt indice} ne peut pas être modifié dans la boucle
    \end{itemize} 
    \end{block}
\end{frame}

\begin{frame}[fragile]{\Ctitle}{\stitle}
	\begin{exampleblock}{Exemple de boucle {\tt for}}
        \begin{itemize}
        \item<1->Ecrire une fonction {\tt harmonique\_asc} qui prend en argument un entier {\tt n} et calcule la somme des inverses des entiers non nuls jusqu'à {\tt n} de façon ascendante.
        \item<2-> Même question de façon descendante avec la fonction {\tt harmonique\_desc}
        \item<3-> Que pensez de l'écart observé entre les deux valeurs pour {\tt n} assez grand ?
        \end{itemize}
    \end{exampleblock}
\end{frame}

\begin{frame}[fragile]{\Ctitle}{\stitle}
	\begin{block}{Boucle  {\tt while}}
    \begin{OCaml*}{fontsize=\small}
while condition do
    expression
done
    \end{OCaml*}
    \onslide<3->{\textcolor{BrickRed}{\small \danger\;} Attention :}
    \begin{itemize}
        \item<4-> {\tt expression} est de type \kw{unit} (ex : affectation, affichage, ...)
        \item<5-> {\tt condition} est de type {\tt bool}
        \item<6-> Pas de \kw{break}
    \end{itemize} 
    \end{block}
\end{frame}

\begin{frame}{\Ctitle}{\stitle}
    \begin{exampleblock}{Exemple}
        \begin{itemize}
            \item<1-> Ecrire une fonction permettant de calculer la somme des n premiers entiers à l'aide d'une boucle {\tt while}. 
            \item<2-> Ecrire une fonction {\tt log2} qui prend en argument un entier $n$ et renvoie l'entier $k$ tel que $2^{k-1}< n \leq 2^k$
        \end{itemize}
    \end{exampleblock}
\end{frame}


\makess{Tableaux}
\begin{frame}[fragile]{\Ctitle}{\stitle}
    \begin{block}{Syntaxe}
    Les tableaux en OCaml, sont identiques à ceux vus en C. C'est à dire qu'il s'agit d'une structure mutable dont les éléments sont rangés de façon contigue en mémoire (et donc avec un accès en $O(1)$).
    \begin{itemize}
        \item<1-> Les tableaux sont notés entre \kw{[|} et \kw{|]}\\
        \onslide<2->{\mintinline{ocaml}{let ex = [|2; 7; 9; 14 |]}}
        \item<3-> L'accès à un élément se fait avec la notation \kw{tab.(i)}\\
        \onslide<4->{\mintinline{ocaml}{print_int ex.(1)} affiche 7}
        \item<5-> La structure est mutable, attention l'affectation se fait avec \kw{<-} (sans {\tt let})\\
        \onslide<6->{\mintinline{ocaml}{ex.(1)<-13} le continu du tableau est maintenant \mintinline{ocaml}{[|2; 13; 9; 14 |]}}
    \end{itemize}
\end{block}
\end{frame}


\begin{frame}[fragile]{\Ctitle}{\stitle}
    \begin{block}{Fonctions de {\tt Array}}
    \begin{itemize}
        \item<1-> \mintinline{ocaml}{Array.make} qui prend en argument un entier et une valeur et crée le tableau contenant n fois cette valeur
        \item<2-> \mintinline{ocaml}{Array.length} donne le nombre d'élément du tableau (en $O(1)$)
        \item<3-> \mintinline{ocaml}{Array.of_list} \mintinline{ocaml}{Array.to_list} permet de convertir depuis ou vers une liste.
        \item<4-> \mintinline{ocaml}{Array.map}, \mintinline{ocaml}{Array.iter}, \mintinline{ocaml}{Array.fold_left} (ou \mintinline{ocaml}{Array.fold_right}) sont les équivalents sur les tableaux des fonctions de même non sur les listes.
    \end{itemize}
\end{block}
\onslide<5->{
    \begin{block}{Aliasing}
        \textcolor{BrickRed}{\small \danger \;} Quand \kw{tab} est un tableau, alors \mintinline{ocaml}{let new_tab = tab} fait pointer \kw{new\_tab} vers la \textit{même} zone mémoire que \kw{tab} et donc toute modification de l'un se répercute sur l'autre ! 
        Faire particulièrement attention lors de la création de matrices (donc de tableaux de tableaux).
    \end{block}}
\end{frame}

\begin{frame}[fragile]{\Ctitle}{\stitle}
    \begin{exampleblock}{Exemples}
    \begin{itemize}
        \item<1-> Crée le tableau contenant les doubles des 100 premiers entiers.
        \item<2-> Ecrire la fonction affiche permettant d'afficher les éléments d'un tableau d'entier. On utilisera une boucle \kw{for} et \mintinline{ocaml}{Array.length}.
        \item<3-> Ecrire une fonction permettant de trouver le minimum des éléments d'un tableau non vide. 
    \end{itemize}
    \end{exampleblock}
\end{frame}

\makess{Entrées-Sorties}
\begin{frame}[fragile]{\Ctitle}{\stitle}
    \begin{block}{Fonction d'affichage}
    \begin{itemize}
        \item<1-> Les fonctions permettant d'afficher les valeurs de type de base \kw{print\_int}, \kw{print\_float}, \dots
        \item<2-> La fonction \kw{printf} du module {\tt Printf} permettant d'afficher, à la façon du C, des données. Les spécificateurs de format sont :
        \begin{itemize}
            \item<3-> \kw{\%d} : un entier
            \item<4-> \kw{\%s} : une chaine de caractères
            \item<5-> \kw{\%s} : un flottant
            \item<6-> \kw{\%b} : un booléen
            \item<7-> \kw{\%c} : un caractère
        \end{itemize}
        Par exemple : \\ \mintinline{ocaml}{Printf.printf "Date de naissance : le %d %s %d" jour mois annee}
    \end{itemize}
\end{block}
\end{frame}


\begin{frame}[fragile]{\Ctitle}{\stitle}
    \begin{block}{Fonction de saisie}
        Plusieurs fonctions permettent de récupérer des informations entrées au clavier :
    \begin{itemize}
        \item<1-> La fonction \kw{read\_line} de type {\tt unit->string} attend la saisie d'une chaine de caractère au clavier (suivie d'un retour chariot) puis renvoie cette chaine (sans le retour chariot final)
        \item<2-> Des fonctions similaires existent qui tentent directement la conversion dans le type attendu comme \kw{read\_int} pour lire un entier. Une exception est levée si cette conversion échoue.
    \end{itemize}
\end{block}
\onslide<3->{
\begin{exampleblock}{Exemple}
    \inputpartOCaml{hello.ml}{}{\small}{1}{5}
\end{exampleblock}}
\end{frame}

\begin{frame}[fragile]{\Ctitle}{\stitle}
    \begin{block}{Lecture et écriture dans un fichier}
        Les accès vers les fichiers sont traités comme des canaux (\kw{in\_channel} pour un canal de lecture et \kw{out\_channel} pour un canal de sortie). Pour lire/écrire dans un fichier on retrouve les trois étapes habituels
    \begin{itemize}
        \item<1-> L'ouverture du fichier de type {\tt string -> in\_channel} pour une ouverture en lecture et {\tt string -> out\_channel} pour une ouverture en écriture.
        \item<2-> La fonction {\tt input\_line} de type {\tt in\_channel -> string} permet de lire une ligne du fichier (sans le retour chariot final) et {\tt output\_string} de type {\tt out\_channel -> string -> unit} permet d'écrire dans un canal de sortie.
        {\small \textcolor{BrickRed}{\danger}\;} Dans le cas d'une lecture, une exception est levée lorsque la fin de fichier est atteinte.
        \item<3-> Les fonctions {\tt close\_in} et {\tt close\_out} permettent de fermer un fichier (respectivement d'entrée ou de sortie). 
    \end{itemize}
\end{block}
\end{frame}

\begin{frame}[fragile]{\Ctitle}{\stitle}
    \begin{exampleblock}{Compte le nombre de lignes dans un fichier}
    Le programme suivant demande le nom d'un fichier puis affiche son nombre de lignes dans le terminal.
    \inputpartOCaml{compte.ml}{}{\small}{1}{13}
    \end{exampleblock}
\end{frame}


\makess{Argument en ligne de commande}
\begin{frame}[fragile]{\Ctitle}{\stitle}
    \begin{block}{Module \textcolor{yellow}{\tt Sys}}
    \onslide<1->{A la façon de ce qui a été vu en C, on peut passer des arguments en ligne de commande à un programme. Les arguments présents sur la ligne de commande sont rangés dans le tableau \kw{Sys.argv}. \\
    \textcolor{BrickRed}{\small \danger}\; Comme en C, le premier élément du tableau (donc celui d'indice 0) contient le nom de l'exécutable.}
    \onslide<2->{Par exemple, si on compile le programme suivante sous le nom {\tt arg.exe}
    \inputpartOCaml{arg.ml}{}{\small}{1}{2}}
    \onslide<3->{Alors, {\tt ./arg.exe blabla} affichera :\\
    {\tt Nombre d'arguments fournis : 2}\\
    {\tt Le premier argument est : ./arg.exe}\\
    }
\end{block}
\end{frame}

\makess{Récupération d'erreurs}
\begin{frame}[fragile]{\Ctitle}{\stitle}
    \begin{block}{Bloc {\tt try-with}}
    \onslide<1->{Les \textcolor{blue}{exceptions} sont un mécanisme de récupération et de gestion des erreurs pouvant se produire lors de l'exécution d'un programme. Un exemple a déjà été vu dans le programme permettant de compter les lignes d'un fichier. Le mécanisme de gestion est toujours de la forme :\\}
    \onslide<2->{
    \kw{try} {\tt <expr>} \\
    \kw{with} \\
    {\tt | <motif1> -> <expr1>} \\
    {\tt ..}\\
    {\tt | <motifn> -> <exprn>} \\
    }
\end{block}
\end{frame}

\begin{frame}[fragile]{\Ctitle}{\stitle}
\begin{exampleblock}{Exemple}
    On peut éventuellement choisir de poursuivre l'exécution d'un programme lors d'une division par zéro :
    \onslide<2->{\inputpartOCaml{divzero.ml}{}{\small}{1}{6}}
    \onslide<3->{Ici on choisit de renvoyer le plus grand entier représentable (ce qui n'est pas forcément un choix judicieux !)}
\end{exampleblock}
\end{frame}

\end{document}
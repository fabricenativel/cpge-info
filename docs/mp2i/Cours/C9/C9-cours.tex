\PassOptionsToPackage{dvipsnames,table}{xcolor}
\documentclass[10pt]{beamer}
\usepackage{Cours}

\begin{document}

\input{\detokenize{/home/fenarius/Travail/Cours/cpge-info/latex/MacrosCours.tex}}

% Numéro et titre de chapitre
\setcounter{numchap}{9}
\newcommand{\Ctitle}{\cnum {OCaml : aspects fonctiimpératifsonnels}}
\newcommand{\SPATH}{/home/fenarius/Travail/Cours/cpge-info/docs/mp2i/files/C\thenumchap/}

% Définition d'une structure de données
\makess{Variables mutables}
\begin{frame}{\Ctitle}{\stitle}
	\begin{block}{Introduction}
        \begin{itemize}
            \item<1-> Pour le moment, nous nous sommes limités aux aspects fonctionnels d'OCaml et donc aux variables et aux structures de données \textcolor{blue}{non mutables}.
            \item<2-> Cependant, OCaml est un langage de programmation multi-paradigme et la programmation impérative (et donc les variables mutables) peuvent être manipulées en OCaml.
        \end{itemize}
    \end{block}
    
    \begin{block}{Champ mutable d'un enregistrement}
        \onslide<3->{On peut déclarer en OCaml un enregistrement ayant des champs mutables grâce au mot-clé \mintinline{ocaml}{mutable}.}
        \onslide<4->{ Par exemple,
        \inputpartOCaml{\SPATH mutable.ml}{}{}{1}{1}}
        \onslide<5->{\textcolor{BrickRed}{\small \danger} Pour modifier la valeur du champ on utilise \textcolor{blue}{\tt <-}. Le symbole {\tt =} est réservé à la comparaison.}
    \end{block}
\end{frame}

\begin{frame}{\Ctitle}{\stitle}
    \begin{exampleblock}{Exemple}
        \inputpartOCaml{mutable.ml}{\small}{}{1}{12}
    \end{exampleblock}
\end{frame}

\begin{frame}{\Ctitle}{\stitle}
    \begin{block}{Les références}
        \begin{itemize}
            \item<1-> Le type \mintinline{ocaml}{ref}  est prédéfini dans OCaml et correspond exactement à ce que nous venons de faire (sauf que le champ mutable s'appelle {\tt contents})\\
            \onslide<2-> Par exemple \mintinline{ocaml}{let a = {contents = 5};} crée une variable ayant son champ mutable {\tt contents} qui vaut 5.
            \item<3-> Afin d'alléger la syntaxe, on peut écrire directement \mintinline{ocaml}{let a = ref 5;;}. \\
            \onslide<4->\textcolor{gray}{On retiendra que l'effet reste le même : on a crée une variable {\tt a} ayant un champ mutable entier qui contient 5.}
            \item<5-> Si on veut accéder à la valeur du champ mutable, une syntaxe "adoucie" est aussi disponible avec \textcolor{BrickRed}{\tt !}. On écrira par exemple  \mintinline{ocaml}{print_int !a} pour afficher le contenu du champ mutable de {\tt a}. \\
            \onslide<6->\textcolor{gray}{On retiendra que c'est identique à \mintinline{ocaml}{print_int a.contents}}
            \item<7-> Pour modifier la valeur de {\tt a}, une syntaxe plus simple  est aussi disponible avec \textcolor{BrickRed}{\tt :=}. On écrira par exemple \mintinline{ocaml}{a := 5}\\
            \onslide<8-> Par exemple, \mintinline{ocaml}{ a:= !a +1} permet d'incrémenter de 1 la valeur (du champ mutable) de {\tt a}.
        \end{itemize}
    \end{block}
\end{frame}


\begin{frame}{\Ctitle}{\stitle}
    \begin{exampleblock}{Exemple}
        Créer une reférence {\tt a} vers 42 et une référence {\tt b} vers 2023. Echanger le contenu de ces deux variables en utilisant une troisième référence {\tt temp}
        \onslide<2->
        \inputpartOCaml{echange.ml}{}{}{1}{5}
    \end{exampleblock}
\end{frame}

\end{document}
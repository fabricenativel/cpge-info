\PassOptionsToPackage{dvipsnames,table}{xcolor}
\documentclass[10pt]{beamer}
\usepackage{Cours}

\begin{document}


\newcounter{numchap}
\setcounter{numchap}{1}
\newcounter{numframe}
\setcounter{numframe}{0}
\newcommand{\mframe}[1]{\frametitle{#1} \addtocounter{numframe}{1}}
\newcommand{\cnum}{\fbox{\textcolor{yellow}{\textbf{C\thenumchap}}}~}
\newcommand{\makess}[1]{\section{#1} \label{ss\thesection}}
\newcommand{\stitle}{\textcolor{yellow}{\textbf{\thesection. \nameref{ss\thesection}}}}

\definecolor{codebg}{gray}{0.90}
\definecolor{grispale}{gray}{0.95}
\definecolor{fluo}{rgb}{1,0.96,0.62}
\newminted[langageC]{c}{linenos=true,escapeinside=||,highlightcolor=fluo,tabsize=2,breaklines=true}
\newminted[codepython]{python}{linenos=true,escapeinside=||,highlightcolor=fluo,tabsize=2,breaklines=true}
% Inclusion complète (ou partiel en indiquant premiere et dernière ligne) d'un fichier C
\newcommand{\inputC}[3]{\begin{mdframed}[backgroundcolor=codebg] \inputminted[breaklines=true,fontsize=#3,linenos=true,highlightcolor=fluo,tabsize=2,highlightlines={#2}]{c}{#1} \end{mdframed}}
\newcommand{\inputpartC}[5]{\begin{mdframed}[backgroundcolor=codebg] \inputminted[breaklines=true,fontsize=#3,linenos=true,highlightcolor=fluo,tabsize=2,highlightlines={#2},firstline=#4,lastline=#5,firstnumber=1]{c}{#1} \end{mdframed}}
\newcommand{\inputpython}[3]{\begin{mdframed}[backgroundcolor=codebg] \inputminted[breaklines=true,fontsize=#3,linenos=true,highlightcolor=fluo,tabsize=2,highlightlines={#2}]{python}{#1} \end{mdframed}}
\newcommand{\inputpartOCaml}[5]{\begin{mdframed}[backgroundcolor=codebg] \inputminted[breaklines=true,fontsize=#3,linenos=true,highlightcolor=fluo,tabsize=2,highlightlines={#2},firstline=#4,lastline=#5,firstnumber=1]{OCaml}{#1} \end{mdframed}}
\BeforeBeginEnvironment{minted}{\begin{mdframed}[backgroundcolor=codebg]}
\AfterEndEnvironment{minted}{\end{mdframed}}
\newcommand{\kw}[1]{\textcolor{blue}{\tt #1}}

\newtcolorbox{rcadre}[4]{halign=center,colback={#1},colframe={#2},width={#3cm},height={#4cm},valign=center,boxrule=1pt,left=0pt,right=0pt}
\newtcolorbox{cadre}[4]{halign=center,colback={#1},colframe={#2},arc=0mm,width={#3cm},height={#4cm},valign=center,boxrule=1pt,left=0pt,right=0pt}
\newcommand{\myem}[1]{\colorbox{fluo}{#1}}
\mdfsetup{skipabove=1pt,skipbelow=-2pt}



% Noeud dans un cadre pour les arbres
\newcommand{\noeud}[2]{\Tr{\fbox{\textcolor{#1}{\tt #2}}}}

\newcommand{\htmlmode}{\lstset{language=html,numbers=left, tabsize=4, frame=single, breaklines=true, keywordstyle=\ttfamily, basicstyle=\small,
   numberstyle=\tiny\ttfamily, framexleftmargin=0mm, backgroundcolor=\color{grispale}, xleftmargin=12mm,showstringspaces=false}}
\newcommand{\pythonmode}{\lstset{
   language=python,
   linewidth=\linewidth,
   numbers=left,
   tabsize=4,
   frame=single,
   breaklines=true,
   keywordstyle=\ttfamily\color{blue},
   basicstyle=\small,
   numberstyle=\tiny\ttfamily,
   framexleftmargin=-2mm,
   numbersep=-0.5mm,
   backgroundcolor=\color{codebg},
   xleftmargin=-1mm, 
   showstringspaces=false,
   commentstyle=\color{gray},
   stringstyle=\color{OliveGreen},
   emph={turtle,Screen,Turtle},
   emphstyle=\color{RawSienna},
   morekeywords={setheading,goto,backward,forward,left,right,pendown,penup,pensize,color,speed,hideturtle,showturtle,forward}}
   }
   \newcommand{\Cmode}{\lstset{
      language=[ANSI]C,
      linewidth=\linewidth,
      numbers=left,
      tabsize=4,
      frame=single,
      breaklines=true,
      keywordstyle=\ttfamily\color{blue},
      basicstyle=\small,
      numberstyle=\tiny\ttfamily,
      framexleftmargin=0mm,
      numbersep=2mm,
      backgroundcolor=\color{codebg},
      xleftmargin=0mm, 
      showstringspaces=false,
      commentstyle=\color{gray},
      stringstyle=\color{OliveGreen},
      emphstyle=\color{RawSienna},
      escapechar=\|,
      morekeywords={}}
      }
\newcommand{\bashmode}{\lstset{language=bash,numbers=left, tabsize=2, frame=single, breaklines=true, basicstyle=\ttfamily,
   numberstyle=\tiny\ttfamily, framexleftmargin=0mm, backgroundcolor=\color{grispale}, xleftmargin=12mm, showstringspaces=false}}
\newcommand{\exomode}{\lstset{language=python,numbers=left, tabsize=2, frame=single, breaklines=true, basicstyle=\ttfamily,
   numberstyle=\tiny\ttfamily, framexleftmargin=13mm, xleftmargin=12mm, basicstyle=\small, showstringspaces=false}}
   
   
  
%tei pour placer les images
%tei{nom de l’image}{échelle de l’image}{sens}{texte a positionner}
%sens ="1" (droite) ou "2" (gauche)
\newlength{\ltxt}
\newcommand{\tei}[4]{
\setlength{\ltxt}{\linewidth}
\setbox0=\hbox{\includegraphics[scale=#2]{#1}}
\addtolength{\ltxt}{-\wd0}
\addtolength{\ltxt}{-10pt}
\ifthenelse{\equal{#3}{1}}{
\begin{minipage}{\wd0}
\includegraphics[scale=#2]{#1}
\end{minipage}
\hfill
\begin{minipage}{\ltxt}
#4
\end{minipage}
}{
\begin{minipage}{\ltxt}
#4
\end{minipage}
\hfill
\begin{minipage}{\wd0}
\includegraphics[scale=#2]{#1}
\end{minipage}
}
}

%Juxtaposition d'une image pspciture et de texte 
%#1: = code pstricks de l'image
%#2: largeur de l'image
%#3: hauteur de l'image
%#4: Texte à écrire
\newcommand{\ptp}[4]{
\setlength{\ltxt}{\linewidth}
\addtolength{\ltxt}{-#2 cm}
\addtolength{\ltxt}{-0.1 cm}
\begin{minipage}[b][#3 cm][t]{\ltxt}
#4
\end{minipage}\hfill
\begin{minipage}[b][#3 cm][c]{#2 cm}
#1
\end{minipage}\par
}



%Macros pour les graphiques
\psset{linewidth=0.5\pslinewidth,PointSymbol=x}
\setlength{\fboxrule}{0.5pt}
\newcounter{tempangle}

%Marque la longueur du segment d'extrémité  #1 et  #2 avec la valeur #3, #4 est la distance par rapport au segment (en %age de la valeur de celui ci) et #5 l'orientation du marquage : +90 ou -90
\newcommand{\afflong}[5]{
\pstRotation[RotAngle=#4,PointSymbol=none,PointName=none]{#1}{#2}[X] 
\pstHomO[PointSymbol=none,PointName=none,HomCoef=#5]{#1}{X}[Y]
\pstTranslation[PointSymbol=none,PointName=none]{#1}{#2}{Y}[Z]
 \ncline{|<->|,linewidth=0.25\pslinewidth}{Y}{Z} \ncput*[nrot=:U]{\footnotesize{#3}}
}
\newcommand{\afflongb}[3]{
\ncline{|<->|,linewidth=0}{#1}{#2} \naput*[nrot=:U]{\footnotesize{#3}}
}

%Construis le point #4 situé à #2 cm du point #1 avant un angle #3 par rapport à l'horizontale. #5 = liste de paramètre
\newcommand{\lsegment}[5]{\pstGeonode[PointSymbol=none,PointName=none](0,0){O'}(#2,0){I'} \pstTranslation[PointSymbol=none,PointName=none]{O'}{I'}{#1}[J'] \pstRotation[RotAngle=#3,PointSymbol=x,#5]{#1}{J'}[#4]}
\newcommand{\tsegment}[5]{\pstGeonode[PointSymbol=none,PointName=none](0,0){O'}(#2,0){I'} \pstTranslation[PointSymbol=none,PointName=none]{O'}{I'}{#1}[J'] \pstRotation[RotAngle=#3,PointSymbol=x,#5]{#1}{J'}[#4] \pstLineAB{#4}{#1}}

%Construis le point #4 situé à #3 cm du point #1 et faisant un angle de  90° avec la droite (#1,#2) #5 = liste de paramètre
\newcommand{\psegment}[5]{
\pstGeonode[PointSymbol=none,PointName=none](0,0){O'}(#3,0){I'}
 \pstTranslation[PointSymbol=none,PointName=none]{O'}{I'}{#1}[J']
 \pstInterLC[PointSymbol=none,PointName=none]{#1}{#2}{#1}{J'}{M1}{M2} \pstRotation[RotAngle=-90,PointSymbol=x,#5]{#1}{M1}[#4]
  }
  
%Construis le point #4 situé à #3 cm du point #1 et faisant un angle de  #5° avec la droite (#1,#2) #6 = liste de paramètre
\newcommand{\mlogo}[6]{
\pstGeonode[PointSymbol=none,PointName=none](0,0){O'}(#3,0){I'}
 \pstTranslation[PointSymbol=none,PointName=none]{O'}{I'}{#1}[J']
 \pstInterLC[PointSymbol=none,PointName=none]{#1}{#2}{#1}{J'}{M1}{M2} \pstRotation[RotAngle=#5,PointSymbol=x,#6]{#1}{M2}[#4]
  }

% Construis un triangle avec #1=liste des 3 sommets séparés par des virgules, #2=liste des 3 longueurs séparés par des virgules, #3 et #4 : paramètre d'affichage des 2e et 3 points et #5 : inclinaison par rapport à l'horizontale
%autre macro identique mais sans tracer les segments joignant les sommets
\noexpandarg
\newcommand{\Triangleccc}[5]{
\StrBefore{#1}{,}[\pointA]
\StrBetween[1,2]{#1}{,}{,}[\pointB]
\StrBehind[2]{#1}{,}[\pointC]
\StrBefore{#2}{,}[\coteA]
\StrBetween[1,2]{#2}{,}{,}[\coteB]
\StrBehind[2]{#2}{,}[\coteC]
\tsegment{\pointA}{\coteA}{#5}{\pointB}{#3} 
\lsegment{\pointA}{\coteB}{0}{Z1}{PointSymbol=none, PointName=none}
\lsegment{\pointB}{\coteC}{0}{Z2}{PointSymbol=none, PointName=none}
\pstInterCC{\pointA}{Z1}{\pointB}{Z2}{\pointC}{Z3} 
\pstLineAB{\pointA}{\pointC} \pstLineAB{\pointB}{\pointC}
\pstSymO[PointName=\pointC,#4]{C}{C}[C]
}
\noexpandarg
\newcommand{\TrianglecccP}[5]{
\StrBefore{#1}{,}[\pointA]
\StrBetween[1,2]{#1}{,}{,}[\pointB]
\StrBehind[2]{#1}{,}[\pointC]
\StrBefore{#2}{,}[\coteA]
\StrBetween[1,2]{#2}{,}{,}[\coteB]
\StrBehind[2]{#2}{,}[\coteC]
\tsegment{\pointA}{\coteA}{#5}{\pointB}{#3} 
\lsegment{\pointA}{\coteB}{0}{Z1}{PointSymbol=none, PointName=none}
\lsegment{\pointB}{\coteC}{0}{Z2}{PointSymbol=none, PointName=none}
\pstInterCC[PointNameB=none,PointSymbolB=none,#4]{\pointA}{Z1}{\pointB}{Z2}{\pointC}{Z1} 
}


% Construis un triangle avec #1=liste des 3 sommets séparés par des virgules, #2=liste formée de 2 longueurs et d'un angle séparés par des virgules, #3 et #4 : paramètre d'affichage des 2e et 3 points et #5 : inclinaison par rapport à l'horizontale
%autre macro identique mais sans tracer les segments joignant les sommets
\newcommand{\Trianglecca}[5]{
\StrBefore{#1}{,}[\pointA]
\StrBetween[1,2]{#1}{,}{,}[\pointB]
\StrBehind[2]{#1}{,}[\pointC]
\StrBefore{#2}{,}[\coteA]
\StrBetween[1,2]{#2}{,}{,}[\coteB]
\StrBehind[2]{#2}{,}[\angleA]
\tsegment{\pointA}{\coteA}{#5}{\pointB}{#3} 
\setcounter{tempangle}{#5}
\addtocounter{tempangle}{\angleA}
\tsegment{\pointA}{\coteB}{\thetempangle}{\pointC}{#4}
\pstLineAB{\pointB}{\pointC}
}
\newcommand{\TriangleccaP}[5]{
\StrBefore{#1}{,}[\pointA]
\StrBetween[1,2]{#1}{,}{,}[\pointB]
\StrBehind[2]{#1}{,}[\pointC]
\StrBefore{#2}{,}[\coteA]
\StrBetween[1,2]{#2}{,}{,}[\coteB]
\StrBehind[2]{#2}{,}[\angleA]
\lsegment{\pointA}{\coteA}{#5}{\pointB}{#3} 
\setcounter{tempangle}{#5}
\addtocounter{tempangle}{\angleA}
\lsegment{\pointA}{\coteB}{\thetempangle}{\pointC}{#4}
}

% Construis un triangle avec #1=liste des 3 sommets séparés par des virgules, #2=liste formée de 1 longueurs et de deux angle séparés par des virgules, #3 et #4 : paramètre d'affichage des 2e et 3 points et #5 : inclinaison par rapport à l'horizontale
%autre macro identique mais sans tracer les segments joignant les sommets
\newcommand{\Trianglecaa}[5]{
\StrBefore{#1}{,}[\pointA]
\StrBetween[1,2]{#1}{,}{,}[\pointB]
\StrBehind[2]{#1}{,}[\pointC]
\StrBefore{#2}{,}[\coteA]
\StrBetween[1,2]{#2}{,}{,}[\angleA]
\StrBehind[2]{#2}{,}[\angleB]
\tsegment{\pointA}{\coteA}{#5}{\pointB}{#3} 
\setcounter{tempangle}{#5}
\addtocounter{tempangle}{\angleA}
\lsegment{\pointA}{1}{\thetempangle}{Z1}{PointSymbol=none, PointName=none}
\setcounter{tempangle}{#5}
\addtocounter{tempangle}{180}
\addtocounter{tempangle}{-\angleB}
\lsegment{\pointB}{1}{\thetempangle}{Z2}{PointSymbol=none, PointName=none}
\pstInterLL[#4]{\pointA}{Z1}{\pointB}{Z2}{\pointC}
\pstLineAB{\pointA}{\pointC}
\pstLineAB{\pointB}{\pointC}
}
\newcommand{\TrianglecaaP}[5]{
\StrBefore{#1}{,}[\pointA]
\StrBetween[1,2]{#1}{,}{,}[\pointB]
\StrBehind[2]{#1}{,}[\pointC]
\StrBefore{#2}{,}[\coteA]
\StrBetween[1,2]{#2}{,}{,}[\angleA]
\StrBehind[2]{#2}{,}[\angleB]
\lsegment{\pointA}{\coteA}{#5}{\pointB}{#3} 
\setcounter{tempangle}{#5}
\addtocounter{tempangle}{\angleA}
\lsegment{\pointA}{1}{\thetempangle}{Z1}{PointSymbol=none, PointName=none}
\setcounter{tempangle}{#5}
\addtocounter{tempangle}{180}
\addtocounter{tempangle}{-\angleB}
\lsegment{\pointB}{1}{\thetempangle}{Z2}{PointSymbol=none, PointName=none}
\pstInterLL[#4]{\pointA}{Z1}{\pointB}{Z2}{\pointC}
}

%Construction d'un cercle de centre #1 et de rayon #2 (en cm)
\newcommand{\Cercle}[2]{
\lsegment{#1}{#2}{0}{Z1}{PointSymbol=none, PointName=none}
\pstCircleOA{#1}{Z1}
}

%construction d'un parallélogramme #1 = liste des sommets, #2 = liste contenant les longueurs de 2 côtés consécutifs et leurs angles;  #3, #4 et #5 : paramètre d'affichage des sommets #6 inclinaison par rapport à l'horizontale 
% meme macro sans le tracé des segements
\newcommand{\Para}[6]{
\StrBefore{#1}{,}[\pointA]
\StrBetween[1,2]{#1}{,}{,}[\pointB]
\StrBetween[2,3]{#1}{,}{,}[\pointC]
\StrBehind[3]{#1}{,}[\pointD]
\StrBefore{#2}{,}[\longueur]
\StrBetween[1,2]{#2}{,}{,}[\largeur]
\StrBehind[2]{#2}{,}[\angle]
\tsegment{\pointA}{\longueur}{#6}{\pointB}{#3} 
\setcounter{tempangle}{#6}
\addtocounter{tempangle}{\angle}
\tsegment{\pointA}{\largeur}{\thetempangle}{\pointD}{#5}
\pstMiddleAB[PointName=none,PointSymbol=none]{\pointB}{\pointD}{Z1}
\pstSymO[#4]{Z1}{\pointA}[\pointC]
\pstLineAB{\pointB}{\pointC}
\pstLineAB{\pointC}{\pointD}
}
\newcommand{\ParaP}[6]{
\StrBefore{#1}{,}[\pointA]
\StrBetween[1,2]{#1}{,}{,}[\pointB]
\StrBetween[2,3]{#1}{,}{,}[\pointC]
\StrBehind[3]{#1}{,}[\pointD]
\StrBefore{#2}{,}[\longueur]
\StrBetween[1,2]{#2}{,}{,}[\largeur]
\StrBehind[2]{#2}{,}[\angle]
\lsegment{\pointA}{\longueur}{#6}{\pointB}{#3} 
\setcounter{tempangle}{#6}
\addtocounter{tempangle}{\angle}
\lsegment{\pointA}{\largeur}{\thetempangle}{\pointD}{#5}
\pstMiddleAB[PointName=none,PointSymbol=none]{\pointB}{\pointD}{Z1}
\pstSymO[#4]{Z1}{\pointA}[\pointC]
}


%construction d'un cerf-volant #1 = liste des sommets, #2 = liste contenant les longueurs de 2 côtés consécutifs et leurs angles;  #3, #4 et #5 : paramètre d'affichage des sommets #6 inclinaison par rapport à l'horizontale 
% meme macro sans le tracé des segements
\newcommand{\CerfVolant}[6]{
\StrBefore{#1}{,}[\pointA]
\StrBetween[1,2]{#1}{,}{,}[\pointB]
\StrBetween[2,3]{#1}{,}{,}[\pointC]
\StrBehind[3]{#1}{,}[\pointD]
\StrBefore{#2}{,}[\longueur]
\StrBetween[1,2]{#2}{,}{,}[\largeur]
\StrBehind[2]{#2}{,}[\angle]
\tsegment{\pointA}{\longueur}{#6}{\pointB}{#3} 
\setcounter{tempangle}{#6}
\addtocounter{tempangle}{\angle}
\tsegment{\pointA}{\largeur}{\thetempangle}{\pointD}{#5}
\pstOrtSym[#4]{\pointB}{\pointD}{\pointA}[\pointC]
\pstLineAB{\pointB}{\pointC}
\pstLineAB{\pointC}{\pointD}
}

%construction d'un quadrilatère quelconque #1 = liste des sommets, #2 = liste contenant les longueurs des 4 côtés et l'angle entre 2 cotés consécutifs  #3, #4 et #5 : paramètre d'affichage des sommets #6 inclinaison par rapport à l'horizontale 
% meme macro sans le tracé des segements
\newcommand{\Quadri}[6]{
\StrBefore{#1}{,}[\pointA]
\StrBetween[1,2]{#1}{,}{,}[\pointB]
\StrBetween[2,3]{#1}{,}{,}[\pointC]
\StrBehind[3]{#1}{,}[\pointD]
\StrBefore{#2}{,}[\coteA]
\StrBetween[1,2]{#2}{,}{,}[\coteB]
\StrBetween[2,3]{#2}{,}{,}[\coteC]
\StrBetween[3,4]{#2}{,}{,}[\coteD]
\StrBehind[4]{#2}{,}[\angle]
\tsegment{\pointA}{\coteA}{#6}{\pointB}{#3} 
\setcounter{tempangle}{#6}
\addtocounter{tempangle}{\angle}
\tsegment{\pointA}{\coteD}{\thetempangle}{\pointD}{#5}
\lsegment{\pointB}{\coteB}{0}{Z1}{PointSymbol=none, PointName=none}
\lsegment{\pointD}{\coteC}{0}{Z2}{PointSymbol=none, PointName=none}
\pstInterCC[PointNameA=none,PointSymbolA=none,#4]{\pointB}{Z1}{\pointD}{Z2}{Z3}{\pointC} 
\pstLineAB{\pointB}{\pointC}
\pstLineAB{\pointC}{\pointD}
}


% Définition des colonnes centrées ou à droite pour tabularx
\newcolumntype{Y}{>{\centering\arraybackslash}X}
\newcolumntype{Z}{>{\flushright\arraybackslash}X}

%Les pointillés à remplir par les élèves
\newcommand{\po}[1]{\makebox[#1 cm]{\dotfill}}
\newcommand{\lpo}[1][3]{%
\multido{}{#1}{\makebox[\linewidth]{\dotfill}
}}

%Liste des pictogrammes utilisés sur la fiche d'exercice ou d'activités
\newcommand{\bombe}{\faBomb}
\newcommand{\livre}{\faBook}
\newcommand{\calculatrice}{\faCalculator}
\newcommand{\oral}{\faCommentO}
\newcommand{\surfeuille}{\faEdit}
\newcommand{\ordinateur}{\faLaptop}
\newcommand{\ordi}{\faDesktop}
\newcommand{\ciseaux}{\faScissors}
\newcommand{\danger}{\faExclamationTriangle}
\newcommand{\out}{\faSignOut}
\newcommand{\cadeau}{\faGift}
\newcommand{\flash}{\faBolt}
\newcommand{\lumiere}{\faLightbulb}
\newcommand{\compas}{\dsmathematical}
\newcommand{\calcullitteral}{\faTimesCircleO}
\newcommand{\raisonnement}{\faCogs}
\newcommand{\recherche}{\faSearch}
\newcommand{\rappel}{\faHistory}
\newcommand{\video}{\faFilm}
\newcommand{\capacite}{\faPuzzlePiece}
\newcommand{\aide}{\faLifeRing}
\newcommand{\loin}{\faExternalLink}
\newcommand{\groupe}{\faUsers}
\newcommand{\bac}{\faGraduationCap}
\newcommand{\histoire}{\faUniversity}
\newcommand{\coeur}{\faSave}
\newcommand{\python}{\faPython}
\newcommand{\os}{\faMicrochip}
\newcommand{\rd}{\faCubes}
\newcommand{\data}{\faColumns}
\newcommand{\web}{\faCode}
\newcommand{\prog}{\faFile}
\newcommand{\algo}{\faCogs}
\newcommand{\important}{\faExclamationCircle}
\newcommand{\maths}{\faTimesCircle}
% Traitement des données en tables
\newcommand{\tables}{\faColumns}
% Types construits
\newcommand{\construits}{\faCubes}
% Type et valeurs de base
\newcommand{\debase}{{\footnotesize \faCube}}
% Systèmes d'exploitation
\newcommand{\linux}{\faLinux}
\newcommand{\sd}{\faProjectDiagram}
\newcommand{\bd}{\faDatabase}

%Les ensembles de nombres
\renewcommand{\N}{\mathbb{N}}
\newcommand{\D}{\mathbb{D}}
\newcommand{\Z}{\mathbb{Z}}
\newcommand{\Q}{\mathbb{Q}}
\newcommand{\R}{\mathbb{R}}
\newcommand{\C}{\mathbb{C}}

%Ecriture des vecteurs
\newcommand{\vect}[1]{\vbox{\halign{##\cr 
  \tiny\rightarrowfill\cr\noalign{\nointerlineskip\vskip1pt} 
  $#1\mskip2mu$\cr}}}


%Compteur activités/exos et question et mise en forme titre et questions
\newcounter{numact}
\setcounter{numact}{1}
\newcounter{numseance}
\setcounter{numseance}{1}
\newcounter{numexo}
\setcounter{numexo}{0}
\newcounter{numprojet}
\setcounter{numprojet}{0}
\newcounter{numquestion}
\newcommand{\espace}[1]{\rule[-1ex]{0pt}{#1 cm}}
\newcommand{\Quest}[3]{
\addtocounter{numquestion}{1}
\begin{tabularx}{\textwidth}{X|m{1cm}|}
\cline{2-2}
\textbf{\sffamily{\alph{numquestion})}} #1 & \dots / #2 \\
\hline 
\multicolumn{2}{|l|}{\espace{#3}} \\
\hline
\end{tabularx}
}
\newcommand{\QuestR}[3]{
\addtocounter{numquestion}{1}
\begin{tabularx}{\textwidth}{X|m{1cm}|}
\cline{2-2}
\textbf{\sffamily{\alph{numquestion})}} #1 & \dots / #2 \\
\hline 
\multicolumn{2}{|l|}{\cor{#3}} \\
\hline
\end{tabularx}
}
\newcommand{\Pre}{{\sc nsi} 1\textsuperscript{e}}
\newcommand{\Term}{{\sc nsi} Terminale}
\newcommand{\Sec}{2\textsuperscript{e}}
\newcommand{\Exo}[2]{ \addtocounter{numexo}{1} \ding{113} \textbf{\sffamily{Exercice \thenumexo}} : \textit{#1} \hfill #2  \setcounter{numquestion}{0}}
\newcommand{\Projet}[1]{ \addtocounter{numprojet}{1} \ding{118} \textbf{\sffamily{Projet \thenumprojet}} : \textit{#1}}
\newcommand{\ExoD}[2]{ \addtocounter{numexo}{1} \ding{113} \textbf{\sffamily{Exercice \thenumexo}}  \textit{(#1 pts)} \hfill #2  \setcounter{numquestion}{0}}
\newcommand{\ExoB}[2]{ \addtocounter{numexo}{1} \ding{113} \textbf{\sffamily{Exercice \thenumexo}}  \textit{(Bonus de +#1 pts maximum)} \hfill #2  \setcounter{numquestion}{0}}
\newcommand{\Act}[2]{ \ding{113} \textbf{\sffamily{Activité \thenumact}} : \textit{#1} \hfill #2  \addtocounter{numact}{1} \setcounter{numquestion}{0}}
\newcommand{\Seance}{ \rule{1.5cm}{0.5pt}\raisebox{-3pt}{\framebox[4cm]{\textbf{\sffamily{Séance \thenumseance}}}}\hrulefill  \\
  \addtocounter{numseance}{1}}
\newcommand{\Acti}[2]{ {\footnotesize \ding{117}} \textbf{\sffamily{Activité \thenumact}} : \textit{#1} \hfill #2  \addtocounter{numact}{1} \setcounter{numquestion}{0}}
\newcommand{\titre}[1]{\begin{Large}\textbf{\ding{118}}\end{Large} \begin{large}\textbf{ #1}\end{large} \vspace{0.2cm}}
\newcommand{\QListe}[1][0]{
\ifthenelse{#1=0}
{\begin{enumerate}[partopsep=0pt,topsep=0pt,parsep=0pt,itemsep=0pt,label=\textbf{\sffamily{\arabic*.}},series=question]}
{\begin{enumerate}[resume*=question]}}
\newcommand{\SQListe}[1][0]{
\ifthenelse{#1=0}
{\begin{enumerate}[partopsep=0pt,topsep=0pt,parsep=0pt,itemsep=0pt,label=\textbf{\sffamily{\alph*)}},series=squestion]}
{\begin{enumerate}[resume*=squestion]}}
\newcommand{\SQListeL}[1][0]{
\ifthenelse{#1=0}
{\begin{enumerate*}[partopsep=0pt,topsep=0pt,parsep=0pt,itemsep=0pt,label=\textbf{\sffamily{\alph*)}},series=squestion]}
{\begin{enumerate*}[resume*=squestion]}}
\newcommand{\FinListe}{\end{enumerate}}
\newcommand{\FinListeL}{\end{enumerate*}}

%Mise en forme de la correction
\newcommand{\cor}[1]{\par \textcolor{OliveGreen}{#1}}
\newcommand{\br}[1]{\cor{\textbf{#1}}}
\newcommand{\tcor}[1]{\begin{tcolorbox}[width=0.92\textwidth,colback={white},colbacktitle=white,coltitle=OliveGreen,colframe=green!75!black,boxrule=0.2mm]   
\cor{#1}
\end{tcolorbox}
}
\newcommand{\rc}[1]{\textcolor{OliveGreen}{#1}}
\newcommand{\pmc}[1]{\textcolor{blue}{\tt #1}}
\newcommand{\tmc}[1]{\textcolor{RawSienna}{\tt #1}}


%Référence aux exercices par leur numéro
\newcommand{\refexo}[1]{
\refstepcounter{numexo}
\addtocounter{numexo}{-1}
\label{#1}}

%Séparation entre deux activités
\newcommand{\separateur}{\begin{center}
\rule{1.5cm}{0.5pt}\raisebox{-3pt}{\ding{117}}\rule{1.5cm}{0.5pt}  \vspace{0.2cm}
\end{center}}

%Entête et pied de page
\newcommand{\snt}[1]{\lhead{\textbf{SNT -- La photographie numérique} \rhead{\textit{Lycée Nord}}}}
\newcommand{\Activites}[2]{\lhead{\textbf{{\sc #1}}}
\rhead{Activités -- \textbf{#2}}
\cfoot{}}
\newcommand{\Exos}[2]{\lhead{\textbf{Fiche d'exercices: {\sc #1}}}
\rhead{Niveau: \textbf{#2}}
\cfoot{}}
\newcommand{\Devoir}[2]{\lhead{\textbf{Devoir de mathématiques : {\sc #1}}}
\rhead{\textbf{#2}} \setlength{\fboxsep}{8pt}
\begin{center}
%Titre de la fiche
\fbox{\parbox[b][1cm][t]{0.3\textwidth}{Nom : \hfill \po{3} \par \vfill Prénom : \hfill \po{3}} } \hfill 
\fbox{\parbox[b][1cm][t]{0.6\textwidth}{Note : \po{1} / 20} }
\end{center} \cfoot{}}

%Devoir programmation en NSI (pas à rendre sur papier)
\newcommand{\PNSI}[2]{\lhead{\textbf{Devoir de {\sc nsi} : \textsf{ #1}}}
\rhead{\textbf{#2}} \setlength{\fboxsep}{8pt}
\begin{tcolorbox}[title=\textcolor{black}{\danger\; A lire attentivement},colbacktitle=lightgray]
{\begin{enumerate}
\item Rendre tous vous programmes en les envoyant par mail à l'adresse {\tt fnativel2@ac-reunion.fr}, en précisant bien dans le sujet vos noms et prénoms
\item Un programme qui fonctionne mal ou pas du tout peut rapporter des points
\item Les bonnes pratiques de programmation (clarté et lisiblité du code) rapportent des points
\end{enumerate}
}
\end{tcolorbox}
 \cfoot{}}


%Devoir de NSI
\newcommand{\DNSI}[2]{\lhead{\textbf{Devoir de {\sc nsi} : \textsf{ #1}}}
\rhead{\textbf{#2}} \setlength{\fboxsep}{8pt}
\begin{center}
%Titre de la fiche
\fbox{\parbox[b][1cm][t]{0.3\textwidth}{Nom : \hfill \po{3} \par \vfill Prénom : \hfill \po{3}} } \hfill 
\fbox{\parbox[b][1cm][t]{0.6\textwidth}{Note : \po{1} / 10} }
\end{center} \cfoot{}}

\newcommand{\DevoirNSI}[2]{\lhead{\textbf{Devoir de {\sc nsi} : {\sc #1}}}
\rhead{\textbf{#2}} \setlength{\fboxsep}{8pt}
\cfoot{}}

%La définition de la commande QCM pour auto-multiple-choice
%En premier argument le sujet du qcm, deuxième argument : la classe, 3e : la durée prévue et #4 : présence ou non de questions avec plusieurs bonnes réponses
\newcommand{\QCM}[4]{
{\large \textbf{\ding{52} QCM : #1}} -- Durée : \textbf{#3 min} \hfill {\large Note : \dots/10} 
\hrule \vspace{0.1cm}\namefield{}
Nom :  \textbf{\textbf{\nom{}}} \qquad \qquad Prénom :  \textbf{\prenom{}}  \hfill Classe: \textbf{#2}
\vspace{0.2cm}
\hrule  
\begin{itemize}[itemsep=0pt]
\item[-] \textit{Une bonne réponse vaut un point, une absence de réponse n'enlève pas de point. }
\item[\danger] \textit{Une mauvaise réponse enlève un point.}
\ifthenelse{#4=1}{\item[-] \textit{Les questions marquées du symbole \multiSymbole{} peuvent avoir plusieurs bonnes réponses possibles.}}{}
\end{itemize}
}
\newcommand{\DevoirC}[2]{
\renewcommand{\footrulewidth}{0.5pt}
\lhead{\textbf{Devoir de mathématiques : {\sc #1}}}
\rhead{\textbf{#2}} \setlength{\fboxsep}{8pt}
\fbox{\parbox[b][0.4cm][t]{0.955\textwidth}{Nom : \po{5} \hfill Prénom : \po{5} \hfill Classe: \textbf{1}\textsuperscript{$\dots$}} } 
\rfoot{\thepage} \cfoot{} \lfoot{Lycée Nord}}
\newcommand{\DevoirInfo}[2]{\lhead{\textbf{Evaluation : {\sc #1}}}
\rhead{\textbf{#2}} \setlength{\fboxsep}{8pt}
 \cfoot{}}
\newcommand{\DM}[2]{\lhead{\textbf{Devoir maison à rendre le #1}} \rhead{\textbf{#2}}}

%Macros permettant l'affichage des touches de la calculatrice
%Touches classiques : #1 = 0 fond blanc pour les nombres et #1= 1gris pour les opérations et entrer, second paramètre=contenu
%Si #2=1 touche arrondi avec fond gris
\newcommand{\TCalc}[2]{
\setlength{\fboxsep}{0.1pt}
\ifthenelse{#1=0}
{\psframebox[fillstyle=solid, fillcolor=white]{\parbox[c][0.25cm][c]{0.6cm}{\centering #2}}}
{\ifthenelse{#1=1}
{\psframebox[fillstyle=solid, fillcolor=lightgray]{\parbox[c][0.25cm][c]{0.6cm}{\centering #2}}}
{\psframebox[framearc=.5,fillstyle=solid, fillcolor=white]{\parbox[c][0.25cm][c]{0.6cm}{\centering #2}}}
}}
\newcommand{\Talpha}{\psdblframebox[fillstyle=solid, fillcolor=white]{\hspace{-0.05cm}\parbox[c][0.25cm][c]{0.65cm}{\centering \scriptsize{alpha}}} \;}
\newcommand{\Tsec}{\psdblframebox[fillstyle=solid, fillcolor=white]{\parbox[c][0.25cm][c]{0.6cm}{\centering \scriptsize 2nde}} \;}
\newcommand{\Tfx}{\psdblframebox[fillstyle=solid, fillcolor=white]{\parbox[c][0.25cm][c]{0.6cm}{\centering \scriptsize $f(x)$}} \;}
\newcommand{\Tvar}{\psframebox[framearc=.5,fillstyle=solid, fillcolor=white]{\hspace{-0.22cm} \parbox[c][0.25cm][c]{0.82cm}{$\scriptscriptstyle{X,T,\theta,n}$}}}
\newcommand{\Tgraphe}{\psdblframebox[fillstyle=solid, fillcolor=white]{\hspace{-0.08cm}\parbox[c][0.25cm][c]{0.68cm}{\centering \tiny{graphe}}} \;}
\newcommand{\Tfen}{\psdblframebox[fillstyle=solid, fillcolor=white]{\hspace{-0.08cm}\parbox[c][0.25cm][c]{0.68cm}{\centering \tiny{fenêtre}}} \;}
\newcommand{\Ttrace}{\psdblframebox[fillstyle=solid, fillcolor=white]{\parbox[c][0.25cm][c]{0.6cm}{\centering \scriptsize{trace}}} \;}

% Macroi pour l'affichage  d'un entier n dans  une base b
\newcommand{\base}[2]{ \overline{#1}^{#2}}
% Intervalle d'entiers
\newcommand{\intN}[2]{\llbracket #1; #2 \rrbracket}}

% Numéro et titre de chapitre
\setcounter{numchap}{5}
\newcommand{\Ctitle}{\cnum {Récursivité}}


\makess{Définition et exemples}
\begin{frame}{\Ctitle}{\stitle}
	\begin{alertblock}{Définition}
		\onslide<2->{En informatique, on dit qu'une fonction est \textcolor{red}{récursive},}
		\onslide<3->{lorsque cette fonction fait appel à elle-même.}
	\end{alertblock}
	\onslide<4->{
		\begin{block}{Remarques}
			\begin{itemize}
				\item<5-> Une fonction récursive permet donc, \textit{comme une boucle}, de répéter des instructions. Une même fonction peut donc souvent se programmer de façon \textcolor{blue}{itérative} (avec des boucles) ou de façon \textcolor{blue}{récursive} (en s'appelant elle-même).
				\item<6-> Une fonction récursive doit toujours \textcolor{blue}{contenir une condition d'arrêt}, dans le cas contraire elle s'appelle elle-même à l'infini et le programme ne se termine jamais.
				\item<7-> Les valeurs passées en paramètres lors des appels successifs doivent être différents, sinon la fonction s'exécute à l'identique à chaque appel et donc boucle à l'infini.
			\end{itemize}
		\end{block}}
\end{frame}

% Récursivité : Exemples des puissances
\begin{frame}{\Ctitle}{\stitle}
	\begin{exampleblock}{Exemple : les puissances positives}
		En mathématiques, pour un nombre quelconque $a$ et un entier positif $n$, on définit $a$ puissance $n$ par :\\
		$ a^n = a \times a \times \dots \times a $, et on convient que $a^0=1$
		\begin{itemize}
			\item<2->{Définir une fonction {\tt puissance} qui prend en argument un flottant $a$ et un entier $n$ et renvoie $a^n$ en effectuant ce calcul de façon itératif}
			\item<3->{Recopier et compléter : $a^n = \dots \times a^{\dots}$}
			\item<4->{En déduire une version récursive de la fonction calculant les puissances}
		\end{itemize}
	\end{exampleblock}
\end{frame}

% Récursivité : Exemples des puissances
\begin{frame}[fragile]{\Ctitle}{\stitle}
	\begin{exampleblock}{Exemple : les puissances positives}
		\begin{itemize}
			\item<1-> \textcolor{OliveGreen}{Puissance : version itérative}
				\inputpartC{/home/fenarius/Travail/Cours/cpge-info/docs/mp2i/files/C5/puissance.c}{}{\small}{3}{7}
			\item<2-> $a^n = \textcolor{OliveGreen}{a} \times a^{\textcolor{OliveGreen}{n-1}}$
			\item<3-> \textcolor{OliveGreen}{Puissance : version récursive}
				\inputpartC{/home/fenarius/Travail/Cours/cpge-info/docs/mp2i/files/C5/puissance.c}{}{\small}{9}{13}
		\end{itemize}
	\end{exampleblock}
\end{frame}

% Récursivité détour par OCaml
\begin{frame}[fragile]{\Ctitle}{\stitle}
	\begin{block}{Un petit détour par OCaml}
		La récursivité est très présente dans le paradigme de programmation fonctionnel que nous verrons en OCaml.
		On présente donc ici la version récursive des puissances en OCaml (même si nous n'avons pas encore commencé son apprentissage)
		\onslide<2->\inputpartOCaml{/home/fenarius/Travail/Cours/cpge-info/docs/mp2i/files/C5/puissance.ml}{}{\small}{1}{2}
	\end{block}
\end{frame}

% Récursivité détour par OCaml
\begin{frame}[fragile]{\Ctitle}{\stitle}
	\begin{block}{Un petit détour par OCaml}
		La récursivité est très présente dans le paradigme de programmation fonctionnel que nous verrons en OCaml.
		On présente donc ici la version récursive des puissances en OCaml (même si nous n'avons pas encore commencé son apprentissage)
		\inputpartOCaml{/home/fenarius/Travail/Cours/cpge-info/docs/mp2i/files/C5/puissance.ml}{1}{\small}{1}{2}
		\begin{itemize}
			\item <2-> La définition d'une fonction commence par \kw{let}, suivi de \kw{rec} si la fonction est récursive.
			\item <3-> Remarquer l'absence des parenthèses autour des arguments, et surtout l'absence du type des variables. En effet, OCaml détecte automatiquement le type des variables utilisés (ici des entiers) par un procédé appelé \textcolor{blue}{inférence de types}.
		\end{itemize}
	\end{block}
\end{frame}

% Récursivité détour par OCaml
\begin{frame}[fragile]{\Ctitle}{\stitle}
	\begin{block}{Un petit détour par OCaml}
		La récursivité est très présente dans le paradigme de programmation fonctionnel que nous verrons en OCaml.
		On présente donc ici la version récursive des puissances en OCaml (même si nous n'avons pas encore commencé son apprentissage)
		\inputpartOCaml{/home/fenarius/Travail/Cours/cpge-info/docs/mp2i/files/C5/puissance.ml}{2}{\small}{1}{2}
		\begin{itemize}
			\item<2-> Le test d'égalité est le \textcolor{blue}{simple} {\tt =}
			\item<3-> Le mot clé \kw{then} suit le test d'égalité.
			\item<3-> Remarquer l'absence de \kw{return}.
		\end{itemize}
	\end{block}
\end{frame}


% Récursivité : petits exemples
\begin{frame}[fragile]{\Ctitle}{\stitle}
	\begin{exampleblock}{Exemple : factorielle}
		\begin{enumerate}
			\item<1-> Ecrire une fonction itérative en C permettant de calculer $n!$.
			\item<2-> Ecrire la relation liant $n!$ et $(n-1)!$.
			\item<3-> Donner une fonction récursive en C permettant de calculer $n!$.
			\item<4-> Même question en Ocaml.
		\end{enumerate}
	\end{exampleblock}
	\begin{exampleblock}{Exemple : dessin d'un triangle}
		\onslide<5->On suppose qu'on dispose déjà d'une fonction {\tt ligne} qui ne renvoie rien prend en argument un entier {\tt n} et affiche {\tt n} caractères {\tt *} suivi d'un saut de ligne.
		\begin{enumerate}
			\item<6-> Ecrire une fonction itérative en C qui prend en argument un entier $n$ et dessine un triangle de caractères
				{\tt *}comme ci-dessous ($n=4$) : \\
				{\tt ****} \\
				{\tt ***} \\
				{\tt **} \\
				{\tt *} \\
			\item<7-> Donner une version récursive de cette fonction.
		\end{enumerate}
	\end{exampleblock}
\end{frame}

\begin{frame}[fragile]{\Ctitle}{\stitle}
	\begin{exampleblock}{Factorielle : correction}
		\begin{enumerate}
			\item<1-> Version itérative  \inputpartC{/home/fenarius/Travail/Cours/cpge-info/docs/mp2i/files/C5/fact.c}{}{\footnotesize}{3}{7}
			\item<2-> $n! = n \times (n-1)!$
			\item<3-> Version récursive en C \inputpartC{/home/fenarius/Travail/Cours/cpge-info/docs/mp2i/files/C5/fact.c}{}{\footnotesize}{9}{12}
			\item<4-> Version récursive en Ocaml \inputpartOCaml{/home/fenarius/Travail/Cours/cpge-info/docs/mp2i/files/C5/fact.ml}{}{\small}{1}{2}
		\end{enumerate}
	\end{exampleblock}
\end{frame}

\begin{frame}[fragile]{\Ctitle}{\stitle}
	\begin{exampleblock}{Triangle : correction}
		\begin{enumerate}
			\item<1-> Version itérative  \inputpartC{/home/fenarius/Travail/Cours/cpge-info/docs/mp2i/files/C5/triangle.c}{}{\small}{13}{16}
			\item<2-> Version récursive \inputpartC{/home/fenarius/Travail/Cours/cpge-info/docs/mp2i/files/C5/triangle.c}{}{\small}{18}{22}
		\end{enumerate}
	\end{exampleblock}
\end{frame}

\makess{Récursivité terminale}
\begin{frame}[fragile]{\Ctitle}{\stitle}
	\begin{alertblock}{Définition}
		Une fonction est dite \textcolor{blue}{récursive terminale} lorsque chaque appel récursif est la dernière instruction a être évaluée. c'est-à-dire qu'aucun calcul n'est effectué avec le résultat de l'appel récursif.
	\end{alertblock}
	\begin{exampleblock}{Exemples}
		\onslide<2->{La fonction \kw{puissance}  vue en début de chapitre :
			\inputpartC{/home/fenarius/Travail/Cours/cpge-info/docs/mp2i/files/C5/puissance.c}{}{\small}{9}{13}}
		\onslide<3->{n'est \textcolor{BrickRed}{pas} récursive terminale, }
		\onslide<4->{en effet bien que l'appel récursif soit placé à la fin, le dernier calcul effectué par la fonction est la multiplication.}
	\end{exampleblock}
\end{frame}

\begin{frame}[fragile]{\Ctitle}{\stitle}
	\begin{exampleblock}{Exemples}
		\begin{enumerate}
			\item<1-> La fonction {\tt factorielle} ci-dessous est-elle récursive terminale ?
				\onslide<1->{\inputpartC{/home/fenarius/Travail/Cours/cpge-info/docs/mp2i/files/C5/fact.c}{}{\small}{9}{12}}
				\onslide<2->\textcolor{OliveGreen}{Non, car le dernier calcul effectué par la fonction est la multiplication.}
			\item<3-> Même question pour la fonction {\tt triangle}
				\onslide<3->{\inputpartC{/home/fenarius/Travail/Cours/cpge-info/docs/mp2i/files/C5/triangle.c}{}{\small}{18}{22}}
				\onslide<4->\textcolor{OliveGreen}{Oui, l'appel récursif est la dernière instruction à être exécutée.}
		\end{enumerate}
	\end{exampleblock}
\end{frame}

\begin{frame}[fragile]{\Ctitle}{\stitle}
	\begin{block}{Remarque}
		Dans le cas d'une récursivité non terminale, les appels successifs sont empilés dans la \textcolor{blue}{pile d'exécution}, en effet le résultat de chaque appel doit être renvoyé vers l'appel précédent. Puis être dépilés une fois le cas de base atteint.\\
		\onslide<2->{ Par exemple pour la fonction {\tt factorielle} récursive :}
		\onslide<3->{\begin{center}
				\renewcommand{\arraystretch}{1.4}
				\begin{tabularx}{5cm}{|Y|}
					\hline
					\alt<12->{\makebox[4cm]{\tt fact(\textcolor{blue}{0})=1}}{\phantom{\makebox[4cm]{\tt fact(0)=1}}}\rnode{A0}{}                 \\
					\hline
					\alt<10->{\makebox[4cm]{\tt fact(1)=1*fact(\textcolor{blue}{0})}}{\phantom{\makebox[4cm]{\tt fact(1)=1*fact(0)}}}\rnode{A1}{} \\
					\hline
					\alt<8->{\makebox[4cm]{\tt fact(2)=2*fact(\textcolor{blue}{1})}}{\phantom{\makebox[4cm]{\tt fact(2)=2*fact(1)}}}\rnode{A2}{}  \\
					\hline
					\alt<6->{\makebox[4cm]{\tt fact(3)=3*fact(\textcolor{blue}{2})}}{\phantom{\makebox[4cm]{\tt fact(3)=3*fact(2)}}}\rnode{A3}{}  \\
					\hline
					\makebox[4cm]{\tt fact(4)=4*fact(\textcolor{blue}{3})} \rnode{A4}{}                                                           \\
					\hline
					\multicolumn{1}{c}{\textcolor{BrickRed}{Pile d'appels}}
				\end{tabularx}
			\end{center}}
		\onslide<5->{\ncangle[arm=0.7cm,nodesep=0.5cm,linewidth=0.7pt,linecolor=BrickRed,offsetB=0.1cm,offsetA=0cm]{->}{A4}{A3} \nbput{\textcolor{blue}{\footnotesize appel n=3}}}
		\onslide<7->{\ncangle[arm=0.7cm,nodesep=0.5cm,linewidth=0.7pt,linecolor=BrickRed,offsetB=0.1cm,offsetA=0cm]{->}{A3}{A2} \nbput{\textcolor{blue}{\footnotesize appel n=2}}}
		\onslide<9->{\ncangle[arm=0.7cm,nodesep=0.5cm,linewidth=0.7pt,linecolor=BrickRed,offsetB=0.1cm,offsetA=0cm]{->}{A2}{A1} \nbput{\textcolor{blue}{\footnotesize appel n=1}}}
		\onslide<11->{\ncangle[arm=0.7cm,nodesep=0.5cm,linewidth=0.7pt,linecolor=BrickRed,offsetB=0.1cm,offsetA=0cm]{->}{A1}{A0} \nbput{\textcolor{blue}{\footnotesize appel n=0}}}
	\end{block}
\end{frame}


\begin{frame}[fragile]{\Ctitle}{\stitle}
	\begin{block}{Remarque}
		Dans le cas d'une récursivité non terminale, les appels successifs sont empilés dans la \textcolor{blue}{pile d'exécution}, en effet le résultat de chaque appel doit être renvoyé vers l'appel précédent. Puis être dépilés une fois le cas de base atteint.\\
		Par exemple pour la fonction {\tt factorielle} récursive :
		\begin{center}
			\renewcommand{\arraystretch}{1.4}
			\begin{tabularx}{5cm}{|Y|}
				\hline
				\alt<2->{\phantom{\makebox[4cm]{\tt fact(0)=1}}}{\makebox[4cm]{\tt fact(0)=1}}\rnode{A0}{}                 \\
				\hline
				\alt<3->{\phantom{\makebox[4cm]{\tt fact(1)=1*fact(0)}}}{\makebox[4cm]{\tt fact(1)=1*fact(0)}}\rnode{A1}{} \\
				\hline
				\alt<4->{\phantom{\makebox[4cm]{\tt fact(2)=2*fact(1)}}}{\makebox[4cm]{\tt fact(2)=2*fact(1)}}\rnode{A2}{} \\
				\hline
				\alt<5->{\phantom{\makebox[4cm]{\tt fact(3)=3*fact(2)}}}{\makebox[4cm]{\tt fact(3)=3*fact(2)}}\rnode{A3}{} \\
				\hline
				\makebox[4cm]{\tt fact(4)=4*fact(3)} \rnode{A4}{}                                                          \\
				\hline
				\multicolumn{1}{c}{\textcolor{BrickRed}{Pile d'appels}}
			\end{tabularx}
		\end{center}
		\onslide<6->{Le dépassement de capacité de la pile d'appel est le fameux \textit{stackoverflow} (débordement de pile).}
		\onslide<7->{Pour l'éviter, on peut transformer une fonction récursive non terminale en récursive terminale ce qui évite l'empilement des appels successifs.}
		\ncangle[arm=0.7cm,nodesep=0.5cm,linewidth=0.7pt,linecolor=BrickRed,offsetB=0.1cm,offsetA=0cm]{->}{A4}{A3} \nbput{\textcolor{blue}{\footnotesize appel n=3}}
		\ncangle[arm=0.7cm,nodesep=0.5cm,linewidth=0.7pt,linecolor=BrickRed,offsetB=0.1cm,offsetA=0cm]{->}{A3}{A2} \nbput{\textcolor{blue}{\footnotesize appel n=2}}
		\ncangle[arm=0.7cm,nodesep=0.5cm,linewidth=0.7pt,linecolor=BrickRed,offsetB=0.1cm,offsetA=0cm]{->}{A2}{A1} \nbput{\textcolor{blue}{\footnotesize appel n=1}}
		\ncangle[arm=0.7cm,nodesep=0.5cm,linewidth=0.7pt,linecolor=BrickRed,offsetB=0.1cm,offsetA=0cm]{->}{A1}{A0} \nbput{\textcolor{blue}{\footnotesize appel n=0}}
		\onslide<2>\ncangle[arm=0.7cm,angle=180,nodesep=4.5cm,linewidth=0.7pt,linecolor=OliveGreen,offsetB=0.1cm,offsetA=0cm]{->}{A0}{A1} \nbput{\textcolor{blue}{\footnotesize renvoie 1}}
		\onslide<3>\ncangle[arm=0.7cm,angle=180,nodesep=4.5cm,linewidth=0.7pt,linecolor=OliveGreen,offsetB=0.1cm,offsetA=0cm]{->}{A1}{A2} \nbput{\textcolor{blue}{\footnotesize renvoie 1}}
		\onslide<4>\ncangle[arm=0.7cm,angle=180,nodesep=4.5cm,linewidth=0.7pt,linecolor=OliveGreen,offsetB=0.1cm,offsetA=0cm]{->}{A2}{A3} \nbput{\textcolor{blue}{\footnotesize renvoie 2}}
		\onslide<5>\ncangle[arm=0.7cm,angle=180,nodesep=4.5cm,linewidth=0.7pt,linecolor=OliveGreen,offsetB=0.1cm,offsetA=0cm]{->}{A3}{A4} \nbput{\textcolor{blue}{\footnotesize renvoie 6}}

	\end{block}

\end{frame}

\begin{frame}{\Ctitle}{\stitle}
	\begin{exampleblock}{Exemple :  factorielle récursive terminale}
		L'idée est d'utiliser une variable auxiliaire (un \textit{accumulateur}) qui est passé en argument à l'appel récursif et dans lequel les calculs sont effectués au fur et à mesure des appels.
		\begin{itemize}
			\item<2->En C :
			\onslide<2->\inputpartC{/home/fenarius/Travail/Cours/cpge-info/docs/mp2i/files/C5/fact.c}{}{\small}{14}{17}
			\item<2->En OCaml :
			\onslide<2->\inputpartOCaml{/home/fenarius/Travail/Cours/cpge-info/docs/mp2i/files/C5/fact.ml}{}{\small}{4}{5}
		\end{itemize}
	\end{exampleblock}
\end{frame}

\begin{frame}{\Ctitle}{\stitle}
	\begin{exampleblock}{Exemple :  puissance}
		Donner une version récursive terminale de la fonction puissance :
		\begin{itemize}
			\item<2-> En C
				\onslide<3->\inputpartC{/home/fenarius/Travail/Cours/cpge-info/docs/mp2i/files/C5/puissance.c}{}{\small}{15}{19}
			\item<4-> En OCaml
				\onslide<4->\inputpartOCaml{/home/fenarius/Travail/Cours/cpge-info/docs/mp2i/files/C5/puissance.ml}{}{\small}{4}{5}
		\end{itemize}
	\end{exampleblock}
\end{frame}

\begin{frame}{\Ctitle}{\stitle}
	\begin{block}{Remarque}
		\begin{itemize}
			\item<1-> La version récursive terminale ne provoquera pas de débordement de pile en OCaml car elle sera automatiquement optimisée par le compilateur.
			\item<2-> En C, on utilisera l'option d'optimisation \kw{-O2} pour que cela soit le cas.
			\item<3-> Le langage Ocaml est plus "adapté" à une programmation fonctionnel (et donc à la récursivité) que le C qui permet la récursivité mais représente avant tout le paradigme impératif.
		\end{itemize}
	\end{block}
\end{frame}

\makess{Récursivité croisée}
\begin{frame}{\Ctitle}{\stitle}
	\begin{alertblock}{Définition}
		Lorsqu'on définit simultanément plusieurs fonctions qui s'appellent mutuellement on parle de \textcolor{blue}{récursivité croisée} (ou de récursivité mutuelle).
	\end{alertblock}
	\onslide<2->{
		\begin{exampleblock}{Exemple : moyenne arithmético-géométrique}
			Etant donné $a \in R^+, b \in \R^+$, on définit les suites : \\
			\begin{tabularx}{\linewidth}{XX}
				$\left\{ \begin{array}{lcl} u_0 &=& a \\ u_{n+1} &=& \sqrt{u_nv_n}\\ \end{array}\right.$ &
				$\left\{ \begin{array}{lcl} v_0 &=& b \\ v_{n+1} &=& \dfrac{u_n+v_n}{2}\\ \end{array}\right.$ \\
			\end{tabularx}
			Ecrire deux fonctions mutuellement récursives {\tt un} et {\tt vn} permettant de calculer les termes de ces deux suites.\\
			\onslide<3->{\textcolor{BrickRed}{\small \danger} En C, une fonction doit être définie avant d'y faire appel. On doit donc dans le cas d'une récursivité croisée donner la signature d'une des deux fonctions avant la définition complète de l'autre.}\\
			\onslide<4->{ En OCaml, on définira les deux fonctions dans le même "{\tt let}" en les séparant par \kw{and}.}
		\end{exampleblock}}
\end{frame}

\begin{frame}[fragile]{\Ctitle}{\stitle}
	\begin{exampleblock}{Correction : En C}
		\inputpartC{/home/fenarius/Travail/Cours/cpge-info/docs/mp2i/files/C5/arithgeo.c}{}{\small}{5}{15}
	\end{exampleblock}
\end{frame}

\begin{frame}[fragile]{\Ctitle}{\stitle}
	\begin{exampleblock}{Correction : En OCaml}
		\inputpartC{/home/fenarius/Travail/Cours/cpge-info/docs/mp2i/files/C5/arithgeo.ml}{}{\small}{1}{5}
		\onslide<2->On remarquera \kw{+.}, \kw{*.}, et \kw{/.} qui correspondent aux opérations sur les flottants (inférence de type)
	\end{exampleblock}
\end{frame}


\makess{Chevauchement des appels récursifs}
\begin{frame}{\Ctitle}{\stitle}
	\begin{exampleblock}{Exemple introductif}
		\begin{enumerate}
			\item<1-> Ecrire une fonction récursive qui prend en argument un entier $n$ et renvoie le $n$ième terme de la suite de Fibonacci défini par :
				$\left\{ \begin{array}{lll}
						f_0   & = & 0,                                                  \\
						f_1   & = & 1,                                                  \\
						f_{n} & = & f_{n-1}+f_{n-2} \mathrm{\ \ pour\ tout\ \ } n\geq2.\end{array} \right.$
			\item<2-> Tracer le graphe des appels récursifs de cette fonction pour $n=5$
			\item<3-> Commenter
		\end{enumerate}
	\end{exampleblock}
\end{frame}



\begin{frame}{\Ctitle}{\stitle}
	\begin{exampleblock}{Correction question 1}
		\begin{itemize}
			\item <1-> En C
			\inputpartC{/home/fenarius/Travail/Cours/cpge-info/docs/mp2i/files/C5/fibo_rec.c}{}{\small}{3}{6}
			\item<2-> En OCaml
			\inputpartOCaml{/home/fenarius/Travail/Cours/cpge-info/docs/mp2i/files/C5/fibo_rec.ml}{}{\small}{1}{2}
		\end{itemize}
		
	\end{exampleblock}
\end{frame}

\begin{frame}{\Ctitle}{\stitle}
	\begin{exampleblock}{Correction questions 2}
		\begin{center}
			\psset{levelsep=1cm,treesep=0.2cm,linecolor=OliveGreen,linewidth=0.6pt}
			\pstree{\Toval{\tiny \tt fibo(5)}}{
				\pstree{\Toval{\tiny \tt fibo(4)}}{
					\pstree{\Toval{\tiny \tt  fibo(3)}}{
						\pstree{\Toval{\textcolor{BrickRed}{\tt \tiny fibo(2)}}}{\Toval{\tt \tiny fibo(1)} \Toval{\tiny \tt  fibo(0)}}
						\Toval{\tiny \tt  fibo(1)}}
					\pstree{\Toval{\textcolor{BrickRed}{\tt \tiny fibo(2)}}}{\Toval{\tiny  \tt fibo(1)} \Toval{\tiny \tt  fibo(0)}}
				}
				\pstree{\Toval{\tiny \tt fibo(3)}}{
					\pstree{\Toval{\textcolor{BrickRed}{\tt \tiny fibo(2)}}}{\Toval{\tiny \tt  fibo(1)} \Toval{\tiny \tt  fibo(0)}}
					\Toval{\tiny \tt  fibo(1)}}
			}
		\end{center}
	\end{exampleblock}
\end{frame}


\begin{frame}{\Ctitle}{\stitle}
	\begin{exampleblock}{Correction questions 3}
		\begin{center}
			\psset{levelsep=1cm,treesep=0.2cm,linecolor=OliveGreen,linewidth=0.6pt}
			\pstree{\Toval{\tiny \tt fibo(5)}}{
				\pstree{\Toval{\tiny \tt fibo(4)}}{
					\pstree{\Toval{\tiny \tt  fibo(3)}}{
						\pstree{\Toval{\textcolor{BrickRed}{\tt \tiny fibo(2)}}}{\Toval{\tt \tiny fibo(1)} \Toval{\tiny \tt  fibo(0)}}
						\Toval{\tiny \tt  fibo(1)}}
					\pstree{\Toval{\textcolor{BrickRed}{\tt \tiny fibo(2)}}}{\Toval{\tiny  \tt fibo(1)} \Toval{\tiny \tt  fibo(0)}}
				}
				\pstree{\Toval{\tiny \tt fibo(3)}}{
					\pstree{\Toval{\textcolor{BrickRed}{\tt \tiny fibo(2)}}}{\Toval{\tiny \tt  fibo(1)} \Toval{\tiny \tt  fibo(0)}}
					\Toval{\tiny \tt  fibo(1)}}
			}
		\end{center} \medskip
		{\small On calcule à plusieurs reprises les \textit{mêmes valeurs}, ici par exemple \textcolor{BrickRed}{\tt fibo(2)} est calculé à trois reprises.}
	\end{exampleblock}
\end{frame}

\begin{frame}[fragile]{\Ctitle}{\stitle}
	\begin{block}{Remarque}
		\begin{itemize}
			\item<1-> La solution récursive est ici simple à mettre en oeuvre (on traduit directement la définition mathématique de la suite)
			\item<2-> L'exécution est problématique car on effectue plusieurs fois les mêmes appels récursifs.
			\item<3-> Pour pallier ce problème :
				\begin{itemize}
					\item<4-> On peut stocker les résultats des appels antérieurs dans une structure de données adaptées (c'est le principe de \textcolor{blue}{mémoïsation})
					\item<5-> On peut utiliser une formulation itérative.
				\end{itemize}
		\end{itemize}
	\end{block}
\end{frame}

\makess{Une technique de programmation élégante}
\begin{frame}[fragile]{\Ctitle}{\stitle}
	\begin{block}{Remarques}
		\begin{itemize}
			\item<1->Bien que la programmation récursive soit parfois gourmande en ressource (débordement de pile, chevauchement d'appels récursifs). Certains problèmes ont une solution récursive très lisible et rapide à programmer.
			\item<2-> La formulation récursive est donc parfois \og plus adaptée \fg{} à un problème et constitue une façon élégante et concise de programmer une résolution.
			\item<3-> On présente ici un exemple simple mais d'autres seront vus en TP (tours de Hanoï, dessins récursifs)
		\end{itemize}
	\end{block}
\end{frame}

\begin{frame}[fragile]{\Ctitle}{\stitle}
	\begin{exampleblock}{Exemple : exponentiation rapide}
		\begin{enumerate}
			\item<1-> Combien faut-il faire de multiplications pour calculer $a^{13}$ avec la fonction
				\inputpartC{/home/fenarius/Travail/Cours/cpge-info/docs/mp2i/files/C5/puissance.c}{}{\footnotesize}{3}{7}
			\item<2-> Combien en faut-il si on procède de la façon suivante :
				\begin{itemize}
					\item<3-> Calculer $a^6$, l'élever au carré et le multiplier par $a$.
					\item<4-> Pour calculer $a^6$,calculer $a^3$ et l'élever au carré.
					\item<5-> Pour calculer $a^3$, élever $a$ au carré et multiplier par $a$.
				\end{itemize}
			\item<6-> Généraliser la méthode précédente au cas d'un exposant quelconque et en déduire une relation de récurrence entre $a^n$ et $a^\frac{n}{2}$ si $n$ et pair et $a^\frac{n-1}{2}$ sinon.
			\item<7-> Proposer une implémentation récursive en C puis en Ocaml.
			\item<8-> Que pensez-vous d'une implémentation itérative ?
		\end{enumerate}
	\end{exampleblock}
\end{frame}


\begin{frame}[fragile]{\Ctitle}{\stitle}
	\begin{exampleblock}{Exemple : exponentiation rapide}
		\begin{enumerate}
			\item<1-> \textcolor{OliveGreen}{Il faut faire 13 multiplications, puisque $a^{13}$ est calculé avec :\\
					$a^{13} = 1 \textcolor{BrickRed}{\times} a \textcolor{BrickRed}{\times} a\textcolor{BrickRed}{\times} a\textcolor{BrickRed}{\times} a\textcolor{BrickRed}{\times} a\textcolor{BrickRed}{\times} a\textcolor{BrickRed}{\times} a\textcolor{BrickRed}{\times} a\textcolor{BrickRed}{\times} a\textcolor{BrickRed}{\times} a\textcolor{BrickRed}{\times} a\textcolor{BrickRed}{\times} a\textcolor{BrickRed}{\times} a$}
			\item<2-> \textcolor{OliveGreen}{Dans ce cas, il ne faut que 5 multiplications en effet, on calcul $a^{13}$ avec : \\
					$a^{13} = \left( \left(a^{\textcolor{BrickRed}{2}} \textcolor{BrickRed}{\times} a \right)^{\textcolor{BrickRed}{2}} \right)^{\textcolor{BrickRed}{2}} \textcolor{BrickRed}{\times} a$}
			\item<3-> \textcolor{OliveGreen}{$\left\{ \begin{array}{lll}
							a^n & = & \left(a^\frac{n}{2}\right)^2, \ \mathrm{si\ } n  \mathrm{\ est\ paire} \\
							a^n & = & \left(a^\frac{n-1}{2}\right)^2\times a, \ \mathrm{sinon\ }\end{array} \right. $}
		\end{enumerate}
	\end{exampleblock}
\end{frame}

\begin{frame}[fragile]{\Ctitle}{\stitle}
	\begin{exampleblock}{Exponentiation rapide}
		\begin{enumerate}
			\addtocounter{enumi}{3}
			\item<1-> \textcolor{OliveGreen}{Implémentation en C :}
				\inputpartC{/home/fenarius/Travail/Cours/cpge-info/docs/mp2i/files/C5/puissance.c}{}{\footnotesize}{21}{29}
			\item<2-> \textcolor{OliveGreen}{Implémentation en OCaml :}
			\inputpartOCaml{/home/fenarius/Travail/Cours/cpge-info/docs/mp2i/files/C5/puissance.ml}{}{\footnotesize}{7}{11}
		\end{enumerate}
	\end{exampleblock}
\end{frame}
\end{document}
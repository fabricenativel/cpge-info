\PassOptionsToPackage{dvipsnames,table}{xcolor}
\documentclass[10pt]{beamer}
\usepackage{Cours}

\begin{document}

\input{\detokenize{/home/fenarius/Travail/Cours/cpge-info/latex//MacrosCours.tex}}

% Numéro et titre de chapitre
\setcounter{numchap}{5}
\newcommand{\Ctitle}{\cnum {Récursivité}}


\makess{Définition et exemples}
\begin{frame}{\Ctitle}{\stitle}
	\begin{alertblock}{Définition}
		\onslide<2->{En informatique, on dit qu'une fonction est \textcolor{red}{récursive},}
		\onslide<3->{lorsque cette fonction fait appel à elle-même.}
	\end{alertblock}
	\onslide<4->{
		\begin{block}{Remarques}
			\begin{itemize}
				\item<5-> Une fonction récursive permet donc, \textit{comme une boucle}, de répéter des instructions. Une même fonction peut donc souvent se programmer de façon \textcolor{blue}{itérative} (avec des boucles) ou de façon \textcolor{blue}{récursive} (en s'appelant elle-même).
				\item<6-> Une fonction récursive doit toujours \textcolor{blue}{contenir une condition d'arrêt}, dans le cas contraire elle s'appelle elle-même à l'infini et le programme ne se termine jamais.
				\item<7-> Les valeurs passées en paramètres lors des appels successifs doivent être différents, sinon la fonction s'exécute à l'identique à chaque appel et donc boucle à l'infini.
			\end{itemize}
		\end{block}}
\end{frame}

% Récursivité : Exemples des puissances
\begin{frame}{\Ctitle}{\stitle}
	\begin{exampleblock}{Exemple : les puissances positives}
		En mathématiques, pour un nombre quelconque $a$ et un entier positif $n$, on définit $a$ puissance $n$ par :\\
		$ a^n = a \times a \times \dots \times a $, et on convient que $a^0=1$
		\begin{itemize}
			\item<2->{Définir une fonction {\tt puissance} qui prend en argument un flottant $a$ et un entier $n$ et renvoie $a^n$ en effectuant ce calcul de façon itératif}
			\item<3->{Recopier et compléter : $a^n = \dots \times a^{\dots}$}
			\item<4->{En déduire une version récursive de la fonction calculant les puissances}
		\end{itemize}
	\end{exampleblock}
\end{frame}

% Récursivité : Exemples des puissances
\begin{frame}[fragile]{\Ctitle}{\stitle}
	\begin{exampleblock}{Exemple : les puissances positives}
		\begin{itemize}
			\item<1-> \textcolor{OliveGreen}{Puissance : version itérative}
\begin{langageC}
float puissance_iteratif(float a, int n):
    p=1
    for (int k=1;k<n;k=k+1){
        p=p*a;}
    return p;
\end{langageC}
			\item<2-> $a^n = \textcolor{OliveGreen}{a} \times a^{\textcolor{OliveGreen}{n-1}}$
			\item<3-> \textcolor{OliveGreen}{Puissance : version récursive}
			      \begin{langageC}
float puissance_recursif(float a, int n):
    if n==0:
        return 1;
    else:
        return a*puissance_recursif(a,n-1);
\end{langageC}
		\end{itemize}
	\end{exampleblock}
\end{frame}


% Récursivité : exemple de factorielle
\begin{frame}[fragile]{\Ctitle}{\stitle}
    \begin{exampleblock}{Exemple : factorielle}
        \begin{enumerate}
            \item<1-> Ecrire une fonction itérative en C permettant de calculer $n!$.
            \item<2-> Ecrire la relation liant $n!$ et $(n-1)$.
            \item<3-> Donner une fonction récursive permettant de calculer $n!$
        \end{enumerate}
    \end{exampleblock}
    \begin{exampleblock}{Exemple : dessin d'un triangle}
        \begin{enumerate}
            \item<4-> Ecrire une fonction itérative en C qui prend en argument un entier $n$ et dessine un triangle de caractères {\tt *}comme ci-dessous ($n=4$) :\begin{verbatim}
                *
                ** 
                *** 
                ****
            \end{verbatim}
            \item<5-> Donner une version récursive de cette fonction.
        \end{enumerate}
    \end{exampleblock}
\end{frame}
\end{document}
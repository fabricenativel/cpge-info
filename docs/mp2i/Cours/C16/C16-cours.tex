\PassOptionsToPackage{dvipsnames,table}{xcolor}
\documentclass[10pt]{beamer}
\usepackage{Cours}
\DeclareMathAlphabet{\mymathbb}{U}{BOONDOX-ds}{m}{n}
\begin{document}



\newcounter{numchap}
\setcounter{numchap}{1}
\newcounter{numframe}
\setcounter{numframe}{0}
\newcommand{\mframe}[1]{\frametitle{#1} \addtocounter{numframe}{1}}
\newcommand{\cnum}{\fbox{\textcolor{yellow}{\textbf{C\thenumchap}}}~}
\newcommand{\makess}[1]{\section{#1} \label{ss\thesection}}
\newcommand{\stitle}{\textcolor{yellow}{\textbf{\thesection. \nameref{ss\thesection}}}}

\definecolor{codebg}{gray}{0.90}
\definecolor{grispale}{gray}{0.95}
\definecolor{fluo}{rgb}{1,0.96,0.62}
\newminted[langageC]{c}{linenos=true,escapeinside=||,highlightcolor=fluo,tabsize=2,breaklines=true}
\newminted[codepython]{python}{linenos=true,escapeinside=||,highlightcolor=fluo,tabsize=2,breaklines=true}
% Inclusion complète (ou partiel en indiquant premiere et dernière ligne) d'un fichier C
\newcommand{\inputC}[3]{\begin{mdframed}[backgroundcolor=codebg] \inputminted[breaklines=true,fontsize=#3,linenos=true,highlightcolor=fluo,tabsize=2,highlightlines={#2}]{c}{#1} \end{mdframed}}
\newcommand{\inputpartC}[5]{\begin{mdframed}[backgroundcolor=codebg] \inputminted[breaklines=true,fontsize=#3,linenos=true,highlightcolor=fluo,tabsize=2,highlightlines={#2},firstline=#4,lastline=#5,firstnumber=1]{c}{#1} \end{mdframed}}
\newcommand{\inputpython}[3]{\begin{mdframed}[backgroundcolor=codebg] \inputminted[breaklines=true,fontsize=#3,linenos=true,highlightcolor=fluo,tabsize=2,highlightlines={#2}]{python}{#1} \end{mdframed}}
\newcommand{\inputpartOCaml}[5]{\begin{mdframed}[backgroundcolor=codebg] \inputminted[breaklines=true,fontsize=#3,linenos=true,highlightcolor=fluo,tabsize=2,highlightlines={#2},firstline=#4,lastline=#5,firstnumber=1]{OCaml}{#1} \end{mdframed}}
\BeforeBeginEnvironment{minted}{\begin{mdframed}[backgroundcolor=codebg]}
\AfterEndEnvironment{minted}{\end{mdframed}}
\newcommand{\kw}[1]{\textcolor{blue}{\tt #1}}

\newtcolorbox{rcadre}[4]{halign=center,colback={#1},colframe={#2},width={#3cm},height={#4cm},valign=center,boxrule=1pt,left=0pt,right=0pt}
\newtcolorbox{cadre}[4]{halign=center,colback={#1},colframe={#2},arc=0mm,width={#3cm},height={#4cm},valign=center,boxrule=1pt,left=0pt,right=0pt}
\newcommand{\myem}[1]{\colorbox{fluo}{#1}}
\mdfsetup{skipabove=1pt,skipbelow=-2pt}



% Noeud dans un cadre pour les arbres
\newcommand{\noeud}[2]{\Tr{\fbox{\textcolor{#1}{\tt #2}}}}

\newcommand{\htmlmode}{\lstset{language=html,numbers=left, tabsize=4, frame=single, breaklines=true, keywordstyle=\ttfamily, basicstyle=\small,
   numberstyle=\tiny\ttfamily, framexleftmargin=0mm, backgroundcolor=\color{grispale}, xleftmargin=12mm,showstringspaces=false}}
\newcommand{\pythonmode}{\lstset{
   language=python,
   linewidth=\linewidth,
   numbers=left,
   tabsize=4,
   frame=single,
   breaklines=true,
   keywordstyle=\ttfamily\color{blue},
   basicstyle=\small,
   numberstyle=\tiny\ttfamily,
   framexleftmargin=-2mm,
   numbersep=-0.5mm,
   backgroundcolor=\color{codebg},
   xleftmargin=-1mm, 
   showstringspaces=false,
   commentstyle=\color{gray},
   stringstyle=\color{OliveGreen},
   emph={turtle,Screen,Turtle},
   emphstyle=\color{RawSienna},
   morekeywords={setheading,goto,backward,forward,left,right,pendown,penup,pensize,color,speed,hideturtle,showturtle,forward}}
   }
   \newcommand{\Cmode}{\lstset{
      language=[ANSI]C,
      linewidth=\linewidth,
      numbers=left,
      tabsize=4,
      frame=single,
      breaklines=true,
      keywordstyle=\ttfamily\color{blue},
      basicstyle=\small,
      numberstyle=\tiny\ttfamily,
      framexleftmargin=0mm,
      numbersep=2mm,
      backgroundcolor=\color{codebg},
      xleftmargin=0mm, 
      showstringspaces=false,
      commentstyle=\color{gray},
      stringstyle=\color{OliveGreen},
      emphstyle=\color{RawSienna},
      escapechar=\|,
      morekeywords={}}
      }
\newcommand{\bashmode}{\lstset{language=bash,numbers=left, tabsize=2, frame=single, breaklines=true, basicstyle=\ttfamily,
   numberstyle=\tiny\ttfamily, framexleftmargin=0mm, backgroundcolor=\color{grispale}, xleftmargin=12mm, showstringspaces=false}}
\newcommand{\exomode}{\lstset{language=python,numbers=left, tabsize=2, frame=single, breaklines=true, basicstyle=\ttfamily,
   numberstyle=\tiny\ttfamily, framexleftmargin=13mm, xleftmargin=12mm, basicstyle=\small, showstringspaces=false}}
   
   
  
%tei pour placer les images
%tei{nom de l’image}{échelle de l’image}{sens}{texte a positionner}
%sens ="1" (droite) ou "2" (gauche)
\newlength{\ltxt}
\newcommand{\tei}[4]{
\setlength{\ltxt}{\linewidth}
\setbox0=\hbox{\includegraphics[scale=#2]{#1}}
\addtolength{\ltxt}{-\wd0}
\addtolength{\ltxt}{-10pt}
\ifthenelse{\equal{#3}{1}}{
\begin{minipage}{\wd0}
\includegraphics[scale=#2]{#1}
\end{minipage}
\hfill
\begin{minipage}{\ltxt}
#4
\end{minipage}
}{
\begin{minipage}{\ltxt}
#4
\end{minipage}
\hfill
\begin{minipage}{\wd0}
\includegraphics[scale=#2]{#1}
\end{minipage}
}
}

%Juxtaposition d'une image pspciture et de texte 
%#1: = code pstricks de l'image
%#2: largeur de l'image
%#3: hauteur de l'image
%#4: Texte à écrire
\newcommand{\ptp}[4]{
\setlength{\ltxt}{\linewidth}
\addtolength{\ltxt}{-#2 cm}
\addtolength{\ltxt}{-0.1 cm}
\begin{minipage}[b][#3 cm][t]{\ltxt}
#4
\end{minipage}\hfill
\begin{minipage}[b][#3 cm][c]{#2 cm}
#1
\end{minipage}\par
}



%Macros pour les graphiques
\psset{linewidth=0.5\pslinewidth,PointSymbol=x}
\setlength{\fboxrule}{0.5pt}
\newcounter{tempangle}

%Marque la longueur du segment d'extrémité  #1 et  #2 avec la valeur #3, #4 est la distance par rapport au segment (en %age de la valeur de celui ci) et #5 l'orientation du marquage : +90 ou -90
\newcommand{\afflong}[5]{
\pstRotation[RotAngle=#4,PointSymbol=none,PointName=none]{#1}{#2}[X] 
\pstHomO[PointSymbol=none,PointName=none,HomCoef=#5]{#1}{X}[Y]
\pstTranslation[PointSymbol=none,PointName=none]{#1}{#2}{Y}[Z]
 \ncline{|<->|,linewidth=0.25\pslinewidth}{Y}{Z} \ncput*[nrot=:U]{\footnotesize{#3}}
}
\newcommand{\afflongb}[3]{
\ncline{|<->|,linewidth=0}{#1}{#2} \naput*[nrot=:U]{\footnotesize{#3}}
}

%Construis le point #4 situé à #2 cm du point #1 avant un angle #3 par rapport à l'horizontale. #5 = liste de paramètre
\newcommand{\lsegment}[5]{\pstGeonode[PointSymbol=none,PointName=none](0,0){O'}(#2,0){I'} \pstTranslation[PointSymbol=none,PointName=none]{O'}{I'}{#1}[J'] \pstRotation[RotAngle=#3,PointSymbol=x,#5]{#1}{J'}[#4]}
\newcommand{\tsegment}[5]{\pstGeonode[PointSymbol=none,PointName=none](0,0){O'}(#2,0){I'} \pstTranslation[PointSymbol=none,PointName=none]{O'}{I'}{#1}[J'] \pstRotation[RotAngle=#3,PointSymbol=x,#5]{#1}{J'}[#4] \pstLineAB{#4}{#1}}

%Construis le point #4 situé à #3 cm du point #1 et faisant un angle de  90° avec la droite (#1,#2) #5 = liste de paramètre
\newcommand{\psegment}[5]{
\pstGeonode[PointSymbol=none,PointName=none](0,0){O'}(#3,0){I'}
 \pstTranslation[PointSymbol=none,PointName=none]{O'}{I'}{#1}[J']
 \pstInterLC[PointSymbol=none,PointName=none]{#1}{#2}{#1}{J'}{M1}{M2} \pstRotation[RotAngle=-90,PointSymbol=x,#5]{#1}{M1}[#4]
  }
  
%Construis le point #4 situé à #3 cm du point #1 et faisant un angle de  #5° avec la droite (#1,#2) #6 = liste de paramètre
\newcommand{\mlogo}[6]{
\pstGeonode[PointSymbol=none,PointName=none](0,0){O'}(#3,0){I'}
 \pstTranslation[PointSymbol=none,PointName=none]{O'}{I'}{#1}[J']
 \pstInterLC[PointSymbol=none,PointName=none]{#1}{#2}{#1}{J'}{M1}{M2} \pstRotation[RotAngle=#5,PointSymbol=x,#6]{#1}{M2}[#4]
  }

% Construis un triangle avec #1=liste des 3 sommets séparés par des virgules, #2=liste des 3 longueurs séparés par des virgules, #3 et #4 : paramètre d'affichage des 2e et 3 points et #5 : inclinaison par rapport à l'horizontale
%autre macro identique mais sans tracer les segments joignant les sommets
\noexpandarg
\newcommand{\Triangleccc}[5]{
\StrBefore{#1}{,}[\pointA]
\StrBetween[1,2]{#1}{,}{,}[\pointB]
\StrBehind[2]{#1}{,}[\pointC]
\StrBefore{#2}{,}[\coteA]
\StrBetween[1,2]{#2}{,}{,}[\coteB]
\StrBehind[2]{#2}{,}[\coteC]
\tsegment{\pointA}{\coteA}{#5}{\pointB}{#3} 
\lsegment{\pointA}{\coteB}{0}{Z1}{PointSymbol=none, PointName=none}
\lsegment{\pointB}{\coteC}{0}{Z2}{PointSymbol=none, PointName=none}
\pstInterCC{\pointA}{Z1}{\pointB}{Z2}{\pointC}{Z3} 
\pstLineAB{\pointA}{\pointC} \pstLineAB{\pointB}{\pointC}
\pstSymO[PointName=\pointC,#4]{C}{C}[C]
}
\noexpandarg
\newcommand{\TrianglecccP}[5]{
\StrBefore{#1}{,}[\pointA]
\StrBetween[1,2]{#1}{,}{,}[\pointB]
\StrBehind[2]{#1}{,}[\pointC]
\StrBefore{#2}{,}[\coteA]
\StrBetween[1,2]{#2}{,}{,}[\coteB]
\StrBehind[2]{#2}{,}[\coteC]
\tsegment{\pointA}{\coteA}{#5}{\pointB}{#3} 
\lsegment{\pointA}{\coteB}{0}{Z1}{PointSymbol=none, PointName=none}
\lsegment{\pointB}{\coteC}{0}{Z2}{PointSymbol=none, PointName=none}
\pstInterCC[PointNameB=none,PointSymbolB=none,#4]{\pointA}{Z1}{\pointB}{Z2}{\pointC}{Z1} 
}


% Construis un triangle avec #1=liste des 3 sommets séparés par des virgules, #2=liste formée de 2 longueurs et d'un angle séparés par des virgules, #3 et #4 : paramètre d'affichage des 2e et 3 points et #5 : inclinaison par rapport à l'horizontale
%autre macro identique mais sans tracer les segments joignant les sommets
\newcommand{\Trianglecca}[5]{
\StrBefore{#1}{,}[\pointA]
\StrBetween[1,2]{#1}{,}{,}[\pointB]
\StrBehind[2]{#1}{,}[\pointC]
\StrBefore{#2}{,}[\coteA]
\StrBetween[1,2]{#2}{,}{,}[\coteB]
\StrBehind[2]{#2}{,}[\angleA]
\tsegment{\pointA}{\coteA}{#5}{\pointB}{#3} 
\setcounter{tempangle}{#5}
\addtocounter{tempangle}{\angleA}
\tsegment{\pointA}{\coteB}{\thetempangle}{\pointC}{#4}
\pstLineAB{\pointB}{\pointC}
}
\newcommand{\TriangleccaP}[5]{
\StrBefore{#1}{,}[\pointA]
\StrBetween[1,2]{#1}{,}{,}[\pointB]
\StrBehind[2]{#1}{,}[\pointC]
\StrBefore{#2}{,}[\coteA]
\StrBetween[1,2]{#2}{,}{,}[\coteB]
\StrBehind[2]{#2}{,}[\angleA]
\lsegment{\pointA}{\coteA}{#5}{\pointB}{#3} 
\setcounter{tempangle}{#5}
\addtocounter{tempangle}{\angleA}
\lsegment{\pointA}{\coteB}{\thetempangle}{\pointC}{#4}
}

% Construis un triangle avec #1=liste des 3 sommets séparés par des virgules, #2=liste formée de 1 longueurs et de deux angle séparés par des virgules, #3 et #4 : paramètre d'affichage des 2e et 3 points et #5 : inclinaison par rapport à l'horizontale
%autre macro identique mais sans tracer les segments joignant les sommets
\newcommand{\Trianglecaa}[5]{
\StrBefore{#1}{,}[\pointA]
\StrBetween[1,2]{#1}{,}{,}[\pointB]
\StrBehind[2]{#1}{,}[\pointC]
\StrBefore{#2}{,}[\coteA]
\StrBetween[1,2]{#2}{,}{,}[\angleA]
\StrBehind[2]{#2}{,}[\angleB]
\tsegment{\pointA}{\coteA}{#5}{\pointB}{#3} 
\setcounter{tempangle}{#5}
\addtocounter{tempangle}{\angleA}
\lsegment{\pointA}{1}{\thetempangle}{Z1}{PointSymbol=none, PointName=none}
\setcounter{tempangle}{#5}
\addtocounter{tempangle}{180}
\addtocounter{tempangle}{-\angleB}
\lsegment{\pointB}{1}{\thetempangle}{Z2}{PointSymbol=none, PointName=none}
\pstInterLL[#4]{\pointA}{Z1}{\pointB}{Z2}{\pointC}
\pstLineAB{\pointA}{\pointC}
\pstLineAB{\pointB}{\pointC}
}
\newcommand{\TrianglecaaP}[5]{
\StrBefore{#1}{,}[\pointA]
\StrBetween[1,2]{#1}{,}{,}[\pointB]
\StrBehind[2]{#1}{,}[\pointC]
\StrBefore{#2}{,}[\coteA]
\StrBetween[1,2]{#2}{,}{,}[\angleA]
\StrBehind[2]{#2}{,}[\angleB]
\lsegment{\pointA}{\coteA}{#5}{\pointB}{#3} 
\setcounter{tempangle}{#5}
\addtocounter{tempangle}{\angleA}
\lsegment{\pointA}{1}{\thetempangle}{Z1}{PointSymbol=none, PointName=none}
\setcounter{tempangle}{#5}
\addtocounter{tempangle}{180}
\addtocounter{tempangle}{-\angleB}
\lsegment{\pointB}{1}{\thetempangle}{Z2}{PointSymbol=none, PointName=none}
\pstInterLL[#4]{\pointA}{Z1}{\pointB}{Z2}{\pointC}
}

%Construction d'un cercle de centre #1 et de rayon #2 (en cm)
\newcommand{\Cercle}[2]{
\lsegment{#1}{#2}{0}{Z1}{PointSymbol=none, PointName=none}
\pstCircleOA{#1}{Z1}
}

%construction d'un parallélogramme #1 = liste des sommets, #2 = liste contenant les longueurs de 2 côtés consécutifs et leurs angles;  #3, #4 et #5 : paramètre d'affichage des sommets #6 inclinaison par rapport à l'horizontale 
% meme macro sans le tracé des segements
\newcommand{\Para}[6]{
\StrBefore{#1}{,}[\pointA]
\StrBetween[1,2]{#1}{,}{,}[\pointB]
\StrBetween[2,3]{#1}{,}{,}[\pointC]
\StrBehind[3]{#1}{,}[\pointD]
\StrBefore{#2}{,}[\longueur]
\StrBetween[1,2]{#2}{,}{,}[\largeur]
\StrBehind[2]{#2}{,}[\angle]
\tsegment{\pointA}{\longueur}{#6}{\pointB}{#3} 
\setcounter{tempangle}{#6}
\addtocounter{tempangle}{\angle}
\tsegment{\pointA}{\largeur}{\thetempangle}{\pointD}{#5}
\pstMiddleAB[PointName=none,PointSymbol=none]{\pointB}{\pointD}{Z1}
\pstSymO[#4]{Z1}{\pointA}[\pointC]
\pstLineAB{\pointB}{\pointC}
\pstLineAB{\pointC}{\pointD}
}
\newcommand{\ParaP}[6]{
\StrBefore{#1}{,}[\pointA]
\StrBetween[1,2]{#1}{,}{,}[\pointB]
\StrBetween[2,3]{#1}{,}{,}[\pointC]
\StrBehind[3]{#1}{,}[\pointD]
\StrBefore{#2}{,}[\longueur]
\StrBetween[1,2]{#2}{,}{,}[\largeur]
\StrBehind[2]{#2}{,}[\angle]
\lsegment{\pointA}{\longueur}{#6}{\pointB}{#3} 
\setcounter{tempangle}{#6}
\addtocounter{tempangle}{\angle}
\lsegment{\pointA}{\largeur}{\thetempangle}{\pointD}{#5}
\pstMiddleAB[PointName=none,PointSymbol=none]{\pointB}{\pointD}{Z1}
\pstSymO[#4]{Z1}{\pointA}[\pointC]
}


%construction d'un cerf-volant #1 = liste des sommets, #2 = liste contenant les longueurs de 2 côtés consécutifs et leurs angles;  #3, #4 et #5 : paramètre d'affichage des sommets #6 inclinaison par rapport à l'horizontale 
% meme macro sans le tracé des segements
\newcommand{\CerfVolant}[6]{
\StrBefore{#1}{,}[\pointA]
\StrBetween[1,2]{#1}{,}{,}[\pointB]
\StrBetween[2,3]{#1}{,}{,}[\pointC]
\StrBehind[3]{#1}{,}[\pointD]
\StrBefore{#2}{,}[\longueur]
\StrBetween[1,2]{#2}{,}{,}[\largeur]
\StrBehind[2]{#2}{,}[\angle]
\tsegment{\pointA}{\longueur}{#6}{\pointB}{#3} 
\setcounter{tempangle}{#6}
\addtocounter{tempangle}{\angle}
\tsegment{\pointA}{\largeur}{\thetempangle}{\pointD}{#5}
\pstOrtSym[#4]{\pointB}{\pointD}{\pointA}[\pointC]
\pstLineAB{\pointB}{\pointC}
\pstLineAB{\pointC}{\pointD}
}

%construction d'un quadrilatère quelconque #1 = liste des sommets, #2 = liste contenant les longueurs des 4 côtés et l'angle entre 2 cotés consécutifs  #3, #4 et #5 : paramètre d'affichage des sommets #6 inclinaison par rapport à l'horizontale 
% meme macro sans le tracé des segements
\newcommand{\Quadri}[6]{
\StrBefore{#1}{,}[\pointA]
\StrBetween[1,2]{#1}{,}{,}[\pointB]
\StrBetween[2,3]{#1}{,}{,}[\pointC]
\StrBehind[3]{#1}{,}[\pointD]
\StrBefore{#2}{,}[\coteA]
\StrBetween[1,2]{#2}{,}{,}[\coteB]
\StrBetween[2,3]{#2}{,}{,}[\coteC]
\StrBetween[3,4]{#2}{,}{,}[\coteD]
\StrBehind[4]{#2}{,}[\angle]
\tsegment{\pointA}{\coteA}{#6}{\pointB}{#3} 
\setcounter{tempangle}{#6}
\addtocounter{tempangle}{\angle}
\tsegment{\pointA}{\coteD}{\thetempangle}{\pointD}{#5}
\lsegment{\pointB}{\coteB}{0}{Z1}{PointSymbol=none, PointName=none}
\lsegment{\pointD}{\coteC}{0}{Z2}{PointSymbol=none, PointName=none}
\pstInterCC[PointNameA=none,PointSymbolA=none,#4]{\pointB}{Z1}{\pointD}{Z2}{Z3}{\pointC} 
\pstLineAB{\pointB}{\pointC}
\pstLineAB{\pointC}{\pointD}
}


% Définition des colonnes centrées ou à droite pour tabularx
\newcolumntype{Y}{>{\centering\arraybackslash}X}
\newcolumntype{Z}{>{\flushright\arraybackslash}X}

%Les pointillés à remplir par les élèves
\newcommand{\po}[1]{\makebox[#1 cm]{\dotfill}}
\newcommand{\lpo}[1][3]{%
\multido{}{#1}{\makebox[\linewidth]{\dotfill}
}}

%Liste des pictogrammes utilisés sur la fiche d'exercice ou d'activités
\newcommand{\bombe}{\faBomb}
\newcommand{\livre}{\faBook}
\newcommand{\calculatrice}{\faCalculator}
\newcommand{\oral}{\faCommentO}
\newcommand{\surfeuille}{\faEdit}
\newcommand{\ordinateur}{\faLaptop}
\newcommand{\ordi}{\faDesktop}
\newcommand{\ciseaux}{\faScissors}
\newcommand{\danger}{\faExclamationTriangle}
\newcommand{\out}{\faSignOut}
\newcommand{\cadeau}{\faGift}
\newcommand{\flash}{\faBolt}
\newcommand{\lumiere}{\faLightbulb}
\newcommand{\compas}{\dsmathematical}
\newcommand{\calcullitteral}{\faTimesCircleO}
\newcommand{\raisonnement}{\faCogs}
\newcommand{\recherche}{\faSearch}
\newcommand{\rappel}{\faHistory}
\newcommand{\video}{\faFilm}
\newcommand{\capacite}{\faPuzzlePiece}
\newcommand{\aide}{\faLifeRing}
\newcommand{\loin}{\faExternalLink}
\newcommand{\groupe}{\faUsers}
\newcommand{\bac}{\faGraduationCap}
\newcommand{\histoire}{\faUniversity}
\newcommand{\coeur}{\faSave}
\newcommand{\python}{\faPython}
\newcommand{\os}{\faMicrochip}
\newcommand{\rd}{\faCubes}
\newcommand{\data}{\faColumns}
\newcommand{\web}{\faCode}
\newcommand{\prog}{\faFile}
\newcommand{\algo}{\faCogs}
\newcommand{\important}{\faExclamationCircle}
\newcommand{\maths}{\faTimesCircle}
% Traitement des données en tables
\newcommand{\tables}{\faColumns}
% Types construits
\newcommand{\construits}{\faCubes}
% Type et valeurs de base
\newcommand{\debase}{{\footnotesize \faCube}}
% Systèmes d'exploitation
\newcommand{\linux}{\faLinux}
\newcommand{\sd}{\faProjectDiagram}
\newcommand{\bd}{\faDatabase}

%Les ensembles de nombres
\renewcommand{\N}{\mathbb{N}}
\newcommand{\D}{\mathbb{D}}
\newcommand{\Z}{\mathbb{Z}}
\newcommand{\Q}{\mathbb{Q}}
\newcommand{\R}{\mathbb{R}}
\newcommand{\C}{\mathbb{C}}

%Ecriture des vecteurs
\newcommand{\vect}[1]{\vbox{\halign{##\cr 
  \tiny\rightarrowfill\cr\noalign{\nointerlineskip\vskip1pt} 
  $#1\mskip2mu$\cr}}}


%Compteur activités/exos et question et mise en forme titre et questions
\newcounter{numact}
\setcounter{numact}{1}
\newcounter{numseance}
\setcounter{numseance}{1}
\newcounter{numexo}
\setcounter{numexo}{0}
\newcounter{numprojet}
\setcounter{numprojet}{0}
\newcounter{numquestion}
\newcommand{\espace}[1]{\rule[-1ex]{0pt}{#1 cm}}
\newcommand{\Quest}[3]{
\addtocounter{numquestion}{1}
\begin{tabularx}{\textwidth}{X|m{1cm}|}
\cline{2-2}
\textbf{\sffamily{\alph{numquestion})}} #1 & \dots / #2 \\
\hline 
\multicolumn{2}{|l|}{\espace{#3}} \\
\hline
\end{tabularx}
}
\newcommand{\QuestR}[3]{
\addtocounter{numquestion}{1}
\begin{tabularx}{\textwidth}{X|m{1cm}|}
\cline{2-2}
\textbf{\sffamily{\alph{numquestion})}} #1 & \dots / #2 \\
\hline 
\multicolumn{2}{|l|}{\cor{#3}} \\
\hline
\end{tabularx}
}
\newcommand{\Pre}{{\sc nsi} 1\textsuperscript{e}}
\newcommand{\Term}{{\sc nsi} Terminale}
\newcommand{\Sec}{2\textsuperscript{e}}
\newcommand{\Exo}[2]{ \addtocounter{numexo}{1} \ding{113} \textbf{\sffamily{Exercice \thenumexo}} : \textit{#1} \hfill #2  \setcounter{numquestion}{0}}
\newcommand{\Projet}[1]{ \addtocounter{numprojet}{1} \ding{118} \textbf{\sffamily{Projet \thenumprojet}} : \textit{#1}}
\newcommand{\ExoD}[2]{ \addtocounter{numexo}{1} \ding{113} \textbf{\sffamily{Exercice \thenumexo}}  \textit{(#1 pts)} \hfill #2  \setcounter{numquestion}{0}}
\newcommand{\ExoB}[2]{ \addtocounter{numexo}{1} \ding{113} \textbf{\sffamily{Exercice \thenumexo}}  \textit{(Bonus de +#1 pts maximum)} \hfill #2  \setcounter{numquestion}{0}}
\newcommand{\Act}[2]{ \ding{113} \textbf{\sffamily{Activité \thenumact}} : \textit{#1} \hfill #2  \addtocounter{numact}{1} \setcounter{numquestion}{0}}
\newcommand{\Seance}{ \rule{1.5cm}{0.5pt}\raisebox{-3pt}{\framebox[4cm]{\textbf{\sffamily{Séance \thenumseance}}}}\hrulefill  \\
  \addtocounter{numseance}{1}}
\newcommand{\Acti}[2]{ {\footnotesize \ding{117}} \textbf{\sffamily{Activité \thenumact}} : \textit{#1} \hfill #2  \addtocounter{numact}{1} \setcounter{numquestion}{0}}
\newcommand{\titre}[1]{\begin{Large}\textbf{\ding{118}}\end{Large} \begin{large}\textbf{ #1}\end{large} \vspace{0.2cm}}
\newcommand{\QListe}[1][0]{
\ifthenelse{#1=0}
{\begin{enumerate}[partopsep=0pt,topsep=0pt,parsep=0pt,itemsep=0pt,label=\textbf{\sffamily{\arabic*.}},series=question]}
{\begin{enumerate}[resume*=question]}}
\newcommand{\SQListe}[1][0]{
\ifthenelse{#1=0}
{\begin{enumerate}[partopsep=0pt,topsep=0pt,parsep=0pt,itemsep=0pt,label=\textbf{\sffamily{\alph*)}},series=squestion]}
{\begin{enumerate}[resume*=squestion]}}
\newcommand{\SQListeL}[1][0]{
\ifthenelse{#1=0}
{\begin{enumerate*}[partopsep=0pt,topsep=0pt,parsep=0pt,itemsep=0pt,label=\textbf{\sffamily{\alph*)}},series=squestion]}
{\begin{enumerate*}[resume*=squestion]}}
\newcommand{\FinListe}{\end{enumerate}}
\newcommand{\FinListeL}{\end{enumerate*}}

%Mise en forme de la correction
\newcommand{\cor}[1]{\par \textcolor{OliveGreen}{#1}}
\newcommand{\br}[1]{\cor{\textbf{#1}}}
\newcommand{\tcor}[1]{\begin{tcolorbox}[width=0.92\textwidth,colback={white},colbacktitle=white,coltitle=OliveGreen,colframe=green!75!black,boxrule=0.2mm]   
\cor{#1}
\end{tcolorbox}
}
\newcommand{\rc}[1]{\textcolor{OliveGreen}{#1}}
\newcommand{\pmc}[1]{\textcolor{blue}{\tt #1}}
\newcommand{\tmc}[1]{\textcolor{RawSienna}{\tt #1}}


%Référence aux exercices par leur numéro
\newcommand{\refexo}[1]{
\refstepcounter{numexo}
\addtocounter{numexo}{-1}
\label{#1}}

%Séparation entre deux activités
\newcommand{\separateur}{\begin{center}
\rule{1.5cm}{0.5pt}\raisebox{-3pt}{\ding{117}}\rule{1.5cm}{0.5pt}  \vspace{0.2cm}
\end{center}}

%Entête et pied de page
\newcommand{\snt}[1]{\lhead{\textbf{SNT -- La photographie numérique} \rhead{\textit{Lycée Nord}}}}
\newcommand{\Activites}[2]{\lhead{\textbf{{\sc #1}}}
\rhead{Activités -- \textbf{#2}}
\cfoot{}}
\newcommand{\Exos}[2]{\lhead{\textbf{Fiche d'exercices: {\sc #1}}}
\rhead{Niveau: \textbf{#2}}
\cfoot{}}
\newcommand{\Devoir}[2]{\lhead{\textbf{Devoir de mathématiques : {\sc #1}}}
\rhead{\textbf{#2}} \setlength{\fboxsep}{8pt}
\begin{center}
%Titre de la fiche
\fbox{\parbox[b][1cm][t]{0.3\textwidth}{Nom : \hfill \po{3} \par \vfill Prénom : \hfill \po{3}} } \hfill 
\fbox{\parbox[b][1cm][t]{0.6\textwidth}{Note : \po{1} / 20} }
\end{center} \cfoot{}}

%Devoir programmation en NSI (pas à rendre sur papier)
\newcommand{\PNSI}[2]{\lhead{\textbf{Devoir de {\sc nsi} : \textsf{ #1}}}
\rhead{\textbf{#2}} \setlength{\fboxsep}{8pt}
\begin{tcolorbox}[title=\textcolor{black}{\danger\; A lire attentivement},colbacktitle=lightgray]
{\begin{enumerate}
\item Rendre tous vous programmes en les envoyant par mail à l'adresse {\tt fnativel2@ac-reunion.fr}, en précisant bien dans le sujet vos noms et prénoms
\item Un programme qui fonctionne mal ou pas du tout peut rapporter des points
\item Les bonnes pratiques de programmation (clarté et lisiblité du code) rapportent des points
\end{enumerate}
}
\end{tcolorbox}
 \cfoot{}}


%Devoir de NSI
\newcommand{\DNSI}[2]{\lhead{\textbf{Devoir de {\sc nsi} : \textsf{ #1}}}
\rhead{\textbf{#2}} \setlength{\fboxsep}{8pt}
\begin{center}
%Titre de la fiche
\fbox{\parbox[b][1cm][t]{0.3\textwidth}{Nom : \hfill \po{3} \par \vfill Prénom : \hfill \po{3}} } \hfill 
\fbox{\parbox[b][1cm][t]{0.6\textwidth}{Note : \po{1} / 10} }
\end{center} \cfoot{}}

\newcommand{\DevoirNSI}[2]{\lhead{\textbf{Devoir de {\sc nsi} : {\sc #1}}}
\rhead{\textbf{#2}} \setlength{\fboxsep}{8pt}
\cfoot{}}

%La définition de la commande QCM pour auto-multiple-choice
%En premier argument le sujet du qcm, deuxième argument : la classe, 3e : la durée prévue et #4 : présence ou non de questions avec plusieurs bonnes réponses
\newcommand{\QCM}[4]{
{\large \textbf{\ding{52} QCM : #1}} -- Durée : \textbf{#3 min} \hfill {\large Note : \dots/10} 
\hrule \vspace{0.1cm}\namefield{}
Nom :  \textbf{\textbf{\nom{}}} \qquad \qquad Prénom :  \textbf{\prenom{}}  \hfill Classe: \textbf{#2}
\vspace{0.2cm}
\hrule  
\begin{itemize}[itemsep=0pt]
\item[-] \textit{Une bonne réponse vaut un point, une absence de réponse n'enlève pas de point. }
\item[\danger] \textit{Une mauvaise réponse enlève un point.}
\ifthenelse{#4=1}{\item[-] \textit{Les questions marquées du symbole \multiSymbole{} peuvent avoir plusieurs bonnes réponses possibles.}}{}
\end{itemize}
}
\newcommand{\DevoirC}[2]{
\renewcommand{\footrulewidth}{0.5pt}
\lhead{\textbf{Devoir de mathématiques : {\sc #1}}}
\rhead{\textbf{#2}} \setlength{\fboxsep}{8pt}
\fbox{\parbox[b][0.4cm][t]{0.955\textwidth}{Nom : \po{5} \hfill Prénom : \po{5} \hfill Classe: \textbf{1}\textsuperscript{$\dots$}} } 
\rfoot{\thepage} \cfoot{} \lfoot{Lycée Nord}}
\newcommand{\DevoirInfo}[2]{\lhead{\textbf{Evaluation : {\sc #1}}}
\rhead{\textbf{#2}} \setlength{\fboxsep}{8pt}
 \cfoot{}}
\newcommand{\DM}[2]{\lhead{\textbf{Devoir maison à rendre le #1}} \rhead{\textbf{#2}}}

%Macros permettant l'affichage des touches de la calculatrice
%Touches classiques : #1 = 0 fond blanc pour les nombres et #1= 1gris pour les opérations et entrer, second paramètre=contenu
%Si #2=1 touche arrondi avec fond gris
\newcommand{\TCalc}[2]{
\setlength{\fboxsep}{0.1pt}
\ifthenelse{#1=0}
{\psframebox[fillstyle=solid, fillcolor=white]{\parbox[c][0.25cm][c]{0.6cm}{\centering #2}}}
{\ifthenelse{#1=1}
{\psframebox[fillstyle=solid, fillcolor=lightgray]{\parbox[c][0.25cm][c]{0.6cm}{\centering #2}}}
{\psframebox[framearc=.5,fillstyle=solid, fillcolor=white]{\parbox[c][0.25cm][c]{0.6cm}{\centering #2}}}
}}
\newcommand{\Talpha}{\psdblframebox[fillstyle=solid, fillcolor=white]{\hspace{-0.05cm}\parbox[c][0.25cm][c]{0.65cm}{\centering \scriptsize{alpha}}} \;}
\newcommand{\Tsec}{\psdblframebox[fillstyle=solid, fillcolor=white]{\parbox[c][0.25cm][c]{0.6cm}{\centering \scriptsize 2nde}} \;}
\newcommand{\Tfx}{\psdblframebox[fillstyle=solid, fillcolor=white]{\parbox[c][0.25cm][c]{0.6cm}{\centering \scriptsize $f(x)$}} \;}
\newcommand{\Tvar}{\psframebox[framearc=.5,fillstyle=solid, fillcolor=white]{\hspace{-0.22cm} \parbox[c][0.25cm][c]{0.82cm}{$\scriptscriptstyle{X,T,\theta,n}$}}}
\newcommand{\Tgraphe}{\psdblframebox[fillstyle=solid, fillcolor=white]{\hspace{-0.08cm}\parbox[c][0.25cm][c]{0.68cm}{\centering \tiny{graphe}}} \;}
\newcommand{\Tfen}{\psdblframebox[fillstyle=solid, fillcolor=white]{\hspace{-0.08cm}\parbox[c][0.25cm][c]{0.68cm}{\centering \tiny{fenêtre}}} \;}
\newcommand{\Ttrace}{\psdblframebox[fillstyle=solid, fillcolor=white]{\parbox[c][0.25cm][c]{0.6cm}{\centering \scriptsize{trace}}} \;}

% Macroi pour l'affichage  d'un entier n dans  une base b
\newcommand{\base}[2]{ \overline{#1}^{#2}}
% Intervalle d'entiers
\newcommand{\intN}[2]{\llbracket #1; #2 \rrbracket}}

% Numéro et titre de chapitre
\setcounter{numchap}{16}
\newcommand{\Ctitle}{\cnum {Logique}}
\newcommand{\SPATH}{/home/fenarius/Travail/Cours/cpge-info/docs/mp2i/files/C\thenumchap/}

\newcommand{\vrai}{\mymathbb{1}}
\newcommand{\faux}{\mymathbb{0}}
\newcommand{\non}{\neg}
\newcommand{\et}{\wedge}
\newcommand{\ou}{\vee}
\newcommand{\imp}{\to}
\newcommand{\eq}{\leftrightarrow}
\newcommand{\lnode}[1]{\TCircle{$#1$}}
\newcommand{\B}{\mathbb{B}}
\newcommand{\vv}[1]{\llbracket #1 \rrbracket_{\varphi}}
\psset{treesep=0.5cm,levelsep=0.8cm}

\makess{Syntaxe des formules logiques}
\begin{frame}{\Ctitle}{\stitle}
	\begin{alertblock}{Définition}
		Soit  $\vrai$ un ensemble au plus dénombrable de variables logiques noté $V = \{p, q, r, \dots \}$. On définit inductivement l'ensemble $P$ des formules logiques sur  $\vrai$ par :
		\begin{itemize}
			\item<2-> L'ensemble d'axiomes $P_0 = \{ \top, \bot \} \cup V$ \\
				\onslide<3->\textcolor{gray}{\small $\top$  se lit \og{} top \fg{} et $\bot$ se lit \og{} bottom \fg{}}
			\item<4-> Les règles d'inférence :
				\begin{itemize}
					\item<4-> \textit{négation} $\non : F \mapsto \non F$
					\item<5-> \textit{conjonction} $\et : F,G \mapsto (F \et G)$
					\item<6-> \textit{disjonction} $\ou : F,G \mapsto (F \ou G)$
					\item<7-> \textit{implication} $\imp : F,G \mapsto (F \imp G)$
					\item<8-> \textit{équivalence} $\eq : F,G \mapsto (F \eq G)$
				\end{itemize}
		\end{itemize}
		On notera avec les lettres minuscules $p$, $q$, \dots \textit{les variables logiques} ie les éléments de  $\vrai$ et avec les lettres majuscules  $\faux$, $G$, \dots \textit{les formules logiques} ie les éléments de $P$.
	\end{alertblock}
\end{frame}

\begin{frame}{\Ctitle}{\stitle}
	\begin{block}{Remarques}
		\begin{itemize}
			\item Afin d'éviter de surcharger les écritures, on pourra omettre certaines parenthèses :
			      \begin{itemize}
				      \item<2-> En utilisant l'associativité à droite de $\et, \ou$ : \\
					      \onslide<3-> \textcolor{gray}{Par exemple, $(p \ou (q \ou r))$ s'écrit plus simplement $p \ou q \ou r$.}
				      \item<4-> En utilisant l'ordre de priorité suivant sur les connecteurs : $\non, \et, \ou, \imp, \eq$ \\
					      \onslide<5-> \textcolor{gray}{Par exemple $ ((\non p) \ou (q \et r))$ s'écrit plus simplement $\non p \ou q \et r$.}
			      \end{itemize}
			      \onslide<6->{En cas de doute, on laissera les parenthèses afin de lever toute ambiguïté.}
			\item<7-> On a définit pour le moment simplement les propositions logiques valables d'un point de vue \textit{syntaxique}, sans leur donner de sens ou de valeur.
		\end{itemize}
	\end{block}
	\onslide<8->{
		\begin{exampleblock}{Exemples}
			\begin{itemize}
				\item<7-> $(((\non p) \ou (\non q)) \et r)$ est une formule logique qu'on pourra écrire plus simplement $(\non p \ou \non q) \et r$.
				\item<8-> $\et p \non pq$ ou encore $( p \et q) \imp (r$ ne sont pas des formules logiques.
				\item<9-> $(p \imp q) \et (q \imp p)$ et $ p \eq q$ sont deux formules logiques \textit{différentes}.
			\end{itemize}
		\end{exampleblock}}
\end{frame}

\begin{frame}{\Ctitle}{\stitle}
	\begin{block}{Arbre syntaxique}
		Les formules logiques admettent naturellement une représentation sous forme d'arbre :
		\begin{itemize}
			\item<2-> les variables logiques et les constantes $\top$ et $\bot$ sont les étiquettes des feuilles
			\item<3-> les noeuds internes ont pour étiquette les règles d'inférence
		\end{itemize}
	\end{block}
	\onslide<4->{
		\begin{exampleblock}{Exemples}
			La  formule logique $( p \imp q) \ou (\non r)$ admet la représentation : \\
			\onslide<5->{
				\begin{center}
					\pstree{\lnode{\ou}}{\pstree{\lnode{\imp}}{\lnode{p} \lnode{q}} \pstree{\lnode{\non}}{\lnode{r}}}
				\end{center}
			}
		\end{exampleblock}}
\end{frame}

\begin{frame}{\Ctitle}{\stitle}
	\begin{exampleblock}{Exemple}
		\begin{itemize}
			\item Quelle est la formule logique ayant pour représentation arborescente :
			      \begin{center}
				      \pstree{\lnode{\et}}{
					      \pstree{\lnode{\ou}}{
						      \pstree{\lnode{\imp}}{\lnode{p} \lnode{q}}
						      \pstree{\lnode{\eq}}{\lnode{r} \lnode{s}}
					      }
					      \pstree{\lnode{\ou}}{
						      \lnode{r}
						      \pstree{\lnode{\non}}{\lnode{q}}
					      }
				      }
			      \end{center}
			      \onslide<2->\textcolor{OliveGreen}{$ ((p \imp q) \ou (r \eq s)) \et (r \ou (\non q)) $}
			\item<3-> Dessiner la représentation arborescente de $ \non(\top \eq (p \ou q))$.
		\end{itemize}
	\end{exampleblock}
\end{frame}

\begin{frame}{\Ctitle}{\stitle}
	\begin{block}{Hauteur, taille et sous formule}
		Etant donnée une formule logique notée  $\faux$,
		\begin{itemize}
			\item<2-> la \textcolor{blue}{hauteur} de  $\faux$ est la hauteur de l'arbre syntaxique associé
			\item<3-> la \textcolor{blue}{taille} de  $\faux$ est le nombre de noeuds de l'arbre syntaxiqué associé
			\item<4-> Une \textcolor{blue}{sous formule} de  $\faux$ est un sous-arbre de l'arbre syntaxique associé
		\end{itemize}
	\end{block}
\end{frame}

\begin{frame}{\Ctitle}{\stitle}
	\begin{block}{Implémentation en OCaml}
		Le type somme de OCaml associé à la correspondance de motif permettent de représenter et de travailler efficacement sur les formules logiques.
		\onslide<2->{\inputpartOCaml{\SPATH/fl.ml}{}{\small}{1}{8}}
		\onslide<3->{On a représenté ici une variable logique par un entier, on pourrait choisir un caractère, ou un type option {\tt 'a}.\\}
		\onslide<4->{La formule logique $(p \et \non q) \imp r$ est alors représentée par :}
		\onslide<5->{\inputpartOCaml{\SPATH/fl.ml}{}{\small}{16}{16}}
	\end{block}
\end{frame}

\begin{frame}{\Ctitle}{\stitle}
	\begin{block}{Implémentation en OCaml}
		Le calcul de la taille s'obtient alors via un \textit{pattern matching} :
		\onslide<2->{\inputpartOCaml{\SPATH/fl.ml}{}{\small}{18}{25}}
	\end{block}
\end{frame}

\makess{Sémantique des formules logiques}
\begin{frame}{\Ctitle}{\stitle}
	\begin{alertblock}{Valuation}
		On note $\mathbb{B} = \{\faux, \vrai\}$, l'ensemble des \textcolor{blue}{valeurs de vérités}\\
		Une \textcolor{blue}{valuation} est une attribution de l'une des deux valeurs de vérités à chaque variable propositionnelle. Une valuation $\varphi$ est une donc une application de  $\vrai$ de $B$.
	\end{alertblock}
	\onslide<2->{
		\begin{exampleblock}{Exemple}
			Si l'ensemble des variables propositionnelle est $V = \{ p, q, r \}$, une valuation possible est : \\
			$\varphi :  V  \mapsto \B$, avec $\varphi(p) = \vrai$, $\varphi(q) = \faux$ et $\varphi(r) = \faux$.    \\
		\end{exampleblock}
	}
	\onslide<3->{
		\begin{block}{Remarque}
			En nontant $| V | = n$,  il y a $2^n$ valuations possibles.
		\end{block}
	}
\end{frame}

\begin{frame}{\Ctitle}{\stitle}
	\begin{block}{Fonction booléennes usuelles}
		On rappelle les fonction booléennes usuelles associées à chaque connecteur :
		\begin{tabularx}{\textwidth}{YY}
			$f_{\et} : \B^2 \mapsto \B$ \newline
			\begin{tabular}{l|l|l}
				$x$ & $y$ & $f_{\et}(x,y)$ \\
				\hline
				 $\faux$ &  $\faux$ &  $\faux$            \\
				 $\faux$ &  $\vrai$ &  $\faux$            \\
				 $\vrai$ &  $\faux$ &  $\faux$            \\
				 $\vrai$ &  $\vrai$ &  $\vrai$            \\
			\end{tabular} &
			$f_{\ou} : \B^2 \mapsto \B$ \newline
			\begin{tabular}{l|l|l}
				$x$ & $y$ & $f_{\ou}(x,y)$ \\
				\hline
				 $\faux$ &  $\faux$ &  $\faux$            \\
				 $\faux$ &  $\vrai$ &  $\vrai$            \\
				 $\vrai$ &  $\faux$ &  $\vrai$            \\
				 $\vrai$ &  $\vrai$ &  $\vrai$            \\
			\end{tabular}      \\
			                                                    & \\
			$f_{\imp} : \B^2 \mapsto \B$ \newline
			\begin{tabular}{l|l|l}
				$x$ & $y$ & $f_{\imp}(x,y)$ \\
				\hline
				 $\faux$ &  $\faux$ &  $\vrai$             \\
				 $\faux$ &  $\vrai$ &  $\vrai$             \\
				 $\vrai$ &  $\faux$ &  $\faux$             \\
				 $\vrai$ &  $\vrai$ &  $\vrai$             \\
			\end{tabular} &
			$f_{\eq} : \B^2 \mapsto \B$ \newline
			\begin{tabular}{l|l|l}
				$x$ & $y$ & $f_{\eq}(x,y)$ \\
				\hline
				 $\faux$ &  $\faux$ &  $\vrai$            \\
				 $\faux$ &  $\vrai$ &  $\faux$            \\
				 $\vrai$ &  $\faux$ &  $\faux$            \\
				 $\vrai$ &  $\vrai$ &  $\vrai$            \\
			\end{tabular}      \\
			                                                    & \\
		\end{tabularx}
		Et $f_{\non} : \B \mapsto \B$, définie par $f_\non(\faux) = \vrai$ et $f_\non(\vrai) = \faux$.
	\end{block}
\end{frame}

\begin{frame}{\Ctitle}{\stitle}
	\begin{alertblock}{Valeur de vérité d'une formule}
		Pour une valuation $\varphi$, on définit la valeur de vérité d'une formule $F$ notée $\vv{F}$ par :
		\begin{itemize}
			\item<2-> $\vv{\top} = V$
			\item<2-> $\vv{\bot} = F$
			\item<3-> si $v \in V$, $\vv{v} = \varphi(v)$
			\item<4-> $\vv{\non F} = f_{\non}(\vv{F})$
			\item<5-> $\vv{(F \et G)} = f_{\et}(\vv{F},\vv{G})$
			\item<5-> $\vv{(F \ou G)} = f_{\ou}(\vv{F},\vv{G})$
			\item<5-> $\vv{(F \imp G)} = f_{\imp}(\vv{F},\vv{G})$
			\item<5-> $\vv{(G \eq G)} = f_{\eq}(\vv{G},\vv{G})$
		\end{itemize}
	\end{alertblock}
\end{frame}

\begin{frame}{\Ctitle}{\stitle}
	\begin{exampleblock}{Exemple}
		Sur $V = \{p, q, r\}$, on considère la valuation $\varphi :  V  \mapsto \B$, telle que $\varphi(p) = \vrai$, $\varphi(q) = \faux$ et $\varphi(r) = \faux$ on peut alors déterminer la valeur de vérité de la formule logique $F = (p \imp q) \et (\non p \ou r )$ associée à cette valuation :
		\onslide<2->{$\vv{F} = f_{\et} (\vv{p \imp q}, \vv{\non p \ou r})$ \\}
		\onslide<3->{$\vv{F} = f_{\et} (f_{\imp}(\vv{p},\vv{q})), f_{\ou}(\vv{\non p},\vv{r})$ \\}
		\onslide<4->{$\vv{F} = f_{\et}(f_{\imp}(\vrai,\faux),f_{\ou}(\faux,\faux))$\\}
		\onslide<5->{$\vv{F} = f_{\et}(\faux,\faux)$\\}
		\onslide<6->{$\vv{F} = \faux$\\}
		\begin{itemize}
			\item<7-> Donner la valeur de vérité de cette proposition pour la valuation $\varphi'$ définie par : $ \varphi'(p) = \faux, \varphi'(q) = \faux, \varphi'(r) = \faux$
			\item<8-> Déterminer la valeur de vérité de $((p \imp q) \ou (r \eq s)) \et (r \ou (\non q))$ pour la valuation $\varphi$.
		\end{itemize}
	\end{exampleblock}
\end{frame}

\begin{frame}{\Ctitle}{\stitle}
	\begin{alertblock}{Tautologie, antilogie, satisfiabilité}
		\begin{itemize}
			\item Une formule est une \textcolor{blue}{tautologie} si sa valeur de vérité est  $\vrai$ pour toute valuation. c'est-à-dire que $F$ est une tautologie ssi pour tout $\varphi \in \B^V$, $\vv{F} = \vrai$.
			\item<2-> Une formule est une \textcolor{blue}{antilogie} si sa valeur de vérité est  $\faux$ pour toute valuation. c'est-à-dire que $F$ est une antilogie ssi pour tout $\varphi \in \B^V$, $\vv{F} = \faux$.
			\item<3-> Une formule est \textcolor{blue}{satisfiable} s'il existe une valuation pour laquelle sa valeur de vérité est  $\vrai$. c'est-à-dire que $F$ est satisfiable ssi il existe $\varphi \in \B^V$ tel que $\vv{F} = \vrai$.
		\end{itemize}
	\end{alertblock}
	\onslide<4->
	{\begin{block}{Remarques}
			\begin{itemize}
				\item<4-> $F$ est une tautologie ssi $\non F$ n'est pas satisfiable.
				\item<5-> $F$ est satisfiable ssi $\non F$ n'est pas une tautologie.
			\end{itemize}
		\end{block}}
\end{frame}

\begin{frame}{\Ctitle}{\stitle}
	\begin{block}{Tables de vérité}
		La \textcolor{blue}{table de vérité} d'une formule logique $F$ contenant $n$ variables logiques consiste à présenter sous forme de tableau la valeur de vérité de $F$  pour chacune des $2^n$ valuations possibles.
	\end{block}
	\onslide<2->{
		\begin{exampleblock}{Exemple}
			\begin{itemize}
				\item<2-> Dresser la table de vérité de $F = (p \imp q) \eq (\non p \ou q)$
					\onslide<4->{
						\begin{tabular}{c|c|c|c|c}
							$p$ & $q$ & $p \imp q$                                             & $\non p \ou q$                                         & $F$                                                    \\
                            \hline
							 $\faux$ &  $\faux$ & \leavevmode{\onslide<5->{\textcolor{OliveGreen}{ $\vrai$}}} & \leavevmode{\onslide<6->{\textcolor{OliveGreen}{ $\vrai$}}} & \leavevmode{\onslide<7->{\textcolor{OliveGreen}{ $\vrai$}}} \\
							 $\faux$ &  $\vrai$ & \leavevmode{\onslide<5->{\textcolor{OliveGreen}{ $\vrai$}}} & \leavevmode{\onslide<6->{\textcolor{OliveGreen}{ $\vrai$}}} & \leavevmode{\onslide<7->{\textcolor{OliveGreen}{ $\vrai$}}} \\
							 $\vrai$ &  $\faux$ & \leavevmode{\onslide<5->{\textcolor{OliveGreen}{ $\faux$}}} & \leavevmode{\onslide<6->{\textcolor{OliveGreen}{ $\faux$}}} & \leavevmode{\onslide<7->{\textcolor{OliveGreen}{ $\vrai$}}} \\
							 $\vrai$ &  $\vrai$ & \leavevmode{\onslide<5->{\textcolor{OliveGreen}{ $\vrai$}}} & \leavevmode{\onslide<6->{\textcolor{OliveGreen}{ $\vrai$}}} & \leavevmode{\onslide<7->{\textcolor{OliveGreen}{ $\vrai$}}} \\
						\end{tabular}
					}
				\item<3-> Que peut-on en déduire ?
				\onslide<8->\textcolor{OliveGreen}{$F$ est une tautologie.}
			\end{itemize}
		\end{exampleblock}}
\end{frame}

\begin{frame}{\Ctitle}{\stitle}
    \begin{alertblock}{Equivalence logique}
        On dit que deux formules logiques $F$ et $G$ sont \textcolor{blue}{logiquement équivalentes} si pour toute valuation $\varphi$, $\vv{F} = \vv{G}$.
        On notera alors $F \equiv G$.\\
        \onslide<2->{Cela traduit l'égalité sémantique de $F$ et $G$, et permet de simplifier les formules.}
        \onslide<3->{A ne pas confondre avec $\eq$ qui est un connecteur logique.}
    \end{alertblock}
    \onslide<3->
    {\begin{block}{Quelques équivalences à connaitre}
        \begin{itemize}
            \item<3-> $\non (\non F) \equiv F$
            \item<4-> $F \ou (G \et H) \equiv (F \ou G) \et (F \ou H)$
            \item<5-> $F \et (G \ou H) \equiv (F \et G) \ou (F \et H)$ 
            \item<6-> $\non (F \ou G) \equiv \non F \et \non G$  (loi de De Morgan)
            \item<7-> $\non (F \et G) \equiv \non F \ou \non G$ (loi de De Morgan)
            \item<8-> $F \imp G \equiv   \non F \ou G $
        \end{itemize}
    \end{block}
}
\end{frame}

\begin{frame}{\Ctitle}{\stitle}
    \begin{exampleblock}{Exemple}
        Montrer que :
        \begin{itemize}
            \item $P \ou \non P \equiv \top$ (tiers exclu)
            \item<2-> $ P \imp Q \equiv \non Q \imp \non P$ (contraposition)
        \end{itemize}
    \end{exampleblock}
    \onslide<3->{
    \begin{alertblock}{Conséquence logique}
        On dit que qu'une formule $G$ est \textcolor{blue}{conséquence logique} d'un ensemble de formules  $\Gamma = \{F_1, \dots F_n\}$  si pour toute valuation $\varphi$, qui rend vraies les formules $(F_i)_{1\leqslant i \leqslant n}$ rend aussi vraie $G$.
        On notera alors $\Gamma \vDash Q$.\\
        \onslide<4->{A ne pas confondre avec $\imp$ qui est un connecteur logique.}
    \end{alertblock}}
    \onslide<5->{
        \begin{block}{Remarques}
            \begin{itemize}
                \item<5-> Une formule $F$ est est une tautologie ssi $\varnothing \vDash F$, on notera simplement, $\vDash F$.
                \item<6-> $F \equiv G$ ssi $F \vDash G$ et $ G \vDash Q$.
            \end{itemize}
        \end{block}
                }
\end{frame}

\makess{Formes normales}
\begin{frame}{\Ctitle}{\stitle}
    \begin{block}{Définitions}
        \begin{itemize}
        \item Un \textcolor{blue}{littéral} est une formule qui est soit une variable propositionnelle $p$, soit sa négation $\non p$.
        \item<2-> Une \textcolor{blue}{forme normale conjonctive} est une formule qui est une conjonction de disjonctions de littéraux.\\
        $(p_{1,1} \ou p_{1,2} \dots \ou p_{1,k_1}) \et \underbrace{(p_{2,1} \ou p_{2,2} \dots \ou p_{2,k_2})}_{\text{\textcolor{blue}{une clause}}} \et \dots \et (p_{m,1} \ou p_{m,2} \dots \ou p_{m,k_m})$
        \item<3-> Une \textcolor{blue}{forme normale disjonctive} est une formule qui est une disjonction de conjonctions de littéraux.\\
        $(p_{1,1} \et p_{1,2} \dots \et p_{1,k_1}) \ou (p_{2,1} \et p_{2,2} \dots \et p_{2,k_2}) \ou \dots \ou (p_{m,1} \et p_{m,2} \dots \et p_{m,k_m})$
        \end{itemize}
    \end{block}
    \onslide<4->{
        \begin{exampleblock}{Exemple}
            $(p \et q \et \non r) \ou (\non p \et r) \ou (p \et \non q \et\non r)$ est une {\sc fnd}.
        \end{exampleblock}
    }
\end{frame}


\begin{frame}{\Ctitle}{\stitle}
    \begin{alertblock}{Propositions}
        \begin{itemize}
            \item<1-> Pour tout formule logique $F$, il existe une {\sc fnc} $G$ et une {\sc fnd} $H$ telles que $F \equiv G \equiv H$.
            \item<2-> On dispose d'un algorithme pour calculer une forme normale :
            \begin{enumerate}
                \item<3-> supprimer les $\bot$ et les $\top$
                \item<4-> Remplacer $\imp$ et $\eq$ par des formules sémantiquement équivalentes  n'utilisant pas ces connecteurs : $A \imp B \equiv \non A \ou B$ et $ A \eq B \equiv (A \et B) \ou (\non A \et \non B)$.
                \item<5-> Utiliser les lois de De Morgan afin de faire descendre les $\non$ au niveau des feuilles de l'arbre syntaxique
                \item<6-> Appliquer les propriétés d'associativité et de distributivité des connecteurs $\et$ et $\ou$.
                \item<7-> simplifier les doublons éventuelles dans les clauses de littéraux ($v \et \non v \equiv \bot$ et $v \ou \non v = \top$)
            \end{enumerate}
            \item<8-> Une autre méthode consiste à utiliser la table de vérité.
        \end{itemize}
    \end{alertblock}
\end{frame}

\begin{frame}{\Ctitle}{\stitle}
    \begin{exampleblock}{Exemple}
        {\small Mise sous forme normale de $ P = (p \imp q) \eq \non r$ \\}
        \begin{itemize}
            \item<1->{\small Méthode 1 : utilisation de l'algorithme \\}
            \onslide<2->{\small $P =  ((p \imp q) \et \non r) \ou (\non (p\imp q) \et \non \non r)$ (équivalence sémantique de $\eq$)}
            \onslide<3->{\small $P =  ((\non p \ou q) \et \non r) \ou (\non (\non p \ou q) \et r)$ (équivalence sémantique de $\imp$)}
            \onslide<4->{\small $P = ((\non p \ou q) \et \non r) \ou ( (p \et \non q) \et r)$ (lois de DeMorgan)} 
            \onslide<5->{\small $P = (\non p \et \non r) \ou (q \et \non r) \ou ( p \et \non q \et r)$ (distributivité et associativité)}
            \item<6->{\small Méthode 2 : utilisation de la table de vérité\\}
            \onslide<7->{\begin{tabular}{>{\small}c|>{\small}c|>{\small}c|>{\small}c|>{\small}c|>{\small}c|>{\small}l} 
                $p$ & $q$ &$r$  & $p \imp q$  & $\non r$&  $P$                                                    \\
                \cline{1-6}
                 $\faux$ &  $\faux$ &  $\faux$ &  $\vrai$ &  $\vrai$&  $\textcolor{BrickRed}{\vrai}$ & \leavevmode\onslide<7->{$\longrightarrow (\non p \et \non q \et \non r) \ou$} \\
                 $\faux$ &  $\faux$ &  $\vrai$ &  $\vrai$ &  $\faux$&   $\faux$ & \\
                 $\faux$ &  $\vrai$ &  $\faux$ &  $\vrai$ &  $\vrai$&  $\textcolor{BrickRed}{\vrai}$ & \leavevmode\onslide<8->{$\longrightarrow (\non p \et  q \et \non r) \ou$}\\
                 $\faux$ &  $\vrai$ &  $\vrai$ &  $\vrai$ &  $\faux$&   $\faux$ & \\
                 $\vrai$ &  $\faux$ &  $\faux$ &  $\faux$ &  $\vrai$&   $\faux$ & \\
                 $\vrai$ &  $\faux$ &  $\vrai$ &  $\faux$ &  $\faux$&  $\textcolor{BrickRed}{\vrai}$ & \leavevmode\onslide<9->{$\longrightarrow ( p \et \non q \et  r) \ou$}\\
                 $\vrai$ &  $\vrai$ &  $\faux$ &  $\vrai$ &  $\vrai$&  $\textcolor{BrickRed}{\vrai}$ & \leavevmode\onslide<10->{$\longrightarrow (p \et  q \et \non r)$}\\
                 $\vrai$ &  $\vrai$ &  $\vrai$ &  $\vrai$ &  $\faux$&   $\faux$ & \\
            \end{tabular}}
        \end{itemize}
    \end{exampleblock}
\end{frame}

\makess{Problème {\sc sat} -- Algorithme de Quine}
\begin{frame}{\Ctitle}{\stitle}
\begin{block}{Définitions}
    \begin{itemize}
		\item<1-> Un \textcolor{blue}{problème de décision} sur un ensemble $E$, est une question sur les éléments de $E$ à laquelle on répond par \textit{oui} ou \textit{non}.\\
		\textcolor{gray}{Par exemple sur $\N$, savoir si un entier $n$ est premier ou non est un problème de décision.}
		\item<2-> La \textcolor{blue}{théorie de la calculabilité} étudie l'existence ou non d'un algorithme capable de répondre à un problème de décision.\\
		\textcolor{gray}{Par exemple le problème de l'arrêt est indécidable}
		\item<3-> La \textcolor{blue}{théorie de la complexité} s'intéresse à la complexité des algorithmes lorsqu'un problème de décision est décidable.		
	\end{itemize}
\end{block}
\end{frame}

\begin{frame}{\Ctitle}{\stitle}
	\begin{alertblock}{Problème {\sc sat}}
		Le problème {\sc sat} (pour satisfiabilité) est le problème de savoir si une formule logique $F$ définie sur un ensemble de variable logique $V = \{p_1, \dots p_n\}$ est satisfiable ou non.
	\end{alertblock}
\end{frame}



\begin{frame}{\Ctitle}{\stitle}
    \begin{alertblock}{Algorithme de Quine}
    Pour tester la satisfiabilité d'une formule logique, on peut construire sa table de vérité ou utiliser l'\textcolor{blue}{algorithme de Quine}. Soit $F$ une formule contenant les variables logiques $p_1, \dots p_n$.
    \begin{itemize}
        \item On fixe $\varphi(p_1) = \faux$ et on teste récursivement la satisfiabilité de $F$ dans laquelle toutes les occurences de $p_1$ sont remplacées par $\bot$ (notée $F[\bot/p_1$]).
        \item En cas d'échec,  on fixe $\varphi(p_1) = \vrai$ et on teste récursivement la satisfiabilité de $P[\top/p_1]$.
        \item En cas d'échec la formule n'est pas satisfiable.
    \end{itemize}
\end{alertblock}
\end{frame}

\begin{frame}{\Ctitle}{\stitle}
    \begin{exampleblock}{Exemple}
        Dérouler l'algorithme de Quine sur $F = (p\ \ou q) \et (\non q \ou \non r) \et (\non p \ou r) \et (p \ou r)$
        \begin{itemize}
            \item On affecte la valeur  $\faux$ à $p$ et on teste la satisfiabilité de : \\
            \onslide<2->{$ F[\bot/p] = (\bot \ou q) \et (\non q \ou \non r) \et (\non \bot \ou r) \et (\bot \ou r) $\\}
            \onslide<3->{$ \phantom{P[\bot/p]} = q \et (\non q \ou \non r) \et \top \et r$ \\}
            \onslide<4->{$ \phantom{P[\bot/p]} = q \et (\non q \ou \non r) \et r$}
            \begin{itemize} 
                \item<5-> On affecte la valeur  $\faux$ à $q$ et on teste la satisfiabilité de : \\
                \onslide<6->{$\bot \et (\non \bot \ou \non r) \et r$ qui est non satisfiable}
                \item<7-> On affecte la valeur  $\vrai$ à $q$ et on teste la satisfiabilité de : \\
                \onslide<8->{$\top \et (\non \top \ou \non r) \et r =  \non r \et r $ donc non satisfiable. }
            \end{itemize}
            \item<9-> On affecte la valeur  $\vrai$ à $p$ et on teste la satisfiabilité de : \\
            \onslide<10->{$ F[\top/p] = (\top \ou q) \et (\non q \ou \non r) \et (\non \top \ou r) \et (\top \ou r) $\\}
            \onslide<11->{$ F[\top/p] = (\non q \ou \non r) \et  r $\\}
            \begin{itemize} 
                \item<12-> On affecte la valeur  $\faux$ à $q$ et on teste la satisfiabilité de : \\
                \onslide<13->{$(\non \bot \ou \non r) \et  r = r $ qui est satisfiable.}
            \end{itemize}
        \end{itemize}
        \onslide<14->{On dispose à la fin d'un valuation  $\varphi$ telle que $\vv{P} = V$ : $\varphi(p) = V, \varphi(q)=F$ et  $\varphi(r)=V$.}
    \end{exampleblock}

\end{frame}



\end{document}  
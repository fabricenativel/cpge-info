\PassOptionsToPackage{dvipsnames,table}{xcolor}
\documentclass[10pt]{beamer}
\usepackage{Cours}

\begin{document}


\input{\detokenize{/home/fenarius/Travail/Cours/cpge-info/latex/MacrosCours.tex}}

% Numéro et titre de chapitre
\setcounter{numchap}{16}
\newcommand{\Ctitle}{\cnum {Logique}}
\newcommand{\SPATH}{/home/fenarius/Travail/Cours/cpge-info/docs/mp2i/files/C\thenumchap/}

\newcommand{\non}{\neg}
\newcommand{\et}{\wedge}
\newcommand{\ou}{\vee}
\newcommand{\imp}{\to}
\newcommand{\eq}{\leftrightarrow}
\newcommand{\lnode}[1]{\TCircle{$#1$}}
\psset{treesep=0.5cm,levelsep=0.8cm}

\makess{Syntaxe des formules logiques}
\begin{frame}{\Ctitle}{\stitle}
	\begin{alertblock}{Définition}
		Soit $V$ un ensemble au plus dénombrable de variables logiques noté $V = \{p, q, r, \dots \}$. On définit inductivement l'ensemble $P$ des formules logiques sur $V$ par :
        \begin{itemize}
            \item<2-> L'ensemble d'axiomes $P_0 = \{ \top, \bot \} \cup V$ \\
            \onslide<3->\textcolor{gray}{\small $\top$  se lit \og{} top \fg{} et $\bot$ se lit \og{} bottom \fg{}}
            \item<4-> Les règles d'inférence :
            \begin{itemize}
                \item<4-> \textit{négation} $\non : p \mapsto \non p$ 
                \item<5-> \textit{conjonction} $\et : p,q \mapsto (p \et q)$ 
                \item<6-> \textit{disjonction} $\ou : p,q \mapsto (p \ou q)$ 
                \item<7-> \textit{implication} $\imp : p,q \mapsto (p \imp q)$ 
                \item<8-> \textit{équivalence} $\eq : p,q \mapsto (p \eq q)$ 
            \end{itemize}
        \end{itemize}
	\end{alertblock}
\end{frame}

\begin{frame}{\Ctitle}{\stitle}
    \begin{block}{Remarques}
        \begin{itemize}
            \item Afin d'éviter de surcharger les écritures, on pourra omettre certaines parenthèses :
            \begin{itemize}
                \item<2-> En utilisant l'associativité à droite de $\et, \ou$ : \\
                \onslide<3-> \textcolor{gray}{Par exemple, $(p \ou (q \ou r))$ s'écrit plus simplement $p \ou q \ou r$.}
                \item<4-> En utilisant l'ordre de priorité suivant sur les connecteurs : $\non, \et, \ou, \imp, \eq$ \\
                \onslide<5-> \textcolor{gray}{Par exemple $ ((\non p) \ou (q \et r))$ s'écrit plus simplement $\non p \ou q \et r$.}
            \end{itemize}
            \onslide<6->{En cas de doute, on laissera les parenthèses afin de lever toute ambiguïté.}
            \item<7-> On a définit pour le moment simplement les propositions logiques valables d'un point de vue \textit{syntaxique}, sans leur donner de sens ou de valeur.
        \end{itemize}
    \end{block}
    \onslide<8->{
    \begin{exampleblock}{Exemples}
        \begin{itemize}
            \item<7-> $(((\non p) \ou (\non q)) \et r)$ est une formule logique qu'on pourra écrire plus simplement $(\non p \ou \non q) \et r$.
            \item<8-> $\et p \non pq$ ou encore $( p \et q) \imp (r$ ne sont pas des formules logiques.
            \item<9-> $(p \imp q) \et (q \imp p)$ et $ p \eq q$ sont deux formules logiques \textit{différentes}.
        \end{itemize}
    \end{exampleblock}}
\end{frame}

\begin{frame}{\Ctitle}{\stitle}
    \begin{block}{Arbre syntaxique}
        Les formules logiques admettent naturellement une représentation sous forme d'arbre :
        \begin{itemize}
            \item<2-> les variables logiques et les constantes $\top$ et $\bot$ sont les étiquettes des feuilles
            \item<3-> les noeuds internes ont pour étiquette les règles d'inférence
        \end{itemize}
    \end{block}
    \onslide<4->{
    \begin{exampleblock}{Exemples}
            La  formule logique $( p \imp q) \ou (\non r)$ admet la représentation : \\
            \onslide<5->{
                \begin{center}
                \pstree{\lnode{\ou}}{\pstree{\lnode{\imp}}{\lnode{p} \lnode{q}} \pstree{\lnode{\non}}{\lnode{r}}}
                \end{center}
                }
    \end{exampleblock}}
\end{frame}

\begin{frame}{\Ctitle}{\stitle}
    \begin{exampleblock}{Exemple}
        \begin{itemize}
        \item Quelle est la formule logique ayant pour représentation arborescente :
        \begin{center}
        \pstree{\lnode{\et}}{
            \pstree{\lnode{\ou}}{
                \pstree{\lnode{\imp}}{\lnode{p} \lnode{q}}
                \pstree{\lnode{\eq}}{\lnode{r} \lnode{s}}
                }
            \pstree{\lnode{\ou}}{
                \lnode{r}
                \pstree{\lnode{\non}}{\lnode{q}}
            }
        }
        \end{center}
        \onslide<2->\textcolor{OliveGreen}{$ ((p \imp q) \ou (r \eq s)) \et (r \ou (\non q)) $}
        \item<3-> Dessiner la représentation arborescente de $ \non(\top \eq (p \ou q))$.
    \end{itemize}
    \end{exampleblock}
\end{frame}

\begin{frame}{\Ctitle}{\stitle}
    \begin{block}{Hauteur, taille et sous formule}
        Etant donnée une formule logique notée $P$,
        \begin{itemize}
            \item<2-> la \textcolor{blue}{hauteur} de $P$ 
        \end{itemize}
    \end{block}
\end{frame}

\begin{frame}{\Ctitle}{\stitle}
    \begin{block}{Implémentation en OCaml}
        Le type somme de OCaml associé à la correspondance de motif permettent de représenter et de travailler efficacement sur les formules logiques
    \end{block}
\end{frame}
\end{document}
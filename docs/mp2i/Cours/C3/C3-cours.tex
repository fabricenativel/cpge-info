\PassOptionsToPackage{dvipsnames,table}{xcolor}
\documentclass[10pt]{beamer}
\usepackage{Cours}

\begin{document}


\newcounter{numchap}
\setcounter{numchap}{1}
\newcounter{numframe}
\setcounter{numframe}{0}
\newcommand{\mframe}[1]{\frametitle{#1} \addtocounter{numframe}{1}}
\newcommand{\cnum}{\fbox{\textcolor{yellow}{\textbf{C\thenumchap}}}~}
\newcommand{\makess}[1]{\section{#1} \label{ss\thesection}}
\newcommand{\stitle}{\textcolor{yellow}{\textbf{\thesection. \nameref{ss\thesection}}}}

\definecolor{codebg}{gray}{0.90}
\definecolor{grispale}{gray}{0.95}
\definecolor{fluo}{rgb}{1,0.96,0.62}
\newminted[langageC]{c}{linenos=true,escapeinside=||,highlightcolor=fluo,tabsize=2,breaklines=true}
\newminted[codepython]{python}{linenos=true,escapeinside=||,highlightcolor=fluo,tabsize=2,breaklines=true}
% Inclusion complète (ou partiel en indiquant premiere et dernière ligne) d'un fichier C
\newcommand{\inputC}[3]{\begin{mdframed}[backgroundcolor=codebg] \inputminted[breaklines=true,fontsize=#3,linenos=true,highlightcolor=fluo,tabsize=2,highlightlines={#2}]{c}{#1} \end{mdframed}}
\newcommand{\inputpartC}[5]{\begin{mdframed}[backgroundcolor=codebg] \inputminted[breaklines=true,fontsize=#3,linenos=true,highlightcolor=fluo,tabsize=2,highlightlines={#2},firstline=#4,lastline=#5,firstnumber=1]{c}{#1} \end{mdframed}}
\newcommand{\inputpython}[3]{\begin{mdframed}[backgroundcolor=codebg] \inputminted[breaklines=true,fontsize=#3,linenos=true,highlightcolor=fluo,tabsize=2,highlightlines={#2}]{python}{#1} \end{mdframed}}
\newcommand{\inputpartOCaml}[5]{\begin{mdframed}[backgroundcolor=codebg] \inputminted[breaklines=true,fontsize=#3,linenos=true,highlightcolor=fluo,tabsize=2,highlightlines={#2},firstline=#4,lastline=#5,firstnumber=1]{OCaml}{#1} \end{mdframed}}
\BeforeBeginEnvironment{minted}{\begin{mdframed}[backgroundcolor=codebg]}
\AfterEndEnvironment{minted}{\end{mdframed}}
\newcommand{\kw}[1]{\textcolor{blue}{\tt #1}}

\newtcolorbox{rcadre}[4]{halign=center,colback={#1},colframe={#2},width={#3cm},height={#4cm},valign=center,boxrule=1pt,left=0pt,right=0pt}
\newtcolorbox{cadre}[4]{halign=center,colback={#1},colframe={#2},arc=0mm,width={#3cm},height={#4cm},valign=center,boxrule=1pt,left=0pt,right=0pt}
\newcommand{\myem}[1]{\colorbox{fluo}{#1}}
\mdfsetup{skipabove=1pt,skipbelow=-2pt}



% Noeud dans un cadre pour les arbres
\newcommand{\noeud}[2]{\Tr{\fbox{\textcolor{#1}{\tt #2}}}}

\newcommand{\htmlmode}{\lstset{language=html,numbers=left, tabsize=4, frame=single, breaklines=true, keywordstyle=\ttfamily, basicstyle=\small,
   numberstyle=\tiny\ttfamily, framexleftmargin=0mm, backgroundcolor=\color{grispale}, xleftmargin=12mm,showstringspaces=false}}
\newcommand{\pythonmode}{\lstset{
   language=python,
   linewidth=\linewidth,
   numbers=left,
   tabsize=4,
   frame=single,
   breaklines=true,
   keywordstyle=\ttfamily\color{blue},
   basicstyle=\small,
   numberstyle=\tiny\ttfamily,
   framexleftmargin=-2mm,
   numbersep=-0.5mm,
   backgroundcolor=\color{codebg},
   xleftmargin=-1mm, 
   showstringspaces=false,
   commentstyle=\color{gray},
   stringstyle=\color{OliveGreen},
   emph={turtle,Screen,Turtle},
   emphstyle=\color{RawSienna},
   morekeywords={setheading,goto,backward,forward,left,right,pendown,penup,pensize,color,speed,hideturtle,showturtle,forward}}
   }
   \newcommand{\Cmode}{\lstset{
      language=[ANSI]C,
      linewidth=\linewidth,
      numbers=left,
      tabsize=4,
      frame=single,
      breaklines=true,
      keywordstyle=\ttfamily\color{blue},
      basicstyle=\small,
      numberstyle=\tiny\ttfamily,
      framexleftmargin=0mm,
      numbersep=2mm,
      backgroundcolor=\color{codebg},
      xleftmargin=0mm, 
      showstringspaces=false,
      commentstyle=\color{gray},
      stringstyle=\color{OliveGreen},
      emphstyle=\color{RawSienna},
      escapechar=\|,
      morekeywords={}}
      }
\newcommand{\bashmode}{\lstset{language=bash,numbers=left, tabsize=2, frame=single, breaklines=true, basicstyle=\ttfamily,
   numberstyle=\tiny\ttfamily, framexleftmargin=0mm, backgroundcolor=\color{grispale}, xleftmargin=12mm, showstringspaces=false}}
\newcommand{\exomode}{\lstset{language=python,numbers=left, tabsize=2, frame=single, breaklines=true, basicstyle=\ttfamily,
   numberstyle=\tiny\ttfamily, framexleftmargin=13mm, xleftmargin=12mm, basicstyle=\small, showstringspaces=false}}
   
   
  
%tei pour placer les images
%tei{nom de l’image}{échelle de l’image}{sens}{texte a positionner}
%sens ="1" (droite) ou "2" (gauche)
\newlength{\ltxt}
\newcommand{\tei}[4]{
\setlength{\ltxt}{\linewidth}
\setbox0=\hbox{\includegraphics[scale=#2]{#1}}
\addtolength{\ltxt}{-\wd0}
\addtolength{\ltxt}{-10pt}
\ifthenelse{\equal{#3}{1}}{
\begin{minipage}{\wd0}
\includegraphics[scale=#2]{#1}
\end{minipage}
\hfill
\begin{minipage}{\ltxt}
#4
\end{minipage}
}{
\begin{minipage}{\ltxt}
#4
\end{minipage}
\hfill
\begin{minipage}{\wd0}
\includegraphics[scale=#2]{#1}
\end{minipage}
}
}

%Juxtaposition d'une image pspciture et de texte 
%#1: = code pstricks de l'image
%#2: largeur de l'image
%#3: hauteur de l'image
%#4: Texte à écrire
\newcommand{\ptp}[4]{
\setlength{\ltxt}{\linewidth}
\addtolength{\ltxt}{-#2 cm}
\addtolength{\ltxt}{-0.1 cm}
\begin{minipage}[b][#3 cm][t]{\ltxt}
#4
\end{minipage}\hfill
\begin{minipage}[b][#3 cm][c]{#2 cm}
#1
\end{minipage}\par
}



%Macros pour les graphiques
\psset{linewidth=0.5\pslinewidth,PointSymbol=x}
\setlength{\fboxrule}{0.5pt}
\newcounter{tempangle}

%Marque la longueur du segment d'extrémité  #1 et  #2 avec la valeur #3, #4 est la distance par rapport au segment (en %age de la valeur de celui ci) et #5 l'orientation du marquage : +90 ou -90
\newcommand{\afflong}[5]{
\pstRotation[RotAngle=#4,PointSymbol=none,PointName=none]{#1}{#2}[X] 
\pstHomO[PointSymbol=none,PointName=none,HomCoef=#5]{#1}{X}[Y]
\pstTranslation[PointSymbol=none,PointName=none]{#1}{#2}{Y}[Z]
 \ncline{|<->|,linewidth=0.25\pslinewidth}{Y}{Z} \ncput*[nrot=:U]{\footnotesize{#3}}
}
\newcommand{\afflongb}[3]{
\ncline{|<->|,linewidth=0}{#1}{#2} \naput*[nrot=:U]{\footnotesize{#3}}
}

%Construis le point #4 situé à #2 cm du point #1 avant un angle #3 par rapport à l'horizontale. #5 = liste de paramètre
\newcommand{\lsegment}[5]{\pstGeonode[PointSymbol=none,PointName=none](0,0){O'}(#2,0){I'} \pstTranslation[PointSymbol=none,PointName=none]{O'}{I'}{#1}[J'] \pstRotation[RotAngle=#3,PointSymbol=x,#5]{#1}{J'}[#4]}
\newcommand{\tsegment}[5]{\pstGeonode[PointSymbol=none,PointName=none](0,0){O'}(#2,0){I'} \pstTranslation[PointSymbol=none,PointName=none]{O'}{I'}{#1}[J'] \pstRotation[RotAngle=#3,PointSymbol=x,#5]{#1}{J'}[#4] \pstLineAB{#4}{#1}}

%Construis le point #4 situé à #3 cm du point #1 et faisant un angle de  90° avec la droite (#1,#2) #5 = liste de paramètre
\newcommand{\psegment}[5]{
\pstGeonode[PointSymbol=none,PointName=none](0,0){O'}(#3,0){I'}
 \pstTranslation[PointSymbol=none,PointName=none]{O'}{I'}{#1}[J']
 \pstInterLC[PointSymbol=none,PointName=none]{#1}{#2}{#1}{J'}{M1}{M2} \pstRotation[RotAngle=-90,PointSymbol=x,#5]{#1}{M1}[#4]
  }
  
%Construis le point #4 situé à #3 cm du point #1 et faisant un angle de  #5° avec la droite (#1,#2) #6 = liste de paramètre
\newcommand{\mlogo}[6]{
\pstGeonode[PointSymbol=none,PointName=none](0,0){O'}(#3,0){I'}
 \pstTranslation[PointSymbol=none,PointName=none]{O'}{I'}{#1}[J']
 \pstInterLC[PointSymbol=none,PointName=none]{#1}{#2}{#1}{J'}{M1}{M2} \pstRotation[RotAngle=#5,PointSymbol=x,#6]{#1}{M2}[#4]
  }

% Construis un triangle avec #1=liste des 3 sommets séparés par des virgules, #2=liste des 3 longueurs séparés par des virgules, #3 et #4 : paramètre d'affichage des 2e et 3 points et #5 : inclinaison par rapport à l'horizontale
%autre macro identique mais sans tracer les segments joignant les sommets
\noexpandarg
\newcommand{\Triangleccc}[5]{
\StrBefore{#1}{,}[\pointA]
\StrBetween[1,2]{#1}{,}{,}[\pointB]
\StrBehind[2]{#1}{,}[\pointC]
\StrBefore{#2}{,}[\coteA]
\StrBetween[1,2]{#2}{,}{,}[\coteB]
\StrBehind[2]{#2}{,}[\coteC]
\tsegment{\pointA}{\coteA}{#5}{\pointB}{#3} 
\lsegment{\pointA}{\coteB}{0}{Z1}{PointSymbol=none, PointName=none}
\lsegment{\pointB}{\coteC}{0}{Z2}{PointSymbol=none, PointName=none}
\pstInterCC{\pointA}{Z1}{\pointB}{Z2}{\pointC}{Z3} 
\pstLineAB{\pointA}{\pointC} \pstLineAB{\pointB}{\pointC}
\pstSymO[PointName=\pointC,#4]{C}{C}[C]
}
\noexpandarg
\newcommand{\TrianglecccP}[5]{
\StrBefore{#1}{,}[\pointA]
\StrBetween[1,2]{#1}{,}{,}[\pointB]
\StrBehind[2]{#1}{,}[\pointC]
\StrBefore{#2}{,}[\coteA]
\StrBetween[1,2]{#2}{,}{,}[\coteB]
\StrBehind[2]{#2}{,}[\coteC]
\tsegment{\pointA}{\coteA}{#5}{\pointB}{#3} 
\lsegment{\pointA}{\coteB}{0}{Z1}{PointSymbol=none, PointName=none}
\lsegment{\pointB}{\coteC}{0}{Z2}{PointSymbol=none, PointName=none}
\pstInterCC[PointNameB=none,PointSymbolB=none,#4]{\pointA}{Z1}{\pointB}{Z2}{\pointC}{Z1} 
}


% Construis un triangle avec #1=liste des 3 sommets séparés par des virgules, #2=liste formée de 2 longueurs et d'un angle séparés par des virgules, #3 et #4 : paramètre d'affichage des 2e et 3 points et #5 : inclinaison par rapport à l'horizontale
%autre macro identique mais sans tracer les segments joignant les sommets
\newcommand{\Trianglecca}[5]{
\StrBefore{#1}{,}[\pointA]
\StrBetween[1,2]{#1}{,}{,}[\pointB]
\StrBehind[2]{#1}{,}[\pointC]
\StrBefore{#2}{,}[\coteA]
\StrBetween[1,2]{#2}{,}{,}[\coteB]
\StrBehind[2]{#2}{,}[\angleA]
\tsegment{\pointA}{\coteA}{#5}{\pointB}{#3} 
\setcounter{tempangle}{#5}
\addtocounter{tempangle}{\angleA}
\tsegment{\pointA}{\coteB}{\thetempangle}{\pointC}{#4}
\pstLineAB{\pointB}{\pointC}
}
\newcommand{\TriangleccaP}[5]{
\StrBefore{#1}{,}[\pointA]
\StrBetween[1,2]{#1}{,}{,}[\pointB]
\StrBehind[2]{#1}{,}[\pointC]
\StrBefore{#2}{,}[\coteA]
\StrBetween[1,2]{#2}{,}{,}[\coteB]
\StrBehind[2]{#2}{,}[\angleA]
\lsegment{\pointA}{\coteA}{#5}{\pointB}{#3} 
\setcounter{tempangle}{#5}
\addtocounter{tempangle}{\angleA}
\lsegment{\pointA}{\coteB}{\thetempangle}{\pointC}{#4}
}

% Construis un triangle avec #1=liste des 3 sommets séparés par des virgules, #2=liste formée de 1 longueurs et de deux angle séparés par des virgules, #3 et #4 : paramètre d'affichage des 2e et 3 points et #5 : inclinaison par rapport à l'horizontale
%autre macro identique mais sans tracer les segments joignant les sommets
\newcommand{\Trianglecaa}[5]{
\StrBefore{#1}{,}[\pointA]
\StrBetween[1,2]{#1}{,}{,}[\pointB]
\StrBehind[2]{#1}{,}[\pointC]
\StrBefore{#2}{,}[\coteA]
\StrBetween[1,2]{#2}{,}{,}[\angleA]
\StrBehind[2]{#2}{,}[\angleB]
\tsegment{\pointA}{\coteA}{#5}{\pointB}{#3} 
\setcounter{tempangle}{#5}
\addtocounter{tempangle}{\angleA}
\lsegment{\pointA}{1}{\thetempangle}{Z1}{PointSymbol=none, PointName=none}
\setcounter{tempangle}{#5}
\addtocounter{tempangle}{180}
\addtocounter{tempangle}{-\angleB}
\lsegment{\pointB}{1}{\thetempangle}{Z2}{PointSymbol=none, PointName=none}
\pstInterLL[#4]{\pointA}{Z1}{\pointB}{Z2}{\pointC}
\pstLineAB{\pointA}{\pointC}
\pstLineAB{\pointB}{\pointC}
}
\newcommand{\TrianglecaaP}[5]{
\StrBefore{#1}{,}[\pointA]
\StrBetween[1,2]{#1}{,}{,}[\pointB]
\StrBehind[2]{#1}{,}[\pointC]
\StrBefore{#2}{,}[\coteA]
\StrBetween[1,2]{#2}{,}{,}[\angleA]
\StrBehind[2]{#2}{,}[\angleB]
\lsegment{\pointA}{\coteA}{#5}{\pointB}{#3} 
\setcounter{tempangle}{#5}
\addtocounter{tempangle}{\angleA}
\lsegment{\pointA}{1}{\thetempangle}{Z1}{PointSymbol=none, PointName=none}
\setcounter{tempangle}{#5}
\addtocounter{tempangle}{180}
\addtocounter{tempangle}{-\angleB}
\lsegment{\pointB}{1}{\thetempangle}{Z2}{PointSymbol=none, PointName=none}
\pstInterLL[#4]{\pointA}{Z1}{\pointB}{Z2}{\pointC}
}

%Construction d'un cercle de centre #1 et de rayon #2 (en cm)
\newcommand{\Cercle}[2]{
\lsegment{#1}{#2}{0}{Z1}{PointSymbol=none, PointName=none}
\pstCircleOA{#1}{Z1}
}

%construction d'un parallélogramme #1 = liste des sommets, #2 = liste contenant les longueurs de 2 côtés consécutifs et leurs angles;  #3, #4 et #5 : paramètre d'affichage des sommets #6 inclinaison par rapport à l'horizontale 
% meme macro sans le tracé des segements
\newcommand{\Para}[6]{
\StrBefore{#1}{,}[\pointA]
\StrBetween[1,2]{#1}{,}{,}[\pointB]
\StrBetween[2,3]{#1}{,}{,}[\pointC]
\StrBehind[3]{#1}{,}[\pointD]
\StrBefore{#2}{,}[\longueur]
\StrBetween[1,2]{#2}{,}{,}[\largeur]
\StrBehind[2]{#2}{,}[\angle]
\tsegment{\pointA}{\longueur}{#6}{\pointB}{#3} 
\setcounter{tempangle}{#6}
\addtocounter{tempangle}{\angle}
\tsegment{\pointA}{\largeur}{\thetempangle}{\pointD}{#5}
\pstMiddleAB[PointName=none,PointSymbol=none]{\pointB}{\pointD}{Z1}
\pstSymO[#4]{Z1}{\pointA}[\pointC]
\pstLineAB{\pointB}{\pointC}
\pstLineAB{\pointC}{\pointD}
}
\newcommand{\ParaP}[6]{
\StrBefore{#1}{,}[\pointA]
\StrBetween[1,2]{#1}{,}{,}[\pointB]
\StrBetween[2,3]{#1}{,}{,}[\pointC]
\StrBehind[3]{#1}{,}[\pointD]
\StrBefore{#2}{,}[\longueur]
\StrBetween[1,2]{#2}{,}{,}[\largeur]
\StrBehind[2]{#2}{,}[\angle]
\lsegment{\pointA}{\longueur}{#6}{\pointB}{#3} 
\setcounter{tempangle}{#6}
\addtocounter{tempangle}{\angle}
\lsegment{\pointA}{\largeur}{\thetempangle}{\pointD}{#5}
\pstMiddleAB[PointName=none,PointSymbol=none]{\pointB}{\pointD}{Z1}
\pstSymO[#4]{Z1}{\pointA}[\pointC]
}


%construction d'un cerf-volant #1 = liste des sommets, #2 = liste contenant les longueurs de 2 côtés consécutifs et leurs angles;  #3, #4 et #5 : paramètre d'affichage des sommets #6 inclinaison par rapport à l'horizontale 
% meme macro sans le tracé des segements
\newcommand{\CerfVolant}[6]{
\StrBefore{#1}{,}[\pointA]
\StrBetween[1,2]{#1}{,}{,}[\pointB]
\StrBetween[2,3]{#1}{,}{,}[\pointC]
\StrBehind[3]{#1}{,}[\pointD]
\StrBefore{#2}{,}[\longueur]
\StrBetween[1,2]{#2}{,}{,}[\largeur]
\StrBehind[2]{#2}{,}[\angle]
\tsegment{\pointA}{\longueur}{#6}{\pointB}{#3} 
\setcounter{tempangle}{#6}
\addtocounter{tempangle}{\angle}
\tsegment{\pointA}{\largeur}{\thetempangle}{\pointD}{#5}
\pstOrtSym[#4]{\pointB}{\pointD}{\pointA}[\pointC]
\pstLineAB{\pointB}{\pointC}
\pstLineAB{\pointC}{\pointD}
}

%construction d'un quadrilatère quelconque #1 = liste des sommets, #2 = liste contenant les longueurs des 4 côtés et l'angle entre 2 cotés consécutifs  #3, #4 et #5 : paramètre d'affichage des sommets #6 inclinaison par rapport à l'horizontale 
% meme macro sans le tracé des segements
\newcommand{\Quadri}[6]{
\StrBefore{#1}{,}[\pointA]
\StrBetween[1,2]{#1}{,}{,}[\pointB]
\StrBetween[2,3]{#1}{,}{,}[\pointC]
\StrBehind[3]{#1}{,}[\pointD]
\StrBefore{#2}{,}[\coteA]
\StrBetween[1,2]{#2}{,}{,}[\coteB]
\StrBetween[2,3]{#2}{,}{,}[\coteC]
\StrBetween[3,4]{#2}{,}{,}[\coteD]
\StrBehind[4]{#2}{,}[\angle]
\tsegment{\pointA}{\coteA}{#6}{\pointB}{#3} 
\setcounter{tempangle}{#6}
\addtocounter{tempangle}{\angle}
\tsegment{\pointA}{\coteD}{\thetempangle}{\pointD}{#5}
\lsegment{\pointB}{\coteB}{0}{Z1}{PointSymbol=none, PointName=none}
\lsegment{\pointD}{\coteC}{0}{Z2}{PointSymbol=none, PointName=none}
\pstInterCC[PointNameA=none,PointSymbolA=none,#4]{\pointB}{Z1}{\pointD}{Z2}{Z3}{\pointC} 
\pstLineAB{\pointB}{\pointC}
\pstLineAB{\pointC}{\pointD}
}


% Définition des colonnes centrées ou à droite pour tabularx
\newcolumntype{Y}{>{\centering\arraybackslash}X}
\newcolumntype{Z}{>{\flushright\arraybackslash}X}

%Les pointillés à remplir par les élèves
\newcommand{\po}[1]{\makebox[#1 cm]{\dotfill}}
\newcommand{\lpo}[1][3]{%
\multido{}{#1}{\makebox[\linewidth]{\dotfill}
}}

%Liste des pictogrammes utilisés sur la fiche d'exercice ou d'activités
\newcommand{\bombe}{\faBomb}
\newcommand{\livre}{\faBook}
\newcommand{\calculatrice}{\faCalculator}
\newcommand{\oral}{\faCommentO}
\newcommand{\surfeuille}{\faEdit}
\newcommand{\ordinateur}{\faLaptop}
\newcommand{\ordi}{\faDesktop}
\newcommand{\ciseaux}{\faScissors}
\newcommand{\danger}{\faExclamationTriangle}
\newcommand{\out}{\faSignOut}
\newcommand{\cadeau}{\faGift}
\newcommand{\flash}{\faBolt}
\newcommand{\lumiere}{\faLightbulb}
\newcommand{\compas}{\dsmathematical}
\newcommand{\calcullitteral}{\faTimesCircleO}
\newcommand{\raisonnement}{\faCogs}
\newcommand{\recherche}{\faSearch}
\newcommand{\rappel}{\faHistory}
\newcommand{\video}{\faFilm}
\newcommand{\capacite}{\faPuzzlePiece}
\newcommand{\aide}{\faLifeRing}
\newcommand{\loin}{\faExternalLink}
\newcommand{\groupe}{\faUsers}
\newcommand{\bac}{\faGraduationCap}
\newcommand{\histoire}{\faUniversity}
\newcommand{\coeur}{\faSave}
\newcommand{\python}{\faPython}
\newcommand{\os}{\faMicrochip}
\newcommand{\rd}{\faCubes}
\newcommand{\data}{\faColumns}
\newcommand{\web}{\faCode}
\newcommand{\prog}{\faFile}
\newcommand{\algo}{\faCogs}
\newcommand{\important}{\faExclamationCircle}
\newcommand{\maths}{\faTimesCircle}
% Traitement des données en tables
\newcommand{\tables}{\faColumns}
% Types construits
\newcommand{\construits}{\faCubes}
% Type et valeurs de base
\newcommand{\debase}{{\footnotesize \faCube}}
% Systèmes d'exploitation
\newcommand{\linux}{\faLinux}
\newcommand{\sd}{\faProjectDiagram}
\newcommand{\bd}{\faDatabase}

%Les ensembles de nombres
\renewcommand{\N}{\mathbb{N}}
\newcommand{\D}{\mathbb{D}}
\newcommand{\Z}{\mathbb{Z}}
\newcommand{\Q}{\mathbb{Q}}
\newcommand{\R}{\mathbb{R}}
\newcommand{\C}{\mathbb{C}}

%Ecriture des vecteurs
\newcommand{\vect}[1]{\vbox{\halign{##\cr 
  \tiny\rightarrowfill\cr\noalign{\nointerlineskip\vskip1pt} 
  $#1\mskip2mu$\cr}}}


%Compteur activités/exos et question et mise en forme titre et questions
\newcounter{numact}
\setcounter{numact}{1}
\newcounter{numseance}
\setcounter{numseance}{1}
\newcounter{numexo}
\setcounter{numexo}{0}
\newcounter{numprojet}
\setcounter{numprojet}{0}
\newcounter{numquestion}
\newcommand{\espace}[1]{\rule[-1ex]{0pt}{#1 cm}}
\newcommand{\Quest}[3]{
\addtocounter{numquestion}{1}
\begin{tabularx}{\textwidth}{X|m{1cm}|}
\cline{2-2}
\textbf{\sffamily{\alph{numquestion})}} #1 & \dots / #2 \\
\hline 
\multicolumn{2}{|l|}{\espace{#3}} \\
\hline
\end{tabularx}
}
\newcommand{\QuestR}[3]{
\addtocounter{numquestion}{1}
\begin{tabularx}{\textwidth}{X|m{1cm}|}
\cline{2-2}
\textbf{\sffamily{\alph{numquestion})}} #1 & \dots / #2 \\
\hline 
\multicolumn{2}{|l|}{\cor{#3}} \\
\hline
\end{tabularx}
}
\newcommand{\Pre}{{\sc nsi} 1\textsuperscript{e}}
\newcommand{\Term}{{\sc nsi} Terminale}
\newcommand{\Sec}{2\textsuperscript{e}}
\newcommand{\Exo}[2]{ \addtocounter{numexo}{1} \ding{113} \textbf{\sffamily{Exercice \thenumexo}} : \textit{#1} \hfill #2  \setcounter{numquestion}{0}}
\newcommand{\Projet}[1]{ \addtocounter{numprojet}{1} \ding{118} \textbf{\sffamily{Projet \thenumprojet}} : \textit{#1}}
\newcommand{\ExoD}[2]{ \addtocounter{numexo}{1} \ding{113} \textbf{\sffamily{Exercice \thenumexo}}  \textit{(#1 pts)} \hfill #2  \setcounter{numquestion}{0}}
\newcommand{\ExoB}[2]{ \addtocounter{numexo}{1} \ding{113} \textbf{\sffamily{Exercice \thenumexo}}  \textit{(Bonus de +#1 pts maximum)} \hfill #2  \setcounter{numquestion}{0}}
\newcommand{\Act}[2]{ \ding{113} \textbf{\sffamily{Activité \thenumact}} : \textit{#1} \hfill #2  \addtocounter{numact}{1} \setcounter{numquestion}{0}}
\newcommand{\Seance}{ \rule{1.5cm}{0.5pt}\raisebox{-3pt}{\framebox[4cm]{\textbf{\sffamily{Séance \thenumseance}}}}\hrulefill  \\
  \addtocounter{numseance}{1}}
\newcommand{\Acti}[2]{ {\footnotesize \ding{117}} \textbf{\sffamily{Activité \thenumact}} : \textit{#1} \hfill #2  \addtocounter{numact}{1} \setcounter{numquestion}{0}}
\newcommand{\titre}[1]{\begin{Large}\textbf{\ding{118}}\end{Large} \begin{large}\textbf{ #1}\end{large} \vspace{0.2cm}}
\newcommand{\QListe}[1][0]{
\ifthenelse{#1=0}
{\begin{enumerate}[partopsep=0pt,topsep=0pt,parsep=0pt,itemsep=0pt,label=\textbf{\sffamily{\arabic*.}},series=question]}
{\begin{enumerate}[resume*=question]}}
\newcommand{\SQListe}[1][0]{
\ifthenelse{#1=0}
{\begin{enumerate}[partopsep=0pt,topsep=0pt,parsep=0pt,itemsep=0pt,label=\textbf{\sffamily{\alph*)}},series=squestion]}
{\begin{enumerate}[resume*=squestion]}}
\newcommand{\SQListeL}[1][0]{
\ifthenelse{#1=0}
{\begin{enumerate*}[partopsep=0pt,topsep=0pt,parsep=0pt,itemsep=0pt,label=\textbf{\sffamily{\alph*)}},series=squestion]}
{\begin{enumerate*}[resume*=squestion]}}
\newcommand{\FinListe}{\end{enumerate}}
\newcommand{\FinListeL}{\end{enumerate*}}

%Mise en forme de la correction
\newcommand{\cor}[1]{\par \textcolor{OliveGreen}{#1}}
\newcommand{\br}[1]{\cor{\textbf{#1}}}
\newcommand{\tcor}[1]{\begin{tcolorbox}[width=0.92\textwidth,colback={white},colbacktitle=white,coltitle=OliveGreen,colframe=green!75!black,boxrule=0.2mm]   
\cor{#1}
\end{tcolorbox}
}
\newcommand{\rc}[1]{\textcolor{OliveGreen}{#1}}
\newcommand{\pmc}[1]{\textcolor{blue}{\tt #1}}
\newcommand{\tmc}[1]{\textcolor{RawSienna}{\tt #1}}


%Référence aux exercices par leur numéro
\newcommand{\refexo}[1]{
\refstepcounter{numexo}
\addtocounter{numexo}{-1}
\label{#1}}

%Séparation entre deux activités
\newcommand{\separateur}{\begin{center}
\rule{1.5cm}{0.5pt}\raisebox{-3pt}{\ding{117}}\rule{1.5cm}{0.5pt}  \vspace{0.2cm}
\end{center}}

%Entête et pied de page
\newcommand{\snt}[1]{\lhead{\textbf{SNT -- La photographie numérique} \rhead{\textit{Lycée Nord}}}}
\newcommand{\Activites}[2]{\lhead{\textbf{{\sc #1}}}
\rhead{Activités -- \textbf{#2}}
\cfoot{}}
\newcommand{\Exos}[2]{\lhead{\textbf{Fiche d'exercices: {\sc #1}}}
\rhead{Niveau: \textbf{#2}}
\cfoot{}}
\newcommand{\Devoir}[2]{\lhead{\textbf{Devoir de mathématiques : {\sc #1}}}
\rhead{\textbf{#2}} \setlength{\fboxsep}{8pt}
\begin{center}
%Titre de la fiche
\fbox{\parbox[b][1cm][t]{0.3\textwidth}{Nom : \hfill \po{3} \par \vfill Prénom : \hfill \po{3}} } \hfill 
\fbox{\parbox[b][1cm][t]{0.6\textwidth}{Note : \po{1} / 20} }
\end{center} \cfoot{}}

%Devoir programmation en NSI (pas à rendre sur papier)
\newcommand{\PNSI}[2]{\lhead{\textbf{Devoir de {\sc nsi} : \textsf{ #1}}}
\rhead{\textbf{#2}} \setlength{\fboxsep}{8pt}
\begin{tcolorbox}[title=\textcolor{black}{\danger\; A lire attentivement},colbacktitle=lightgray]
{\begin{enumerate}
\item Rendre tous vous programmes en les envoyant par mail à l'adresse {\tt fnativel2@ac-reunion.fr}, en précisant bien dans le sujet vos noms et prénoms
\item Un programme qui fonctionne mal ou pas du tout peut rapporter des points
\item Les bonnes pratiques de programmation (clarté et lisiblité du code) rapportent des points
\end{enumerate}
}
\end{tcolorbox}
 \cfoot{}}


%Devoir de NSI
\newcommand{\DNSI}[2]{\lhead{\textbf{Devoir de {\sc nsi} : \textsf{ #1}}}
\rhead{\textbf{#2}} \setlength{\fboxsep}{8pt}
\begin{center}
%Titre de la fiche
\fbox{\parbox[b][1cm][t]{0.3\textwidth}{Nom : \hfill \po{3} \par \vfill Prénom : \hfill \po{3}} } \hfill 
\fbox{\parbox[b][1cm][t]{0.6\textwidth}{Note : \po{1} / 10} }
\end{center} \cfoot{}}

\newcommand{\DevoirNSI}[2]{\lhead{\textbf{Devoir de {\sc nsi} : {\sc #1}}}
\rhead{\textbf{#2}} \setlength{\fboxsep}{8pt}
\cfoot{}}

%La définition de la commande QCM pour auto-multiple-choice
%En premier argument le sujet du qcm, deuxième argument : la classe, 3e : la durée prévue et #4 : présence ou non de questions avec plusieurs bonnes réponses
\newcommand{\QCM}[4]{
{\large \textbf{\ding{52} QCM : #1}} -- Durée : \textbf{#3 min} \hfill {\large Note : \dots/10} 
\hrule \vspace{0.1cm}\namefield{}
Nom :  \textbf{\textbf{\nom{}}} \qquad \qquad Prénom :  \textbf{\prenom{}}  \hfill Classe: \textbf{#2}
\vspace{0.2cm}
\hrule  
\begin{itemize}[itemsep=0pt]
\item[-] \textit{Une bonne réponse vaut un point, une absence de réponse n'enlève pas de point. }
\item[\danger] \textit{Une mauvaise réponse enlève un point.}
\ifthenelse{#4=1}{\item[-] \textit{Les questions marquées du symbole \multiSymbole{} peuvent avoir plusieurs bonnes réponses possibles.}}{}
\end{itemize}
}
\newcommand{\DevoirC}[2]{
\renewcommand{\footrulewidth}{0.5pt}
\lhead{\textbf{Devoir de mathématiques : {\sc #1}}}
\rhead{\textbf{#2}} \setlength{\fboxsep}{8pt}
\fbox{\parbox[b][0.4cm][t]{0.955\textwidth}{Nom : \po{5} \hfill Prénom : \po{5} \hfill Classe: \textbf{1}\textsuperscript{$\dots$}} } 
\rfoot{\thepage} \cfoot{} \lfoot{Lycée Nord}}
\newcommand{\DevoirInfo}[2]{\lhead{\textbf{Evaluation : {\sc #1}}}
\rhead{\textbf{#2}} \setlength{\fboxsep}{8pt}
 \cfoot{}}
\newcommand{\DM}[2]{\lhead{\textbf{Devoir maison à rendre le #1}} \rhead{\textbf{#2}}}

%Macros permettant l'affichage des touches de la calculatrice
%Touches classiques : #1 = 0 fond blanc pour les nombres et #1= 1gris pour les opérations et entrer, second paramètre=contenu
%Si #2=1 touche arrondi avec fond gris
\newcommand{\TCalc}[2]{
\setlength{\fboxsep}{0.1pt}
\ifthenelse{#1=0}
{\psframebox[fillstyle=solid, fillcolor=white]{\parbox[c][0.25cm][c]{0.6cm}{\centering #2}}}
{\ifthenelse{#1=1}
{\psframebox[fillstyle=solid, fillcolor=lightgray]{\parbox[c][0.25cm][c]{0.6cm}{\centering #2}}}
{\psframebox[framearc=.5,fillstyle=solid, fillcolor=white]{\parbox[c][0.25cm][c]{0.6cm}{\centering #2}}}
}}
\newcommand{\Talpha}{\psdblframebox[fillstyle=solid, fillcolor=white]{\hspace{-0.05cm}\parbox[c][0.25cm][c]{0.65cm}{\centering \scriptsize{alpha}}} \;}
\newcommand{\Tsec}{\psdblframebox[fillstyle=solid, fillcolor=white]{\parbox[c][0.25cm][c]{0.6cm}{\centering \scriptsize 2nde}} \;}
\newcommand{\Tfx}{\psdblframebox[fillstyle=solid, fillcolor=white]{\parbox[c][0.25cm][c]{0.6cm}{\centering \scriptsize $f(x)$}} \;}
\newcommand{\Tvar}{\psframebox[framearc=.5,fillstyle=solid, fillcolor=white]{\hspace{-0.22cm} \parbox[c][0.25cm][c]{0.82cm}{$\scriptscriptstyle{X,T,\theta,n}$}}}
\newcommand{\Tgraphe}{\psdblframebox[fillstyle=solid, fillcolor=white]{\hspace{-0.08cm}\parbox[c][0.25cm][c]{0.68cm}{\centering \tiny{graphe}}} \;}
\newcommand{\Tfen}{\psdblframebox[fillstyle=solid, fillcolor=white]{\hspace{-0.08cm}\parbox[c][0.25cm][c]{0.68cm}{\centering \tiny{fenêtre}}} \;}
\newcommand{\Ttrace}{\psdblframebox[fillstyle=solid, fillcolor=white]{\parbox[c][0.25cm][c]{0.6cm}{\centering \scriptsize{trace}}} \;}

% Macroi pour l'affichage  d'un entier n dans  une base b
\newcommand{\base}[2]{ \overline{#1}^{#2}}
% Intervalle d'entiers
\newcommand{\intN}[2]{\llbracket #1; #2 \rrbracket}}
% Numéro et titre de chapitre
\setcounter{numchap}{3}
\newcommand{\Ctitle}{\cnum {Pointeurs, types structurés}}

\makess{Mémoire en C}
\begin{frame}{\Ctitle}{\stitle}
	\begin{block}{Introduction}
		\begin{itemize}
			\item<1-> De façon schématique, la mémoire d'un ordinateur s'apparente à un immense tableau dont chaque case a une adresse.
			\item<2-> Un \textcolor{blue}{pointeur} est une variable contenant une de ces adresses.
			\item<3-> Le schéma ci-dessous représente un pointeur \kw{p} contenant l'adresse d'une variable \kw{x} qui vaut 42. \\
				\onslide<4->{
					\renewcommand{\arraystretch}{1.3}
					\begin{tabular}{ccccccccc}
						\textcolor{blue}{variable} &       &                                      & \kw{x}                                                                                &                                      &       &       & \kw{p}                                                                        &       \\
						\cline{3-5} \cline{8-8}
						\cdashline{2-2} \cdashline{6-7} \cdashline{9-9}
						valeur                     &       & \multicolumn{1}{|c|}{\phantom{4242}} & \multicolumn{1}{|c|}{\rnode{x}{42}}                                                   & \multicolumn{1}{|c|}{\phantom{4242}} &       &       & \multicolumn{1}{|c|}{\ovalnode[linecolor=BrickRed,linewidth=0.7pt]{pp}{2741}} &       \\
						\cline{3-5} \cline{8-8}
						\cdashline{2-2} \cdashline{6-7} \cdashline{9-9}
						\textcolor{Sepia}{adresse} & \dots & \textcolor{Sepia}{\scriptsize 2740}  & \ovalnode[linecolor=BrickRed,linewidth=0.7pt]{p}{\textcolor{Sepia}{\scriptsize 2741}} & \textcolor{Sepia}{\scriptsize 2742}  & \dots & \dots & \textcolor{Sepia}{\scriptsize 3154}                                           & \dots \\
					\end{tabular}
					\ncline[linecolor=BrickRed,linewidth=1pt]{->}{p}{x}
				}
			\item<5-> La valeur particulière \kw{NULL}  indique qu'un pointeur ne pointe sur rien.
			\item<6-> En C, la gestion de la mémoire n'est pas totalement automatique (comme en Python ou en OCaml). Certains aspects reviennent au programmeur, ce qui impose de comprendre le modèle mémoire du C.
		\end{itemize}
	\end{block}
\end{frame}




\begin{frame}{\Ctitle}{\stitle}
	\begin{block}{Schéma de l'organisation de la mémoire en C}
		\vspace{5.8cm}
	\end{block}
\end{frame}

\begin{frame}{\Ctitle}{\stitle}
	\begin{block}{Schéma de l'organisation de la mémoire en C}
		\vspace{5.8cm}
		\rput(1,0.0){\rnode{bas}{}}
		\rput(1,5.9){\rnode{haut}{}}
		\ncline[linecolor=gray,linestyle=dashed]{->}{bas}{haut}
		\naput[nrot=:U,labelsep=0.1]{\textcolor{gray}{\scriptsize adresses croissantes}}
	\end{block}
\end{frame}

\begin{frame}{\Ctitle}{\stitle}
	\begin{block}{Schéma de l'organisation de la mémoire en C}
		\vspace{5.8cm}
		\rput(1,0.0){\rnode{bas}{}}
		\rput(1,5.9){\rnode{haut}{}}
		\ncline[linecolor=gray,linestyle=dashed]{->}{bas}{haut}
		\naput[nrot=:U,labelsep=0.1]{\textcolor{gray}{\scriptsize adresses croissantes}}
		\rput(4.4,0.3){\rnode{code}{\begin{cadre}{codebg}{OliveGreen}{4.8}{0.6}   Code compilé \end{cadre}}}
		\rput(9.5,0.3){\parbox{6cm}{\center \scriptsize En lecture seule}}
	\end{block}
\end{frame}

\begin{frame}{\Ctitle}{\stitle}
	\begin{block}{Schéma de l'organisation de la mémoire en C}
		\vspace{5.8cm}
		\rput(1,0.0){\rnode{bas}{}}
		\rput(1,5.9){\rnode{haut}{}}
		\ncline[linecolor=gray,linestyle=dashed]{->}{bas}{haut}
		\naput[nrot=:U,labelsep=0.1]{\textcolor{gray}{\scriptsize adresses croissantes}}
		\rput(4.4,0.3){\rnode{code}{\begin{cadre}{codebg}{OliveGreen}{4.8}{0.6}   Code compilé \end{cadre}}}
		\rput(9.5,0.3){\parbox{6cm}{\center \scriptsize En lecture seule}}
		\rput(4.4,0.9){\rnode{stat}{\begin{cadre}{codebg}{OliveGreen}{4.8}{0.6}   Données statiques  \end{cadre}}}
		\rput(9.5,0.9){\parbox{6cm}{\center \scriptsize Taille connue à la compilation \\ initialisées ou non.}}
	\end{block}
\end{frame}

\begin{frame}{\Ctitle}{\stitle}
	\begin{block}{Schéma de l'organisation de la mémoire en C}
		\vspace{5.8cm}
		\rput(1,0.0){\rnode{bas}{}}
		\rput(1,5.9){\rnode{haut}{}}
		\ncline[linecolor=gray,linestyle=dashed]{->}{bas}{haut}
		\naput[nrot=:U,labelsep=0.1]{\textcolor{gray}{\scriptsize adresses croissantes}}
		\rput(4.4,0.3){\rnode{code}{\begin{cadre}{codebg}{OliveGreen}{4.8}{0.6}   Code compilé \end{cadre}}}
		\rput(9.5,0.3){\parbox{6cm}{\center \scriptsize En lecture seule}}
		\rput(4.4,0.9){\rnode{stat}{\begin{cadre}{codebg}{OliveGreen}{4.8}{0.6}   Données statiques  \end{cadre}}}
		\rput(9.5,0.9){\parbox{6cm}{\center \scriptsize Taille connue à la compilation \\ initialisées ou non.}}
		\rput(4.4,2.2){\rnode{tas}{\begin{cadre}{white}{BrickRed}{4.8}{2}   Tas (\textit{heap}) \end{cadre}}}
		\rput(9.5,2.2){\parbox{6cm}{\center \scriptsize Taille variable \\ Allouée dynamiquement par le programmeur avec \kw{malloc} et libéré avec \kw{free} }}
	\end{block}
\end{frame}

\begin{frame}{\Ctitle}{\stitle}
	\begin{block}{Schéma de l'organisation de la mémoire en C}
		\vspace{5.8cm}
		\rput(1,0.0){\rnode{bas}{}}
		\rput(1,5.9){\rnode{haut}{}}
		\ncline[linecolor=gray,linestyle=dashed]{->}{bas}{haut}
		\naput[nrot=:U,labelsep=0.1]{\textcolor{gray}{\scriptsize adresses croissantes}}
		\rput(4.4,0.3){\rnode{code}{\begin{cadre}{codebg}{OliveGreen}{4.8}{0.6}   Code compilé \end{cadre}}}
		\rput(9.5,0.3){\parbox{6cm}{\center \scriptsize En lecture seule}}
		\rput(4.4,0.9){\rnode{stat}{\begin{cadre}{codebg}{OliveGreen}{4.8}{0.6}   Données statiques  \end{cadre}}}
		\rput(9.5,0.9){\parbox{6cm}{\center \scriptsize Taille connue à la compilation \\ initialisées ou non.}}
		\rput(4.4,2.2){\rnode{tas}{\begin{cadre}{white}{BrickRed}{4.8}{2}   Tas (\textit{heap}) \end{cadre}}}
		\rput(9.5,2.2){\parbox{6cm}{\center \scriptsize Taille variable \\ Allouée dynamiquement par le programmeur avec \kw{malloc} et libéré avec \kw{free} }}
		\rput(4.4,5.6){\rnode{pile}{\begin{cadre}{white}{BrickRed}{4.8}{0.6}   Pile (\textit{stack}) \end{cadre}}}
		\rput(9.5,5.6){\parbox{6cm}{\center \scriptsize Empile les contextes d'appel des fonctions \\ (variables locales, adresse de retour, \dots) }}
	\end{block}
\end{frame}

\begin{frame}{\Ctitle}{\stitle}
	\begin{block}{Schéma de l'organisation de la mémoire en C}
		\vspace{5.8cm}
		\rput(1,0.0){\rnode{bas}{}}
		\rput(1,5.9){\rnode{haut}{}}
		\ncline[linecolor=gray,linestyle=dashed]{->}{bas}{haut}
		\naput[nrot=:U,labelsep=0.1]{\textcolor{gray}{\scriptsize adresses croissantes}}
		\rput(4.4,0.3){\rnode{code}{\begin{cadre}{codebg}{OliveGreen}{4.8}{0.6}   Code compilé \end{cadre}}}
		\rput(9.5,0.3){\parbox{6cm}{\center \scriptsize En lecture seule}}
		\rput(4.4,0.9){\rnode{stat}{\begin{cadre}{codebg}{OliveGreen}{4.8}{0.6}   Données statiques  \end{cadre}}}
		\rput(9.5,0.9){\parbox{6cm}{\center \scriptsize Taille connue à la compilation \\ initialisées ou non.}}
		\rput(4.4,2.2){\rnode{tas}{\begin{cadre}{white}{BrickRed}{4.8}{2}   Tas (\textit{heap}) \end{cadre}}}
		\rput(9.5,2.2){\parbox{6cm}{\center \scriptsize Taille variable \\ Allouée dynamiquement par le programmeur avec \kw{malloc} et libéré avec \kw{free} }}
		\rput(4.4,5.6){\rnode{pile}{\begin{cadre}{white}{BrickRed}{4.8}{0.6}   Pile (\textit{stack}) \end{cadre}}}
		\rput(9.5,5.6){\parbox{6cm}{\center \scriptsize Empile les contextes d'appel des fonctions \\ (variables locales, adresse de retour, \dots) }}
		\psline[linecolor=BrickRed,linestyle=dashed]{}(2,5.3)(2,3.2)
		\psline[linecolor=BrickRed,linestyle=dashed]{}(6.78,5.3)(6.78,3.2)
		\rput(4.5,4.4){Mémoire libre}
	\end{block}
\end{frame}

\begin{frame}{\Ctitle}{\stitle}
	\begin{block}{Schéma de l'organisation de la mémoire en C}
		\vspace{5.8cm}
		\rput(1,0.0){\rnode{bas}{}}
		\rput(1,5.9){\rnode{haut}{}}
		\ncline[linecolor=gray,linestyle=dashed]{->}{bas}{haut}
		\naput[nrot=:U,labelsep=0.1]{\textcolor{gray}{\scriptsize adresses croissantes}}
		\rput(4.4,0.3){\rnode{code}{\begin{cadre}{codebg}{OliveGreen}{4.8}{0.6}   Code compilé \end{cadre}}}
		\rput(9.5,0.3){\parbox{6cm}{\center \scriptsize En lecture seule}}
		\rput(4.4,0.9){\rnode{stat}{\begin{cadre}{codebg}{OliveGreen}{4.8}{0.6}   Données statiques  \end{cadre}}}
		\rput(9.5,0.9){\parbox{6cm}{\center \scriptsize Taille connue à la compilation \\ initialisées ou non.}}
		\rput(4.4,2.2){\rnode{tas}{\begin{cadre}{white}{BrickRed}{4.8}{2}   Tas (\textit{heap}) \end{cadre}}}
		\rput(9.5,2.2){\parbox{6cm}{\center \scriptsize Taille variable \\ Allouée dynamiquement par le programmeur avec \kw{malloc} et libéré avec \kw{free} }}
		\rput(4.4,5.6){\rnode{pile}{\begin{cadre}{white}{BrickRed}{4.8}{0.6}   Pile (\textit{stack}) \end{cadre}}}
		\rput(9.5,5.6){\parbox{6cm}{\center \scriptsize Empile les contextes d'appel des fonctions \\ (variables locales, adresse de retour, \dots) }}
		\psline[linecolor=BrickRed,linestyle=dashed]{}(2,5.3)(2,3.2)
		\psline[linecolor=BrickRed,linestyle=dashed]{}(6.78,5.3)(6.78,3.2)
		\rput(4.5,4.4){Mémoire libre}
		\ncline[nodesepB=1.2,linewidth=1pt,linestyle=dashed,linecolor=BrickRed]{->}{tas}{pile}
		\ncline[nodesepB=1.5,linewidth=1pt,linestyle=dashed,linecolor=BrickRed]{->}{pile}{tas}
	\end{block}
\end{frame}

\begin{frame}{\Ctitle}{\stitle}
	\begin{block}{Schéma de l'organisation de la mémoire en C}
		\vspace{5.8cm}
		\rput(1,0.0){\rnode{bas}{}}
		\rput(1,5.9){\rnode{haut}{}}
		\ncline[linecolor=gray,linestyle=dashed]{->}{bas}{haut}
		\naput[nrot=:U,labelsep=0.1]{\textcolor{gray}{\scriptsize adresses croissantes}}
		\rput(4.4,0.3){\rnode{code}{\begin{cadre}{codebg}{OliveGreen}{4.8}{0.6}   Code compilé \end{cadre}}}
		\rput(9.5,0.3){\parbox{6cm}{\center \scriptsize En lecture seule}}
		\rput(4.4,0.9){\rnode{stat}{\begin{cadre}{codebg}{OliveGreen}{4.8}{0.6}   Données statiques  \end{cadre}}}
		\rput(9.5,0.9){\parbox{6cm}{\center \scriptsize Taille connue à la compilation \\ initialisées ou non.}}
		\rput(4.4,2.2){\rnode{tas}{\begin{cadre}{white}{BrickRed}{4.8}{2}   Tas (\textit{heap}) \end{cadre}}}
		\rput(9.5,2.2){\parbox{6cm}{\center \scriptsize Taille variable \\ Allouée dynamiquement par le programmeur avec \kw{malloc} et libéré avec \kw{free}}}
		\rput(4.4,5.6){\rnode{pile}{\begin{cadre}{white}{BrickRed}{4.8}{0.6}   Pile (\textit{stack}) \end{cadre}}}
		\rput(9.5,5.6){\parbox{6cm}{\center \scriptsize Empile les contextes d'appel des fonctions \\ (variables locales, adresse de retour, \dots) }}
		\psline[linecolor=BrickRed,linestyle=dashed]{}(2,5.3)(2,3.2)
		\psline[linecolor=BrickRed,linestyle=dashed]{}(6.78,5.3)(6.78,3.2)
		\rput(4.5,4.4){Mémoire libre}
		\ncline[nodesepB=1.2,linewidth=1pt,linestyle=dashed,linecolor=BrickRed]{->}{tas}{pile}
		\ncline[nodesepB=1.5,linewidth=1pt,linestyle=dashed,linecolor=BrickRed]{->}{pile}{tas}
		\ncbar[angle=180,offsetA=0.26,offsetB=0.28,linecolor=OliveGreen,linestyle=dashed]{code}{stat} \naput[nrot=:U,labelsep=0]{\textcolor{OliveGreen}{\scriptsize statique}}
		\ncbar[angle=180,offsetA=0.98,offsetB=0.28,linecolor=BrickRed,linestyle=dashed]{tas}{pile} \naput[nrot=:U,labelsep=0]{\textcolor{BrickRed}{\scriptsize  dynamique}}
	\end{block}
\end{frame}


\begin{frame}{\Ctitle}{\stitle}
	\begin{block}{Conséquences}
		Cette organisation de la mémoire a des conséquences importantes
		\begin{itemize}
			\item<1-> La taille de la pile est limitée (bien plus que celle du tas), donc une variable locale de taille importante risque de provoquer un débordement de pile (\textit{stackoverflow}). Il est nettement préférable de l'allouer dans le tas.
			\item<2-> Lors de l'appel à une fonction, les variables locales (et autres informations) sont stockés dans la pile. A la fin de l'exécution, ces informations sont supprimés de la pile. Conserver des pointeurs vers des adresses de variables locales est donc problématique.
			\item<3-> De la mémoire alloué par le programmeur dans le tas et non libérée est considérée comme non disponible, créant des fuites mémoires (\textit{memory leak}).
		\end{itemize}
	\end{block}
\end{frame}

\begin{frame}[fragile]{\Ctitle}{\stitle}
	\begin{exampleblock}{Un premier exemple}
		\begin{langageC}
			int main()  {
			double big_array[1500000];
			return 0;}
		\end{langageC}
		{\small
		\begin{enumerate}
			\item<1-> Rappeler la taille d'un \kw{double}, en déduire la taille du tableau {\tt big\_array} \\
				\onslide<5-> {\textcolor{OliveGreen}{Un double occupe 8 octects, donc ce tableau $8 \times 1,5 = 12$ Mo.}}
			\item<2-> Comment expliquer que programme a provoqué une {\tt erreur de segmentation}, alors qu'il a été exécuté sur une machine possédant 8 Go de mémoire vive ?\\
				\onslide<6-> {\textcolor{OliveGreen}{La taille du tableau dépasse celle de la pile sur laquelle il est alloué.}}
			\item<3-> En déduire une information sur la taille de la pile d'appel sur cet ordinateur. \\
				\onslide<7-> {\textcolor{OliveGreen}{La pile fait moins de 12Mo (sa taille est de l'ordre de 8Mo sur l'ordinateur utilisé)}}
			\item<4-> Comment résoudre ce problème ? \\
				\onslide<8-> {\textcolor{OliveGreen}{La mémoire occupée par le tableau doit être alloué sur le tas.}}
		\end{enumerate}}
	\end{exampleblock}
\end{frame}

\makess{Pointeurs}
\begin{frame}[fragile]{\Ctitle}{\stitle}
	\begin{alertblock}{Opérateurs {\tt \&} et {\tt *}}
		\begin{itemize}
			\item<2-> L'opérateur unaire \kw{\&}, appelé opérateur d'adresse, permet en C de récupérer l'adresse mémoire d'une variable.
			\item<3-> L'opérateur unaire \kw{*}, appelé opérateur de déréférencement, permet en C de récupérer la valeur stockée dans une adresse mémoire. \\
				\textcolor{BrickRed}{\important} Déréférencer un pointeur \kw{NULL} est un comportement indéfini.
			\item<4-> Pour résumer : \\
				\begin{tabularx}{\linewidth}{|c|c|X|}
					\cline{2-3}
					\multicolumn{1}{c|}{} & S'applique à   & Permet de                                           \\
					\hline
					\kw{\&}               & à une variable & récupérer son adresse                               \\
					\hline
					\kw{*}                & à un pointeur  & récupérer la valeur à l'emplacement mémoire désigné \\
					\hline
				\end{tabularx}
			\item<5-> Si \kw{t} est un type du C, par exemple (\kw{int}, \kw{char}, \dots), alors \kw{t *} est du type pointeur vers \kw{t}. Par exemple : \\
				\onslide<6->{\mintinline{c}{int ma_fonction(int *n)}} \\
				\onslide<7->{est une fonction qui prend en paramètre l'adresse d'un \kw{int}}
		\end{itemize}
	\end{alertblock}
\end{frame}


\begin{frame}[fragile]{\Ctitle}{\stitle}
	\begin{exampleblock}{Exemple 1}
		\onslide<1->\begin{langageC*}{fontsize=\small}
			int n = 42;
			int p = &n;
			printf("Valeur stockée à l'adresse de p = %d",*p);
		\end{langageC*}
		\onslide<2->{\small Dans l'exemple ci-dessus, \kw{p} est un pointeur qui contient l'adresse de \kw{n}.}
		\onslide<3->{\small Le \kw{printf} affiche le contenu de l'adresse pointée par \kw{p} donc 42.}
	\end{exampleblock}
	\begin{exampleblock}{Exemple 2}
		\begin{enumerate}
			\item<4-> \small Ecrire une fonction {\tt echange} qui prend en argument deux adresses vers des entiers et échange les valeurs de ces deux entiers.
			\item<5-> \small Ajouter un appel à {\tt echange} dans le programme ci-dessous de façon à échanger les valeurs des variables \kw{a} et \kw{b}.
				\onslide<3->\begin{langageC*}{fontsize=\small}
					int a = 42;
					int b = 14;
				\end{langageC*}
		\end{enumerate}
	\end{exampleblock}
\end{frame}

\begin{frame}[fragile]{\Ctitle}{\stitle}
	\begin{exampleblock}{Correction de l'exemple}
		\textcolor{OliveGreen}{\aide} \onslide<2->{Récupérer les valeurs stockées aux adresses \kw{p1} et \kw{p2}.\\}
		\onslide<3->{\phantom{\aide} Utiliser une variable temporaire et les échanger\\}
		\onslide<4->{\phantom{\aide} Pour l'appel, récupérer les adresses de {\tt a} et {\tt b}	afin de les passer en paramètres.}
		\onslide<5->\inputC{/home/fenarius/Travail/Cours/cpge-info/docs/mp2i/files/C3/echange.c}{}{\small}
		\onslide<6->{
			Pour l'appel : \kw{echange}{\tt (\&a,\&b);}
		}
	\end{exampleblock}
\end{frame}

\makess{Fonctions {\tt malloc} et {\tt free}}
\begin{frame}[fragile]{\Ctitle}{\stitle}
	\begin{alertblock}{{\tt malloc}}
		\begin{itemize}
			\item<1-> La fonction \kw{malloc} permet d'allouer sur le tas, un bloc mémoire dont on donne la taille
			\item<2-> Elle s'utilise donc souvent conjointement à \kw{sizeof} qui donne la taille d'un objet en C.
			\item<3-> Comme pour les tableaux, accéder en dehors des limites du bloc alloué est une comportement indéfini.
		\end{itemize}
	\end{alertblock}
	\begin{exampleblock}{Exemple}
		\onslide<4->
		\begin{langageC}
			double *t =  malloc(sizeof(int)*100); //alloue le bloc mémoire
			t[5] = 12; // affecte la valeur 12 au 6eme élément du bloc
			t[113] = 27; // Comportement indéfini
		\end{langageC}
	\end{exampleblock}
\end{frame}

\begin{frame}[fragile]{\Ctitle}{\stitle}
	\begin{alertblock}{{\tt free}}
		\begin{itemize}
			\item<1-> La fonction \kw{free} permet de libérer un bloc mémoire précédemment alloué grâce à {\tt malloc}
			\item<2-> On appelle donc \kw{free} sur un pointeur crée \kw{p} par \kw{malloc}. Cet appel doit ompérativement se faire sur la portée de \kw{p}. En dehors, le bloc mémoire n'est plus libérable
			\item<3-> Des blocs mémoire non libérés (ou non libérables) créent des fuites de mémoire.
			\item<4-> On utilisera toujours l'option \kw{fsanitize = adress} du compilateur pour détecter ses fuites mémoires.
		\end{itemize}
	\end{alertblock}
\end{frame}


\begin{frame}[fragile]{\Ctitle}{\stitle}
	\begin{exampleblock}{Exemple : fonction renvoyant un tableau}
		\begin{itemize} \item<1-> Ecrire une fonction {\tt make\_tab} qui prend en argument deux entiers {\tt size} et {\tt init} et renvoie un pointeur vers un tableau de {\tt size} cases initialisés à la valeur {\tt init}.
				\onslide<2->\inputpartC{/home/fenarius/Travail/Cours/cpge-info/docs/mp2i/files/C3/renvoie_tab.c}{}{\small}{4}{10}
			\item<3-> \textcolor{BrickRed}{\small \danger} L'allocation \textcolor{BrickRed}{doit} se faire avec \kw{malloc}, sinon elle est faite sur la pile et donc disparait lorsqu'on quitte la fonction.
			\item<4-> La fonction appelante doit libérer la mémoire allouée avec \kw{free} (sous peine de fuites mémoires.)
		\end{itemize}
	\end{exampleblock}
\end{frame}

\makess{Argument en ligne de commande}
\begin{frame}{\Ctitle}{\stitle}
	\begin{block}{Arguments de {\tt main}}
		\begin{itemize}
			\item<1-> La fonction \kw{main} d'un programme C peut prendre en arguments un entier habituellement noté \kw{argc} (\textit{argument count}) et un tableau habituellement noté \kw{argv[]} (\textit{argument vector}) de chaines de caractères.
			\item<2-> Ces arguments doivent alors être fournis à l'exécutable produit lors de la compilation.
			\item<3-> Ces arguments sont traités comme des chaines de caractères et doivent donc être convertis dans le type adéquat si besoin grâce aux fonctions suivantes disponibles dans \kw{stdlib}.
				\begin{itemize}
					\item<4-> La fonction \kw{atoi} (\textit{{\sc ascii} to integer}) permet de convertir une chaine de caractères en un \kw{int}
					\item<5-> La fonction \kw{atof} (\textit{{\sc ascii} to float}) permet de convertir une chaine de caractères en un \kw{double}
				\end{itemize}

		\end{itemize}
	\end{block}
\end{frame}


\begin{frame}[fragile]{\Ctitle}{\stitle}
	\begin{exampleblock}{Exemple}
		Ecrire un exécutable {\tt moyenne.exe} en C qui prend en argument sur la ligne de commande des flottants et écrit dans le terminal la moyenne de ces nombres.
		Par exemple {\tt ./moyenne.exe 12 10.5 16.5} doit écrire  {\tt 13.0}.
		\onslide<2->{
			\inputC{/home/fenarius/Travail/Cours/cpge-info/docs/mp2i/files/C1/moyenne.c}{}{\small}
		}
	\end{exampleblock}
\end{frame}

\makess{Types structurés}
\begin{frame}[fragile]{\Ctitle}{\stitle}
	\begin{block}{Définitions}
		\begin{itemize}
			\item<1-> On peut définir en C, des types structurés, appelé \kw{struct} composés de plusieurs champs.
			\item<2-> La syntaxe générale de définition d'un type structuré est :
				\begin{langageC}
					struct nom_type_struct {
					type1 elt1;
					type2 elt2;
					...}
				\end{langageC}
			\item<3-> Un nom de type (qui peut être celui du {\tt struct}) peut être associé à un type structuré de façon à y faire référence plus rapidement.\\
				\mintinline{c}{typedef struct nom_type_struct nom_type}
		\end{itemize}
	\end{block}
\end{frame}

\begin{frame}[fragile]{\Ctitle}{\stitle}
	\begin{exampleblock}{Exemple}
		Pour créer le type structuré \kw{personne\_struct} contenant les trois champs nom (chaine de caractères), poids et taille (float) :
		\begin{langageC}
			struct personne_struct {
			char nom[50];
			poids float;
			taille float;
			}
		\end{langageC}
		On peut donner un nom à ce type : \\
		\mintinline{c}{typedef struct personne_struct personne;}
	\end{exampleblock}
\end{frame}

\begin{frame}[fragile]{\Ctitle}{\stitle}
	\begin{exampleblock}{Déclaration, lecture et écriture d'un champ}
		\begin{itemize}
			\item<1-> La déclaration d'un variable de type personne peut se faire maintenant avec : \\
				\mintinline{c}{personne bruce_banner;}
			\item<2-> Eventuellement avec initialisation immédiate avec la notation \kw{\{} et \kw{\}} déjà rencontré sur les tableaux : \\
				\mintinline{c}{personne hulk = {.nom="Hulk",.poids = 635,.taille=2.43};}
			\item<3-> On accède aux champs avec la notion \kw{.}, pour les lire comme par exemple :\\
				\mintinline{c}{imc_hulk = hulk.poids / (hulk.taille*hulk.taille);}
			\item<4-> Ou pour les modifier, comme par exemple : \\
				\mintinline{c}{bruce_banner.taille = 1.75;}
		\end{itemize}
	\end{exampleblock}
\end{frame}

\begin{frame}[fragile]{\Ctitle}{\stitle}
	\begin{exampleblock}{Exemple : Hulk fait un régime}
		Quel sera l'affichage produit par le programme ci-dessous, pourquoi ?
		\inputC{/home/fenarius/Travail/Cours/cpge-info/docs/mp2i/files/C3/personne.c}{}{\footnotesize}
	\end{exampleblock}
\end{frame}

\begin{frame}[fragile]{\Ctitle}{\stitle}
	\begin{block}{Pointeur sur un {\tt struct}}
		\begin{itemize}
			\item<1-> L'utilisation de pointeurs sur des \kw{struct} est courante en C.
			\item<2-> Si \kw{p} est un pointeur sur un {\tt struct} (c'est à dire \kw{*p} est un {\tt struct}) alors on accède aux champs avec la notation : \textcolor{BrickRed}{\tt (*p).nom\_champ}
			\item<3-> La notation précédente est raccourcie en \textcolor{BrickRed}{\tt p->nom\_champ}.
		\end{itemize}
	\end{block}
	\begin{exampleblock}{Exemple}
		\onslide<4->{Proposer une version correcte de la fonction modifiant le champ poids d'une variable de type {\tt struct personne}.}
	\end{exampleblock}
\end{frame}

\begin{frame}[fragile]{\Ctitle}{\stitle}
	\begin{exampleblock}{Correction}
		\inputC{/home/fenarius/Travail/Cours/cpge-info/docs/mp2i/files/C3/personne2.c}{}{\footnotesize}
	\end{exampleblock}
\end{frame}

\makess{Fichiers}
\begin{frame}[fragile]{\Ctitle}{\stitle}
	\begin{block}{Manipulation des fichiers en C}
		\begin{itemize}
			\item<1-> L'ouverture d'un fichier se fait à l'aide de \kw{fopen} qui prend comme arguments le nom du fichier et le mode d'ouverture :
				\begin{itemize}
					\item<2-> \kw{"r"} pour un accès en lecture
					\item<3-> \kw{"w"} pour un accès en écriture (le fichier est détruit s'il existait)
				\end{itemize}
			\item<4-> Cette fonction renvoie un pointeur vers un objet de type \kw{FILE} qui correspond à un flux de données sur lequel on peut lire ou écrire.
			\item<5-> Les fonctions \kw{fscanf}  et \kw{fprint} permet respectivement de lire et d'écrire sur le flux de données.
			\item<6-> Dans les deux cas, il faut fournir en argument le flux de données ainsi que les spécificateurs de format des données à lire et un pointeur vers les variables lues/écrites.
			\item<7-> La valeur spéciale \kw{EOF} est renvoyée par \kw{scanf} lorsque la fin du fichier est atteinte.
			\item<8-> Pour fermer un fichier, on utilise \kw{fclose}.
		\end{itemize}
	\end{block}
\end{frame}

\begin{frame}[fragile]{\Ctitle}{\stitle}
	\begin{exampleblock}{Somme des entiers}
		\small On suppose qu'un fichier {\tt entiers.txt} contient des entiers séparés par des espaces.
		\begin{itemize}
			\item<1-> \small Ecrire l'instruction permettant d'ouvrir ce fichier en mode lecture
				\onslide<5->\begin{langageC*}{fontsize=\small}
					FILE *fichier = fopen("entiers.txt","r");
				\end{langageC*}
			\item<2-> \small Déclarer deux entiers {\tt n} (qui va contenir les entiers lus) et {\tt somme} initialisé à 0.
				\onslide<6->\begin{langageC*}{fontsize=\small}
					int n, somme = 0;
				\end{langageC*}
			\item<3-> \small Ecrire une boucle while permettant de lire chacun des entiers jusqu'à la fin du fichier et d'en faire la somme dans la variable {\tt somme}
				\onslide<7->\begin{langageC*}{fontsize=\small}
					while (fscanf(fichier,"%d",&n)!=EOF) {
					somme = somme + n;}
				\end{langageC*}
				\item<4->\small Ecrire l'instruction permettant de fermer le fichier
				\onslide<8->\begin{langageC*}{fontsize=\small}
					fclose(fichier);
				\end{langageC*}
		\end{itemize}
	\end{exampleblock}
\end{frame}

\begin{frame}[fragile]{\Ctitle}{\stitle}
	\begin{exampleblock}{Programme complet}
		\inputC{/home/fenarius/Travail/Cours/cpge-info/docs/mp2i/files/C3/somme.c}{}{\small}
	\end{exampleblock}
\end{frame}

\makess{Compilation séparée}
\begin{frame}[fragile]{\Ctitle}{\stitle}
	\begin{block}{Principe}
		\begin{itemize}
			\item<1-> De la même façon que certaines fonctions du langage C sont écrites dans des modules séparés et inclus à l'aide de la directive {\tt \#include} en début de programme, on peut écrire ses propres modules et y écrire des fonctions destinées à être réutilisées dans différents programmes.
			\item<2-> Ces modules peuvent être \textcolor{blue}{compilés séparément}, ce qui induit de nombreux avantages (réduction de la taille du programme principal, structuration de l'application, maintenance facilitée du programme, \dots)
		\end{itemize}
	\end{block}
\end{frame}

\begin{frame}[fragile]{\Ctitle}{\stitle}
	\begin{alertblock}{Méthode}
		\begin{itemize}
			\item<2-> On commence par simplement écrire les signatures des fonctions dans un fichier {\tt <module.h>}. C'est le fichier d'en-tête (le \kw{h} vient de l'anglais \textit{header}).
			\item<3-> Les corps des fonctions sont  écrites dans {\tt <module.c>} et ce fichier est compilé de façon à obtenir un fichier objet \kw{module.o} grâce à l'option \kw{-c} de {\tt gcc}.
			\item<4-> Dans le programme principal utilisant le module, on écrit au début \mintinline{c}{#include "module.h"} ce qui permet de faire référence à ces fonctions sans déclencher de \textit{warning} à la compilation. \\
				\onslide<5->\textcolor{BrickRed}{\small \danger} Ce sont bien des guillemets et pas \kw{<} et \kw{>}. \\
			\item<5-> Le programme principal est lui aussi compilé avec l'option \kw{-c} de {\tt gcc} afin de produire un fichier \kw{main.o}.
			\item<6-> Enfin, on lie ensemble les deux fichiers objets pour produire l'exécutable grâce à la ligne de compilation :\\
				\onslide<6-> {\tt gcc \textcolor{blue}{module.o  main.o} -o main.exe}
		\end{itemize}
	\end{alertblock}
\end{frame}

\begin{frame}[fragile]{\Ctitle}{\stitle}
	\begin{block}{Schéma des étapes d'une compilation séparée}
		\renewcommand{\arraystretch}{3}
		\begin{tabular}{cccc|c|}
			\cline{5-5}
			\rnode{MH}{\textcolor{blue}{\fbox{\makebox[2cm]{\tt \small \faFile* \; module.h}}}} & \hspace{1cm} & \rnode{PC}{\textcolor{blue}{\fbox{\makebox[2cm]{\tt \small \faFile* \; main.c}}}} & \hspace{1cm}         & \rnode{PO}{\textcolor{Sepia}{\fbox{\makebox[2cm]{\tt \small \faFileArchive \; main.o}}}}                   \\
			                                                                                    &              & \rnode{MC}{\textcolor{blue}{\fbox{\makebox[2cm]{\tt \small \faFile* module.c}}}}  & \hspace{1cm}         & \rnode{MO}{\textcolor{Sepia}{\fbox{\makebox[2cm]{\tt \small \faFileArchive\; module.o}}}}                  \\
			\cline{5-5}
			                                                                                    &              &                                                                                   & \multicolumn{1}{c}{} & \multicolumn{1}{c}{}                                                                                       \\
			                                                                                    &              &                                                                                   & \multicolumn{1}{c}{} & \multicolumn{1}{c}{\rnode{EX}{\textcolor{BrickRed}{\fbox{\makebox[2cm]{\tt \small \faCogs \; main.exe}}}}} \\
		\end{tabular}
		\onslide<2->{\ncline[doubleline=true,doublesep=2pt,doublecolor=gray,linecolor=gray,linewidth=1pt,arrowsize=10pt,arrowinset=0.2,arrowlength=1.2]{->}{MH}{PC} \naput[labelsep=0cm]{{\scriptsize \tt \#include}}}
		\onslide<3->{\ncline[doubleline=true,doublesep=2pt,doublecolor=OliveGreen,linecolor=OliveGreen,linewidth=1pt,arrowsize=10pt,arrowinset=0.2,arrowlength=1.2]{->}{PC}{PO} \naput[labelsep=0cm]{{\scriptsize \tt gcc -c}}}
		\onslide<4->{\ncline[doubleline=true,doublesep=2pt,doublecolor=OliveGreen,linecolor=OliveGreen,linewidth=1pt,arrowsize=10pt,arrowinset=0.2,arrowlength=1.2]{->}{MC}{MO} \naput[labelsep=0cm]{{\scriptsize \tt gcc -c}}}
		\onslide<5->{\ncline[nodesepA=0.2cm,doubleline=true,doublesep=2pt,doublecolor=OliveGreen,linecolor=OliveGreen,linewidth=1pt,arrowsize=10pt,arrowinset=0.2,arrowlength=1.2]{->}{MO}{EX} \naput[labelsep=0cm]{{\scriptsize \tt gcc}}}
	\end{block}
\end{frame}

\begin{frame}[fragile]{\Ctitle}{\stitle}
	\begin{block}{Remarques}
		\begin{itemize}
			\item<1-> Des types composés \kw{struct} peuvent être définis dans le fichier d'en-tête sans donner leur structure. Ce point sera détaillé plus loin dans le cours lorsqu'on abordera les structures de données.
			\item<2-> Afin d'éviter des redondances, on peut spécifier dans le fichier d'en-tête
				\begin{langageC}
					#ifndef NONMODULE
					#define NONMODULE
					// ici définition des signatures des fonctions
					// et ici des types structurés éventuels
					#endif
				\end{langageC}
				Cela évite d'avoir des problèmes en cas de double inclusion malheureuse du même fichier.
			\item<3-> Afin d'éviter des conflits entre des fonctions portant le même nom et définie dans des modules ou dans le programme principale, on peut prefixer tous les noms de fonctions ou de type d'un module par le nom de ce module.
		\end{itemize}
	\end{block}
\end{frame}

\begin{frame}[fragile]{\Ctitle}{\stitle}
	\begin{exampleblock}{Un exemple}
		On décide de créer un module \kw{entiers} afin d'y définir des fonctions usuelles sur les entiers non présentes en C. Par exemple le calcul de puissances ou encore le calcul du {\sc pgcd} :
		\begin{enumerate}
			\item<2-> On commence par créer \kw{entiers.h} et y écrire les signatures de nos fonctions \\
				\onslide<3->{\tt On y trouvera par exemple \mintinline{c}{int entiers_puissance(int a, int n);}}
			\item<3-> On crée le fichier \kw{entiers.c} qui contient ces fonctions et on le compile avec \kw{gcc -c entiers.c}, cela produit un fichier {\tt entiers.o}
			\item<4-> Dans le programme principal on inclus le fichier \kw{entiers.h} avec : \\
				\mintinline{c}{#include "entiers.h"}
			\item<5-> On compile le programme principal avec \kw{gcc -c main.c}, cela produit un fichier {\tt main.o}
			\item<6-> Enfin, on produit l'executable grâce à la compilation \kw{gcc main.o entiers.o -o main.exe}
		\end{enumerate}
	\end{exampleblock}
\end{frame}

\makess{Tableaux multidimensionnels}
\begin{frame}[fragile]{\Ctitle}{\stitle}
	\begin{block}{Allocation sur la pile}
		\begin{itemize}
			\item<1-> En C, on peut allouer des tableaux multidimensionnels sur la pile en précisant les tailles attendues pour chaque dimension. Par exemple :\\
				\onslide<2->{\mintinline{c}{float matrice[4][4];} \\déclare un tableau de flottants de 4 lignes sur 4 colonnes. L'accès aux élément se fait alors avec la notation {\tt matrice[i][j]}.\\}
				\onslide<3->{La taille de la pile étant limitée, cela est réservé aux tableaux de petites tailles.}
			\item<4-> Pour une allocation sur le tas (qui sera nécessaire pour les tableaux de taille importante), deux options sont envisageables :
				\begin{itemize}
					\item<5-> Allouer un tableau de tableau à une dimension.
					\item<6-> Allouer un tableau de pointeurs, chacun des pointeurs pointe alors vers une ligne du tableau.
				\end{itemize}
		\end{itemize}
	\end{block}
\end{frame}

\begin{frame}[fragile]{\Ctitle}{\stitle}
	\begin{block}{Allocation d'un tableau de tableaux}
		\begin{itemize}
			\item<1-> Un tableau d'entier de type {\tt int} et de taille $n$ à la structure suivante en mémoire (où chaque case à la taille prévue pour accueillir un {\tt int})
				\begin{center}
					\begin{tabularx}{0.7\linewidth}{|p{0.6cm}|p{0.6cm}|X|p{0.6cm}|p{0.6cm}|}
						\hline
						                                    &                                     & \dotfill             &                      &                                         \\
						\hline
						\multicolumn{1}{c}{\footnotesize 0} & \multicolumn{1}{c}{\footnotesize 1} & \multicolumn{1}{c}{} & \multicolumn{1}{c}{} & \multicolumn{1}{c}{\footnotesize $n-1$} \\
					\end{tabularx}
				\end{center}
				\item<2->{Pour allouer une matrice de dimension $n \times m$, on peut "linéariser" le tableau et prévoir que chaque case à maintenant la capacité d'accueillir $m$ entiers. \\
				            \textcolor{BrickRed}{\small \danger\;} on doit donc connaître $m$ au moment de l'allocation.}.\\
				\item<3->{La structure de données correspondante et "à visualiser" est alors la suivante :}
				\newcolumntype{"}{@{\vrule width 1pt}}
				\newcommand{\minitab}{\begin{tabular}{|c|p{0.74cm}|c|}  & \  \dots  &  \\ \end{tabular}}
				\begin{tabularx}{\linewidth}{@{\vrule width 1pt}p{2cm}"p{2cm}"X"p{2cm}@{\vrule width 1pt}}
					\Xhline{2\arrayrulewidth}
					 \minitab &               \minitab                      & \dotfill             &    \minitab                                     \\
					\Xhline{2\arrayrulewidth}
					\multicolumn{1}{c}{\small $m$ {\tt int}} & \multicolumn{1}{c}{\small $m$ {\tt int}}& \multicolumn{1}{c}{}  & \multicolumn{1}{c}{\small $m$ {\tt int}} \\
					\multicolumn{1}{c}{\footnotesize 0}                 & \multicolumn{1}{c}{\footnotesize 1} & \multicolumn{1}{c}{} & \multicolumn{1}{c}{\footnotesize $n-1$} \\
				\end{tabularx}
				Chaque case contient du "grand" tableau contient un tableau de $m$ entiers.
				\item<4->{Cette allocation s'effectue avec : \\ \mintinline{c}{int (*mat)[n] =  malloc(sizeof(*mat)*m);}}
		\end{itemize}
	\end{block}
\end{frame}

\begin{frame}[fragile]{\Ctitle}{\stitle}
	\begin{block}{Allocation d'un tableau de pointeurs}
		\begin{itemize}
			\item<1-> On peut aussi, allouer un tableau de pointeurs, chaque pointeur pointe sur une ligne de la matrice. La structure de données correspondante est alors :
			\onslide<2->{\begin{center}
				\renewcommand{\arraystretch}{1.4}
			\begin{tabularx}{0.4\textwidth}{c|c|X|c|c|c|}
				\multicolumn{1}{c}{}& \multicolumn{1}{c}{} & \multicolumn{1}{c}{} & \multicolumn{1}{c}{0} & \multicolumn{1}{c}{\dots} & \multicolumn{1}{c}{$m-1$}\\
				\cline{2-2} \cline{4-6}
				{\footnotesize 0} & \rnode{P1}{} & & \rnode{C1}{} & \dots& \\

				\cline{2-2} \cline{4-6}
				{\footnotesize 1} & \rnode{P2}{} & & \rnode{C2}{} & \dots& \\
				\cline{2-2} \cline{4-6} 
				\multicolumn{6}{c}{\dots} \\
				\cline{2-2} \cline{4-6} 
				{\footnotesize $n-1$} & \rnode{PN}{} & & \rnode{CN}{} & \dots & \\
				\cline{2-2} \cline{4-6} 
			\end{tabularx}
			\ncline{*->}{P1}{C1}
			\ncline{*->}{P2}{C2}
			\ncline{*->}{PN}{CN}
		\end{center}}
		\item<3-> Dans ce cas il faut allouer le tableau de pointeurs : \\
		\mintinline{c}{int **mat = malloc(sizeof(int*)*n);}
		\item<4-> Puis chacune des lignes de la matrice individuellement : \\
		\mintinline{c}{mat[i] = malloc(sizeof(int) * m);} (à faire pour $ 0 \leq i \leq n$)
		\end{itemize}
	\end{block}
\end{frame}

\begin{frame}{\Ctitle}{\stitle}
	\begin{block}{Comparatif}
		\begin{itemize}
			\item<1-> Dans le premier cas :
			\begin{itemize}
				\item<2-> lne seule allocation (donc une seule libération est nécessaire)
				\item<3-> les éléments sont contigus en mémoire
				\item<4-> chaque ligne doit avoir la même taille
				\item<5-> cette taille doit être connue à l'allocation
			\end{itemize}
			\item<6-> Dans le second cas :
			\begin{itemize}
				\item<6-> Chaque ligne est alloué séparément et peut avoir une taille différente				
				\item<7-> On stocke en plus des éléments du tableau le tableau de pointeurs (donc $n$ pointeurs)
			\end{itemize}
		\end{itemize}
		\textcolor{BrickRed}{\small \danger \;} On se limitera cette année à la première méthode, l'autre est à la limite du programme et est utilisée surtout dans le cas où chaque ligne n'a pas la même taille (par exemple un tableau de chaines de caractères).
	\end{block}
\end{frame}

\end{document}


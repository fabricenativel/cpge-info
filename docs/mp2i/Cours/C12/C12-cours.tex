\PassOptionsToPackage{dvipsnames,table}{xcolor}
\documentclass[10pt]{beamer}
\usepackage{Cours}

\begin{document}


\newcounter{numchap}
\setcounter{numchap}{1}
\newcounter{numframe}
\setcounter{numframe}{0}
\newcommand{\mframe}[1]{\frametitle{#1} \addtocounter{numframe}{1}}
\newcommand{\cnum}{\fbox{\textcolor{yellow}{\textbf{C\thenumchap}}}~}
\newcommand{\makess}[1]{\section{#1} \label{ss\thesection}}
\newcommand{\stitle}{\textcolor{yellow}{\textbf{\thesection. \nameref{ss\thesection}}}}

\definecolor{codebg}{gray}{0.90}
\definecolor{grispale}{gray}{0.95}
\definecolor{fluo}{rgb}{1,0.96,0.62}
\newminted[langageC]{c}{linenos=true,escapeinside=||,highlightcolor=fluo,tabsize=2,breaklines=true}
\newminted[codepython]{python}{linenos=true,escapeinside=||,highlightcolor=fluo,tabsize=2,breaklines=true}
% Inclusion complète (ou partiel en indiquant premiere et dernière ligne) d'un fichier C
\newcommand{\inputC}[3]{\begin{mdframed}[backgroundcolor=codebg] \inputminted[breaklines=true,fontsize=#3,linenos=true,highlightcolor=fluo,tabsize=2,highlightlines={#2}]{c}{#1} \end{mdframed}}
\newcommand{\inputpartC}[5]{\begin{mdframed}[backgroundcolor=codebg] \inputminted[breaklines=true,fontsize=#3,linenos=true,highlightcolor=fluo,tabsize=2,highlightlines={#2},firstline=#4,lastline=#5,firstnumber=1]{c}{#1} \end{mdframed}}
\newcommand{\inputpython}[3]{\begin{mdframed}[backgroundcolor=codebg] \inputminted[breaklines=true,fontsize=#3,linenos=true,highlightcolor=fluo,tabsize=2,highlightlines={#2}]{python}{#1} \end{mdframed}}
\newcommand{\inputpartOCaml}[5]{\begin{mdframed}[backgroundcolor=codebg] \inputminted[breaklines=true,fontsize=#3,linenos=true,highlightcolor=fluo,tabsize=2,highlightlines={#2},firstline=#4,lastline=#5,firstnumber=1]{OCaml}{#1} \end{mdframed}}
\BeforeBeginEnvironment{minted}{\begin{mdframed}[backgroundcolor=codebg]}
\AfterEndEnvironment{minted}{\end{mdframed}}
\newcommand{\kw}[1]{\textcolor{blue}{\tt #1}}

\newtcolorbox{rcadre}[4]{halign=center,colback={#1},colframe={#2},width={#3cm},height={#4cm},valign=center,boxrule=1pt,left=0pt,right=0pt}
\newtcolorbox{cadre}[4]{halign=center,colback={#1},colframe={#2},arc=0mm,width={#3cm},height={#4cm},valign=center,boxrule=1pt,left=0pt,right=0pt}
\newcommand{\myem}[1]{\colorbox{fluo}{#1}}
\mdfsetup{skipabove=1pt,skipbelow=-2pt}



% Noeud dans un cadre pour les arbres
\newcommand{\noeud}[2]{\Tr{\fbox{\textcolor{#1}{\tt #2}}}}

\newcommand{\htmlmode}{\lstset{language=html,numbers=left, tabsize=4, frame=single, breaklines=true, keywordstyle=\ttfamily, basicstyle=\small,
   numberstyle=\tiny\ttfamily, framexleftmargin=0mm, backgroundcolor=\color{grispale}, xleftmargin=12mm,showstringspaces=false}}
\newcommand{\pythonmode}{\lstset{
   language=python,
   linewidth=\linewidth,
   numbers=left,
   tabsize=4,
   frame=single,
   breaklines=true,
   keywordstyle=\ttfamily\color{blue},
   basicstyle=\small,
   numberstyle=\tiny\ttfamily,
   framexleftmargin=-2mm,
   numbersep=-0.5mm,
   backgroundcolor=\color{codebg},
   xleftmargin=-1mm, 
   showstringspaces=false,
   commentstyle=\color{gray},
   stringstyle=\color{OliveGreen},
   emph={turtle,Screen,Turtle},
   emphstyle=\color{RawSienna},
   morekeywords={setheading,goto,backward,forward,left,right,pendown,penup,pensize,color,speed,hideturtle,showturtle,forward}}
   }
   \newcommand{\Cmode}{\lstset{
      language=[ANSI]C,
      linewidth=\linewidth,
      numbers=left,
      tabsize=4,
      frame=single,
      breaklines=true,
      keywordstyle=\ttfamily\color{blue},
      basicstyle=\small,
      numberstyle=\tiny\ttfamily,
      framexleftmargin=0mm,
      numbersep=2mm,
      backgroundcolor=\color{codebg},
      xleftmargin=0mm, 
      showstringspaces=false,
      commentstyle=\color{gray},
      stringstyle=\color{OliveGreen},
      emphstyle=\color{RawSienna},
      escapechar=\|,
      morekeywords={}}
      }
\newcommand{\bashmode}{\lstset{language=bash,numbers=left, tabsize=2, frame=single, breaklines=true, basicstyle=\ttfamily,
   numberstyle=\tiny\ttfamily, framexleftmargin=0mm, backgroundcolor=\color{grispale}, xleftmargin=12mm, showstringspaces=false}}
\newcommand{\exomode}{\lstset{language=python,numbers=left, tabsize=2, frame=single, breaklines=true, basicstyle=\ttfamily,
   numberstyle=\tiny\ttfamily, framexleftmargin=13mm, xleftmargin=12mm, basicstyle=\small, showstringspaces=false}}
   
   
  
%tei pour placer les images
%tei{nom de l’image}{échelle de l’image}{sens}{texte a positionner}
%sens ="1" (droite) ou "2" (gauche)
\newlength{\ltxt}
\newcommand{\tei}[4]{
\setlength{\ltxt}{\linewidth}
\setbox0=\hbox{\includegraphics[scale=#2]{#1}}
\addtolength{\ltxt}{-\wd0}
\addtolength{\ltxt}{-10pt}
\ifthenelse{\equal{#3}{1}}{
\begin{minipage}{\wd0}
\includegraphics[scale=#2]{#1}
\end{minipage}
\hfill
\begin{minipage}{\ltxt}
#4
\end{minipage}
}{
\begin{minipage}{\ltxt}
#4
\end{minipage}
\hfill
\begin{minipage}{\wd0}
\includegraphics[scale=#2]{#1}
\end{minipage}
}
}

%Juxtaposition d'une image pspciture et de texte 
%#1: = code pstricks de l'image
%#2: largeur de l'image
%#3: hauteur de l'image
%#4: Texte à écrire
\newcommand{\ptp}[4]{
\setlength{\ltxt}{\linewidth}
\addtolength{\ltxt}{-#2 cm}
\addtolength{\ltxt}{-0.1 cm}
\begin{minipage}[b][#3 cm][t]{\ltxt}
#4
\end{minipage}\hfill
\begin{minipage}[b][#3 cm][c]{#2 cm}
#1
\end{minipage}\par
}



%Macros pour les graphiques
\psset{linewidth=0.5\pslinewidth,PointSymbol=x}
\setlength{\fboxrule}{0.5pt}
\newcounter{tempangle}

%Marque la longueur du segment d'extrémité  #1 et  #2 avec la valeur #3, #4 est la distance par rapport au segment (en %age de la valeur de celui ci) et #5 l'orientation du marquage : +90 ou -90
\newcommand{\afflong}[5]{
\pstRotation[RotAngle=#4,PointSymbol=none,PointName=none]{#1}{#2}[X] 
\pstHomO[PointSymbol=none,PointName=none,HomCoef=#5]{#1}{X}[Y]
\pstTranslation[PointSymbol=none,PointName=none]{#1}{#2}{Y}[Z]
 \ncline{|<->|,linewidth=0.25\pslinewidth}{Y}{Z} \ncput*[nrot=:U]{\footnotesize{#3}}
}
\newcommand{\afflongb}[3]{
\ncline{|<->|,linewidth=0}{#1}{#2} \naput*[nrot=:U]{\footnotesize{#3}}
}

%Construis le point #4 situé à #2 cm du point #1 avant un angle #3 par rapport à l'horizontale. #5 = liste de paramètre
\newcommand{\lsegment}[5]{\pstGeonode[PointSymbol=none,PointName=none](0,0){O'}(#2,0){I'} \pstTranslation[PointSymbol=none,PointName=none]{O'}{I'}{#1}[J'] \pstRotation[RotAngle=#3,PointSymbol=x,#5]{#1}{J'}[#4]}
\newcommand{\tsegment}[5]{\pstGeonode[PointSymbol=none,PointName=none](0,0){O'}(#2,0){I'} \pstTranslation[PointSymbol=none,PointName=none]{O'}{I'}{#1}[J'] \pstRotation[RotAngle=#3,PointSymbol=x,#5]{#1}{J'}[#4] \pstLineAB{#4}{#1}}

%Construis le point #4 situé à #3 cm du point #1 et faisant un angle de  90° avec la droite (#1,#2) #5 = liste de paramètre
\newcommand{\psegment}[5]{
\pstGeonode[PointSymbol=none,PointName=none](0,0){O'}(#3,0){I'}
 \pstTranslation[PointSymbol=none,PointName=none]{O'}{I'}{#1}[J']
 \pstInterLC[PointSymbol=none,PointName=none]{#1}{#2}{#1}{J'}{M1}{M2} \pstRotation[RotAngle=-90,PointSymbol=x,#5]{#1}{M1}[#4]
  }
  
%Construis le point #4 situé à #3 cm du point #1 et faisant un angle de  #5° avec la droite (#1,#2) #6 = liste de paramètre
\newcommand{\mlogo}[6]{
\pstGeonode[PointSymbol=none,PointName=none](0,0){O'}(#3,0){I'}
 \pstTranslation[PointSymbol=none,PointName=none]{O'}{I'}{#1}[J']
 \pstInterLC[PointSymbol=none,PointName=none]{#1}{#2}{#1}{J'}{M1}{M2} \pstRotation[RotAngle=#5,PointSymbol=x,#6]{#1}{M2}[#4]
  }

% Construis un triangle avec #1=liste des 3 sommets séparés par des virgules, #2=liste des 3 longueurs séparés par des virgules, #3 et #4 : paramètre d'affichage des 2e et 3 points et #5 : inclinaison par rapport à l'horizontale
%autre macro identique mais sans tracer les segments joignant les sommets
\noexpandarg
\newcommand{\Triangleccc}[5]{
\StrBefore{#1}{,}[\pointA]
\StrBetween[1,2]{#1}{,}{,}[\pointB]
\StrBehind[2]{#1}{,}[\pointC]
\StrBefore{#2}{,}[\coteA]
\StrBetween[1,2]{#2}{,}{,}[\coteB]
\StrBehind[2]{#2}{,}[\coteC]
\tsegment{\pointA}{\coteA}{#5}{\pointB}{#3} 
\lsegment{\pointA}{\coteB}{0}{Z1}{PointSymbol=none, PointName=none}
\lsegment{\pointB}{\coteC}{0}{Z2}{PointSymbol=none, PointName=none}
\pstInterCC{\pointA}{Z1}{\pointB}{Z2}{\pointC}{Z3} 
\pstLineAB{\pointA}{\pointC} \pstLineAB{\pointB}{\pointC}
\pstSymO[PointName=\pointC,#4]{C}{C}[C]
}
\noexpandarg
\newcommand{\TrianglecccP}[5]{
\StrBefore{#1}{,}[\pointA]
\StrBetween[1,2]{#1}{,}{,}[\pointB]
\StrBehind[2]{#1}{,}[\pointC]
\StrBefore{#2}{,}[\coteA]
\StrBetween[1,2]{#2}{,}{,}[\coteB]
\StrBehind[2]{#2}{,}[\coteC]
\tsegment{\pointA}{\coteA}{#5}{\pointB}{#3} 
\lsegment{\pointA}{\coteB}{0}{Z1}{PointSymbol=none, PointName=none}
\lsegment{\pointB}{\coteC}{0}{Z2}{PointSymbol=none, PointName=none}
\pstInterCC[PointNameB=none,PointSymbolB=none,#4]{\pointA}{Z1}{\pointB}{Z2}{\pointC}{Z1} 
}


% Construis un triangle avec #1=liste des 3 sommets séparés par des virgules, #2=liste formée de 2 longueurs et d'un angle séparés par des virgules, #3 et #4 : paramètre d'affichage des 2e et 3 points et #5 : inclinaison par rapport à l'horizontale
%autre macro identique mais sans tracer les segments joignant les sommets
\newcommand{\Trianglecca}[5]{
\StrBefore{#1}{,}[\pointA]
\StrBetween[1,2]{#1}{,}{,}[\pointB]
\StrBehind[2]{#1}{,}[\pointC]
\StrBefore{#2}{,}[\coteA]
\StrBetween[1,2]{#2}{,}{,}[\coteB]
\StrBehind[2]{#2}{,}[\angleA]
\tsegment{\pointA}{\coteA}{#5}{\pointB}{#3} 
\setcounter{tempangle}{#5}
\addtocounter{tempangle}{\angleA}
\tsegment{\pointA}{\coteB}{\thetempangle}{\pointC}{#4}
\pstLineAB{\pointB}{\pointC}
}
\newcommand{\TriangleccaP}[5]{
\StrBefore{#1}{,}[\pointA]
\StrBetween[1,2]{#1}{,}{,}[\pointB]
\StrBehind[2]{#1}{,}[\pointC]
\StrBefore{#2}{,}[\coteA]
\StrBetween[1,2]{#2}{,}{,}[\coteB]
\StrBehind[2]{#2}{,}[\angleA]
\lsegment{\pointA}{\coteA}{#5}{\pointB}{#3} 
\setcounter{tempangle}{#5}
\addtocounter{tempangle}{\angleA}
\lsegment{\pointA}{\coteB}{\thetempangle}{\pointC}{#4}
}

% Construis un triangle avec #1=liste des 3 sommets séparés par des virgules, #2=liste formée de 1 longueurs et de deux angle séparés par des virgules, #3 et #4 : paramètre d'affichage des 2e et 3 points et #5 : inclinaison par rapport à l'horizontale
%autre macro identique mais sans tracer les segments joignant les sommets
\newcommand{\Trianglecaa}[5]{
\StrBefore{#1}{,}[\pointA]
\StrBetween[1,2]{#1}{,}{,}[\pointB]
\StrBehind[2]{#1}{,}[\pointC]
\StrBefore{#2}{,}[\coteA]
\StrBetween[1,2]{#2}{,}{,}[\angleA]
\StrBehind[2]{#2}{,}[\angleB]
\tsegment{\pointA}{\coteA}{#5}{\pointB}{#3} 
\setcounter{tempangle}{#5}
\addtocounter{tempangle}{\angleA}
\lsegment{\pointA}{1}{\thetempangle}{Z1}{PointSymbol=none, PointName=none}
\setcounter{tempangle}{#5}
\addtocounter{tempangle}{180}
\addtocounter{tempangle}{-\angleB}
\lsegment{\pointB}{1}{\thetempangle}{Z2}{PointSymbol=none, PointName=none}
\pstInterLL[#4]{\pointA}{Z1}{\pointB}{Z2}{\pointC}
\pstLineAB{\pointA}{\pointC}
\pstLineAB{\pointB}{\pointC}
}
\newcommand{\TrianglecaaP}[5]{
\StrBefore{#1}{,}[\pointA]
\StrBetween[1,2]{#1}{,}{,}[\pointB]
\StrBehind[2]{#1}{,}[\pointC]
\StrBefore{#2}{,}[\coteA]
\StrBetween[1,2]{#2}{,}{,}[\angleA]
\StrBehind[2]{#2}{,}[\angleB]
\lsegment{\pointA}{\coteA}{#5}{\pointB}{#3} 
\setcounter{tempangle}{#5}
\addtocounter{tempangle}{\angleA}
\lsegment{\pointA}{1}{\thetempangle}{Z1}{PointSymbol=none, PointName=none}
\setcounter{tempangle}{#5}
\addtocounter{tempangle}{180}
\addtocounter{tempangle}{-\angleB}
\lsegment{\pointB}{1}{\thetempangle}{Z2}{PointSymbol=none, PointName=none}
\pstInterLL[#4]{\pointA}{Z1}{\pointB}{Z2}{\pointC}
}

%Construction d'un cercle de centre #1 et de rayon #2 (en cm)
\newcommand{\Cercle}[2]{
\lsegment{#1}{#2}{0}{Z1}{PointSymbol=none, PointName=none}
\pstCircleOA{#1}{Z1}
}

%construction d'un parallélogramme #1 = liste des sommets, #2 = liste contenant les longueurs de 2 côtés consécutifs et leurs angles;  #3, #4 et #5 : paramètre d'affichage des sommets #6 inclinaison par rapport à l'horizontale 
% meme macro sans le tracé des segements
\newcommand{\Para}[6]{
\StrBefore{#1}{,}[\pointA]
\StrBetween[1,2]{#1}{,}{,}[\pointB]
\StrBetween[2,3]{#1}{,}{,}[\pointC]
\StrBehind[3]{#1}{,}[\pointD]
\StrBefore{#2}{,}[\longueur]
\StrBetween[1,2]{#2}{,}{,}[\largeur]
\StrBehind[2]{#2}{,}[\angle]
\tsegment{\pointA}{\longueur}{#6}{\pointB}{#3} 
\setcounter{tempangle}{#6}
\addtocounter{tempangle}{\angle}
\tsegment{\pointA}{\largeur}{\thetempangle}{\pointD}{#5}
\pstMiddleAB[PointName=none,PointSymbol=none]{\pointB}{\pointD}{Z1}
\pstSymO[#4]{Z1}{\pointA}[\pointC]
\pstLineAB{\pointB}{\pointC}
\pstLineAB{\pointC}{\pointD}
}
\newcommand{\ParaP}[6]{
\StrBefore{#1}{,}[\pointA]
\StrBetween[1,2]{#1}{,}{,}[\pointB]
\StrBetween[2,3]{#1}{,}{,}[\pointC]
\StrBehind[3]{#1}{,}[\pointD]
\StrBefore{#2}{,}[\longueur]
\StrBetween[1,2]{#2}{,}{,}[\largeur]
\StrBehind[2]{#2}{,}[\angle]
\lsegment{\pointA}{\longueur}{#6}{\pointB}{#3} 
\setcounter{tempangle}{#6}
\addtocounter{tempangle}{\angle}
\lsegment{\pointA}{\largeur}{\thetempangle}{\pointD}{#5}
\pstMiddleAB[PointName=none,PointSymbol=none]{\pointB}{\pointD}{Z1}
\pstSymO[#4]{Z1}{\pointA}[\pointC]
}


%construction d'un cerf-volant #1 = liste des sommets, #2 = liste contenant les longueurs de 2 côtés consécutifs et leurs angles;  #3, #4 et #5 : paramètre d'affichage des sommets #6 inclinaison par rapport à l'horizontale 
% meme macro sans le tracé des segements
\newcommand{\CerfVolant}[6]{
\StrBefore{#1}{,}[\pointA]
\StrBetween[1,2]{#1}{,}{,}[\pointB]
\StrBetween[2,3]{#1}{,}{,}[\pointC]
\StrBehind[3]{#1}{,}[\pointD]
\StrBefore{#2}{,}[\longueur]
\StrBetween[1,2]{#2}{,}{,}[\largeur]
\StrBehind[2]{#2}{,}[\angle]
\tsegment{\pointA}{\longueur}{#6}{\pointB}{#3} 
\setcounter{tempangle}{#6}
\addtocounter{tempangle}{\angle}
\tsegment{\pointA}{\largeur}{\thetempangle}{\pointD}{#5}
\pstOrtSym[#4]{\pointB}{\pointD}{\pointA}[\pointC]
\pstLineAB{\pointB}{\pointC}
\pstLineAB{\pointC}{\pointD}
}

%construction d'un quadrilatère quelconque #1 = liste des sommets, #2 = liste contenant les longueurs des 4 côtés et l'angle entre 2 cotés consécutifs  #3, #4 et #5 : paramètre d'affichage des sommets #6 inclinaison par rapport à l'horizontale 
% meme macro sans le tracé des segements
\newcommand{\Quadri}[6]{
\StrBefore{#1}{,}[\pointA]
\StrBetween[1,2]{#1}{,}{,}[\pointB]
\StrBetween[2,3]{#1}{,}{,}[\pointC]
\StrBehind[3]{#1}{,}[\pointD]
\StrBefore{#2}{,}[\coteA]
\StrBetween[1,2]{#2}{,}{,}[\coteB]
\StrBetween[2,3]{#2}{,}{,}[\coteC]
\StrBetween[3,4]{#2}{,}{,}[\coteD]
\StrBehind[4]{#2}{,}[\angle]
\tsegment{\pointA}{\coteA}{#6}{\pointB}{#3} 
\setcounter{tempangle}{#6}
\addtocounter{tempangle}{\angle}
\tsegment{\pointA}{\coteD}{\thetempangle}{\pointD}{#5}
\lsegment{\pointB}{\coteB}{0}{Z1}{PointSymbol=none, PointName=none}
\lsegment{\pointD}{\coteC}{0}{Z2}{PointSymbol=none, PointName=none}
\pstInterCC[PointNameA=none,PointSymbolA=none,#4]{\pointB}{Z1}{\pointD}{Z2}{Z3}{\pointC} 
\pstLineAB{\pointB}{\pointC}
\pstLineAB{\pointC}{\pointD}
}


% Définition des colonnes centrées ou à droite pour tabularx
\newcolumntype{Y}{>{\centering\arraybackslash}X}
\newcolumntype{Z}{>{\flushright\arraybackslash}X}

%Les pointillés à remplir par les élèves
\newcommand{\po}[1]{\makebox[#1 cm]{\dotfill}}
\newcommand{\lpo}[1][3]{%
\multido{}{#1}{\makebox[\linewidth]{\dotfill}
}}

%Liste des pictogrammes utilisés sur la fiche d'exercice ou d'activités
\newcommand{\bombe}{\faBomb}
\newcommand{\livre}{\faBook}
\newcommand{\calculatrice}{\faCalculator}
\newcommand{\oral}{\faCommentO}
\newcommand{\surfeuille}{\faEdit}
\newcommand{\ordinateur}{\faLaptop}
\newcommand{\ordi}{\faDesktop}
\newcommand{\ciseaux}{\faScissors}
\newcommand{\danger}{\faExclamationTriangle}
\newcommand{\out}{\faSignOut}
\newcommand{\cadeau}{\faGift}
\newcommand{\flash}{\faBolt}
\newcommand{\lumiere}{\faLightbulb}
\newcommand{\compas}{\dsmathematical}
\newcommand{\calcullitteral}{\faTimesCircleO}
\newcommand{\raisonnement}{\faCogs}
\newcommand{\recherche}{\faSearch}
\newcommand{\rappel}{\faHistory}
\newcommand{\video}{\faFilm}
\newcommand{\capacite}{\faPuzzlePiece}
\newcommand{\aide}{\faLifeRing}
\newcommand{\loin}{\faExternalLink}
\newcommand{\groupe}{\faUsers}
\newcommand{\bac}{\faGraduationCap}
\newcommand{\histoire}{\faUniversity}
\newcommand{\coeur}{\faSave}
\newcommand{\python}{\faPython}
\newcommand{\os}{\faMicrochip}
\newcommand{\rd}{\faCubes}
\newcommand{\data}{\faColumns}
\newcommand{\web}{\faCode}
\newcommand{\prog}{\faFile}
\newcommand{\algo}{\faCogs}
\newcommand{\important}{\faExclamationCircle}
\newcommand{\maths}{\faTimesCircle}
% Traitement des données en tables
\newcommand{\tables}{\faColumns}
% Types construits
\newcommand{\construits}{\faCubes}
% Type et valeurs de base
\newcommand{\debase}{{\footnotesize \faCube}}
% Systèmes d'exploitation
\newcommand{\linux}{\faLinux}
\newcommand{\sd}{\faProjectDiagram}
\newcommand{\bd}{\faDatabase}

%Les ensembles de nombres
\renewcommand{\N}{\mathbb{N}}
\newcommand{\D}{\mathbb{D}}
\newcommand{\Z}{\mathbb{Z}}
\newcommand{\Q}{\mathbb{Q}}
\newcommand{\R}{\mathbb{R}}
\newcommand{\C}{\mathbb{C}}

%Ecriture des vecteurs
\newcommand{\vect}[1]{\vbox{\halign{##\cr 
  \tiny\rightarrowfill\cr\noalign{\nointerlineskip\vskip1pt} 
  $#1\mskip2mu$\cr}}}


%Compteur activités/exos et question et mise en forme titre et questions
\newcounter{numact}
\setcounter{numact}{1}
\newcounter{numseance}
\setcounter{numseance}{1}
\newcounter{numexo}
\setcounter{numexo}{0}
\newcounter{numprojet}
\setcounter{numprojet}{0}
\newcounter{numquestion}
\newcommand{\espace}[1]{\rule[-1ex]{0pt}{#1 cm}}
\newcommand{\Quest}[3]{
\addtocounter{numquestion}{1}
\begin{tabularx}{\textwidth}{X|m{1cm}|}
\cline{2-2}
\textbf{\sffamily{\alph{numquestion})}} #1 & \dots / #2 \\
\hline 
\multicolumn{2}{|l|}{\espace{#3}} \\
\hline
\end{tabularx}
}
\newcommand{\QuestR}[3]{
\addtocounter{numquestion}{1}
\begin{tabularx}{\textwidth}{X|m{1cm}|}
\cline{2-2}
\textbf{\sffamily{\alph{numquestion})}} #1 & \dots / #2 \\
\hline 
\multicolumn{2}{|l|}{\cor{#3}} \\
\hline
\end{tabularx}
}
\newcommand{\Pre}{{\sc nsi} 1\textsuperscript{e}}
\newcommand{\Term}{{\sc nsi} Terminale}
\newcommand{\Sec}{2\textsuperscript{e}}
\newcommand{\Exo}[2]{ \addtocounter{numexo}{1} \ding{113} \textbf{\sffamily{Exercice \thenumexo}} : \textit{#1} \hfill #2  \setcounter{numquestion}{0}}
\newcommand{\Projet}[1]{ \addtocounter{numprojet}{1} \ding{118} \textbf{\sffamily{Projet \thenumprojet}} : \textit{#1}}
\newcommand{\ExoD}[2]{ \addtocounter{numexo}{1} \ding{113} \textbf{\sffamily{Exercice \thenumexo}}  \textit{(#1 pts)} \hfill #2  \setcounter{numquestion}{0}}
\newcommand{\ExoB}[2]{ \addtocounter{numexo}{1} \ding{113} \textbf{\sffamily{Exercice \thenumexo}}  \textit{(Bonus de +#1 pts maximum)} \hfill #2  \setcounter{numquestion}{0}}
\newcommand{\Act}[2]{ \ding{113} \textbf{\sffamily{Activité \thenumact}} : \textit{#1} \hfill #2  \addtocounter{numact}{1} \setcounter{numquestion}{0}}
\newcommand{\Seance}{ \rule{1.5cm}{0.5pt}\raisebox{-3pt}{\framebox[4cm]{\textbf{\sffamily{Séance \thenumseance}}}}\hrulefill  \\
  \addtocounter{numseance}{1}}
\newcommand{\Acti}[2]{ {\footnotesize \ding{117}} \textbf{\sffamily{Activité \thenumact}} : \textit{#1} \hfill #2  \addtocounter{numact}{1} \setcounter{numquestion}{0}}
\newcommand{\titre}[1]{\begin{Large}\textbf{\ding{118}}\end{Large} \begin{large}\textbf{ #1}\end{large} \vspace{0.2cm}}
\newcommand{\QListe}[1][0]{
\ifthenelse{#1=0}
{\begin{enumerate}[partopsep=0pt,topsep=0pt,parsep=0pt,itemsep=0pt,label=\textbf{\sffamily{\arabic*.}},series=question]}
{\begin{enumerate}[resume*=question]}}
\newcommand{\SQListe}[1][0]{
\ifthenelse{#1=0}
{\begin{enumerate}[partopsep=0pt,topsep=0pt,parsep=0pt,itemsep=0pt,label=\textbf{\sffamily{\alph*)}},series=squestion]}
{\begin{enumerate}[resume*=squestion]}}
\newcommand{\SQListeL}[1][0]{
\ifthenelse{#1=0}
{\begin{enumerate*}[partopsep=0pt,topsep=0pt,parsep=0pt,itemsep=0pt,label=\textbf{\sffamily{\alph*)}},series=squestion]}
{\begin{enumerate*}[resume*=squestion]}}
\newcommand{\FinListe}{\end{enumerate}}
\newcommand{\FinListeL}{\end{enumerate*}}

%Mise en forme de la correction
\newcommand{\cor}[1]{\par \textcolor{OliveGreen}{#1}}
\newcommand{\br}[1]{\cor{\textbf{#1}}}
\newcommand{\tcor}[1]{\begin{tcolorbox}[width=0.92\textwidth,colback={white},colbacktitle=white,coltitle=OliveGreen,colframe=green!75!black,boxrule=0.2mm]   
\cor{#1}
\end{tcolorbox}
}
\newcommand{\rc}[1]{\textcolor{OliveGreen}{#1}}
\newcommand{\pmc}[1]{\textcolor{blue}{\tt #1}}
\newcommand{\tmc}[1]{\textcolor{RawSienna}{\tt #1}}


%Référence aux exercices par leur numéro
\newcommand{\refexo}[1]{
\refstepcounter{numexo}
\addtocounter{numexo}{-1}
\label{#1}}

%Séparation entre deux activités
\newcommand{\separateur}{\begin{center}
\rule{1.5cm}{0.5pt}\raisebox{-3pt}{\ding{117}}\rule{1.5cm}{0.5pt}  \vspace{0.2cm}
\end{center}}

%Entête et pied de page
\newcommand{\snt}[1]{\lhead{\textbf{SNT -- La photographie numérique} \rhead{\textit{Lycée Nord}}}}
\newcommand{\Activites}[2]{\lhead{\textbf{{\sc #1}}}
\rhead{Activités -- \textbf{#2}}
\cfoot{}}
\newcommand{\Exos}[2]{\lhead{\textbf{Fiche d'exercices: {\sc #1}}}
\rhead{Niveau: \textbf{#2}}
\cfoot{}}
\newcommand{\Devoir}[2]{\lhead{\textbf{Devoir de mathématiques : {\sc #1}}}
\rhead{\textbf{#2}} \setlength{\fboxsep}{8pt}
\begin{center}
%Titre de la fiche
\fbox{\parbox[b][1cm][t]{0.3\textwidth}{Nom : \hfill \po{3} \par \vfill Prénom : \hfill \po{3}} } \hfill 
\fbox{\parbox[b][1cm][t]{0.6\textwidth}{Note : \po{1} / 20} }
\end{center} \cfoot{}}

%Devoir programmation en NSI (pas à rendre sur papier)
\newcommand{\PNSI}[2]{\lhead{\textbf{Devoir de {\sc nsi} : \textsf{ #1}}}
\rhead{\textbf{#2}} \setlength{\fboxsep}{8pt}
\begin{tcolorbox}[title=\textcolor{black}{\danger\; A lire attentivement},colbacktitle=lightgray]
{\begin{enumerate}
\item Rendre tous vous programmes en les envoyant par mail à l'adresse {\tt fnativel2@ac-reunion.fr}, en précisant bien dans le sujet vos noms et prénoms
\item Un programme qui fonctionne mal ou pas du tout peut rapporter des points
\item Les bonnes pratiques de programmation (clarté et lisiblité du code) rapportent des points
\end{enumerate}
}
\end{tcolorbox}
 \cfoot{}}


%Devoir de NSI
\newcommand{\DNSI}[2]{\lhead{\textbf{Devoir de {\sc nsi} : \textsf{ #1}}}
\rhead{\textbf{#2}} \setlength{\fboxsep}{8pt}
\begin{center}
%Titre de la fiche
\fbox{\parbox[b][1cm][t]{0.3\textwidth}{Nom : \hfill \po{3} \par \vfill Prénom : \hfill \po{3}} } \hfill 
\fbox{\parbox[b][1cm][t]{0.6\textwidth}{Note : \po{1} / 10} }
\end{center} \cfoot{}}

\newcommand{\DevoirNSI}[2]{\lhead{\textbf{Devoir de {\sc nsi} : {\sc #1}}}
\rhead{\textbf{#2}} \setlength{\fboxsep}{8pt}
\cfoot{}}

%La définition de la commande QCM pour auto-multiple-choice
%En premier argument le sujet du qcm, deuxième argument : la classe, 3e : la durée prévue et #4 : présence ou non de questions avec plusieurs bonnes réponses
\newcommand{\QCM}[4]{
{\large \textbf{\ding{52} QCM : #1}} -- Durée : \textbf{#3 min} \hfill {\large Note : \dots/10} 
\hrule \vspace{0.1cm}\namefield{}
Nom :  \textbf{\textbf{\nom{}}} \qquad \qquad Prénom :  \textbf{\prenom{}}  \hfill Classe: \textbf{#2}
\vspace{0.2cm}
\hrule  
\begin{itemize}[itemsep=0pt]
\item[-] \textit{Une bonne réponse vaut un point, une absence de réponse n'enlève pas de point. }
\item[\danger] \textit{Une mauvaise réponse enlève un point.}
\ifthenelse{#4=1}{\item[-] \textit{Les questions marquées du symbole \multiSymbole{} peuvent avoir plusieurs bonnes réponses possibles.}}{}
\end{itemize}
}
\newcommand{\DevoirC}[2]{
\renewcommand{\footrulewidth}{0.5pt}
\lhead{\textbf{Devoir de mathématiques : {\sc #1}}}
\rhead{\textbf{#2}} \setlength{\fboxsep}{8pt}
\fbox{\parbox[b][0.4cm][t]{0.955\textwidth}{Nom : \po{5} \hfill Prénom : \po{5} \hfill Classe: \textbf{1}\textsuperscript{$\dots$}} } 
\rfoot{\thepage} \cfoot{} \lfoot{Lycée Nord}}
\newcommand{\DevoirInfo}[2]{\lhead{\textbf{Evaluation : {\sc #1}}}
\rhead{\textbf{#2}} \setlength{\fboxsep}{8pt}
 \cfoot{}}
\newcommand{\DM}[2]{\lhead{\textbf{Devoir maison à rendre le #1}} \rhead{\textbf{#2}}}

%Macros permettant l'affichage des touches de la calculatrice
%Touches classiques : #1 = 0 fond blanc pour les nombres et #1= 1gris pour les opérations et entrer, second paramètre=contenu
%Si #2=1 touche arrondi avec fond gris
\newcommand{\TCalc}[2]{
\setlength{\fboxsep}{0.1pt}
\ifthenelse{#1=0}
{\psframebox[fillstyle=solid, fillcolor=white]{\parbox[c][0.25cm][c]{0.6cm}{\centering #2}}}
{\ifthenelse{#1=1}
{\psframebox[fillstyle=solid, fillcolor=lightgray]{\parbox[c][0.25cm][c]{0.6cm}{\centering #2}}}
{\psframebox[framearc=.5,fillstyle=solid, fillcolor=white]{\parbox[c][0.25cm][c]{0.6cm}{\centering #2}}}
}}
\newcommand{\Talpha}{\psdblframebox[fillstyle=solid, fillcolor=white]{\hspace{-0.05cm}\parbox[c][0.25cm][c]{0.65cm}{\centering \scriptsize{alpha}}} \;}
\newcommand{\Tsec}{\psdblframebox[fillstyle=solid, fillcolor=white]{\parbox[c][0.25cm][c]{0.6cm}{\centering \scriptsize 2nde}} \;}
\newcommand{\Tfx}{\psdblframebox[fillstyle=solid, fillcolor=white]{\parbox[c][0.25cm][c]{0.6cm}{\centering \scriptsize $f(x)$}} \;}
\newcommand{\Tvar}{\psframebox[framearc=.5,fillstyle=solid, fillcolor=white]{\hspace{-0.22cm} \parbox[c][0.25cm][c]{0.82cm}{$\scriptscriptstyle{X,T,\theta,n}$}}}
\newcommand{\Tgraphe}{\psdblframebox[fillstyle=solid, fillcolor=white]{\hspace{-0.08cm}\parbox[c][0.25cm][c]{0.68cm}{\centering \tiny{graphe}}} \;}
\newcommand{\Tfen}{\psdblframebox[fillstyle=solid, fillcolor=white]{\hspace{-0.08cm}\parbox[c][0.25cm][c]{0.68cm}{\centering \tiny{fenêtre}}} \;}
\newcommand{\Ttrace}{\psdblframebox[fillstyle=solid, fillcolor=white]{\parbox[c][0.25cm][c]{0.6cm}{\centering \scriptsize{trace}}} \;}

% Macroi pour l'affichage  d'un entier n dans  une base b
\newcommand{\base}[2]{ \overline{#1}^{#2}}
% Intervalle d'entiers
\newcommand{\intN}[2]{\llbracket #1; #2 \rrbracket}}

% Numéro et titre de chapitre
\setcounter{numchap}{12}
\newcommand{\Ctitle}{\cnum {Arbres binaires}}
\newcommand{\SPATH}{/home/fenarius/Travail/Cours/cpge-info/docs/mp2i/files/C\thenumchap/}

% Exemple introductif
\makess{Définition}
\begin{frame}[fragile]{\Ctitle}{\stitle}
	\begin{alertblock}{Définition}
		Un \textcolor{blue}{arbre binaire}, est une structure de données \textit{hiérarchique} (les éléments, appelés \textcolor{blue}{noeuds} sont rangés par niveau) qui peut se définir récursivement.\\
		En effet, un arbre binaire est
		\begin{itemize}
			\item<2-> soit vide, on le note alors $\varnothing$
			\item<3-> soit un noeud $(sag,r,sad)$  appelé \textcolor{blue}{racine} où $r$ est l'étiquette de la racine et $sag$ et $sad$ sont deux arbres binaires (le sous arbre gauche, et le sous arbre droit)
		\end{itemize}
	\end{alertblock}
	\begin{exampleblock}{Exemples}
		\onslide<4->{\small L'arbre $(\varnothing,a,(\varnothing,b,\varnothing))$\\}
		\onslide<5->{
			\pstree[arrows=->,treesep=0.5cm,levelsep=0.7cm]{\TCircle[radius=0.25cm]{$a$}}{
				\Tr{$\varnothing$}
				\pstree{\TCircle[radius=0.25cm]{$b$}}{
					\Tr{$\varnothing$}
					\Tr{$\varnothing$}
				}
			} \quad \quad \quad }
		\onslide<6->
		{
			\pstree[arrows=->,treesep=0.5cm,levelsep=0.7cm]{\TCircle[radius=0.25cm]{$a$}}{
				\Tn{}
				\TCircle[radius=0.25cm]{$b$}
			}
			\\}
		\onslide<7->{\small Représenté avec (à gauche) ou sans (à droite) les sous arbres vides.}
	\end{exampleblock}
\end{frame}

\begin{frame}[fragile]{\Ctitle}{\stitle}
	\begin{block}{Remarques}
		\begin{itemize}
			\item<1->{\small \textcolor{BrickRed}{\danger}} Les deux arbres ci-dessous sont \textcolor{blue}{différents} !\\
			\pstree[arrows=->,treesep=0.5cm,levelsep=0.7cm]{\TCircle[radius=0.25cm]{$a$}}{
				\Tr{$\varnothing$}
				\pstree[arrows=->]{\TCircle[radius=0.25cm]{$b$}}{
					\Tr{$\varnothing$}
					\Tr{$\varnothing$}
				}
			} \quad \quad \quad
			\pstree[arrows=->,treesep=0.5cm,levelsep=0.7cm]{\TCircle[radius=0.25cm]{$a$}}{
				\pstree{\TCircle[radius=0.25cm]{$b$}}{
					\Tr{$\varnothing$}
					\Tr{$\varnothing$}
				}
				\Tr{$\varnothing$}
			} \quad \quad \quad
			\item<2-> On omet parfois de représenter les sous arbres vides, mais on doit garder à l'esprit qu'un noeud non vide est \textit{toujours} un triplet. Et que donc les sous arbres gauche et droit même vide, sont toujours présents.
			\item<3-> Lorsque qu'un noeud $a$ possède un sous arbre non vide dont la racine est $b$, on dit que $a$ est le \textcolor{blue}{père} de $b$ et que $b$ est le \textcolor{blue}{fils} $a$.
			\item<4-> Un noeud dont les deux sous arbres sont vides s'appelle une \textcolor{blue}{feuille}.
			\item<5-> un noeud qui n'est pas une feuille s'appelle un \textcolor{blue}{noeud interne}.
		\end{itemize}
	\end{block}
\end{frame}

\begin{frame}[fragile]{\Ctitle}{\stitle}
	\begin{block}{Définition récursive du nombre de noeuds et de la hauteur}
		\begin{itemize}
			\item Le \textcolor{blue}{nombre de noeuds} d'un arbre binaire $A$, noté $n(A)$, se définit récursivement par :
			      \onslide<2->{
				      $$ \left\{
					      \begin{array}{llll}
						      n(A) & = & 0               & \text{si $A$ est vide}  \\
						      n(A) & = & 1 + n(g) + n(d) & \text{si $A = (g,a,d)$} \\
					      \end{array}
					      \right.
				      $$}
			\item<3-> La \textcolor{blue}{hauteur} d'un arbre binaire $A$, noté $h(A)$, se définit récursivement par :
				\onslide<4->{
					$$ \left\{
						\begin{array}{llll}
							h(A) & = & -1                  & \text{si $A$ est vide}  \\
							n(A) & = & 1 + \max(h(g),h(d)) & \text{si $A = (g,a,d)$} \\
						\end{array}
						\right.
					$$}
				\onslide<5->{\textcolor{BrickRed}{\small \danger} Certains auteurs prennent $0$ comment hauteur de l'arbre vide.}
				\item<6->La \textcolor{blue}{profondeur} d'un noeud est sa distance à la racine.
		\end{itemize}
	\end{block}
\end{frame}

\begin{frame}[fragile]{\Ctitle}{\stitle}
	\begin{exampleblock}{Exemple}
		\begin{center}
			\pstree[arrows=->,treesep=0.5cm,levelsep=0.7cm]{\TCircle[radius=0.25cm]{$r$}}{
				\Tn{}
				\pstree[treesep=0.5cm,levelsep=0.7cm]{\TCircle[radius=0.25cm]{$k$}}
				{
					\pstree[treesep=0.5cm,levelsep=0.7cm]{\TCircle[radius=0.25cm]{$e$}}{
						\pstree[treesep=0.5cm,levelsep=0.7cm]{\TCircle[radius=0.25cm]{$p$}}{
							\TCircle[radius=0.25cm]{$l$}
							\TCircle[radius=0.25cm]{$u$}
						}
						\Tn{}
					}
					\pstree[treesep=0.5cm,levelsep=0.7cm]{\TCircle[radius=0.25cm]{$c$}}{
						\Tn{}
						\TCircle[radius=0.25cm]{$z$}}
				}
			}
		\end{center}
		\begin{itemize}
			\item<2-> Nommer les feuilles et les noeuds internes
			\item<3-> Donner le nombre de noeuds
			\item<4-> Donner la hauteur de cet arbre
			\item<5-> Donner un noeud de profondeur 2
			\item<6-> Donner l'écriture de cet arbre sous forme de triplet
		\end{itemize}
	\end{exampleblock}
\end{frame}


% Cas particuliers d'arbres binaires
\begin{frame}[fragile]{\Ctitle}{\stitle}
	\begin{block}{Quelques cas particuliers}
		\begin{itemize}
			\item Un arbre binaire est dit \textcolor{blue}{dégénéré} lorsque tous les noeuds à l'exception des feuilles n'ont qu'un fils. \\
			      \onslide<2->{
				      \begin{tabularx}{\linewidth}{YYY}
					      \pstree[arrows=->,treesep=0.5cm,levelsep=0.7cm]{\TCircle[radius=0.25cm]{}}{
						      \Tn{}
						      \pstree[treesep=0.5cm,levelsep=0.7cm]{\TCircle[radius=0.25cm]{}}{
							      \Tn{}
							      \pstree[treesep=0.5cm,levelsep=0.7cm]{\TCircle[radius=0.25cm]{}}{
								      \Tn{}
								      \pstree[treesep=0.5cm,levelsep=0.7cm]{\TCircle[radius=0.25cm]{}}{
									      \Tn{}
								      }}}}
					       &
					      \pstree[arrows=->,treesep=0.5cm,levelsep=0.7cm]{\TCircle[radius=0.25cm]{}}{
						      \Tn{}
						      \pstree[treesep=0.5cm,levelsep=0.7cm]{\TCircle[radius=0.25cm]{}}{
							      \Tn{}
							      \pstree[treesep=0.5cm,levelsep=0.7cm]{\TCircle[radius=0.25cm]{}}{

								      \pstree[treesep=0.5cm,levelsep=0.7cm]{\TCircle[radius=0.25cm]{}}{
									      \Tn{}
								      }\Tn{}}}}
					       & \pstree[arrows=->,treesep=0.5cm,levelsep=0.7cm]{\TCircle[radius=0.25cm]{}}{
						      \pstree[treesep=0.5cm,levelsep=0.7cm]{\TCircle[radius=0.25cm]{}}{
							      \pstree[treesep=0.5cm,levelsep=0.7cm]{\TCircle[radius=0.25cm]{}}{
								      \pstree[treesep=0.5cm,levelsep=0.7cm]{\TCircle[radius=0.25cm]{}}{
									      \Tn{} } \Tn{}} \Tn{}}                    \Tn{}
					      }
				      \end{tabularx}}
			      \onslide<2-> Pour les arbres représentés à gauche et à droite on parle de \textit{peigne}, à rapprocher de la liste chainée.
		\end{itemize}
	\end{block}
\end{frame}
\begin{frame}[fragile]{\Ctitle}{\stitle}
	\begin{block}{Quelques cas particuliers}
		\begin{itemize}
			\item<2-> Un arbre binaire est dit \textcolor{blue}{parfait} lorsque tous les niveaux sont remplis : \\
				\onslide<3->{
					\begin{center}
						\pstree[arrows=->,treesep=0.5cm,levelsep=0.7cm]{\TCircle[radius=0.25cm]{}}
						{
							\pstree[treesep=0.5cm,levelsep=0.7cm]{\TCircle[radius=0.25cm]{}}{
								\TCircle[radius=0.25cm]{}
								\TCircle[radius=0.25cm]{}
							}
							\pstree[treesep=0.5cm,levelsep=0.7cm]{\TCircle[radius=0.25cm]{}}{
								\TCircle[radius=0.25cm]{}
								\TCircle[radius=0.25cm]{}
							}
						}
					\end{center}}
			\item<4-> Un arbre binaire est dit \textcolor{blue}{complet} lorsque tous les niveaux à l'exception du dernier sont remplis et que le dernier niveau est rempli à parti de la gauche. \\
				\onslide<5->{
					\begin{center}
						\pstree[arrows=->,treesep=0.5cm,levelsep=0.7cm]{\TCircle[radius=0.25cm]{}}
						{
							\pstree[treesep=0.5cm,levelsep=0.7cm]{\TCircle[radius=0.25cm]{}}{
								\TCircle[radius=0.25cm]{}
								\TCircle[radius=0.25cm]{}
							}
							\pstree[treesep=0.5cm,levelsep=0.7cm]{\TCircle[radius=0.25cm]{}}{
								\TCircle[radius=0.25cm]{}
								\Tn{}
							}
						}
					\end{center}}
		\end{itemize}
	\end{block}
\end{frame}

% Relation entre hauteur et taille
\begin{frame}[fragile]{\Ctitle}{\stitle}
	\begin{alertblock}{Nombre de sous abres vides}
		Le nombre de sous arbres vides d'un arbre binaire de taille $n$ est $n+1$.
	\end{alertblock}
	\onslide<2->{
		\begin{alertblock}{Relation entre hauteur et taille}
			En notant $n$ la taille et $h$ la hauteur d'un arbre binaire, on a la relation suivante : \\
			\onslide<3->{$$ { h+1 \leq n \leq 2^{h+1}-1} $$}
		\end{alertblock}}
	\onslide<4->{
		\begin{exampleblock}{Exercice}
			\begin{enumerate}
				\item<5-> Dessiner tous les arbres binaires de taille 4.
				\item<6-> Dessiner un arbre binaire de hauteur taille 4 et de hauteur 3.
				\item<7-> Dessiner un arbre binaire de hauteur 2 et de taille $2^3-1$.
			\end{enumerate}
		\end{exampleblock}}
\end{frame}

\makess{Représentation  en machine}
\begin{frame}[fragile]{\Ctitle}{\stitle}
	\begin{block}{Type structuré en C}
		En C, on représente un arbre binaire par un pointeur vers un type structuré contenant trois champs : l'étiquette (un {\tt int}) de la racine, et les pointeurs vers les deux sous arbres (gauche et droit). L'arbre vide est le pointeur {\sc null}.
		\onslide<2->{\inputpartC{\SPATH/arbres_binaires.c}{}{}{4}{11}}
	\end{block}
\end{frame}

\begin{frame}[fragile]{\Ctitle}{\stitle}
	\begin{block}{Création d'un arbre}
		\onslide<2->On écrit alors une fonction \mintinline{c}{ab cree_arbre(ab g, int v, ab d)}  qui renvoie un arbre donne on donne l'étiquette de la racine et les deux sous arbres :
		\onslide<3>{\inputpartC{\SPATH/arbres_binaires.c}{}{}{15}{22}}
	\end{block}
\end{frame}

\begin{frame}[fragile]{\Ctitle}{\stitle}
	\begin{exampleblock}{Exemple}
		\begin{enumerate}
			\item<1-> En utilisant cette représentation, créer l'arbre binaire suivant :
				\begin{center}
					\pstree[arrows=->,treesep=0.5cm,levelsep=0.7cm]{\TCircle[radius=0.25cm]{9}}
					{
						\pstree[treesep=0.5cm,levelsep=0.7cm]{\TCircle[radius=0.25cm]{5}}{
							\Tn{}
							\TCircle[radius=0.25cm]{7}
						}
						\pstree[treesep=0.5cm,levelsep=0.7cm]{\TCircle[radius=0.25cm]{3}}{
							\TCircle[radius=0.25cm]{4}
							\Tn{}
						}
					}
				\end{center}
			\item<2-> Ecrire la fonction permettant de calculer le nombre de noeuds d'un arbre binaire.
				\onslide<3->{\inputpartC{\SPATH/arbres_binaires.c}{}{}{24}{29}}
		\end{enumerate}
	\end{exampleblock}
\end{frame}


\begin{frame}[fragile]{\Ctitle}{\stitle}
	\begin{block}{Type structuré en OCaml}
		Un arbre binaire étant soit vide soit constituée d'une étiquette ({\tt int} pour simplifier), on le définit en OCaml en envisageant les 2 cas :
		\onslide<2->{\inputpartOCaml{\SPATH/arbres_binaires.ml}{}{}{1}{3}}
	\end{block}
\end{frame}

\begin{frame}[fragile]{\Ctitle}{\stitle}
	\begin{exampleblock}{Exemple}
		\begin{enumerate}
			\item<1-> En utilisant cette représentation, créer l'arbre binaire suivant :
				\begin{center}
					\pstree[arrows=->,treesep=0.5cm,levelsep=0.7cm]{\TCircle[radius=0.25cm]{2}}
					{
						\pstree[treesep=0.5cm,levelsep=0.7cm]{\TCircle[radius=0.25cm]{7}}{
							\Tn{}
							\pstree[treesep=0.5cm,levelsep=0.7cm]{\TCircle[radius=0.25cm]{1}}{
								\TCircle[radius=0.25cm]{5}
								\TCircle[radius=0.25cm]{6}}
						}
						\pstree[treesep=0.5cm,levelsep=0.7cm]{\TCircle[radius=0.25cm]{3}}{
							\Tn{}
							\TCircle[radius=0.25cm]{4}
						}
					}
				\end{center}
			\item<2-> Ecrire la fonction permettant de calculer le nombre de noeuds d'un arbre binaire.
				\onslide<3->{\inputpartOCaml{\SPATH/arbres_binaires.ml}{}{}{6}{9}}
		\end{enumerate}
	\end{exampleblock}
\end{frame}

\begin{frame}[fragile]{\Ctitle}{\stitle}
	\begin{block}{Cas particulier d'un arbre binaire complet}
		\onslide<2->{Un arbre binaire complet de taille $n$ peut être représenté de façon compacte à l'aide d'un tableau de taille $n$.}
		\onslide<3->{On numérote les noeuds depuis la racine, de gauche à droite et de haut en bas, le noeud numeroté $i$ est placé dans la case d'indice $i$.}
		\onslide<4->{Par exemple :
			\begin{center}
				\pstree[arrows=->,treesep=0.5cm,levelsep=0.7cm]{\TCircle[radius=0.25cm]{4} \alt<5->{\nput[labelsep=1 pt]{0}{\pssucc}{\textcolor{blue}{\scriptsize 0}}}{}}
				{
					\pstree[treesep=0.5cm,levelsep=0.7cm]{
						\TCircle[radius=0.25cm]{5} \alt<5->{\nput[labelsep=1 pt]{0}{\pssucc}{\textcolor{blue}{\scriptsize 1}}}{}}{
						\TCircle[radius=0.25cm]{2} \alt<5->{\nput[labelsep=1 pt]{0}{\pssucc}{\textcolor{blue}{\scriptsize 3}}}{}
						\TCircle[radius=0.25cm]{7} \alt<5->{\nput[labelsep=1 pt]{0}{\pssucc}{\textcolor{blue}{\scriptsize 4}}}{}
					}
					\pstree[treesep=0.5cm,levelsep=0.7cm]{\TCircle[radius=0.25cm]{9} \alt<5->{\nput[labelsep=1 pt]{0}{\pssucc}{\textcolor{blue}{\scriptsize 2}}}{}}{
						\TCircle[radius=0.25cm]{8} \alt<5->{\nput[labelsep=1 pt]{0}{\pssucc}{\textcolor{blue}{\scriptsize 5}}}{}
						\Tn{}
					}
				}\vspace{0.2cm}
			\end{center}}
		\onslide<6->{sera représenté par le tableau :}
		\onslide<7->{
			\begin{tabular}{|c|c|c|c|c|c|}
				\hline
				4                                                   & 5                                                   & 9                                                   & 2                                                   & 7                                                   & 8                                                   \\
				\hline
				\multicolumn{1}{c}{\textcolor{blue}{\scriptsize 0}} & \multicolumn{1}{c}{\textcolor{blue}{\scriptsize 1}} & \multicolumn{1}{c}{\textcolor{blue}{\scriptsize 2}} & \multicolumn{1}{c}{\textcolor{blue}{\scriptsize 3}} & \multicolumn{1}{c}{\textcolor{blue}{\scriptsize 4}} & \multicolumn{1}{c}{\textcolor{blue}{\scriptsize 5}} \\
			\end{tabular}}
		\begin{itemize}
			\item Le fils gauche (resp. droit) du noeud $i$ se trouve à l'indice $2i+1$ (resp.) $2i+2$.
			\item Le père du noeud d'indice i se trouve à l'indice $\pe{\dfrac{i-1}{2}}$.
		\end{itemize}
	\end{block}
\end{frame}

\makess{Parcours d'un arbre binaire}
\begin{frame}[fragile]{\Ctitle}{\stitle}
	\begin{alertblock}{Parcours récursifs}
		On appelle \textit{parcours d'un arbre binaire} un algorithme permettant de visiter chaque noeud de cet arbre une et une seule fois afin d'y effectuer un traitement (tester la présence d'une valeur, chercher la plus petite valeur, \dots).
		\onslide<2->{Compte tenu de la structure récursive des arbres binaires, trois parcours récursifs émergent suivant le choix du moment où on traite la racine du noeud ($g,r,d$) : }
		\begin{itemize}
			\item<3-> Dans le parcours \textcolor{blue}{préfixe}, la racine est traitée avant de relancer le parcours sur  le sous arbre gauche $g$ et le sous arbre droit $d$.
			\item<4-> Dans le parcours \textcolor{blue}{infixe}, la racine est traitée après le parcours du sous arbre gauche $g$ mais avant celui du sous arbre droit  $d$.
			\item<5-> Dans le parcours \textcolor{blue}{suffixe}, la racine est traitée  après le parcours du sous arbre gauche $g$ et du sous arbre droit $d$.
		\end{itemize}
	\end{alertblock}
\end{frame}


\begin{frame}[fragile]{\Ctitle}{\stitle}
	\begin{exampleblock}{Exemple}
		\begin{center}
			\begin{tabular}{p{0.3cm}p{0.3cm}p{0.3cm}p{0.3cm}p{0.3cm}p{0.3cm}p{0.3cm}}
				                    &                     &                     & \circlenode{A}{{R}} &                     &                     & \phantom{0}\vspace{0.5cm} \\
				                    & \circlenode{B}{T}   &                     &                     &                     & \circlenode{C}{{C}} & \phantom{0}\vspace{0.5cm} \\
				\circlenode{D}{{V}} &                     & \circlenode{E}{{I}} &                     & \circlenode{F}{{F}} &                     & \phantom{0}\vspace{0.5cm} \\
				                    & \circlenode{I}{{M}} &                     & \circlenode{G}{{P}} &                     & \circlenode{H}{{A}} &                           \\
				\ncline{->}{A}{B} \ncline{->}{A}{C} \ncline{->}{B}{D} \ncline{->}{B}{E} \ncline{->}{F}{G} \ncline{->}{F}{H} \ncline{C}{F} \ncline{D}{I}
			\end{tabular}
		\end{center}
		Donner l'ordre des noeuds lorsqu'on parcourt l'arbre ci-dessus :
		\begin{itemize}
			\item<2-> En profondeur préfixé : \onslide<3->{R, T, V, M, I, C, F, P, A}
			\item<4-> En profondeur infixé : \onslide<5->{V, M, T, I, R, P, F, A, C}
			\item<6-> En profondeur suffixé : \onslide<7->{M, V, I, T, P, A, F, C, R}
		\end{itemize}
	\end{exampleblock}
\end{frame}


\begin{frame}[fragile]{\Ctitle}{\stitle}
	\begin{alertblock}{Parcours en largeur}
		La parcours en largeur revient à lister les noeuds par ordre croissant de profondeur et de gauche à droite \\
		\onslide<2-> L'implémentation de ce parcours peut se faire à l'aide d'une file dans laquelle on stocke les noeuds restants à parcourir. A chaque fois qu'on traite un noeud, on le defile et on enfile ses fils.
	\end{alertblock}
	\onslide<3->{
	\begin{exampleblock}{Exemple}
		\begin{center}
			\pstree[arrows=->,treesep=0.5cm,levelsep=0.7cm]{\TCircle[radius=0.25cm]{S}}
			{
				\pstree[treesep=0.5cm,levelsep=0.7cm]{\TCircle[radius=0.25cm]{U}}{
					\Tn{}
					\TCircle[radius=0.25cm]{E}
				}
				\pstree[treesep=0.5cm,levelsep=0.7cm]{\TCircle[radius=0.25cm]{P}}{
					\TCircle[radius=0.25cm]{R}
					\Tn{}
				}
			}
		\end{center}\vspace{0.2cm}}
		\onslide<4-> {Le parcours en largeur donne : S, U, P, E, R.}
	\end{exampleblock}
\end{frame}

\begin{frame}[fragile]{\Ctitle}{\stitle}
	\begin{block}{Exemple d'implémentation}
		\begin{itemize}
			\item<1-> Parcours prefixe en C (affichage des étiquettes)
				\onslide<2->{\inputpartC{\SPATH/arbres_binaires.c}{}{\footnotesize}{88}{96}}
			\item<3-> Parcours infixe en OCaml (affichage des étiquettes)
				\onslide<2->{\inputpartOCaml{\SPATH/arbres_binaires.ml}{}{\footnotesize}{56}{59}}
		\end{itemize}
	\end{block}
\end{frame}

\makess{Arbres binaires de recherche}
\begin{frame}[fragile]{\Ctitle}{\stitle}
	\begin{alertblock}{Arbre binaire de recherche}
		Un arbre binaire \textcolor{blue}{de recherche} (noté {\sc abr}), est un arbre binaire tel que :
		\begin{itemize}
			\item<2-> Les étiquettes des noeuds, appelées \textcolor{blue}{clés} sont toutes comparables entre elles. \\
				\onslide<4->{\textcolor{gray}{Par exemple, les étiquettes sont toutes des nombres ou encore des chaines de caractères (comparées par ordre alphabétique).}}
			\item<5-> Pour tous les noeuds $N(g,v,d)$ l'ensemble des clés présentes dans le sous arbre gauche $g$ (resp. droit $d$) sont strictement inférieures (resp. supérieures) à $v$.\\
				\item<6->{Les clés sont \textcolor{blue}{uniques}.}
		\end{itemize}
	\end{alertblock}
	\onslide<7->{
		\begin{block}{Caractérisation par le parcours infixe}
			Un arbre binaire est un {\sc abr} si et seulement si le parcours infixe fournit les clés dans l'ordre croissant.
		\end{block}}
\end{frame}


\begin{frame}[fragile]{\Ctitle}{\stitle}
	\begin{exampleblock}{Exemple}
		\begin{center}
			\begin{tabular}{p{0.3cm}p{0.3cm}p{0.3cm}p{0.3cm}p{0.3cm}p{0.3cm}p{0.3cm}}
				                    &                     &                     & \circlenode{A}{{10}} &                      &                      & \phantom{0}\vspace{0.5cm} \\
				                    & \circlenode{B}{{6}} &                     &                      &                      & \circlenode{C}{{19}} & \phantom{0}\vspace{0.5cm} \\
				\circlenode{D}{{4}} &                     & \circlenode{E}{{?}} &                      & \circlenode{F}{{16}} &                      & \phantom{0}\vspace{0.5cm} \\
				                    & \circlenode{I}{{5}} &                     & \circlenode{G}{{13}} &                      & \circlenode{H}{{17}} &                           \\
				\ncline{->}{A}{B} \ncline{->}{A}{C} \ncline{->}{B}{D} \ncline{->}{B}{E} \ncline{->}{F}{G} \ncline{->}{F}{H} \ncline{C}{F} \ncline{D}{I}
			\end{tabular}\vspace{-0.5cm}
		\end{center}
		{\small
		\begin{itemize}
			\item<1-> Cet arbre est-il un {\sc abr} si la clé manquante est 2 ? 9 ? 12 ?
			\item<2-> On suppose que la clé manquante est 9. Proposer une nouvelle valeur pour le noeud de clé 16 de façon à ce que cet arbre reste un {\sc abr}.
			\item<3-> Proposer une valeur pour le noeud de clé 16 de façon à ce que cet arbre ne soit pas un {\sc abr}.
			\item<4-> Donner l'ordre des clés lors d'un parcours infixe.
		\end{itemize}}
	\end{exampleblock}
\end{frame}

\begin{frame}[fragile]{\Ctitle}{\stitle}
	\begin{alertblock}{Insertion dans un {\sc abr}}
		Pour insérer une nouvelle valeur $u$ dans un {\sc abr} $A$ :
		\begin{itemize}
			\item<2-> Si $A$ est vide on renvoie $(\varnothing,u,\varnothing)$.
			\item<3-> Sinon $A = (g,v,d)$ et on insère dans $g$ si $u<v$ et dans $d$ sinon.
		\end{itemize}
	\end{alertblock}
	\begin{block}{Implémentation en OCaml}
		\inputpartOCaml{\SPATH/abr.ml}{}{}{70}{73}
	\end{block}
\end{frame}

\begin{frame}[fragile]{\Ctitle}{\stitle}
	\begin{exampleblock}{Exercice}
		\begin{itemize}
			\item<1-> Dessiner l'arbre obtenu en partant de l'arbre vide puis en insérant dans cet ordre les valeurs :
				\begin{itemize}
					\item<2-> 2, 5, 7, 9 et 11
					\item<3-> 9, 5, 11, 2, 7
					\item<4-> 7, 5, 9, 2, 11
				\end{itemize}
				\item<5->En vous inspirant de la fonction d'insertion, écrire une fonction {\tt in\_abr} ({\tt abr -> int -> bool}) qui renvoie un booléen indiquant si la valeur passé en argument se trouve ou non dans l'{\sc abr}.
				\onslide<6->{\inputpartOCaml{\SPATH/abr.ml}{}{}{75}{79}}
		\end{itemize}
	\end{exampleblock}
\end{frame}

\begin{frame}[fragile]{\Ctitle}{\stitle}
	\begin{block}{Complexité}
		La complexité des opérations d'insertion et de recherche dans un {\sc abr} est majorée par la hauteur $h$ de l'arbre.\onslide<2->\textcolor{gray}{ On descend d'un niveau dans l'arbre à chaque appel récursif et la profondeur d'un noeud est inférieure à $h$.}\\
		\onslide<3->{Or on sait que $ h+1 \leq n \leq 2^{h+1}-1$, et les deux bornes sont atteintes}
		\begin{itemize}
			\item<4-> Dans le cas d'un peigne ($n=h+1$) les opérations seront en $O(n)$.
			\item<5-> Dans le cas d'un arbre complet ($n=2^{h+1}-1$), les opérations seront en $O(\log(n))$.
		\end{itemize}
	\end{block}
	\onslide<6->{
		\begin{alertblock}{Définition}
			Soit $S$, un ensemble d'abres binaires. On dit que les arbres de $S$ sont \textcolor{blue}{équilibrés} s'il existe une constante $C$ telle que, pour tout arbre $s \in S$ :
			$$ h(s) \leq C \log(n(s))$$
		\end{alertblock}}
\end{frame}


\begin{frame}[fragile]{\Ctitle}{\stitle}
	\begin{block}{Rotation d'un {\sc abr}}
		On considère l'{\sc abr} suivant où $u$ et $v$ sont les étiquettes des noeuds représentés et $t_1$, $t_2$, $t_3$ des arbres binaires :
		\onslide<2->{\begin{center}\pstree[arrows=->,treesep=0.5cm,levelsep=0.7cm]{\TCircle[radius=0.25cm]{$v$}}{
					\pstree{\TCircle[radius=0.25cm]{$u$}}{
						\Tr{$t_1$}
						\Tr{$t_2$}}
					\Tr{$t_3$}
				}
			\end{center}}
		\onslide<3->{La \textcolor{blue}{rotation droite} de cet arbre, consiste à réorganiser les noeuds \textit{en conservant la propriété d'{\sc abr}} de la façon suivante :}
		\onslide<4->{
			\begin{center}
				\pstree[arrows=->,treesep=0.5cm,levelsep=0.7cm]{\TCircle[radius=0.25cm]{$u$}}{
					\Tr{$t_1$}
					\pstree{\TCircle[radius=0.25cm]{$v$}}{
						\Tr{$t_2$}
						\Tr{$t_3$}
					}
				}
			\end{center}}
		\onslide<5->{De façon symétrique, la \textcolor{blue}{rotation gauche} consiste en partant de cet arbre à revenir à l'arbre initial.}
	\end{block}
\end{frame}

\begin{frame}[fragile]{\Ctitle}{\stitle}
	\begin{exampleblock}{Exemple}
		On considère l'arbre binaire suivant :
		\begin{center}\pstree[arrows=->,treesep=0.5cm,levelsep=0.7cm]{\TCircle[radius=0.25cm]{$7$}}{
				\pstree{\TCircle[radius=0.25cm]{$3$}}{
					\TCircle[radius=0.25cm]{$2$}
					\pstree{\Tcircle[radius=0.25cm]{$5$}}
					{
						\TCircle[radius=0.25cm]{$4$}
						\TCircle[radius=0.25cm]{$6$}
					}
				}
				\TCircle[radius=0.25cm]{$9$}
			}
		\end{center}
		\begin{enumerate}
			\item<2-> Vérifier qu'il s'agit d'un {\sc abr}
			\item<3-> Montrer qu'un utilisant des rotations, on peut transformer cet arbre en un arbre binaire parfait.
		\end{enumerate}
	\end{exampleblock}
\end{frame}

\begin{frame}[fragile]{\Ctitle}{\stitle}
	\begin{exampleblock}{Correction}
		\begin{center}
			\pstree[arrows=->,treesep=0.5cm,levelsep=0.7cm]{\TCircle[radius=0.25cm]{$7$}}{
				\pstree{\TCircle[linecolor=BrickRed,linewidth=1pt,radius=0.25cm]{\textcolor{BrickRed}{$3$}} \nput[labelsep=1 pt]{0}{\pssucc}{\textcolor{BrickRed}{$u$}}}{
					\TCircle[name=D,radius=0.25cm]{$\ 2\ $}
					\ncbox[linecolor=gray,nodesep=0.1,boxsize=0.3,linestyle=dashed]{D}{D}
					\nbput[labelsep=0]{\textcolor{gray}{$\scriptstyle t_1$}}
					\pstree{\Tcircle[linecolor=BrickRed,linewidth=1pt,radius=0.25cm]{\textcolor{BrickRed}{$5$}} \nput[labelsep=1 pt]{0}{\pssucc}{\textcolor{BrickRed}{$v$}}}
					{
						\TCircle[name=Q,radius=0.25cm]{$4$}
						\ncbox[linecolor=gray,nodesep=0.1,boxsize=0.3,linestyle=dashed]{Q}{Q}
						\nbput[labelsep=0]{\textcolor{gray}{$\scriptstyle t_2$}}
						\TCircle[name=S,radius=0.25cm]{$6$}
						\ncbox[linecolor=gray,nodesep=0.1,boxsize=0.3,linestyle=dashed]{S}{S}
						\nbput[labelsep=0]{\textcolor{gray}{$\scriptstyle t_3$}}
					}
				}
				\TCircle[radius=0.25cm]{$9$} 
			} \hspace{1.5cm}
			\onslide<3->{
			\pstree[arrows=->,treesep=0.5cm,levelsep=0.7cm]{\TCircle[radius=0.25cm]{$7$}}{
				\pstree{\TCircle[linecolor=BrickRed,linewidth=1pt,radius=0.25cm]{\textcolor{BrickRed}{$5$}} \nput[labelsep=1 pt]{0}{\pssucc}{\textcolor{BrickRed}{$v$}}}{
					\pstree{\Tcircle[linecolor=BrickRed,linewidth=1pt,radius=0.25cm]{\textcolor{BrickRed}{$3$}} \nput[labelsep=1 pt]{0}{\pssucc}{\textcolor{BrickRed}{$u$}}}
					{
						\TCircle[name=D,radius=0.25cm]{$2$}
						\ncbox[linecolor=gray,nodesep=0.1,boxsize=0.3,linestyle=dashed]{D}{D}
						\nbput[labelsep=0]{\textcolor{gray}{$\scriptstyle t_1$}}
						\TCircle[name=Q,radius=0.25cm]{$4$}
						\ncbox[linecolor=gray,nodesep=0.1,boxsize=0.3,linestyle=dashed]{Q}{Q}
						\nbput[labelsep=0]{\textcolor{gray}{$\scriptstyle t_2$}}
					}
					\TCircle[name=S,radius=0.25cm]{$6$}
					\ncbox[linecolor=gray,nodesep=0.1,boxsize=0.3,linestyle=dashed]{S}{S}
					\nbput[labelsep=0]{\textcolor{gray}{$\scriptstyle t_3$}}
				}
				\TCircle[radius=0.25cm]{$9$} 
			}} \vspace{0.2cm}\\
			\onslide<4->{
			\pstree[arrows=->,treesep=0.7cm,levelsep=1cm]{\TCircle[linecolor=BrickRed,linewidth=1pt,radius=0.25cm]{$7$} \nput[labelsep=1 pt]{0}{\pssucc}{\textcolor{BrickRed}{$u$}}}{
				\pstree{\TCircle[linecolor=BrickRed,linewidth=1pt,radius=0.25cm]{\textcolor{BrickRed}{$5$}} \nput[labelsep=1 pt]{0}{\pssucc}{\textcolor{BrickRed}{$v$}}}{
						\pstree{\Tcircle[radius=0.25cm]{$3$}}
					{  
						\TCircle[name=D,radius=0.25cm]{$2$}
						\TCircle[name=Q,radius=0.25cm]{$4$}
						
					}
					\psframe[linecolor=gray,linestyle=dashed](0.8,0.25)(-1.25,-1.3)
					\nbput[labelsep=1.6cm]{\textcolor{gray}{$\scriptstyle t_1$}}
					\TCircle[name=S,radius=0.25cm]{$6$}
					\ncbox[linecolor=gray,nodesep=0.1,boxsize=0.3,linestyle=dashed]{S}{S}
					\nbput[labelsep=0]{\textcolor{gray}{$\scriptstyle t_2$}}
				}
				\TCircle[name=N,radius=0.25cm]{$9$}
				\ncbox[linecolor=gray,nodesep=0.1,boxsize=0.3,linestyle=dashed]{N}{N}
				\nbput[labelsep=0]{\textcolor{gray}{$\scriptstyle t_3$}}
			}}
			\vspace{1cm}
		\end{center}
	\end{exampleblock}
\end{frame}

\end{document}
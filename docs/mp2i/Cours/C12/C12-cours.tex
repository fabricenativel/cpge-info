\PassOptionsToPackage{dvipsnames,table}{xcolor}
\documentclass[10pt]{beamer}
\usepackage{Cours}

\begin{document}

\input{\detokenize{/home/fenarius/Travail/Cours/cpge-info/latex/MacrosCours.tex}}

% Numéro et titre de chapitre
\setcounter{numchap}{12}
\newcommand{\Ctitle}{\cnum {Arbres binaires}}
\newcommand{\SPATH}{/home/fenarius/Travail/Cours/cpge-info/docs/mp2i/files/C\thenumchap/}

% Exemple introductif
\makess{Définition}
\begin{frame}[fragile]{\Ctitle}{\stitle}
    \begin{alertblock}{Définition}
        Un \textcolor{blue}{arbre binaire}, est une structure de données \textit{hiérarchique} (les éléments sont rangés par niveau) qui peut se définir récursivement.\\
        En effet, un arbre binaire est 
        \begin{itemize}
            \item<2-> soit vide, on le note alors $\varnothing$
            \item<3-> soit un triplet $(sag,r,sad)$ où $r$ est la racine et $sag$ et $sad$ sont deux arbres binaires (le sous arbre gauche, et le sous arbre droit)
        \end{itemize}
    \end{alertblock}
\end{frame}

\end{document}
\PassOptionsToPackage{dvipsnames,table}{xcolor}
\documentclass[10pt]{beamer}
\usepackage{Cours}

\begin{document}

\input{\detokenize{/home/fenarius/Travail/Cours/cpge-info/latex/MacrosCours.tex}}

% Numéro et titre de chapitre
\setcounter{numchap}{12}
\newcommand{\Ctitle}{\cnum {Arbres binaires}}
\newcommand{\SPATH}{/home/fenarius/Travail/Cours/cpge-info/docs/mp2i/files/C\thenumchap/}

% Exemple introductif
\makess{Définition}
\begin{frame}[fragile]{\Ctitle}{\stitle}
    \begin{alertblock}{Définition}
        Un \textcolor{blue}{arbre binaire}, est une structure de données \textit{hiérarchique} (les éléments, appelés \textcolor{blue}{noeuds} sont rangés par niveau) qui peut se définir récursivement.\\
        En effet, un arbre binaire est 
        \begin{itemize}
            \item<2-> soit vide, on le note alors $\varnothing$
            \item<3-> soit un noeud $(sag,r,sad)$  appelé \textcolor{blue}{racine} où $r$ est l'étiquette de la racine et $sag$ et $sad$ sont deux arbres binaires (le sous arbre gauche, et le sous arbre droit)
        \end{itemize}
    \end{alertblock}
    \begin{exampleblock}{Exemples}
        \onslide<4->{\small L'arbre $(\varnothing,a,(\varnothing,b,\varnothing))$\\}
        \onslide<5->{
        \pstree[treesep=0.5cm,levelsep=0.7cm]{\TCircle[radius=0.25cm]{$a$}}{
            \Tr{$\varnothing$}
            \pstree{\TCircle[radius=0.25cm]{$b$}}{
                \Tr{$\varnothing$}
                \Tr{$\varnothing$}
            }
        } \quad \quad \quad }
        \onslide<6->
        {
            \pstree[treesep=0.5cm,levelsep=0.7cm]{\TCircle[radius=0.25cm]{$a$}}{
                \Tn{}
            \TCircle[radius=0.25cm]{$b$}
            }
        \\}
        \onslide<7->{\small Représenté avec (à gauche) ou sans (à droite) les sous arbres vides.}
    \end{exampleblock}
\end{frame}

\begin{frame}[fragile]{\Ctitle}{\stitle}
    \begin{block}{Remarques}
        \begin{itemize}
        \item<1->{\small \textcolor{BrickRed}{\danger}} Les deux arbres ci-dessous sont \textcolor{blue}{différents} !\\
        \pstree[treesep=0.5cm,levelsep=0.7cm]{\TCircle[radius=0.25cm]{$a$}}{
            \Tr{$\varnothing$}
            \pstree{\TCircle[radius=0.25cm]{$b$}}{
                \Tr{$\varnothing$}
                \Tr{$\varnothing$}
            }
        } \quad \quad \quad 
        \pstree[treesep=0.5cm,levelsep=0.7cm]{\TCircle[radius=0.25cm]{$a$}}{
            \pstree{\TCircle[radius=0.25cm]{$b$}}{
                \Tr{$\varnothing$}
                \Tr{$\varnothing$}
            }
            \Tr{$\varnothing$}
        } \quad \quad \quad
        \item<2-> On omet parfois de représenter les sous arbres vides, mais on doit garder à l'esprit qu'un noeud non vide est \textit{toujours} un triplet. Et que donc les sous arbres gauche et droit même vide, sont toujours présents.
        \end{itemize}
    \end{block}
\end{frame}

\end{document}
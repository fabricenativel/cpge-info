\PassOptionsToPackage{dvipsnames,table}{xcolor}
\documentclass[10pt]{beamer}
\usepackage{Cours}

\begin{document}


\newcounter{numchap}
\setcounter{numchap}{1}
\newcounter{numframe}
\setcounter{numframe}{0}
\newcommand{\mframe}[1]{\frametitle{#1} \addtocounter{numframe}{1}}
\newcommand{\cnum}{\fbox{\textcolor{yellow}{\textbf{C\thenumchap}}}~}
\newcommand{\makess}[1]{\section{#1} \label{ss\thesection}}
\newcommand{\stitle}{\textcolor{yellow}{\textbf{\thesection. \nameref{ss\thesection}}}}

\definecolor{codebg}{gray}{0.90}
\definecolor{grispale}{gray}{0.95}
\definecolor{fluo}{rgb}{1,0.96,0.62}
\newminted[langageC]{c}{linenos=true,escapeinside=||,highlightcolor=fluo,tabsize=2,breaklines=true}
\newminted[codepython]{python}{linenos=true,escapeinside=||,highlightcolor=fluo,tabsize=2,breaklines=true}
% Inclusion complète (ou partiel en indiquant premiere et dernière ligne) d'un fichier C
\newcommand{\inputC}[3]{\begin{mdframed}[backgroundcolor=codebg] \inputminted[breaklines=true,fontsize=#3,linenos=true,highlightcolor=fluo,tabsize=2,highlightlines={#2}]{c}{#1} \end{mdframed}}
\newcommand{\inputpartC}[5]{\begin{mdframed}[backgroundcolor=codebg] \inputminted[breaklines=true,fontsize=#3,linenos=true,highlightcolor=fluo,tabsize=2,highlightlines={#2},firstline=#4,lastline=#5,firstnumber=1]{c}{#1} \end{mdframed}}
\newcommand{\inputpython}[3]{\begin{mdframed}[backgroundcolor=codebg] \inputminted[breaklines=true,fontsize=#3,linenos=true,highlightcolor=fluo,tabsize=2,highlightlines={#2}]{python}{#1} \end{mdframed}}
\newcommand{\inputpartOCaml}[5]{\begin{mdframed}[backgroundcolor=codebg] \inputminted[breaklines=true,fontsize=#3,linenos=true,highlightcolor=fluo,tabsize=2,highlightlines={#2},firstline=#4,lastline=#5,firstnumber=1]{OCaml}{#1} \end{mdframed}}
\BeforeBeginEnvironment{minted}{\begin{mdframed}[backgroundcolor=codebg]}
\AfterEndEnvironment{minted}{\end{mdframed}}
\newcommand{\kw}[1]{\textcolor{blue}{\tt #1}}

\newtcolorbox{rcadre}[4]{halign=center,colback={#1},colframe={#2},width={#3cm},height={#4cm},valign=center,boxrule=1pt,left=0pt,right=0pt}
\newtcolorbox{cadre}[4]{halign=center,colback={#1},colframe={#2},arc=0mm,width={#3cm},height={#4cm},valign=center,boxrule=1pt,left=0pt,right=0pt}
\newcommand{\myem}[1]{\colorbox{fluo}{#1}}
\mdfsetup{skipabove=1pt,skipbelow=-2pt}



% Noeud dans un cadre pour les arbres
\newcommand{\noeud}[2]{\Tr{\fbox{\textcolor{#1}{\tt #2}}}}

\newcommand{\htmlmode}{\lstset{language=html,numbers=left, tabsize=4, frame=single, breaklines=true, keywordstyle=\ttfamily, basicstyle=\small,
   numberstyle=\tiny\ttfamily, framexleftmargin=0mm, backgroundcolor=\color{grispale}, xleftmargin=12mm,showstringspaces=false}}
\newcommand{\pythonmode}{\lstset{
   language=python,
   linewidth=\linewidth,
   numbers=left,
   tabsize=4,
   frame=single,
   breaklines=true,
   keywordstyle=\ttfamily\color{blue},
   basicstyle=\small,
   numberstyle=\tiny\ttfamily,
   framexleftmargin=-2mm,
   numbersep=-0.5mm,
   backgroundcolor=\color{codebg},
   xleftmargin=-1mm, 
   showstringspaces=false,
   commentstyle=\color{gray},
   stringstyle=\color{OliveGreen},
   emph={turtle,Screen,Turtle},
   emphstyle=\color{RawSienna},
   morekeywords={setheading,goto,backward,forward,left,right,pendown,penup,pensize,color,speed,hideturtle,showturtle,forward}}
   }
   \newcommand{\Cmode}{\lstset{
      language=[ANSI]C,
      linewidth=\linewidth,
      numbers=left,
      tabsize=4,
      frame=single,
      breaklines=true,
      keywordstyle=\ttfamily\color{blue},
      basicstyle=\small,
      numberstyle=\tiny\ttfamily,
      framexleftmargin=0mm,
      numbersep=2mm,
      backgroundcolor=\color{codebg},
      xleftmargin=0mm, 
      showstringspaces=false,
      commentstyle=\color{gray},
      stringstyle=\color{OliveGreen},
      emphstyle=\color{RawSienna},
      escapechar=\|,
      morekeywords={}}
      }
\newcommand{\bashmode}{\lstset{language=bash,numbers=left, tabsize=2, frame=single, breaklines=true, basicstyle=\ttfamily,
   numberstyle=\tiny\ttfamily, framexleftmargin=0mm, backgroundcolor=\color{grispale}, xleftmargin=12mm, showstringspaces=false}}
\newcommand{\exomode}{\lstset{language=python,numbers=left, tabsize=2, frame=single, breaklines=true, basicstyle=\ttfamily,
   numberstyle=\tiny\ttfamily, framexleftmargin=13mm, xleftmargin=12mm, basicstyle=\small, showstringspaces=false}}
   
   
  
%tei pour placer les images
%tei{nom de l’image}{échelle de l’image}{sens}{texte a positionner}
%sens ="1" (droite) ou "2" (gauche)
\newlength{\ltxt}
\newcommand{\tei}[4]{
\setlength{\ltxt}{\linewidth}
\setbox0=\hbox{\includegraphics[scale=#2]{#1}}
\addtolength{\ltxt}{-\wd0}
\addtolength{\ltxt}{-10pt}
\ifthenelse{\equal{#3}{1}}{
\begin{minipage}{\wd0}
\includegraphics[scale=#2]{#1}
\end{minipage}
\hfill
\begin{minipage}{\ltxt}
#4
\end{minipage}
}{
\begin{minipage}{\ltxt}
#4
\end{minipage}
\hfill
\begin{minipage}{\wd0}
\includegraphics[scale=#2]{#1}
\end{minipage}
}
}

%Juxtaposition d'une image pspciture et de texte 
%#1: = code pstricks de l'image
%#2: largeur de l'image
%#3: hauteur de l'image
%#4: Texte à écrire
\newcommand{\ptp}[4]{
\setlength{\ltxt}{\linewidth}
\addtolength{\ltxt}{-#2 cm}
\addtolength{\ltxt}{-0.1 cm}
\begin{minipage}[b][#3 cm][t]{\ltxt}
#4
\end{minipage}\hfill
\begin{minipage}[b][#3 cm][c]{#2 cm}
#1
\end{minipage}\par
}



%Macros pour les graphiques
\psset{linewidth=0.5\pslinewidth,PointSymbol=x}
\setlength{\fboxrule}{0.5pt}
\newcounter{tempangle}

%Marque la longueur du segment d'extrémité  #1 et  #2 avec la valeur #3, #4 est la distance par rapport au segment (en %age de la valeur de celui ci) et #5 l'orientation du marquage : +90 ou -90
\newcommand{\afflong}[5]{
\pstRotation[RotAngle=#4,PointSymbol=none,PointName=none]{#1}{#2}[X] 
\pstHomO[PointSymbol=none,PointName=none,HomCoef=#5]{#1}{X}[Y]
\pstTranslation[PointSymbol=none,PointName=none]{#1}{#2}{Y}[Z]
 \ncline{|<->|,linewidth=0.25\pslinewidth}{Y}{Z} \ncput*[nrot=:U]{\footnotesize{#3}}
}
\newcommand{\afflongb}[3]{
\ncline{|<->|,linewidth=0}{#1}{#2} \naput*[nrot=:U]{\footnotesize{#3}}
}

%Construis le point #4 situé à #2 cm du point #1 avant un angle #3 par rapport à l'horizontale. #5 = liste de paramètre
\newcommand{\lsegment}[5]{\pstGeonode[PointSymbol=none,PointName=none](0,0){O'}(#2,0){I'} \pstTranslation[PointSymbol=none,PointName=none]{O'}{I'}{#1}[J'] \pstRotation[RotAngle=#3,PointSymbol=x,#5]{#1}{J'}[#4]}
\newcommand{\tsegment}[5]{\pstGeonode[PointSymbol=none,PointName=none](0,0){O'}(#2,0){I'} \pstTranslation[PointSymbol=none,PointName=none]{O'}{I'}{#1}[J'] \pstRotation[RotAngle=#3,PointSymbol=x,#5]{#1}{J'}[#4] \pstLineAB{#4}{#1}}

%Construis le point #4 situé à #3 cm du point #1 et faisant un angle de  90° avec la droite (#1,#2) #5 = liste de paramètre
\newcommand{\psegment}[5]{
\pstGeonode[PointSymbol=none,PointName=none](0,0){O'}(#3,0){I'}
 \pstTranslation[PointSymbol=none,PointName=none]{O'}{I'}{#1}[J']
 \pstInterLC[PointSymbol=none,PointName=none]{#1}{#2}{#1}{J'}{M1}{M2} \pstRotation[RotAngle=-90,PointSymbol=x,#5]{#1}{M1}[#4]
  }
  
%Construis le point #4 situé à #3 cm du point #1 et faisant un angle de  #5° avec la droite (#1,#2) #6 = liste de paramètre
\newcommand{\mlogo}[6]{
\pstGeonode[PointSymbol=none,PointName=none](0,0){O'}(#3,0){I'}
 \pstTranslation[PointSymbol=none,PointName=none]{O'}{I'}{#1}[J']
 \pstInterLC[PointSymbol=none,PointName=none]{#1}{#2}{#1}{J'}{M1}{M2} \pstRotation[RotAngle=#5,PointSymbol=x,#6]{#1}{M2}[#4]
  }

% Construis un triangle avec #1=liste des 3 sommets séparés par des virgules, #2=liste des 3 longueurs séparés par des virgules, #3 et #4 : paramètre d'affichage des 2e et 3 points et #5 : inclinaison par rapport à l'horizontale
%autre macro identique mais sans tracer les segments joignant les sommets
\noexpandarg
\newcommand{\Triangleccc}[5]{
\StrBefore{#1}{,}[\pointA]
\StrBetween[1,2]{#1}{,}{,}[\pointB]
\StrBehind[2]{#1}{,}[\pointC]
\StrBefore{#2}{,}[\coteA]
\StrBetween[1,2]{#2}{,}{,}[\coteB]
\StrBehind[2]{#2}{,}[\coteC]
\tsegment{\pointA}{\coteA}{#5}{\pointB}{#3} 
\lsegment{\pointA}{\coteB}{0}{Z1}{PointSymbol=none, PointName=none}
\lsegment{\pointB}{\coteC}{0}{Z2}{PointSymbol=none, PointName=none}
\pstInterCC{\pointA}{Z1}{\pointB}{Z2}{\pointC}{Z3} 
\pstLineAB{\pointA}{\pointC} \pstLineAB{\pointB}{\pointC}
\pstSymO[PointName=\pointC,#4]{C}{C}[C]
}
\noexpandarg
\newcommand{\TrianglecccP}[5]{
\StrBefore{#1}{,}[\pointA]
\StrBetween[1,2]{#1}{,}{,}[\pointB]
\StrBehind[2]{#1}{,}[\pointC]
\StrBefore{#2}{,}[\coteA]
\StrBetween[1,2]{#2}{,}{,}[\coteB]
\StrBehind[2]{#2}{,}[\coteC]
\tsegment{\pointA}{\coteA}{#5}{\pointB}{#3} 
\lsegment{\pointA}{\coteB}{0}{Z1}{PointSymbol=none, PointName=none}
\lsegment{\pointB}{\coteC}{0}{Z2}{PointSymbol=none, PointName=none}
\pstInterCC[PointNameB=none,PointSymbolB=none,#4]{\pointA}{Z1}{\pointB}{Z2}{\pointC}{Z1} 
}


% Construis un triangle avec #1=liste des 3 sommets séparés par des virgules, #2=liste formée de 2 longueurs et d'un angle séparés par des virgules, #3 et #4 : paramètre d'affichage des 2e et 3 points et #5 : inclinaison par rapport à l'horizontale
%autre macro identique mais sans tracer les segments joignant les sommets
\newcommand{\Trianglecca}[5]{
\StrBefore{#1}{,}[\pointA]
\StrBetween[1,2]{#1}{,}{,}[\pointB]
\StrBehind[2]{#1}{,}[\pointC]
\StrBefore{#2}{,}[\coteA]
\StrBetween[1,2]{#2}{,}{,}[\coteB]
\StrBehind[2]{#2}{,}[\angleA]
\tsegment{\pointA}{\coteA}{#5}{\pointB}{#3} 
\setcounter{tempangle}{#5}
\addtocounter{tempangle}{\angleA}
\tsegment{\pointA}{\coteB}{\thetempangle}{\pointC}{#4}
\pstLineAB{\pointB}{\pointC}
}
\newcommand{\TriangleccaP}[5]{
\StrBefore{#1}{,}[\pointA]
\StrBetween[1,2]{#1}{,}{,}[\pointB]
\StrBehind[2]{#1}{,}[\pointC]
\StrBefore{#2}{,}[\coteA]
\StrBetween[1,2]{#2}{,}{,}[\coteB]
\StrBehind[2]{#2}{,}[\angleA]
\lsegment{\pointA}{\coteA}{#5}{\pointB}{#3} 
\setcounter{tempangle}{#5}
\addtocounter{tempangle}{\angleA}
\lsegment{\pointA}{\coteB}{\thetempangle}{\pointC}{#4}
}

% Construis un triangle avec #1=liste des 3 sommets séparés par des virgules, #2=liste formée de 1 longueurs et de deux angle séparés par des virgules, #3 et #4 : paramètre d'affichage des 2e et 3 points et #5 : inclinaison par rapport à l'horizontale
%autre macro identique mais sans tracer les segments joignant les sommets
\newcommand{\Trianglecaa}[5]{
\StrBefore{#1}{,}[\pointA]
\StrBetween[1,2]{#1}{,}{,}[\pointB]
\StrBehind[2]{#1}{,}[\pointC]
\StrBefore{#2}{,}[\coteA]
\StrBetween[1,2]{#2}{,}{,}[\angleA]
\StrBehind[2]{#2}{,}[\angleB]
\tsegment{\pointA}{\coteA}{#5}{\pointB}{#3} 
\setcounter{tempangle}{#5}
\addtocounter{tempangle}{\angleA}
\lsegment{\pointA}{1}{\thetempangle}{Z1}{PointSymbol=none, PointName=none}
\setcounter{tempangle}{#5}
\addtocounter{tempangle}{180}
\addtocounter{tempangle}{-\angleB}
\lsegment{\pointB}{1}{\thetempangle}{Z2}{PointSymbol=none, PointName=none}
\pstInterLL[#4]{\pointA}{Z1}{\pointB}{Z2}{\pointC}
\pstLineAB{\pointA}{\pointC}
\pstLineAB{\pointB}{\pointC}
}
\newcommand{\TrianglecaaP}[5]{
\StrBefore{#1}{,}[\pointA]
\StrBetween[1,2]{#1}{,}{,}[\pointB]
\StrBehind[2]{#1}{,}[\pointC]
\StrBefore{#2}{,}[\coteA]
\StrBetween[1,2]{#2}{,}{,}[\angleA]
\StrBehind[2]{#2}{,}[\angleB]
\lsegment{\pointA}{\coteA}{#5}{\pointB}{#3} 
\setcounter{tempangle}{#5}
\addtocounter{tempangle}{\angleA}
\lsegment{\pointA}{1}{\thetempangle}{Z1}{PointSymbol=none, PointName=none}
\setcounter{tempangle}{#5}
\addtocounter{tempangle}{180}
\addtocounter{tempangle}{-\angleB}
\lsegment{\pointB}{1}{\thetempangle}{Z2}{PointSymbol=none, PointName=none}
\pstInterLL[#4]{\pointA}{Z1}{\pointB}{Z2}{\pointC}
}

%Construction d'un cercle de centre #1 et de rayon #2 (en cm)
\newcommand{\Cercle}[2]{
\lsegment{#1}{#2}{0}{Z1}{PointSymbol=none, PointName=none}
\pstCircleOA{#1}{Z1}
}

%construction d'un parallélogramme #1 = liste des sommets, #2 = liste contenant les longueurs de 2 côtés consécutifs et leurs angles;  #3, #4 et #5 : paramètre d'affichage des sommets #6 inclinaison par rapport à l'horizontale 
% meme macro sans le tracé des segements
\newcommand{\Para}[6]{
\StrBefore{#1}{,}[\pointA]
\StrBetween[1,2]{#1}{,}{,}[\pointB]
\StrBetween[2,3]{#1}{,}{,}[\pointC]
\StrBehind[3]{#1}{,}[\pointD]
\StrBefore{#2}{,}[\longueur]
\StrBetween[1,2]{#2}{,}{,}[\largeur]
\StrBehind[2]{#2}{,}[\angle]
\tsegment{\pointA}{\longueur}{#6}{\pointB}{#3} 
\setcounter{tempangle}{#6}
\addtocounter{tempangle}{\angle}
\tsegment{\pointA}{\largeur}{\thetempangle}{\pointD}{#5}
\pstMiddleAB[PointName=none,PointSymbol=none]{\pointB}{\pointD}{Z1}
\pstSymO[#4]{Z1}{\pointA}[\pointC]
\pstLineAB{\pointB}{\pointC}
\pstLineAB{\pointC}{\pointD}
}
\newcommand{\ParaP}[6]{
\StrBefore{#1}{,}[\pointA]
\StrBetween[1,2]{#1}{,}{,}[\pointB]
\StrBetween[2,3]{#1}{,}{,}[\pointC]
\StrBehind[3]{#1}{,}[\pointD]
\StrBefore{#2}{,}[\longueur]
\StrBetween[1,2]{#2}{,}{,}[\largeur]
\StrBehind[2]{#2}{,}[\angle]
\lsegment{\pointA}{\longueur}{#6}{\pointB}{#3} 
\setcounter{tempangle}{#6}
\addtocounter{tempangle}{\angle}
\lsegment{\pointA}{\largeur}{\thetempangle}{\pointD}{#5}
\pstMiddleAB[PointName=none,PointSymbol=none]{\pointB}{\pointD}{Z1}
\pstSymO[#4]{Z1}{\pointA}[\pointC]
}


%construction d'un cerf-volant #1 = liste des sommets, #2 = liste contenant les longueurs de 2 côtés consécutifs et leurs angles;  #3, #4 et #5 : paramètre d'affichage des sommets #6 inclinaison par rapport à l'horizontale 
% meme macro sans le tracé des segements
\newcommand{\CerfVolant}[6]{
\StrBefore{#1}{,}[\pointA]
\StrBetween[1,2]{#1}{,}{,}[\pointB]
\StrBetween[2,3]{#1}{,}{,}[\pointC]
\StrBehind[3]{#1}{,}[\pointD]
\StrBefore{#2}{,}[\longueur]
\StrBetween[1,2]{#2}{,}{,}[\largeur]
\StrBehind[2]{#2}{,}[\angle]
\tsegment{\pointA}{\longueur}{#6}{\pointB}{#3} 
\setcounter{tempangle}{#6}
\addtocounter{tempangle}{\angle}
\tsegment{\pointA}{\largeur}{\thetempangle}{\pointD}{#5}
\pstOrtSym[#4]{\pointB}{\pointD}{\pointA}[\pointC]
\pstLineAB{\pointB}{\pointC}
\pstLineAB{\pointC}{\pointD}
}

%construction d'un quadrilatère quelconque #1 = liste des sommets, #2 = liste contenant les longueurs des 4 côtés et l'angle entre 2 cotés consécutifs  #3, #4 et #5 : paramètre d'affichage des sommets #6 inclinaison par rapport à l'horizontale 
% meme macro sans le tracé des segements
\newcommand{\Quadri}[6]{
\StrBefore{#1}{,}[\pointA]
\StrBetween[1,2]{#1}{,}{,}[\pointB]
\StrBetween[2,3]{#1}{,}{,}[\pointC]
\StrBehind[3]{#1}{,}[\pointD]
\StrBefore{#2}{,}[\coteA]
\StrBetween[1,2]{#2}{,}{,}[\coteB]
\StrBetween[2,3]{#2}{,}{,}[\coteC]
\StrBetween[3,4]{#2}{,}{,}[\coteD]
\StrBehind[4]{#2}{,}[\angle]
\tsegment{\pointA}{\coteA}{#6}{\pointB}{#3} 
\setcounter{tempangle}{#6}
\addtocounter{tempangle}{\angle}
\tsegment{\pointA}{\coteD}{\thetempangle}{\pointD}{#5}
\lsegment{\pointB}{\coteB}{0}{Z1}{PointSymbol=none, PointName=none}
\lsegment{\pointD}{\coteC}{0}{Z2}{PointSymbol=none, PointName=none}
\pstInterCC[PointNameA=none,PointSymbolA=none,#4]{\pointB}{Z1}{\pointD}{Z2}{Z3}{\pointC} 
\pstLineAB{\pointB}{\pointC}
\pstLineAB{\pointC}{\pointD}
}


% Définition des colonnes centrées ou à droite pour tabularx
\newcolumntype{Y}{>{\centering\arraybackslash}X}
\newcolumntype{Z}{>{\flushright\arraybackslash}X}

%Les pointillés à remplir par les élèves
\newcommand{\po}[1]{\makebox[#1 cm]{\dotfill}}
\newcommand{\lpo}[1][3]{%
\multido{}{#1}{\makebox[\linewidth]{\dotfill}
}}

%Liste des pictogrammes utilisés sur la fiche d'exercice ou d'activités
\newcommand{\bombe}{\faBomb}
\newcommand{\livre}{\faBook}
\newcommand{\calculatrice}{\faCalculator}
\newcommand{\oral}{\faCommentO}
\newcommand{\surfeuille}{\faEdit}
\newcommand{\ordinateur}{\faLaptop}
\newcommand{\ordi}{\faDesktop}
\newcommand{\ciseaux}{\faScissors}
\newcommand{\danger}{\faExclamationTriangle}
\newcommand{\out}{\faSignOut}
\newcommand{\cadeau}{\faGift}
\newcommand{\flash}{\faBolt}
\newcommand{\lumiere}{\faLightbulb}
\newcommand{\compas}{\dsmathematical}
\newcommand{\calcullitteral}{\faTimesCircleO}
\newcommand{\raisonnement}{\faCogs}
\newcommand{\recherche}{\faSearch}
\newcommand{\rappel}{\faHistory}
\newcommand{\video}{\faFilm}
\newcommand{\capacite}{\faPuzzlePiece}
\newcommand{\aide}{\faLifeRing}
\newcommand{\loin}{\faExternalLink}
\newcommand{\groupe}{\faUsers}
\newcommand{\bac}{\faGraduationCap}
\newcommand{\histoire}{\faUniversity}
\newcommand{\coeur}{\faSave}
\newcommand{\python}{\faPython}
\newcommand{\os}{\faMicrochip}
\newcommand{\rd}{\faCubes}
\newcommand{\data}{\faColumns}
\newcommand{\web}{\faCode}
\newcommand{\prog}{\faFile}
\newcommand{\algo}{\faCogs}
\newcommand{\important}{\faExclamationCircle}
\newcommand{\maths}{\faTimesCircle}
% Traitement des données en tables
\newcommand{\tables}{\faColumns}
% Types construits
\newcommand{\construits}{\faCubes}
% Type et valeurs de base
\newcommand{\debase}{{\footnotesize \faCube}}
% Systèmes d'exploitation
\newcommand{\linux}{\faLinux}
\newcommand{\sd}{\faProjectDiagram}
\newcommand{\bd}{\faDatabase}

%Les ensembles de nombres
\renewcommand{\N}{\mathbb{N}}
\newcommand{\D}{\mathbb{D}}
\newcommand{\Z}{\mathbb{Z}}
\newcommand{\Q}{\mathbb{Q}}
\newcommand{\R}{\mathbb{R}}
\newcommand{\C}{\mathbb{C}}

%Ecriture des vecteurs
\newcommand{\vect}[1]{\vbox{\halign{##\cr 
  \tiny\rightarrowfill\cr\noalign{\nointerlineskip\vskip1pt} 
  $#1\mskip2mu$\cr}}}


%Compteur activités/exos et question et mise en forme titre et questions
\newcounter{numact}
\setcounter{numact}{1}
\newcounter{numseance}
\setcounter{numseance}{1}
\newcounter{numexo}
\setcounter{numexo}{0}
\newcounter{numprojet}
\setcounter{numprojet}{0}
\newcounter{numquestion}
\newcommand{\espace}[1]{\rule[-1ex]{0pt}{#1 cm}}
\newcommand{\Quest}[3]{
\addtocounter{numquestion}{1}
\begin{tabularx}{\textwidth}{X|m{1cm}|}
\cline{2-2}
\textbf{\sffamily{\alph{numquestion})}} #1 & \dots / #2 \\
\hline 
\multicolumn{2}{|l|}{\espace{#3}} \\
\hline
\end{tabularx}
}
\newcommand{\QuestR}[3]{
\addtocounter{numquestion}{1}
\begin{tabularx}{\textwidth}{X|m{1cm}|}
\cline{2-2}
\textbf{\sffamily{\alph{numquestion})}} #1 & \dots / #2 \\
\hline 
\multicolumn{2}{|l|}{\cor{#3}} \\
\hline
\end{tabularx}
}
\newcommand{\Pre}{{\sc nsi} 1\textsuperscript{e}}
\newcommand{\Term}{{\sc nsi} Terminale}
\newcommand{\Sec}{2\textsuperscript{e}}
\newcommand{\Exo}[2]{ \addtocounter{numexo}{1} \ding{113} \textbf{\sffamily{Exercice \thenumexo}} : \textit{#1} \hfill #2  \setcounter{numquestion}{0}}
\newcommand{\Projet}[1]{ \addtocounter{numprojet}{1} \ding{118} \textbf{\sffamily{Projet \thenumprojet}} : \textit{#1}}
\newcommand{\ExoD}[2]{ \addtocounter{numexo}{1} \ding{113} \textbf{\sffamily{Exercice \thenumexo}}  \textit{(#1 pts)} \hfill #2  \setcounter{numquestion}{0}}
\newcommand{\ExoB}[2]{ \addtocounter{numexo}{1} \ding{113} \textbf{\sffamily{Exercice \thenumexo}}  \textit{(Bonus de +#1 pts maximum)} \hfill #2  \setcounter{numquestion}{0}}
\newcommand{\Act}[2]{ \ding{113} \textbf{\sffamily{Activité \thenumact}} : \textit{#1} \hfill #2  \addtocounter{numact}{1} \setcounter{numquestion}{0}}
\newcommand{\Seance}{ \rule{1.5cm}{0.5pt}\raisebox{-3pt}{\framebox[4cm]{\textbf{\sffamily{Séance \thenumseance}}}}\hrulefill  \\
  \addtocounter{numseance}{1}}
\newcommand{\Acti}[2]{ {\footnotesize \ding{117}} \textbf{\sffamily{Activité \thenumact}} : \textit{#1} \hfill #2  \addtocounter{numact}{1} \setcounter{numquestion}{0}}
\newcommand{\titre}[1]{\begin{Large}\textbf{\ding{118}}\end{Large} \begin{large}\textbf{ #1}\end{large} \vspace{0.2cm}}
\newcommand{\QListe}[1][0]{
\ifthenelse{#1=0}
{\begin{enumerate}[partopsep=0pt,topsep=0pt,parsep=0pt,itemsep=0pt,label=\textbf{\sffamily{\arabic*.}},series=question]}
{\begin{enumerate}[resume*=question]}}
\newcommand{\SQListe}[1][0]{
\ifthenelse{#1=0}
{\begin{enumerate}[partopsep=0pt,topsep=0pt,parsep=0pt,itemsep=0pt,label=\textbf{\sffamily{\alph*)}},series=squestion]}
{\begin{enumerate}[resume*=squestion]}}
\newcommand{\SQListeL}[1][0]{
\ifthenelse{#1=0}
{\begin{enumerate*}[partopsep=0pt,topsep=0pt,parsep=0pt,itemsep=0pt,label=\textbf{\sffamily{\alph*)}},series=squestion]}
{\begin{enumerate*}[resume*=squestion]}}
\newcommand{\FinListe}{\end{enumerate}}
\newcommand{\FinListeL}{\end{enumerate*}}

%Mise en forme de la correction
\newcommand{\cor}[1]{\par \textcolor{OliveGreen}{#1}}
\newcommand{\br}[1]{\cor{\textbf{#1}}}
\newcommand{\tcor}[1]{\begin{tcolorbox}[width=0.92\textwidth,colback={white},colbacktitle=white,coltitle=OliveGreen,colframe=green!75!black,boxrule=0.2mm]   
\cor{#1}
\end{tcolorbox}
}
\newcommand{\rc}[1]{\textcolor{OliveGreen}{#1}}
\newcommand{\pmc}[1]{\textcolor{blue}{\tt #1}}
\newcommand{\tmc}[1]{\textcolor{RawSienna}{\tt #1}}


%Référence aux exercices par leur numéro
\newcommand{\refexo}[1]{
\refstepcounter{numexo}
\addtocounter{numexo}{-1}
\label{#1}}

%Séparation entre deux activités
\newcommand{\separateur}{\begin{center}
\rule{1.5cm}{0.5pt}\raisebox{-3pt}{\ding{117}}\rule{1.5cm}{0.5pt}  \vspace{0.2cm}
\end{center}}

%Entête et pied de page
\newcommand{\snt}[1]{\lhead{\textbf{SNT -- La photographie numérique} \rhead{\textit{Lycée Nord}}}}
\newcommand{\Activites}[2]{\lhead{\textbf{{\sc #1}}}
\rhead{Activités -- \textbf{#2}}
\cfoot{}}
\newcommand{\Exos}[2]{\lhead{\textbf{Fiche d'exercices: {\sc #1}}}
\rhead{Niveau: \textbf{#2}}
\cfoot{}}
\newcommand{\Devoir}[2]{\lhead{\textbf{Devoir de mathématiques : {\sc #1}}}
\rhead{\textbf{#2}} \setlength{\fboxsep}{8pt}
\begin{center}
%Titre de la fiche
\fbox{\parbox[b][1cm][t]{0.3\textwidth}{Nom : \hfill \po{3} \par \vfill Prénom : \hfill \po{3}} } \hfill 
\fbox{\parbox[b][1cm][t]{0.6\textwidth}{Note : \po{1} / 20} }
\end{center} \cfoot{}}

%Devoir programmation en NSI (pas à rendre sur papier)
\newcommand{\PNSI}[2]{\lhead{\textbf{Devoir de {\sc nsi} : \textsf{ #1}}}
\rhead{\textbf{#2}} \setlength{\fboxsep}{8pt}
\begin{tcolorbox}[title=\textcolor{black}{\danger\; A lire attentivement},colbacktitle=lightgray]
{\begin{enumerate}
\item Rendre tous vous programmes en les envoyant par mail à l'adresse {\tt fnativel2@ac-reunion.fr}, en précisant bien dans le sujet vos noms et prénoms
\item Un programme qui fonctionne mal ou pas du tout peut rapporter des points
\item Les bonnes pratiques de programmation (clarté et lisiblité du code) rapportent des points
\end{enumerate}
}
\end{tcolorbox}
 \cfoot{}}


%Devoir de NSI
\newcommand{\DNSI}[2]{\lhead{\textbf{Devoir de {\sc nsi} : \textsf{ #1}}}
\rhead{\textbf{#2}} \setlength{\fboxsep}{8pt}
\begin{center}
%Titre de la fiche
\fbox{\parbox[b][1cm][t]{0.3\textwidth}{Nom : \hfill \po{3} \par \vfill Prénom : \hfill \po{3}} } \hfill 
\fbox{\parbox[b][1cm][t]{0.6\textwidth}{Note : \po{1} / 10} }
\end{center} \cfoot{}}

\newcommand{\DevoirNSI}[2]{\lhead{\textbf{Devoir de {\sc nsi} : {\sc #1}}}
\rhead{\textbf{#2}} \setlength{\fboxsep}{8pt}
\cfoot{}}

%La définition de la commande QCM pour auto-multiple-choice
%En premier argument le sujet du qcm, deuxième argument : la classe, 3e : la durée prévue et #4 : présence ou non de questions avec plusieurs bonnes réponses
\newcommand{\QCM}[4]{
{\large \textbf{\ding{52} QCM : #1}} -- Durée : \textbf{#3 min} \hfill {\large Note : \dots/10} 
\hrule \vspace{0.1cm}\namefield{}
Nom :  \textbf{\textbf{\nom{}}} \qquad \qquad Prénom :  \textbf{\prenom{}}  \hfill Classe: \textbf{#2}
\vspace{0.2cm}
\hrule  
\begin{itemize}[itemsep=0pt]
\item[-] \textit{Une bonne réponse vaut un point, une absence de réponse n'enlève pas de point. }
\item[\danger] \textit{Une mauvaise réponse enlève un point.}
\ifthenelse{#4=1}{\item[-] \textit{Les questions marquées du symbole \multiSymbole{} peuvent avoir plusieurs bonnes réponses possibles.}}{}
\end{itemize}
}
\newcommand{\DevoirC}[2]{
\renewcommand{\footrulewidth}{0.5pt}
\lhead{\textbf{Devoir de mathématiques : {\sc #1}}}
\rhead{\textbf{#2}} \setlength{\fboxsep}{8pt}
\fbox{\parbox[b][0.4cm][t]{0.955\textwidth}{Nom : \po{5} \hfill Prénom : \po{5} \hfill Classe: \textbf{1}\textsuperscript{$\dots$}} } 
\rfoot{\thepage} \cfoot{} \lfoot{Lycée Nord}}
\newcommand{\DevoirInfo}[2]{\lhead{\textbf{Evaluation : {\sc #1}}}
\rhead{\textbf{#2}} \setlength{\fboxsep}{8pt}
 \cfoot{}}
\newcommand{\DM}[2]{\lhead{\textbf{Devoir maison à rendre le #1}} \rhead{\textbf{#2}}}

%Macros permettant l'affichage des touches de la calculatrice
%Touches classiques : #1 = 0 fond blanc pour les nombres et #1= 1gris pour les opérations et entrer, second paramètre=contenu
%Si #2=1 touche arrondi avec fond gris
\newcommand{\TCalc}[2]{
\setlength{\fboxsep}{0.1pt}
\ifthenelse{#1=0}
{\psframebox[fillstyle=solid, fillcolor=white]{\parbox[c][0.25cm][c]{0.6cm}{\centering #2}}}
{\ifthenelse{#1=1}
{\psframebox[fillstyle=solid, fillcolor=lightgray]{\parbox[c][0.25cm][c]{0.6cm}{\centering #2}}}
{\psframebox[framearc=.5,fillstyle=solid, fillcolor=white]{\parbox[c][0.25cm][c]{0.6cm}{\centering #2}}}
}}
\newcommand{\Talpha}{\psdblframebox[fillstyle=solid, fillcolor=white]{\hspace{-0.05cm}\parbox[c][0.25cm][c]{0.65cm}{\centering \scriptsize{alpha}}} \;}
\newcommand{\Tsec}{\psdblframebox[fillstyle=solid, fillcolor=white]{\parbox[c][0.25cm][c]{0.6cm}{\centering \scriptsize 2nde}} \;}
\newcommand{\Tfx}{\psdblframebox[fillstyle=solid, fillcolor=white]{\parbox[c][0.25cm][c]{0.6cm}{\centering \scriptsize $f(x)$}} \;}
\newcommand{\Tvar}{\psframebox[framearc=.5,fillstyle=solid, fillcolor=white]{\hspace{-0.22cm} \parbox[c][0.25cm][c]{0.82cm}{$\scriptscriptstyle{X,T,\theta,n}$}}}
\newcommand{\Tgraphe}{\psdblframebox[fillstyle=solid, fillcolor=white]{\hspace{-0.08cm}\parbox[c][0.25cm][c]{0.68cm}{\centering \tiny{graphe}}} \;}
\newcommand{\Tfen}{\psdblframebox[fillstyle=solid, fillcolor=white]{\hspace{-0.08cm}\parbox[c][0.25cm][c]{0.68cm}{\centering \tiny{fenêtre}}} \;}
\newcommand{\Ttrace}{\psdblframebox[fillstyle=solid, fillcolor=white]{\parbox[c][0.25cm][c]{0.6cm}{\centering \scriptsize{trace}}} \;}

% Macroi pour l'affichage  d'un entier n dans  une base b
\newcommand{\base}[2]{ \overline{#1}^{#2}}
% Intervalle d'entiers
\newcommand{\intN}[2]{\llbracket #1; #2 \rrbracket}}

% Numéro et titre de chapitre
\setcounter{numchap}{10}
\newcommand{\Ctitle}{\cnum {Tableaux associatifs, hachage}}
\newcommand{\SPATH}{/home/fenarius/Travail/Cours/cpge-info/docs/mp2i/files/C\thenumchap/}

% Exemple introductif
\makess{Exemple introductif}
\begin{frame}[fragile]{\Ctitle}{\stitle}
	\begin{block}{Nombre d'occurence des mots d'un texte}
		A partir du texte de l'oeuvre de J. Verne, \textit{\og{} \numprint{20000} lieux sous les mers \fg{}}, on a construit la liste des mots (sans accent, ni majuscules) qui apparaissent dans cet oeuvre. A titre d'exemple voici un extrait du fichier obtenu
		\begin{langageC*}{fontsize=\footnotesize,tabsize=0}
			cetace
			extraordinaire
			pouvait
			se
			transporter
			un
			endroit
			un
			autre
		\end{langageC*}
		\onslide<2->{Le but du problème est de trouver une méthode \textcolor{blue}{efficace} afin d'obtenir le nombre d'occurrence de chaque mot}
	\end{block}
\end{frame}

\makess{Définition}
\begin{frame}[fragile]{\Ctitle}{\stitle}
	\begin{block}{Définition}
		Un \textcolor{BrickRed}{tableau associatif} (ou \textcolor{BrickRed}{dictionnaire}) est une structure de données constituée d'un ensemble de \textcolor{blue}{clés} $C$ et d'un ensemble de \textcolor{blue}{valeurs} $V$. Chaque clé n'apparait qu'une fois dans la structure et est associée à un élément de $V$.\\
		\textcolor{gray}{Les dictionnaires étendent en quelque sorte la notion de tableau, les indices des éléments d'un tableau associatif de taille $n$ n'étant plus nécessairement les entiers $\intN{0}{n-1}$ comme pour les tableaux classiques.}
	\end{block}
	\onslide<2->{
		\begin{exampleblock}{Exemple}
			On peut associer à chaque mot d'un texte, son nombre d'occurrence dans ce texte :
			\begin{itemize}
				\item<3-> L'ensemble des clés est l'ensemble des mots du texte.
				\item<4-> L'ensemble des valeurs est $\mathbb{N}$.
			\end{itemize}
			\onslide<5->{Ce tableau associatif peut se noter :
				\mintinline{c}{{("un",10), ("cours",3), ("exercice",7) ...}}}
		\end{exampleblock}}
\end{frame}


\begin{frame}[fragile]{\Ctitle}{\stitle}
	\begin{block}{Interface d'un tableau associatif}
		En notant $T$ un tableau associatif d'ensemble de clé $C$ et de valeur $V$,  $c \in C$ et $v \in V$ :
		\begin{itemize}
			\item<2-> Tester si une clé $c$ appartient à $T$
			\item<3-> Ajouter à $T$ une association $(c,v)$
			\item<4-> Supprimer une association $(c,v)$ de $T$
			\item<5-> Obtenir la valeur $v$ associée à une clé $c$.
			\item<6-> Modifier la valeur associée à une clé.
		\end{itemize}
	\end{block}
\end{frame}

\makess{Techniques d'implémentation}
\begin{frame}[fragile]{\Ctitle}{\stitle}
	\begin{block}{Possibilités d'implémentation}
		\begin{itemize}
			\item<1-> Une implémentation naïve à l'aide d'une liste chainée de maillons contenant les couples (clé, valeur) est possible mais clairement inefficace \onslide<2->(le test d'appartenance est alors en $\mathcal{O}(n)$)
			\item<3-> Une implémentation utilisant les arbres sera vue ultérieurement.
			\item<4-> Lorsque l'ensemble des clés est inclus dans $\intN{0}{N-1}$, on peut utiliser un tableau de taille $N$, pour un couple $(c,v)$, on stocke alors la valeur $v$ dans la case d'indice $c$ du tableau.
			\item<5-> Pour un ensemble de clés quelconque $C$, on se ramène au cas précédent en deux étapes :
				\begin{itemize}
					\item<6-> On utilise une fonction $h : C \rightarrow \N$, dite \textcolor{BrickRed}{fonction de hachage} (\textit{hash function}).
					\item<7-> L'indice de la clé $c$ dans le tableau s'obtient alors en prenant $h(c) \mod N$.
				\end{itemize}
				\onslide<8->{\textcolor{BrickRed}{\small \danger} Deux clés différentes peuvent alors donner le \textit{même} indice dans le tableau, on parle alors de \textcolor{BrickRed}{collision}.}
		\end{itemize}
	\end{block}
\end{frame}


\begin{frame}{\Ctitle}{\stitle}
	\begin{block}{Visualisation}
		\mintinline{c}{{ "dans":12 , "le" : 8, "un" : 16, "bol" : 10} } \\ \vspace{0.2cm}
		\begin{tabularx}{\textwidth}{X|c|X}
			\multicolumn{1}{l}{\quad {\textcolor{blue}{Clés}}}                                     & \multicolumn{1}{c}{\textcolor{blue}{Indice}} & \multicolumn{1}{c}{\textcolor{blue}{Contenu}}                      \\
			\cline{2-2}
			{\rnode{dans}{\begin{cadre}{codebg}{blue}{2.2}{0.4}{\footnotesize "dans"}\end{cadre}}} & 0                                            &                                                                    \\
			\cline{2-2}
			                                                                                       & \rnode{i1}{1}                                & \leavevmode{\onslide<3->{\quad \quad \rnode{v1}{\tt ("dans",12)}}} \\
			\cline{2-2}
			                                                                                       & \rnode{i2}{2}                                & \leavevmode{\onslide<5->{\quad \quad \rnode{v2}{\tt ("un",16)}}}   \\
			\cline{2-2}
			{\rnode{le}{\begin{cadre}{codebg}{blue}{2.2}{0.4}{\footnotesize "le"}\end{cadre}}}     & \vdots                                       &                                                                    \\
			\cline{2-2}
			                                                                                       & \rnode{i42}{42}                              & \leavevmode{\onslide<4->{\quad \quad \rnode{v42}{\tt ("le",8)}}}   \\
			\cline{2-2}
			{\rnode{un}{\begin{cadre}{codebg}{blue}{2.2}{0.4}{\footnotesize "un"}\end{cadre}}}     & \vdots                                       &                                                                    \\
			\cline{2-2}
			                                                                                       & $N$-1                                        &                                                                    \\
			\cline{2-2}
		\end{tabularx}
		\onslide<2->{\ncline[nodesepB=0.45,offsetA=0.1,offsetB=-0.05,linewidth=0.8pt,linecolor=brown]{->}{dans}{i1} \naput[nrot=:U,labelsep=0.05]{\textcolor{brown}{\footnotesize hash + mod}}}
		\onslide<5->{\ncline[nodesepB=0.42,nodesepA=0.13,offsetA=-0.35,offsetB=0.15,linewidth=0.8pt,linecolor=brown]{->}{un}{i2}}
		\onslide<4->{\ncline[nodesepB=0.42,offsetA=0.1,offsetB=-0.05,linewidth=0.8pt,linecolor=brown]{->}{le}{i42}}
		\onslide<3->{\ncline[nodesepA=0.3]{o->}{i1}{v1}}
		\onslide<5->{\ncline[nodesepA=0.3]{o->}{i2}{v2}}
		\onslide<4->{\ncline[nodesepA=0.3]{o->}{i42}{v42}}
		\vspace{0.2cm}\\
		\onslide<6->{Certaines cases du tableau peuvent restées vides ! Le \textit{taux de charge} de la table est défini comme le rapport entre le nombres de cases occupés et $N$.}
	\end{block}
\end{frame}


\begin{frame}{\Ctitle}{\stitle}
	\begin{block}{Collision}
		\mintinline{c}{{ "dans":12 , "le" : 8, "un" : 16, "bol" : 10} } \\ \vspace{0.2cm}
		\begin{tabularx}{\textwidth}{X|c|X}
			\multicolumn{1}{l}{\quad {\textcolor{blue}{Clés}}}                                                            & \multicolumn{1}{c}{\textcolor{blue}{Indice}} & \multicolumn{1}{c}{\textcolor{blue}{Contenu}}                                                                                          \\
			\cline{2-2}
			{\rnode{dans}{\begin{cadre}{codebg}{blue}{2.2}{0.4}{\footnotesize "dans"}\end{cadre}}}                        & 0                                            &                                                                                                                                        \\
			\cline{2-2}
			                                                                                                              & \rnode{i1}{1}                                & \quad \quad \rnode{v1}{\tt ("dans",12)}                                                                                                \\
			\cline{2-2}
			                                                                                                              & \rnode{i2}{2}                                & \quad \quad \rnode{v2}{\tt ("un",16)}                                                                                                  \\
			\cline{2-2}
			{\rnode{le}{\begin{cadre}{codebg}{blue}{2.2}{0.4}{\footnotesize "le"}\end{cadre}}}                            & \vdots                                       &                                                                                                                                        \\
			\cline{2-2}
			                                                                                                              & \rnode{i42}{42}                              & \alt<7->{\quad \quad \rnode{v42}{\textcolor{BrickRed}{\tt ("bol",10) $\rightarrow$ ("le",8) }}}{\quad \quad \rnode{v42}{\tt ("le",8)}} \\
			\cline{2-2}
			{\rnode{un}{\begin{cadre}{codebg}{blue}{2.2}{0.4}{\footnotesize "un"}\end{cadre}}}                            & \vdots                                       &                                                                                                                                        \\
			\cline{2-2}
			\leavevmode{\onslide<2->{\rnode{bol}{\begin{cadre}{codebg}{blue}{2.2}{0.4}{\footnotesize "bol"}\end{cadre}}}} & $N$-1                                        &                                                                                                                                        \\
			\cline{2-2}
		\end{tabularx}
		\ncline[nodesepB=0.45,offsetA=0.1,offsetB=-0.05,linewidth=0.8pt,linecolor=brown]{->}{dans}{i1} \naput[nrot=:U,labelsep=0.05]{\textcolor{brown}{\footnotesize hash + mod}}
		\ncline[nodesepB=0.42,nodesepA=0.13,offsetA=-0.35,offsetB=0.15,linewidth=0.8pt,linecolor=brown]{->}{un}{i2}
		\ncline[nodesepB=0.42,offsetA=0.1,offsetB=-0.05,linewidth=0.8pt,linecolor=brown]{->}{le}{i42}
		\ncline[nodesepA=0.3]{o->}{i1}{v1}
		\ncline[nodesepA=0.3]{o->}{i2}{v2}
		\ncline[nodesepA=0.3]{o->}{i42}{v42}
		\onslide<3->{\ncline[nodesepB=0.42,offsetA=-0.2,nodesepA=0.04,offsetB=-0.05,linewidth=0.8pt,linecolor=brown]{->}{bol}{i42}}
		\begin{itemize}
			\item<4->{\small Deux clés différentes ("bol" et "le"), doivent être rangées au même indice dans une tableau, c'est une \textcolor{BrickRed}{collision}.}
			\item<5->{\small La résolution par chaînage consiste à stocker des listes chaînées, dans le tableau. On dit alors que chaque case du tableau est un \textcolor{blue}{seau} (ou \textcolor{gray}{bucket} en anglais) .}
		\end{itemize}
	\end{block}
\end{frame}

\makess{Complexité}
\begin{frame}{\Ctitle}{\stitle}
	\begin{block}{Complexité des opérations}
		\textcolor{BrickRed}{\small \danger} On se place dans le cas d'une \textit{résolution des collisions par chainage}.
		\begin{itemize}
			\item<1-> Le nombre de collisions influence directement la complexité des opérations \\
			\onslide<2->\textcolor{gray}{\small Dans le cas extrême où toutes les valeurs sont en collision, le tableau associatif est représentée par une liste chainée et les opérations sont en $\mathcal{O}(n)$}
			\item<3-> On s'intéresse donc à la probabilité d'apparitions de collisions dans le cas d'un \textcolor{blue}{hachage uniforme}, c'est à dire dans le cas où les valeurs de hachage ont la même probabilité d'apparition.
		\end{itemize}
	\end{block}
	\onslide<4->{
	\begin{block}{Probabilité de collision}
		On considère $n$ clés hachées uniformément sur $N$ valeurs ($N \geqslant n$), alors la probabilité d'absence de collision est :
		$$ P_0 = \dfrac{N!}{N^n\,(N-n)!}$$
	\end{block}}
\end{frame}

\begin{frame}{\Ctitle}{\stitle}
	\begin{exampleblock}{Applications numériques}
		\renewcommand{\arraystretch}{1.4}
		En notant $P$ la probabilité qu'au moins une collisions survienne :
		\begin{tabular}{|c|c|c|c|c}
			\cline{1-4}
			$n$ & $N$ & $P_0$ & $P = 1 - P_0$ & \\
			\cline{1-4}
			$23$ & $365$ & $0.49$ & $0.51$ & \textcolor{gray}{\small (paradoxe des anniversaires)} \\
			\cline{1-4}
			$1000$ & $10^6$ & $0.6$  & $0.4$ & \\
			\cline{1-4}
			$5000$ & $10^6$ & $0.3\,10^{-6}$ &  $0.999996$ &\\
			\cline{1-4}
		\end{tabular} \\
	Les collisions sont donc difficilement évitables, ceci dit on dispose du résultat suivant (admis)
	\end{exampleblock}
	\onslide<2->\begin{alertblock}{Complexité d'une recherche}
		Si on suppose le hachage uniforme et les collisions résolues par chainage, le temps moyen d'une recherche est $\mathcal{O}(1+\alpha)$, où $\alpha = \frac{n}{N}$ est le taux de charge de la table.
	\end{alertblock}
\end{frame}

\begin{frame}{\Ctitle}{\stitle}
	\begin{block}{A retenir ...}
			Sous les hypothèses suivantes :
			\begin{itemize}
			\item<2-> le calcul de la fonction de hachage  est en $\mathcal{O}(1)$,
			\item<2-> le hachage est uniforme,
			\item<2-> le taux de charge de la table est majoré indépendamment de $n$ (par exemple $N$ proportionnel à $n$),
			\end{itemize}
			\onslide<3->la complexité moyenne des opérations (appartenance, ajout, suppression, \dots) est en $\mathcal{O}(1)$.
	\end{block}
\end{frame}


\makess{Implémentation en langage C}
\begin{frame}{\Ctitle}{\stitle}
	\begin{block}{Les listes chainées de clés/valeurs}
		\begin{itemize}
			\item<1-> On commence par créer des listes chaînées de clés/valeurs. Les clés sont des chaines de caractères, et les valeurs des entiers positifs. Pour simplifier, on suppose qu'un mot a au maximum 26 lettres :
				\onslide<2->{\inputpartC{\SPATH/count.c}{}{\scriptsize}{9}{16}}
			\item<3-> Les prototypes des fonctions nécessaires :
				\begin{itemize}
					\item<4-> Test si une clé est présente : \mintinline{c}{bool is_in(list l, char w[26])}
					\item<5-> Ajout d'une nouvelle clé : \mintinline{c}{void insert(list *l, char w[26])}
					\item<6-> Récupérer la valeur associée à une clé : \mintinline{c}{int value(list l, char w[26])}
					\item<7-> Modification d'une valeur: \mintinline{c}{void update(list *l, char w[26],int v)}
				\end{itemize}
		\end{itemize}
	\end{block}
\end{frame}

\begin{frame}{\Ctitle}{\stitle}
	\begin{block}{La table de hachage}
		On définit alors la table de hachage comme un tableau de {\sc size} alvéoles contenant chacune une liste chainée (la constante {\sc size} pouvant être définie en début de programme) les prototypes des fonctions à écrire sont :
		\begin{itemize}
			\item<1-> \mintinline{c}{bool is_in_hashtable(list ht[SIZE], char w[26])}
			\item<2-> \mintinline{c}{void insert_in_hashtable(list ht[SIZE], char w[26])}
			\item<3-> \mintinline{c}{void update_hashtable(list ht[SIZE], char w[26], int n)}
			\item<4-> \mintinline{c}{int get_val_hashtable(list ht[SIZE], char w[26])}
		\end{itemize}
		\onslide<5->{Cette implémentation sera vue en détail en TP.}
	\end{block}
\end{frame}

\makess{Implémentation en OCaml}
\begin{frame}{\Ctitle}{\stitle}
	\begin{block}{Implémentation avec le type \textcolor{yellow}{\tt array}}
		On peut créer un type représentant un tableau de liste. Les éléments des listes étant des couples ($c$,$v$) où $c$ est de type \kw{string} et $v$ de type \kw{int} :
		\inputpartOCaml{\SPATH/count2.ml}{}{\small}{1}{1}
		\onslide<2->{Sur les listes, les opérations nécessaires sont les mêmes que celles vu sur l'implémentation en C :}
		\begin{itemize}
			\item<3-> Le test d'appartenance :
				\inputpartOCaml{\SPATH/count2.ml}{}{\small}{14}{17}
			\item<4-> La mise à jour de la valeur associée à une clé (on utilise \kw{failwith} en cas d'absence de la clé).
			\item<5-> La récupération de la valeur associée à une clé.
		\end{itemize}
	\end{block}
\end{frame}

\begin{frame}{\Ctitle}{\stitle}
	\begin{block}{Implémentation avec le type \textcolor{yellow}{\tt array}}
		\begin{itemize}
			\item<1->On se fixe une taille pour la table de hachage et la définit comme un tableau de liste de couples \mintinline{ocaml}{string*int} :
			\inputpartOCaml{\SPATH/count2.ml}{}{\small}{2}{3}
			\item<2-> La fonction de hachage {\tt string -> int} transforme une chaine de caractère en un entier compris entre 0 (inclus) et {\tt size} (exclus)
			\item<3-> On doit alors écrire les fonctions pour :
				\begin{itemize}
					\item<4->  test si la clé $c$ est présente ({\tt hashtable -> string -> bool}) : \\
						\mintinline{ocaml}{let is_in_ht (ht:hashtable) c}
					\item<5-> ajout de ($c$,$v$) dans la table ({\tt hashtable -> string -> int -> unit}) : \\
						\mintinline{ocaml}{let add_ht (ht:hashtable) c v}
					\item<6-> renvoie la valeur associé à $c$ ({\tt hastable -> string -> int}) : \\
						\mintinline{ocaml}{let get_value_ht (ht:hashtable) c}
					\item<7-> met à jour la valeur associée  à c ({\tt hashtable -> string -> int -> unit}) : \\
						\mintinline{ocaml}{let update_ht (ht:hashtable) c v}
				\end{itemize}
		\end{itemize}
	\end{block}
\end{frame}

\begin{frame}{\Ctitle}{\stitle}
	\begin{block}{Module {\tt Hashtbl}}
		La bibliothèque standard d'OCaml propose une implémentation des tables de hachage via le module \kw{Hashtbl}.
		\begin{itemize}
			\item<1-> On inclus ce module avec :
				\inputpartOCaml{\SPATH/count.ml}{}{\small}{1}{1}
			\item<2-> La fonction de hachage (\kw{Hashtbl.hash}) est prédéfinie par OCaml et s'applique à une valeur de n'importe quel type.
			\item<3-> On doit donner une taille initiale lors de la création de la table de hachage :
				\inputpartOCaml{\SPATH/count.ml}{}{\small}{2}{2}
				A noter que lorsque le taux de remplissage de cette table devient trop important, elle est redimensionnée (la taille double), c'est donc un tableau dynamique.
		\end{itemize}
	\end{block}
\end{frame}

\begin{frame}{\Ctitle}{\stitle}
	\begin{block}{Fonctions du module {\tt Hashtbl}}
		Parmi les fonctions disponibles, on retrouve celles vues dans l'implémentation en langage C :
		\begin{itemize}
			\item<2-> \mintinline{ocaml}{Hashtbl.mem} pour le test d'appartenance.
			\item<3-> \mintinline{ocaml}{Hashtbl.add} pour ajouter une couple (clé, valeur).
			\item<4-> \mintinline{ocaml}{Hashtbl.find} pour trouver la valeur associée à une clé.
			\item<5-> \mintinline{ocaml}{Hashtbl.replace} pour modifier la valeur associée à une clé.
		\end{itemize}
	\end{block}
\end{frame}

\makess{Une application}
\begin{frame}{\Ctitle}{\stitle}
	\begin{block}{Calcul des coefficients du binôme}
		\begin{enumerate}
			\item<1-> Rappeler la définition des coefficients binomiaux à l'aide de factoriel ainsi que la relation de récurrence liant les coefficients binomiaux
			\item<2-> Ecrire une fonction en OCaml utilisant la définition à base de factoriels permettant de calculer $\binom{n}{k}$ (avec $n$ et $k$ deux entiers naturels $n\geq k$). Donner sa complexité.
			\item<3-> Proposer une version récursive utilisant la relation de récurrence vue à la question 1.
			\item<4-> On note $N(n,k)$ le nombre d'appels  nécessaires pour calculer $\binom{n}{k}$. Donner $N(n,0)$, $N(n,n)$ et une relation de récurrence liant $N(n,k), N(n-1,k)$ et $N(n-1,k-1)$.
			\item<5-> Prouver par récurrence que $N(n,k) = 2\binom{n}{k} -1$. En déduire la complexité de la fonction récursive écrite à la question 3.
		\end{enumerate}
	\end{block}
\end{frame}

\begin{frame}{\Ctitle}{\stitle}
	\begin{block}{Mémoïsation avec une table de hachage}
		\begin{itemize}
			\item<1-> La complexité exponentielle de la version récursive est lié à l'apparition de nombreux appels récursifs identiques.
			\item<2-> L'idée est donc de mémoriser les résultats des appels récursifs déjà effectués afin d'en avoir le résultat sans les relancer. Cette technique de programmation s'appelle la \textcolor{blue}{mémoïsation}
			\item<3-> Quelle structure de données vous paraît adaptée ?
			\item<4-> Proposer une implémentation en OCaml
			\item<5-> Donner la complexité de cette nouvelle implémentation.
		\end{itemize}
	\end{block}
\end{frame}

\begin{frame}{\Ctitle}{\stitle}
	\begin{alertblock}{Mémoïsation}
		\begin{itemize}
			\item<1-> La \textcolor{blue}{mémoïsation} consiste à stocker dans une structure de données les valeurs renvoyées par une fonction afin de ne pas les recalculer lors des appels identiques suivant.\\
			\item<2-> Les tableaux associatifs sont alors des structures de données adaptées, les clés sont les paramètres d'appels de la fonction et les valeurs le résultat de l'appel.
			\item<3-> On peut alors tester si la valeur a déjà été calculée (présence de la clé) en $\mathcal{O}(1)$ et suivant le cas de figure calculer la valeur associée puis la stocker ($\mathcal{O}(1)$) ou alors la récupérer ($\mathcal{O}(1)$).
		\end{itemize}
	\end{alertblock}
\end{frame}


\end{document}
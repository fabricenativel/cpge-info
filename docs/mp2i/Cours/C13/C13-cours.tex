\PassOptionsToPackage{dvipsnames,table}{xcolor}
\documentclass[10pt]{beamer}
\usepackage{Cours}

\begin{document}

\input{\detokenize{/home/fenarius/Travail/Cours/cpge-info/latex/MacrosCours.tex}}

% Numéro et titre de chapitre
\setcounter{numchap}{13}
\newcommand{\Ctitle}{\cnum {Force brute, retour sur trace}}
\newcommand{\SPATH}{/home/fenarius/Travail/Cours/cpge-info/docs/mp2i/files/C\thenumchap/}

\makess{Force brute : généralités}
\begin{frame}[fragile]{\Ctitle}{\stitle}
	\begin{alertblock}{Définition}
		La recherche par \textcolor{blue}{force brute} (\textit{brute force}) consiste à parcourir toutes les solutions possibles d'un problème en testant si elles conviennent.\\
		\onslide<2->{De façon formelle, si on note $V$ l'ensemble des candidats, et $P$ une propriété des éléments de $V$, on teste (en les énumérant) les $x \in V$ jusqu'à en trouver un qui vérifie $P$.}
	\end{alertblock}
	\onslide<3->{
		\begin{block}{Remarques}
			\begin{itemize}
				\item<3-> Dans certains problèmes on cherche à déterminer \textit{toutes} les solutions et donc on ne s'arrête pas à la première rencontrée
				\item<4-> On doit pouvoir \textcolor{blue}{énumérer de façon exhaustive} les éléments de $V$ (ce qui peut être difficile dans certains cas).
			\end{itemize}
		\end{block}}
\end{frame}

\begin{frame}[fragile]{\Ctitle}{\stitle}
	\begin{exampleblock}{Exemple}
		\begin{itemize}
			\item La recherche d'un mot de passe par force brute : $V$ est l'ensemble des chaines de caractères et $P(x)$ est vérifié si $x$ est le mot de passe cherché. Dans ce cas on pourra réaliser l'énumération des candidats en commençant par les chaines de longueur 1, puis 2, \dots
			\item<2-> $V$ est l'ensemble des grilles complétées d'un sudoku et $P$ vérifie si la grille est valide.
			\item<3-> $V$ est l'ensemble des permutations possibles d'une liste d'entiers et $P$ vérifie si la permutation est triée en ordre croissant (\textcolor{blue}{bogosort}).
		\end{itemize}
	\end{exampleblock}
\end{frame}

\begin{frame}[fragile]{\Ctitle}{\stitle}
	\begin{block}{Complexité}
		En supposant l'ensemble $V$ fini, une recherche exhaustive effectue au plus $|V|$ itérations.\\
		\textcolor{BrickRed}{\small \danger} Cela ne signifie pas que la complexité est toujours en $O(V)$ car tester si une solution vérifie $P$ ou non peut avoir un coût non constant.
	\end{block}
    \onslide<2->{
	\begin{block}{Problème d'optimisation}
		Dans les problèmes du type \og{} \textit{déterminer $x \in V$ tel que $f(x)$ soit minimale (ou maximale)} \fg{}, l'exploration exhaustive va résoudre le problème en calculant toutes les images par $f$ des éléments de V.
	\end{block}}
\end{frame}

\makess{Mot de passe par force brute}
\begin{frame}[fragile]{\Ctitle}{\stitle}
	\begin{exampleblock}{Exercice}
		Tableau des temps pour la recherche de mot de passe par force brute :
		\begin{center}
			\includegraphics[width=130px]{mdp.eps}
		\end{center}
		On s'intéresse à la recherche d'un mot de passe de 8 lettres minusucles.
		\begin{itemize}
			\item<2-> Combien il y a-t-il de mots de passes possibles ? Quel est le temps indiqué dans le tableau ?
			\item<3-> Proposer un algorithme permettant d'énumérer les possibilités.
			\item<4-> Ecrire en C, une fonction de signature \mintinline{c}{char *bruteforce(char *mdp, char *charset, int size)
				} qui implémente une recherche de mot de passe par force brute.
		\end{itemize}
	\end{exampleblock}
\end{frame}

\begin{frame}[fragile]{\Ctitle}{\stitle}
	\begin{exampleblock}{Proposition de solution}
		\inputpartC{\SPATH/mdp.c}{}{\tiny}{5}{35}
	\end{exampleblock}
\end{frame}

\makess{Résolution par retour sur trace}
\begin{frame}[fragile]{\Ctitle}{\stitle}
	\begin{alertblock}{Définition}
		Le \textcolor{blue}{retour sur trace} (\textit{backtracking}) consiste à construire la solution d'un problème à partir d'une solution partielle. On s'arrête dès qu'une incohérence est rencontrée dans la solution partielle et on revient en arrière afin de modifier une décision prise précédemment.
	\end{alertblock}
	\onslide<2->{
		\begin{exampleblock}{Exemple}
			Pour résoudre un Sudoku :
			\begin{itemize}
				\item<3-> La force brute parcourt l'ensemble des valeurs possibles pour toutes les cases restantes
				\item<4-> Le \textit{backtracking} place des valeurs au fur et à mesure et revient en arrière si une impossibilité est rencontrée.
			\end{itemize}
		\end{exampleblock}}
\end{frame}

\makess{Résolution du problème des n reines par \textit{backtracking}}
\newcommand{\queen}{\textcolor{BrickRed}{\!\faChessQueen}}
\begin{frame}[fragile]{\Ctitle}{\stitle}
	\begin{exampleblock}{Exemple}
		{\small Le problème des $n$ reines consiste à placer $n$ reines sur une échiquier de taille $n$ de façon à ce que deux reines ne soit pas sur la même ligne, même colonne ou même diagonale (c'est-à-dire qu'aucune reine n'en menace une autre) \\}
		\smallskip
		\onslide<2->{\small Une solution dans le cas $n=8$ est proposée ci-dessous :
			\begin{center}
				\input{\SPATH/sol42}
			\end{center}}
	\end{exampleblock}
\end{frame}


\begin{frame}[fragile]{\Ctitle}{\stitle}
	\begin{exampleblock}{Représentation du problème}
        \begin{itemize}
		\item <1-> On sait qu'il y a une seule reine par colonne, on peut donc représenter une position de jeu par un tableau de taille $n$ contenant les numéros de ligne des $k$ reines déjà placées.
		\item <2-> L'algorithme va alors consister à tenter de placer la reine $k+1$ sur chacune des lignes $0, \dots, k-1$
		\begin{itemize}
            \item si cela génère une menace, on essayer la possibilité suivante, en revenant récursivement à la reine précédente si nécesssaire
            \item sinon on place la reine suivante, si c'est la dernière reine, une solution est trouvée.
        \end{itemize}
    \end{itemize}
	\end{exampleblock}
\end{frame}

\begin{frame}[fragile]{\Ctitle}{\stitle}
	\begin{exampleblock}{Début de l'algorithme ($n=4$)}
        \begin{tabularx}{\textwidth}{YYY}
            \renewcommand{\arraystretch}{1.5}
 \begin{tabular}{|P{0.2cm}|P{0.2cm}|P{0.2cm}|P{0.2cm}|P{0.2cm}|P{0.2cm}|P{0.2cm}|P{0.2cm}|} 
 \hline\cellcolor{white}{}&\cellcolor{black!30}{}&\cellcolor{white}{}&\cellcolor{black!30}{}\\
 \hline 
\cellcolor{black!30}{}&\cellcolor{white}{}&\cellcolor{black!30}{}&\cellcolor{white}{}\\
 \hline 
\cellcolor{white}{}&\cellcolor{black!30}{}&\cellcolor{white}{}&\cellcolor{black!30}{}\\
 \hline 
\cellcolor{black!30}{\queen}&\cellcolor{white}{}&\cellcolor{black!30}{}&\cellcolor{white}{}\\
 \hline 
\end{tabular}  & \leavevmode{\onslide<2->{\renewcommand{\arraystretch}{1.5}\begin{tabular}{|P{0.2cm}|P{0.2cm}|P{0.2cm}|P{0.2cm}|P{0.2cm}|P{0.2cm}|P{0.2cm}|P{0.2cm}|} 
    \hline\cellcolor{white}{}&\cellcolor{black!30}{}&\cellcolor{white}{}&\cellcolor{black!30}{}\\
    \hline 
   \cellcolor{black!30}{}&\cellcolor{white}{}&\cellcolor{black!30}{}&\cellcolor{white}{}\\
    \hline 
   \cellcolor{white}{}&\cellcolor{black!30}{}&\cellcolor{white}{}&\cellcolor{black!30}{}\\
    \hline 
   \cellcolor{black!30}{\queen}&\cellcolor{white}{\queen}&\cellcolor{black!30}{}&\cellcolor{white}{}\\
    \hline 
   \end{tabular}}} &
   \leavevmode{\onslide<3->{\renewcommand{\arraystretch}{1.5}\begin{tabular}{|P{0.2cm}|P{0.2cm}|P{0.2cm}|P{0.2cm}|P{0.2cm}|P{0.2cm}|P{0.2cm}|P{0.2cm}|} 
    \hline\cellcolor{white}{}&\cellcolor{black!30}{}&\cellcolor{white}{}&\cellcolor{black!30}{}\\
    \hline 
   \cellcolor{black!30}{}&\cellcolor{white}{}&\cellcolor{black!30}{}&\cellcolor{white}{}\\
    \hline 
   \cellcolor{white}{}&\cellcolor{black!30}{\queen}&\cellcolor{white}{}&\cellcolor{black!30}{}\\
    \hline 
   \cellcolor{black!30}{\queen}&\cellcolor{white}{}&\cellcolor{black!30}{}&\cellcolor{white}{}\\
    \hline 
   \end{tabular}}}\\
   & & \\
   \leavevmode{\onslide<4->{\renewcommand{\arraystretch}{1.5}\begin{tabular}{|P{0.2cm}|P{0.2cm}|P{0.2cm}|P{0.2cm}|P{0.2cm}|P{0.2cm}|P{0.2cm}|P{0.2cm}|} 
    \hline\cellcolor{white}{}&\cellcolor{black!30}{}&\cellcolor{white}{}&\cellcolor{black!30}{}\\
    \hline 
   \cellcolor{black!30}{}&\cellcolor{white}{\queen}&\cellcolor{black!30}{}&\cellcolor{white}{}\\
    \hline 
   \cellcolor{white}{}&\cellcolor{black!30}{}&\cellcolor{white}{}&\cellcolor{black!30}{}\\
    \hline 
   \cellcolor{black!30}{\queen}&\cellcolor{white}{}&\cellcolor{black!30}{}&\cellcolor{white}{}\\
    \hline 
   \end{tabular}}} & 
   \leavevmode{\onslide<5->{\renewcommand{\arraystretch}{1.5}\begin{tabular}{|P{0.2cm}|P{0.2cm}|P{0.2cm}|P{0.2cm}|P{0.2cm}|P{0.2cm}|P{0.2cm}|P{0.2cm}|} 
    \hline\cellcolor{white}{}&\cellcolor{black!30}{}&\cellcolor{white}{}&\cellcolor{black!30}{}\\
    \hline 
   \cellcolor{black!30}{}&\cellcolor{white}{\queen}&\cellcolor{black!30}{}&\cellcolor{white}{}\\
    \hline 
   \cellcolor{white}{}&\cellcolor{black!30}{}&\cellcolor{white}{}&\cellcolor{black!30}{}\\
    \hline 
   \cellcolor{black!30}{\queen}&\cellcolor{white}{}&\cellcolor{black!30}{\queen}&\cellcolor{white}{}\\
    \hline 
   \end{tabular}}}
   & 
   \leavevmode{\onslide<6->{\renewcommand{\arraystretch}{1.5}\begin{tabular}{|P{0.2cm}|P{0.2cm}|P{0.2cm}|P{0.2cm}|P{0.2cm}|P{0.2cm}|P{0.2cm}|P{0.2cm}|} 
    \hline\cellcolor{white}{}&\cellcolor{black!30}{}&\cellcolor{white}{}&\cellcolor{black!30}{}\\
    \hline 
   \cellcolor{black!30}{}&\cellcolor{white}{\queen}&\cellcolor{black!30}{}&\cellcolor{white}{}\\
    \hline 
   \cellcolor{white}{}&\cellcolor{black!30}{}&\cellcolor{white}{\queen}&\cellcolor{black!30}{}\\
    \hline 
   \cellcolor{black!30}{\queen}&\cellcolor{white}{}&\cellcolor{black!30}{}&\cellcolor{white}{}\\
    \hline 
   \end{tabular}}}
   \\
        \end{tabularx}
	\end{exampleblock}
\end{frame}

\begin{frame}[fragile]{\Ctitle}{\stitle}
    \begin{exampleblock}{Implémentation en langage C}
        \begin{enumerate}
            \item Ecrire une fonction de signature \mintinline{c}{bool menace(int tab[], int idx)} qui renvoie \mintinline{c}{true} si la reine située en colonne {\tt idx} menace l'une des reines situés en colonne {\tt 0, \dots, idx-1}.\\
            \onslide<2->{\inputpartC{\SPATH/reines.c}{}{\footnotesize}{63}{73}}
            \item<3-> Ecrire une fonction qui renvoie la première solution rencontrée sous la forme du tableau des positions par colonne des $n$ reines.
        \end{enumerate}
    \end{exampleblock}
\end{frame}


\begin{frame}[fragile]{\Ctitle}{\stitle}
    \begin{exampleblock}{Proposition de solution}
            \inputpartC{\SPATH/reines.c}{}{\footnotesize}{100}{118}
    \end{exampleblock}
\end{frame}

\begin{frame}[fragile]{\Ctitle}{\stitle}
    \begin{exampleblock}{Exercice}
            Modifier la fonction précédente afin qu'elle calcule le nombre total de solutions au problème.
            \onslide<2->{\inputpartC{\SPATH/reines.c}{}{\footnotesize}{120}{137}}
    \end{exampleblock}
\end{frame}



\end{document}
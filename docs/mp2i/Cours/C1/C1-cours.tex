\PassOptionsToPackage{dvipsnames,table}{xcolor}
\documentclass[10pt]{beamer}
\usepackage{Cours}

\begin{document}

\input{\detokenize{/home/fenarius/Travail/Cours/cpge-info/latex//MacrosCours.tex}}

% Numéro et titre de chapitre
\setcounter{numchap}{1}
\newcommand{\Ctitle}{\cnum {Introduction au langage C}}
\newcommand{\SPATH}{/home/fenarius/Travail/Cours/cpge-info/docs/mp2i/files/C\thenumchap/}

\makess{Historique}
\begin{frame}{\Ctitle}{\stitle}
	\begin{block}{Bref historique}
		\begin{itemize}
			\item<1-> 1972 : début du développement du langage C par Dennis Ritchie et Ken Thomson aux laboratoires Bell parallèlement à la création du système d'exploitation {\sc unix}.
			\item<2-> 1978 : première édition du livre "The C programming language" (Kernighan \& Ritchie)
			\item<3-> 1983 : première standardisation du langage par l'{\sc ansi} qui assure la compatibilité et la portabilité entre différentes plateformes. La dernière standardisation date de 2018 (C18)
			\item<4-> A partir de 1983 : développement de plusieurs dérivés de C, parmi lesquels C++ (B. Strousrtup, 1983), C\# (Microsoft, 2000), Go (Google, 2007), Rust (Mozilla, 2010)
		\end{itemize}
	\end{block}
\end{frame}

\makess{Caractéristiques du C}
\begin{frame}{\Ctitle}{\stitle}
	\begin{block}{Quelques aspects du C}
		\begin{itemize}
			\item<1-> Langage \textcolor{blue}{impératif} : séquence d'instructions exécutées par l'ordinateur pour modifier l'état du programme. C n'est ni orienté objet, ni fonctionnel.
			\item<2-> Les variables sont \textcolor{blue}{mutables} c'est à dire qu'elles peuvent changer de valeur pendant l'exécution.
			\item<3-> Le langage C est \textcolor{blue}{statiquement typé} c'est à dire qu'une variable appartient à un type défini à la déclaration et durant toute sa durée de vie.
			\item<4-> Equipé d'une librairie standard : la \textcolor{blue}{libc}.
			\item<5-> \textcolor{BrickRed}{\small \danger} Le standard précise un certain nombres de \textcolor{blue}{comportements indéfinis}, c'est à dire de programmes dont le résultat est imprévisible.
			\item<6-> Plus \textcolor{blue}{proche de la machine} que bien d'autres langages de haut niveau, ce qui induit une certaine efficacité.
			\item<7-> Souvent utilisé pour le développement de systèmes d'exploitation, de pilotes de périphériques, de logiciels embarqués.
		\end{itemize}
	\end{block}
\end{frame}

\begin{frame}{\Ctitle}{\stitle}
	\begin{block}{Compilation}
		Le langage C est \textcolor{blue}{compilé} : \\ \medskip
		\begin{tabular}{ccccc}
			\rnode{CS}{\begin{rcadre}{lightgray}{Sepia}{2.4}{1.4}
					           \textcolor{Sepia}{\small \faFile\; Code source \\
					           {\footnotesize (fichier(s) texte .c)}}
				           \end{rcadre}} & \hspace{0.8cm} & \onslide<3->{\rnode{{CO}}{\begin{cadre}{white}{black}{2.8}{1.4}  {\small \textbf{\faCog\;} Compilateur} \\ {\footnotesize (gcc, clang, ..)} \end{cadre}}} & \hspace{0.8cm} &
			\onslide<5->{\rnode{EX}{\begin{rcadre}{lightgray}{blue}{2.4}{1.4}
						                        \textcolor{blue}{\small \faFileArchive\; Exécutable \\ {\footnotesize (fichier binaire)}}
					                        \end{rcadre}}}                                                                                                             \\
		\end{tabular}
		\onslide<3->{\ncline[doubleline=true,doublesep=2pt,doublecolor=OliveGreen,linecolor=OliveGreen,linewidth=1pt,arrowsize=10pt,arrowinset=0.2,arrowlength=1.2]{->}{CS}{CO}}
		\onslide<6->{\ncline[doubleline=true,doublesep=2pt,doublecolor=OliveGreen,linecolor=OliveGreen,linewidth=1pt,arrowsize=10pt,arrowinset=0.2,arrowlength=1.2]{->}{CO}{EX}}
		\begin{enumerate}
			\item<2-> Les IDE comme VS Code signalent certaines erreurs dans le code.
			\item<4-> La compilation peut produire des erreurs ou des avertissement (\textit{warning}) \\
				\textcolor{gray}{\small La compilation se déroule en 4 étapes : préprocesseur, compilation, assemblage, editions de lien}
			\item<6-> Une compilation sans erreur (mais éventuellement des \textit{warning}) produit un exécutable.
			\item<7-> Les erreurs dans l'exécution ne feront pas référence aux instructions du code source.
		\end{enumerate}
	\end{block}
\end{frame}



\makess{Exemples de programmes}
\begin{frame}[fragile]{\Ctitle}{\stitle}
	\begin{exampleblock}{Programme minimal}
		\begin{overprint}
			\onslide<1>
			\begin{langageC}
				#include <stdio.h>

				int main()
				{
						printf("Hello world \n");
						return 0;
					}
			\end{langageC}
			\medskip
			Si le fichier texte s'appelle {\tt hello.c}, on lance la compilation avec {\tt gcc hello.c}, l'exécutable produit s'appelle par défaut {\tt a.out}, on peut modifier ce nom avec l'option {\tt -o}. Par exemple : {\tt gcc -o hello.exe hello.c}
			\onslide<2>
			\begin{langageC*}{highlightlines=1}
				#include <stdio.h>

				int main()
				{
						printf("Hello world \n");
						return 0;
					}
			\end{langageC*}
			\medskip
			Appel aux fonctions \textcolor{blue}{\textbf{st}}andar\textcolor{blue}{\textbf d} d'entrées et de sorties (\textcolor{blue}{\textbf i}nput et \textcolor{blue}{\textbf o}utput)  de la libc.
			\onslide<3>
			\begin{langageC}
				#include <stdio.h>

				int |\myem{main}|()
				{
						printf("Hello world \n");
						return 0;
					}
			\end{langageC}
			\medskip
			Un programme C contient une fonction \kw{ main} par laquelle l'exécution du programme commence.
			\onslide<4>
			\begin{langageC}
				#include <stdio.h>

				|\myem{int}| main|\myem{()}|
				{
				printf("Hello world \n");
				return 0;
				}
			\end{langageC}
			\medskip
			Avant le nom d'une fonction on trouve le type de variable qu'elle renvoie (ici \kw{ int}) et après entre parenthèses, les arguments éventuels de la fonction (ici aucun).
			\onslide<5>
			\begin{langageC}
				#include <stdio.h>

				int main()
				|\myem{\{}|
				printf("Hello world \n")|\myem{;}|
				return 0|\myem{;}|
				|\myem{\}}|
			\end{langageC}
			\medskip
			Les blocs d'instructions sont délimités par des accolades (\kw{ \{} et \kw{ \}}). Les instructions doivent se terminer par un point virgule \kw{ ;}. Les espaces, sauts de ligne et indentation sont ignorés par le compilateur, mais sont nécessaires pour une bonne lisibilité.
			\onslide<6>
			\begin{langageC*}{highlightlines=5}
				#include <stdio.h>

				int main()
				{
						printf("Hello world \n");
						return 0;
					}
			\end{langageC*}
			\medskip
			L'instruction \kw{ printf} permet d'afficher dans le terminal. On notera les guillemets (\textcolor{blue}{"}) pour délimiter une chaîne de caractères et le caractère \kw{ \textbackslash n} pour indiquer un retour à la ligne.
			\onslide<7>
			\begin{langageC*}{highlightlines=6}
				#include <stdio.h>

				int main()
				{
						printf("Hello world \n");
						return 0|\myem{;}|
					}
			\end{langageC*}
			\medskip
			L'instruction \kw{ return} quitte la fonction en renvoyant la valeur donnée. Ici, on renvoie {\tt 0}, qui indique traditionnellement que le programme se  termine sans erreurs.
		\end{overprint}
	\end{exampleblock}

\end{frame}

\begin{frame}{\Ctitle}{\stitle}
	\begin{exampleblock}{Exemple de boucle}
		\inputC{\SPATH/cours_ex2.c}{4}{\small}
		\medskip
		Déclaration de la variable {\tt somme} de type \kw{int} et initialisation à zéro. A noter qu'on peut déclarer une variable sans l'initialiser.
	\end{exampleblock}
\end{frame}

\begin{frame}{\Ctitle}{\stitle}
	\begin{exampleblock}{Exemple de boucle}
		\inputC{\SPATH/cours_ex2.c}{5}{\small}
		\medskip
		Une variable dont la valeur ne sera pas modifiée peut être déclaré avec \kw{const}.\\
		\onslide<2-> \textcolor{gray}{On peut aussi utiliser une directive de précompilation \\ {\tt \#define NMAX 100}}
	\end{exampleblock}
\end{frame}

\begin{frame}{\Ctitle}{\stitle}
	\begin{exampleblock}{Exemple de boucle}
		\inputC{\SPATH/cours_ex2.c}{6}{\small}
		\medskip
		On remarque que la boucle \kw{for} est de la forme {\tt for} \textit{(init; fin; incr)}. Les opérateurs de comparaison en C sont \kw{==}, \kw{!=}, \kw{<}, \kw{>}, \kw{<=} et \kw{>=}.
	\end{exampleblock}
\end{frame}

\begin{frame}{\Ctitle}{\stitle}
	\begin{exampleblock}{Exemple de boucle}
		\inputC{\SPATH/cours_ex2.c}{8}{\small}
		\medskip
		On veut afficher un \kw{int} dans la réponse, on utilise \kw{\%{}d} dans \kw{printf} à l'emplacement souhaité.
	\end{exampleblock}
\end{frame}

\begin{frame}{\Ctitle}{\stitle}
	\begin{exampleblock}{Exemple d'instruction conditionnelle}
		\inputC{\SPATH/cours_ex3.c}{2}{\small}
		\medskip
		Une ligne de commentaire commence avec \kw{//}, un commentaire multiligne est encadré par \kw{/*} et \kw{*/}
	\end{exampleblock}
\end{frame}

\begin{frame}{\Ctitle}{\stitle}
	\begin{exampleblock}{Exemple d'instruction conditionnelle}
		\inputC{\SPATH/cours_ex3.c}{3}{\small}
		\medskip
		Définition d'une fonction \kw{syracuse} qui prend comme paramètre un entier et renvoie un entier. C'est la \textcolor{blue}{signature} de la fonction. \\
		\textcolor{BrickRed}{\important} En C, les paramètres sont passés par \textcolor{blue}{valeur}.
	\end{exampleblock}
\end{frame}

\begin{frame}{\Ctitle}{\stitle}
	\begin{exampleblock}{Exemple d'instruction conditionnelle}
		\inputC{\SPATH/cours_ex3.c}{4-7}{\small}
		\medskip
		Instruction conditionnelle : on exécute le bloc qui suit la condition si celle-ci est vérifiée et sinon le bloc qui suit le \kw{else} (s'il est présent).
		Noter les parenthèses autour de la condition.
	\end{exampleblock}
\end{frame}

\makess{Types de base en C}
\begin{frame}{\Ctitle}{\stitle}
	\begin{alertblock}{Types de base}
		\begin{tabularx}{\linewidth}{|l|p{1.8cm}|>{\footnotesize}X|}
			\hline
			Type                                        & Opérations                                                                                                & Commentaires                                                                                                                                       \\
			\hline
			\leavevmode\onslide<2->{\kw{int} et \kw{unsigned int}}               & \leavevmode\onslide<3->{\kw{+}, \kw{-}, \kw{*}, \kw{/}, \kw{\%}  \newline} \leavevmode\onslide<4->{\textcolor{BrickRed}{\small \danger}\textcolor{gray}{{\tt ++},{\tt -{}-}}} & \leavevmode\onslide<5->{Entiers signés ou non signés codés sur un minimum de 16 bits.}                                                                                      \\
			\leavevmode\onslide<6->{\kw{int}\textcolor{Sepia}{$N$}\kw{\_t} et \kw{uint}\textcolor{Sepia}{$N$}\kw{\_t}} &                                                                                                           & \leavevmode\onslide<7->{Entiers codés sur \textcolor{Sepia}{$N$} bits accessibles dans \kw{stdint.h} ($\textcolor{Sepia}{N=8}$, \textcolor{Sepia}{$32$} ou \textcolor{Sepia}{$64$}).}                                                                        \\
			\hline
			\leavevmode\onslide<8->{\kw{float} et \kw{double}}                   & \leavevmode\onslide<8->{\kw{+}, \kw{-}, \kw{*}, \kw{/}}                                                                            & \leavevmode\onslide<8->{Représentation des nombres en virgules flottantes en simple ou double précision de la norme {\sc ieee754}. Fonctions élémentaires dans \kw{math.h}} \\
			\hline
			\leavevmode\onslide<9->{\kw{bool}, valeurs \kw{true} ou \kw{false}.}                                   & \leavevmode\onslide<9->{\kw{||},  \kw{\&\&}, \kw{!}}                                                                                & \leavevmode\onslide<9->{Booléens accessibles dans \kw{stdbool.h}. Evaluations paresseuses des expressions.}                                                                 \\
			\hline
			\leavevmode\onslide<10->{\kw{char}}                                   & \leavevmode\onslide<10->{\textcolor{gray}{\tt +, -}}                                                                                & \leavevmode\onslide<10->{Caractères noté entre quotes (\kw{'}), uniquement ceux de la table {\sc ascii}. Caractère nul : \kw{'\textbackslash{}0'}}                           \\
			\hline
		\end{tabularx}
		\onslide<11-> Pour indiquer l'absence de type, notamment pour les fonctions ne renvoyant rien (par exemple une fonction d'affichage) on utilise \kw{void}.
	\end{alertblock}
\end{frame}

\begin{frame}{\Ctitle}{\stitle}
	\begin{exampleblock}{Exemples}
		\begin{enumerate}
		\item<1-> Déclarer deux variables entières \kw{a} et \kw{b} initialisées respectivement à 2024 et 137. Déclarer \kw{c} ayant comme valeur le reste dans division euclidienne de \kw{a} par \kw{b}.
		\item<2-> Déclarer et initialiser \kw{d} ayant comme valeur $b^2 - 4ac$, en supposant que $a$, $b$ et $c$ sont des variables flottantes existantes et initialisées.
		\item<3-> On suppose déjà déclarées deux variables booléennes \kw{x} et \kw{y}, écrire une expression booléenne correspondant à \kw{x} {\tt xor} \kw{y}. 
		\item<4-> Discuter l'effet de l'instruction conditionnelle \mintinline{c}{if (a > 0 || a / b < 1)} suivant les valeurs des variables entières {\tt a} et {\tt b}.
		\item<5-> La déclaration de variable suivante est-elle correcte ? \mintinline{c}{char c = "a";} 
		\item<6-> Quelle est selon vous la cause du message : \textit{\textcolor{darkgray}{warning: ‘return’ with a value, in function returning void}} lors d'une compilation ?
	\end{enumerate}
	\end{exampleblock}
\end{frame}



\begin{frame}{\Ctitle}{\stitle}
	\begin{block}{Affichage et spécificateur de format}
		En C, l'affichage des variables se fait à l'aide de spécificateurs de format suivant le type de la variable \\
		\textcolor{gray}{
			\begin{tabular}{|l|l|}
				\hline
				Type                                               & Spécificateur \\
				\hline
				\kw{char}                                          & \kw{\%{}c}    \\
				\hline
				\kw{char[]}                                        & \kw{\%{}s}    \\
				\hline
				\kw{unsigned int}, \kw{uint8\_t} et \kw{uint32\_t} & \kw{\%{}u}    \\
				\hline
				\kw{int}, \kw{int8\_t} et \kw{int32\_t}            & \kw{\%{}d}    \\
				\hline
				\kw{float}                                         & \kw{\%{}f}    \\
				\hline
				\kw{uint64\_t}                                     & \kw{\%{}lu}   \\
				\hline
				\kw{int64\_t}                                      & \kw{\%{}ld}   \\
				\hline
			\end{tabular}}
	\end{block}
\end{frame}

\begin{frame}{\Ctitle}{\stitle}
	\begin{alertblock}{Définition}
		\begin{itemize}
			\item<1-> La \textcolor{blue}{portée} d'une variable est la partie du programme  dans laquelle cette variable est visible (on peut y faire référence).
			\item<2-> La portée peut-être :
				\begin{itemize}
					\item<3-> \textcolor{blue}{globale}, c'est à dire que la variable est accessible depuis tout le programme. En C, c'est le cas des variables déclarées en début de programme en dehors de tout bloc d'instructions.
					\item<4-> \textcolor{blue}{locale} lorsque la variable est déclarée dans un bloc d'instruction alors sa portée est limitée à ce bloc. C'est le cas des paramètres d'une fonction ou d'une variable de boucle.
				\end{itemize}
			\item<5-> Lorsque deux variables ont le même identifiant, c'est la variable ayant la plus petite portée (celle définie dans le bloc le plus intérieur) qui est accessible.
		\end{itemize}
	\end{alertblock}
\end{frame}

\begin{frame}[fragile]{\Ctitle}{\stitle}
	\begin{exampleblock}{Exemples}
		\begin{enumerate}
		\item<1->Dans le programme suivant, donner les portées de \kw{maxn}, \kw{n}, \kw{somme}, \kw{i}
		\inputC{\SPATH/cours_portee.c}{}{\footnotesize}
		\item<2-> On définit une variable entière {\tt i}, juste après la ligne 6, donner la portée de cette nouvelle variable. Le programme fonctionne-t-il encore correctement ?
		\end{enumerate}
	\end{exampleblock}
\end{frame}

\begin{frame}[fragile]{\Ctitle}{\stitle}
	\begin{exampleblock}{Exemples}
		\kw{maxn} est une variable globale
		\inputC{\SPATH/cours_portee.c}{3-14}{\small}
	\end{exampleblock}
\end{frame}

\begin{frame}[fragile]{\Ctitle}{\stitle}
	\begin{exampleblock}{Exemples}
		\kw{n} est un paramètre de la fonction \kw{harmo}
		\inputC{\SPATH/cours_portee.c}{5-9}{\small}
	\end{exampleblock}
\end{frame}

\begin{frame}[fragile]{\Ctitle}{\stitle}
	\begin{exampleblock}{Exemples}
		\kw{somme} est déclarée dans la fonction \kw{harmo}
		\inputC{\SPATH/cours_portee.c}{6-9}{\small}
	\end{exampleblock}
\end{frame}

\begin{frame}[fragile]{\Ctitle}{\stitle}
	\begin{exampleblock}{Exemples}
		\kw{i} est locale à la boucle
		\inputC{\SPATH/cours_portee.c}{7-8}{\small}
	\end{exampleblock}
\end{frame}

\begin{frame}[fragile]{\Ctitle}{\stitle}
	\begin{exampleblock}{Exemples}
		\kw{s} est locale au \kw{main}
		\inputC{\SPATH/cours_portee.c}{12-14}{\small}
	\end{exampleblock}
\end{frame}

\begin{frame}[fragile]{\Ctitle}{\stitle}

	\begin{block}{Conversion implicite de type}
		La ligne \mintinline{c}{double somme = 0;} est une \textcolor{blue}{conversion implicite de type}. En effet, 0 est de type entier mais est converti en flottant pour être affecté à la variable \kw{somme} qui est de type double.
	\end{block}
	\onslide<2->{
		\begin{block}{Conversion explicite : \textit{cast}}
			On aurait pu réaliser une \textcolor{blue}{conversion explicite} ou \textit{cast} en spécifiant le type de destination entre parenthèses : \mintinline{c}{double somme = (double) 0;}
		\end{block}}
\end{frame}

\begin{frame}[fragile]{\Ctitle}{\stitle}
	\begin{exampleblock}{Exemple}
		\inputC{\SPATH/cast.c}{}{\small}
		\begin{itemize}
			\item<2-> Quel est le résultat de ce programme ? Pourquoi ?
			\item<3-> Comment afficher le résultat de la division décimale ?
		\end{itemize}
	\end{exampleblock}
\end{frame}

\begin{frame}[fragile]{\Ctitle}{\stitle}
	\begin{exampleblock}{Exercice}
		Prévoir (et éventuellement observer) le résultat du programme suivant, expliquer.
		\inputC{\SPATH/pvaleur.c}{}{\small}
	\end{exampleblock}
\end{frame}

\begin{frame}{\Ctitle}{\stitle}
	\begin{block}{Remarques}
		Afin de repérer \textit{dès la compilation} le maximum de problèmes potentiels, il est \textcolor{BrickRed}{très fortement recommandé} de toujours utiliser \kw{gcc} avec les options :
		\begin{itemize}
			\item<2->\kw{-Wall} affichage de tous les \textit{warning}
			\item<3->\kw{-Wextra} affichage de \textit{warning} supplémentaires
			\item<4->\kw{-Wconversion} pour signaler les problèmes éventuels de conversion implicite
		\end{itemize}
		\onslide<5-> D'autre part, il est préférable de spécifier un fichier un nom pour l'exécutable produit grâce à l'option \kw{-o}
	\end{block}
	\onslide<6->{
		\begin{exampleblock}{Exemple}
			Pour compiler le programme {\tt exemple.c}, la ligne de compilation devrait donc être : \\
			{\tt gcc exemple.c -o exemple.exe -Wall -Wextra -Wconversion}
		\end{exampleblock}}
\end{frame}


\makess{Structures de contrôle}
\begin{frame}[fragile]{\Ctitle}{\stitle}
	\begin{alertblock}{Conditionnelle}
		\begin{itemize}
			\item<1-> \kw{if} \kw{(}\textit{condition}\kw{)}  \kw{\{} \textit{instruction} \kw{\}}
			\item<2-> \kw{if} \kw{(}\textit{condition}\kw{)}  \kw{\{} \textit{instruction} \kw{\}} else \kw{\{} \textit{instruction} \kw{\}}
		\end{itemize}
	\onslide<3-> \textcolor{BrickRed}{\small \danger} Pas de {\kw ;} après la condition !
	\end{alertblock}
	\begin{exampleblock}{Exemple}
		\onslide<3->{Ecrire une fonction \kw{compare} en C, prenant comme paramètre deux entiers \tt{a} et \tt{b} et renvoyant \tt{-1} si \tt{a<b}, \tt{0} si \tt{a=b} et 1 sinon.}
	\end{exampleblock}
\end{frame}

\begin{frame}[fragile]{\Ctitle}{\stitle}
	\begin{exampleblock}{Correction de l'exemple}
		\inputpartC{\SPATH/compare.c}{}{\small}{4}{16}
	\end{exampleblock}
\end{frame}

\begin{frame}[fragile]{\Ctitle}{\stitle}
	\begin{exampleblock}{Exemples}
		Ecrire les instructions conditionnelles suivantes (les variables utilisées sont supposées déjà déclarées):
		\begin{enumerate}
			\item<1-> Affiche "Ok" si \kw{a} est positif.
			\item<2-> Affecte \kw{nb} à 2 si \kw{d} est strictement positif, 1 si \kw{d} est nul et 0 sinon.
			\item<3-> Affecte \kw{e} à 1 si \kw{a} et \kw{b} ont la même parité et 0 sinon.
		\end{enumerate}
	\end{exampleblock}
\end{frame}

\begin{frame}[fragile]{\Ctitle}{\stitle}
	\begin{alertblock}{Boucles}
		\begin{itemize}
			\item<1-> \kw{for} \kw{(}\textit{init}\kw{;}\textit{fin};\textit{increment}\kw{)}  \kw{\{} \textit{instruction} \kw{\}} \\
			permet de répéter le bloc d'instruction pour chaque valeur prise par la variable de boucle.
				\onslide<2->{Généralement utilisé sous la forme : \mintinline{c}{for (int i=0; i<n; i=i+1)} {\tt\{} \dots {\tt\} }}\\
			\onslide<3->{\textcolor{gray}{On tolère {\tt i++} pour l'incrémentation, on recommande fortement de ne {\textit pas} utilisé cet opérateur dans un autre contexte.}}
			\item<4-> \kw{while} \kw{(}\textit{condition}\kw{)}  \kw{\{} \textit{instruction} \kw{\}} \\
			permet de répéter le bloc d'instruction tant que la condition est vérifiée.
			\item<5-> Lorsqu'une boucle se trouve dans le corps d'une fonction, une instruction \kw{return} a pour effet de quitter immédiatement cette boucle (et le corps de la fonction) et de revenir au point d'appel de la fonction.
			\item<6-> Une boucle peut-être interrompue avec l'instruction \kw{break}.
			\item<7-> \textcolor{BrickRed}{\small \danger \;} Pas de \kw{;} après la condition.
		\end{itemize}
	\end{alertblock}
\end{frame}

\begin{frame}[fragile]{\Ctitle}{\stitle}
	\begin{exampleblock}{Exemple}
		Le type \kw{char} correspond en fait à une valeur entière, les caractères imprimables vont de 32 (l'espace) à 127 ({\sc del}). Sachant que l'affichage d'un caractère avec \kw{printf} se fait à l'aide de \kw{\%c}
		\begin{itemize}
			\item<2-> Ecrire une boucle \kw{for} permettant d'afficher ces caractères.
			\item<3-> Faire de même avec une boucle \kw{while}.
		\end{itemize}
	\end{exampleblock}
\end{frame}

\begin{frame}[fragile]{\Ctitle}{\stitle}
	\begin{exampleblock}{Correction de l'exemple}
		\begin{itemize}
			\item<1-> Avec une boucle \kw{for} \\
				\inputC{\SPATH/cours_ex5.c}{}{\small}
			\item<2-> Avec une boucle \kw{while} \\
				\inputC{\SPATH/cours_ex6.c}{}{\small}
		\end{itemize}
	\end{exampleblock}
\end{frame}

\begin{frame}[fragile]{\Ctitle}{\stitle}
	\begin{exampleblock}{Exercices}
	Ecrire une boucle permettant :
	\begin{enumerate}
		\item<1-> d'afficher les 10 premiers multiples de 42.
		\item<2-> d'afficher les entiers de 10 à 1 (dans cet ordre).
		\item<3-> de calculer la somme des entiers impairs de 1 à 999.
		\item<4-> de déterminer le plus petit entier $n$ tel que $1 + \dfrac{1}{2} + \dfrac{1}{3} + \dots +\dfrac{1}{n} > 7$.
	\end{enumerate}
	\end{exampleblock}
\end{frame}


\makess{Tableaux à une dimension, chaines de caractères}
\begin{frame}[fragile]{\Ctitle}{\stitle}
	\begin{block}{Tableaux}
		\textcolor{BrickRed}{\small \important} \textit{\footnotesize La notion de tableau en C est intimement liée à celle de \textit{pointeur}. Ces derniers seront abordés plus tard, aussi on fait ici une présentation élémentaire des tableaux, en particulier, on s'interdit pour le moment d'écrire des fonctions qui renvoient un tableau.}
		\begin{itemize}
			\item<1-> Un tableau se déclare en donnant sa longueur et le type de ses éléments. \\
				\onslide<2->{\mintinline[fontsize=\small]{c}{bool est_premier[1000]; //un tableau de 1000 booléens}}
			\item<3-> On peut initialiser le tableau en donnant une liste de valeurs entre accolades. \\
				\onslide<4->{\mintinline[fontsize=\small]{c}{double notes[4]={5.5, 12.0, 13.5, 7.0}; //un tableau de 4 flottants}}
			\item<5-> Les éléments sont numérotés à partir de 0
			\item<6-> On accède à un élément en donnant son numéro (son indice) entre crochet.\\
				\onslide<7->{\mintinline[fontsize=\small]{c}{est_premier[0]; //Le premier élément du tableau est_premier}}
			\item<7-> \textcolor{BrickRed}{\small \danger\;} Un accès en dehors des bornes du tableau est un \textcolor{blue}{comportement indéfini}\\
			\item<8-> On ne peut pas accéder à la taille d'un tableau en C. En conséquence, une fonction qui travaille sur un tableau doit aussi recevoir en argument la taille de ce dernier. 
		\end{itemize}
	\end{block}
\end{frame}


\begin{frame}[fragile]{\Ctitle}{\stitle}
	\begin{exampleblock}{Exemple}
		\onslide<1-> Ecrire une fonction \kw{croissant} qui prend un argument un tableau et sa taille et renvoie \kw{true} si le tableau est trié et false sinon.
		\onslide<2->{\inputpartC{\SPATH/cours_ex7.c}{}{}{4}{10}}
	\end{exampleblock}
\end{frame}

\begin{frame}[fragile]{\Ctitle}{\stitle}
	\begin{exampleblock}{Exemple}
		\onslide<1-> Ecrire une fonction \kw{echange} qui prend un argument un tableau et deux indices  \kw{i} et \kw{j} ne renvoie rien et échange les éléments d'indice \kw{i} et \kw{j} de ce tableau.\\
		\onslide<2->{\inputpartC{\SPATH/cours_ex9.c}{}{}{4}{9}}
		\onslide<3->\textcolor{gray}{Mais .... en C, les paramètres sont passés par valeur non ?}
	\end{exampleblock}
\end{frame}

\begin{frame}[fragile]{\Ctitle}{\stitle}
	\begin{exampleblock}{Exercices}
		\begin{enumerate}
		\item<1-> Déclarer un tableau {\tt tab} de 20 entiers.
		\item<2-> Déclarer un tableau {\tt test} de 5 booléens initialisées aux valeurs {\tt false, true, false, true, false}.
		\item<3-> Déclarer un tableau de 100 entiers, écrire une boucle permettant d'initialiser {\tt t[i]} à la valeur {\tt 2*i}.
		\item<4-> Commenter le programme suivant :
		\inputpartC{\SPATH/ub_tab.c}{}{\small}{3}{8}
		\end{enumerate}
	\end{exampleblock}
\end{frame}

\begin{frame}[fragile]{\Ctitle}{\stitle}
	\begin{block}{Chaines de caractères}
		\begin{itemize}
			\item<1-> En C, les chaines de caractères (notées entre guillemets \kw{"}) sont des tableaux de caractères (type \kw{char[]}) dont le dernier élément est le caractère spécial \kw{'\textbackslash{}0'} qui marque la fin de la chaine.
			\item<2-> Par exemple \mintinline{c}{char exemple[] = "Hello !";} crée le tableau :\\
				\begin{tabular}{|>{\tt}c|>{\tt}c|>{\tt}c|>{\tt}c|>{\tt}c|>{\tt}c|>{\tt}c|>{\tt}c|}
					\hline
					H & e & l & l & o & \; & ! & \textbackslash{}0 \\
					\hline
				\end{tabular}
			\item<3-> Le module \kw{string.h} fournit des fonctions usuelles de manipulation de caractères, notamment :
				\begin{itemize}
					\item<4-> \kw{strlen} : renvoie la longueur de la chaine de caractères
					\item<5-> \kw{strcpy} : copie une chaine de caractères
					\item<6-> \kw{strcat} : concaténation de chaines de caractères
				\end{itemize}
			\item<4-> Contrairement à un tableau \og{} classique \fg{}, on peut donc connaitre la longueur d'une chaine de caractères, grâce à la présence du caractère sentinelle \kw{\textbackslash{0}} qui en indique la fin.
		\end{itemize}
	\end{block}
\end{frame}

\begin{frame}[fragile]{\Ctitle}{\stitle}
	\begin{exampleblock}{Exemple}
		\begin{enumerate}
		\item<1->Quel est l'affichage produit par le programme suivant ?
		\onslide<2->{\inputC{\SPATH/cours_ex10.c}{}{}}
		\item<2->Quel est l'affichage produit en remplaçant la ligne 8 par \mintinline{c}{test[3]='\0';} ?
	\end{enumerate}
	\end{exampleblock}
\end{frame}

\makess{Saisie de valeurs au clavier}
\begin{frame}[fragile]{\Ctitle}{\stitle}
	\begin{block}{Fonction {\tt scanf}}
		\begin{itemize}
			\item<1-> La fonction \kw{scanf} permet la saisie de valeurs des variables depuis le clavier.
			\item<2-> Elle prend en argument un \textcolor{blue}{spécificateur de format} (comme \kw{printf}) qui permet de préciser le type de la variable attendue.
			\item<3-> On fera précéder la variable qui reçoit la valeur saisie au clavier du caractère \kw{\&} \\
				\onslide<4-> Ce point sera expliqué plus loin dans le cours.
			\item<5-> Cette fonction renvoie le nombre de valeurs correctement lues.
		\end{itemize}
	\end{block}
	\begin{exampleblock}{Exemple}
		\begin{itemize}
			\item<6-> Ecrire un programme qui demande à l'utilisateur de saisir au clavier deux entiers {\tt a} et {\tt b} puis affiche leur somme.
			\item<7-> Modifier ce programme pour que les valeurs saisies soient des flottants.
		\end{itemize}
	\end{exampleblock}
\end{frame}

\begin{frame}{\Ctitle}{\stitle}
	\begin{exampleblock}{Correction}
		\inputC{\SPATH/somme.c}{}{\small}
	\end{exampleblock}
\end{frame}

\begin{frame}{\Ctitle}{\stitle}
	\begin{exampleblock}{Correction}
		\inputC{\SPATH/sommef.c}{}{\small}
	\end{exampleblock}
\end{frame}

\end{document}

\PassOptionsToPackage{dvipsnames,table}{xcolor}
\documentclass[10pt]{beamer}
\usepackage{Cours}

\begin{document}


\newcounter{numchap}
\setcounter{numchap}{1}
\newcounter{numframe}
\setcounter{numframe}{0}
\newcommand{\mframe}[1]{\frametitle{#1} \addtocounter{numframe}{1}}
\newcommand{\cnum}{\fbox{\textcolor{yellow}{\textbf{C\thenumchap}}}~}
\newcommand{\makess}[1]{\section{#1} \label{ss\thesection}}
\newcommand{\stitle}{\textcolor{yellow}{\textbf{\thesection. \nameref{ss\thesection}}}}

\definecolor{codebg}{gray}{0.90}
\definecolor{grispale}{gray}{0.95}
\definecolor{fluo}{rgb}{1,0.96,0.62}
\newminted[langageC]{c}{linenos=true,escapeinside=||,highlightcolor=fluo,tabsize=2,breaklines=true}
\newminted[codepython]{python}{linenos=true,escapeinside=||,highlightcolor=fluo,tabsize=2,breaklines=true}
% Inclusion complète (ou partiel en indiquant premiere et dernière ligne) d'un fichier C
\newcommand{\inputC}[3]{\begin{mdframed}[backgroundcolor=codebg] \inputminted[breaklines=true,fontsize=#3,linenos=true,highlightcolor=fluo,tabsize=2,highlightlines={#2}]{c}{#1} \end{mdframed}}
\newcommand{\inputpartC}[5]{\begin{mdframed}[backgroundcolor=codebg] \inputminted[breaklines=true,fontsize=#3,linenos=true,highlightcolor=fluo,tabsize=2,highlightlines={#2},firstline=#4,lastline=#5,firstnumber=1]{c}{#1} \end{mdframed}}
\newcommand{\inputpython}[3]{\begin{mdframed}[backgroundcolor=codebg] \inputminted[breaklines=true,fontsize=#3,linenos=true,highlightcolor=fluo,tabsize=2,highlightlines={#2}]{python}{#1} \end{mdframed}}
\newcommand{\inputpartOCaml}[5]{\begin{mdframed}[backgroundcolor=codebg] \inputminted[breaklines=true,fontsize=#3,linenos=true,highlightcolor=fluo,tabsize=2,highlightlines={#2},firstline=#4,lastline=#5,firstnumber=1]{OCaml}{#1} \end{mdframed}}
\BeforeBeginEnvironment{minted}{\begin{mdframed}[backgroundcolor=codebg]}
\AfterEndEnvironment{minted}{\end{mdframed}}
\newcommand{\kw}[1]{\textcolor{blue}{\tt #1}}

\newtcolorbox{rcadre}[4]{halign=center,colback={#1},colframe={#2},width={#3cm},height={#4cm},valign=center,boxrule=1pt,left=0pt,right=0pt}
\newtcolorbox{cadre}[4]{halign=center,colback={#1},colframe={#2},arc=0mm,width={#3cm},height={#4cm},valign=center,boxrule=1pt,left=0pt,right=0pt}
\newcommand{\myem}[1]{\colorbox{fluo}{#1}}
\mdfsetup{skipabove=1pt,skipbelow=-2pt}



% Noeud dans un cadre pour les arbres
\newcommand{\noeud}[2]{\Tr{\fbox{\textcolor{#1}{\tt #2}}}}

\newcommand{\htmlmode}{\lstset{language=html,numbers=left, tabsize=4, frame=single, breaklines=true, keywordstyle=\ttfamily, basicstyle=\small,
   numberstyle=\tiny\ttfamily, framexleftmargin=0mm, backgroundcolor=\color{grispale}, xleftmargin=12mm,showstringspaces=false}}
\newcommand{\pythonmode}{\lstset{
   language=python,
   linewidth=\linewidth,
   numbers=left,
   tabsize=4,
   frame=single,
   breaklines=true,
   keywordstyle=\ttfamily\color{blue},
   basicstyle=\small,
   numberstyle=\tiny\ttfamily,
   framexleftmargin=-2mm,
   numbersep=-0.5mm,
   backgroundcolor=\color{codebg},
   xleftmargin=-1mm, 
   showstringspaces=false,
   commentstyle=\color{gray},
   stringstyle=\color{OliveGreen},
   emph={turtle,Screen,Turtle},
   emphstyle=\color{RawSienna},
   morekeywords={setheading,goto,backward,forward,left,right,pendown,penup,pensize,color,speed,hideturtle,showturtle,forward}}
   }
   \newcommand{\Cmode}{\lstset{
      language=[ANSI]C,
      linewidth=\linewidth,
      numbers=left,
      tabsize=4,
      frame=single,
      breaklines=true,
      keywordstyle=\ttfamily\color{blue},
      basicstyle=\small,
      numberstyle=\tiny\ttfamily,
      framexleftmargin=0mm,
      numbersep=2mm,
      backgroundcolor=\color{codebg},
      xleftmargin=0mm, 
      showstringspaces=false,
      commentstyle=\color{gray},
      stringstyle=\color{OliveGreen},
      emphstyle=\color{RawSienna},
      escapechar=\|,
      morekeywords={}}
      }
\newcommand{\bashmode}{\lstset{language=bash,numbers=left, tabsize=2, frame=single, breaklines=true, basicstyle=\ttfamily,
   numberstyle=\tiny\ttfamily, framexleftmargin=0mm, backgroundcolor=\color{grispale}, xleftmargin=12mm, showstringspaces=false}}
\newcommand{\exomode}{\lstset{language=python,numbers=left, tabsize=2, frame=single, breaklines=true, basicstyle=\ttfamily,
   numberstyle=\tiny\ttfamily, framexleftmargin=13mm, xleftmargin=12mm, basicstyle=\small, showstringspaces=false}}
   
   
  
%tei pour placer les images
%tei{nom de l’image}{échelle de l’image}{sens}{texte a positionner}
%sens ="1" (droite) ou "2" (gauche)
\newlength{\ltxt}
\newcommand{\tei}[4]{
\setlength{\ltxt}{\linewidth}
\setbox0=\hbox{\includegraphics[scale=#2]{#1}}
\addtolength{\ltxt}{-\wd0}
\addtolength{\ltxt}{-10pt}
\ifthenelse{\equal{#3}{1}}{
\begin{minipage}{\wd0}
\includegraphics[scale=#2]{#1}
\end{minipage}
\hfill
\begin{minipage}{\ltxt}
#4
\end{minipage}
}{
\begin{minipage}{\ltxt}
#4
\end{minipage}
\hfill
\begin{minipage}{\wd0}
\includegraphics[scale=#2]{#1}
\end{minipage}
}
}

%Juxtaposition d'une image pspciture et de texte 
%#1: = code pstricks de l'image
%#2: largeur de l'image
%#3: hauteur de l'image
%#4: Texte à écrire
\newcommand{\ptp}[4]{
\setlength{\ltxt}{\linewidth}
\addtolength{\ltxt}{-#2 cm}
\addtolength{\ltxt}{-0.1 cm}
\begin{minipage}[b][#3 cm][t]{\ltxt}
#4
\end{minipage}\hfill
\begin{minipage}[b][#3 cm][c]{#2 cm}
#1
\end{minipage}\par
}



%Macros pour les graphiques
\psset{linewidth=0.5\pslinewidth,PointSymbol=x}
\setlength{\fboxrule}{0.5pt}
\newcounter{tempangle}

%Marque la longueur du segment d'extrémité  #1 et  #2 avec la valeur #3, #4 est la distance par rapport au segment (en %age de la valeur de celui ci) et #5 l'orientation du marquage : +90 ou -90
\newcommand{\afflong}[5]{
\pstRotation[RotAngle=#4,PointSymbol=none,PointName=none]{#1}{#2}[X] 
\pstHomO[PointSymbol=none,PointName=none,HomCoef=#5]{#1}{X}[Y]
\pstTranslation[PointSymbol=none,PointName=none]{#1}{#2}{Y}[Z]
 \ncline{|<->|,linewidth=0.25\pslinewidth}{Y}{Z} \ncput*[nrot=:U]{\footnotesize{#3}}
}
\newcommand{\afflongb}[3]{
\ncline{|<->|,linewidth=0}{#1}{#2} \naput*[nrot=:U]{\footnotesize{#3}}
}

%Construis le point #4 situé à #2 cm du point #1 avant un angle #3 par rapport à l'horizontale. #5 = liste de paramètre
\newcommand{\lsegment}[5]{\pstGeonode[PointSymbol=none,PointName=none](0,0){O'}(#2,0){I'} \pstTranslation[PointSymbol=none,PointName=none]{O'}{I'}{#1}[J'] \pstRotation[RotAngle=#3,PointSymbol=x,#5]{#1}{J'}[#4]}
\newcommand{\tsegment}[5]{\pstGeonode[PointSymbol=none,PointName=none](0,0){O'}(#2,0){I'} \pstTranslation[PointSymbol=none,PointName=none]{O'}{I'}{#1}[J'] \pstRotation[RotAngle=#3,PointSymbol=x,#5]{#1}{J'}[#4] \pstLineAB{#4}{#1}}

%Construis le point #4 situé à #3 cm du point #1 et faisant un angle de  90° avec la droite (#1,#2) #5 = liste de paramètre
\newcommand{\psegment}[5]{
\pstGeonode[PointSymbol=none,PointName=none](0,0){O'}(#3,0){I'}
 \pstTranslation[PointSymbol=none,PointName=none]{O'}{I'}{#1}[J']
 \pstInterLC[PointSymbol=none,PointName=none]{#1}{#2}{#1}{J'}{M1}{M2} \pstRotation[RotAngle=-90,PointSymbol=x,#5]{#1}{M1}[#4]
  }
  
%Construis le point #4 situé à #3 cm du point #1 et faisant un angle de  #5° avec la droite (#1,#2) #6 = liste de paramètre
\newcommand{\mlogo}[6]{
\pstGeonode[PointSymbol=none,PointName=none](0,0){O'}(#3,0){I'}
 \pstTranslation[PointSymbol=none,PointName=none]{O'}{I'}{#1}[J']
 \pstInterLC[PointSymbol=none,PointName=none]{#1}{#2}{#1}{J'}{M1}{M2} \pstRotation[RotAngle=#5,PointSymbol=x,#6]{#1}{M2}[#4]
  }

% Construis un triangle avec #1=liste des 3 sommets séparés par des virgules, #2=liste des 3 longueurs séparés par des virgules, #3 et #4 : paramètre d'affichage des 2e et 3 points et #5 : inclinaison par rapport à l'horizontale
%autre macro identique mais sans tracer les segments joignant les sommets
\noexpandarg
\newcommand{\Triangleccc}[5]{
\StrBefore{#1}{,}[\pointA]
\StrBetween[1,2]{#1}{,}{,}[\pointB]
\StrBehind[2]{#1}{,}[\pointC]
\StrBefore{#2}{,}[\coteA]
\StrBetween[1,2]{#2}{,}{,}[\coteB]
\StrBehind[2]{#2}{,}[\coteC]
\tsegment{\pointA}{\coteA}{#5}{\pointB}{#3} 
\lsegment{\pointA}{\coteB}{0}{Z1}{PointSymbol=none, PointName=none}
\lsegment{\pointB}{\coteC}{0}{Z2}{PointSymbol=none, PointName=none}
\pstInterCC{\pointA}{Z1}{\pointB}{Z2}{\pointC}{Z3} 
\pstLineAB{\pointA}{\pointC} \pstLineAB{\pointB}{\pointC}
\pstSymO[PointName=\pointC,#4]{C}{C}[C]
}
\noexpandarg
\newcommand{\TrianglecccP}[5]{
\StrBefore{#1}{,}[\pointA]
\StrBetween[1,2]{#1}{,}{,}[\pointB]
\StrBehind[2]{#1}{,}[\pointC]
\StrBefore{#2}{,}[\coteA]
\StrBetween[1,2]{#2}{,}{,}[\coteB]
\StrBehind[2]{#2}{,}[\coteC]
\tsegment{\pointA}{\coteA}{#5}{\pointB}{#3} 
\lsegment{\pointA}{\coteB}{0}{Z1}{PointSymbol=none, PointName=none}
\lsegment{\pointB}{\coteC}{0}{Z2}{PointSymbol=none, PointName=none}
\pstInterCC[PointNameB=none,PointSymbolB=none,#4]{\pointA}{Z1}{\pointB}{Z2}{\pointC}{Z1} 
}


% Construis un triangle avec #1=liste des 3 sommets séparés par des virgules, #2=liste formée de 2 longueurs et d'un angle séparés par des virgules, #3 et #4 : paramètre d'affichage des 2e et 3 points et #5 : inclinaison par rapport à l'horizontale
%autre macro identique mais sans tracer les segments joignant les sommets
\newcommand{\Trianglecca}[5]{
\StrBefore{#1}{,}[\pointA]
\StrBetween[1,2]{#1}{,}{,}[\pointB]
\StrBehind[2]{#1}{,}[\pointC]
\StrBefore{#2}{,}[\coteA]
\StrBetween[1,2]{#2}{,}{,}[\coteB]
\StrBehind[2]{#2}{,}[\angleA]
\tsegment{\pointA}{\coteA}{#5}{\pointB}{#3} 
\setcounter{tempangle}{#5}
\addtocounter{tempangle}{\angleA}
\tsegment{\pointA}{\coteB}{\thetempangle}{\pointC}{#4}
\pstLineAB{\pointB}{\pointC}
}
\newcommand{\TriangleccaP}[5]{
\StrBefore{#1}{,}[\pointA]
\StrBetween[1,2]{#1}{,}{,}[\pointB]
\StrBehind[2]{#1}{,}[\pointC]
\StrBefore{#2}{,}[\coteA]
\StrBetween[1,2]{#2}{,}{,}[\coteB]
\StrBehind[2]{#2}{,}[\angleA]
\lsegment{\pointA}{\coteA}{#5}{\pointB}{#3} 
\setcounter{tempangle}{#5}
\addtocounter{tempangle}{\angleA}
\lsegment{\pointA}{\coteB}{\thetempangle}{\pointC}{#4}
}

% Construis un triangle avec #1=liste des 3 sommets séparés par des virgules, #2=liste formée de 1 longueurs et de deux angle séparés par des virgules, #3 et #4 : paramètre d'affichage des 2e et 3 points et #5 : inclinaison par rapport à l'horizontale
%autre macro identique mais sans tracer les segments joignant les sommets
\newcommand{\Trianglecaa}[5]{
\StrBefore{#1}{,}[\pointA]
\StrBetween[1,2]{#1}{,}{,}[\pointB]
\StrBehind[2]{#1}{,}[\pointC]
\StrBefore{#2}{,}[\coteA]
\StrBetween[1,2]{#2}{,}{,}[\angleA]
\StrBehind[2]{#2}{,}[\angleB]
\tsegment{\pointA}{\coteA}{#5}{\pointB}{#3} 
\setcounter{tempangle}{#5}
\addtocounter{tempangle}{\angleA}
\lsegment{\pointA}{1}{\thetempangle}{Z1}{PointSymbol=none, PointName=none}
\setcounter{tempangle}{#5}
\addtocounter{tempangle}{180}
\addtocounter{tempangle}{-\angleB}
\lsegment{\pointB}{1}{\thetempangle}{Z2}{PointSymbol=none, PointName=none}
\pstInterLL[#4]{\pointA}{Z1}{\pointB}{Z2}{\pointC}
\pstLineAB{\pointA}{\pointC}
\pstLineAB{\pointB}{\pointC}
}
\newcommand{\TrianglecaaP}[5]{
\StrBefore{#1}{,}[\pointA]
\StrBetween[1,2]{#1}{,}{,}[\pointB]
\StrBehind[2]{#1}{,}[\pointC]
\StrBefore{#2}{,}[\coteA]
\StrBetween[1,2]{#2}{,}{,}[\angleA]
\StrBehind[2]{#2}{,}[\angleB]
\lsegment{\pointA}{\coteA}{#5}{\pointB}{#3} 
\setcounter{tempangle}{#5}
\addtocounter{tempangle}{\angleA}
\lsegment{\pointA}{1}{\thetempangle}{Z1}{PointSymbol=none, PointName=none}
\setcounter{tempangle}{#5}
\addtocounter{tempangle}{180}
\addtocounter{tempangle}{-\angleB}
\lsegment{\pointB}{1}{\thetempangle}{Z2}{PointSymbol=none, PointName=none}
\pstInterLL[#4]{\pointA}{Z1}{\pointB}{Z2}{\pointC}
}

%Construction d'un cercle de centre #1 et de rayon #2 (en cm)
\newcommand{\Cercle}[2]{
\lsegment{#1}{#2}{0}{Z1}{PointSymbol=none, PointName=none}
\pstCircleOA{#1}{Z1}
}

%construction d'un parallélogramme #1 = liste des sommets, #2 = liste contenant les longueurs de 2 côtés consécutifs et leurs angles;  #3, #4 et #5 : paramètre d'affichage des sommets #6 inclinaison par rapport à l'horizontale 
% meme macro sans le tracé des segements
\newcommand{\Para}[6]{
\StrBefore{#1}{,}[\pointA]
\StrBetween[1,2]{#1}{,}{,}[\pointB]
\StrBetween[2,3]{#1}{,}{,}[\pointC]
\StrBehind[3]{#1}{,}[\pointD]
\StrBefore{#2}{,}[\longueur]
\StrBetween[1,2]{#2}{,}{,}[\largeur]
\StrBehind[2]{#2}{,}[\angle]
\tsegment{\pointA}{\longueur}{#6}{\pointB}{#3} 
\setcounter{tempangle}{#6}
\addtocounter{tempangle}{\angle}
\tsegment{\pointA}{\largeur}{\thetempangle}{\pointD}{#5}
\pstMiddleAB[PointName=none,PointSymbol=none]{\pointB}{\pointD}{Z1}
\pstSymO[#4]{Z1}{\pointA}[\pointC]
\pstLineAB{\pointB}{\pointC}
\pstLineAB{\pointC}{\pointD}
}
\newcommand{\ParaP}[6]{
\StrBefore{#1}{,}[\pointA]
\StrBetween[1,2]{#1}{,}{,}[\pointB]
\StrBetween[2,3]{#1}{,}{,}[\pointC]
\StrBehind[3]{#1}{,}[\pointD]
\StrBefore{#2}{,}[\longueur]
\StrBetween[1,2]{#2}{,}{,}[\largeur]
\StrBehind[2]{#2}{,}[\angle]
\lsegment{\pointA}{\longueur}{#6}{\pointB}{#3} 
\setcounter{tempangle}{#6}
\addtocounter{tempangle}{\angle}
\lsegment{\pointA}{\largeur}{\thetempangle}{\pointD}{#5}
\pstMiddleAB[PointName=none,PointSymbol=none]{\pointB}{\pointD}{Z1}
\pstSymO[#4]{Z1}{\pointA}[\pointC]
}


%construction d'un cerf-volant #1 = liste des sommets, #2 = liste contenant les longueurs de 2 côtés consécutifs et leurs angles;  #3, #4 et #5 : paramètre d'affichage des sommets #6 inclinaison par rapport à l'horizontale 
% meme macro sans le tracé des segements
\newcommand{\CerfVolant}[6]{
\StrBefore{#1}{,}[\pointA]
\StrBetween[1,2]{#1}{,}{,}[\pointB]
\StrBetween[2,3]{#1}{,}{,}[\pointC]
\StrBehind[3]{#1}{,}[\pointD]
\StrBefore{#2}{,}[\longueur]
\StrBetween[1,2]{#2}{,}{,}[\largeur]
\StrBehind[2]{#2}{,}[\angle]
\tsegment{\pointA}{\longueur}{#6}{\pointB}{#3} 
\setcounter{tempangle}{#6}
\addtocounter{tempangle}{\angle}
\tsegment{\pointA}{\largeur}{\thetempangle}{\pointD}{#5}
\pstOrtSym[#4]{\pointB}{\pointD}{\pointA}[\pointC]
\pstLineAB{\pointB}{\pointC}
\pstLineAB{\pointC}{\pointD}
}

%construction d'un quadrilatère quelconque #1 = liste des sommets, #2 = liste contenant les longueurs des 4 côtés et l'angle entre 2 cotés consécutifs  #3, #4 et #5 : paramètre d'affichage des sommets #6 inclinaison par rapport à l'horizontale 
% meme macro sans le tracé des segements
\newcommand{\Quadri}[6]{
\StrBefore{#1}{,}[\pointA]
\StrBetween[1,2]{#1}{,}{,}[\pointB]
\StrBetween[2,3]{#1}{,}{,}[\pointC]
\StrBehind[3]{#1}{,}[\pointD]
\StrBefore{#2}{,}[\coteA]
\StrBetween[1,2]{#2}{,}{,}[\coteB]
\StrBetween[2,3]{#2}{,}{,}[\coteC]
\StrBetween[3,4]{#2}{,}{,}[\coteD]
\StrBehind[4]{#2}{,}[\angle]
\tsegment{\pointA}{\coteA}{#6}{\pointB}{#3} 
\setcounter{tempangle}{#6}
\addtocounter{tempangle}{\angle}
\tsegment{\pointA}{\coteD}{\thetempangle}{\pointD}{#5}
\lsegment{\pointB}{\coteB}{0}{Z1}{PointSymbol=none, PointName=none}
\lsegment{\pointD}{\coteC}{0}{Z2}{PointSymbol=none, PointName=none}
\pstInterCC[PointNameA=none,PointSymbolA=none,#4]{\pointB}{Z1}{\pointD}{Z2}{Z3}{\pointC} 
\pstLineAB{\pointB}{\pointC}
\pstLineAB{\pointC}{\pointD}
}


% Définition des colonnes centrées ou à droite pour tabularx
\newcolumntype{Y}{>{\centering\arraybackslash}X}
\newcolumntype{Z}{>{\flushright\arraybackslash}X}

%Les pointillés à remplir par les élèves
\newcommand{\po}[1]{\makebox[#1 cm]{\dotfill}}
\newcommand{\lpo}[1][3]{%
\multido{}{#1}{\makebox[\linewidth]{\dotfill}
}}

%Liste des pictogrammes utilisés sur la fiche d'exercice ou d'activités
\newcommand{\bombe}{\faBomb}
\newcommand{\livre}{\faBook}
\newcommand{\calculatrice}{\faCalculator}
\newcommand{\oral}{\faCommentO}
\newcommand{\surfeuille}{\faEdit}
\newcommand{\ordinateur}{\faLaptop}
\newcommand{\ordi}{\faDesktop}
\newcommand{\ciseaux}{\faScissors}
\newcommand{\danger}{\faExclamationTriangle}
\newcommand{\out}{\faSignOut}
\newcommand{\cadeau}{\faGift}
\newcommand{\flash}{\faBolt}
\newcommand{\lumiere}{\faLightbulb}
\newcommand{\compas}{\dsmathematical}
\newcommand{\calcullitteral}{\faTimesCircleO}
\newcommand{\raisonnement}{\faCogs}
\newcommand{\recherche}{\faSearch}
\newcommand{\rappel}{\faHistory}
\newcommand{\video}{\faFilm}
\newcommand{\capacite}{\faPuzzlePiece}
\newcommand{\aide}{\faLifeRing}
\newcommand{\loin}{\faExternalLink}
\newcommand{\groupe}{\faUsers}
\newcommand{\bac}{\faGraduationCap}
\newcommand{\histoire}{\faUniversity}
\newcommand{\coeur}{\faSave}
\newcommand{\python}{\faPython}
\newcommand{\os}{\faMicrochip}
\newcommand{\rd}{\faCubes}
\newcommand{\data}{\faColumns}
\newcommand{\web}{\faCode}
\newcommand{\prog}{\faFile}
\newcommand{\algo}{\faCogs}
\newcommand{\important}{\faExclamationCircle}
\newcommand{\maths}{\faTimesCircle}
% Traitement des données en tables
\newcommand{\tables}{\faColumns}
% Types construits
\newcommand{\construits}{\faCubes}
% Type et valeurs de base
\newcommand{\debase}{{\footnotesize \faCube}}
% Systèmes d'exploitation
\newcommand{\linux}{\faLinux}
\newcommand{\sd}{\faProjectDiagram}
\newcommand{\bd}{\faDatabase}

%Les ensembles de nombres
\renewcommand{\N}{\mathbb{N}}
\newcommand{\D}{\mathbb{D}}
\newcommand{\Z}{\mathbb{Z}}
\newcommand{\Q}{\mathbb{Q}}
\newcommand{\R}{\mathbb{R}}
\newcommand{\C}{\mathbb{C}}

%Ecriture des vecteurs
\newcommand{\vect}[1]{\vbox{\halign{##\cr 
  \tiny\rightarrowfill\cr\noalign{\nointerlineskip\vskip1pt} 
  $#1\mskip2mu$\cr}}}


%Compteur activités/exos et question et mise en forme titre et questions
\newcounter{numact}
\setcounter{numact}{1}
\newcounter{numseance}
\setcounter{numseance}{1}
\newcounter{numexo}
\setcounter{numexo}{0}
\newcounter{numprojet}
\setcounter{numprojet}{0}
\newcounter{numquestion}
\newcommand{\espace}[1]{\rule[-1ex]{0pt}{#1 cm}}
\newcommand{\Quest}[3]{
\addtocounter{numquestion}{1}
\begin{tabularx}{\textwidth}{X|m{1cm}|}
\cline{2-2}
\textbf{\sffamily{\alph{numquestion})}} #1 & \dots / #2 \\
\hline 
\multicolumn{2}{|l|}{\espace{#3}} \\
\hline
\end{tabularx}
}
\newcommand{\QuestR}[3]{
\addtocounter{numquestion}{1}
\begin{tabularx}{\textwidth}{X|m{1cm}|}
\cline{2-2}
\textbf{\sffamily{\alph{numquestion})}} #1 & \dots / #2 \\
\hline 
\multicolumn{2}{|l|}{\cor{#3}} \\
\hline
\end{tabularx}
}
\newcommand{\Pre}{{\sc nsi} 1\textsuperscript{e}}
\newcommand{\Term}{{\sc nsi} Terminale}
\newcommand{\Sec}{2\textsuperscript{e}}
\newcommand{\Exo}[2]{ \addtocounter{numexo}{1} \ding{113} \textbf{\sffamily{Exercice \thenumexo}} : \textit{#1} \hfill #2  \setcounter{numquestion}{0}}
\newcommand{\Projet}[1]{ \addtocounter{numprojet}{1} \ding{118} \textbf{\sffamily{Projet \thenumprojet}} : \textit{#1}}
\newcommand{\ExoD}[2]{ \addtocounter{numexo}{1} \ding{113} \textbf{\sffamily{Exercice \thenumexo}}  \textit{(#1 pts)} \hfill #2  \setcounter{numquestion}{0}}
\newcommand{\ExoB}[2]{ \addtocounter{numexo}{1} \ding{113} \textbf{\sffamily{Exercice \thenumexo}}  \textit{(Bonus de +#1 pts maximum)} \hfill #2  \setcounter{numquestion}{0}}
\newcommand{\Act}[2]{ \ding{113} \textbf{\sffamily{Activité \thenumact}} : \textit{#1} \hfill #2  \addtocounter{numact}{1} \setcounter{numquestion}{0}}
\newcommand{\Seance}{ \rule{1.5cm}{0.5pt}\raisebox{-3pt}{\framebox[4cm]{\textbf{\sffamily{Séance \thenumseance}}}}\hrulefill  \\
  \addtocounter{numseance}{1}}
\newcommand{\Acti}[2]{ {\footnotesize \ding{117}} \textbf{\sffamily{Activité \thenumact}} : \textit{#1} \hfill #2  \addtocounter{numact}{1} \setcounter{numquestion}{0}}
\newcommand{\titre}[1]{\begin{Large}\textbf{\ding{118}}\end{Large} \begin{large}\textbf{ #1}\end{large} \vspace{0.2cm}}
\newcommand{\QListe}[1][0]{
\ifthenelse{#1=0}
{\begin{enumerate}[partopsep=0pt,topsep=0pt,parsep=0pt,itemsep=0pt,label=\textbf{\sffamily{\arabic*.}},series=question]}
{\begin{enumerate}[resume*=question]}}
\newcommand{\SQListe}[1][0]{
\ifthenelse{#1=0}
{\begin{enumerate}[partopsep=0pt,topsep=0pt,parsep=0pt,itemsep=0pt,label=\textbf{\sffamily{\alph*)}},series=squestion]}
{\begin{enumerate}[resume*=squestion]}}
\newcommand{\SQListeL}[1][0]{
\ifthenelse{#1=0}
{\begin{enumerate*}[partopsep=0pt,topsep=0pt,parsep=0pt,itemsep=0pt,label=\textbf{\sffamily{\alph*)}},series=squestion]}
{\begin{enumerate*}[resume*=squestion]}}
\newcommand{\FinListe}{\end{enumerate}}
\newcommand{\FinListeL}{\end{enumerate*}}

%Mise en forme de la correction
\newcommand{\cor}[1]{\par \textcolor{OliveGreen}{#1}}
\newcommand{\br}[1]{\cor{\textbf{#1}}}
\newcommand{\tcor}[1]{\begin{tcolorbox}[width=0.92\textwidth,colback={white},colbacktitle=white,coltitle=OliveGreen,colframe=green!75!black,boxrule=0.2mm]   
\cor{#1}
\end{tcolorbox}
}
\newcommand{\rc}[1]{\textcolor{OliveGreen}{#1}}
\newcommand{\pmc}[1]{\textcolor{blue}{\tt #1}}
\newcommand{\tmc}[1]{\textcolor{RawSienna}{\tt #1}}


%Référence aux exercices par leur numéro
\newcommand{\refexo}[1]{
\refstepcounter{numexo}
\addtocounter{numexo}{-1}
\label{#1}}

%Séparation entre deux activités
\newcommand{\separateur}{\begin{center}
\rule{1.5cm}{0.5pt}\raisebox{-3pt}{\ding{117}}\rule{1.5cm}{0.5pt}  \vspace{0.2cm}
\end{center}}

%Entête et pied de page
\newcommand{\snt}[1]{\lhead{\textbf{SNT -- La photographie numérique} \rhead{\textit{Lycée Nord}}}}
\newcommand{\Activites}[2]{\lhead{\textbf{{\sc #1}}}
\rhead{Activités -- \textbf{#2}}
\cfoot{}}
\newcommand{\Exos}[2]{\lhead{\textbf{Fiche d'exercices: {\sc #1}}}
\rhead{Niveau: \textbf{#2}}
\cfoot{}}
\newcommand{\Devoir}[2]{\lhead{\textbf{Devoir de mathématiques : {\sc #1}}}
\rhead{\textbf{#2}} \setlength{\fboxsep}{8pt}
\begin{center}
%Titre de la fiche
\fbox{\parbox[b][1cm][t]{0.3\textwidth}{Nom : \hfill \po{3} \par \vfill Prénom : \hfill \po{3}} } \hfill 
\fbox{\parbox[b][1cm][t]{0.6\textwidth}{Note : \po{1} / 20} }
\end{center} \cfoot{}}

%Devoir programmation en NSI (pas à rendre sur papier)
\newcommand{\PNSI}[2]{\lhead{\textbf{Devoir de {\sc nsi} : \textsf{ #1}}}
\rhead{\textbf{#2}} \setlength{\fboxsep}{8pt}
\begin{tcolorbox}[title=\textcolor{black}{\danger\; A lire attentivement},colbacktitle=lightgray]
{\begin{enumerate}
\item Rendre tous vous programmes en les envoyant par mail à l'adresse {\tt fnativel2@ac-reunion.fr}, en précisant bien dans le sujet vos noms et prénoms
\item Un programme qui fonctionne mal ou pas du tout peut rapporter des points
\item Les bonnes pratiques de programmation (clarté et lisiblité du code) rapportent des points
\end{enumerate}
}
\end{tcolorbox}
 \cfoot{}}


%Devoir de NSI
\newcommand{\DNSI}[2]{\lhead{\textbf{Devoir de {\sc nsi} : \textsf{ #1}}}
\rhead{\textbf{#2}} \setlength{\fboxsep}{8pt}
\begin{center}
%Titre de la fiche
\fbox{\parbox[b][1cm][t]{0.3\textwidth}{Nom : \hfill \po{3} \par \vfill Prénom : \hfill \po{3}} } \hfill 
\fbox{\parbox[b][1cm][t]{0.6\textwidth}{Note : \po{1} / 10} }
\end{center} \cfoot{}}

\newcommand{\DevoirNSI}[2]{\lhead{\textbf{Devoir de {\sc nsi} : {\sc #1}}}
\rhead{\textbf{#2}} \setlength{\fboxsep}{8pt}
\cfoot{}}

%La définition de la commande QCM pour auto-multiple-choice
%En premier argument le sujet du qcm, deuxième argument : la classe, 3e : la durée prévue et #4 : présence ou non de questions avec plusieurs bonnes réponses
\newcommand{\QCM}[4]{
{\large \textbf{\ding{52} QCM : #1}} -- Durée : \textbf{#3 min} \hfill {\large Note : \dots/10} 
\hrule \vspace{0.1cm}\namefield{}
Nom :  \textbf{\textbf{\nom{}}} \qquad \qquad Prénom :  \textbf{\prenom{}}  \hfill Classe: \textbf{#2}
\vspace{0.2cm}
\hrule  
\begin{itemize}[itemsep=0pt]
\item[-] \textit{Une bonne réponse vaut un point, une absence de réponse n'enlève pas de point. }
\item[\danger] \textit{Une mauvaise réponse enlève un point.}
\ifthenelse{#4=1}{\item[-] \textit{Les questions marquées du symbole \multiSymbole{} peuvent avoir plusieurs bonnes réponses possibles.}}{}
\end{itemize}
}
\newcommand{\DevoirC}[2]{
\renewcommand{\footrulewidth}{0.5pt}
\lhead{\textbf{Devoir de mathématiques : {\sc #1}}}
\rhead{\textbf{#2}} \setlength{\fboxsep}{8pt}
\fbox{\parbox[b][0.4cm][t]{0.955\textwidth}{Nom : \po{5} \hfill Prénom : \po{5} \hfill Classe: \textbf{1}\textsuperscript{$\dots$}} } 
\rfoot{\thepage} \cfoot{} \lfoot{Lycée Nord}}
\newcommand{\DevoirInfo}[2]{\lhead{\textbf{Evaluation : {\sc #1}}}
\rhead{\textbf{#2}} \setlength{\fboxsep}{8pt}
 \cfoot{}}
\newcommand{\DM}[2]{\lhead{\textbf{Devoir maison à rendre le #1}} \rhead{\textbf{#2}}}

%Macros permettant l'affichage des touches de la calculatrice
%Touches classiques : #1 = 0 fond blanc pour les nombres et #1= 1gris pour les opérations et entrer, second paramètre=contenu
%Si #2=1 touche arrondi avec fond gris
\newcommand{\TCalc}[2]{
\setlength{\fboxsep}{0.1pt}
\ifthenelse{#1=0}
{\psframebox[fillstyle=solid, fillcolor=white]{\parbox[c][0.25cm][c]{0.6cm}{\centering #2}}}
{\ifthenelse{#1=1}
{\psframebox[fillstyle=solid, fillcolor=lightgray]{\parbox[c][0.25cm][c]{0.6cm}{\centering #2}}}
{\psframebox[framearc=.5,fillstyle=solid, fillcolor=white]{\parbox[c][0.25cm][c]{0.6cm}{\centering #2}}}
}}
\newcommand{\Talpha}{\psdblframebox[fillstyle=solid, fillcolor=white]{\hspace{-0.05cm}\parbox[c][0.25cm][c]{0.65cm}{\centering \scriptsize{alpha}}} \;}
\newcommand{\Tsec}{\psdblframebox[fillstyle=solid, fillcolor=white]{\parbox[c][0.25cm][c]{0.6cm}{\centering \scriptsize 2nde}} \;}
\newcommand{\Tfx}{\psdblframebox[fillstyle=solid, fillcolor=white]{\parbox[c][0.25cm][c]{0.6cm}{\centering \scriptsize $f(x)$}} \;}
\newcommand{\Tvar}{\psframebox[framearc=.5,fillstyle=solid, fillcolor=white]{\hspace{-0.22cm} \parbox[c][0.25cm][c]{0.82cm}{$\scriptscriptstyle{X,T,\theta,n}$}}}
\newcommand{\Tgraphe}{\psdblframebox[fillstyle=solid, fillcolor=white]{\hspace{-0.08cm}\parbox[c][0.25cm][c]{0.68cm}{\centering \tiny{graphe}}} \;}
\newcommand{\Tfen}{\psdblframebox[fillstyle=solid, fillcolor=white]{\hspace{-0.08cm}\parbox[c][0.25cm][c]{0.68cm}{\centering \tiny{fenêtre}}} \;}
\newcommand{\Ttrace}{\psdblframebox[fillstyle=solid, fillcolor=white]{\parbox[c][0.25cm][c]{0.6cm}{\centering \scriptsize{trace}}} \;}

% Macroi pour l'affichage  d'un entier n dans  une base b
\newcommand{\base}[2]{ \overline{#1}^{#2}}
% Intervalle d'entiers
\newcommand{\intN}[2]{\llbracket #1; #2 \rrbracket}}

% Numéro et titre de chapitre
\setcounter{numchap}{1}
\newcommand{\Ctitle}{\cnum {Introduction au langage C}}
\newcommand{\SPATH}{/home/fenarius/Travail/Cours/cpge-info/docs/mp2i/files/C\thenumchap/}

\makess{Historique}
\begin{frame}{\Ctitle}{\stitle}
	\begin{block}{Bref historique}
		\begin{itemize}
			\item<1-> 1972 : début du développement du langage C par Dennis Ritchie et Ken Thomson aux laboratoires Bell parallèlement à la création du système d'exploitation {\sc unix}.
			\item<2-> 1978 : première édition du livre "The C programming language" (Kernighan \& Ritchie)
			\item<3-> 1983 : première standardisation du langage par l'{\sc ansi} qui assure la compatibilité et la portabilité entre différentes plateformes. La dernière standardisation date de 2018 (C18)
			\item<4-> A partir de 1983 : développement de plusieurs dérivés de C, parmi lesquels C++ (B. Strousrtup, 1983), C\# (Microsoft, 2000), Go (Google, 2007), Rust (Mozilla, 2010)
		\end{itemize}
	\end{block}
\end{frame}

\makess{Caractéristiques du C}
\begin{frame}{\Ctitle}{\stitle}
	\begin{block}{Quelques aspects du C}
		\begin{itemize}
			\item<1-> Langage \textcolor{blue}{impératif} : séquence d'instructions exécutées par l'ordinateur pour modifier l'état du programme. C n'est ni orienté objet, ni fonctionnel.
			\item<2-> Les variables sont \textcolor{blue}{mutables} c'est à dire qu'elles peuvent changer de valeur pendant l'exécution.
			\item<3-> Le langage C est \textcolor{blue}{statiquement typé} c'est à dire qu'une variable appartient à un type défini à la déclaration et durant toute sa durée de vie.
			\item<4-> Equipé d'une librairie standard : la \textcolor{blue}{libc}.
			\item<5-> \textcolor{BrickRed}{\small \danger} Le standard précise un certain nombres de \textcolor{blue}{comportements indéfinis}, c'est à dire de programmes dont le résultat est imprévisible.
			\item<6-> Plus \textcolor{blue}{proche de la machine} que bien d'autres langages de haut niveau, ce qui induit une certaine efficacité.
			\item<7-> Souvent utilisé pour le développement de systèmes d'exploitation, de pilotes de périphériques, de logiciels embarqués.
		\end{itemize}
	\end{block}
\end{frame}

\begin{frame}{\Ctitle}{\stitle}
	\begin{block}{Compilation}
		Le langage C est \textcolor{blue}{compilé} : \\ \medskip
		\begin{tabular}{ccccc}
			\rnode{CS}{\begin{rcadre}{lightgray}{Sepia}{2.4}{1.4}
					           \textcolor{Sepia}{\small \faFile\; Code source \\
					           {\footnotesize (fichier(s) texte .c)}}
				           \end{rcadre}} & \hspace{0.8cm} & \onslide<3->{\rnode{{CO}}{\begin{cadre}{white}{black}{2.8}{1.4}  {\small \textbf{\faCog\;} Compilateur} \\ {\footnotesize (gcc, clang, ..)} \end{cadre}}} & \hspace{0.8cm} &
			\onslide<5->{\rnode{EX}{\begin{rcadre}{lightgray}{blue}{2.4}{1.4}
						                        \textcolor{blue}{\small \faFileArchive\; Exécutable \\ {\footnotesize (fichier binaire)}}
					                        \end{rcadre}}}                                                                                                             \\
		\end{tabular}
		\onslide<3->{\ncline[doubleline=true,doublesep=2pt,doublecolor=OliveGreen,linecolor=OliveGreen,linewidth=1pt,arrowsize=10pt,arrowinset=0.2,arrowlength=1.2]{->}{CS}{CO}}
		\onslide<6->{\ncline[doubleline=true,doublesep=2pt,doublecolor=OliveGreen,linecolor=OliveGreen,linewidth=1pt,arrowsize=10pt,arrowinset=0.2,arrowlength=1.2]{->}{CO}{EX}}
		\begin{enumerate}
			\item<2-> Les IDE comme VS Code signalent certaines erreurs dans le code.
			\item<4-> La compilation peut produire des erreurs ou des avertissement (\textit{warning}) \\
				\textcolor{gray}{\small La compilation se déroule en 4 étapes : préprocesseur, compilation, assemblage, editions de lien}
			\item<6-> Une compilation sans erreur (mais éventuellement des \textit{warning}) produit un exécutable.
			\item<7-> Les erreurs dans l'exécution ne feront pas référence aux instructions du code source.
		\end{enumerate}
	\end{block}
\end{frame}



\makess{Exemples de programmes}
\begin{frame}[fragile]{\Ctitle}{\stitle}
	\begin{exampleblock}{Programme minimal}
		\begin{overprint}
			\onslide<1>
			\begin{langageC}
				#include <stdio.h>

				int main()
				{
						printf("Hello world \n");
						return 0;
					}
			\end{langageC}
			\medskip
			Si le fichier texte s'appelle {\tt hello.c}, on lance la compilation avec {\tt gcc hello.c}, l'exécutable produit s'appelle par défaut {\tt a.out}, on peut modifier ce nom avec l'option {\tt -o}. Par exemple : {\tt gcc -o hello.exe hello.c}
			\onslide<2>
			\begin{langageC*}{highlightlines=1}
				#include <stdio.h>

				int main()
				{
						printf("Hello world \n");
						return 0;
					}
			\end{langageC*}
			\medskip
			Appel aux fonctions \textcolor{blue}{\textbf{st}}andar\textcolor{blue}{\textbf d} d'entrées et de sorties (\textcolor{blue}{\textbf i}nput et \textcolor{blue}{\textbf o}utput)  de la libc.
			\onslide<3>
			\begin{langageC}
				#include <stdio.h>

				int |\myem{main}|()
				{
						printf("Hello world \n");
						return 0;
					}
			\end{langageC}
			\medskip
			Un programme C contient une fonction \kw{ main} par laquelle l'exécution du programme commence.
			\onslide<4>
			\begin{langageC}
				#include <stdio.h>

				|\myem{int}| main|\myem{()}|
				{
				printf("Hello world \n");
				return 0;
				}
			\end{langageC}
			\medskip
			Avant le nom d'une fonction on trouve le type de variable qu'elle renvoie (ici \kw{ int}) et après entre parenthèses, les arguments éventuels de la fonction (ici aucun).
			\onslide<5>
			\begin{langageC}
				#include <stdio.h>

				int main()
				|\myem{\{}|
				printf("Hello world \n")|\myem{;}|
				return 0|\myem{;}|
				|\myem{\}}|
			\end{langageC}
			\medskip
			Les blocs d'instructions sont délimités par des accolades (\kw{ \{} et \kw{ \}}). Les instructions doivent se terminer par un point virgule \kw{ ;}. Les espaces, sauts de ligne et indentation sont ignorés par le compilateur, mais sont nécessaires pour une bonne lisibilité.
			\onslide<6>
			\begin{langageC*}{highlightlines=5}
				#include <stdio.h>

				int main()
				{
						printf("Hello world \n");
						return 0;
					}
			\end{langageC*}
			\medskip
			L'instruction \kw{ printf} permet d'afficher dans le terminal. On notera les guillemets (\textcolor{blue}{"}) pour délimiter une chaîne de caractères et le caractère \kw{ \textbackslash n} pour indiquer un retour à la ligne.
			\onslide<7>
			\begin{langageC*}{highlightlines=6}
				#include <stdio.h>

				int main()
				{
						printf("Hello world \n");
						return 0|\myem{;}|
					}
			\end{langageC*}
			\medskip
			L'instruction \kw{ return} quitte la fonction en renvoyant la valeur donnée. Ici, on renvoie {\tt 0}, qui indique traditionnellement que le programme se  termine sans erreurs.
		\end{overprint}
	\end{exampleblock}

\end{frame}

\begin{frame}{\Ctitle}{\stitle}
	\begin{exampleblock}{Exemple de boucle}
		\inputC{\SPATH/cours_ex2.c}{4}{\small}
		\medskip
		Déclaration de la variable {\tt somme} de type \kw{int} et initialisation à zéro. A noter qu'on peut déclarer une variable sans l'initialiser.
	\end{exampleblock}
\end{frame}

\begin{frame}{\Ctitle}{\stitle}
	\begin{exampleblock}{Exemple de boucle}
		\inputC{\SPATH/cours_ex2.c}{5}{\small}
		\medskip
		Une variable dont la valeur ne sera pas modifiée peut être déclaré avec \kw{const}.\\
		\onslide<2-> \textcolor{gray}{On peut aussi utiliser une directive de précompilation \\ {\tt \#define NMAX 100}}
	\end{exampleblock}
\end{frame}

\begin{frame}{\Ctitle}{\stitle}
	\begin{exampleblock}{Exemple de boucle}
		\inputC{\SPATH/cours_ex2.c}{6}{\small}
		\medskip
		On remarque que la boucle \kw{for} est de la forme {\tt for} \textit{(init; fin; incr)}. Les opérateurs de comparaison en C sont \kw{==}, \kw{!=}, \kw{<}, \kw{>}, \kw{<=} et \kw{>=}.
	\end{exampleblock}
\end{frame}

\begin{frame}{\Ctitle}{\stitle}
	\begin{exampleblock}{Exemple de boucle}
		\inputC{\SPATH/cours_ex2.c}{8}{\small}
		\medskip
		On veut afficher un \kw{int} dans la réponse, on utilise \kw{\%{}d} dans \kw{printf} à l'emplacement souhaité.
	\end{exampleblock}
\end{frame}

\begin{frame}{\Ctitle}{\stitle}
	\begin{exampleblock}{Exemple d'instruction conditionnelle}
		\inputC{\SPATH/cours_ex3.c}{2}{\small}
		\medskip
		Une ligne de commentaire commence avec \kw{//}, un commentaire multiligne est encadré par \kw{/*} et \kw{*/}
	\end{exampleblock}
\end{frame}

\begin{frame}{\Ctitle}{\stitle}
	\begin{exampleblock}{Exemple d'instruction conditionnelle}
		\inputC{\SPATH/cours_ex3.c}{3}{\small}
		\medskip
		Définition d'une fonction \kw{syracuse} qui prend comme paramètre un entier et renvoie un entier. C'est la \textcolor{blue}{signature} de la fonction. \\
		\textcolor{BrickRed}{\important} En C, les paramètres sont passés par \textcolor{blue}{valeur}.
	\end{exampleblock}
\end{frame}

\begin{frame}{\Ctitle}{\stitle}
	\begin{exampleblock}{Exemple d'instruction conditionnelle}
		\inputC{\SPATH/cours_ex3.c}{4-7}{\small}
		\medskip
		Instruction conditionnelle : on exécute le bloc qui suit la condition si celle-ci est vérifiée et sinon le bloc qui suit le \kw{else} (s'il est présent).
		Noter les parenthèses autour de la condition.
	\end{exampleblock}
\end{frame}

\makess{Types de base en C}
\begin{frame}{\Ctitle}{\stitle}
	\begin{alertblock}{Types de base}
		\begin{tabularx}{\linewidth}{|l|p{1.8cm}|>{\footnotesize}X|}
			\hline
			Type                                        & Opérations                                                                                                & Commentaires                                                                                                                                       \\
			\hline
			\leavevmode\onslide<2->{\kw{int} et \kw{unsigned int}}               & \leavevmode\onslide<3->{\kw{+}, \kw{-}, \kw{*}, \kw{/}, \kw{\%}  \newline} \leavevmode\onslide<4->{\textcolor{BrickRed}{\small \danger}\textcolor{gray}{{\tt ++},{\tt -{}-}}} & \leavevmode\onslide<5->{Entiers signés ou non signés codés sur un minimum de 16 bits.}                                                                                      \\
			\leavevmode\onslide<6->{\kw{int}\textcolor{Sepia}{$N$}\kw{\_t} et \kw{uint}\textcolor{Sepia}{$N$}\kw{\_t}} &                                                                                                           & \leavevmode\onslide<7->{Entiers codés sur \textcolor{Sepia}{$N$} bits accessibles dans \kw{stdint.h} ($\textcolor{Sepia}{N=8}$, \textcolor{Sepia}{$32$} ou \textcolor{Sepia}{$64$}).}                                                                        \\
			\hline
			\leavevmode\onslide<8->{\kw{float} et \kw{double}}                   & \leavevmode\onslide<8->{\kw{+}, \kw{-}, \kw{*}, \kw{/}}                                                                            & \leavevmode\onslide<8->{Représentation des nombres en virgules flottantes en simple ou double précision de la norme {\sc ieee754}. Fonctions élémentaires dans \kw{math.h}} \\
			\hline
			\leavevmode\onslide<9->{\kw{bool}, valeurs \kw{true} ou \kw{false}.}                                   & \leavevmode\onslide<9->{\kw{||},  \kw{\&\&}, \kw{!}}                                                                                & \leavevmode\onslide<9->{Booléens accessibles dans \kw{stdbool.h}. Evaluations paresseuses des expressions.}                                                                 \\
			\hline
			\leavevmode\onslide<10->{\kw{char}}                                   & \leavevmode\onslide<10->{\textcolor{gray}{\tt +, -}}                                                                                & \leavevmode\onslide<10->{Caractères noté entre quotes (\kw{'}), uniquement ceux de la table {\sc ascii}. Caractère nul : \kw{'\textbackslash{}0'}}                           \\
			\hline
		\end{tabularx}
		\onslide<11-> Pour indiquer l'absence de type, notamment pour les fonctions ne renvoyant rien (par exemple une fonction d'affichage) on utilise \kw{void}.
	\end{alertblock}
\end{frame}

\begin{frame}{\Ctitle}{\stitle}
	\begin{exampleblock}{Exemples}
		\begin{enumerate}
		\item<1-> Déclarer deux variables entières \kw{a} et \kw{b} initialisées respectivement à 2024 et 137. Déclarer \kw{c} ayant comme valeur le reste dans division euclidienne de \kw{a} par \kw{b}.
		\item<2-> Déclarer et initialiser \kw{d} ayant comme valeur $b^2 - 4ac$, en supposant que $a$, $b$ et $c$ sont des variables flottantes existantes et initialisées.
		\item<3-> On suppose déjà déclarées deux variables booléennes \kw{x} et \kw{y}, écrire une expression booléenne correspondant à \kw{x} {\tt xor} \kw{y}. 
		\item<4-> Discuter l'effet de l'instruction conditionnelle \mintinline{c}{if (a > 0 || a / b < 1)} suivant les valeurs des variables entières {\tt a} et {\tt b}.
		\item<5-> La déclaration de variable suivante est-elle correcte ? \mintinline{c}{char c = "a";} 
		\item<6-> Quelle est selon vous la cause du message : \textit{\textcolor{darkgray}{warning: ‘return’ with a value, in function returning void}} lors d'une compilation ?
	\end{enumerate}
	\end{exampleblock}
\end{frame}



\begin{frame}{\Ctitle}{\stitle}
	\begin{block}{Affichage et spécificateur de format}
		En C, l'affichage des variables se fait à l'aide de spécificateurs de format suivant le type de la variable \\
		\textcolor{gray}{
			\begin{tabular}{|l|l|}
				\hline
				Type                                               & Spécificateur \\
				\hline
				\kw{char}                                          & \kw{\%{}c}    \\
				\hline
				\kw{char[]}                                        & \kw{\%{}s}    \\
				\hline
				\kw{unsigned int}, \kw{uint8\_t} et \kw{uint32\_t} & \kw{\%{}u}    \\
				\hline
				\kw{int}, \kw{int8\_t} et \kw{int32\_t}            & \kw{\%{}d}    \\
				\hline
				\kw{float}                                         & \kw{\%{}f}    \\
				\hline
				\kw{uint64\_t}                                     & \kw{\%{}lu}   \\
				\hline
				\kw{int64\_t}                                      & \kw{\%{}ld}   \\
				\hline
			\end{tabular}}
	\end{block}
\end{frame}

\begin{frame}{\Ctitle}{\stitle}
	\begin{alertblock}{Définition}
		\begin{itemize}
			\item<1-> La \textcolor{blue}{portée} d'une variable est la partie du programme  dans laquelle cette variable est visible (on peut y faire référence).
			\item<2-> La portée peut-être :
				\begin{itemize}
					\item<3-> \textcolor{blue}{globale}, c'est à dire que la variable est accessible depuis tout le programme. En C, c'est le cas des variables déclarées en début de programme en dehors de tout bloc d'instructions.
					\item<4-> \textcolor{blue}{locale} lorsque la variable est déclarée dans un bloc d'instruction alors sa portée est limitée à ce bloc. C'est le cas des paramètres d'une fonction ou d'une variable de boucle.
				\end{itemize}
			\item<5-> Lorsque deux variables ont le même identifiant, c'est la variable ayant la plus petite portée (celle définie dans le bloc le plus intérieur) qui est accessible.
		\end{itemize}
	\end{alertblock}
\end{frame}

\begin{frame}[fragile]{\Ctitle}{\stitle}
	\begin{exampleblock}{Exemples}
		\begin{enumerate}
		\item<1->Dans le programme suivant, donner les portées de \kw{maxn}, \kw{n}, \kw{somme}, \kw{i}
		\inputC{\SPATH/cours_portee.c}{}{\footnotesize}
		\item<2-> On définit une variable entière {\tt i}, juste après la ligne 6, donner la portée de cette nouvelle variable. Le programme fonctionne-t-il encore correctement ?
		\end{enumerate}
	\end{exampleblock}
\end{frame}

\begin{frame}[fragile]{\Ctitle}{\stitle}
	\begin{exampleblock}{Exemples}
		\kw{maxn} est une variable globale
		\inputC{\SPATH/cours_portee.c}{3-14}{\small}
	\end{exampleblock}
\end{frame}

\begin{frame}[fragile]{\Ctitle}{\stitle}
	\begin{exampleblock}{Exemples}
		\kw{n} est un paramètre de la fonction \kw{harmo}
		\inputC{\SPATH/cours_portee.c}{5-9}{\small}
	\end{exampleblock}
\end{frame}

\begin{frame}[fragile]{\Ctitle}{\stitle}
	\begin{exampleblock}{Exemples}
		\kw{somme} est déclarée dans la fonction \kw{harmo}
		\inputC{\SPATH/cours_portee.c}{6-9}{\small}
	\end{exampleblock}
\end{frame}

\begin{frame}[fragile]{\Ctitle}{\stitle}
	\begin{exampleblock}{Exemples}
		\kw{i} est locale à la boucle
		\inputC{\SPATH/cours_portee.c}{7-8}{\small}
	\end{exampleblock}
\end{frame}

\begin{frame}[fragile]{\Ctitle}{\stitle}
	\begin{exampleblock}{Exemples}
		\kw{s} est locale au \kw{main}
		\inputC{\SPATH/cours_portee.c}{12-14}{\small}
	\end{exampleblock}
\end{frame}

\begin{frame}[fragile]{\Ctitle}{\stitle}

	\begin{block}{Conversion implicite de type}
		La ligne \mintinline{c}{double somme = 0;} est une \textcolor{blue}{conversion implicite de type}. En effet, 0 est de type entier mais est converti en flottant pour être affecté à la variable \kw{somme} qui est de type double.
	\end{block}
	\onslide<2->{
		\begin{block}{Conversion explicite : \textit{cast}}
			On aurait pu réaliser une \textcolor{blue}{conversion explicite} ou \textit{cast} en spécifiant le type de destination entre parenthèses : \mintinline{c}{double somme = (double) 0;}
		\end{block}}
\end{frame}

\begin{frame}[fragile]{\Ctitle}{\stitle}
	\begin{exampleblock}{Exemple}
		\inputC{\SPATH/cast.c}{}{\small}
		\begin{itemize}
			\item<2-> Quel est le résultat de ce programme ? Pourquoi ?
			\item<3-> Comment afficher le résultat de la division décimale ?
		\end{itemize}
	\end{exampleblock}
\end{frame}

\begin{frame}[fragile]{\Ctitle}{\stitle}
	\begin{exampleblock}{Exercice}
		Prévoir (et éventuellement observer) le résultat du programme suivant, expliquer.
		\inputC{\SPATH/pvaleur.c}{}{\small}
	\end{exampleblock}
\end{frame}

\begin{frame}{\Ctitle}{\stitle}
	\begin{block}{Remarques}
		Afin de repérer \textit{dès la compilation} le maximum de problèmes potentiels, il est \textcolor{BrickRed}{très fortement recommandé} de toujours utiliser \kw{gcc} avec les options :
		\begin{itemize}
			\item<2->\kw{-Wall} affichage de tous les \textit{warning}
			\item<3->\kw{-Wextra} affichage de \textit{warning} supplémentaires
			\item<4->\kw{-Wconversion} pour signaler les problèmes éventuels de conversion implicite
		\end{itemize}
		\onslide<5-> D'autre part, il est préférable de spécifier un fichier un nom pour l'exécutable produit grâce à l'option \kw{-o}
	\end{block}
	\onslide<6->{
		\begin{exampleblock}{Exemple}
			Pour compiler le programme {\tt exemple.c}, la ligne de compilation devrait donc être : \\
			{\tt gcc exemple.c -o exemple.exe -Wall -Wextra -Wconversion}
		\end{exampleblock}}
\end{frame}


\makess{Structures de contrôle}
\begin{frame}[fragile]{\Ctitle}{\stitle}
	\begin{alertblock}{Conditionnelle}
		\begin{itemize}
			\item<1-> \kw{if} \kw{(}\textit{condition}\kw{)}  \kw{\{} \textit{instruction} \kw{\}}
			\item<2-> \kw{if} \kw{(}\textit{condition}\kw{)}  \kw{\{} \textit{instruction} \kw{\}} else \kw{\{} \textit{instruction} \kw{\}}
		\end{itemize}
	\onslide<3-> \textcolor{BrickRed}{\small \danger} Pas de {\kw ;} après la condition !
	\end{alertblock}
	\begin{exampleblock}{Exemple}
		\onslide<3->{Ecrire une fonction \kw{compare} en C, prenant comme paramètre deux entiers \tt{a} et \tt{b} et renvoyant \tt{-1} si \tt{a<b}, \tt{0} si \tt{a=b} et 1 sinon.}
	\end{exampleblock}
\end{frame}

\begin{frame}[fragile]{\Ctitle}{\stitle}
	\begin{exampleblock}{Correction de l'exemple}
		\inputpartC{\SPATH/compare.c}{}{\small}{4}{16}
	\end{exampleblock}
\end{frame}

\begin{frame}[fragile]{\Ctitle}{\stitle}
	\begin{exampleblock}{Exemples}
		Ecrire les instructions conditionnelles suivantes (les variables utilisées sont supposées déjà déclarées):
		\begin{enumerate}
			\item<1-> Affiche "Ok" si \kw{a} est positif.
			\item<2-> Affecte \kw{nb} à 2 si \kw{d} est strictement positif, 1 si \kw{d} est nul et 0 sinon.
			\item<3-> Affecte \kw{e} à 1 si \kw{a} et \kw{b} ont la même parité et 0 sinon.
		\end{enumerate}
	\end{exampleblock}
\end{frame}

\begin{frame}[fragile]{\Ctitle}{\stitle}
	\begin{alertblock}{Boucles}
		\begin{itemize}
			\item<1-> \kw{for} \kw{(}\textit{init}\kw{;}\textit{fin};\textit{increment}\kw{)}  \kw{\{} \textit{instruction} \kw{\}} \\
			permet de répéter le bloc d'instruction pour chaque valeur prise par la variable de boucle.
				\onslide<2->{Généralement utilisé sous la forme : \mintinline{c}{for (int i=0; i<n; i=i+1)} {\tt\{} \dots {\tt\} }}\\
			\onslide<3->{\textcolor{gray}{On tolère {\tt i++} pour l'incrémentation, on recommande fortement de ne {\textit pas} utilisé cet opérateur dans un autre contexte.}}
			\item<4-> \kw{while} \kw{(}\textit{condition}\kw{)}  \kw{\{} \textit{instruction} \kw{\}} \\
			permet de répéter le bloc d'instruction tant que la condition est vérifiée.
			\item<5-> Lorsqu'une boucle se trouve dans le corps d'une fonction, une instruction \kw{return} a pour effet de quitter immédiatement cette boucle (et le corps de la fonction) et de revenir au point d'appel de la fonction.
			\item<6-> Une boucle peut-être interrompue avec l'instruction \kw{break}.
			\item<7-> \textcolor{BrickRed}{\small \danger \;} Pas de \kw{;} après la condition.
		\end{itemize}
	\end{alertblock}
\end{frame}

\begin{frame}[fragile]{\Ctitle}{\stitle}
	\begin{exampleblock}{Exemple}
		Le type \kw{char} correspond en fait à une valeur entière, les caractères imprimables vont de 32 (l'espace) à 127 ({\sc del}). Sachant que l'affichage d'un caractère avec \kw{printf} se fait à l'aide de \kw{\%c}
		\begin{itemize}
			\item<2-> Ecrire une boucle \kw{for} permettant d'afficher ces caractères.
			\item<3-> Faire de même avec une boucle \kw{while}.
		\end{itemize}
	\end{exampleblock}
\end{frame}

\begin{frame}[fragile]{\Ctitle}{\stitle}
	\begin{exampleblock}{Correction de l'exemple}
		\begin{itemize}
			\item<1-> Avec une boucle \kw{for} \\
				\inputC{\SPATH/cours_ex5.c}{}{\small}
			\item<2-> Avec une boucle \kw{while} \\
				\inputC{\SPATH/cours_ex6.c}{}{\small}
		\end{itemize}
	\end{exampleblock}
\end{frame}

\begin{frame}[fragile]{\Ctitle}{\stitle}
	\begin{exampleblock}{Exercices}
	Ecrire une boucle permettant :
	\begin{enumerate}
		\item<1-> d'afficher les 10 premiers multiples de 42.
		\item<2-> d'afficher les entiers de 10 à 1 (dans cet ordre).
		\item<3-> de calculer la somme des entiers impairs de 1 à 999.
		\item<4-> de déterminer le plus petit entier $n$ tel que $1 + \dfrac{1}{2} + \dfrac{1}{3} + \dots +\dfrac{1}{n} > 7$.
	\end{enumerate}
	\end{exampleblock}
\end{frame}


\makess{Tableaux à une dimension, chaines de caractères}
\begin{frame}[fragile]{\Ctitle}{\stitle}
	\begin{block}{Tableaux}
		\textcolor{BrickRed}{\small \important} \textit{\footnotesize La notion de tableau en C est intimement liée à celle de \textit{pointeur}. Ces derniers seront abordés plus tard, aussi on fait ici une présentation élémentaire des tableaux, en particulier, on s'interdit pour le moment d'écrire des fonctions qui renvoient un tableau.}
		\begin{itemize}
			\item<1-> Un tableau se déclare en donnant sa longueur et le type de ses éléments. \\
				\onslide<2->{\mintinline[fontsize=\small]{c}{bool est_premier[1000]; //un tableau de 1000 booléens}}
			\item<3-> On peut initialiser le tableau en donnant une liste de valeurs entre accolades. \\
				\onslide<4->{\mintinline[fontsize=\small]{c}{double notes[4]={5.5, 12.0, 13.5, 7.0}; //un tableau de 4 flottants}}
			\item<5-> Les éléments sont numérotés à partir de 0
			\item<6-> On accède à un élément en donnant son numéro (son indice) entre crochet.\\
				\onslide<7->{\mintinline[fontsize=\small]{c}{est_premier[0]; //Le premier élément du tableau est_premier}}
			\item<7-> \textcolor{BrickRed}{\small \danger\;} Un accès en dehors des bornes du tableau est un \textcolor{blue}{comportement indéfini}\\
			\item<8-> On ne peut pas accéder à la taille d'un tableau en C. En conséquence, une fonction qui travaille sur un tableau doit aussi recevoir en argument la taille de ce dernier. 
		\end{itemize}
	\end{block}
\end{frame}


\begin{frame}[fragile]{\Ctitle}{\stitle}
	\begin{exampleblock}{Exemple}
		\onslide<1-> Ecrire une fonction \kw{croissant} qui prend un argument un tableau et sa taille et renvoie \kw{true} si le tableau est trié et false sinon.
		\onslide<2->{\inputpartC{\SPATH/cours_ex7.c}{}{}{4}{10}}
	\end{exampleblock}
\end{frame}

\begin{frame}[fragile]{\Ctitle}{\stitle}
	\begin{exampleblock}{Exemple}
		\onslide<1-> Ecrire une fonction \kw{echange} qui prend un argument un tableau et deux indices  \kw{i} et \kw{j} ne renvoie rien et échange les éléments d'indice \kw{i} et \kw{j} de ce tableau.\\
		\onslide<2->{\inputpartC{\SPATH/cours_ex9.c}{}{}{4}{9}}
		\onslide<3->\textcolor{gray}{Mais .... en C, les paramètres sont passés par valeur non ?}
	\end{exampleblock}
\end{frame}

\begin{frame}[fragile]{\Ctitle}{\stitle}
	\begin{exampleblock}{Exercices}
		\begin{enumerate}
		\item<1-> Déclarer un tableau {\tt tab} de 20 entiers.
		\item<2-> Déclarer un tableau {\tt test} de 5 booléens initialisées aux valeurs {\tt false, true, false, true, false}.
		\item<3-> Déclarer un tableau de 100 entiers, écrire une boucle permettant d'initialiser {\tt t[i]} à la valeur {\tt 2*i}.
		\item<4-> Commenter le programme suivant :
		\inputpartC{\SPATH/ub_tab.c}{}{\small}{3}{8}
		\end{enumerate}
	\end{exampleblock}
\end{frame}

\begin{frame}[fragile]{\Ctitle}{\stitle}
	\begin{block}{Chaines de caractères}
		\begin{itemize}
			\item<1-> En C, les chaines de caractères (notées entre guillemets \kw{"}) sont des tableaux de caractères (type \kw{char[]}) dont le dernier élément est le caractère spécial \kw{'\textbackslash{}0'} qui marque la fin de la chaine.
			\item<2-> Par exemple \mintinline{c}{char exemple[] = "Hello !";} crée le tableau :\\
				\begin{tabular}{|>{\tt}c|>{\tt}c|>{\tt}c|>{\tt}c|>{\tt}c|>{\tt}c|>{\tt}c|>{\tt}c|}
					\hline
					H & e & l & l & o & \; & ! & \textbackslash{}0 \\
					\hline
				\end{tabular}
			\item<3-> Le module \kw{string.h} fournit des fonctions usuelles de manipulation de caractères, notamment :
				\begin{itemize}
					\item<4-> \kw{strlen} : renvoie la longueur de la chaine de caractères
					\item<5-> \kw{strcpy} : copie une chaine de caractères
					\item<6-> \kw{strcat} : concaténation de chaines de caractères
				\end{itemize}
			\item<4-> Contrairement à un tableau \og{} classique \fg{}, on peut donc connaitre la longueur d'une chaine de caractères, grâce à la présence du caractère sentinelle \kw{\textbackslash{0}} qui en indique la fin.
		\end{itemize}
	\end{block}
\end{frame}

\begin{frame}[fragile]{\Ctitle}{\stitle}
	\begin{exampleblock}{Exemple}
		\begin{enumerate}
		\item<1->Quel est l'affichage produit par le programme suivant ?
		\onslide<2->{\inputC{\SPATH/cours_ex10.c}{}{}}
		\item<2->Quel est l'affichage produit en remplaçant la ligne 8 par \mintinline{c}{test[3]='\0';} ?
	\end{enumerate}
	\end{exampleblock}
\end{frame}

\makess{Saisie de valeurs au clavier}
\begin{frame}[fragile]{\Ctitle}{\stitle}
	\begin{block}{Fonction {\tt scanf}}
		\begin{itemize}
			\item<1-> La fonction \kw{scanf} permet la saisie de valeurs des variables depuis le clavier.
			\item<2-> Elle prend en argument un \textcolor{blue}{spécificateur de format} (comme \kw{printf}) qui permet de préciser le type de la variable attendue.
			\item<3-> On fera précéder la variable qui reçoit la valeur saisie au clavier du caractère \kw{\&} \\
				\onslide<4-> Ce point sera expliqué plus loin dans le cours.
			\item<5-> Cette fonction renvoie le nombre de valeurs correctement lues.
		\end{itemize}
	\end{block}
	\begin{exampleblock}{Exemple}
		\begin{itemize}
			\item<6-> Ecrire un programme qui demande à l'utilisateur de saisir au clavier deux entiers {\tt a} et {\tt b} puis affiche leur somme.
			\item<7-> Modifier ce programme pour que les valeurs saisies soient des flottants.
		\end{itemize}
	\end{exampleblock}
\end{frame}

\begin{frame}{\Ctitle}{\stitle}
	\begin{exampleblock}{Correction}
		\inputC{\SPATH/somme.c}{}{\small}
	\end{exampleblock}
\end{frame}

\begin{frame}{\Ctitle}{\stitle}
	\begin{exampleblock}{Correction}
		\inputC{\SPATH/sommef.c}{}{\small}
	\end{exampleblock}
\end{frame}

\end{document}

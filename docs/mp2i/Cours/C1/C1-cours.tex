\PassOptionsToPackage{dvipsnames,table}{xcolor}
\documentclass[10pt]{beamer}
\usepackage{Cours}

\begin{document}

\input{\detokenize{/home/fenarius/Travail/Cours/cpge-info/latex//MacrosCours.tex}}

% Numéro et titre de chapitre
\setcounter{numchap}{1}
\newcommand{\Ctitle}{\cnum {Introduction au langage C}}

\makess{Historique}
\begin{frame}{\Ctitle}{\stitle}
	\begin{block}{Bref historique}
		\begin{itemize}
			\item<1-> 1972 : début du développement du langage C par Dennis Ritchie et Ken Thomson aux laboratoires Bell parallèlement à la création du système d'exploitation {\sc unix}.
			\item<2-> 1978 : première édition du livre "The C programming language" (Kernighan \& Ritchie)
			\item<3-> 1983 : première standardisation du langage par l'{\sc ansi} qui assure la compatibilité et la portabilité entre différentes plateformes. La dernière standardisation date de 2018 (C18)
			\item<4-> A partir de 1983 : développement de plusieurs dérivés de C, parmi lesquels C++ (B. Strousrtup, 1983), C\# (Microsoft, 2000), Go (Google, 2007), Rust (Mozilla, 2010)
		\end{itemize}
	\end{block}
\end{frame}

\makess{Caractéristiques du C}
\begin{frame}{\Ctitle}{\stitle}
	\begin{block}{Quelques aspects du C}
		\begin{itemize}
			\item<1-> Langage \textcolor{blue}{impératif} : séquence d'instructions exécutées par l'ordinateur pour modifier l'état du programme,. C n'est ni orienté objet, ni fonctionnel.
			\item<2-> Les variables sont \textcolor{blue}{mutables} c'est à dire qu'elles peuvent changer de valeur pendant l'exécution.
			\item<3-> Le langage C est \textcolor{blue}{statiquement typé} c'est à dire qu'une variable appartient à un type défini durant toute sa durée de vie.
			\item<4-> Equipé d'une librairie standard : la \textcolor{blue}{libc}.
			\item<5-> Le standard précise un certain nombres de \textcolor{blue}{comportements indéfinis}, c'est à dire de programmes dont le résultat est imprévisible. 
			\item<6-> Plus \textcolor{blue}{proche de la machine} que bien d'autres langages de haut niveau, ce qui induit une certaine efficacité.
			\item<7-> Souvent utilisé pour le développement de systèmes d'exploitation, de pilotes de périphériques, de logiciels embarqués,
		\end{itemize}
	\end{block}
\end{frame}

\begin{frame}{\Ctitle}{\stitle}
	\begin{block}{Compilation}
		Le langage C est \textcolor{blue}{compilé} : \\ \medskip
		\begin{tabular}{ccccc}
			\rnode{CS}{\begin{rcadre}{lightgray}{Sepia}{2.4}{1.4}
					           \textcolor{Sepia}{\small \faFile\; Code source \\
					           {\footnotesize (fichier(s) texte .c)}}
				           \end{rcadre}} & \hspace{0.8cm} & \onslide<3->{\rnode{{CO}}{\begin{cadre}{white}{black}{2.8}{1.4}  {\small \textbf{\faCog\;} Compilateur} \\ {\footnotesize (gcc, clang, ..)} \end{cadre}}} & \hspace{0.8cm} &
			\onslide<5->{\rnode{EX}{\begin{rcadre}{lightgray}{blue}{2.4}{1.4}
						                        \textcolor{blue}{\small \faFileArchive\; Exécutable \\ {\footnotesize (fichier binaire)}}
					                        \end{rcadre}}}                                                                                                             \\
		\end{tabular}
		\onslide<3->{\ncline[doubleline=true,doublesep=2pt,doublecolor=OliveGreen,linecolor=OliveGreen,linewidth=1pt,arrowsize=10pt,arrowinset=0.2,arrowlength=1.2]{->}{CS}{CO}}
		\onslide<6->{\ncline[doubleline=true,doublesep=2pt,doublecolor=OliveGreen,linecolor=OliveGreen,linewidth=1pt,arrowsize=10pt,arrowinset=0.2,arrowlength=1.2]{->}{CO}{EX}}
		\begin{enumerate}
			\item<2-> Les IDE comme VS Code signalent certaines erreurs dans le code.
			\item<4-> La compilation peut produire des erreurs ou des avertissement (\textit{warning}) \\
			\textcolor{gray}{\small La compilation se déroule en 4 étapes : préprocesseur, compilation, assemblage, editions de lien}
			\item<6-> Une compilation sans erreur (mais éventuellement des \textit{warning}) produit un exécutable.
			\item<7-> Les erreurs dans l'exécution ne feront pas référence aux instructions du code source.
		\end{enumerate}
	\end{block}
\end{frame}

\makess{Exemples de programmes}
\begin{frame}[fragile]{\Ctitle}{\stitle}
	\begin{exampleblock}{Programme minimal}
		\begin{overprint}
			\onslide<1>
			\begin{langageC}
#include <stdio.h>

int main()
{
	printf("Hello world \n");
	return 0;
}
			\end{langageC}
\medskip
Si le fichier texte s'appelle {\tt hello.c}, on lance la compilation avec {\tt gcc hello.c}, l'exécutable produit s'appelle par défaut {\tt a.out}, on peut modifier ce nom avec l'option {\tt -o}. Par exemple : {\tt gcc -o hello.exe hello.c}
			\onslide<2>
			\begin{langageC*}{highlightlines=1}
#include <stdio.h>

int main()
{
	printf("Hello world \n");
	return 0;
}
			\end{langageC*}
			\medskip
			Appel aux fonctions \textcolor{blue}{\textbf{st}}andar\textcolor{blue}{\textbf d} d'entrées et de sorties (\textcolor{blue}{\textbf i}nput et \textcolor{blue}{\textbf o}utput)  de la libc.
			\onslide<3>
			\begin{langageC}
#include <stdio.h>

int |\myem{main}|()
{
	printf("Hello world \n");
	return 0;
}
			\end{langageC}
			\medskip
			Un programme C contient une fonction \kw{ main} par laquelle l'exécution du programme commence.
			\onslide<4>
			\begin{langageC}
#include <stdio.h>

|\myem{int}| main|\myem{()}|
{
	printf("Hello world \n");
	return 0;
}
			\end{langageC}
			\medskip
			Avant le nom d'une fonction on trouve le type de variable qu'elle renvoie (ici \kw{ int}) et après entre parenthèses, les arguments éventuels de la fonction (ici aucun).
			\onslide<5>
			\begin{langageC}
#include <stdio.h>

int main()
|\myem{\{}|
	printf("Hello world \n")|\myem{;}|
	return 0|\myem{;}|
|\myem{\}}|
			\end{langageC}
			\medskip
			Les blocs d'instructions sont délimités par des accolades (\kw{ \{} et \kw{ \}}). Les instructions doivent se terminer par un point virgule \kw{ ;}. Les espaces, sauts de ligne et indentation sont ignorés par le compilateur, mais sont nécessaires pour une bonne lisibilité.
			\onslide<6>
			\begin{langageC*}{highlightlines=5}
#include <stdio.h>

int main()
{
	printf("Hello world \n");
	return 0;
}
			\end{langageC*}
			\medskip
			L'instruction \kw{ printf} permet d'afficher dans le terminal. On notera les guillemets (\textcolor{blue}{"}) pour délimiter une chaîne de caractères et le caractère \kw{ \textbackslash n} pour indiquer un retour à la ligne.
			\onslide<7>
			\begin{langageC*}{highlightlines=6}
#include <stdio.h>

int main()
{
	printf("Hello world \n");
	return 0|\myem{;}|
}
			\end{langageC*}
			\medskip
			L'instruction \kw{ return} quitte la fonction en renvoyant la valeur donnée. Ici, on renvoie {\tt 0}, qui indique traditionnellement que le programme se  termine sans erreurs.
		\end{overprint}
	\end{exampleblock}

\end{frame}

\begin{frame}{\Ctitle}{\stitle}
	\begin{exampleblock}{Exemple de boucle}
		\inputC{/home/fenarius/Travail/Cours/cpge-info/docs/mp2i/files/C1/cours_ex2.c}{4}{\small}
	\medskip
	Déclaration de la variable {\tt somme} de type \kw{int} et initialisation à zéro. A noter qu'on peut déclarer une variable sans l'initialiser.
	\end{exampleblock}
\end{frame}

\begin{frame}{\Ctitle}{\stitle}
	\begin{exampleblock}{Exemple de boucle}
		\inputC{/home/fenarius/Travail/Cours/cpge-info/docs/mp2i/files/C1/cours_ex2.c}{5}{\small}
	\medskip
	Une variable dont la valeur ne sera pas modifiée peut être déclaré avec \kw{const}.\\
	\onslide<2-> \textcolor{gray}{On peut aussi utiliser une directive de précompilation \\ {\tt \#define NMAX 100}}
	\end{exampleblock}
\end{frame}

\begin{frame}{\Ctitle}{\stitle}
	\begin{exampleblock}{Exemple de boucle}
		\inputC{/home/fenarius/Travail/Cours/cpge-info/docs/mp2i/files/C1/cours_ex2.c}{6}{\small}
	\medskip
	On remarque que la boucle \kw{for} est de la forme {\tt for} \textit{(init; fin; incr)}. Les opérateurs de comparaison en C sont \kw{==}, \kw{!=}, \kw{<}, \kw{>}, \kw{<=} et \kw{>=}.
	\end{exampleblock}
\end{frame}

\begin{frame}{\Ctitle}{\stitle}
	\begin{exampleblock}{Exemple de boucle}
		\inputC{/home/fenarius/Travail/Cours/cpge-info/docs/mp2i/files/C1/cours_ex2.c}{8}{\small}
	\medskip
	On veut afficher un \kw{int} dans la réponse, on utilise \kw{\%{}d} dans \kw{printf} à l'emplacement souhaité.
	\end{exampleblock}
\end{frame}

\begin{frame}{\Ctitle}{\stitle}
	\begin{exampleblock}{Exemple d'instruction conditionnelle}
		\inputC{/home/fenarius/Travail/Cours/cpge-info/docs/mp2i/files/C1/cours_ex3.c}{2}{\small}
	\medskip
	Une ligne de commentaire commence avec \kw{//}, un commentaire multiligne est encadré par \kw{/*} et \kw{*/}
	\end{exampleblock}
\end{frame}

\begin{frame}{\Ctitle}{\stitle}
	\begin{exampleblock}{Exemple d'instruction conditionnelle}
		\inputC{/home/fenarius/Travail/Cours/cpge-info/docs/mp2i/files/C1/cours_ex3.c}{3}{\small}
	\medskip
	Définition d'une fonction \kw{syracuse} qui prend comme paramètre un entier et renvoie un entier. C'est la \textcolor{blue}{signature} de la fonction. \\
	\textcolor{BrickRed}{\important} En C, les paramètres sont passés par \textcolor{blue}{valeur}.
	\end{exampleblock}
\end{frame}

\begin{frame}{\Ctitle}{\stitle}
	\begin{exampleblock}{Exemple d'instruction conditionnelle}
		\inputC{/home/fenarius/Travail/Cours/cpge-info/docs/mp2i/files/C1/cours_ex3.c}{4-7}{\small}
	\medskip
	Instruction conditionnelle : on exécute le bloc qui suit la condition si celle-ci est vérifiée et sinon le bloc qui suit le \kw{else} (s'il est présent).
	Noter les parenthèses autour de la condition.
	\end{exampleblock}
\end{frame}

\makess{Définitions et types de base}
\begin{frame}{\Ctitle}{\stitle}
	\begin{alertblock}{Types de base}
		\begin{tabularx}{\linewidth}{|l|c|>{\footnotesize}X|}
			\hline
			Type & Opérations & Commentaires \\
			\hline
			\kw{int} et \kw{unsigned int}& \kw{+}, \kw{-}, \kw{*}, \kw{/}, \kw{\%} & Entiers signés ou non signés codés sur un minimum de 16 bits.\\ 
			\kw{int}$N$\kw{\_t} et \kw{uint}$N$\kw{\_t}&  & Entiers codés sur $N$ bits accessibles dans \kw{stdint.h} ($N=8$, 32 ou 64)\\
			\hline
			& & \  \newline \ \newline	\\
			\hline
			&&  \  \newline \  \\
			\hline
			 &  & \  \newline \  \\
			\hline
		\end{tabularx}
		\vspace{1cm}
	\end{alertblock}
\end{frame}

\begin{frame}{\Ctitle}{\stitle}
	\begin{alertblock}{Types de base}
		\begin{tabularx}{\linewidth}{|l|c|>{\footnotesize}X|}
			\hline
			Type & Opérations & Commentaires \\
			\hline
			\kw{int} et \kw{unsigned int}& \kw{+}, \kw{-}, \kw{*}, \kw{/}, \kw{\%} & Entiers signés ou non signés codés sur un minimum de 16 bits.\\ 
			\kw{int}$N$\kw{\_t} et \kw{uint}$N$\kw{\_t}&  & Entiers codés sur $N$ bits accessibles dans \kw{stdint.h} ($N=8$, 32 ou 64) \\
			\hline
			\kw{float} et \kw{double} & \kw{+}, \kw{-}, \kw{*}, \kw{/} & 	Représentation des nombres en virgules flottantes en simple ou double précision de la norme {\sc ieee754}. Fonctions élémentaires dans \kw{math.h}\\
			\hline
			&&  \  \newline \  \\
			\hline
			 &  & \  \newline \  \\
			\hline
		\end{tabularx}
		\vspace{0.7cm}
	\end{alertblock}
\end{frame}

\begin{frame}{\Ctitle}{\stitle}
	\begin{alertblock}{Types de base}
		\begin{tabularx}{\linewidth}{|l|c|>{\footnotesize}X|}
			\hline
			Type & Opérations & Commentaires \\
			\hline
			\kw{int} et \kw{unsigned int}& \kw{+}, \kw{-}, \kw{*}, \kw{/}, \kw{\%} & Entiers signés ou non signés codés sur un minimum de 16 bits.\\ 
			\kw{int}$N$\kw{\_t} et \kw{uint}$N$\kw{\_t}&  & Entiers codés sur $N$ bits accessibles dans \kw{stdint.h} ($N=8$, 32 ou 64)\\
			\hline
			\kw{float} et \kw{double} & \kw{+}, \kw{-}, \kw{*}, \kw{/} & 	Représentation des nombres en virgules flottantes en simple ou double précision de la norme {\sc ieee754}. Fonctions élémentaires dans \kw{math.h}\\
			\hline
			\kw{bool} & \kw{||}  \kw{\&\&}, \kw{!}  & Booléens accessibles dans \kw{stdbool.h}. Evaluations paresseuses des expressions. \\
			\hline
			 &  & \  \newline \  \\
			\hline
		\end{tabularx}
		\vspace{0.7cm}
	\end{alertblock}
\end{frame}

\begin{frame}{\Ctitle}{\stitle}
	\begin{alertblock}{Types de base}
		\begin{tabularx}{\linewidth}{|l|p{1.8cm}|>{\footnotesize}X|}
			\hline
			Type & Opérations & Commentaires \\
			\hline
			\kw{int} et \kw{unsigned int}& \kw{+}, \kw{-}, \kw{*}, \kw{/}, \kw{\%}  \newline \textcolor{gray}{{\tt ++},{\tt -{}-},{\tt +=},{\tt -=}}& Entiers signés ou non signés codés sur un minimum de 16 bits.\\ 
			\kw{int}$N$\kw{\_t} et \kw{uint}$N$\kw{\_t}&  & Entiers codés sur $N$ bits accessibles dans \kw{stdint.h} ($N=8$, 32 ou 64)\\
			\hline
			\kw{float} et \kw{double} & \kw{+}, \kw{-}, \kw{*}, \kw{/} & 	Représentation des nombres en virgules flottantes en simple ou double précision de la norme {\sc ieee754}. Fonctions élémentaires dans \kw{math.h}\\
			\hline
			\kw{bool} & \kw{||}  \kw{\&\&}, \kw{!}  & Booléens accessibles dans \kw{stdbool.h}. Evaluations paresseuses des expressions. \\
			\hline
			\kw{char} &  \textcolor{gray}{\tt +, -} & Caractères noté entre quotes (\kw{'}), uniquement ceux de la table {\sc ascii}. Caractère nul : \kw{'\textbackslash{}0'}\\
			\hline
		\end{tabularx}
		\onslide<2-> Pour indiquer l'absence de type, notamment pour les fonctions ne renvoyant rien (par exemple une fonction d'affichage) on utilise \kw{void}.
	\end{alertblock}
\end{frame}


\begin{frame}{\Ctitle}{\stitle}
	\begin{block}{Affichage et spécificateur de format}
		En C, l'affichage des variables se fait à l'aide de spécificateurs de format suivant le type de la variable \\
		\textcolor{gray}{
		\begin{tabular}{|l|l|}
			\hline
				Type & Spécificateur \\
			\hline
				\kw{char} & \kw{\%{}c} \\
			\hline
			   \kw{char[]} & \kw{\%{}s} \\
			\hline
				\kw{unsigned int}, \kw{uint8\_t} et \kw{uint32\_t}& \kw{\%{}u} \\
			\hline
				\kw{int}, \kw{int8\_t} et \kw{int32\_t} & \kw{\%{}d} \\
			\hline
				\kw{float} & \kw{\%{}f} \\
			\hline
			\kw{uint64\_t} & \kw{\%{}lu} \\
			\hline
			\kw{int64\_t} & \kw{\%{}ld} \\
			\hline
		\end{tabular}}
	\end{block}
\end{frame}

\makess{Structures de contrôle}
\begin{frame}[fragile]{\Ctitle}{\stitle}
	\begin{alertblock}{Conditionnelle}
		\begin{itemize}
		\item<1-> \kw{if} \kw{(}\textit{condition}\kw{)}  \kw{\{} \textit{instruction} \kw{\}}
		\item<2-> \kw{if} \kw{(}\textit{condition}\kw{)}  \kw{\{} \textit{instruction} \kw{\}} else \kw{\{} \textit{instruction} \kw{\}}
		\end{itemize}
	\end{alertblock}
	\begin{exampleblock}{Exemple}
		\onslide<3->{Ecrire une fonction \kw{compare} en C, prenant comme paramètre deux entiers \tt{a} et \tt{b} et renvoyant \tt{-1} si \tt{a<b}, \tt{0} si \tt{a=b} et 1 sinon.}
	\end{exampleblock}
\end{frame}

\begin{frame}[fragile]{\Ctitle}{\stitle}
	\begin{exampleblock}{Correction de l'exemple}		
		\begin{langageC}
int compare(int a, int b) 
{
	if (a<b) 
	{return -1;}
	else if (a==b)
	{return 0;}
	else
	return 1;
}
		\end{langageC}
	\end{exampleblock}
\end{frame}

\begin{frame}[fragile]{\Ctitle}{\stitle}
	\begin{alertblock}{Boucles}
		\begin{itemize}
		\item<1-> \kw{for} \kw{(}\textit{init}\kw{;}\textit{fin};\textit{increment}\kw{)}  \kw{\{} \textit{instruction} \kw{\}} \\
		\onslide<2->{Généralement utilisé sous la forme : \mintinline{c}{for (int i=0; i<n; i=i+1)} {\tt\{} \dots {\tt\} }}
		\item<3-> \kw{while} \kw{(}\textit{condition}\kw{)}  \kw{\{} \textit{instruction} \kw{\}}
		\item<4-> Une boucle peut-être interrompue avec l'instruction \kw{break}
		\end{itemize}
	\end{alertblock}
	\begin{exampleblock}{Exemple}
		\onslide<5->{Le type \kw{char} correspond en fait à une valeur entière, les caractères imprimables vont de 32 (l'espace) à 127 ({\sc del}). Sachant que l'affichage d'un caractère avec \kw{printf} se fait à l'aide de \kw{\%c}}
		\begin{itemize}
			\item<6-> Ecrire une boucle \kw{for} permettant d'afficher ces caractères.
			\item<7-> Faire de même avec une boucle \kw{while}.
		\end{itemize}
	\end{exampleblock}
\end{frame}

\begin{frame}[fragile]{\Ctitle}{\stitle}
	\begin{exampleblock}{Correction de l'exemple}		
		\begin{itemize}
			\item<1-> Avec une boucle \kw{for} \\
			\inputC{/home/fenarius/Travail/Cours/cpge-info/docs/mp2i/files/C1/cours_ex5.c}{}{\small}
			\item<2-> Avec une boucle \kw{while} \\
			\inputC{/home/fenarius/Travail/Cours/cpge-info/docs/mp2i/files/C1/cours_ex6.c}{}{\small}
		\end{itemize}
	\end{exampleblock}
\end{frame}

\makess{Tableaux à une dimension, chaines de caractères}
\begin{frame}[fragile]{\Ctitle}{\stitle}
	\begin{block}{Tableaux}
		\begin{itemize}
			\item<1-> Un tableau se déclare en donnant sa longueur et le type de ses éléments. \\
			\onslide<2->{\mintinline[fontsize=\small]{c}{bool est_premier[1000]; //un tableau de 1000 booléens}}
			\item<3-> On peut initialiser le tableau en donnant une liste de valeurs entre accolades. \\
			\onslide<4->{\mintinline[fontsize=\small]{c}{double notes[4]={5.5, 12.0, 13.5, 7.0}; //un tableau de 4 flottants}}
			\item<5-> Les éléments sont numérotés à partir de 0
			\item<6-> On accède à un élément en donnant son numéro (son indice) entre crochet.\\
			\onslide<7->{\mintinline[fontsize=\small]{c}{est_premier[0]; //Le premier élément du tableau est_premier}}
		\end{itemize}
	\end{block}
	\begin{alertblock}{\textcolor{yellow}{\important}\; Attention}
		\begin{itemize}
			\item<8-> Un accès en dehors des bornes du tableau est un \textcolor{blue}{comportement indéfini}\\
			\item<9-> La gestion de la taille du tableau est de la \textit{responsabilité du programmeur}. Il n'y a pas de fonctions permettant d'y accéder. En conséquence lorsqu'un tableau est passé en paramètre à une fonction on passe aussi sa taille.
		\end{itemize} 
	\end{alertblock}
\end{frame}


\begin{frame}[fragile]{\Ctitle}{\stitle}
	\begin{exampleblock}{Exemple}
		\onslide<1-> Ecrire une fonction \kw{croissant} qui prend un argument un tableau et sa taille et renvoie \kw{true} si le tableau est trié et false sinon.
		\onslide<2->{\inputpartC{/home/fenarius/Travail/Cours/cpge-info/docs/mp2i/files/C1/cours_ex7.c}{}{}{4}{10}}
	\end{exampleblock}
\end{frame}

\begin{frame}[fragile]{\Ctitle}{\stitle}
	\begin{exampleblock}{Exemple}
		\onslide<1-> Ecrire une fonction \kw{echange} qui prend un argument un tableau et deux indices  \kw{i} et \kw{j} ne renvoie rien et échange les éléments d'indice \kw{i} et \kw{j} de ce tableau.\\
		\onslide<2->{\inputpartC{/home/fenarius/Travail/Cours/cpge-info/docs/mp2i/files/C1/cours_ex9.c}{}{}{4}{9}}
		\onslide<3->\textcolor{gray}{Mais .... en C, les paramètres sont passés par valeur non ?}
	\end{exampleblock}
\end{frame}


\begin{frame}[fragile]{\Ctitle}{\stitle}
	\begin{block}{Chaines de caractères}
		\begin{itemize}
			\item<1-> En C, les chaines de caractères (notées entre guillemets \kw{"}) sont des tableaux de caractères (type \kw{char[]}) dont le dernier élément est le caractère spécial \kw{'\textbackslash{}0'} qui marque la fin de la chaine.
			\item<2-> Par exemple \mintinline{c}{char exemple[] = "Hello !";} crée le tableau :\\
			\begin{tabular}{|>{\tt}c|>{\tt}c|>{\tt}c|>{\tt}c|>{\tt}c|>{\tt}c|>{\tt}c|>{\tt}c|}
				\hline
				H & e & l & l & o & \;  & ! & \textbackslash{}0 \\
				\hline
			\end{tabular}
			\item<3-> Le module \kw{string.h} fournit des fonctions usuelles de manipulation de caractères, notamment :
			\begin{itemize}
				\item<4-> \kw{strlen} : renvoie la longueur de la chaine de caractères
				\item<5-> \kw{strcpy} : copie une chaine de caractères
				\item<6-> \kw{strcat} : concaténation de chaines de caractères
			\end{itemize}
		\end{itemize}
	\end{block}
\end{frame}

\begin{frame}[fragile]{\Ctitle}{\stitle}
	\begin{exampleblock}{Exemple}
		Quel est l'affichage produit par le programme suivant ?
		\onslide<2->{\inputC{/home/fenarius/Travail/Cours/cpge-info/docs/mp2i/files/C1/cours_ex10.c}{}{}}	
	\end{exampleblock}
\end{frame}


\end{document}

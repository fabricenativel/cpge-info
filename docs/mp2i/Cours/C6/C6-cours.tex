\PassOptionsToPackage{dvipsnames,table}{xcolor}
\documentclass[10pt]{beamer}
\usepackage{Cours}

\begin{document}


\newcounter{numchap}
\setcounter{numchap}{1}
\newcounter{numframe}
\setcounter{numframe}{0}
\newcommand{\mframe}[1]{\frametitle{#1} \addtocounter{numframe}{1}}
\newcommand{\cnum}{\fbox{\textcolor{yellow}{\textbf{C\thenumchap}}}~}
\newcommand{\makess}[1]{\section{#1} \label{ss\thesection}}
\newcommand{\stitle}{\textcolor{yellow}{\textbf{\thesection. \nameref{ss\thesection}}}}

\definecolor{codebg}{gray}{0.90}
\definecolor{grispale}{gray}{0.95}
\definecolor{fluo}{rgb}{1,0.96,0.62}
\newminted[langageC]{c}{linenos=true,escapeinside=||,highlightcolor=fluo,tabsize=2,breaklines=true}
\newminted[codepython]{python}{linenos=true,escapeinside=||,highlightcolor=fluo,tabsize=2,breaklines=true}
% Inclusion complète (ou partiel en indiquant premiere et dernière ligne) d'un fichier C
\newcommand{\inputC}[3]{\begin{mdframed}[backgroundcolor=codebg] \inputminted[breaklines=true,fontsize=#3,linenos=true,highlightcolor=fluo,tabsize=2,highlightlines={#2}]{c}{#1} \end{mdframed}}
\newcommand{\inputpartC}[5]{\begin{mdframed}[backgroundcolor=codebg] \inputminted[breaklines=true,fontsize=#3,linenos=true,highlightcolor=fluo,tabsize=2,highlightlines={#2},firstline=#4,lastline=#5,firstnumber=1]{c}{#1} \end{mdframed}}
\newcommand{\inputpython}[3]{\begin{mdframed}[backgroundcolor=codebg] \inputminted[breaklines=true,fontsize=#3,linenos=true,highlightcolor=fluo,tabsize=2,highlightlines={#2}]{python}{#1} \end{mdframed}}
\newcommand{\inputpartOCaml}[5]{\begin{mdframed}[backgroundcolor=codebg] \inputminted[breaklines=true,fontsize=#3,linenos=true,highlightcolor=fluo,tabsize=2,highlightlines={#2},firstline=#4,lastline=#5,firstnumber=1]{OCaml}{#1} \end{mdframed}}
\BeforeBeginEnvironment{minted}{\begin{mdframed}[backgroundcolor=codebg]}
\AfterEndEnvironment{minted}{\end{mdframed}}
\newcommand{\kw}[1]{\textcolor{blue}{\tt #1}}

\newtcolorbox{rcadre}[4]{halign=center,colback={#1},colframe={#2},width={#3cm},height={#4cm},valign=center,boxrule=1pt,left=0pt,right=0pt}
\newtcolorbox{cadre}[4]{halign=center,colback={#1},colframe={#2},arc=0mm,width={#3cm},height={#4cm},valign=center,boxrule=1pt,left=0pt,right=0pt}
\newcommand{\myem}[1]{\colorbox{fluo}{#1}}
\mdfsetup{skipabove=1pt,skipbelow=-2pt}



% Noeud dans un cadre pour les arbres
\newcommand{\noeud}[2]{\Tr{\fbox{\textcolor{#1}{\tt #2}}}}

\newcommand{\htmlmode}{\lstset{language=html,numbers=left, tabsize=4, frame=single, breaklines=true, keywordstyle=\ttfamily, basicstyle=\small,
   numberstyle=\tiny\ttfamily, framexleftmargin=0mm, backgroundcolor=\color{grispale}, xleftmargin=12mm,showstringspaces=false}}
\newcommand{\pythonmode}{\lstset{
   language=python,
   linewidth=\linewidth,
   numbers=left,
   tabsize=4,
   frame=single,
   breaklines=true,
   keywordstyle=\ttfamily\color{blue},
   basicstyle=\small,
   numberstyle=\tiny\ttfamily,
   framexleftmargin=-2mm,
   numbersep=-0.5mm,
   backgroundcolor=\color{codebg},
   xleftmargin=-1mm, 
   showstringspaces=false,
   commentstyle=\color{gray},
   stringstyle=\color{OliveGreen},
   emph={turtle,Screen,Turtle},
   emphstyle=\color{RawSienna},
   morekeywords={setheading,goto,backward,forward,left,right,pendown,penup,pensize,color,speed,hideturtle,showturtle,forward}}
   }
   \newcommand{\Cmode}{\lstset{
      language=[ANSI]C,
      linewidth=\linewidth,
      numbers=left,
      tabsize=4,
      frame=single,
      breaklines=true,
      keywordstyle=\ttfamily\color{blue},
      basicstyle=\small,
      numberstyle=\tiny\ttfamily,
      framexleftmargin=0mm,
      numbersep=2mm,
      backgroundcolor=\color{codebg},
      xleftmargin=0mm, 
      showstringspaces=false,
      commentstyle=\color{gray},
      stringstyle=\color{OliveGreen},
      emphstyle=\color{RawSienna},
      escapechar=\|,
      morekeywords={}}
      }
\newcommand{\bashmode}{\lstset{language=bash,numbers=left, tabsize=2, frame=single, breaklines=true, basicstyle=\ttfamily,
   numberstyle=\tiny\ttfamily, framexleftmargin=0mm, backgroundcolor=\color{grispale}, xleftmargin=12mm, showstringspaces=false}}
\newcommand{\exomode}{\lstset{language=python,numbers=left, tabsize=2, frame=single, breaklines=true, basicstyle=\ttfamily,
   numberstyle=\tiny\ttfamily, framexleftmargin=13mm, xleftmargin=12mm, basicstyle=\small, showstringspaces=false}}
   
   
  
%tei pour placer les images
%tei{nom de l’image}{échelle de l’image}{sens}{texte a positionner}
%sens ="1" (droite) ou "2" (gauche)
\newlength{\ltxt}
\newcommand{\tei}[4]{
\setlength{\ltxt}{\linewidth}
\setbox0=\hbox{\includegraphics[scale=#2]{#1}}
\addtolength{\ltxt}{-\wd0}
\addtolength{\ltxt}{-10pt}
\ifthenelse{\equal{#3}{1}}{
\begin{minipage}{\wd0}
\includegraphics[scale=#2]{#1}
\end{minipage}
\hfill
\begin{minipage}{\ltxt}
#4
\end{minipage}
}{
\begin{minipage}{\ltxt}
#4
\end{minipage}
\hfill
\begin{minipage}{\wd0}
\includegraphics[scale=#2]{#1}
\end{minipage}
}
}

%Juxtaposition d'une image pspciture et de texte 
%#1: = code pstricks de l'image
%#2: largeur de l'image
%#3: hauteur de l'image
%#4: Texte à écrire
\newcommand{\ptp}[4]{
\setlength{\ltxt}{\linewidth}
\addtolength{\ltxt}{-#2 cm}
\addtolength{\ltxt}{-0.1 cm}
\begin{minipage}[b][#3 cm][t]{\ltxt}
#4
\end{minipage}\hfill
\begin{minipage}[b][#3 cm][c]{#2 cm}
#1
\end{minipage}\par
}



%Macros pour les graphiques
\psset{linewidth=0.5\pslinewidth,PointSymbol=x}
\setlength{\fboxrule}{0.5pt}
\newcounter{tempangle}

%Marque la longueur du segment d'extrémité  #1 et  #2 avec la valeur #3, #4 est la distance par rapport au segment (en %age de la valeur de celui ci) et #5 l'orientation du marquage : +90 ou -90
\newcommand{\afflong}[5]{
\pstRotation[RotAngle=#4,PointSymbol=none,PointName=none]{#1}{#2}[X] 
\pstHomO[PointSymbol=none,PointName=none,HomCoef=#5]{#1}{X}[Y]
\pstTranslation[PointSymbol=none,PointName=none]{#1}{#2}{Y}[Z]
 \ncline{|<->|,linewidth=0.25\pslinewidth}{Y}{Z} \ncput*[nrot=:U]{\footnotesize{#3}}
}
\newcommand{\afflongb}[3]{
\ncline{|<->|,linewidth=0}{#1}{#2} \naput*[nrot=:U]{\footnotesize{#3}}
}

%Construis le point #4 situé à #2 cm du point #1 avant un angle #3 par rapport à l'horizontale. #5 = liste de paramètre
\newcommand{\lsegment}[5]{\pstGeonode[PointSymbol=none,PointName=none](0,0){O'}(#2,0){I'} \pstTranslation[PointSymbol=none,PointName=none]{O'}{I'}{#1}[J'] \pstRotation[RotAngle=#3,PointSymbol=x,#5]{#1}{J'}[#4]}
\newcommand{\tsegment}[5]{\pstGeonode[PointSymbol=none,PointName=none](0,0){O'}(#2,0){I'} \pstTranslation[PointSymbol=none,PointName=none]{O'}{I'}{#1}[J'] \pstRotation[RotAngle=#3,PointSymbol=x,#5]{#1}{J'}[#4] \pstLineAB{#4}{#1}}

%Construis le point #4 situé à #3 cm du point #1 et faisant un angle de  90° avec la droite (#1,#2) #5 = liste de paramètre
\newcommand{\psegment}[5]{
\pstGeonode[PointSymbol=none,PointName=none](0,0){O'}(#3,0){I'}
 \pstTranslation[PointSymbol=none,PointName=none]{O'}{I'}{#1}[J']
 \pstInterLC[PointSymbol=none,PointName=none]{#1}{#2}{#1}{J'}{M1}{M2} \pstRotation[RotAngle=-90,PointSymbol=x,#5]{#1}{M1}[#4]
  }
  
%Construis le point #4 situé à #3 cm du point #1 et faisant un angle de  #5° avec la droite (#1,#2) #6 = liste de paramètre
\newcommand{\mlogo}[6]{
\pstGeonode[PointSymbol=none,PointName=none](0,0){O'}(#3,0){I'}
 \pstTranslation[PointSymbol=none,PointName=none]{O'}{I'}{#1}[J']
 \pstInterLC[PointSymbol=none,PointName=none]{#1}{#2}{#1}{J'}{M1}{M2} \pstRotation[RotAngle=#5,PointSymbol=x,#6]{#1}{M2}[#4]
  }

% Construis un triangle avec #1=liste des 3 sommets séparés par des virgules, #2=liste des 3 longueurs séparés par des virgules, #3 et #4 : paramètre d'affichage des 2e et 3 points et #5 : inclinaison par rapport à l'horizontale
%autre macro identique mais sans tracer les segments joignant les sommets
\noexpandarg
\newcommand{\Triangleccc}[5]{
\StrBefore{#1}{,}[\pointA]
\StrBetween[1,2]{#1}{,}{,}[\pointB]
\StrBehind[2]{#1}{,}[\pointC]
\StrBefore{#2}{,}[\coteA]
\StrBetween[1,2]{#2}{,}{,}[\coteB]
\StrBehind[2]{#2}{,}[\coteC]
\tsegment{\pointA}{\coteA}{#5}{\pointB}{#3} 
\lsegment{\pointA}{\coteB}{0}{Z1}{PointSymbol=none, PointName=none}
\lsegment{\pointB}{\coteC}{0}{Z2}{PointSymbol=none, PointName=none}
\pstInterCC{\pointA}{Z1}{\pointB}{Z2}{\pointC}{Z3} 
\pstLineAB{\pointA}{\pointC} \pstLineAB{\pointB}{\pointC}
\pstSymO[PointName=\pointC,#4]{C}{C}[C]
}
\noexpandarg
\newcommand{\TrianglecccP}[5]{
\StrBefore{#1}{,}[\pointA]
\StrBetween[1,2]{#1}{,}{,}[\pointB]
\StrBehind[2]{#1}{,}[\pointC]
\StrBefore{#2}{,}[\coteA]
\StrBetween[1,2]{#2}{,}{,}[\coteB]
\StrBehind[2]{#2}{,}[\coteC]
\tsegment{\pointA}{\coteA}{#5}{\pointB}{#3} 
\lsegment{\pointA}{\coteB}{0}{Z1}{PointSymbol=none, PointName=none}
\lsegment{\pointB}{\coteC}{0}{Z2}{PointSymbol=none, PointName=none}
\pstInterCC[PointNameB=none,PointSymbolB=none,#4]{\pointA}{Z1}{\pointB}{Z2}{\pointC}{Z1} 
}


% Construis un triangle avec #1=liste des 3 sommets séparés par des virgules, #2=liste formée de 2 longueurs et d'un angle séparés par des virgules, #3 et #4 : paramètre d'affichage des 2e et 3 points et #5 : inclinaison par rapport à l'horizontale
%autre macro identique mais sans tracer les segments joignant les sommets
\newcommand{\Trianglecca}[5]{
\StrBefore{#1}{,}[\pointA]
\StrBetween[1,2]{#1}{,}{,}[\pointB]
\StrBehind[2]{#1}{,}[\pointC]
\StrBefore{#2}{,}[\coteA]
\StrBetween[1,2]{#2}{,}{,}[\coteB]
\StrBehind[2]{#2}{,}[\angleA]
\tsegment{\pointA}{\coteA}{#5}{\pointB}{#3} 
\setcounter{tempangle}{#5}
\addtocounter{tempangle}{\angleA}
\tsegment{\pointA}{\coteB}{\thetempangle}{\pointC}{#4}
\pstLineAB{\pointB}{\pointC}
}
\newcommand{\TriangleccaP}[5]{
\StrBefore{#1}{,}[\pointA]
\StrBetween[1,2]{#1}{,}{,}[\pointB]
\StrBehind[2]{#1}{,}[\pointC]
\StrBefore{#2}{,}[\coteA]
\StrBetween[1,2]{#2}{,}{,}[\coteB]
\StrBehind[2]{#2}{,}[\angleA]
\lsegment{\pointA}{\coteA}{#5}{\pointB}{#3} 
\setcounter{tempangle}{#5}
\addtocounter{tempangle}{\angleA}
\lsegment{\pointA}{\coteB}{\thetempangle}{\pointC}{#4}
}

% Construis un triangle avec #1=liste des 3 sommets séparés par des virgules, #2=liste formée de 1 longueurs et de deux angle séparés par des virgules, #3 et #4 : paramètre d'affichage des 2e et 3 points et #5 : inclinaison par rapport à l'horizontale
%autre macro identique mais sans tracer les segments joignant les sommets
\newcommand{\Trianglecaa}[5]{
\StrBefore{#1}{,}[\pointA]
\StrBetween[1,2]{#1}{,}{,}[\pointB]
\StrBehind[2]{#1}{,}[\pointC]
\StrBefore{#2}{,}[\coteA]
\StrBetween[1,2]{#2}{,}{,}[\angleA]
\StrBehind[2]{#2}{,}[\angleB]
\tsegment{\pointA}{\coteA}{#5}{\pointB}{#3} 
\setcounter{tempangle}{#5}
\addtocounter{tempangle}{\angleA}
\lsegment{\pointA}{1}{\thetempangle}{Z1}{PointSymbol=none, PointName=none}
\setcounter{tempangle}{#5}
\addtocounter{tempangle}{180}
\addtocounter{tempangle}{-\angleB}
\lsegment{\pointB}{1}{\thetempangle}{Z2}{PointSymbol=none, PointName=none}
\pstInterLL[#4]{\pointA}{Z1}{\pointB}{Z2}{\pointC}
\pstLineAB{\pointA}{\pointC}
\pstLineAB{\pointB}{\pointC}
}
\newcommand{\TrianglecaaP}[5]{
\StrBefore{#1}{,}[\pointA]
\StrBetween[1,2]{#1}{,}{,}[\pointB]
\StrBehind[2]{#1}{,}[\pointC]
\StrBefore{#2}{,}[\coteA]
\StrBetween[1,2]{#2}{,}{,}[\angleA]
\StrBehind[2]{#2}{,}[\angleB]
\lsegment{\pointA}{\coteA}{#5}{\pointB}{#3} 
\setcounter{tempangle}{#5}
\addtocounter{tempangle}{\angleA}
\lsegment{\pointA}{1}{\thetempangle}{Z1}{PointSymbol=none, PointName=none}
\setcounter{tempangle}{#5}
\addtocounter{tempangle}{180}
\addtocounter{tempangle}{-\angleB}
\lsegment{\pointB}{1}{\thetempangle}{Z2}{PointSymbol=none, PointName=none}
\pstInterLL[#4]{\pointA}{Z1}{\pointB}{Z2}{\pointC}
}

%Construction d'un cercle de centre #1 et de rayon #2 (en cm)
\newcommand{\Cercle}[2]{
\lsegment{#1}{#2}{0}{Z1}{PointSymbol=none, PointName=none}
\pstCircleOA{#1}{Z1}
}

%construction d'un parallélogramme #1 = liste des sommets, #2 = liste contenant les longueurs de 2 côtés consécutifs et leurs angles;  #3, #4 et #5 : paramètre d'affichage des sommets #6 inclinaison par rapport à l'horizontale 
% meme macro sans le tracé des segements
\newcommand{\Para}[6]{
\StrBefore{#1}{,}[\pointA]
\StrBetween[1,2]{#1}{,}{,}[\pointB]
\StrBetween[2,3]{#1}{,}{,}[\pointC]
\StrBehind[3]{#1}{,}[\pointD]
\StrBefore{#2}{,}[\longueur]
\StrBetween[1,2]{#2}{,}{,}[\largeur]
\StrBehind[2]{#2}{,}[\angle]
\tsegment{\pointA}{\longueur}{#6}{\pointB}{#3} 
\setcounter{tempangle}{#6}
\addtocounter{tempangle}{\angle}
\tsegment{\pointA}{\largeur}{\thetempangle}{\pointD}{#5}
\pstMiddleAB[PointName=none,PointSymbol=none]{\pointB}{\pointD}{Z1}
\pstSymO[#4]{Z1}{\pointA}[\pointC]
\pstLineAB{\pointB}{\pointC}
\pstLineAB{\pointC}{\pointD}
}
\newcommand{\ParaP}[6]{
\StrBefore{#1}{,}[\pointA]
\StrBetween[1,2]{#1}{,}{,}[\pointB]
\StrBetween[2,3]{#1}{,}{,}[\pointC]
\StrBehind[3]{#1}{,}[\pointD]
\StrBefore{#2}{,}[\longueur]
\StrBetween[1,2]{#2}{,}{,}[\largeur]
\StrBehind[2]{#2}{,}[\angle]
\lsegment{\pointA}{\longueur}{#6}{\pointB}{#3} 
\setcounter{tempangle}{#6}
\addtocounter{tempangle}{\angle}
\lsegment{\pointA}{\largeur}{\thetempangle}{\pointD}{#5}
\pstMiddleAB[PointName=none,PointSymbol=none]{\pointB}{\pointD}{Z1}
\pstSymO[#4]{Z1}{\pointA}[\pointC]
}


%construction d'un cerf-volant #1 = liste des sommets, #2 = liste contenant les longueurs de 2 côtés consécutifs et leurs angles;  #3, #4 et #5 : paramètre d'affichage des sommets #6 inclinaison par rapport à l'horizontale 
% meme macro sans le tracé des segements
\newcommand{\CerfVolant}[6]{
\StrBefore{#1}{,}[\pointA]
\StrBetween[1,2]{#1}{,}{,}[\pointB]
\StrBetween[2,3]{#1}{,}{,}[\pointC]
\StrBehind[3]{#1}{,}[\pointD]
\StrBefore{#2}{,}[\longueur]
\StrBetween[1,2]{#2}{,}{,}[\largeur]
\StrBehind[2]{#2}{,}[\angle]
\tsegment{\pointA}{\longueur}{#6}{\pointB}{#3} 
\setcounter{tempangle}{#6}
\addtocounter{tempangle}{\angle}
\tsegment{\pointA}{\largeur}{\thetempangle}{\pointD}{#5}
\pstOrtSym[#4]{\pointB}{\pointD}{\pointA}[\pointC]
\pstLineAB{\pointB}{\pointC}
\pstLineAB{\pointC}{\pointD}
}

%construction d'un quadrilatère quelconque #1 = liste des sommets, #2 = liste contenant les longueurs des 4 côtés et l'angle entre 2 cotés consécutifs  #3, #4 et #5 : paramètre d'affichage des sommets #6 inclinaison par rapport à l'horizontale 
% meme macro sans le tracé des segements
\newcommand{\Quadri}[6]{
\StrBefore{#1}{,}[\pointA]
\StrBetween[1,2]{#1}{,}{,}[\pointB]
\StrBetween[2,3]{#1}{,}{,}[\pointC]
\StrBehind[3]{#1}{,}[\pointD]
\StrBefore{#2}{,}[\coteA]
\StrBetween[1,2]{#2}{,}{,}[\coteB]
\StrBetween[2,3]{#2}{,}{,}[\coteC]
\StrBetween[3,4]{#2}{,}{,}[\coteD]
\StrBehind[4]{#2}{,}[\angle]
\tsegment{\pointA}{\coteA}{#6}{\pointB}{#3} 
\setcounter{tempangle}{#6}
\addtocounter{tempangle}{\angle}
\tsegment{\pointA}{\coteD}{\thetempangle}{\pointD}{#5}
\lsegment{\pointB}{\coteB}{0}{Z1}{PointSymbol=none, PointName=none}
\lsegment{\pointD}{\coteC}{0}{Z2}{PointSymbol=none, PointName=none}
\pstInterCC[PointNameA=none,PointSymbolA=none,#4]{\pointB}{Z1}{\pointD}{Z2}{Z3}{\pointC} 
\pstLineAB{\pointB}{\pointC}
\pstLineAB{\pointC}{\pointD}
}


% Définition des colonnes centrées ou à droite pour tabularx
\newcolumntype{Y}{>{\centering\arraybackslash}X}
\newcolumntype{Z}{>{\flushright\arraybackslash}X}

%Les pointillés à remplir par les élèves
\newcommand{\po}[1]{\makebox[#1 cm]{\dotfill}}
\newcommand{\lpo}[1][3]{%
\multido{}{#1}{\makebox[\linewidth]{\dotfill}
}}

%Liste des pictogrammes utilisés sur la fiche d'exercice ou d'activités
\newcommand{\bombe}{\faBomb}
\newcommand{\livre}{\faBook}
\newcommand{\calculatrice}{\faCalculator}
\newcommand{\oral}{\faCommentO}
\newcommand{\surfeuille}{\faEdit}
\newcommand{\ordinateur}{\faLaptop}
\newcommand{\ordi}{\faDesktop}
\newcommand{\ciseaux}{\faScissors}
\newcommand{\danger}{\faExclamationTriangle}
\newcommand{\out}{\faSignOut}
\newcommand{\cadeau}{\faGift}
\newcommand{\flash}{\faBolt}
\newcommand{\lumiere}{\faLightbulb}
\newcommand{\compas}{\dsmathematical}
\newcommand{\calcullitteral}{\faTimesCircleO}
\newcommand{\raisonnement}{\faCogs}
\newcommand{\recherche}{\faSearch}
\newcommand{\rappel}{\faHistory}
\newcommand{\video}{\faFilm}
\newcommand{\capacite}{\faPuzzlePiece}
\newcommand{\aide}{\faLifeRing}
\newcommand{\loin}{\faExternalLink}
\newcommand{\groupe}{\faUsers}
\newcommand{\bac}{\faGraduationCap}
\newcommand{\histoire}{\faUniversity}
\newcommand{\coeur}{\faSave}
\newcommand{\python}{\faPython}
\newcommand{\os}{\faMicrochip}
\newcommand{\rd}{\faCubes}
\newcommand{\data}{\faColumns}
\newcommand{\web}{\faCode}
\newcommand{\prog}{\faFile}
\newcommand{\algo}{\faCogs}
\newcommand{\important}{\faExclamationCircle}
\newcommand{\maths}{\faTimesCircle}
% Traitement des données en tables
\newcommand{\tables}{\faColumns}
% Types construits
\newcommand{\construits}{\faCubes}
% Type et valeurs de base
\newcommand{\debase}{{\footnotesize \faCube}}
% Systèmes d'exploitation
\newcommand{\linux}{\faLinux}
\newcommand{\sd}{\faProjectDiagram}
\newcommand{\bd}{\faDatabase}

%Les ensembles de nombres
\renewcommand{\N}{\mathbb{N}}
\newcommand{\D}{\mathbb{D}}
\newcommand{\Z}{\mathbb{Z}}
\newcommand{\Q}{\mathbb{Q}}
\newcommand{\R}{\mathbb{R}}
\newcommand{\C}{\mathbb{C}}

%Ecriture des vecteurs
\newcommand{\vect}[1]{\vbox{\halign{##\cr 
  \tiny\rightarrowfill\cr\noalign{\nointerlineskip\vskip1pt} 
  $#1\mskip2mu$\cr}}}


%Compteur activités/exos et question et mise en forme titre et questions
\newcounter{numact}
\setcounter{numact}{1}
\newcounter{numseance}
\setcounter{numseance}{1}
\newcounter{numexo}
\setcounter{numexo}{0}
\newcounter{numprojet}
\setcounter{numprojet}{0}
\newcounter{numquestion}
\newcommand{\espace}[1]{\rule[-1ex]{0pt}{#1 cm}}
\newcommand{\Quest}[3]{
\addtocounter{numquestion}{1}
\begin{tabularx}{\textwidth}{X|m{1cm}|}
\cline{2-2}
\textbf{\sffamily{\alph{numquestion})}} #1 & \dots / #2 \\
\hline 
\multicolumn{2}{|l|}{\espace{#3}} \\
\hline
\end{tabularx}
}
\newcommand{\QuestR}[3]{
\addtocounter{numquestion}{1}
\begin{tabularx}{\textwidth}{X|m{1cm}|}
\cline{2-2}
\textbf{\sffamily{\alph{numquestion})}} #1 & \dots / #2 \\
\hline 
\multicolumn{2}{|l|}{\cor{#3}} \\
\hline
\end{tabularx}
}
\newcommand{\Pre}{{\sc nsi} 1\textsuperscript{e}}
\newcommand{\Term}{{\sc nsi} Terminale}
\newcommand{\Sec}{2\textsuperscript{e}}
\newcommand{\Exo}[2]{ \addtocounter{numexo}{1} \ding{113} \textbf{\sffamily{Exercice \thenumexo}} : \textit{#1} \hfill #2  \setcounter{numquestion}{0}}
\newcommand{\Projet}[1]{ \addtocounter{numprojet}{1} \ding{118} \textbf{\sffamily{Projet \thenumprojet}} : \textit{#1}}
\newcommand{\ExoD}[2]{ \addtocounter{numexo}{1} \ding{113} \textbf{\sffamily{Exercice \thenumexo}}  \textit{(#1 pts)} \hfill #2  \setcounter{numquestion}{0}}
\newcommand{\ExoB}[2]{ \addtocounter{numexo}{1} \ding{113} \textbf{\sffamily{Exercice \thenumexo}}  \textit{(Bonus de +#1 pts maximum)} \hfill #2  \setcounter{numquestion}{0}}
\newcommand{\Act}[2]{ \ding{113} \textbf{\sffamily{Activité \thenumact}} : \textit{#1} \hfill #2  \addtocounter{numact}{1} \setcounter{numquestion}{0}}
\newcommand{\Seance}{ \rule{1.5cm}{0.5pt}\raisebox{-3pt}{\framebox[4cm]{\textbf{\sffamily{Séance \thenumseance}}}}\hrulefill  \\
  \addtocounter{numseance}{1}}
\newcommand{\Acti}[2]{ {\footnotesize \ding{117}} \textbf{\sffamily{Activité \thenumact}} : \textit{#1} \hfill #2  \addtocounter{numact}{1} \setcounter{numquestion}{0}}
\newcommand{\titre}[1]{\begin{Large}\textbf{\ding{118}}\end{Large} \begin{large}\textbf{ #1}\end{large} \vspace{0.2cm}}
\newcommand{\QListe}[1][0]{
\ifthenelse{#1=0}
{\begin{enumerate}[partopsep=0pt,topsep=0pt,parsep=0pt,itemsep=0pt,label=\textbf{\sffamily{\arabic*.}},series=question]}
{\begin{enumerate}[resume*=question]}}
\newcommand{\SQListe}[1][0]{
\ifthenelse{#1=0}
{\begin{enumerate}[partopsep=0pt,topsep=0pt,parsep=0pt,itemsep=0pt,label=\textbf{\sffamily{\alph*)}},series=squestion]}
{\begin{enumerate}[resume*=squestion]}}
\newcommand{\SQListeL}[1][0]{
\ifthenelse{#1=0}
{\begin{enumerate*}[partopsep=0pt,topsep=0pt,parsep=0pt,itemsep=0pt,label=\textbf{\sffamily{\alph*)}},series=squestion]}
{\begin{enumerate*}[resume*=squestion]}}
\newcommand{\FinListe}{\end{enumerate}}
\newcommand{\FinListeL}{\end{enumerate*}}

%Mise en forme de la correction
\newcommand{\cor}[1]{\par \textcolor{OliveGreen}{#1}}
\newcommand{\br}[1]{\cor{\textbf{#1}}}
\newcommand{\tcor}[1]{\begin{tcolorbox}[width=0.92\textwidth,colback={white},colbacktitle=white,coltitle=OliveGreen,colframe=green!75!black,boxrule=0.2mm]   
\cor{#1}
\end{tcolorbox}
}
\newcommand{\rc}[1]{\textcolor{OliveGreen}{#1}}
\newcommand{\pmc}[1]{\textcolor{blue}{\tt #1}}
\newcommand{\tmc}[1]{\textcolor{RawSienna}{\tt #1}}


%Référence aux exercices par leur numéro
\newcommand{\refexo}[1]{
\refstepcounter{numexo}
\addtocounter{numexo}{-1}
\label{#1}}

%Séparation entre deux activités
\newcommand{\separateur}{\begin{center}
\rule{1.5cm}{0.5pt}\raisebox{-3pt}{\ding{117}}\rule{1.5cm}{0.5pt}  \vspace{0.2cm}
\end{center}}

%Entête et pied de page
\newcommand{\snt}[1]{\lhead{\textbf{SNT -- La photographie numérique} \rhead{\textit{Lycée Nord}}}}
\newcommand{\Activites}[2]{\lhead{\textbf{{\sc #1}}}
\rhead{Activités -- \textbf{#2}}
\cfoot{}}
\newcommand{\Exos}[2]{\lhead{\textbf{Fiche d'exercices: {\sc #1}}}
\rhead{Niveau: \textbf{#2}}
\cfoot{}}
\newcommand{\Devoir}[2]{\lhead{\textbf{Devoir de mathématiques : {\sc #1}}}
\rhead{\textbf{#2}} \setlength{\fboxsep}{8pt}
\begin{center}
%Titre de la fiche
\fbox{\parbox[b][1cm][t]{0.3\textwidth}{Nom : \hfill \po{3} \par \vfill Prénom : \hfill \po{3}} } \hfill 
\fbox{\parbox[b][1cm][t]{0.6\textwidth}{Note : \po{1} / 20} }
\end{center} \cfoot{}}

%Devoir programmation en NSI (pas à rendre sur papier)
\newcommand{\PNSI}[2]{\lhead{\textbf{Devoir de {\sc nsi} : \textsf{ #1}}}
\rhead{\textbf{#2}} \setlength{\fboxsep}{8pt}
\begin{tcolorbox}[title=\textcolor{black}{\danger\; A lire attentivement},colbacktitle=lightgray]
{\begin{enumerate}
\item Rendre tous vous programmes en les envoyant par mail à l'adresse {\tt fnativel2@ac-reunion.fr}, en précisant bien dans le sujet vos noms et prénoms
\item Un programme qui fonctionne mal ou pas du tout peut rapporter des points
\item Les bonnes pratiques de programmation (clarté et lisiblité du code) rapportent des points
\end{enumerate}
}
\end{tcolorbox}
 \cfoot{}}


%Devoir de NSI
\newcommand{\DNSI}[2]{\lhead{\textbf{Devoir de {\sc nsi} : \textsf{ #1}}}
\rhead{\textbf{#2}} \setlength{\fboxsep}{8pt}
\begin{center}
%Titre de la fiche
\fbox{\parbox[b][1cm][t]{0.3\textwidth}{Nom : \hfill \po{3} \par \vfill Prénom : \hfill \po{3}} } \hfill 
\fbox{\parbox[b][1cm][t]{0.6\textwidth}{Note : \po{1} / 10} }
\end{center} \cfoot{}}

\newcommand{\DevoirNSI}[2]{\lhead{\textbf{Devoir de {\sc nsi} : {\sc #1}}}
\rhead{\textbf{#2}} \setlength{\fboxsep}{8pt}
\cfoot{}}

%La définition de la commande QCM pour auto-multiple-choice
%En premier argument le sujet du qcm, deuxième argument : la classe, 3e : la durée prévue et #4 : présence ou non de questions avec plusieurs bonnes réponses
\newcommand{\QCM}[4]{
{\large \textbf{\ding{52} QCM : #1}} -- Durée : \textbf{#3 min} \hfill {\large Note : \dots/10} 
\hrule \vspace{0.1cm}\namefield{}
Nom :  \textbf{\textbf{\nom{}}} \qquad \qquad Prénom :  \textbf{\prenom{}}  \hfill Classe: \textbf{#2}
\vspace{0.2cm}
\hrule  
\begin{itemize}[itemsep=0pt]
\item[-] \textit{Une bonne réponse vaut un point, une absence de réponse n'enlève pas de point. }
\item[\danger] \textit{Une mauvaise réponse enlève un point.}
\ifthenelse{#4=1}{\item[-] \textit{Les questions marquées du symbole \multiSymbole{} peuvent avoir plusieurs bonnes réponses possibles.}}{}
\end{itemize}
}
\newcommand{\DevoirC}[2]{
\renewcommand{\footrulewidth}{0.5pt}
\lhead{\textbf{Devoir de mathématiques : {\sc #1}}}
\rhead{\textbf{#2}} \setlength{\fboxsep}{8pt}
\fbox{\parbox[b][0.4cm][t]{0.955\textwidth}{Nom : \po{5} \hfill Prénom : \po{5} \hfill Classe: \textbf{1}\textsuperscript{$\dots$}} } 
\rfoot{\thepage} \cfoot{} \lfoot{Lycée Nord}}
\newcommand{\DevoirInfo}[2]{\lhead{\textbf{Evaluation : {\sc #1}}}
\rhead{\textbf{#2}} \setlength{\fboxsep}{8pt}
 \cfoot{}}
\newcommand{\DM}[2]{\lhead{\textbf{Devoir maison à rendre le #1}} \rhead{\textbf{#2}}}

%Macros permettant l'affichage des touches de la calculatrice
%Touches classiques : #1 = 0 fond blanc pour les nombres et #1= 1gris pour les opérations et entrer, second paramètre=contenu
%Si #2=1 touche arrondi avec fond gris
\newcommand{\TCalc}[2]{
\setlength{\fboxsep}{0.1pt}
\ifthenelse{#1=0}
{\psframebox[fillstyle=solid, fillcolor=white]{\parbox[c][0.25cm][c]{0.6cm}{\centering #2}}}
{\ifthenelse{#1=1}
{\psframebox[fillstyle=solid, fillcolor=lightgray]{\parbox[c][0.25cm][c]{0.6cm}{\centering #2}}}
{\psframebox[framearc=.5,fillstyle=solid, fillcolor=white]{\parbox[c][0.25cm][c]{0.6cm}{\centering #2}}}
}}
\newcommand{\Talpha}{\psdblframebox[fillstyle=solid, fillcolor=white]{\hspace{-0.05cm}\parbox[c][0.25cm][c]{0.65cm}{\centering \scriptsize{alpha}}} \;}
\newcommand{\Tsec}{\psdblframebox[fillstyle=solid, fillcolor=white]{\parbox[c][0.25cm][c]{0.6cm}{\centering \scriptsize 2nde}} \;}
\newcommand{\Tfx}{\psdblframebox[fillstyle=solid, fillcolor=white]{\parbox[c][0.25cm][c]{0.6cm}{\centering \scriptsize $f(x)$}} \;}
\newcommand{\Tvar}{\psframebox[framearc=.5,fillstyle=solid, fillcolor=white]{\hspace{-0.22cm} \parbox[c][0.25cm][c]{0.82cm}{$\scriptscriptstyle{X,T,\theta,n}$}}}
\newcommand{\Tgraphe}{\psdblframebox[fillstyle=solid, fillcolor=white]{\hspace{-0.08cm}\parbox[c][0.25cm][c]{0.68cm}{\centering \tiny{graphe}}} \;}
\newcommand{\Tfen}{\psdblframebox[fillstyle=solid, fillcolor=white]{\hspace{-0.08cm}\parbox[c][0.25cm][c]{0.68cm}{\centering \tiny{fenêtre}}} \;}
\newcommand{\Ttrace}{\psdblframebox[fillstyle=solid, fillcolor=white]{\parbox[c][0.25cm][c]{0.6cm}{\centering \scriptsize{trace}}} \;}

% Macroi pour l'affichage  d'un entier n dans  une base b
\newcommand{\base}[2]{ \overline{#1}^{#2}}
% Intervalle d'entiers
\newcommand{\intN}[2]{\llbracket #1; #2 \rrbracket}}

% Numéro et titre de chapitre
\setcounter{numchap}{6}
\newcommand{\Ctitle}{\cnum {OCaml : aspects fonctionnels}}
\newcommand{\SPATH}{/home/fenarius/Travail/Cours/cpge-info/docs/mp2i/files/C\thenumchap/}

% Définition d'une structure de données
\makess{Généralités}
\begin{frame}{\Ctitle}{\stitle}
	\begin{block}{Bref historique}
		\begin{itemize}
			\item<1-> Années 70 : développement du langage de preuve de programme  {\sc ml} (Meta Language) (R. Milner).
			\item<2-> 1985 : première implémentation de Caml (Categorical Abstract Machine Language) à l'{\sc inria} (organisme de recherche français).
			\item<3-> 1996 : première version de OCaml (Objective Caml) par X. Leroy.
			\item<4-> 2005 : Première version de F\#, variante de OCaml développé par Microsoft.
			\item<5-> 2016 : Première version de Reason, variant de Ocaml développé par Facebook.
			\item<6-> 2022 : OCaml version 5.0.0
		\end{itemize}
	\end{block}
\end{frame}

\begin{frame}{\Ctitle}{\stitle}
	\begin{block}{Quelques caractéristiques}
		\begin{itemize}
			\item<1-> OCaml est un langage de \textcolor{blue}{programmation fonctionnel}, les fonctions sont au coeur de ce paradigme de programmation.
			\item<2-> Les variables sont \textit{non modifiables} en conséquence la \textcolor{blue}{récursivité} est fondamentale car l'écriture de boucle devient impossible. La motivation est de produire un code plus lisible, plus facile à maintenir et moins sujet aux bugs.
			\item<3-> OCaml est \textcolor{blue}{typé statiquement}, une variable ne peut pas changer de type au cours de l'exécution. De plus les erreurs de type seront systématiquement détectées à la compilation.
			\item<4-> Le type des variables n'a pas besoin d'être précisé, il sera automatiquement détecté par le compilateur grâce à un procédé appelé \textcolor{blue}{inférence de type}.
			\item<5-> Ocaml est un langage \textcolor{blue}{compilé}, cependant un environnement interactif \kw{utop} est disponible.
			\item<6-> La gestion de la mémoire est automatique  (via un \textcolor{blue}{ramasse-miettes} \textit{garbage collector}).
		\end{itemize}
	\end{block}
\end{frame}


\makess{Quelques exemples}
\begin{frame}{\Ctitle}{\stitle}
	\begin{exampleblock}{Fonctions sur les entiers et les flottants}
		\begin{enumerate}
			\item<1-> Fonction qui pour un entier $n$, renvoie $n(n-1)$.
				\inputpartOCaml{\SPATH ex_cours.ml}{}{\small}{1}{1}
				\begin{itemize}
					\item<2->{\textcolor{BrickRed}{\danger} Dans le paradigme fonctionnel, on écrit des \textcolor{blue}{expressions} et pas des \textcolor{blue}{instructions} (paradigme impératif). Contrairement à une instruction, une expression est évaluée et renvoie toujours une valeur.}
					\item<3->{ On remarquera la proximité avec  $f : n \mapsto n(n-1)$ (et l'absence de  \kw{return}).}
					\item<4->{ Les opérateurs \kw{*}, \kw{-} portent sur les entiers et permettent d'inférer le type de l'argument et du résultat.}
					\item<5->{ Pour calculer $f(5)$ : \kw{f 5 ;;}}
				\end{itemize}
			\item<6-> Fonction qui pour un flottant $x$, renvoie $x^2-3x+7$.
				\inputpartOCaml{\SPATH ex_cours.ml}{}{\small}{3}{3}
				\begin{itemize}
					\item<7-> Les opérateurs \kw{+.}, \kw{*.} et \kw{-.} concernent les flottants.
					\item<8-> L'exponentiation est \kw{**} (sur les flottants).
				\end{itemize}
		\end{enumerate}
	\end{exampleblock}
\end{frame}



\begin{frame}{\Ctitle}{\stitle}
	\begin{exampleblock}{Polymorphisme, booléens, conditionnelle}
		\begin{enumerate}
			\item<1-> Fonction qui renvoie \kw{true} lorsque deux  des trois arguments sont égaux.
				\inputpartOCaml{\SPATH ex_cours.ml}{}{\small}{5}{5}
				\begin{itemize}
					\item<2-> Les opérateurs logiques sont \kw{\&\&}, \kw{||} et \kw{not}.
					\item<3-> Ici l'inférence de type ne permet pas de déterminer le type des arguments, on dit que la fonction est \textcolor{BrickRed}{polymorphe}.
					\item<4-> L'appel s'effectue en donnant les paramètres séparés par des espaces : \kw{deux\_egaux 4 6 4;;}
				\end{itemize}
			\item<4-> Terme suivant de la suite de syracuse :
				\inputpartOCaml{\SPATH ex_cours.ml}{}{\small}{7}{8}
				\begin{itemize}
					\item<5-> On notera la construction \kw{if} \dots \kw{then} \dots \kw{else}.
					\item<6-> Attention, le test d'égalité est le \kw{=} simple.
					\item<7-> Le modulo s'obtient avec \kw{mod}.
				\end{itemize}
		\end{enumerate}
	\end{exampleblock}
\end{frame}

\begin{frame}{\Ctitle}{\stitle}
	\begin{exampleblock}{Récursivité exemple 1}
		Fonction qui calcule la somme des $n$ premiers carrés.
		\inputpartOCaml{\SPATH ex_cours.ml}{}{\small}{10}{11}
		\begin{itemize}
			\item<2-> On notera la mot clé \kw{rec} pour préciser que la fonction est récursive.
			\item<3-> Les parenthèses autour de \kw{n-1} permettent d'éviter la confusion avec \kw{(somme carre n)-1} (et donc d'avoir une récursion infinie)
		\end{itemize}
	\end{exampleblock}
\end{frame}

\begin{frame}{\Ctitle}{\stitle}
	\begin{exampleblock}{Récursivité exemple 2}
		Fonction qui compte à rebours depuis $n$ et affiche "Partez" !
		\inputpartOCaml{\SPATH ex_cours.ml}{}{\small}{19}{23}
		\begin{itemize}
			\item<2-> Regroupement de la clause du \kw{else} dans un bloc délimité par \kw{(} \kw{)}. \\
				\textcolor{gray}{On peut de façon équivalent délimiter par \kw{begin} et \kw{end}}
			\item<3-> Les résultats des affichages, sont aussi des expressions et donc renvoient une valeur. Ce sont des valeurs  de type \kw{unit} (et on tous la même valeur : \kw{()}).
			\item<4-> Les \kw{;} séparent les différents affichages, ils permettent d'ignorer la valeur renvoyée par les différents affichages. L'évaluation de l'expression se poursuit donc jusqu'à l'appel récursif.
		\end{itemize}
	\end{exampleblock}
\end{frame}

\begin{frame}{\Ctitle}{\stitle}
	\begin{block}{\textcolor{yellow}{\danger} A retenir !}
		\begin{itemize}
			\item<1-> Un programme OCaml consiste en l'évaluation d'une ou plusieurs expressions, \textbf{il n'y a pas d'instructions}.\\
				\onslide<2->\textcolor{gray}{Une structure \kw{if} \dots \kw{then} \dots \kw{else} est évaluée et renvoie une expression. En impératif, ce sont des instructions conditionnelles.}
			\item<2-> Les types n'ont pas a être spécifiés, ils sont déterminés automatiquement par le mécanisme d'inférence.
			\item<3-> Toute expression a une valeur, la détermination de cette valeur s'appelle l'évaluation.
		\end{itemize}
	\end{block}
\end{frame}

\makess{Définitions et types de base}
\begin{frame}{\Ctitle}{\stitle}
	\begin{alertblock}{Types de base}
		\begin{tabularx}{\linewidth}{|l|c|>{\footnotesize}X|}
			\hline
			Type     & Opérations                                         & Commentaires                                                       \\
			\hline
			\kw{int} & \kw{+}, \kw{-}, \kw{*}, \kw{/}, \kw{mod}, \kw{abs} & Entiers signés sur 64 bits valeurs dans $\intN{-2^{62}}{2^{62}-1}$ \\
			\hline
			         &                                                    & \ \newline                                                         \\
			\hline
			         &                                                    &                                                                    \\
			\hline
			         &                                                    & \  \newline                                                        \\
			\hline
			         &                                                    & \ \newline                                                         \\
			\hline
		\end{tabularx}
		\vspace{1cm}
	\end{alertblock}
\end{frame}

\begin{frame}{\Ctitle}{\stitle}
	\begin{alertblock}{Types de base}
		\begin{tabularx}{\linewidth}{|l|c|>{\footnotesize}X|}
			\hline
			Type       & Opérations                                         & Commentaires                                                                                                                \\
			\hline
			\kw{int}   & \kw{+}, \kw{-}, \kw{*}, \kw{/}, \kw{mod}, \kw{abs} & Entiers signés sur 64 bits valeurs dans $\intN{-2^{62}}{2^{62}-1}$                                                          \\
			\hline
			\kw{float} & \kw{+.}, \kw{-.}, \kw{*.}, \kw{/.}, \kw{**}        & Correspond au type double de la norme {\sc ieee-754}. \newline Fonctions mathématiques usuelles ($\sin, \cos, \exp, \dots$) \\
			\hline
			           &                                                    &                                                                                                                             \\
			\hline
			           &                                                    & \  \newline                                                                                                                 \\
			\hline
			           &                                                    & \ \newline                                                                                                                  \\
			\hline
		\end{tabularx}
		\vspace{1cm}
	\end{alertblock}
\end{frame}

\begin{frame}{\Ctitle}{\stitle}
	\begin{alertblock}{Types de base}
		\begin{tabularx}{\linewidth}{|l|c|>{\footnotesize}X|}
			\hline
			Type       & Opérations                                         & Commentaires                                                                                                                \\
			\hline
			\kw{int}   & \kw{+}, \kw{-}, \kw{*}, \kw{/}, \kw{mod}, \kw{abs} & Entiers signés sur 64 bits valeurs dans $\intN{-2^{62}}{2^{62}-1}$                                                          \\
			\hline
			\kw{float} & \kw{+.}, \kw{-.}, \kw{*.}, \kw{/.}, \kw{**}        & Correspond au type double de la norme {\sc ieee-754}. \newline Fonctions mathématiques usuelles ($\sin, \cos, \exp, \dots$) \\
			\hline
			\kw{bool}  & \kw{\&\&}, \kw{||}, \kw{not}                       & Evaluations paresseuses.                                                                                                    \\
			\hline
			           &                                                    & \  \newline                                                                                                                 \\
			\hline
			           &                                                    & \ \newline                                                                                                                  \\
			\hline
		\end{tabularx}
		\vspace{1cm}
	\end{alertblock}
\end{frame}


\begin{frame}{\Ctitle}{\stitle}
	\begin{alertblock}{Types de base}
		\begin{tabularx}{\linewidth}{|l|c|>{\footnotesize}X|}
			\hline
			Type       & Opérations                                         & Commentaires                                                                                                                \\
			\hline
			\kw{int}   & \kw{+}, \kw{-}, \kw{*}, \kw{/}, \kw{mod}, \kw{abs} & Entiers signés sur 64 bits valeurs dans $\intN{-2^{62}}{2^{62}-1}$                                                          \\
			\hline
			\kw{float} & \kw{+.}, \kw{-.}, \kw{*.}, \kw{/.}, \kw{**}        & Correspond au type double de la norme {\sc ieee-754}. \newline Fonctions mathématiques usuelles ($\sin, \cos, \exp, \dots$) \\
			\hline
			\kw{bool}  & \kw{\&\&}, \kw{||}, \kw{not}                       & Evaluations paresseuses.                                                                                                    \\
			\hline
			\kw{char}  &                                                    & Se note entre apostrophe (\kw{'}). Les \kw{char} sont comparables (ordre du code {\sc ascii}).                              \\
			\hline
			           &                                                    & \ \newline                                                                                                                  \\
			\hline
		\end{tabularx}
		\vspace{1cm}
	\end{alertblock}
\end{frame}

\begin{frame}{\Ctitle}{\stitle}
	\begin{alertblock}{Types de base}
		\begin{tabularx}{\linewidth}{|l|c|>{\footnotesize}X|}
			\hline
			Type        & Opérations                                         & Commentaires                                                                                                                \\
			\hline
			\kw{int}    & \kw{+}, \kw{-}, \kw{*}, \kw{/}, \kw{mod}, \kw{abs} & Entiers signés sur 64 bits valeurs dans $\intN{-2^{62}}{2^{62}-1}$                                                          \\
			\hline
			\kw{float}  & \kw{+.}, \kw{-.}, \kw{*.}, \kw{/.}, \kw{**}        & Correspond au type double de la norme {\sc ieee-754}. \newline Fonctions mathématiques usuelles ($\sin, \cos, \exp, \dots$) \\
			\hline
			\kw{bool}   & \kw{\&\&}, \kw{||}, \kw{not}                       & Evaluations paresseuses.                                                                                                    \\
			\hline
			\kw{char}   &                                                    & Se note entre apostrophe (\kw{''}). Les \kw{char} sont comparables (ordre du code {\sc ascii}).                             \\
			\hline
			\kw{string} & \kw{\^{}}, \kw{.[]}, \kw{String.length}            & Immutabilité. Concaténation de deux chaines : \kw{"Bon"\^{}"jour"}. Accès au ième avec \kw{.[i]}                            \\
			\hline
		\end{tabularx}
		\vspace{1cm}
	\end{alertblock}
\end{frame}

\begin{frame}{\Ctitle}{\stitle}
	\begin{alertblock}{Types de base}
		\begin{tabularx}{\linewidth}{|l|c|>{\footnotesize}X|}
			\hline
			Type        & Opérations                                         & Commentaires                                                                                                                \\
			\hline
			\kw{int}    & \kw{+}, \kw{-}, \kw{*}, \kw{/}, \kw{mod}, \kw{abs} & Entiers signés sur 64 bits valeurs dans $\intN{-2^{62}}{2^{62}-1}$                                                          \\
			\hline
			\kw{float}  & \kw{+.}, \kw{-.}, \kw{*.}, \kw{/.}, \kw{**}        & Correspond au type double de la norme {\sc ieee-754}. \newline Fonctions mathématiques usuelles ($\sin, \cos, \exp, \dots$) \\
			\hline
			\kw{bool}   & \kw{\&\&}, \kw{||}, \kw{not}                       & Evaluations paresseuses.                                                                                                    \\
			\hline
			\kw{char}   &                                                    & Se note entre apostrophe (\kw{''}). Les \kw{char} sont comparables (ordre du code {\sc ascii}).                             \\
			\hline
			\kw{string} & \kw{\^{}}, \kw{.[]}, \kw{String.length}            & Immutabilité. Concaténation de deux chaines : \kw{"Bon"\^{}"jour"}. Accès au ième avec \kw{.[i]}                            \\
			\hline
		\end{tabularx}
		\begin{itemize}
			\item<1-> Le type \kw{unit} possède une seule valeur notée \kw{()} à rapprocher du type \kw{void} du C. Un affichage renvoie \kw{()}.
			\item<2-> Les opérateurs de comparaison (\kw{=}, \kw{<>}, \kw{>}, \kw{>=}, \kw{<}, \kw{<=}) sont polymorphes mais s'appliquent à deux objets \textit{de même type}.
		\end{itemize}
	\end{alertblock}
\end{frame}


\begin{frame}{\Ctitle}{\stitle}
	\begin{block}{Conversion de types}
		\begin{center}
			\renewcommand{\arraystretch}{3}
			\begin{tabular}{p{2.cm}p{3cm}p{3cm}}
				                                                                                                  & \Rnode{int}{\begin{rcadre}{lightgray}{Sepia}{2}{0.8} \textcolor{Sepia}{\tt int} \end{rcadre}} & \Rnode{char}{\begin{rcadre}{lightgray}{Sepia}{2}{0.8} \textcolor{Sepia}{\tt char} \end{rcadre}}     \\
				\Rnode{float}{\begin{rcadre}{lightgray}{Sepia}{2}{0.8} \textcolor{Sepia}{\tt float} \end{rcadre}} &                                                                                               & \Rnode{string}{\begin{rcadre}{lightgray}{Sepia}{2}{0.8} \textcolor{Sepia}{\tt string} \end{rcadre}} \\
				                                                                                                  &                                                                                               & \Rnode{bool}{\begin{rcadre}{lightgray}{Sepia}{2}{0.8} \textcolor{Sepia}{\tt bool} \end{rcadre}}     \\
			\end{tabular}
			\onslide<2->{\ncline[offset=-0.2cm,linecolor=blue,linewidth=0.7pt]{->}{string}{float}
				\ncline[offset=0.4cm,linecolor=blue,linewidth=0.7pt]{->}{float}{string}}
			\onslide<3->{\ncline[offset=-0.2cm,nodesepA=0.2cm,linecolor=blue,linewidth=0.7pt]{->}{int}{float}
				\ncline[linecolor=blue,linewidth=0.7pt]{->}{float}{int} }
			\onslide<4->{\ncline[offset=0.2cm,nodesepA=0.2cm,linecolor=blue,linewidth=0.7pt]{->}{int}{string}
				\ncline[linecolor=blue,linewidth=0.7pt,nodesepA=0.2cm]{->}{string}{int} }
			\onslide<5->{\ncline[offset=0.1cm,linecolor=blue,linewidth=0.7pt]{->}{bool}{string}
				\ncline[offset=0.1cm,linecolor=blue,linewidth=0.7pt]{->}{string}{bool} }
			\onslide<6->{\ncline[offset=-0.2cm,linecolor=blue,linewidth=0.7pt,nodesep=0.5pt]{->}{char}{int}
				\ncline[offset=0.4cm,linecolor=blue,linewidth=0.7pt]{->}{int}{char}}
		\end{center}
		\begin{itemize}
			\item<7-> Les fonctions de conversion sont de la forme \kw{<type1>\_of\_<type2>} par exemple, \kw{string\_of\_float}.
			\item<8-> L'affichage s'obtient avec \kw{print\_<type>} par exemple \kw{print\_string} (excepté pour les booléens).
		\end{itemize}
	\end{block}
\end{frame}

\makess{Expressions conditionnelles}
\begin{frame}[fragile]{\Ctitle}{\stitle}
	\begin{block}{Syntaxe et évaluation}
		\begin{itemize}
		\item<1-> En Ocaml, on parlera d'\textcolor{blue}{expressions conditionnelles} (et pas d'instructions conditonnelles).
		\item<2-> La syntaxe d'une expression conditionnelle est :
		\begin{minted}{ocaml}
if expr_bool then expr1 else expr2;;
		\end{minted}
		\item<3-> Cette expression est évaluée de la façon suivante:
		\begin{itemize}
			\item<4-> On évalue {\tt expr\_bool}
			\item<5-> Si le résultat est vrai alors la valeur de l'expression conditionnelle est {\tt expr1}
			\item<6-> Sinon c'est {\tt expr2}
		\end{itemize}
		\item<7-> {\tt expr1} et {\tt expr2} doivent toujours être du même type.
		\item<8-> On doit donc toujours écrire la clause {\tt else} \\
		\onslide<9->\textcolor{gray}{\small Sauf en fait dans le cas où {\tt expr1} est de type {\tt unit}, dans ce cas en cas d'omission, {\tt expr2} est par défaut {\tt ()} (seule valeur du type {\tt unit})}
	\end{itemize}
	\end{block}
\end{frame}

\begin{frame}{\Ctitle}{\stitle}
	\begin{exampleblock}{Exemple}
		\begin{enumerate}
			\item<1-> Ecrire une fonction \kw{abs\_entier} qui renvoie la valeur absolue de l'entier donné en argument .
				\onslide<2->\inputpartOCaml{\SPATH ex_cours.ml}{}{\small}{25}{26}
			\item<3-> Ecrire une fonction \kw{abs\_flottant} qui renvoie la valeur absolue du flottant donné en argument .
				\onslide<4->\inputpartOCaml{\SPATH ex_cours.ml}{}{\small}{28}{29}
		\end{enumerate}
	\end{exampleblock}
\end{frame}

\makess{Structure d'un programme Ocaml}
\begin{frame}{\Ctitle}{\stitle}
	\begin{block}{Structure générale d'un programme OCaml}
		\begin{itemize}
			\item <1-> Un programme OCaml repose sur l'évaluation d'une ou plusieurs expressions : \\
			      {\tt expr1 ;;} \\
			      {\tt expr2 ;;}\\
			      \dots\\
			\item <2-> Les expressions sont évaluées dans un \textcolor{blue}{environnement} (qui lie des identificateurs à des valeurs) \\
			      \textcolor{gray}{ {\tt a + b} est une expression qui sera évaluée à 10 si {\tt a} est lié à la valeur 4 et {\tt b} à la valeur 6.}
			\item  <3-> On manipule l'environnement \textcolor{BrickRed}{global} grâce au mot clef \kw{let} : \mintinline{ocaml}{let id = expr;;}
			\item <4-> Le mot clef {\tt in} permet de créer un environnement \textcolor{BrickRed}{local} : \mintinline{ocaml}{let id = expr1 in expr2;;} \\
			      \textcolor{gray}{Si {\tt id} existe existe déjà dans l'environnement courant il est provisoirement masqué.}
		\end{itemize}
	\end{block}
\end{frame}

\begin{frame}{\Ctitle}{\stitle}
	\begin{block}{Fonctions}
		\begin{itemize}
			\item <1-> L'appel d'une fonction \kw{f} à $n$ arguments s'écrit en OCaml : \kw{f x1 x2 ... xn}  \\
			      \textcolor{BrickRed}{\danger \;} Attention au parenthésage ! \\
			      {\tt f 2 * 3} est équivalent à {\tt (f 2) * 3} et vaut $f(2) \times 3$ \\
			      {\tt f (2*3)} vaut $f(2\times3)$.
			\item<2-> OCaml traite les fonctions à plusieurs variables comme une fonction à un argument qui renvoie une fonction sur le reste des variables. C'est ce qu'on appelle la \textcolor{BrickRed}{curryfication} (du nom du mathématicien américain Haskemll Curry). C'est à dire que par exemple : \\
				$\begin{array}{llll}
						f : & E \times F & \leftarrow & G \\
						    & (x,y)      & \mapsto    & z \\
					\end{array}$ s'interprète comme
				$\begin{array}{llll}
						f : & E & \leftarrow & \mathcal{A}{(F,G)} \\
						    & x & \mapsto    & (y \mapsto z)      \\
					\end{array}$
			\item<3-> En conséquence, appeler une fonction sans préciser tous ses arguments, crée une nouvelle fonction !
		\end{itemize}
	\end{block}
\end{frame}

\begin{frame}{\Ctitle}{\stitle}
	\begin{exampleblock}{Exemple}
		\inputpartOCaml{\SPATH ex_pgm.ml}{}{}{1}{9}
	\end{exampleblock}
\end{frame}

\begin{frame}{\Ctitle}{\stitle}
	\begin{exampleblock}{Exemple}
		\inputpartOCaml{\SPATH ex_curry.ml}{}{}{1}{10}
	\end{exampleblock}
\end{frame}

\makess{Les types construits}
\begin{frame}{\Ctitle}{\stitle}
	\begin{block}{Principe}
		Les trois mécanismes suivants permettent de construire de nouveaux types en OCaml à partir des types de base ({\tt int}, {\tt float}, {\tt bool}, \dots)
		\begin{itemize}
			\item<2-> Les couples, triplets et plus généralement les n-uplets, définissent des types de la forme $(x_1,\dots,x_n)$ ou chacun des $x_i$ peut avoir son propre type de base. \\
				\onslide<3->\textcolor{gray}{Ex : \mintinline{ocaml}{type point = float * float;;} définit un couple de flottant.}\\
				\onslide<4->Les fonctions \kw{fst } et \kw{snd } permettent d'accéder au premier et au second élément d'un n-uplet.
			\item<5-> Les types enregistrements (ou type produits).\\
				\onslide<6->\textcolor{gray}{Ex : \mintinline{ocaml}{type date = {jour : int; mois : string; annee : int}}}\\
				\onslide<7-> Accès aux champs avec la notation \kw{.} (comme en C).
			\item<6-> Les types unions (ou encore types sommes) où on liste les valeurs possibles séparés par des \kw{|}. \\
				\onslide<7->\textcolor{gray}{Ex : \mintinline{ocaml}{type signe = Positif | Negatif | Nul;;}}\\
				\onslide<8->\textcolor{gray}{Ex : \mintinline{ocaml}{type carte = Roi | Dame | Valet | Nombre of int;;}}\\
				\onslide<9->\textcolor{BrickRed}{\small \danger \;} Les constructeurs commencent par une majuscule.
		\end{itemize}
	\end{block}
\end{frame}

\begin{frame}{\Ctitle}{\stitle}
	\begin{exampleblock}{Exemple}
		\begin{enumerate}
			\item<1-> Définir le type complexe comme couple de flottant, écrire la fonction module pour ce type.
			\item<2-> Définir le type menu comme un type enregistrement composé des champs {\tt entree} (string), {\tt plat} (string), {\tt dessert}(string) et {\tt prix} (float).
			\item<3-> Définir le type rbarre ($\overline{\R}$), comme un type somme pouvant être {\tt Plusinfini}, {\tt Moinsinfini} et les {\tt flottants}.
		\end{enumerate}
	\end{exampleblock}
\end{frame}

\begin{frame}[fragile]{\Ctitle}{\stitle}
	\begin{exampleblock}{Exemple : correction}
		\begin{enumerate}
			\item<1-> Type produit complexe
			\inputpartOCaml{\SPATH ex_type.ml}{}{\small}{1}{6}
			\item<2-> Type menu enregistrement
			\inputpartOCaml{\SPATH ex_type.ml}{}{\small}{8}{8}
			\item<3-> Type somme rbarre
			\inputpartOCaml{\SPATH ex_type.ml}{}{\small}{10}{10}
		\end{enumerate}
	\end{exampleblock}
\end{frame}

\makess{Filtrage par motifs}
\begin{frame}{\Ctitle}{\stitle}
	\begin{block}{Définition}
			\begin{itemize}
			\item<2->{Le filtrage par motif (\textit{pattern matching}) est un mécanisme permettant de travailler efficacement sur les types construits en les décomposant et les analysant suivant leur structure. On peut ainsi indiquer l'expression  à évaluer suivant la forme spécifique de l'entrée.}
			\item<3->{La syntaxe générale est :\\
			 \kw{match} {\tt expr} \kw{with} \\
			{\tt | motif1 -> expr1}
			\dots \\
			{\tt | motifn -> exprn}}\\
			\onslide<4->{\textcolor{BrickRed}{\small \danger\;} Le filtrage doit être exhaustif, le caractère spécial \kw{\_} indique un motif qui correspond à toutes les entrées.}
			\item<5-> L'ordre du filtrage est important car si deux motifs correspondent à une entrée c'est l'expression du premier rencontré qui sera évalué.
			\item<6-> Chaque identifiant apparaît une fois au plus dans un motif.
			\item<7-> On peut filter un n-uplet sur un n-uplet de motif.
			\end{itemize}
	\end{block}
\end{frame}

\begin{frame}{\Ctitle}{\stitle}
	\begin{exampleblock}{Exemple}
		On a déjà rencontré le type carte : \\
		\mintinline{ocaml}{type carte = Roi | Dame | Valet | Nombre of int;;}\\
		On peut associer chaque carte à sa valeur (à la belote) à l'aide d'un pattern matching :
		\inputpartOCaml{\SPATH ex_type.ml}{}{\small}{12}{21}
	\end{exampleblock}
\end{frame}

\begin{frame}{\Ctitle}{\stitle}
	\begin{exampleblock}{Exemple}
		On peut aussi filtrer un couple de carte afin de détecter une éventuelle paire (deux cartes identiques) :
		\inputpartOCaml{\SPATH ex_type.ml}{}{\small}{23}{29}
	\end{exampleblock}
\end{frame}


\begin{frame}{\Ctitle}{\stitle}
	\begin{exampleblock}{Exercice}
		\begin{enumerate}
			\item<1-> Définir un type somme {\tt signe} avec les valeurs {\tt Negatif}, {\tt Positif} et {\tt Nul}.
			\onslide<2-> \inputpartOCaml{\SPATH ex_type.ml}{}{\small}{31}{31}
			\item<3-> Ecrire une fonction {\tt produit} qui renvoie le signe d'un produit 
		\onslide<4-> \inputpartOCaml{\SPATH ex_type.ml}{}{\small}{33}{39}
		\end{enumerate}
	\end{exampleblock}
\end{frame}


\end{document}
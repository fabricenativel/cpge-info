\PassOptionsToPackage{dvipsnames,table}{xcolor}
\documentclass[10pt]{beamer}
\usepackage{Cours}

\begin{document}


\input{\detokenize{/home/fenarius/Travail/Cours/cpge-info/latex/MacrosCours.tex}}

% Numéro et titre de chapitre
\setcounter{numchap}{15}
\newcommand{\Ctitle}{\cnum {Décomposition en sous problèmes}}
\newcommand{\SPATH}{/home/fenarius/Travail/Cours/cpge-info/docs/mp2i/files/C\thenumchap/}

\makess{Introduction}
\begin{frame}{\Ctitle}{\stitle}
	\begin{block}{Principe général}
		Pour résoudre un problème donné, une stratégie peut-être de se ramener à un ou plusieurs sous problèmes de \textit{même type} mais \textit{plus petit}. En notant $P(n)$, un problème de taille $n$, la résolution de $P_n$ conduit à la résolution de $k$ problèmes de tailles $P(\frac{n}{p})$. Un fois ces problèmes résolus, leurs solutions sont combinées afin de former celle du problème initial. \\
		\smallskip
		\onslide<2->{On distingue :}
		\begin{itemize}
			\item<3-> la méthode \textcolor{blue}{diviser pour régner}, dans laquelle les sous-problèmes sont indépendants
			\item<4-> la méthode de \textcolor{blue}{programmation dynamique} dans laquelle, certains sous problèmes se chevauchent.
		\end{itemize}
	\end{block}
\end{frame}

\makess{Diviser pour regner}
\begin{frame}{\Ctitle}{\stitle}
	\begin{exampleblock}{Exemple introductif}
		Le tri fusion est l'exemple typique d'une résolution par la méthode diviser pour régner. En effet, pour trier une liste $l$ de taille $n$,
		\begin{itemize}
			\item<2-> \textcolor{blue}{Diviser} : on sépare $l$ en deux moitiés (à une unité près) $l_1$ et $l_2$. Dans cet exemple $P(n)$ se ramène à la résolution de $2$ instances de résolution de $P(n/2)$.
			\item<3-> \textcolor{blue}{Régner } : on trie $l_1$ et $l_2$
			\item<4-> \textcolor{blue}{Combiner } : on fusionne les listes triées afin de construire la solution au problème initial
		\end{itemize}
	\end{exampleblock}
\end{frame}

\begin{frame}{\Ctitle}{\stitle}
	\begin{exampleblock}{Tri fusion en OCaml}
		\begin{enumerate}
			\item<1->{\small Ecrire une fonction {\tt separe int list -> int list * int list} qui prend en argument une liste d'entiers et renvoie les deux moitiés de cette liste. }\\
			\onslide<4->{\inputpartOCaml{\SPATH/tri_fusion.ml}{}{\small}{1}{5}}
			\item<2->{\small Ecrire une fonction {\tt fusion int list -> int list -> int list} qui fusionne les deux listes données en arugments.}
			\onslide<5->{\inputpartOCaml{\SPATH/tri_fusion.ml}{}{\small}{7}{11}}
		\end{enumerate}
	\end{exampleblock}
\end{frame}


\begin{frame}{\Ctitle}{\stitle}
	\begin{exampleblock}{Tri fusion en OCaml}
		\begin{enumerate}
			\addtocounter{enumi}{2}
			\item{\small Ecrire une fonction {\tt tri\_fusion int list -> int list} qui renvoie la liste donnée en argument triée.}
			\onslide<2->{\inputpartOCaml{\SPATH/tri_fusion.ml}{}{\small}{13}{19}}
		\end{enumerate}
	\end{exampleblock}
\end{frame}


\begin{frame}{\Ctitle}{\stitle}
	\begin{block}{Equation de complexité}
		On note :
		\begin{itemize}
			\item<1-> $C(n)$ la complexité de la résolution d'un problème de taille $n$.
			\item<2-> $k$ le nombre de sous problèmes à résoudre et $\frac{n}{p}$ leur taille.
			\item<3-> $T(n)$ le coût de construction de la solution de taille $n$.
		\end{itemize}
		\onslide<4->{On en déduit l' équation de complexité : \\}
		\onslide<5-> \textcolor{blue}{$C(n) = k\,C(\frac{n}{p}) + T(n)$\\}
		\onslide<6-> On suppose de plus que la résolution d'un problème de taille inférieure à un entier $m$ donnée s'effectue en temps constant
	\end{block}
	\onslide<7->{
		\begin{exampleblock}{Exemple}
			Dans le cas du tri fusion,
			\begin{itemize}
				\item<8-> Donner les valeurs de $k$, $p$ $m$ et écrire les équations de complexité.
				\item<9-> Donner un $O$ de $T(n)$.
			\end{itemize}
		\end{exampleblock}}
\end{frame}

\begin{frame}{\Ctitle}{\stitle}
	\begin{alertblock}{Résolution}
		\onslide<2->\textcolor{gray}{Si la complexité d'un problème est définie par une équation de la forme :
			$ C(n) = k C(\frac{n}{p}) + f(n)$, où $f$ est un polynôme de degré $d$, alors}
		\begin{itemize}
			\item<3->\textcolor{gray}{Si $d < \frac{\log(k)}{\log(p)}$, alors $C \in O(n^\frac{\log(k)}{\log(p)})$}
			\item<4->\textcolor{gray}{Si $d = \frac{\log(k)}{\log(p)}$, alors $C \in O(n^d \log(n))$}
			\item<5->\textcolor{gray}{Si $d > \frac{\log(k)}{\log(p)}$, alors $C \in O(n^d)$}
		\end{itemize}
		\onslide<6->\textcolor{BrickRed}{\small \danger} Ce théorème appelé (\textit{master theorem}) est hors-programme. \\
		\onslide<7-> Mais il permet de résoudre instantanément les équations de complexité de la méthode diviser pour régner !
	\end{alertblock}
\end{frame}

\begin{frame}{\Ctitle}{\stitle}
	\begin{block}{Complexité d'une méthode diviser pour régner}
		\begin{itemize}
			\item<1-> Dans la plupart des cas, on peut résoudre l'équation de complexité \textbf{sans} le \textit{master theorem}.\\
				\onslide<2->\textcolor{OliveGreen}{\small Par exemple, dans le cas du tri fusion, les équations sont :
					$\left\{
						\begin{array}{lll}
							C(0)  & \in & O(1)          \\
							C(2n) & =   & 2C(n)  + O(n) \\
						\end{array}
						\right.$}\\
				\onslide<3->\textcolor{OliveGreen}{\small Pour simplifier on suppose que $n$ est une puissance exacte de $2$ et on obtient : $C(2^{k+1}) \leq 2C(2^k) + M2^k$ et en divisant par $2^{k+1}$, on obtient \\}
				\onslide<4->\textcolor{OliveGreen}{\small $u_{k+1} \leq u_{k} + M$ où $u_k = \frac{C(2^k)}{2^k}$.\\}
				\onslide<5->\textcolor{OliveGreen}{\small par récurrence immédiate, $u_{k} \leq u_0 + kM$}
				\onslide<6->\textcolor{OliveGreen}{\small et donc $C(n) \leq n u_0  + M\,n\log_2(n) $\\}
				\onslide<7->\textcolor{OliveGreen}{\small c'est à dire $C(n) \in O(n\log n)$.}
			\item<2-> Sinon, on peut utiliser le \textit{master theorem} afin d'obtenir la complexité, puis la prouver par récurrence.
		\end{itemize}
	\end{block}
\end{frame}


\makess{Exemples résolus de la méthode diviser pou régner}
\begin{frame}{\Ctitle}{\stitle}
	\begin{exampleblock}{Tranche maximale dans un tableau}
		On considère un tableau $T$ de $n$ entiers, le but du problème est de déterminer la somme maximale d'une tranche (c'est à dire d'éléments contigus de $T$).
		Par exemple si $T = [2, -7, -5, 4, -1, 10, -4, 9, -2]$ alors la somme maximale d'une tranche est : \onslide<2->{\textcolor{OliveGreen}{18, et elle est obtenue en prenant la tranche $[ 4, -1, 10, -4, 9]$}}. \\
		\begin{enumerate}
			\item<4-> Proposer un algorithme de complexité quadratique permettant de résoudre ce problème.
			\item<5-> En donner une implémentation en langage C
			\item<6-> Une solution plus efficace :
				\begin{enumerate}
					\item<6-> Proposer un nouvel algorithme basé sur la méthode \textit{diviser pour régner}
					\item<7-> Donner une implémentation en C de ce nouvel algorithme
					\item<8-> Déterminer sa complexité
				\end{enumerate}
		\end{enumerate}
	\end{exampleblock}
\end{frame}

\begin{frame}[fragile]{\Ctitle}{\stitle}
	\begin{exampleblock}{Résolution avec une complexité quadratique}
		\begin{enumerate}
			\item On calcule les $S_{i,j}$ (somme de la tranche des éléments du tableau compris entre les indices $i$ et $j$ inclus) de proche en proche, en utilisant $S_{ij} = S_{i,j-1} + t_j$ et on prend le maximum des valeurs obtenus.\\
			      \onslide<2->{On utilise donc deux boucles imbriquées dans laquelle on effectue uniquement des opérations élémentaires, la complexité est donc quadratique.}
			\item<3-> Implémentation :
				\inputpartC{\SPATH/tmax_quadratique.c}{}{\small}{5}{14}
		\end{enumerate}
	\end{exampleblock}
\end{frame}

\begin{frame}[fragile]{\Ctitle}{\stitle}
	\begin{exampleblock}{Résolution : diviser pour régner}
		\begin{enumerate}
			\item<1-> On note $k = \left\lfloor \frac{n}{2} \right\rfloor$.
				\begin{itemize}
					\item<2-> \textcolor{blue}{Diviser} : on sépare $T$ en deux sous tableaux $T_g = [t_0 \dots t_{k-1}]$ et $T_d = [t_{k+1}\dots t_{n-1}]$. \\
						\textcolor{BrickRed}{\small \danger} Attention : on remarquera bien que $t_k$ n'est dans aucun des deux sous tableaux !
					\item<3-> \textcolor{blue}{Régner} : on recherche les tranche maximales des sous tableaux $T_g$ et $T_d$ ainsi que celle des tranches contenant l'élément $t_k$.
					\item<4-> \textcolor{blue}{Combiner} on prend le maximum des trois valeurs obtenues.
				\end{itemize}
		\end{enumerate}
	\end{exampleblock}
\end{frame}

\begin{frame}[fragile]{\Ctitle}{\stitle}
	\begin{exampleblock}{Résolution : diviser pour régner}
		\begin{enumerate}
			\addtocounter{enumi}{1}
			\item<2-> Implémentation :
				\inputpartC{\SPATH/tmax.c}{}{\small}{35}{45}
		\end{enumerate}
	\end{exampleblock}
\end{frame}


\begin{frame}[fragile]{\Ctitle}{\stitle}
	\begin{exampleblock}{Résolution : diviser pour régner}
		\begin{enumerate}
			\addtocounter{enumi}{2}
			\item<2-> Calcul de la complexité\\
				\onslide<3-> Pour résoudre un problème de taille $n$, on doit en résoudre deux de tailles $\lfloor \frac{n}{2} \rfloor$ et rechercher le maximum des tranches contenant l'élément $t_k$. Cette opération a une complexité linéaire, on a donc :
				$\left\{
					\begin{array}{lll}
						C(0)  & \in & O(1)          \\
						C(2n) & =   & 2C(n)  + O(n) \\
					\end{array}
					\right.$\\
				\onslide<4->{On retrouve les mêmes équations de complexité que dans le cas du tri fusion. Et donc la complexité est la même : $O(n\,\log n)$. }
		\end{enumerate}
	\end{exampleblock}
\end{frame}

\begin{frame}[fragile]{\Ctitle}{\stitle}
	\begin{exampleblock}{Multiplication matricielle}
		Soient $A$ et $B$ deux matrices carrés de tailles $n$, on note $C = A\times B$
		\begin{enumerate}
			\item<2-> Rappeler l'expression de $C_{ij}$.
			\item<3-> Quelle est la complexité d'un algorithme calculant les coefficients de $C$ en utilisant l'expression précédente ?
			\item<4-> on sépare $A$ et $B$ en blocs de tailles égales : \\
			$A =
				\begin{bmatrix}
					A_{1,1} & A_{1,2} \\
					A_{2,1} & A_{2,2}
				\end{bmatrix}$ et
			$B =
				\begin{bmatrix}
					B_{1,1} & B_{1,2} \\
					B_{2,1} & B_{2,2}
				\end{bmatrix}$ \\
			Déterminer la complexité d'un algorithme qui effectue la multiplication par bloc : \\
			$\begin{bmatrix}
				A_{1,1} & A_{1,2} \\
				A_{2,1} & A_{2,2}
			\end{bmatrix} \times 
				\begin{bmatrix}
					B_{1,1} & B_{1,2} \\
					B_{2,1} & B_{2,2}
				\end{bmatrix}
			$ 
		\end{enumerate}
	\end{exampleblock}
\end{frame}

\begin{frame}[fragile]{\Ctitle}{\stitle}
	\begin{exampleblock}{Algorithme de Strassen}
		{\small L'algorithme de Strassen est une approche diviser pour régner :}
				\begin{itemize}
					\item<2->{\small \textcolor{blue}{diviser} : on sépare $A$ et $B$ en blocs de tailles égales : \\
						$A =
							\begin{bmatrix}
								A_{1,1} & A_{1,2} \\
								A_{2,1} & A_{2,2}
							\end{bmatrix}$ et
						$B =
							\begin{bmatrix}
								B_{1,1} & B_{1,2} \\
								B_{2,1} & B_{2,2}
							\end{bmatrix}$}
			\item<3->{\small \textcolor{blue}{régner} On calcule \textcolor{BrickRed}{seulement \textbf{7}} produits matriciels : \\
$M_{1} = (A_{1,1} + A_{2,2}) (B_{1,1} + B_{2,2})$ \\
$M_{2} = (A_{2,1} + A_{2,2}) B_{1,1}$\\
$M_{3} = A_{1,1} (B_{1,2} - B_{2,2})$\\
$M_{4} = A_{2,2} (B_{2,1} - B_{1,1})$\\
$M_{5} = (A_{1,1} + A_{1,2}) B_{2,2}$\\
$M_{6} = (A_{2,1} - A_{1,1}) (B_{1,1} + B_{1,2})$\\
$M_{7} = (A_{1,2} - A_{2,2}) (B_{2,1} + B_{2,2})$}
			\item<4->{\small \textcolor{blue}{combiner} On combine les solutions afin de construire les blocs de la matrice $C$ :\\
			$C_{1,1} = M_{1} + M_{4} - M_{5} + M_{7}$\\
			$C_{1,2} = M_{3} + M_{5}$\\
			$C_{2,1} = M_{2} + M_{4}$\\
			$C_{2,2} = M_{1} - M_{2} + M_{3} + M_{6}$}
\end{itemize}
	\end{exampleblock}
\end{frame}

\begin{frame}[fragile]{\Ctitle}{\stitle}
	\begin{exampleblock}{Complexité}
		L'équation de complexité s'écrit alors : \\
		$C(2n) = 7\,C(n) + O(n^2)$ \\
		\onslide<2->{Et on montre que $C(n) \in O(n^{\log_2 7}$), et $\log_2 7 \simeq 2,807$\\}
		\onslide<3->{On obtient donc une complexité meilleure que l'algorithme "naïf" \\ \smallskip}
		\onslide<4->{A noter qu'à cause  des tailles respectives des \textit{facteurs cachés} dans l'algorithme de naif et dans l'algorithme de Strassen, ce dernier ne devient plus efficace en terme de temps de calcul que pour de grandes valeurs de $n$.}
		\onslide<5->{
			\begin{center}
				\includegraphics[width=120px]{\SPATH/strassen.eps}
			\end{center}
		}
	\end{exampleblock}
\end{frame}

\makess{Rappel : mémoïsation}
\begin{frame}{\Ctitle}{\stitle}
	\begin{exampleblock}{Exemple}
		\begin{enumerate}
			\item<1-> Ecrire une fonction récursive \textit{naïve} en Ocaml qui prend en argument un entier $n$ et renvoie le $n$ième terme de la suite de Fibonacci définie par :
				$\left\{ \begin{array}{lll}
						f_0   & = & 1,                                                  \\
						f_1   & = & 1,                                                  \\
						f_{n} & = & f_{n-1}+f_{n-2} \mathrm{\ \ pour\ tout\ \ } n\geq2.\end{array} \right.$
					\onslide<2->{\inputpartOCaml{\SPATH/fibo.ml}{}{}{1}{2}}
			\item<3-> Tracer le graphe des appels récursifs de cette fonction pour $n=5$
			\item<4-> Commenter
		\end{enumerate}
	\end{exampleblock}
\end{frame}



\begin{frame}{\Ctitle}{\stitle}
	\begin{exampleblock}{Exemple}
		\begin{center}
			\psset{levelsep=1cm,treesep=0.2cm,linecolor=OliveGreen,linewidth=0.6pt}
			\pstree{\Toval{\tiny \tt fibo(5)}}{
				\pstree{\Toval{\tiny \tt fibo(4)}}{
					\pstree{\Toval{\tiny \tt  fibo(3)}}{
						\pstree{\Toval{\textcolor{BrickRed}{\tt \tiny fibo(2)}}}{\Toval{\tt \tiny fibo(1)} \Toval{\tiny \tt  fibo(0)}}
						\Toval{\tiny \tt  fibo(1)}}
					\pstree{\Toval{\textcolor{BrickRed}{\tt \tiny fibo(2)}}}{\Toval{\tiny  \tt fibo(1)} \Toval{\tiny \tt  fibo(0)}}
				}
				\pstree{\Toval{\tiny \tt fibo(3)}}{
					\pstree{\Toval{\textcolor{BrickRed}{\tt \tiny fibo(2)}}}{\Toval{\tiny \tt  fibo(1)} \Toval{\tiny \tt  fibo(0)}}
					\Toval{\tiny \tt  fibo(1)}}
			}
		\end{center}
		\onslide<2-> Les mêmes appels récursifs apparaissent dans plusieurs branche, on dit qu'il y a \textcolor{blue}{\textit{chevauchement des appels récursifs}}.
		\onslide<3-> (On peut montrer que le nombre d'appels récursifs $a_n$ pour calculer $f_n$ est $a_n = 2\,f_n -1$ et donc la complexité est exponentielle)
	\end{exampleblock}
\end{frame}



\begin{frame}{\Ctitle}{\stitle}
	\begin{alertblock}{Mémoïsation}
		\begin{itemize}
			\item<1-> La \textcolor{blue}{mémoïsation} consiste à stocker dans une structure de données les valeurs renvoyées par une fonction afin de ne pas les recalculer lors des appels identiques suivant.\\
			\item<2-> Les tableaux associatifs dont les clés sont les arguments de la fonction et les valeurs les résultats correspondant sont des structures de données adaptées à ce stockage car on teste l'appartenance et on retrouve une valeur efficacement.
			\item<3-> On rappelle qu'un tableau associatif peut-être implémenté de façon efficace par :
			\begin{itemize}
				\item une table de hachage 
				\item un arbre binaire de recherche lorsque les clés sont ordonnées.
			\end{itemize}
		\end{itemize}
	\end{alertblock}
\end{frame}


\makess{Programmation dynamique : exemple introductif}
\begin{frame}{\Ctitle}{\stitle}
	\begin{exampleblock}{Position du problème}
		\onslide<1->{\small On considère une barre de métal de longueur entière 12 et pouvant être découpée en morceaux de longueurs entières ayant chacun un prix comme indiqué ci-dessous :
			\begin{center}
				\begin{tabular}{|l|p{0.2cm}|p{0.2cm}|p{0.2cm}|p{0.2cm}|p{0.2cm}|p{0.2cm}|p{0.2cm}|p{0.2cm}|p{0.2cm}|p{0.2cm}|p{0.2cm}|p{0.2cm}|}
					\hline
					longueur ($i$) & 1 & 2 & 3 & 4 & 5  & 6  & 7  & 8  & 9  & 10 & 11 & 12 \\
					\hline
					prix     ($p_i$)& 2 & 4 & 7 & 8 & 12 & 14 & 18 & 23 & 24 & 25 & 26 & 31 \\
					\hline
				\end{tabular}
			\end{center}}
		\onslide<2->{\small Le prix de vente des différents morceaux varie donc suivant la découpe utilisée, par exemples :
			la découpe $(2, 4, 6)$ a un prix de vente de $4+8+14=26$, tandis que la découpe $(7, 5)$ a un prix de vente de $18+12=30$\\}
		\onslide<3->\textcolor{blue}{\small Le but du problème est de trouver la valeur maximale des découpes possibles.\\}
		\onslide<4->{\small On note $N$ la longueur de la barre, $(v_i)_{1\leq i \leq N}$, la valeur maximale de la découpe d'une barre de taille $i$ et $(p_i)_{1 \leq i \leq N}$ le prix d'un morceaux de longueur $i$.}
		\begin{enumerate}
			\item<5-> {\small Donner les valeurs de $v_0$, $v_1$, $v_2$ et $v_3$.}
			\item<6-> {\small Etablir une relation de récurrence liant les $(v_i)_{0\leq i \leq N}$.}
			\item<7-> {\small En déduire une fonction  récursive en C calculant la valeur de la découpe maximale.}
			\item<8-> {\small Vérifier qu'on se trouve dans une situation de chevauchement des appels récursifs et proposer une nouvelle version de votre fonction utilisant la mémoïsation.}
			\item<9-> {\small Etudier les complexités des deux versions.}
		\end{enumerate}
	\end{exampleblock}
\end{frame}

\begin{frame}{\Ctitle}{\stitle}
	\begin{exampleblock}{Résolution}
		\begin{enumerate}
			\item<1-> \textcolor{OliveGreen}{\small $v_0=0$, $v_1=2$, $v_2 = 4$ et $v_3 = 7$ }
			\item<2-> \textcolor{OliveGreen}{\small En supposant qu'on connaisse les valeurs maximales de découpe pour \textit{toutes} les tailles inférieures à $n$, la découpe maximale pour la taille $n$ s'en déduit en prenant le maximum parmi les découpes maximales d'une barre de longueur $n-k \leq n-1$  et du prix d'un morceau de taille $k$, c'est à dire :
					$v_n = \max\left\{ v_{n-k} + p_{k},  1 \leq k \leq n\right\}$}
			\item<3-> \textcolor{OliveGreen}{\small Implémentation en C:}
				\inputpartC{\SPATH/barre.c}{}{\footnotesize}{19}{28}
		\end{enumerate}
	\end{exampleblock}
\end{frame}


\begin{frame}{\Ctitle}{\stitle}
	\begin{exampleblock}{Résolution}
		\begin{enumerate}
			\addtocounter{enumi}{3}
			\item Pour calculer $v_5$, on doit calculer $v_4, v_3, \dots v_0$. Mais l'appel à $v_4$ demande aussi le calcul de $v_3, v_2, \dots v_0$. On se trouve donc bien dans le cas d'un chevauchement d'appels récursifs. \\
			\medskip
			On peut proposer la version avec memoisation suivante :
			\inputpartC{\SPATH/barre.c}{}{\footnotesize}{30}{40}
		\end{enumerate}
	\end{exampleblock}
\end{frame}

\begin{frame}{\Ctitle}{\stitle}
	\begin{exampleblock}{Résolution}
		\begin{enumerate}
			\addtocounter{enumi}{4}
			\item On peut construire l'arbre des appels récursifs pour $n=5$ :\\
			\onslide<2->{\begin{center}
			\psset{levelsep=1cm,treesep=0.2cm,linecolor=OliveGreen,linewidth=0.6pt}
			\pstree{\TCircle{$v_5$}}{
				\TCircle{$v_1$}
				\pstree{\TCircle{$v_2$}}{
					\TCircle{$v_1$}}
				\pstree{\TCircle{$v_3$}}{
						\TCircle{$v_1$}
						\pstree{\TCircle{$v_2$}}{
						\TCircle{$v_1$}}
						}
				\pstree{\TCircle{$v_4$}}{
					\TCircle{$v_1$}
					\pstree{\TCircle{$v_2$}}{
					\TCircle{$v_1$}}
					\pstree{\TCircle{$v_3$}}{
						\TCircle{$v_1$}
						\pstree{\TCircle{$v_2$}}{
						\TCircle{$v_1$}}
						}
				}
			}
			\end{center}}
			\onslide<3->{En notant $a_n$ le nombre d'appels pour calculer $v_n$, on a $\displaystyle{a_n = 1 + \sum_{k=0}^{n-1} a_k}$.}
			\onslide<4->{On obtient $a_n = 2^n$ et donc la complexité est au moins quadratique !}
		\end{enumerate}
	\end{exampleblock}
\end{frame}

\begin{frame}{\Ctitle}{\stitle}
	\begin{exampleblock}{Résolution}
		\begin{enumerate}
			\addtocounter{enumi}{4}
			\item Dans le cas de la mémoïsation, les appels récursifs déjà calculés sont obtenus directement, ce qui donne l'arbre d'appel suivant :
			\onslide<2->{\begin{center}
				\psset{levelsep=1cm,treesep=0.2cm,linecolor=OliveGreen,linewidth=0.6pt}
				\pstree{\TCircle{$v_5$}}{
					\TCircle{$v_1$}
					\pstree{\TCircle{$v_2$}}{
						\TCircle{$v_1$}}
					\pstree{\TCircle{$v_3$}}{
							\TCircle{$v_1$}
							\TCircle{$v_2$}
							}
					\pstree{\TCircle{$v_4$}}{
						\TCircle{$v_1$}
						\TCircle{$v_2$}
						\TCircle{$v_3$}
						}
					}
				\end{center}}
				\onslide<3->{En notant $b_n$ le nombre d'appels pour calculer $v_n$, on a $\displaystyle{b_n = \sum_{k=0}^{n-1} k}$.}
			\onslide<4->{La complexité est donc quadratique.}
		\end{enumerate}
	\end{exampleblock}
\end{frame}

\begin{frame}{\Ctitle}{\stitle}
	\begin{exampleblock}{Calcul de bas en haut (\textit{bottom up})}
		\onslide<1->{\small La mémoïsation construit la solution de façon "descendante", on lance les appels récursif sur les plus grandes valeurs de taille de la barre. Une autre stratégie dite \textcolor{blue}{ascendante} ou \textcolor{blue}{de bas en haut (\textit{bottom up})} consiste à construire la solution en partant des instances les plus petites du problème.\\}
		\onslide<2->{\small Pour la découpe de la barre on part donc des valeurs connues $v_0$ et $v_1$ et on construit $v_2$ puis $v_3$, en utilisant la relation de récurrence $v_n = \max\left\{ v_{n-k} + p_{k},  1 \leq k \leq n\right\}$}\\
		\onslide<3->{\small Ce qui se traduit  par une solution \textcolor{blue}{itérative} :}
		\onslide<4->{\inputpartC{\SPATH/barre.c}{}{\footnotesize}{43}{52}}
	\end{exampleblock}
\end{frame}

\begin{frame}{\Ctitle}{\stitle}
	\begin{exampleblock}{Construction d'une solution}
		On a pour le moment déterminé la valeur maximale de la découpe, mais pas la découpe elle-même. D'autre part, plusieurs découpes différentes peuvent avoir cette même valeur maximale. Pour rechercher \textit{une} découpe de valeur maximale, on peut par exemple :
		\begin{itemize}
			\item<1-> construire le tableau $(v_k)_{0\leq k \leq\N}$  et l'utiliser afin d'en déduire la découpe. \\
				\onslide<2->{\textcolor{gray}{\small Par exemple, si $v_{12} = v_8 + p_4$, cela signifie que pour avoir la valeur maximale de la découpe d'une barre de taille 12, une possibilité est d'utiliser une découpe  maximale d'une barre de taille 8 et un morceau de taille 4. En remontant ainsi de proche en proche, on obtient une découpe maximale possible}}
			\item<3-> Modifier notre fonction afin qu'elle renvoie la découpe maximale et non pas la valeur de cette découpe. \\
		\end{itemize}
		\onslide<4->{Ces deux possibilités seront abordées en TP.}
	\end{exampleblock}
\end{frame}


\makess{Programmation dynamique}
\begin{frame}{\Ctitle}{\stitle}
	\begin{alertblock}{Principes généraux}
		La programmation dynamique s'applique généralement à la résolution d'un problème d'optimisation vérifiant les conditions suivantes :
		\begin{enumerate}
			\item<2-> \textcolor{blue}{Sous structure optimale} : ce problème peut-être résolu à partir de problèmes similaires mais plus petits \\
				\onslide<3->\textcolor{gray}{\small La découpe maximale d'une barre de taille $N$ s'obtient comme découpe maximale  d'une barre de taille strictement inférieure $k$ et d'un morceau de taille $N-k$.}
				\item<4->\textcolor{blue}{Chevauchement de sous problème} : une solution récursive produit des appels identiques. Pour pallier ce problème, on utilise la mémoïsation dans les solutions récursives ou une solution de bas en haut itérative.\\
				\onslide<5->\textcolor{gray}{\small Pour rechercher la découpe maximale d'un barre de taille 5, on est amené à chercher celle d'une barre de taille 4,3,2,1. Et pour chercher celle d'une barre de taille 4, on fera de nouveau appel à celle d'une barre de taille 3,2,1 ...}
		\end{enumerate}
		\onslide<6->{\textcolor{BrickRed}{\small \important} L'étape cruciale est de déterminer les relations de récurrence entre les différentes instances du problème.}
	\end{alertblock}
\end{frame}

\begin{frame}{\Ctitle}{\stitle}
	\begin{exampleblock}{Sous structure optimale : contre-exemples}
	On donne ici deux exemples de problèmes n'ayant \textit{pas} la propriété de \textcolor{blue}{sous structure optimale} :
		\begin{itemize}
			\item<2-> Nombre minimal de multiplications dans le calcul de $a^n$\\
			\onslide<3->\textcolor{gray}{\small Par exemple le calcul optimal de $a^{15}$ demande 5 multiplications : $a^{15} = a^3 \times \left((a^3)^2\right)^2$ \\ $a^6$ doit donc être donc calculé en utilisant le calcul de $a^3$ au lieu de par exemple $a^6 = (a^2)^3$ qui demande aussi 3 multiplications. C'est à dire qu'on peut avoir trouvé une solution optimale $n=6$ mais qui n'intervient \textit{pas} dans le calcul pour $n=15$.}
			\item<4-> Recherche du plus long chemin élémentaire dans un graphe\\
			\onslide<5->{
			\begin{tabularx}{\linewidth}{p{2.5cm}X}
				\begin{tabular}{lr}
					\circlenode[linecolor=gray]{a}{\textcolor{gray}{$a$}} \hspace{1cm} & \circlenode[linecolor=gray]{b}{\textcolor{gray}{$b$}} \vspace{1cm}\\
					\multicolumn{2}{c}{\circlenode[linecolor=gray]{c}{\textcolor{gray}{$c$}}}\\
				\end{tabular} 
				\ncarc[linecolor=gray]{->}{a}{b} \ncarc[linecolor=gray]{->}{a}{c} \ncarc[linecolor=gray]{->}{b}{a} \ncarc[linecolor=gray]{->}{b}{c} \ncarc[linecolor=gray]{->}{c}{a} \ncarc[linecolor=gray]{->}{c}{b}
				& \vspace{-1cm}
				\textcolor{gray}{\small Dans le graphe ci-contre, le plus long chemin élémentaire de $a$ vers $c$ est $a \rightarrow b \rightarrow c$ et son sous-chemin $a \rightarrow b$ n'est pas une la solution du sous problème consistant à rechercher le plus long chemin élémentaire de $a$ vers $b$. Et de même pour son sous chemin $b \rightarrow c$.}
			\end{tabularx}}
		\end{itemize}
	\end{exampleblock}
\end{frame}



\makess{Exemple résolu : plus longue sous séquence commune}
\begin{frame}{\Ctitle}{\stitle}
	\begin{exampleblock}{Position du problème}
		On considère deux chaines de caractères $u$ et $v$ de longueurs respectives $n$ et $m$. On cherche à déterminer la longueur de leur \textcolor{blue}{p}lus \textcolor{blue}{l}ongue \textcolor{blue}{s}ous \textcolor{blue}{s}équence \textcolor{blue}{c}ommune (\textcolor{blue}{\sc plssc}), c'est à dire la chaine $w$ telle que :
		\begin{itemize}
			\item<2-> $w$ est une sous séquence (c'est à dire une suite extraite) de $u$,
			\item<3-> $w$ est une sous séquence de $w$,
			\item<4-> $w$ est de longueur maximale.
		\end{itemize}
		\onslide<5->{Par exemple,  $u$="{\sc programmation}" et $v$="{\sc dynamique}" ont comme sous séquence commune "{\sc ami}" (et c'est la plus longue)}
		\begin{itemize}
			\item<6-> {\sc progr\textcolor{BrickRed}{a}m\textcolor{BrickRed}{m}at\textcolor{BrickRed}{i}on}
			\item<7-> {\sc dyn\textcolor{BrickRed}{ami}que}
		\end{itemize}
		\onslide<8->{Donc ici, la longueur de la {\sc plssc} est 3.}
	\end{exampleblock}
\end{frame}

\begin{frame}{\Ctitle}{\stitle}
	\begin{exampleblock}{Résolution}
		On cherche les relations de récurrence entre des instances du sous-problème. Pour cela on note $u_i$ ($0\leq i \leq n$) la chaine composée des $i$ premiers caractères de $u$, et $v_j$ ($0\leq j \leq m$) celle composée des $j$ premiers caractères de $v$. Et on note $\mathrm{lplssc}(u_i,v_j)$ la longueur de la {\sc plssc} de $u_i$ et de $v_j$.
		\begin{itemize}
			\item<2->{Si $u[i] = v[j]$ alors, quelle est la relation entre $\mathrm{lplssc}(u_i,v_j)$ et $\mathrm{lplssc}(u_{i-1},v_{j-1})$ ? \\}
			\onslide<5->{\textcolor{OliveGreen}{$\mathrm{lplssc}(u_i,v_j) = 1 + \mathrm{lplssc}(u_{i-1},v_{j-1})$}}
			\item<3->{Sinon, exprimer $\mathrm{plssc}(u_i,v_j)$ en fonction de  $\mathrm{plssc}(u_{i},v_{j-1})$  et $\mathrm{plssc}(u_{i-1},v_j)$}
			\onslide<6->{\textcolor{OliveGreen}{$\mathrm{lplssc}(u_i,v_j) = \max\left(\mathrm{lplssc}(u_{i},v_{j-1}) ,\mathrm{lplssc}(u_{i-1},v_{j})\right)$}}
			\item<4->{Déterminer les cas de base (ceux où $u$ et $v$ sont des chaines vides notés $\epsilon$):\\}
			\onslide<7->{\textcolor{OliveGreen}{$\mathrm{lplssc}(u_{i},\epsilon) = 0$} \\
				\textcolor{OliveGreen}{$\mathrm{lplssc}(\epsilon,v_j) = 0$} }
		\end{itemize}
	\end{exampleblock}
\end{frame}


\begin{frame}{\Ctitle}{\stitle}
	\begin{exampleblock}{Implémentation en OCaml}
		On doit donc écrire une fonction {\tt lplssc string -> string -> int} qui prend en argument deux chaines de caractères {\tt u} et {\tt v} et renvoie la longueur de leur plus longue sous séquence commune.\\
		\onslide<2->{\textcolor{OliveGreen}{\small \aide} Comme on travaille récursivement sur la longueur des prefixes on pourra écrire une fonction auxiliaire {\tt aux  string -> string -> int -> int -> int} qui prend deux entiers supplémentaires en arguments : les longueurs de chacune des deux chaines}
		\onslide<3->{\inputpartOCaml{\SPATH/plssc.ml}{}{}{1}{7}}
	\end{exampleblock}
\end{frame}

\begin{frame}{\Ctitle}{\stitle}
	\begin{exampleblock}{Mémoïsation}
		{\small Modifier la fonction précédente afin de mémoriser les résultats déjà calculés. On pourra utiliser une matrice {\tt memo} crée en OCaml avec \mintinline{Ocaml}{let memo = Array.make_matrix (n+1) (m+1) (-1)}\\ La valeur initiale $-1$ indiquant que la {\sc lplssc} de $u_i$ et $v_j$ n'a pas encore été calculée.}
		\onslide<3->{\inputpartOCaml{\SPATH/plssc.ml}{}{\footnotesize}{10}{22}}
	\end{exampleblock}
\end{frame}

\begin{frame}{\Ctitle}{\stitle}
	\begin{exampleblock}{Version itérative}
		\onslide<2->{\inputpartOCaml{\SPATH/plssc.ml}{}{\footnotesize}{37}{54}}
	\end{exampleblock}
\end{frame}

\begin{frame}{\Ctitle}{\stitle}
	\begin{exampleblock}{Reconstruction de la solution}
		{\small La matrice obtenue pour les mots : {\sc imperial} et {\sc empire} est : \\}
		\quad \quad \begin{tabular}{|>{\tt}c|>{\tt}c|>{\tt}c|>{\tt}c|>{\tt}c|>{\tt}c|>{\tt}c|>{\tt}c|}
			
			\cline{2-8}
		\multicolumn{1}{c|}{}	& \textcolor{blue}{$\epsilon$}  & \textcolor{blue}{E}  &\alt<17->{\textcolor{red}{M}}{\textcolor{blue}{M}} &\alt<15->{\textcolor{red}{P}}{\textcolor{blue}{P}} &\alt<11->{\textcolor{red}{I}}{\textcolor{blue}{I}} &\textcolor{blue}{R} & \textcolor{blue}{E} \\
			\hline
			\textcolor{blue}{$\epsilon$}	& 0 &0 &0 &0 &0 &0 &0 \\ 
		\textcolor{blue}{I}	& 0 & \alt<18->{\textcolor{red}{0}}{0} &0 &0 &1 &1 &1 \\ 
		\alt<17->{\textcolor{red}{M}}{\textcolor{blue}{M}}	& 0 &0 &\alt<16->{\textcolor{red}{\underline 1}}{1} &1 &1 &1 &1 \\ 
		\alt<15->{\textcolor{red}{P}}{\textcolor{blue}{P}}	& 0 &0 & 1 &\alt<14->{\textcolor{red}{\underline 2}}{2} &2 &2 &2 \\ 
		\textcolor{blue}{E}	& 0 &1 &1 &\alt<13->{\textcolor{red}{2}}{2} &2 &2 &3 \\ 
		\textcolor{blue}{R}	& 0 &1 &1 &\alt<12->{\textcolor{red}{2}}{2} &2 &3 &3 \\ 
		\alt<11->{\textcolor{red}{I}}{\textcolor{blue}{I}}	& 0 &1 &1 &2 &\alt<10->{\textcolor{red}{\underline 3}}{3} &3 &3 \\ 
		\textcolor{blue}{A}	& 0 &1 &1 &2 &\alt<9->{\textcolor{red}{3}}{3} &3 &3 \\ 
		\textcolor{blue}{L}	& 0 &1 &1 &2 &\alt<8->{\textcolor{red}{3}}{3} &\alt<7->{\textcolor{red}{3}}{3} &\alt<6->{\textcolor{red}{3}}{3} \\
		\hline
		\end{tabular}\\
	\onslide<2->{\small On peut utiliser ce résultat pour construire la {\sc plssc} :}
	\begin{itemize}
		\item<3-> {\small on part de la dernière case en bas et à droite}
		\item<4-> {\small si les lettres sur la ligne et la colonne sont identiques on ajoute à la {\sc plssc} et on remonte en diagonale}
		\item<5-> {\small sinon on va à gauche ou en haut suivant la case qui a plus grande valeur}
	\end{itemize}
	\end{exampleblock}
\end{frame}


\begin{frame}{\Ctitle}{\stitle}
	\begin{exampleblock}{Reconstruction de la solution}
		\onslide<2->{\inputpartOCaml{\SPATH/plssc.ml}{}{\footnotesize}{59}{74}}
	\end{exampleblock}
\end{frame}


\makess{Exemple résolu : rendu de monnaie}
\begin{frame}{\Ctitle}{\stitle}
	\begin{exampleblock}{Position du problème}
		On dispose d'un \textit{système monétaire} c'est à dire d'un ensemble de valeurs possibles pour les pièces et les billets. Le problème du rendu de monnaie consiste à déterminer le nombre minimal de pièces à utiliser pour former une somme donnée. \\
		\onslide<2->{Par exemple si le système monétaire est $\{ 1, 3, 4, 5 \}$ et la somme 7,}
		\onslide<3->{alors on peut utiliser au minimum 2 pièces ($4+3$).}\\
		\onslide<5->{\textcolor{blue}{\small \rappel \; Rappel : }{\textcolor{gray}{ l'algorithme glouton qui consiste à rendre à tout moment la pièce de plus forte valeur possible ne fournit pas toujours la solution optimale. Ici, on obtiendrait $5, 1, 1$ et donc 3 pièces.}}}
		\begin{enumerate}
			\item<6-> Ecrire une relation de récurrence entre les différentes instances du problème en donnant les solutions des cas de base.
			\item<7-> Ecrire un programme en C permettant de répondre au problème.
			\item<8-> Construire la liste effective des pièces à rendre.
		\end{enumerate}
	\end{exampleblock}
\end{frame}


\begin{frame}{\Ctitle}{\stitle}
	\begin{exampleblock}{Résolution}
		\begin{enumerate}
			\item<2->\textcolor{black}{On note} :
			\begin{itemize}
				\item<3->\textcolor{black}{$s$ la somme à rendre,}
				\item<4->\textcolor{black}{$(p_i)_{1 \leq i \leq n}$ les valeurs des pièces rangées dans l'ordre \textit{croissant}}
				\item<5->\textcolor{black}{$m(s,k)$ le nombre minimal de pièce pour rendre la somme~$s$ en utilisant les pièces $(p_i)_{1 \leq i \leq k}$}
			\end{itemize}
			\onslide<6->\textcolor{black}{Avec ces notations, on doit donc trouver $m(s,n)$ et on dispose des relations suivantes :} \\
			\onslide<7->{\textcolor{black}{$\left\{ \begin{array}{lll}
							m(0,k)   & = & 0 \text{ pour tout } 1\leq k \leq n,                         \\
							m(s,0) & = & +\infty \text{ pour tout } s \in \N^*,                                                     \\
							m(s,k)   & = & m(s,k-1)    \text{ si }  s<p_k,                              \\
							m(s,k)   & = & \min\left\{ 1 + m(s-p_k,k), m(s,k-1) \text{  }\right\} \text{ sinon.}\end{array} \right.$}\medskip \\}
		\end{enumerate}
	\end{exampleblock}
\end{frame}

\begin{frame}{\Ctitle}{\stitle}
	\begin{exampleblock}{Résolution}
		\begin{enumerate}
			\addtocounter{enumi}{1}
			\item Programme en C 
			\onslide<2->{\inputpartC{\SPATH/rendu.c}{}{\small}{7}{16}}
			\onslide<3->{Comme pour la {\sc plssc}, on peut écrire une solution itérative ou une solution utilisant la mémoïsation.}
		\end{enumerate}
	\end{exampleblock}
\end{frame}

\begin{frame}{\Ctitle}{\stitle}
	\begin{exampleblock}{Reconstruction de la solution}
		\begin{enumerate}
			\addtocounter{enumi}{2}
			\item On construit une solution effective à partir des valeurs de la matrice~$m(s,k)$ \\
			      \renewcommand{\arraystretch}{1.3}
			      \begin{tabular}{|c|c|c|c|c|c|c|}
				      \cline{2-6}
					  \multicolumn{1}{c|}{} & \textcolor{blue}{0} &  \textcolor{blue}{1}  {\tiny ($\scriptstyle p_1=1$)}  & \textcolor{blue}{2} {\tiny ($\scriptstyle p_2=3$)} & \textcolor{blue}{3} {\tiny ($\scriptstyle p_3=4$)}& \textcolor{blue}{4} {\tiny ($\scriptstyle p_4=5$)} \\
					  \hline
				      	\textcolor{blue}{0}& 0 & 0 & 0 & 0 & 0 \\ 
						\textcolor{blue}{1}& $\infty$ & 1 & 1 & 1 & 1 \\ 
						\textcolor{blue}{2}& $\infty$ & 2 & 2 & 2 & \alt<2->{\circlenode[linecolor=blue]{r1}{\textcolor{blue}{2}}}{2} \\ 
						\textcolor{blue}{3}& $\infty$ & 3 & 1 & 1 & 1 \\ 
						\textcolor{blue}{4}& $\infty$ & 4 & 2 & 1 & 1 \\ 
						\textcolor{blue}{5}& $\infty$ & 5 & 3 & 2 & 1 \\ 
						\textcolor{blue}{6}& $\infty$ & 6 & 2 & 2 & 2 \\ 
						\textcolor{blue}{7}& $\infty$ & 7 & 3 & \alt<4->{\circlenode[linecolor=blue]{r0}{\textcolor{blue}{2}}}{2} & \circlenode[linecolor=red]{r2}{\textcolor{red}{2}} \\ 
					\hline
			      \end{tabular}
				  \onslide<3->{\ncbar[angleA=0, angleB=0, linestyle=dashed, linecolor=red]{->}{r2}{r1} \nbput{$\boxed{p_4}$}}
				  \onslide<4->{\ncline[linestyle=dashed, linecolor=red]{->}{r2}{r0}}
		\end{enumerate}
	\end{exampleblock}
\end{frame}


\end{document}
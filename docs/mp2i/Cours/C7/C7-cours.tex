\PassOptionsToPackage{dvipsnames,table}{xcolor}
\documentclass[10pt]{beamer}
\usepackage{Cours}

\begin{document}

\input{\detokenize{/home/fenarius/Travail/Cours/cpge-info/latex/MacrosCours.tex}}

% Numéro et titre de chapitre
\setcounter{numchap}{7}
\newcommand{\Ctitle}{\cnum {Terminaison, correction, complexité.}}
\newcommand{\SPATH}{/home/fenarius/Travail/Cours/cpge-info/docs/mp2i/files/C\thenumchap}

% Définition d'une structure de données
\makess{Objectifs}
\begin{frame}[fragile]{\Ctitle}{\stitle}
    \begin{block}{Algorithmique}
        Dans l'étude des algorithmes (\textcolor{blue}{algorithmique}), on s'intéresse aux trois problèmes suivantes :
        \begin{enumerate}
            \item<2-> \textcolor{blue}{terminaison} : peut-on garantir que l'algorithme se termine en un temps fini ? (ne concerne que les algorithmes avec des boucles non bornées)
            \item<3-> \textcolor{blue}{correction} : peut-on garantir que l'algorithme fournit la réponse attendue ?
            \item<4-> \textcolor{blue}{complexité} : evolution du temps d'exécution de l'algorithme en fonction de la taille des données. En particulier, le temps d'exécution d'un algorithme sur une entrée donnée sera-t-il \og raisonnable \fg ?
        \end{enumerate}
        \onslide<5->{L'algorithme étudié doit avoir une \textcolor{blue}{spécification} précise (entrées, sorties, préconditions, postconditions, effets de bord). On parle d'algorithmes (et non de programmes) car ces questions sont indépendantes de l'implémentation dans un langage de programmation quelconque.}
    \end{block}	
\end{frame}

\begin{frame}[fragile]{\Ctitle}{\stitle}
    \begin{exampleblock}{Exemple}
        Nous avons déjà rencontrés plusieurs algorithmes de tri (tri par insertion, tri par sélection, tri fusion). On cherche maintenant :
        \begin{itemize}
            \item <2-> A obtenir une \textcolor{blue}{preuve mathématique} que ces algorithmes se terminent et donc n'entrent jamais dans une boucle infinie quelques soient les données.
            \item <3-> A obtenir une \textcolor{blue}{preuve mathématique} que ces algorithmes trient effectivement les listes de nombres données en argument et ce quelques soient leur taille et les valeurs qu'elles contiennent.
            \item <4-> A comparer ces algorithmes en quantifiant leur efficacité (qui peut être mesuré de diverses façons).  
        \end{itemize}
    \end{exampleblock}
\end{frame}

\makess{Terminaison}
\begin{frame}[fragile]{\Ctitle}{\stitle}
\begin{exampleblock}{Exemple introductif}
    Que dire de la terminaison de la fonction suivante ?
    \inputpartC{\SPATH/intro_terminaison.c}{}{\small}{4}{12}
    \begin{itemize}
        \item<2-> \textcolor{OliveGreen}{Si {\tt n} est strictement négatif, c'est une boucle infinie.\\}
        \onslide<3->{\textcolor{gray}{\small On ne tient pas compte des dépassements de capacités sur les entiers.}}
        \item<4-> \textcolor{OliveGreen}{Sinon  elle termine.\\}
        \onslide<5->\textcolor{gray}{\small Dans le cas contraire, les valeurs prises par {\tt n} formeraient une suite strictement décroissante d'entiers naturels.}
    \end{itemize}
\end{exampleblock}
\end{frame}


\begin{frame}[fragile]{\Ctitle}{\stitle}
\begin{block}{Définitions}
    \begin{itemize}
        \item<2-> On dit qu'un algorithme \textcolor{blue}{termine} si il renvoie un résultat en un nombre fini d'étapes quels que soient les valeurs des entrées.
        \item<3-> Un \textcolor{blue}{variant de boucle} est une quantité :
            \begin{itemize}
            \item<4-> à valeurs dans $\N$.
            \item<5-> qui décroît \textit{strictement} à chaque passage dans la boucle.
            \end{itemize}
        \item<6-> Pour un algorithme récursif, un \textcolor{blue}{variant} est une quantité :
        \begin{itemize}
            \item<7-> à valeurs dans $\N$.
            \item<8-> qui décroît \textit{strictement} à chaque appel récursif.
        \end{itemize}
    \end{itemize}
\end{block}
\onslide<9->{
\begin{alertblock}{Preuve de la terminaison d'un algorithme}
    Pour prouver la terminaison d'un algorithme, il suffit de trouver un \textcolor{blue}{variant de boucle} pour chaque boucle non bornée qu'il contient. Et un \textcolor{blue}{variant} pour chaque fonction récursive.
\end{alertblock}}
\end{frame}

\begin{frame}[fragile]{\Ctitle}{\stitle}
\begin{exampleblock}{Exemple 1}
    On considère la fonction ci-dessous :
    \inputpartC{\SPATH/terminaison.c}{}{\small}{4}{14}
    \onslide<2-> En trouvant un variant de boucle, prouver la terminaison de cette fonction.
\end{exampleblock}
\end{frame}

\begin{frame}[fragile]{\Ctitle}{\stitle}
\begin{exampleblock}{Correction de l'exemple 1}
    \textcolor{OliveGreen}{Montrons que la variable {\tt a} est un variant de boucle, c'est à dire qu'elle prend ses valeurs dans $\N$ et  décroît strictement à chaque passage dans la boucle.}
    \begin{itemize}
        \item<2->{\textcolor{OliveGreen}{La valeur initiale de \texttt{a} est un entier naturel (précondition)}}
        \item<3->{\textcolor{OliveGreen}{A chaque tour de boucle la valeur de \texttt{a} diminue (de \texttt{b}$>0$).}}
        \item<4->{\textcolor{OliveGreen}{La nouvelle valeur de \texttt{a} est \texttt{a-b} qui est garantie positive par condition d'entrée dans la boucle}}
    \end{itemize}
    \onslide<5->\textcolor{OliveGreen}{Les trois éléments ci-dessus prouvent que la variable {\tt a} est un variant de la boucle {\tt while} de ce programme, par conséquent cette boucle se termine.}
\end{exampleblock}
\end{frame}

\begin{frame}[fragile]{\Ctitle}{\stitle}
    \begin{exampleblock}{Exemple 2}
        On considère la fonction ci-dessous :
        \inputpartOCaml{\SPATH/terminaison.ml}{}{\small}{1}{4}
        \onslide<2-> Prouver que cette fonction récursive termine
    \end{exampleblock}
    \end{frame}
    
    \begin{frame}[fragile]{\Ctitle}{\stitle}
    \begin{exampleblock}{Correction de l'exemple 2}
        \textcolor{OliveGreen}{Montrons que la longueur de la liste \kw{liste} est un variant, c'est à dire qu'elle prend ses valeurs dans $\N$ et  décroît strictement à chaque appel récursif.}
        \begin{itemize}
            \item<2->{\textcolor{OliveGreen}{La longueur d'une liste est à valeur dans $\N$}}
            \item<3->{\textcolor{OliveGreen}{A chaque appel récursif, on enlève un élément de \kw{liste} (sa tête) et donc la taille de la liste diminue de 1.}}
        \end{itemize}
        \onslide<4->\textcolor{OliveGreen}{Les deux éléments ci-dessus prouvent que la longueur de la liste est un variant et que donc cette fonction récursive termine.}
    \end{exampleblock}
    \end{frame}

    \begin{frame}[fragile]{\Ctitle}{\stitle}
        \begin{block}{Remarque}
            Dans les exemples ci-dessus la mise en évidence du variant est facile (et ce sera le cas en général dans nos algorithmes). Mais, les preuves de terminaison sont loin d'être toujours aussi évidentes.\\
            \onslide<2->{Par exemple, si on considère la fonction suivante :
            \inputpartOCaml{\SPATH/terminaison.ml}{}{\small}{6}{8}}
            \onslide<3-> Prouver sa terminaison reviendrait à prouver la conjecture de syracuse qui résiste aux mathématiciens depuis un siècle !
        \end{block}
    \end{frame}

\makess{Correction}
\begin{frame}[fragile]{\Ctitle}{\stitle}
    \begin{exampleblock}{Exemple introductif}
        Comment garantir que cette fonction renvoie bien la somme des éléments du tableau donné en argument ?
        \inputpartC{\SPATH/intro_correction.c}{}{\small}{5}{13}
        \textcolor{OliveGreen}{Par un raisonnement par récurrence sur $k$, on  prouve que \og{} après $k$ tours de boucle, {\tt s} contient la somme des $k$ premiers éléments de {\tt tab}\fg{}.}
    \end{exampleblock}
\end{frame}

\begin{frame}[fragile]{\Ctitle}{\stitle}
\begin{alertblock}{Correction d'un algorithme}
    On dira qu'un algorithme est \textcolor{red}{correct}
    \onslide<2-> lorsqu'il renvoie la réponse attendue pour n'importe quel donnée en entrée.\\
    \onslide<3-> On parle de \textcolor{blue}{correction partielle} quand le résultat est correct lorsque l'algorithme s'arrête. Et de \textcolor{blue}{correction totale} lorsque la correction est partielle et que l'algorithme se termine.
\end{alertblock}
\onslide<3->{\begin{block}{Tests et correction}
        \onslide<4->{Des tests ne permettent \textcolor{red}{pas} de prouver qu'un algorithme est correct.}
        \onslide<5->{En effet, ils ne permettent de valider le comportement de l'algorithme que dans quelques cas particuliers et jamais dans le cas général}
    \end{block}}
\end{frame}

\begin{frame}[fragile]{\Ctitle}{\stitle}
\begin{alertblock}{Preuve de la correction d'un algorithme}
    Pour prouver la correction d'un algorithme itératif, on utilise la notion \textcolor{red}{d'invariant de boucle}. C'est une propriété du programme qui
    \begin{itemize}
        \item<2-> est vraie à l'entrée dans la boucle.
        \item<3-> reste vraie à chaque itération si elle l'était à l'itération précédente.
    \end{itemize}
    \onslide<4->{La méthode est similaire à une récurrence mathématique (les deux étapes précédentes correspondent à l’initialisation et à l'hérédité).}
\end{alertblock}
\end{frame}

\begin{frame}[fragile]{\Ctitle}{\stitle}
\begin{exampleblock}{Exemple 1}
    On considère la fonction ci-dessous :
    \inputpartC{\SPATH/correction.c}{}{\small}{3}{15}
    \onslide<2-> En trouvant un invariant de boucle, montrer qu'à la sortie de la boucle, la variable \kw{cpt} contient le nombre de fois où \kw{elt} apparaît dans \kw{tab}
\end{exampleblock}
\end{frame}


\begin{frame}[fragile]{\Ctitle}{\stitle}
\begin{exampleblock}{Correction de l'exemple 1}
    \textcolor{OliveGreen}{On note  $\{ e_1, e_2, \dots, e_n \}$ les élements de \kw{tab} et  $k$ le nombre de tours de boucle\\}
    \onslide<2->{\textcolor{OliveGreen}{Montrons que la propriété : \\}   \og{} \textit{\kw{nb} contient le nombre de de fois où \kw{elt} apparaît dans les $k$ premiers éléments de \kw{tab}} \fg{}  \textcolor{OliveGreen}{est un invariant de boucle.}} 
    \begin{itemize}
        \item<4-> \textcolor{OliveGreen}{En entrant dans la boucle ($k=0$) la propriété est vraie car \kw{cpt} vaut 0.}
        \item<5-> \textcolor{OliveGreen}{Supposons la propriété vraie au $k$-ième tour de boucle alors, au tour suivant elle reste vraie puisqu'on ajoute 1 au compteur lorsque \kw{elt}$=e_{k+1}$}
    \end{itemize}
    \onslide<6->\textcolor{OliveGreen}{Cette propriété est donc bien un invariant de boucle.}
    \onslide<7->\textcolor{OliveGreen}{L'invariant de boucle reste vraie en sortie de boucle ce qui prouve que l'algorithme est correct.}
\end{exampleblock}
\end{frame}

\begin{frame}[fragile]{\Ctitle}{\stitle}
    \begin{block}{Correction d'un algorithme récursif}
        Dans le cas d'un algorithme récursif, la preuve de la correction peut s'obtenir :
        \begin{itemize}
            \item<1-> Directement à partir des identités mathématiques issues du code de la fonction.
            \item<2-> Par un raisonnement inductif, c'est à dire similaire à une récurrence.
        \end{itemize}
    \end{block}
    \onslide<3->{
    \begin{exampleblock}{Exemple 2 : factoriel récursif}
        \onslide<4->\inputpartOCaml{\SPATH/correction.ml}{}{\small}{1}{2}
        \onslide<5-> Les identités {\tt fact 0 = 1} et {\tt fact n =  n * fact (n-1)} si $n>0$, correspondent bien à la définition mathématique de la factorielle c'est à dire $0!=1$ et pour tout $n \in \N^*$, $n! = n \times (n-1)!$, donc cette fonction est correcte
    \end{exampleblock}}
\end{frame}

\begin{frame}[fragile]{\Ctitle}{\stitle}
    \begin{exampleblock}{Exemple 2 : liste des termes pairs}
        {\small On considère la fonction récursive suivante qui renvoie la liste des termes pairs.}
        \onslide<2->\inputpartOCaml{\SPATH/correction.ml}{}{\small}{4}{8}
        \onslide<3->{\small Pour prouver  que cette fonction est correcte on peut faire le raisonnement inductif suivant :}
        \onslide<4->{\small On note $P(n)$, la propriété : \og{} \textit{pour une liste de taille $n$, \kw{pairs} renvoie la liste des termes pairs de cette liste} \fg{}. Alors :}
        \begin{itemize}
            \item<5-> {\small $P(0)$ est vérifiée d'après le cas de base}
            \item<6-> {\small On suppose que $P(n)$ vérifié au rang $n$, et on considère une liste de taille $n+1$ notée {\tt h::t}. Comme  {\tt t} est de taille $n$, on lui applique l'hypothèse de récurrence et {\tt pair t} renvoie bien la liste des termes pairs.}
            \onslide<7->{\small  La formule de récursivité permet alors de conclure que $P(n+1)$ est vérifiée puisque {\tt pair h::t} renvoie h::(pair t) si {\tt h} est pair et {\tt pair t} sinon. }
        \end{itemize}
    \end{exampleblock}
\end{frame}

\makess{Complexité}
\begin{frame}[fragile]{\Ctitle}{\stitle}
    \begin{exampleblock}{Exemple introductif}
        Ecrire en C,
        \begin{enumerate}
            \item<2-> Une fonction {\tt recherche\_simple} qui prend en argument un tableau {\tt tab} et un entier {\tt n} et renvoie {\tt true} si {\tt n} est dans {\tt tab} et {\tt false} sinon. \\
            \onslide<3->\inputpartC{\SPATH/complexite.c}{}{\small}{6}{10}
            \item<4-> On suppose maintenant que le tableau est \textit{triée}, et pour effectuer la recherche, on procède par dichotomie. \onslide<5->{c'est à dire qu'on compare la valeur cherchée à l'élément situé au milieu du tableau, on relance alors la recherche sur la \textit{"bonne"} moitié.} \onslide<6->{Ecrire en C, une fonction itérative {\tt recherche\_dicho} qui correspond à ce nouvel algorithme.}
        \end{enumerate}
    \end{exampleblock}
\end{frame}

\begin{frame}[fragile]{\Ctitle}{\stitle}
    \begin{exampleblock}{Exemple introductif}
        \inputpartC{\SPATH/complexite.c}{}{\small}{12}{26}    
    \end{exampleblock}
\end{frame}

\begin{frame}[fragile]{\Ctitle}{\stitle}
    \begin{exampleblock}{Exemple introductif}
        Graphe des temps d'exécution (l'élément cherché est absent du tableau).
        \begin{center}
            \includegraphics[height=130px]{recherche.eps}
        \end{center}
        \begin{itemize}
            \item<2-> La recherche dichotomique est plus rapide.
            \item<3-> Temps proportionnel à la taille du tableau pour la recherche simple.
        \end{itemize}
    \end{exampleblock}
\end{frame}

\begin{frame}[fragile]{\Ctitle}{\stitle}
    \begin{alertblock}{Définition}
        La \textcolor{red}{complexité} d'un algorithme est une mesure de son efficacité.
        \onslide<2-> On parle notamment de :
        \begin{itemize}
            \item<3-> \textcolor{blue}{Complexité en temps} : le nombre d'opérations nécessaire à l'exécution d'un algorithme.
            \item<4-> Complexité en mémoire : l'occupation mémoire en fonction de la taille des données.
        \end{itemize}
        \onslide<5->{Ces deux éléments varient en fonction de la taille et de la nature des données. On s'intéressera principalement à la complexité temporelle.}
    \end{alertblock}
    \onslide<6->{
    \begin{exampleblock}{Exemple}
        La recherche dichotomique est plus rapide que la recherche simple, elle demande moins d'opérations, sa complexité doit donc être meilleure
    \end{exampleblock}}
\end{frame}


\begin{frame}[fragile]{\Ctitle}{\stitle}
    \begin{alertblock}{Calcul de la complexité temporelle}
        \begin{itemize}
            \item<1-> On considère certaines opérations comme élémentaires, leur coût est alors majoré par une constante.\\
            \onslide<2->{\textcolor{OliveGreen}{\small Par exemple les opérations arithmétiques, les tests, les affectations \dots }}
            \item<3-> On exprime le coût de l'algorithme pour un entrée de taille {\tt n} en nombre d'opérations élémentaires nécessaires à sa réalisation.
            \onslide<4->{\textcolor{OliveGreen}{
                \begin{itemize}
                    \item<5->\textcolor{OliveGreen}{La fonction \mintinline{ocaml}{let f x = x*x + 2*x + 3;;} demande 5 opérations quelque soit la taille de l'entrée {\tt x}. }
                    \item<6->\textcolor{OliveGreen}{La fonction suivante en C :}
                    \inputpartC{\SPATH/intro_correction.c}{}{\small}{16}{20}    
                    \onslide<7->\textcolor{OliveGreen}{demande $4\,n+3$ opérations où $n$ est la taille du tableau.}
                \end{itemize}}}
        \end{itemize}
    \end{alertblock}
\end{frame}

\begin{frame}[fragile]{\Ctitle}{\stitle}
    \begin{alertblock}{Types de complexité}
    Le nombre d'opérations d'un algorithme peut varier en fonction de la nature des données, on est donc à amener à définir :
        \begin{itemize}
            \item<4-> la \textcolor{blue}{complexité dans le meilleur cas}, le nombre minimal d'operations effectuées par un algorithme sur une entrée de taille $n$.
            \item<5-> la \textcolor{blue}{complexité dans le pire cas}, le nombre maximal d'operations effectuées par un algorithme sur une entrée de taille $n$.
            \item<6-> la \textcolor{blue}{complexité en moyenne}, le nombre moyen d'operations effectuées par un algorithme sur un ensemble d'entrées de taille $n$.
        \end{itemize}
    \onslide<7->{En général, on s'intéresse à la \textcolor{blue}{complexité dans le pire cas}, car on cherche à \textcolor{blue}{majorer} le nombre d'opérations effectuées par l'algorithme.}
    \end{alertblock}
\end{frame} 



\begin{frame}[fragile]{\Ctitle}{\stitle}
    \begin{exampleblock}{Exemple : recherche simple dans un tableau}
        {\small On considère la fonction ci dessous qui recherche si l'\kw{elt} est ou non dans {\kw{tab}}}
        \inputpartC{\SPATH/complexite.c}{}{\small}{6}{10}
    \begin{enumerate}
        \item<3-> {\small En donnant un exemple, montrer que cette fonction peut renvoyer le résultat en effectuant une seule comparaison.}
        \item<4-> {\small En donnant un exemple, montrer que cette fonction peut renvoyer le résultat en effectuant $n$ comparaison.}
        \item<5-> {\small On suppose à présent qu'on cherche un élément $a$ qui se trouve à un seul exemplaire dans le tableau et que les positions sont équiprobables. C'est à dire que pour tout $i \in \intN{0}{n-1}$ $a$ se trouve à l'indice $i$ avec la probabilité $\dfrac{1}{n}$. Quel sera le nombre \textit{moyen} de comparaison à effectuer avec de renvoyer le résultat ?}
    \end{enumerate}
\end{exampleblock}
\end{frame}

\begin{frame}[fragile]{\Ctitle}{\stitle}
    \begin{exampleblock}{Exemple : recherche simple dans un tableau}
    \begin{enumerate}
        \item<2-> \textcolor{OliveGreen}{\small Si l'élément cherché est en première position dans le tableau on effectue une seule comparaison.}
        \item<3-> \textcolor{OliveGreen}{\small Si l'élément cherché n'est pas dans le tableau (ou qu'il y figure en dernière position) on effectue $n$ comparaison.}
        \item<4-> \textcolor{OliveGreen}{\small on note $X$ le nombre de comparaisons avant de trouver $a$, alors $p(X=k) = \dfrac{1}{n}$. Donc,\\
        $\begin{array}{lll}
         \onslide<5->{E(X) &=& \displaystyle{\sum_{k=1}^{n} k\,\dfrac{1}{n}} \\}
         \onslide<6->{E(X) &=&  \dfrac{n+1}{2}}
        \end{array}$
        }
    \end{enumerate}
    \onslide<7->\textcolor{OliveGreen}{Le nombre de comparaisons varie donc avec les données du problème.}
\end{exampleblock}
\end{frame}

\begin{frame}[fragile]{\Ctitle}{\stitle}
    \begin{block}{Majoration asymptotique}
        \begin{itemize}
            \item En pratique, seul une \textcolor{blue}{majoration asymptotique} du coût $C(n)$ d'un algorithme nous intéresse et pas sa détermination exacte.\\
            \onslide<2->\textcolor{OliveGreen}{Par exemple, si le coût de l'algorithme est $C(n) = 3n + 15$ opérations élémentaires, on dira  que $C(n)$ est majoré asymptotiquement par $n$.} 
            \item<3-> L'outil mathématique associé est la notion de \textcolor{blue}{domination} d'une suite : \\
				Etant donné deux suites $(u_n)_{n \in \N}$ et $(v_n)_{n \in \N}$ à valeurs strictement positives. On dit que $(u_n)_{n \in \N}$ est dominée par $(v_n)_{n \in \N}$ lorsqu'il existe un entier $K>0$ et un rang $N \in \N$ tel que : \\ \onslide<3->{$\forall n \in \N, n>N$, on a  $u_n \leq k v_n$.\\}
				\onslide<4->{On note alors \textcolor{BrickRed}{$u = O(v)$} (ou encore $u \in O(v)$) et on dit que $u$ est un grand $O$ de $v$.\\}
                \onslide<5->{\og{} $u_n$ est inférieur à $v_n$ à une constante multiplicative près et pour $n$ assez grand \fg.}
        \end{itemize}
    \end{block}
\end{frame}

\begin{frame}[fragile]{\Ctitle}{\stitle}
    \begin{exampleblock}{Exemples}
        \begin{itemize}
            \item<1-> Montrer que $(a_n)$ de terme général $10n + 3$ est un $O(n)$\\
            \onslide<4->{\textcolor{OliveGreen}{$10 n + 3 < 11 n$ pour $n>3$}}
            \item<2-> Montrer que $(b_n)$ de terme général $n^2 + n + 1$ est un $O(n^2)$\\
            \onslide<5->{\textcolor{OliveGreen}{$n^2 + n + 1 < 2 n^2$ pour $n>2$}}
            \item<3-> Déterminer un grand $O$ de $(c_n)$ de terme général $ 7 n + \ln(n)$\\
            \onslide<6->{\textcolor{OliveGreen}{Comme $\ln(n) < n$, $c_n<8n$ et donc $(c_n)$ est un $O(n)$.}}
        \end{itemize}
    \end{exampleblock}
\end{frame}

\begin{frame}[fragile]{\Ctitle}{\stitle}
    \begin{block}{Autres expressions de la complexité}
        \begin{itemize}
            \item<1-> On note $u_n = \Omega(v_n)$ si $v_n = O(v_n)$ 
            \item<2-> On note $v_n = \Theta(v_n)$ si $u_n = O(v_n)$ et $v_n = O(u_n)$ \\
            \onslide<3-> Cette notation peut être utilisé pour donner un équivalent (plutôt qu'une majoration de la complexité)
        \end{itemize}
    \end{block}
\end{frame}



\begin{frame}[fragile]{\Ctitle}{\stitle}
	\begin{alertblock}{Complexités usuelles}
		\renewcommand{\arraystretch}{1.3}
		\begin{tabularx}{\textwidth}{|c|l|X|}
			\hline
			\textcolor{blue}{Complexité}   & \textcolor{blue}{Nom}                  & \textcolor{blue}{Exemple}                                             \\
			\hline
			\leavevmode\onslide<2->{$O(1)$}           & \leavevmode\onslide<2->{Constant}      & \leavevmode\onslide<2->{Accéder à un élément d'une liste}             \\
			\hline
			\leavevmode\onslide<3->{$O(\log(n))$}     & \leavevmode\onslide<3->{Logarithmique}   & \leavevmode\onslide<3->{Recherche dichotomique dans une liste}        \\
			\hline
			\leavevmode\onslide<4->{$O(n)$          } & \leavevmode\onslide<4->{Linéaire     } & \leavevmode\onslide<4->{Recherche simple dans une liste}              \\
			\hline
			\leavevmode\onslide<5->{$O(n\log(n))$   } & \leavevmode\onslide<5->{Linéaritmique} & \leavevmode\onslide<5->{Tri fusion}                                   \\
			\hline
			\leavevmode\onslide<6->{$O(n^2)$        } & \leavevmode\onslide<6->{Quadratique  } & \leavevmode\onslide<6->{Tri par insertion d'une liste }               \\
			\hline
			\leavevmode\onslide<7->{$O(2^n)$        } & \leavevmode\onslide<7->{Exponentielle} & \leavevmode\onslide<7->{Algorithme par force brute pour le sac à dos} \\
			\hline
		\end{tabularx}
	\end{alertblock}
\end{frame}

\begin{frame}[fragile]{\Ctitle}{\stitle}
	\begin{block}{Représentation graphique}
		\begin{center}
			\includegraphics[height=6cm]{complexite.eps}
		\end{center}
	\end{block}
\end{frame}

\begin{frame}[fragile]{\Ctitle}{\stitle}
	\begin{block}{Temps de calcul effectif}
		Sur un ordinateur réalisant 100 million d'opérations par seconde :
		\renewcommand{\arraystretch}{1.3}
		\begin{tabular}{|c|c|c|c|c|c|}
			\hline
			\textcolor{blue}{Complexité} & $n=10$                      & $n=100$                     & $n=1000$                    & $n=10^6$                    & $n=10^9$                      \\
			\hline
			$O(\log(n))$   & \textcolor{green}{\faCheck} & \textcolor{green}{\faCheck} & \textcolor{green}{\faCheck} & \textcolor{green}{\faCheck} & \textcolor{green}{\faCheck} \\
			\hline
			$O(n)$   & \textcolor{green}{\faCheck} & \textcolor{green}{\faCheck} & \textcolor{green}{\faCheck} & \textcolor{green}{\faCheck} & $\simeq 10$s \\
			\hline
			$O(n)\log(n)$   & \textcolor{green}{\faCheck} & \textcolor{green}{\faCheck} & \textcolor{green}{\faCheck} & \textcolor{green}{\faCheck} & $\simeq 1,5$ mn \\
			\hline
			$O(n^2)$   & \textcolor{green}{\faCheck} & \textcolor{green}{\faCheck} & \textcolor{green}{\faCheck} & $\simeq 3$ h & $\simeq 300$ ans \\
			\hline
			$O(2^n)$   & \textcolor{green}{\faCheck} &  \textcolor{red}{\faTimes} &  \textcolor{red}{\faTimes} &  \textcolor{red}{\faTimes} &  \textcolor{red}{\faTimes} \\
			\hline
		\end{tabular}
	\end{block}
\end{frame}

\begin{frame}[fragile]{\Ctitle}{\stitle}
	\begin{exampleblock}{Exemples}
		\begin{itemize}
			\item<1-> On suppose qu'on dispose d'un algorithme de complexité linéaire travaillant sur une liste, il traite une liste de \numprint{1000} éléments en \numprint{0.015} secondes. Donner une estimation du temps de calcul pour une liste de \numprint{250000} éléments.\\
				\onslide<2-> {\textcolor{OliveGreen}{La taille des données a été multiplié par 250, la complexité étant lineaire le temps de calcul sera aussi approximativement multiplié par 250. \\}}
				\onslide<3->{\textcolor{OliveGreen}{$0.015 \times 250 = 3.75$, on peut donc prévoir un temps de calcul d'environ 3,75 secondes}}
			\item<4-> Même question pour un algorithme de complexité quadratique qui traite une liste de \numprint{1000} éléments en \numprint{0.07} secondes.\\
				\onslide<5-> {\textcolor{OliveGreen}{La taille des données a été multiplié par 250, la complexité étant quadratique le temps de calcul sera  approximativement multiplié par $250^2=62500$ \\}}
				\onslide<6->{\textcolor{OliveGreen}{$0.07 \times 62\,500 = 4375$, on peut donc prévoir un temps de calcul d'environ $4\,375$ secondes, c'est à dire près d'une heure et 15 minutes !}}
		\end{itemize}
	\end{exampleblock}
\end{frame}

\makess{Exemple résolu: recherche dichotomique}
\begin{frame}[fragile]{\Ctitle}{\stitle}
    \begin{exampleblock}{Enoncé : recherche dichotomique}
    On reprend l'exemple de la recherche dichotomique dans un tableau trié:
    \inputpartC{\SPATH/complexite.c}{}{\tiny}{12}{26}
    \begin{itemize}
        \item<2-> Prouver que cet algorithme termine
        \item<3-> Prouver qu'il est correct
        \item<4-> Donner sa complexité.
    \end{itemize}
\end{exampleblock}
\end{frame}

\begin{frame}[fragile]{\Ctitle}{\stitle}
    \begin{exampleblock}{Correction : recherche dichotomique}
        \begin{itemize}
            \item<2->\textcolor{OliveGreen}{Montrons que les valeurs prises par {\tt fin - deb} est un variant de boucle.
            \begin{itemize}
                \item\textcolor{OliveGreen}{{\tt fin-deb} est positif à l'entrée dans la boucle (le tableau est supposé non vide)}
                \item\textcolor{OliveGreen}{{\tt fin-deb} décroît strictement à chaque itération car soit {\tt fin} diminue, soit {\tt }deb augmente}
            \end{itemize}
            }
            \item<3->\textcolor{OliveGreen}{Si la fonction renvoie {\tt true} alors l'élément cherché est bien présent dans le tableau car le test {\tt tab[milieu]==elt} est vraie. Montrons maintenant que si la fonction renvoie {\tt false} alors l'élément n'est pas dans le tableau. Pour cela on montre la propriété suivante : $P$ : \og{} Si {\tt elt} est dans {\tt tab} alors il se trouve entre les indices {\tt deb} et {\tt fin} \fg{}. En effet, :
            \begin{itemize}
                \item<4->\textcolor{OliveGreen}{Cette propriété est vraie à l'entrée dans la boucle puisque {\tt deb=0} et {\tt fin=size-1} la totalité du tableau est couverte.}
                \item<5->\textcolor{OliveGreen}{Cette propriété reste vraie à chaque tour de boucle car puisque le tableau est trié, si {\tt elt} est strictement plus grand que {\tt tab[milieu]} alors il se situe forcément après l'indice {\tt milieu} (et strictement avant dans le cas contraire)}
            \end{itemize}}
        \end{itemize}
    \end{exampleblock}
\end{frame}

\begin{frame}[fragile]{\Ctitle}{\stitle}
    \begin{exampleblock}{Correction : recherche dichotomique}
        \begin{itemize}
        \item<1->\textcolor{OliveGreen}{La boucle ne contient que des opérations élémentaires, le coût d'un tour de boucle est donc $O(1)$.}
        \item<2->\textcolor{OliveGreen}{La taille de l'intervalle de recherche est initialement la taille {\tt n} du tableau et est divisé par deux à chaque itération. Donc le nombre de tours de boucle est un $O(\log(n))$.}
        \item<3->\textcolor{OliveGreen}{En conclusion, la recherche dichotomique a une complexité logarithmique.}
        \end{itemize}
    \end{exampleblock}
\end{frame}


\makess{Exemple résolu : tri sélection}
\begin{frame}[fragile]{\Ctitle}{\stitle}
    \begin{exampleblock}{Tri sélection}
        \begin{itemize}
            \item<1-> Rappeler l'algorithme du tri par sélection \\
            \onslide<2->{\textcolor{OliveGreen}{On note $n$ la taille du tableau, pour chaque entier $i=0 \dots n-1$, on échange l'élément situé à l'indice $i$ avec le minimum du tableau depuis l'indice $i$.}}
            \item<3-> Montrer que l'algorithme termine\\
            \onslide<4->{\textcolor{OliveGreen}{L'algorithme ne contient pas de boucle non bornées donc sa terminaison est garantie.}}
            \item<5-> Montrer qu'il est correct
            \item<6-> Déterminer sa complexité
        \end{itemize}
    \end{exampleblock}
\end{frame}


\begin{frame}[fragile]{\Ctitle}{\stitle}
    \begin{exampleblock}{Correction du tri sélection}
        \textcolor{OliveGreen}{Pour montrer que l'algorithme est correct, on prouve  la propriété : $P(k)$ : \og{} après $k$ itérations, les $k$ premiers éléments du tableau sont ceux du tableau trié pour $k=0 \dots n$.}
        \begin{itemize}
            \item<2->\textcolor{OliveGreen}{Pour $k=0$, la propriété est vraie (aucun élément n'est encore trié)}
            \item<3->\textcolor{OliveGreen}{En supposant $P(k)$ vraie (c'est à dire les $k$ premiers éléments du tableau sont ceux du tableau trié), on montre $P(k+1)$. L'algorithme consiste à placer le minimum des éléments restants à l'indice $k+1$, cet élément est bien celui d'indice $k+1$ dans le tableau trié (il est supérieur aux éléments situés avant et inférieur à ceux qui restent à trier). Et donc $P(k+1)$ est vraie.}
        \end{itemize}
    \end{exampleblock}
\end{frame}

\begin{frame}[fragile]{\Ctitle}{\stitle}
    \begin{exampleblock}{Complexité du tri sélection}
        En notant $n$, la taille du tableau
        \begin{itemize}
            \item<1-> Chaque recherche de minimum parcourt au plus la totalité du tableau et est donc un $O(n)$.
            \item<2-> Cette recherche est effectuée $n-1$ fois et est donc aussi un $O(n)$.
            \item<3-> En conclusion la tri par sélection a un complexité en $O(n^2)$.
        \end{itemize}
    \end{exampleblock}
\end{frame}


\end{document}
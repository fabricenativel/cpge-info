\documentclass[11pt,a4paper]{article}

\usepackage{Act}

\begin{document}
\input{\detokenize{/home/fenarius/Travail/Cours/cpge-info/latex/Macros.tex}}
\newcommand{\SPATH}{/home/fenarius/Travail/Cours/cpge-info/docs/mp2i/files/}

\ModeExercice
\Colles{S21/S22/S23}{Structures de données}

\setcounter{Exercise}{0}

\begin{Exercise}[title = {Nombre d'arêtes}]
    \Question{Rappeler la définition d'un arbre binaire.}
    \Question{Soit $a$ un arbre binaire à $n$ noeuds ($n \geqslant 1$), montrer que $a$ possède $n-1$ arêtes.}
    \Question{On rappelle l'implémentation des arbres en OCaml utilisée en cours :
    \inputpartOCaml{\SPATH/C12/abi.ml}{}{}{1}{3}
    En utilisant cette implémentation, écrire une fonction {\tt nb\_aretes} de signature {\tt ab -> int} et qui renvoie le nombre d'arêtes d'un arbre binaire
    }
\end{Exercise}

\begin{Exercise}[title = {Reconstruction}]

\end{Exercise}


\begin{Exercise}[title = {Tester si un arbre est un {\sc abr}}]

\Question{Rappeler la définition d'un arbre binaire de recherche}
\Question{Proposer deux méthodes de complexité linéaire permettant de vérifier qu'un arbre est bien un {\sc abr}.}
\Question{Donner l'implémentation de l'une au moins des méthodes.}

\end{Exercise}


\end{document}
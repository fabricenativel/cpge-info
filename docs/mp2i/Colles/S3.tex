\documentclass[11pt,a4paper]{article}

\usepackage{Act}


\begin{document}
\input{\detokenize{/home/fenarius/Travail/Cours/cpge-info/latex/Macros.tex}}
\newcommand{\SPATH}{/home/fenarius/Travail/Cours/cpge-info/docs/mp2i/files/}

\ModeExercice

\Colles{S24/S25/S26}{Tas, induction}

\setboolean{corrige}{false}

\setcounter{Exercise}{0}


\begin{Exercise}[title = {Insertion dans un tas}]
	\Question{Rappeler la défintion d'un tas binaire (min)}
	\Question{On suppose qu'un tas est représenté par un tableau $t = (t_0,\dots t_{n-1})$. Lorsqu'ils existent quels sont les indices des fils de $t_i$ ?}
	\Question{Quel est l'indice (lorsqu'il existe) du père de $t_i$ ?}
	\Question{Rappeler le principe d'insertion d'un nouvel élément dans un tas binaire}
	\Question{Vérifier que l'arbre binaire suivant possède bien la structure de tas :
		\begin{center}
			\pstree[arrows=->,treesep=1cm,levelsep=1cm]{\TCircle{3}}
			{\pstree{\TCircle{6}}
				{\pstree{\TCircle{10}}
					{\TCircle{17} \TCircle{11}}
					\pstree{\TCircle{9}}
					{\TCircle{15}
						\TCircle{18}
					}}\pstree{\TCircle{5}}
				{\TCircle{7}
					\TCircle{13}
				}}
		\end{center}
	}
    \Question{Détailler les étapes de l'insertion de 4 dans le tas précédent.}
    \Question{On rappelle les structures de données vues en cours et permettant de représenter un tas :
    \begin{itemize}
        \item En langage C :
        \inputpartC{tas.c}{}{}{6}{12}
        \item En OCaml : 
        \inputpartOCaml{tas.ml}{}{}{1}{1}
    \end{itemize}
    Dans le langage de votre choix, implémenter l'algorithme d'insertion dans un tas.
    }
\end{Exercise}

\begin{Exercise}[title = {Extraction du minimum dans un tas}]
    \Question{Rappeler la défintion d'un tas binaire (min)}
	\Question{On suppose qu'un tas est représenté par un tableau $t = (t_0,\dots t_{n-1})$. Lorsqu'ils existent quels sont les indices des fils de $t_i$ ?}
	\Question{Quel est l'indice (lorsqu'il existe) du père de $t_i$ ?}
	\Question{Rappeler le principe d'extraction du minimum d'un tas binaire}
	\Question{Vérifier que l'arbre binaire suivant possède bien la structure de tas :
		\begin{center}
			\pstree[arrows=->,treesep=1cm,levelsep=1cm]{\TCircle{3}}
			{\pstree{\TCircle{6}}
				{\pstree{\TCircle{10}}
					{\TCircle{17} \TCircle{11}}
					\pstree{\TCircle{9}}
					{\TCircle{15}
						\TCircle{18}
					}}\pstree{\TCircle{5}}
				{\TCircle{7}
					\TCircle{13}
				}}
		\end{center}
	}
    \Question{Détailler les étapes de l'extraction du minimum dans le tas précédent.}
    \Question{On rappelle les structures de données vues en cours et permettant de représenter un tas :
    \begin{itemize}
        \item En langage C :
        \inputpartC{tas.c}{}{}{6}{12}
        \item En OCaml : 
        \inputpartOCaml{tas.ml}{}{}{1}{1}
    \end{itemize}
    Dans le langage de votre choix, implémenter l'algorithme d'extraction du minimum dans un tas.
    }
\end{Exercise}

\begin{Exercise}[title = {Ensemble inductif}]\\
    Soit $X$ l'ensemble construit par induction à partir :
    \begin{itemize}
        \item des axiomes $X_0 = {0}$
        \item des règles d'inférence $r_1 : n \mapsto n+5$, $r_2 : n \mapsto n+2$ et $r_3 : n \mapsto -n$
    \end{itemize}
    \Question{En donnant une suite de règle permettant de l'obtenir, montrer que $11 \in X$.}
    \Question{La définition de $X$ est-elle ambigüe ?}
    \Question{Prouver par induction structurelle que $X \subset \Z$.}
    \Question{Montrer que $X = \Z$.}
    \Question{On définit en OCaml le type :
    \inputpartOCaml{def_inductive.ml}{}{}{1}{5}
    Ecrire une fonction {\tt res nb -> int} qui prend en renvoie l'entier associée à la suite d'applications des règles d'inférences. Par exemple {\tt res (R3 (R1 (R2 Z))) renvoie $-7$.}
    }
\end{Exercise}

\begin{Exercise}[title = {Recherche d'un pic par la méthode diviser pour régner}]\\
    On dit qu'un tableau $t$ de taille $n$ présente un \textit{pic} en position $p$ si et seulement si :
    \begin{itemize}
        \item toutes les valeurs de $t$ sont distinctes
        \item le sous tableau $t_0,\dots t_{p}$ est trié par ordre croissant
        \item le sous tableau $t_p,\dots t_{n-1}$ est trié par ordre décroissant.
    \end{itemize}
    \Question{Justifier rapidement que si un tableau présente un pic alors ce pic est unique.}
    \Question{Ecrire dans le langage de votre choix, une fonction itérative de complexité linéaire qui prend en entrée un tableau présentant un pic et renvoyant la position de ce pic.}
    \Question{On veut maintenant utiliser une méthode diviser pour régner.}
    \subQuestion{Donner le cas de base et sa solution}
    \subQuestion{Si on divise le tableau en deux moitiés, montrer qu'on peut déterminer dans laquelle se trouve le pic.}
    \subQuestion{Donner l'implémentation de la méthode diviser pour régner dans le langage de votre choix.}
    \subQuestion{Donner sa complexité.}

\end{Exercise}

\end{document}
\PassOptionsToPackage{dvipsnames,table}{xcolor}
\documentclass[11pt,a4paper]{article}

\usepackage{DS}

\begin{document}
\input{\detokenize{/home/fenarius/Travail/Cours/cpge-info/latex/Macros.tex}}
\ModeConcours

\DS{PC}{2}{Février 2025}
\newcommand{\SPATH}{/home/fenarius/Travail/Cours/cpge-info/docs/itc/Evaluations/DS/DS2}
\setboolean{corrige}{false}
\alertbox{\danger}{Remarques et consignes importantes}{
	\begin{itemize}
		\item[\textbullet] On pourra toujours librement utiliser une fonction demandée à une question précédente même si cette question n'a pas été traitée.
		\item[\textbullet] Veillez à présenter vos idées et vos réponses partielles même si vous ne trouvez pas la solution complète à une question.
		\item[\textbullet] La clarté et la lisibilité de la rédaction et des programmes sont des éléments de notation.
	\end{itemize}
}

\begin{Exercise}[title={Problème du sac à dos}]

	\medskip
	On dispose d'un sac à dos pouvant contenir un poids maximal noté $P$ et de $n$ objets ayant chacun un poids $\left(p_i\right)_{0\leqslant i \leqslant n-1}$ et une valeur $\left(v_i\right)_{0\leqslant i \leqslant n-1}$. On cherche à remplir le sac à dos de manière à maximiser la valeur totale des objets contenus dans le sac sans dépasser le poids maximal $P$.
	Par exemple, si on dispose des objets suivants :
	\begin{itemize}
		\item un objet de poids $p_0 = 4$ et de valeur $v_0 = 20$,
		\item un objet de poids $p_1 = 5$ et de valeur $v_1 = 28$,
		\item un objet de poids $p_2 = 6$ et de valeur $v_2 = 36$,
		\item un objet de poids $p_3 = 7$ et de valeur $v_3 = 50$,
	\end{itemize}
	et qu'on suppose que le poids maximal du sac est $10$ alors un choix possible serait de prendre l'objet 3, aucun autre objet ne rentre alors dans le sac et la valeur du sac est  de $50$ avec un poids de 7. Une autre possibilité plus intéressante serait de choisir les objets 0  et 2, la valeur totale serait alors de $56$ et le poids du sac de $10$.\smallskip

	Dans toute la suite de l'exercice on supposera que les poids et les valeurs des objets sont fournis sous la forme d'une liste de Python contenant les tuples {\tt (poids, valeur)} représentant les objets. Par exemple, les objets précédents seraient représentés par la liste suivante :
	\begin{minted}{python}
objets = [(4, 20), (5, 28), (6, 36), (7, 50)]
\end{minted}

	On propose de représenter un choix d'objets par une liste contenant des 0 et des 1. Si le $i$-ème élément de la liste vaut $1$ alors l'objet $i$ est choisi, s'il vaut $0$ alors l'objet $i$ n'est pas choisi. Par exemple, pour les objets précédents, le choix de prendre uniquement l'objet 3 serait représenté par la liste \mintinline{python}{[0, 0, 0, 1]} et le choix de prendre les objets 0 et 2 serait représenté par la liste \mintinline{python}{[1, 0, 1, 0]}.\medskip\\
	\ExePart[name={Approche par recherche exhaustive}]\\
	La recherche exhaustive consiste à énumérer tous les choix possibles d'objet et à calculer la valeur ainsi que le poids pour chaque choix, on retient alors le choix qui maximise la valeur du sac sans dépasser le poids maximal. 
	\Question{Justifier rapidement que le nombre possible de choix d'objet est $2^n$.}
	\tcor{Pour chaque objet, on a deux choix possibles, le prendre ou non, comme il y a $n$ objets, le nombre total de choix est $2^n$.}
	\Question{Ecrire une fonction {\tt valeur\_poids} qui prend en arguments, une liste d'objets ainsi qu'un choix d'objet (sous la forme indiquée ci-dessus) et qui renvoie le poids et la valeur du sac correspondant à ce choix. Par exemple avec la liste {\tt objets} donnée en exemple plus haut,  {\tt valeur\_poids(objets, [1, 0, 1, 0])} doit renvoyer $(56, 10)$.}
	\ifcorrige
	\corpartPython{sac.py}{}{}{1}{9}
	\fi
	\Question{Donner la complexité de la fonction {\tt valeur\_poids} en fonction de $n$.}
	\tcor{On parcourt la liste des objets en effectuant uniquement des opérations élémentaires, la complexité est donc en $\mathcal{O}(n)$.}
	\Question{En déduire la complexité d'une méthode qui pour chaque choix possible d'objet calculerait la valeur du sac ainsi que son poids et renverrait le choix optimal.}
	\tcor{Il y a $2^n$ choix d'objets possibles, et pour chacun de ces choix on doit effectuer $n$ opérations afin de calculer la valeur et le poids du sac. La complexité de cette méthode est donc en $\mathcal{O}(n2^n)$.}

	\medskip
	\ExePart[name={Stratégie gloutonne}]

	On considère la stratégie gloutonne suivante : on trie les objets par ordre décroissant de leur rapport valeur/poids et on les prend dans cet ordre jusqu'à ce que le poids maximal soit atteint.
	\Question{Vérifier qu'en appliquant cette stratégie à la liste d'objets :
		\begin{minted}{python}
[(4, 30), (5, 34), (6, 36), (7, 49), (10, 74)]
\end{minted}
		et un poids maximal de 10, on n'obtient pas la meilleure solution.}
	\tcor{En triant les objets par rapport poids/valeur, on obtient l'ordre suivant :
	\begin{itemize}
	\item objet 0 (poids : 4, valeur : 30, rapport : 7.5)
	\item objet 4 (poids : 10, valeur : 74, rapport : 7.4)
	\item objet 3 (poids : 7, valeur : 49, rapport : 7.0)
	\item objet 1 (poids : 5, valeur : 34, rapport : 6.8)
	\item objet 2 (poids : 6, valeur : 36, rapport : 6.0)
	\end{itemize}
	On parcourt donc cette liste en prenant les objets tant qu'ils ne dépassent pas la contrainte de poids. Cela conduit à choisir les objets 0 et 1 pour une valeur de 64. Cette combinaison n'est pas optimale puisque l'on peut obtenir une valeur de 74 en prenant uniquement l'objet 4.}
	\Question{Ecrire une fonction {\tt glouton}, qui prend en arguments une liste d'objets qu'on suppose \textit{déjà triée par ordre décroissant du rapport valeur/poids} et un poids maximal et qui renvoie la valeur maximale que l'on peut obtenir en appliquant la stratégie gloutonne.}
	\ifcorrige
	\corpartPython{sac.py}{}{}{12}{20}
	\fi
	\ExePart[name={Approche par programmation dynamique}]\\
	On propose de résoudre le problème du sac à dos par programmation dynamique. Pour tout $i \in \intN{0}{n}$, on note $V(i, p)$ la valeur maximale que l'on peut obtenir avec les objets à partir de celui d'indice $i$ et un poids maximal $p$. Par exemple s'il y a 5 objets (numérotés de 0 à 4), $v(2,10)$ est le poids maximal atteint pour un sac de poids maximal 10 en ne considérant que les objets 2, 3 et 4.
	\Question{Donner $V(i, 0)$ pour $i \in \intN{0}{n-1}$ et $V(n, p)$ pour $p \in \intN{0}{P}$.}
	\tcor{
		$V(i,0)$ est la valeur maximal d'un sac de poids maximal nul c'est donc $0$.
		$V(n,p)$ vaut 0 aussi car on n'utilise aucun objet (les numéros des objets vont de $0$ à $n-1$)
	}
	\Question{Donner les relations de récurrence liant $v(i, p)$ à $v(i+1, p)$ et $v(i+1, p-p_i)$.\\
	{\small \aide \;} Indication : on pourra considérer deux cas, celui où l'objet $i$ est n'est pas pris et celui où il l'est (dans ce cas on a nécessairement $p \geqslant p_i$).}
	\tcor{Pour déterminer $v(i,p)$, si $p\geqslant p_i$, on choisit la valeur maximale entre les deux possiblités suivantes :
	\begin{itemize}
		\item ne pas prendre l'objet $i$, la valeur du sac est alors $v(i+1, p)$,
		\item prendre l'objet $i$, la valeur du sac est alors $v_i + v(i+1, p-p_i)$.
	\end{itemize}
	Sinon, on ne peut pas prendre l'objet $i$ et donc la valeur du sac est $v(i,p) = v(i+1, p)$.
	On a donc la relation de récurrence suivante : $v(i,p) = \max(v(i+1, p), v_i + v(i+1, p-p_i))$ si $p \geqslant pi$ et $v(i,p) = v(i+1, p)$ sinon.}
	\Question{Écrire une fonction récursive {\tt sac\_dynamique} qui prend en arguments, une liste d'objets, un indice $i$ et un poids maximal $p$ et qui renvoie la valeur maximale que l'on peut obtenir avec les objets $i$ à $n-1$ et un poids maximal $p$.}
	\ifcorrige
	\corpartPython{sac.py}{}{}{23}{32}
	\fi
	\Question{Proposer une version de la fonction précédente utilisant la mémoïsation afin de ne pas recalculer les instances du problème déjà résolues.}
\end{Exercise}




\end{document}
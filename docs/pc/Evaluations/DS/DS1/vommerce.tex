
\begin{Exercise}[title = {Stratégie gloutonne pour le problème du voyageur de commerce}] \\
	On considère un graphe \textit{complet} (c'est à dire dans lequel il existe un arc entre toutes les paires de sommet) non orienté, pondéré,  et on s'intéresse  au problème du voyageur de commerce c'est à dire à la recherche d'un chemin de longueur minimale qui démarre d'un sommet donné, passe par tous les sommets et revient à son point de départ. Dans toutes la suite du problème, on note, $N$ le nombre de sommet du graphe et $(s_0, \dots, s_{N-1}$) les sommets et on supposera que le sommet de départ est $s_0$.
	Par exemple si on considère le graphe suivant :
	\begin{center}
	\psset{unit=1.6cm}
	\begin{pspicture}(-2,0)(5,2.4)
				\rput(0.7,0.5){\Circlenode{S1}{$s_1$}}
				\rput(1,1.5){\Circlenode{S2}{$s_2$}}
				\rput(4,0.5){\Circlenode{S3}{$s_3$}}
				\rput(2.5,2){\Circlenode{S0}{$s_0$}} 
				\ncline{-}{S0}{S1} \naput{9}
				\ncline{-}{S0}{S2} \nbput{2}
				\ncline{-}{S0}{S3} \naput{5}
				\ncline{-}{S1}{S2} \naput{4}
				\ncline{-}{S1}{S3} \nbput{3}
				\ncline{-}{S2}{S3} \naput{4}
	\end{pspicture}
	\end{center}
	Le problème est de trouver la longueur d'un chemin de longueur minimal partant de $s_0$, passant par tous les autres sommets et revenant à $s_0$.
	Dans les questions de programmation, on suppose que les graphes sont représentés par matrice d'adjacence en utilisant des listes de listes de Python, par exemple le graphe donné en exemple est représenté par la matrice \mintinline{python}{[[0, 9, 2, 5], [9, 0, 4, 3], [2, 4, 0, 4], [5, 3, 4, 0]]}.
	\Question{Montrer que dans l'exemple précédent, la longueur minimale d'un chemin solution est 14 et donner un chemin réalisant ce minimum.}
	\Question{Ecrire une fonction de signature \mintinline{python}{longueur(mat_adj, chemin)} qui prend en argument un graphe pondéré représenté par sa matrice d'adjacence {\tt mat\_adj} ainsi qu'un chemin (représenté par une liste de sommet) et calcule la longueur de ce chemin. Par exemple si {\tt mat\_adj} est la matrice du graphe précédent alors \mintinline{python}{longueur{mat_adj,[0,1,3,2,0]}} renvoie 9+3+4+2=16.}
	\Question{Déterminer la complexité de la longueur de la fonction {\tt longueur} en fonction de la longueur de la liste chemin. En déduire la complexité en fonction de $N$ de la méthode par force brute qui consiste à énumérer tous les chemins possibles partant et revenant à $s_0$ et passant par tous les autres sommets, à calculer leur longueur afin d'en  trouver un de longueur minimale.}
	\Question{On propose à présent de mettre en place la stratégie de résolution gloutonne suivante : on part de $s_0$ et à chaque étape on se dirige vers le sommet \textit{non encore parcourue} le plus proche, et on revient à $s_0$ une fois tous les sommets parcourus. Quel est le chemin obtenu sur le graphe donné en exemple ?}
	\Question{En dessinant un exemple montrer que cette stratégie ne conduit pas forcément à la solution optimale.}
	\Question{Ecrire une fonction de signature \mintinline{python}{glouton(mat_adj)} qui prend en argument une matrice d'adjacence représentant un graphe et qui renvoie le chemin obtenu en utilisant la stratégie gloutonne.}
\end{Exercise}
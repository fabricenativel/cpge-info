\documentclass[11pt,a4paper]{article}

\usepackage{Act}


\begin{document}
\newcommand{\ModeExercice}{
% Traduction des noms pour le package exercise
\renewcommand{\ExerciseName}{Exercice}
\renewcommand{\thesubQuestion}{\theQuestion.\alph{subQuestion}}
\renewcommand{\AnswerName}{Réponses à l'exercise }
}
\newcommand{\Fiche}[2]{\lhead{\textbf{{\sc #1}}}
\rhead{Niveau: \textbf{#2}}
\cfoot{}}
\definecolor{cfond}{gray}{0.4}
\renewcommand{\thealgocf}{}

\newcommand{\ModeActivite}{
% Traduction des noms pour le package exercise
\renewcommand{\ExerciseName}{Activité}
}
% Réglages de la mise en forme des exercices 
\renewcommand{\ExerciseHeaderTitle}{\ExerciseTitle}
\renewcommand{\ExerciseHeaderOrigin}{\ExerciseOrigin}
% Un : sépare le numéro de l'exerice de son titre ... si le titre existe. On utilise Origine pour placer les pictogrammes en fin de ligne
\renewcommand{\ExerciseHeader}{\ding{113} \textbf{\sffamily{\ExerciseName \ \ExerciseHeaderNB}} \ifthenelse{\equal{\ExerciseTitle}{\empty}}{}{:} \textit{\ExerciseHeaderTitle} \hfill \ExerciseHeaderOrigin}
\renewcommand{\ExePartHeader}{\quad {\footnotesize \ding{110}} \textbf{Partie \textbf{\ExePartHeaderNB}} : \ExePartName}


% Mode Concours
\newcommand{\ModeConcours}{
   \newcounter{qconcours}
   \setcounter{qconcours}{1}
   \renewcommand{\ExerciseName}{Exercice}
   \renewcommand{\ExerciseHeaderTitle}{\ExerciseTitle}
\renewcommand{\ExerciseHeaderOrigin}{\ExerciseOrigin}
\renewcommand{\ExerciseHeader}{\ding{113} \textbf{\sffamily{\ExerciseName \ \ExerciseHeaderNB}} \ifthenelse{\equal{\ExerciseTitle}{\empty}}{}{:} \textit{\ExerciseHeaderTitle} \hfill \ExerciseHeaderOrigin}
\renewcommand{\QuestionNB}{\textbf{Q\arabic{qconcours}--}\ \addtocounter{qconcours}{1}}}

\newcommand{\noeud}[1]{\Tr{\fbox{\tt #1}}}
\newcommand{\FPATH}{/home/fenarius/Travail/Cours/cpge-info/docs/mp2i/}
\newcommand{\spath}[2]{\FPATH Evaluations/#1/#1#2}
\definecolor{codebg}{gray}{0.90}
\newcommand{\inputpartOCaml}[5]{\begin{mdframed}[backgroundcolor=codebg] \inputminted[breaklines=true,fontsize=#3,linenos=true,highlightcolor=fluo,tabsize=2,highlightlines={#2},firstline=#4,lastline=#5,firstnumber=1]{OCaml}{#1} \end{mdframed}}
\newcommand{\inputpartPython}[5]{\begin{mdframed}[backgroundcolor=codebg] \inputminted[breaklines=true,fontsize=#3,linenos=true,highlightcolor=fluo,tabsize=2,highlightlines={#2},firstline=#4,lastline=#5,firstnumber=1]{python}{#1} \end{mdframed}}
\newcommand{\inputpartC}[5]{\begin{mdframed}[backgroundcolor=codebg] \inputminted[breaklines=true,fontsize=#3,linenos=true,highlightcolor=fluo,tabsize=2,highlightlines={#2},firstline=#4,lastline=#5,firstnumber=1]{c}{#1} \end{mdframed}}
\newcommand{\inputC}[2]{\begin{mdframed}[backgroundcolor=codebg] \inputminted[breaklines=true,fontsize=#2,linenos=true,highlightcolor=fluo,tabsize=2]{c}{#1} \end{mdframed}}
\newminted[langageC]{c}{linenos=true,escapeinside=``,highlightcolor=fluo,tabsize=2}
\newminted[python]{python}{linenos=true,escapeinside=``,highlightcolor=fluo,tabsize=4}
\BeforeBeginEnvironment{minted}{\begin{mdframed}[backgroundcolor=codebg,skipabove=0cm]}
   \AfterEndEnvironment{minted}{\end{mdframed}}

% Font light medium et bold pour tt :
\newcommand{\ttl}[1]{\ttfamily \fontseries{l}\selectfont #1}
\newcommand{\ttm}[1]{\ttfamily \fontseries{m}\selectfont #1}
\newcommand{\ttb}[1]{\ttfamily \fontseries{b}\selectfont #1}

%QCM de NSI \QNSI{Question}{R1}{R2}{R3}{R4}
\newcommand{\QNSI}[5]{
#1
\begin{enumerate}[label=\alph{enumi})]
\item #2
\item #3
\item #4
\item #5
\end{enumerate}
}



\definecolor{grispale}{gray}{0.95}
\newcommand{\htmlmode}{\lstset{language=html,numbers=left, tabsize=2, frame=single, breaklines=true, keywordstyle=\ttfamily, basicstyle=\small,
   numberstyle=\tiny\ttfamily, framexleftmargin=0mm, backgroundcolor=\color{grispale}, xleftmargin=12mm,showstringspaces=false}}
\newcommand{\pythonmode}{\lstset{language=python,numbers=left, tabsize=4, frame=single, breaklines=true, keywordstyle=\ttfamily, basicstyle=\small,
   numberstyle=\tiny\ttfamily, framexleftmargin=0mm, backgroundcolor=\color{grispale}, xleftmargin=12mm, showstringspaces=false}}
\newcommand{\bashmode}{\lstset{language=bash,numbers=left, tabsize=2, frame=single, breaklines=true, basicstyle=\ttfamily,
   numberstyle=\tiny\ttfamily, framexleftmargin=0mm, backgroundcolor=\color{grispale}, xleftmargin=12mm, showstringspaces=false}}
\newcommand{\exomode}{\lstset{language=python,numbers=left, tabsize=2, frame=single, breaklines=true, basicstyle=\ttfamily,
   numberstyle=\tiny\ttfamily, framexleftmargin=13mm, xleftmargin=12mm, basicstyle=\small, showstringspaces=false}}
   
   
  \lstset{%
        inputencoding=utf8,
        extendedchars=true,
        literate=%
        {é}{{\'{e}}}1
        {è}{{\`{e}}}1
        {ê}{{\^{e}}}1
        {ë}{{\¨{e}}}1
        {É}{{\'{E}}}1
        {Ê}{{\^{E}}}1
        {û}{{\^{u}}}1
        {ù}{{\`{u}}}1
        {ú}{{\'{u}}}1
        {â}{{\^{a}}}1
        {à}{{\`{a}}}1
        {á}{{\'{a}}}1
        {ã}{{\~{a}}}1
        {Á}{{\'{A}}}1
        {Â}{{\^{A}}}1
        {Ã}{{\~{A}}}1
        {ç}{{\c{c}}}1
        {Ç}{{\c{C}}}1
        {õ}{{\~{o}}}1
        {ó}{{\'{o}}}1
        {ô}{{\^{o}}}1
        {Õ}{{\~{O}}}1
        {Ó}{{\'{O}}}1
        {Ô}{{\^{O}}}1
        {î}{{\^{i}}}1
        {Î}{{\^{I}}}1
        {í}{{\'{i}}}1
        {Í}{{\~{Í}}}1
}

%tei pour placer les images
%tei{nom de l’image}{échelle de l’image}{sens}{texte a positionner}
%sens ="1" (droite) ou "2" (gauche)
\newlength{\ltxt}
\newcommand{\tei}[4]{
\setlength{\ltxt}{\linewidth}
\setbox0=\hbox{\includegraphics[scale=#2]{#1}}
\addtolength{\ltxt}{-\wd0}
\addtolength{\ltxt}{-10pt}
\ifthenelse{\equal{#3}{1}}{
\begin{minipage}{\wd0}
\includegraphics[scale=#2]{#1}
\end{minipage}
\hfill
\begin{minipage}{\ltxt}
#4
\end{minipage}
}{
\begin{minipage}{\ltxt}
#4
\end{minipage}
\hfill
\begin{minipage}{\wd0}
\includegraphics[scale=#2]{#1}
\end{minipage}
}
}

%Juxtaposition d'une image pspciture et de texte 
%#1: = code pstricks de l'image
%#2: largeur de l'image
%#3: hauteur de l'image
%#4: Texte à écrire
\newcommand{\ptp}[4]{
\setlength{\ltxt}{\linewidth}
\addtolength{\ltxt}{-#2 cm}
\addtolength{\ltxt}{-0.1 cm}
\begin{minipage}[b][#3 cm][t]{\ltxt}
#4
\end{minipage}\hfill
\begin{minipage}[b][#3 cm][c]{#2 cm}
#1
\end{minipage}\par
}



%Macros pour les graphiques
\psset{linewidth=0.5\pslinewidth,PointSymbol=x}
\setlength{\fboxrule}{0.5pt}
\newcounter{tempangle}

%Marque la longueur du segment d'extrémité  #1 et  #2 avec la valeur #3, #4 est la distance par rapport au segment (en %age de la valeur de celui ci) et #5 l'orientation du marquage : +90 ou -90
\newcommand{\afflong}[5]{
\pstRotation[RotAngle=#4,PointSymbol=none,PointName=none]{#1}{#2}[X] 
\pstHomO[PointSymbol=none,PointName=none,HomCoef=#5]{#1}{X}[Y]
\pstTranslation[PointSymbol=none,PointName=none]{#1}{#2}{Y}[Z]
 \ncline{|<->|,linewidth=0.25\pslinewidth}{Y}{Z} \ncput*[nrot=:U]{\footnotesize{#3}}
}
\newcommand{\afflongb}[3]{
\ncline{|<->|,linewidth=0}{#1}{#2} \naput*[nrot=:U]{\footnotesize{#3}}
}

%Construis le point #4 situé à #2 cm du point #1 avant un angle #3 par rapport à l'horizontale. #5 = liste de paramètre
\newcommand{\lsegment}[5]{\pstGeonode[PointSymbol=none,PointName=none](0,0){O'}(#2,0){I'} \pstTranslation[PointSymbol=none,PointName=none]{O'}{I'}{#1}[J'] \pstRotation[RotAngle=#3,PointSymbol=x,#5]{#1}{J'}[#4]}
\newcommand{\tsegment}[5]{\pstGeonode[PointSymbol=none,PointName=none](0,0){O'}(#2,0){I'} \pstTranslation[PointSymbol=none,PointName=none]{O'}{I'}{#1}[J'] \pstRotation[RotAngle=#3,PointSymbol=x,#5]{#1}{J'}[#4] \pstLineAB{#4}{#1}}

%Construis le point #4 situé à #3 cm du point #1 et faisant un angle de  90° avec la droite (#1,#2) #5 = liste de paramètre
\newcommand{\psegment}[5]{
\pstGeonode[PointSymbol=none,PointName=none](0,0){O'}(#3,0){I'}
 \pstTranslation[PointSymbol=none,PointName=none]{O'}{I'}{#1}[J']
 \pstInterLC[PointSymbol=none,PointName=none]{#1}{#2}{#1}{J'}{M1}{M2} \pstRotation[RotAngle=-90,PointSymbol=x,#5]{#1}{M1}[#4]
  }
  
%Construis le point #4 situé à #3 cm du point #1 et faisant un angle de  #5° avec la droite (#1,#2) #6 = liste de paramètre
\newcommand{\mlogo}[6]{
\pstGeonode[PointSymbol=none,PointName=none](0,0){O'}(#3,0){I'}
 \pstTranslation[PointSymbol=none,PointName=none]{O'}{I'}{#1}[J']
 \pstInterLC[PointSymbol=none,PointName=none]{#1}{#2}{#1}{J'}{M1}{M2} \pstRotation[RotAngle=#5,PointSymbol=x,#6]{#1}{M2}[#4]
  }

% Construis un triangle avec #1=liste des 3 sommets séparés par des virgules, #2=liste des 3 longueurs séparés par des virgules, #3 et #4 : paramètre d'affichage des 2e et 3 points et #5 : inclinaison par rapport à l'horizontale
%autre macro identique mais sans tracer les segments joignant les sommets
\noexpandarg
\newcommand{\Triangleccc}[5]{
\StrBefore{#1}{,}[\pointA]
\StrBetween[1,2]{#1}{,}{,}[\pointB]
\StrBehind[2]{#1}{,}[\pointC]
\StrBefore{#2}{,}[\coteA]
\StrBetween[1,2]{#2}{,}{,}[\coteB]
\StrBehind[2]{#2}{,}[\coteC]
\tsegment{\pointA}{\coteA}{#5}{\pointB}{#3} 
\lsegment{\pointA}{\coteB}{0}{Z1}{PointSymbol=none, PointName=none}
\lsegment{\pointB}{\coteC}{0}{Z2}{PointSymbol=none, PointName=none}
\pstInterCC{\pointA}{Z1}{\pointB}{Z2}{\pointC}{Z3} 
\pstLineAB{\pointA}{\pointC} \pstLineAB{\pointB}{\pointC}
\pstSymO[PointName=\pointC,#4]{C}{C}[C]
}
\noexpandarg
\newcommand{\TrianglecccP}[5]{
\StrBefore{#1}{,}[\pointA]
\StrBetween[1,2]{#1}{,}{,}[\pointB]
\StrBehind[2]{#1}{,}[\pointC]
\StrBefore{#2}{,}[\coteA]
\StrBetween[1,2]{#2}{,}{,}[\coteB]
\StrBehind[2]{#2}{,}[\coteC]
\tsegment{\pointA}{\coteA}{#5}{\pointB}{#3} 
\lsegment{\pointA}{\coteB}{0}{Z1}{PointSymbol=none, PointName=none}
\lsegment{\pointB}{\coteC}{0}{Z2}{PointSymbol=none, PointName=none}
\pstInterCC[PointNameB=none,PointSymbolB=none,#4]{\pointA}{Z1}{\pointB}{Z2}{\pointC}{Z1} 
}


% Construis un triangle avec #1=liste des 3 sommets séparés par des virgules, #2=liste formée de 2 longueurs et d'un angle séparés par des virgules, #3 et #4 : paramètre d'affichage des 2e et 3 points et #5 : inclinaison par rapport à l'horizontale
%autre macro identique mais sans tracer les segments joignant les sommets
\newcommand{\Trianglecca}[5]{
\StrBefore{#1}{,}[\pointA]
\StrBetween[1,2]{#1}{,}{,}[\pointB]
\StrBehind[2]{#1}{,}[\pointC]
\StrBefore{#2}{,}[\coteA]
\StrBetween[1,2]{#2}{,}{,}[\coteB]
\StrBehind[2]{#2}{,}[\angleA]
\tsegment{\pointA}{\coteA}{#5}{\pointB}{#3} 
\setcounter{tempangle}{#5}
\addtocounter{tempangle}{\angleA}
\tsegment{\pointA}{\coteB}{\thetempangle}{\pointC}{#4}
\pstLineAB{\pointB}{\pointC}
}
\newcommand{\TriangleccaP}[5]{
\StrBefore{#1}{,}[\pointA]
\StrBetween[1,2]{#1}{,}{,}[\pointB]
\StrBehind[2]{#1}{,}[\pointC]
\StrBefore{#2}{,}[\coteA]
\StrBetween[1,2]{#2}{,}{,}[\coteB]
\StrBehind[2]{#2}{,}[\angleA]
\lsegment{\pointA}{\coteA}{#5}{\pointB}{#3} 
\setcounter{tempangle}{#5}
\addtocounter{tempangle}{\angleA}
\lsegment{\pointA}{\coteB}{\thetempangle}{\pointC}{#4}
}

% Construis un triangle avec #1=liste des 3 sommets séparés par des virgules, #2=liste formée de 1 longueurs et de deux angle séparés par des virgules, #3 et #4 : paramètre d'affichage des 2e et 3 points et #5 : inclinaison par rapport à l'horizontale
%autre macro identique mais sans tracer les segments joignant les sommets
\newcommand{\Trianglecaa}[5]{
\StrBefore{#1}{,}[\pointA]
\StrBetween[1,2]{#1}{,}{,}[\pointB]
\StrBehind[2]{#1}{,}[\pointC]
\StrBefore{#2}{,}[\coteA]
\StrBetween[1,2]{#2}{,}{,}[\angleA]
\StrBehind[2]{#2}{,}[\angleB]
\tsegment{\pointA}{\coteA}{#5}{\pointB}{#3} 
\setcounter{tempangle}{#5}
\addtocounter{tempangle}{\angleA}
\lsegment{\pointA}{1}{\thetempangle}{Z1}{PointSymbol=none, PointName=none}
\setcounter{tempangle}{#5}
\addtocounter{tempangle}{180}
\addtocounter{tempangle}{-\angleB}
\lsegment{\pointB}{1}{\thetempangle}{Z2}{PointSymbol=none, PointName=none}
\pstInterLL[#4]{\pointA}{Z1}{\pointB}{Z2}{\pointC}
\pstLineAB{\pointA}{\pointC}
\pstLineAB{\pointB}{\pointC}
}
\newcommand{\TrianglecaaP}[5]{
\StrBefore{#1}{,}[\pointA]
\StrBetween[1,2]{#1}{,}{,}[\pointB]
\StrBehind[2]{#1}{,}[\pointC]
\StrBefore{#2}{,}[\coteA]
\StrBetween[1,2]{#2}{,}{,}[\angleA]
\StrBehind[2]{#2}{,}[\angleB]
\lsegment{\pointA}{\coteA}{#5}{\pointB}{#3} 
\setcounter{tempangle}{#5}
\addtocounter{tempangle}{\angleA}
\lsegment{\pointA}{1}{\thetempangle}{Z1}{PointSymbol=none, PointName=none}
\setcounter{tempangle}{#5}
\addtocounter{tempangle}{180}
\addtocounter{tempangle}{-\angleB}
\lsegment{\pointB}{1}{\thetempangle}{Z2}{PointSymbol=none, PointName=none}
\pstInterLL[#4]{\pointA}{Z1}{\pointB}{Z2}{\pointC}
}

%Construction d'un cercle de centre #1 et de rayon #2 (en cm)
\newcommand{\Cercle}[2]{
\lsegment{#1}{#2}{0}{Z1}{PointSymbol=none, PointName=none}
\pstCircleOA{#1}{Z1}
}

%construction d'un parallélogramme #1 = liste des sommets, #2 = liste contenant les longueurs de 2 côtés consécutifs et leurs angles;  #3, #4 et #5 : paramètre d'affichage des sommets #6 inclinaison par rapport à l'horizontale 
% meme macro sans le tracé des segements
\newcommand{\Para}[6]{
\StrBefore{#1}{,}[\pointA]
\StrBetween[1,2]{#1}{,}{,}[\pointB]
\StrBetween[2,3]{#1}{,}{,}[\pointC]
\StrBehind[3]{#1}{,}[\pointD]
\StrBefore{#2}{,}[\longueur]
\StrBetween[1,2]{#2}{,}{,}[\largeur]
\StrBehind[2]{#2}{,}[\angle]
\tsegment{\pointA}{\longueur}{#6}{\pointB}{#3} 
\setcounter{tempangle}{#6}
\addtocounter{tempangle}{\angle}
\tsegment{\pointA}{\largeur}{\thetempangle}{\pointD}{#5}
\pstMiddleAB[PointName=none,PointSymbol=none]{\pointB}{\pointD}{Z1}
\pstSymO[#4]{Z1}{\pointA}[\pointC]
\pstLineAB{\pointB}{\pointC}
\pstLineAB{\pointC}{\pointD}
}
\newcommand{\ParaP}[6]{
\StrBefore{#1}{,}[\pointA]
\StrBetween[1,2]{#1}{,}{,}[\pointB]
\StrBetween[2,3]{#1}{,}{,}[\pointC]
\StrBehind[3]{#1}{,}[\pointD]
\StrBefore{#2}{,}[\longueur]
\StrBetween[1,2]{#2}{,}{,}[\largeur]
\StrBehind[2]{#2}{,}[\angle]
\lsegment{\pointA}{\longueur}{#6}{\pointB}{#3} 
\setcounter{tempangle}{#6}
\addtocounter{tempangle}{\angle}
\lsegment{\pointA}{\largeur}{\thetempangle}{\pointD}{#5}
\pstMiddleAB[PointName=none,PointSymbol=none]{\pointB}{\pointD}{Z1}
\pstSymO[#4]{Z1}{\pointA}[\pointC]
}


%construction d'un cerf-volant #1 = liste des sommets, #2 = liste contenant les longueurs de 2 côtés consécutifs et leurs angles;  #3, #4 et #5 : paramètre d'affichage des sommets #6 inclinaison par rapport à l'horizontale 
% meme macro sans le tracé des segements
\newcommand{\CerfVolant}[6]{
\StrBefore{#1}{,}[\pointA]
\StrBetween[1,2]{#1}{,}{,}[\pointB]
\StrBetween[2,3]{#1}{,}{,}[\pointC]
\StrBehind[3]{#1}{,}[\pointD]
\StrBefore{#2}{,}[\longueur]
\StrBetween[1,2]{#2}{,}{,}[\largeur]
\StrBehind[2]{#2}{,}[\angle]
\tsegment{\pointA}{\longueur}{#6}{\pointB}{#3} 
\setcounter{tempangle}{#6}
\addtocounter{tempangle}{\angle}
\tsegment{\pointA}{\largeur}{\thetempangle}{\pointD}{#5}
\pstOrtSym[#4]{\pointB}{\pointD}{\pointA}[\pointC]
\pstLineAB{\pointB}{\pointC}
\pstLineAB{\pointC}{\pointD}
}

%construction d'un quadrilatère quelconque #1 = liste des sommets, #2 = liste contenant les longueurs des 4 côtés et l'angle entre 2 cotés consécutifs  #3, #4 et #5 : paramètre d'affichage des sommets #6 inclinaison par rapport à l'horizontale 
% meme macro sans le tracé des segements
\newcommand{\Quadri}[6]{
\StrBefore{#1}{,}[\pointA]
\StrBetween[1,2]{#1}{,}{,}[\pointB]
\StrBetween[2,3]{#1}{,}{,}[\pointC]
\StrBehind[3]{#1}{,}[\pointD]
\StrBefore{#2}{,}[\coteA]
\StrBetween[1,2]{#2}{,}{,}[\coteB]
\StrBetween[2,3]{#2}{,}{,}[\coteC]
\StrBetween[3,4]{#2}{,}{,}[\coteD]
\StrBehind[4]{#2}{,}[\angle]
\tsegment{\pointA}{\coteA}{#6}{\pointB}{#3} 
\setcounter{tempangle}{#6}
\addtocounter{tempangle}{\angle}
\tsegment{\pointA}{\coteD}{\thetempangle}{\pointD}{#5}
\lsegment{\pointB}{\coteB}{0}{Z1}{PointSymbol=none, PointName=none}
\lsegment{\pointD}{\coteC}{0}{Z2}{PointSymbol=none, PointName=none}
\pstInterCC[PointNameA=none,PointSymbolA=none,#4]{\pointB}{Z1}{\pointD}{Z2}{Z3}{\pointC} 
\pstLineAB{\pointB}{\pointC}
\pstLineAB{\pointC}{\pointD}
}


% Définition des colonnes centrées ou à droite pour tabularx
\newcolumntype{Y}{>{\centering\arraybackslash}X}
\newcolumntype{Z}{>{\flushright\arraybackslash}X}

%Les pointillés à remplir par les élèves
\newcommand{\po}[1]{\makebox[#1 cm]{\dotfill}}
\newcommand{\lpo}[1][3]{%
\multido{}{#1}{\makebox[\linewidth]{\dotfill}
}}

%Liste des pictogrammes utilisés sur la fiche d'exercice ou d'activités
\newcommand{\bombe}{\faBomb}
\newcommand{\livre}{\faBook}
\newcommand{\calculatrice}{\faCalculator}
\newcommand{\oral}{\faCommentO}
\newcommand{\surfeuille}{\faEdit}
\newcommand{\ordinateur}{\faLaptop}
\newcommand{\ordi}{\faDesktop}
\newcommand{\ciseaux}{\faScissors}
\newcommand{\danger}{\faExclamationTriangle}
\newcommand{\out}{\faSignOut}
\newcommand{\cadeau}{\faGift}
\newcommand{\flash}{\faBolt}
\newcommand{\lumiere}{\faLightbulb}
\newcommand{\compas}{\dsmathematical}
\newcommand{\calcullitteral}{\faTimesCircleO}
\newcommand{\raisonnement}{\faCogs}
\newcommand{\recherche}{\faSearch}
\newcommand{\rappel}{\faHistory}
\newcommand{\video}{\faFilm}
\newcommand{\capacite}{\faPuzzlePiece}
\newcommand{\aide}{\faLifeRing}
\newcommand{\loin}{\faExternalLink}
\newcommand{\groupe}{\faUsers}
\newcommand{\bac}{\faGraduationCap}
\newcommand{\histoire}{\faUniversity}
\newcommand{\coeur}{\faSave}
\newcommand{\os}{\faMicrochip}
\newcommand{\rd}{\faCubes}
\newcommand{\data}{\faColumns}
\newcommand{\web}{\faCode}
\newcommand{\prog}{\faFile}
\newcommand{\algo}{\faCogs}
\newcommand{\important}{\faExclamationCircle}
\newcommand{\maths}{\faTimesCircle}
% Traitement des données en tables
\newcommand{\tables}{\faColumns}
% Types construits
\newcommand{\construits}{\faCubes}
% Type et valeurs de base
\newcommand{\debase}{{\footnotesize \faCube}}
% Systèmes d'exploitation
\newcommand{\linux}{\faLinux}
\newcommand{\sd}{\faProjectDiagram}
\newcommand{\bd}{\faDatabase}

%Les ensembles de nombres
\renewcommand{\N}{\mathbb{N}}
\newcommand{\D}{\mathbb{D}}
\newcommand{\Z}{\mathbb{Z}}
\newcommand{\Q}{\mathbb{Q}}
\newcommand{\R}{\mathbb{R}}
\newcommand{\C}{\mathbb{C}}

%Ecriture des vecteurs
\newcommand{\vect}[1]{\vbox{\halign{##\cr 
  \tiny\rightarrowfill\cr\noalign{\nointerlineskip\vskip1pt} 
  $#1\mskip2mu$\cr}}}


%Compteur activités/exos et question et mise en forme titre et questions
\newcounter{numact}
\setcounter{numact}{1}
\newcounter{numseance}
\setcounter{numseance}{1}
\newcounter{numexo}
\setcounter{numexo}{0}
\newcounter{numprojet}
\setcounter{numprojet}{0}
\newcounter{numquestion}
\newcommand{\espace}[1]{\rule[-1ex]{0pt}{#1 cm}}
\newcommand{\Quest}[3]{
\addtocounter{numquestion}{1}
\begin{tabularx}{\textwidth}{X|m{1cm}|}
\cline{2-2}
\textbf{\sffamily{\alph{numquestion})}} #1 & \dots / #2 \\
\hline 
\multicolumn{2}{|l|}{\espace{#3}} \\
\hline
\end{tabularx}
}
\newcommand{\mq}[1]
{\ding{113} \addtocounter{numquestion}{1}
\textbf{Question \arabic{numquestion}} \\ #1}
\newcommand{\QuestR}[3]{
\addtocounter{numquestion}{1}
\begin{tabularx}{\textwidth}{X|m{1cm}|}
\cline{2-2}
\textbf{\sffamily{\alph{numquestion})}} #1 & \dots / #2 \\
\hline 
\multicolumn{2}{|l|}{\cor{#3}} \\
\hline
\end{tabularx}
}
\newcommand{\Pre}{{\sc nsi} 1\textsuperscript{e}}
\newcommand{\Term}{{\sc nsi} Terminale}
\newcommand{\Sec}{2\textsuperscript{e}}
\newcommand{\Exo}[2]{ \addtocounter{numexo}{1} \ding{113} \textbf{\sffamily{Exercice \thenumexo}} : \textit{#1} \hfill #2  \setcounter{numquestion}{0}}
\newcommand{\Projet}[1]{ \addtocounter{numprojet}{1} \ding{118} \textbf{\sffamily{Projet \thenumprojet}} : \textit{#1}}
\newcommand{\ExoD}[2]{ \addtocounter{numexo}{1} \ding{113} \textbf{\sffamily{Exercice \thenumexo}}  \textit{(#1 pts)} \hfill #2  \setcounter{numquestion}{0}}
\newcommand{\ExoB}[2]{ \addtocounter{numexo}{1} \ding{113} \textbf{\sffamily{Exercice \thenumexo}}  \textit{(Bonus de +#1 pts maximum)} \hfill #2  \setcounter{numquestion}{0}}
\newcommand{\Act}[2]{ \ding{113} \textbf{\sffamily{Activité \thenumact}} : \textit{#1} \hfill #2  \addtocounter{numact}{1} \setcounter{numquestion}{0}}
\newcommand{\Seance}{ \rule{1.5cm}{0.5pt}\raisebox{-3pt}{\framebox[4cm]{\textbf{\sffamily{Séance \thenumseance}}}}\hrulefill  \\
  \addtocounter{numseance}{1}}
\newcommand{\Acti}[2]{ {\footnotesize \ding{117}} \textbf{\sffamily{Activité \thenumact}} : \textit{#1} \hfill #2  \addtocounter{numact}{1} \setcounter{numquestion}{0}}
\newcommand{\titre}[1]{\begin{Large}\textbf{\ding{118}}\end{Large} \begin{large}\textbf{ #1}\end{large} \vspace{0.2cm}}
\newcommand{\QListe}[1][0]{
\ifthenelse{#1=0}
{\begin{enumerate}[partopsep=0pt,topsep=0pt,parsep=0pt,itemsep=0pt,label=\textbf{\sffamily{\arabic*.}},series=question]}
{\begin{enumerate}[resume*=question]}}
\newcommand{\SQListe}[1][0]{
\ifthenelse{#1=0}
{\begin{enumerate}[partopsep=0pt,topsep=0pt,parsep=0pt,itemsep=0pt,label=\textbf{\sffamily{\alph*)}},series=squestion]}
{\begin{enumerate}[resume*=squestion]}}
\newcommand{\SQListeL}[1][0]{
\ifthenelse{#1=0}
{\begin{enumerate*}[partopsep=0pt,topsep=0pt,parsep=0pt,itemsep=0pt,label=\textbf{\sffamily{\alph*)}},series=squestion]}
{\begin{enumerate*}[resume*=squestion]}}
\newcommand{\FinListe}{\end{enumerate}}
\newcommand{\FinListeL}{\end{enumerate*}}

%Mise en forme de la correction
\newboolean{corrige}
\setboolean{corrige}{false}
\newcommand{\scor}[1]{\par \textcolor{blue!75!black}{\small #1}}
\newcommand{\cor}[1]{\par \textcolor{blue!75!black}{#1}}
\newcommand{\br}[1]{\cor{\textbf{#1}}}
\newcommand{\tcor}[1]{
\ifthenelse{\boolean{corrige}}{\begin{tcolorbox}[width=\linewidth,colback={white},colbacktitle=white,coltitle=green!50!black,colframe=green!50!black,boxrule=0.2mm]   
\cor{#1}
\end{tcolorbox}}{}
}
\newcommand{\iscor}[1]{\ifthenelse{\boolean{corrige}}{#1}}
\newcommand{\rc}[1]{\textcolor{OliveGreen}{#1}}

%Référence aux exercices par leur numéro
\newcommand{\refexo}[1]{
\refstepcounter{numexo}
\addtocounter{numexo}{-1}
\label{#1}}

%Séparation entre deux activités
\newcommand{\separateur}{\begin{center}
\rule{1.5cm}{0.5pt}\raisebox{-3pt}{\ding{117}}\rule{1.5cm}{0.5pt}  \vspace{0.2cm}
\end{center}}

%Entête et pied de page
\newcommand{\snt}[1]{\lhead{\textbf{SNT -- La photographie numérique} \rhead{\textit{Lycée Nord}}}}
\newcommand{\Activites}[2]{\lhead{\textbf{{\sc #1}}}
\rhead{Activités -- \textbf{#2}}
\cfoot{}}
\newcommand{\Exos}[2]{\lhead{\textbf{Fiche d'exercices: {\sc #1}}}
\rhead{Niveau: \textbf{#2}}
\cfoot{}}
\newcommand{\TD}[2]{\lhead{\textbf{TD #1} : {\sc #2} }
\rhead{{\sc mp2i -- Lycée Leconte de Lisle}}
\cfoot{}}
\newcommand{\Colles}[2]{\lhead{{\sc mp2i -- }\textbf{Colles d'informatique #1}} 
\rhead{{\sc #2}}
\cfoot{}}
\newcommand{\Devoir}[2]{\lhead{\textbf{Devoir de mathématiques : {\sc #1}}}
\rhead{\textbf{#2}} \setlength{\fboxsep}{8pt}
\begin{center}
%Titre de la fiche
\fbox{\parbox[b][1cm][t]{0.3\textwidth}{Nom : \hfill \po{3} \par \vfill Prénom : \hfill \po{3}} } \hfill 
\fbox{\parbox[b][1cm][t]{0.6\textwidth}{Note : \po{1} / 20} }
\end{center} \cfoot{}}
\newcommand{\TPnote}[3]{\lhead{\textbf{TP noté d'informatique n° #1}}
\rhead{\textbf{#2}} \setlength{\fboxsep}{8pt}
\ifthenelse{\boolean{corrige}}{}
{\begin{center}
\fbox{\parbox[b][1cm][t]{0.3\textwidth}{Nom : \hfill \po{3} \par \vfill Prénom : \hfill \po{3}} } \hfill 
\fbox{\parbox[b][1cm][t]{0.6\textwidth}{Note : \po{1} / #3} }
\end{center}} \cfoot{}}
\newcommand{\IC}[2]{\lhead{\textbf{Interro de cours n° #1}}
\rhead{{\sc mp2i --} \textbf{#2}} \setlength{\fboxsep}{8pt}
\ifthenelse{\boolean{corrige}}{}
{\begin{center}
%Titre de la fiche
\fbox{\parbox[b][1cm][t]{0.3\textwidth}{Nom : \hfill \po{3} \par \vfill Prénom : \hfill \po{3}} } \hfill 
\fbox{\parbox[b][1cm][t]{0.6\textwidth}{Note : \po{1} / 10} }
\end{center}}\cfoot{}}
\newcommand{\DS}[3]{\lhead{{#1} : \textbf{DS d'informatique n° #2}}
\rhead{Lycée Leconte de Lisle -- #3} \setlength{\fboxsep}{8pt}
%Titre de la fiche
\begin{center}
   {\Large \textbf{Devoir surveillé d'informatique}}
\end{center} \cfoot{\thepage/\pageref{LastPage}}}

\newcommand{\CB}[2]{\lhead{{#1} : \textbf{Councours blanc - Informatique}}
\rhead{Lycée Leconte de Lisle -- #2} \setlength{\fboxsep}{8pt}
%Titre de la fiche
\begin{center}
   {\Large \textbf{Concours Blanc - Epreuve d'informatique}}
\end{center} \cfoot{\thepage/\pageref{LastPage}}}

\newcommand{\PC}[3]{\lhead{Concours {#1} -- #2}
\rhead{Lycée Leconte de Lisle} \setlength{\fboxsep}{8pt}
%Titre de la fiche
\begin{center}
   {\Large \textbf{Proposition de corrigé}}
\end{center} \cfoot{\thepage/\pageref{LastPage}}}

\newcounter{numdspart}
\setcounter{numdspart}{1}
\newcommand{\DSPart}{\bigskip
   \hrulefill\raisebox{-3pt}{\framebox[4cm]{\textbf{\textbf{Partie \thenumdspart}}}}\hrulefill
   \addtocounter{numdspart}{1}
   \bigskip}

\newcommand{\Sauvegarde}[1]{
   \begin{tcolorbox}[title=\textcolor{black}{\danger\; Attention},colbacktitle=lightgray]
      {Tous vos programmes doivent être enregistrés dans votre dossier personnel, dans {\tt Evaluations}{\tt \textbackslash}{\tt #1}
      }
   \end{tcolorbox}
}

\newcommand{\alertbox}[3]{
   \begin{tcolorbox}[title=\textcolor{black}{#1\; #2},colbacktitle=lightgray]
      {#3}
   \end{tcolorbox}
}

%Devoir programmation en NSI (pas à rendre sur papier)
\newcommand{\PNSI}[2]{\lhead{\textbf{Devoir de {\sc nsi} : \textsf{ #1}}
}
\rhead{\textbf{#2}} \setlength{\fboxsep}{8pt}
 \cfoot{}
 \begin{center}{\Large \textbf{Evaluation de {\sc nsi}}}\end{center}}


%Devoir de NSI
\newcommand{\DNSI}[2]{\lhead{\textbf{Devoir de {\sc nsi} : \textsf{ #1}}}
\rhead{\textbf{#2}} \setlength{\fboxsep}{8pt}
\begin{center}
%Titre de la fiche
\fbox{\parbox[b][1cm][t]{0.3\textwidth}{Nom : \hfill \po{3} \par \vfill Prénom : \hfill \po{3}} } \hfill 
\fbox{\parbox[b][1cm][t]{0.6\textwidth}{Note : \po{1} / 10} }
\end{center} \cfoot{}}

\newcommand{\DevoirNSI}[2]{\lhead{\textbf{Devoir de {\sc nsi} : {\sc #1}}}
\rhead{\textbf{#2}} \setlength{\fboxsep}{8pt}
\cfoot{}}

%La définition de la commande QCM pour auto-multiple-choice
%En premier argument le sujet du qcm, deuxième argument : la classe, 3e : la durée prévue et #4 : présence ou non de questions avec plusieurs bonnes réponses
\newcommand{\QCM}[4]{
{\large \textbf{\ding{52} QCM : #1}} -- Durée : \textbf{#3 min} \hfill {\large Note : \dots/10} 
\hrule \vspace{0.1cm}\namefield{}
Nom :  \textbf{\textbf{\nom{}}} \qquad \qquad Prénom :  \textbf{\prenom{}}  \hfill Classe: \textbf{#2}
\vspace{0.2cm}
\hrule  
\begin{itemize}[itemsep=0pt]
\item[-] \textit{Une bonne réponse vaut un point, une absence de réponse n'enlève pas de point. }
\item[\danger] \textit{Une mauvaise réponse enlève un point.}
\ifthenelse{#4=1}{\item[-] \textit{Les questions marquées du symbole \multiSymbole{} peuvent avoir plusieurs bonnes réponses possibles.}}{}
\end{itemize}
}
\newcommand{\DevoirC}[2]{
\renewcommand{\footrulewidth}{0.5pt}
\lhead{\textbf{Devoir de mathématiques : {\sc #1}}}
\rhead{\textbf{#2}} \setlength{\fboxsep}{8pt}
\fbox{\parbox[b][0.4cm][t]{0.955\textwidth}{Nom : \po{5} \hfill Prénom : \po{5} \hfill Classe: \textbf{1}\textsuperscript{$\dots$}} } 
\rfoot{\thepage} \cfoot{} \lfoot{Lycée Nord}}
\newcommand{\DevoirInfo}[2]{\lhead{\textbf{Evaluation : {\sc #1}}}
\rhead{\textbf{#2}} \setlength{\fboxsep}{8pt}
 \cfoot{}}
\newcommand{\DM}[2]{\lhead{\textbf{Devoir maison à rendre le #1}} \rhead{\textbf{#2}}}

%Macros permettant l'affichage des touches de la calculatrice
%Touches classiques : #1 = 0 fond blanc pour les nombres et #1= 1gris pour les opérations et entrer, second paramètre=contenu
%Si #2=1 touche arrondi avec fond gris
\newcommand{\TCalc}[2]{
\setlength{\fboxsep}{0.1pt}
\ifthenelse{#1=0}
{\psframebox[fillstyle=solid, fillcolor=white]{\parbox[c][0.25cm][c]{0.6cm}{\centering #2}}}
{\ifthenelse{#1=1}
{\psframebox[fillstyle=solid, fillcolor=lightgray]{\parbox[c][0.25cm][c]{0.6cm}{\centering #2}}}
{\psframebox[framearc=.5,fillstyle=solid, fillcolor=white]{\parbox[c][0.25cm][c]{0.6cm}{\centering #2}}}
}}
\newcommand{\Talpha}{\psdblframebox[fillstyle=solid, fillcolor=white]{\hspace{-0.05cm}\parbox[c][0.25cm][c]{0.65cm}{\centering \scriptsize{alpha}}} \;}
\newcommand{\Tsec}{\psdblframebox[fillstyle=solid, fillcolor=white]{\parbox[c][0.25cm][c]{0.6cm}{\centering \scriptsize 2nde}} \;}
\newcommand{\Tfx}{\psdblframebox[fillstyle=solid, fillcolor=white]{\parbox[c][0.25cm][c]{0.6cm}{\centering \scriptsize $f(x)$}} \;}
\newcommand{\Tvar}{\psframebox[framearc=.5,fillstyle=solid, fillcolor=white]{\hspace{-0.22cm} \parbox[c][0.25cm][c]{0.82cm}{$\scriptscriptstyle{X,T,\theta,n}$}}}
\newcommand{\Tgraphe}{\psdblframebox[fillstyle=solid, fillcolor=white]{\hspace{-0.08cm}\parbox[c][0.25cm][c]{0.68cm}{\centering \tiny{graphe}}} \;}
\newcommand{\Tfen}{\psdblframebox[fillstyle=solid, fillcolor=white]{\hspace{-0.08cm}\parbox[c][0.25cm][c]{0.68cm}{\centering \tiny{fenêtre}}} \;}
\newcommand{\Ttrace}{\psdblframebox[fillstyle=solid, fillcolor=white]{\parbox[c][0.25cm][c]{0.6cm}{\centering \scriptsize{trace}}} \;}

% Macroi pour l'affichage  d'un entier n dans  une base b
\newcommand{\base}[2]{ \overline{#1}^{#2}}
% Intervalle d'entiers
\newcommand{\intN}[2]{\llbracket #1; #2 \rrbracket}
% Cadre avec lignes réponses
\def\gaddtotok#1{\global\tabtok\expandafter{\the\tabtok#1}}
\newtoks\tabtok
\newcommand*\reponse[2]{%
   \ifthenelse{\boolean{corrige}}{}{
	\global\tabtok{\\ \renewcommand{\arraystretch}{1.4}\begin{tabularx}{\linewidth}{|X|p{1cm}|}\hline \dotfill & \cellcolor{gray!30}{\small \dots/#2} \\ \cline{2-2}}%
	\multido{}{#1}{\gaddtotok{ \multicolumn{2}{|>{\hsize=\dimexpr1\hsize+2\tabcolsep+\arrayrulewidth+1cm\relax}X|}{\dotfill}\\ }}%
	\gaddtotok{\hline \end{tabularx}}%
	\the\tabtok
   }}

\newcommand{\PE}[1]{\left \lfloor #1 \right \lfloor}}

% Entête de la fiche (à modifier avec la macro correspondante dans le fichier macros)
\Fiche{Les listes de Python}{\sc pcsi}

\pythonmode


\begin{tcolorbox}[left=0cm,title=\bf{\faPython \; Tranches},colbacktitle=cfond]
\begin{enumerate}
 \item[\ding{182}] Accès à un caractère par son indice \\
 La notation {\tt [i]} déjà rencontrée sur les chaines de caractères permet d'accéder au {\tt i}-ème caractère d'une chaine où les caractères sont numérotés \textit{à partir de 0}. Par exemple,
 si \mintinline{python}{exemple = "Un petit exemple"} : \\
 \newcommand{\ind}[1]{\multicolumn{1}{c}{\textbf{\tt \footnotesize #1}}}
 \begin{tabular}{|c|c|c|c|c|c|c|c|c|c|c|c|c|c|c|c|}
    \hline
    U & n & & p & e & t & i & t&  & e & x & e & m & p & l & e \\
    \hline
    \ind{0} & \ind{1}& \ind{2}& \ind{3}& \ind{4}& \ind{5}& \ind{6}& \ind{7}& \ind{8}& \ind{9}& \ind{10}& \ind{11}& \ind{12}& \ind{13}& \ind{14} & \ind{15}
 \end{tabular}\\
 alors : \mintinline{python}{exemple[0]} est {\tt 'U'}, \mintinline{python}{exemple[1]} est {\tt 'n'}, \dots \\
 {\small \important} \; On remarquera que l'indice du dernier élément est  \textit{la longueur de la chaine moins 1}. La longueur s'obtenant avec {\tt len}, ici on a par exemple \mintinline{python}{exemple[len(exemple)-1]} qui vaut {\tt 'e'}.
 \item[\ding{183}] Tranches \\
 On peut aussi prendre une tranche en précisant dans les {\tt []} le début de la tranche (inclus) et sa fin (exclue) séparé par le caractère {\tt :} si le début ou la fin sont absents alors ils correspondent respectivement au premier et au dernier indice. Par exemples :
 \begin{itemize}
  \item[\textbullet] \mintinline{python}{exemple[3:8]} est {\tt "petit"}
  \item[\textbullet] \mintinline{python}{exemple[:2]} est {\tt "Un"} (le début étant absent, on commence au premier caractère)
  \item[\textbullet] \mintinline{python}{exemple[13:]} est {\tt "ple"} (la fin étant absente, on termine au dernier caractère)
 \end{itemize}	
\item[\ding{184}] Pas de progression \\
Une tranche peut prendre un troisième paramètre qui indique alors un \textit{pas de progression}, par exemple si ce pas vaut 2, on ne prend qu'un caractère sur 2. D'autre part si le pas est négatif alors on progresse de la fin de la chaine vers le début. Par exemples :
\begin{itemize}
    \item[\textbullet] \mintinline{python}{exemple[4:10:2]} est {\tt "ei "}
    \item[\textbullet] \mintinline{python}{exemple[15:8:-1]} est {\tt "elpmexe"} 
    \item[\textbullet] \mintinline{python}{exemple[::-1]} est {\tt "elpmexe titep nU"} le pas étant négatif on progresse de la fin (absente donc dernier caractère) jusqu'au début (absent donc premier caractère).
\end{itemize}	
\end{enumerate}
\end{tcolorbox}

\begin{tcolorbox}[left=0cm,title=\bf{\faPython \; Tuples},colbacktitle=cfond]
    \begin{enumerate}
        \item[\ding{182}] Un tuple est une suite de valeurs repérées par leur indice (à la façon des caractères d'une chaine). Un tuple se note entre {\tt ()} et les valeurs sont séparées par des virgules. Par exemple \mintinline{python}{date = (2, "décembre", 1815)} est un tuple constituées de trois valeurs.
        \item[\ding{183}] On retrouve pour les tuples, la fonction {\tt len}, l'accès au  ième élément avec {\tt [i]} et les tranches déjà vues sur les chaines de caractères.
        \item[\ding{184}] Les valeurs d'un tuple ne sont \textit{pas modifiables} (comme les caractères d'une chaines), une tentative en ce sens produit un {\tt TypeError}
        \item[\ding{185}] Un tuple peut être décompacté afin d'affecter chacune de ses valeurs à une variable. Par exemple \mintinline{python}{jour, mois, annee = date}.
    \end{enumerate}
\end{tcolorbox}

\begin{tcolorbox}[left=0cm,title=\bf{\faPython \; Importation de fonctions},colbacktitle=cfond]
    En Python, on peut importer des fonctions se trouvant dans d'autres modules, deux syntaxes sont possibles :
    \begin{itemize}
        \item[\textbullet] {\tt from <module> import <fonction>}, cela rend directement utilisable {\tt <fonction>} dans la suite du programme. Par exemple la fonction racine carrée s'appelle \mintinline{python}{sqrt} et doit être importé depuis le module \mintinline{python}{math} avec \mintinline{python}{from math import sqrt} pour être utilisable.
        \item[\textbullet] {\tt import <module>}, dans ce cas, toutes les fonctions du module sont utilisables mais on doit préfixer leur nom par celui du module. Par exemple après un \mintinline{python}{import math} pour utiliser la fonction racine carrée, on doit écrire \mintinline{python}{math.sqrt}.
    \end{itemize}
\end{tcolorbox}

\begin{tcolorbox}[left=0cm,title=\bf{\faPython \; Listes},colbacktitle=cfond]
    \begin{enumerate}
        \item[\ding{182}] Une liste est une suite de valeurs repérées par leur indice. Une liste se note entre {\tt [} et {\tt]} et les valeurs sont séparées par des virgules. Par exemple \mintinline{python}{premiers = [2, 3, 5, 7, 11, 13, 15]} est une liste. La liste vide est {\tt []}.
        \item[\ding{183}] On retrouve pour les listes, la fonction {\tt len}, l'accès au  ième élément avec {\tt [i]} et les tranches déjà vues sur les chaines de caractères et les tuples.
        \item[\ding{184}] Les valeurs d'une liste, à la différence de celles d'un tuple,sont \textit{modifiables}, on peut donc écrire \mintinline{python}{premiers[6]=17} afin que la liste ci-dessus devienne \mintinline{python}{premiers = [2, 3, 5, 7, 11, 13, 17]}.
        \item[\ding{185}] On peut ajouter un élément à une liste avec {\tt append}, la syntaxe est {\tt <liste>.append(<element>)}. Par exemple, après exécution de \mintinline{python}{premiers.append(19)} la liste ci-dessus devient \mintinline{python}{premiers = [2, 3, 5, 7, 11, 13, 17, 19]}.
        \item[\ding{186}] On peut retirer le dernier élément d'une liste avec {\tt pop}, la syntaxe est {\tt <liste>.pop()}. L'élément retiré est renvoyé par cette instruction et peut-être récupéré, ainsi \mintinline{python}{n = premiers.pop()} aura deux effets : supprimer 19 de la liste {\tt premiers} et affecter cette valeur à {\tt n}.
        \item[\ding{187}] Création de listes :
        \begin{itemize}
            \item[\textbullet] en donnant explicitement ses éléments (comme la liste {\tt premiers} ci-dessus).
            \item[\textbullet] par répétition avec {\tt *}, par exemple {\tt [77]*10} est la liste constitué de 10 fois le nombre {\tt 77}.
            \item[\textbullet] par ajout successif, on part d'une liste vide et on ajoute (généralement à l'aide d'une boucle {\tt for}) successivement avec {\tt append} les éléments à la liste.
            \item[\textbullet] par compréhension, à la façon dont on définit parfois les ensembles en mathématiques. Par exemple,
            \mintinline{python}{[i for i in range(50) if i%10==7]} est la liste {\tt [7, 17, 27, 37, 47]} (les nombres entre 0 et 49 dont le reste dans la division euclidienne par 10 est 7).
        \end{itemize}
    \end{enumerate}
\end{tcolorbox}


\begin{tcolorbox}[left=0cm,title=\bf{\faPython \; Mutables et non mutables},colbacktitle=cfond]
    En Python, certains types de données sont \textit{mutables} et d'autres non, cela a des conséquences importantes lorsqu'on les manipule. Les listes de Python sont mutables au contraire de tous les autres rencontrés jusqu'ici ({\tt int}, {\tt float}, {\tt bool}, {\tt str}, {\tt tuple}).
    \begin{itemize}
        \item[\textbullet] Un type mutable est modifié lorsqu'on le passe en argument à une fonction pas un type non mutable.
        \setlength{\multicolsep}{0pt}
        \begin{multicols}{2}
            Cas non mutable :
            \begin{python}
def incremente(n):
    n = n + 1

n= 42
incremente(n)
# n vaut toujours 42
\end{python}
            Cas mutable: 
            \begin{python} 
def ajoute(x,l):
    l.append(x)

l = []
ajoute(2,l)
# l vaut [2]
            \end{python}
        \end{multicols}
    \item[\textbullet] Si donne un autre nom à une variable mutable (avec {\tt =}), toute modification de l'une des variables affecte aussi l'autre.
    \setlength{\multicolsep}{0pt}
    \begin{multicols}{2}
        Cas non mutable :
        \begin{python}
n = 42
m = n
m = m + 1
# n vaut toujours 42
\end{python}
        Cas mutable: 
        \begin{python} 
n = [12, 15]
m = n
m.append(17)
#n vaut maintenant [12, 15, 17]
        \end{python}
    \end{multicols}
    \end{itemize}
\end{tcolorbox}

\end{document}
\PassOptionsToPackage{dvipsnames,table}{xcolor}
\documentclass[10pt]{beamer}
\usepackage{Cours}


\begin{document}

\input{\detokenize{/home/fenarius/Travail/Cours/cpge-info/latex//MacrosCours.tex}}

% Numéro et titre de chapitre
\setcounter{numchap}{2}
\newcommand{\Ctitle}{\cnum Programmation dynamique}

\makess{Les dictionnaires de Python}
\begin{frame}{\Ctitle}{\stitle}
	\begin{alertblock}{Les dictionnaires de Python}
		\begin{itemize}
			\item<1-> Les \textcolor{blue}{dictionnaires} de Python permettent de stocker des données sous forme de tableau associant une clé à une valeur : \vspace{0.2cm} \\
				\begin{tabularx}{0.8\textwidth}{l|Y|Y|Y|Y|Y|}
					\cline{2-6}
					\textcolor{blue}{Valeurs}                   & {\tt v1}                       & {\tt v2}                       & {\tt v3}                       & {\tt v4}                       & {\tt \dots}                    \\
					\cline{2-6}
					\multicolumn{1}{c}{$\uparrow$}              & \multicolumn{1}{c}{$\uparrow$} & \multicolumn{1}{c}{$\uparrow$} & \multicolumn{1}{c}{$\uparrow$} & \multicolumn{1}{c}{$\uparrow$} & \multicolumn{1}{c}{$\uparrow$} \\
					\cline{2-6}
					\multicolumn{1}{c|}{\textcolor{blue}{Clés}} & {\tt c1 }                      & {\tt c2}                       & {\tt c3}                       & {\tt c4}                       & {\tt \dots}                    \\
					\cline{2-6}
				\end{tabularx}
			\item<2-> Un dictionnaire se note entre accolades : \kw{\{} et \kw{\}}
			\item<3-> Les paires clés/valeurs sont séparés par des virgules \kw{,}
			\item<4-> Le caractère \kw{:} sépare une clé de la valeur associée.
		\end{itemize}
	\end{alertblock}
	\begin{exampleblock}{Exemples}
		\begin{itemize}
			\item<5-> Un dictionnaire contenant des objets et leurs prix :\\
				\onslide<6-> {\tt \footnotesize prix = \{ "verre":12 , "tasse" : 8, "assiette" : 16\} }
			\item<7-> Un dictionnaire traduisant des couleurs du français vers l'anglais \\
				\onslide<8-> {\tt \footnotesize couleurs = \{ "vert":"green" , "bleu" : "blue", "rouge" : "red" \} }
		\end{itemize}
	\end{exampleblock}
\end{frame}

% Opérations sur un dictionnaire
\begin{frame}{\Ctitle}{\stitle}
	\begin{alertblock}{Opérations sur un dictionnaire}
		\begin{itemize}
			\item<1-> On accède aux éléments d'un dictionnaire avec la syntaxe \textcolor{blue}{\tt nom\_dictionnaire[cle]}\\
				\onslide<2->\textcolor{gray}{\footnotesize {\tt prix = \{ "verre":12 , "tasse" : 8, "assiette" : 16, "plat" : 30 \} } \\
					Par exemple, {\tt prix["verre"]} contient 12}
			\item<3-> On peut ajouter une clé à un dictionnaire existant en effectuant une affectation \textcolor{blue}{\tt nom\_dictionnaire[nouvelle\_cle]=nouvelle\_valeur} \\
				\onslide<4->\textcolor{gray}{\footnotesize On ajoute un nouvel objet avec son prix : \\
				{\tt prix["couteau"]=20}
				}
			\item<5-> On peut modifier la valeur associée à une clé avec une affectation \textcolor{blue}{\tt nom\_dictionnaire[cle]=nouvelle\_valeur}\\
				\onslide<6->\textcolor{gray}{\footnotesize Le pris d'une tasse passe à 10 : \\
				{\tt prix["tasse"]=10}
				}
		\end{itemize}
	\end{alertblock}
\end{frame}

\begin{frame}{\Ctitle}{\stitle}
	\begin{block}{Présence dans un dictionnaire}
		\begin{itemize}
			\item<1-> Attention, essayer d'accéder à une clé qui n'est pas dans un dictionnaire renvoie une erreur !\\
				\onslide<2->\textcolor{gray}{\footnotesize Il n'y a pas de clé {\tt "fourchette"} dans le dictionnaire prix, donc \textcolor{blue}{\tt prix["fourchette"]} renvoie une erreur ({\tt \textcolor{red}{KeyError}}).}
			\item<3-> On teste la présence d'une clé dans un dictionnaire avec \textcolor{blue}{\tt cle in nom\_dictionnaire}\\
				\onslide<4->\textcolor{gray}{\footnotesize la fourchette n'est pas dans le dictionnaire prix \\
					Le test \textcolor{blue}{\tt fourchette in prix} renvoie \textcolor{blue}{\tt False}\\}
				\onslide<5->\textcolor{BrickRed}{\footnotesize \important} \textcolor{BrickRed}{Ce test d'appartenance s'effectue en temps constant (indépendant de la taille du dictionnaire)}
			\item<5-> On peut supprimer une clé existante dans un dictionnaire avec \textcolor{blue}{\tt del nom\_dictionnaire[cle]}\\
				\onslide<6->\textcolor{gray}{\footnotesize On supprimer le couteau : \\
					\textcolor{blue}{\tt del prix["couteau"]}
				}
		\end{itemize}
	\end{block}
\end{frame}


% Opérations sur un dictionnaire
\begin{frame}{\Ctitle}{\stitle}
	\begin{alertblock}{Parcours d'un dictionnaire}
		\begin{itemize}
			\item<1-> Le parcours par clé s'effectue directement avec \textcolor{blue}{\tt for cle in nom\_dictionnaire}\\
				\onslide<2->\textcolor{gray}{{\footnotesize \tt prix = \{ "verre":12 , "tasse" : 8, "assiette" : 16, "plat" : 30 \} } \\
				Par exemple, {\tt for objet in prix} permettra à la variable {\tt objet} de prendre successivement les valeurs des clés : {\tt "verre", "tasse", "assiette"} et {\tt "plat"}.}
			\item<3-> Le parcours par valeur s'effectue en ajoutant \textcolor{blue}{\tt .values()} au nom du dictionnaire : \textcolor{blue}{\tt for valeur in nom\_dictionnaire.values() \\}
				\onslide<4->\textcolor{gray}{\footnotesize
				Par exemple, {\tt for p in prix.values()} permettra à la variable {\tt p} de prendre successivement les valeurs du dictionnaire : {\tt 12, 8 , 16} et {\tt 30}.
				}
		\end{itemize}
	\end{alertblock}
\end{frame}

\begin{frame}{\Ctitle}{\stitle}
	\begin{exampleblock}{Exemple}
		On dispose d'une liste de nombres entiers et on veut obtenir le nombre d'occurence du (ou des) entiers(s) les plus fréquents dans cette liste. Par exemple si la liste est {\tt [1,7,1,3,4,1,3,4,3,1,5,108,2,3]} alors la réponse est 4, car les entiers les plus fréquents sont 1 et 3 qui apparaissent tous les deux à 4 reprises.
		\begin{enumerate}
			\item<1-> Proposer une solution qui pour chaque élément de la liste calcule son nombre d'apparitions à l'aide d'une fonction {\tt compte\_occurence}
			\item<2-> Proposer une solution utilisant un dictionnaire dont les clés sont les entiers présents dans la liste et les valeurs leurs nombre d'apparitions
			\item<3-> Commenter l'efficacité de ces deux solutions.
		\end{enumerate}
	\end{exampleblock}
\end{frame}

\begin{frame}{\Ctitle}{\stitle}
	\begin{exampleblock}{Correction question 1}
		\inputpython{/home/fenarius/Travail/Cours/cpge-info/docs/itc/files/C2/plus_frequent1.py}{}{\footnotesize}
	\end{exampleblock}
\end{frame}

\begin{frame}{\Ctitle}{\stitle}
	\begin{exampleblock}{Correction question 2}
		\inputpython{/home/fenarius/Travail/Cours/cpge-info/docs/itc/files/C2/plus_frequent2.py}{}{\footnotesize}
	\end{exampleblock}
\end{frame}

\begin{frame}{\Ctitle}{\stitle}
	\begin{exampleblock}{Correction question 2}
		\inputpython{/home/fenarius/Travail/Cours/cpge-info/docs/itc/files/C2/plus_frequent2.py}{}{\footnotesize}
	\end{exampleblock}
\end{frame}


\begin{frame}{\Ctitle}{\stitle}
	\begin{exampleblock}{Correction question 3}
		La solution avec les dictionnaires est bien plus efficace car on effectue un seul parcours de la liste et que le test d'appartenance au dictionnaire est une opération élémentaire (temps constant en moyenne).

	\end{exampleblock}
\end{frame}

\makess{Table de hachage}
\begin{frame}{\Ctitle}{\stitle}
	\begin{block}{Implémentation des dictionnaires}
		\begin{itemize}
			\item<1-> On crée un tableau $T$ de liste de longueur $N$ (donc indicé par les entiers $\intN{0}{N-1})$.
			\item<2-> Une \textcolor{blue}{fonction de hachage} $h$ transforme les clés en entier. Les clés doivent donc être \textcolor{blue}{non mutables} (ce qui exclu les listes). Ces entiers sont ramenés dans l'intervalle $\intN{0}{N-1}$ à l'aide d'un modulo.
			\item<3-> Chaque paire de clé/valeur $(c,v)$ est stockée dans le tableau $T$ à l'indice $h(c)$ (modulo $N$)
			\item<4-> Le cas où dont deux clés différentes $c1$ et $c2$ produisent le même indice s'appelle une \textcolor{blue}{collision}.
		\end{itemize}
	\end{block}
\end{frame}



\begin{frame}{\Ctitle}{\stitle}
	\begin{block}{Visualisation}
		{\tt \footnotesize prix = \{ "verre":12 , "tasse" : 8, "assiette" : 16, "bol" : 10\} } \\ \vspace{0.2cm}
		\begin{tabularx}{\textwidth}{X|c|X}
			\cline{2-2}
			{\rnode{verre}{\begin{cadre}{codebg}{blue}{2.2}{0.4}{\footnotesize "verre"}\end{cadre}}} & 0               &                                            \\
			\cline{2-2}
			                                                                                  & \rnode{i1}{1}   & \quad \quad \rnode{v1}{\tt [("Verre",12)]} \\
			\cline{2-2}
			                                                                                  & \rnode{i2}{2}   &                                            \\
			\cline{2-2}
			                                                                                  & \vdots          &                                            \\
			\cline{2-2}
			                                                                                  & \rnode{i42}{42} &                                            \\
			\cline{2-2}
			                                                                                  & \vdots          &                                            \\
			\cline{2-2}
			                                                                                  & $N$-1           &                                            \\
			\cline{2-2}
		\end{tabularx}
		\ncline[nodesepB=0.45,offsetA=0.1,offsetB=-0.05,linewidth=0.8pt,linecolor=brown]{->}{verre}{i1} \naput[nrot=:U,labelsep=0.05]{\textcolor{brown}{\footnotesize hash}}
		\ncline[nodesepA=0.3]{o->}{i1}{v1}
	\end{block}
\end{frame}

\begin{frame}{\Ctitle}{\stitle}
	\begin{block}{Visualisation}
		{\tt \footnotesize prix = \{ "verre":12 , "tasse" : 8, "assiette" : 16, "bol" : 10\} } \\ \vspace{0.2cm}
		\begin{tabularx}{\textwidth}{X|c|X}
			\cline{2-2}
			{\rnode{verre}{\begin{cadre}{codebg}{blue}{2.2}{0.4}{\footnotesize "verre"}\end{cadre}}}       & 0               &                                               \\
			\cline{2-2}
			                                                                                        & \rnode{i1}{1}   & \quad \quad \rnode{v1}{\tt [("Verre",12)]}    \\
			\cline{2-2}
			                                                                                        & \rnode{i2}{2}   & \quad \quad \rnode{v2}{\tt [("assiette",16)]} \\
			\cline{2-2}
			{\rnode{tasse}{\begin{cadre}{codebg}{blue}{2.2}{0.4}{\footnotesize "tasse"}\end{cadre}}}       & \vdots          &                                               \\
			\cline{2-2}
			                                                                                        & \rnode{i42}{42} & \quad \quad \rnode{v42}{\tt [("tasse",8)]}    \\
			\cline{2-2}
			{\rnode{assiette}{\begin{cadre}{codebg}{blue}{2.2}{0.4}{\footnotesize "assiette"}\end{cadre}}} & \vdots          &                                               \\
			\cline{2-2}
			                                                                                        & $N$-1           &                                               \\
			\cline{2-2}
		\end{tabularx}
		\ncline[nodesepB=0.45,offsetA=0.1,offsetB=-0.05,linewidth=0.8pt,linecolor=brown]{->}{verre}{i1} \naput[nrot=:U,labelsep=0.05]{\textcolor{brown}{\footnotesize hash}}
		\ncline[nodesepB=0.42,nodesepA=0.13,offsetA=-0.35,offsetB=0.15,linewidth=0.8pt,linecolor=brown]{->}{assiette}{i2}
		\ncline[nodesepB=0.42,offsetA=0.1,offsetB=-0.05,linewidth=0.8pt,linecolor=brown]{->}{tasse}{i42}
		\ncline[nodesepA=0.3]{o->}{i1}{v1}
		\ncline[nodesepA=0.3]{o->}{i2}{v2}
		\ncline[nodesepA=0.3]{o->}{i42}{v42}
	\end{block}
\end{frame}

\begin{frame}{\Ctitle}{\stitle}
	\begin{block}{Visualisation d'une collision}
		{\tt \footnotesize prix = \{ "verre":12 , "tasse" : 8, "assiette" : 16, "bol" : 10\} } \\ \vspace{0.2cm}
		\begin{tabularx}{\textwidth}{X|c|X}
			\cline{2-2}
			{\rnode{verre}{\begin{cadre}{codebg}{blue}{2.2}{0.4}{\footnotesize "verre"}\end{cadre}}}       & 0                                                                        &                                                       \\
			\cline{2-2}
			                                                                                        & \rnode{i1}{1}                                                            & \quad \quad \rnode{v1}{\tt [("Verre",12)]}            \\
			\cline{2-2}
			                                                                                        & \rnode{i2}{2}                                                            & \quad \quad \rnode{v2}{\tt [("assiette",16)]}         \\
			\cline{2-2}
			{\rnode{tasse}{\begin{cadre}{codebg}{blue}{2.2}{0.4}{\footnotesize "tasse"}\end{cadre}}}       & \vdots                                                                   &                                                       \\
			\cline{2-2}
			                                                                                        & \rnode[linecolor=BrickRed,linewidth=0.02]{i42}{\textcolor{BrickRed}{42}} & \quad \quad \rnode{v42}{\tt [("tasse",8),("bol",10)]} \\
			\cline{2-2}
			{\rnode{assiette}{\begin{cadre}{codebg}{blue}{2.2}{0.4}{\footnotesize "assiette"}\end{cadre}}} & \vdots                                                                   &                                                       \\
			\cline{2-2}
			{\rnode{bol}{\begin{cadre}{codebg}{blue}{2.2}{0.4}{\footnotesize "bol"}\end{cadre}}}           & $N$-1                                                                    &                                                       \\
			\cline{2-2}
		\end{tabularx}
		\ncline[nodesepB=0.45,offsetA=0.1,offsetB=-0.05,linewidth=0.8pt,linecolor=brown]{->}{verre}{i1} \naput[nrot=:U,labelsep=0.05]{\textcolor{brown}{\footnotesize hash}}
		\ncline[nodesepB=0.42,nodesepA=0.13,offsetA=-0.35,offsetB=0.15,linewidth=0.8pt,linecolor=brown]{->}{assiette}{i2}
		\ncline[nodesepB=0.42,offsetA=0.1,offsetB=-0.05,linewidth=0.8pt,linecolor=brown]{->}{tasse}{i42}
		\ncline[nodesepB=0.42,offsetA=-0.2,nodesepA=0.04,offsetB=-0.05,linewidth=0.8pt,linecolor=brown]{->}{bol}{i42}
		\ncline[nodesepA=0.3]{o->}{i1}{v1}
		\ncline[nodesepA=0.3]{o->}{i2}{v2}
		\ncline[nodesepA=0.3]{o->}{i42}{v42} \rput(7.1,1.65){\textcolor{BrickRed}{\scriptsize Collision}}
	\end{block}
\end{frame}


\begin{frame}{\Ctitle}{\stitle}
	\begin{block}{Visualisation d'une collision}
		{\tt \footnotesize prix = \{ "verre":12 , "tasse" : 8, "assiette" : 16, "bol" : 10\} } \\ \vspace{0.2cm}
		\begin{tabularx}{\textwidth}{X|c|X}
			\cline{2-2}
			{\rnode{verre}{\begin{cadre}{codebg}{blue}{2.2}{0.4}{\footnotesize "verre"}\end{cadre}}}       & 0                                                                        &                                                       \\
			\cline{2-2}
			                                                                                        & \rnode{i1}{1}                                                            & \quad \quad \rnode{v1}{\tt [("Verre",12)]}            \\
			\cline{2-2}
			                                                                                        & \rnode{i2}{2}                                                            & \quad \quad \rnode{v2}{\tt [("assiette",16)]}         \\
			\cline{2-2}
			{\rnode{tasse}{\begin{cadre}{codebg}{blue}{2.2}{0.4}{\footnotesize "tasse"}\end{cadre}}}       & \vdots                                                                   &                                                       \\
			\cline{2-2}
			                                                                                        & \rnode[linecolor=BrickRed,linewidth=0.02]{i42}{\textcolor{BrickRed}{42}} & \quad \quad \rnode{v42}{\tt [("tasse",8),("bol",10)]} \\
			\cline{2-2}
			{\rnode{assiette}{\begin{cadre}{codebg}{blue}{2.2}{0.4}{\footnotesize "assiette"}\end{cadre}}} & \vdots                                                                   &                                                       \\
			\cline{2-2}
			{\rnode{bol}{\begin{cadre}{codebg}{blue}{2.2}{0.4}{\footnotesize "bol"}\end{cadre}}}           & $N$-1                                                                    &                                                       \\
			\cline{2-2}
		\end{tabularx}
		\ncline[nodesepB=0.45,offsetA=0.1,offsetB=-0.05,linewidth=0.8pt,linecolor=brown]{->}{verre}{i1} \naput[nrot=:U,labelsep=0.05]{\textcolor{brown}{\footnotesize hash}}
		\ncline[nodesepB=0.42,nodesepA=0.13,offsetA=-0.35,offsetB=0.15,linewidth=0.8pt,linecolor=brown]{->}{assiette}{i2}
		\ncline[nodesepB=0.42,offsetA=0.1,offsetB=-0.05,linewidth=0.8pt,linecolor=brown]{->}{tasse}{i42}
		\ncline[nodesepB=0.42,offsetA=-0.2,nodesepA=0.04,offsetB=-0.05,linewidth=0.8pt,linecolor=brown]{->}{bol}{i42}
		\ncline[nodesepA=0.3]{o->}{i1}{v1}
		\ncline[nodesepA=0.3]{o->}{i2}{v2}
		\ncline[nodesepA=0.3]{o->}{i42}{v42} \rput(7.1,1.85){\textcolor{BrickRed}{\scriptsize Collision}}
		Pour rechercher si une clé est présente dans le dictionnaire il suffit de calculer son \textit{hash} et de regarder à l'indice correspondant dans le tableau.
	\end{block}
\end{frame}

\makess{Programmation dynamique : exemples}
\begin{frame}{\Ctitle}{\stitle}
	\begin{exampleblock}{Exemple introductif}
		\begin{enumerate}
			\item<1-> Ecrire une fonction récursive qui prend en argument un entier $n$ et renvoie le $n$ième terme de la suite de Fibonacci défini par :
				$\left\{ \begin{array}{lll}
						f_0   & = & 0,                                                  \\
						f_1   & = & 1,                                                  \\
						f_{n} & = & f_{n-1}+f_{n-2} \mathrm{\ \ pour\ tout\ \ } n\geq2.\end{array} \right.$
			\item<2-> Tracer le graphe des appels récursifs de cette fonction pour $n=5$
			\item<3-> Conclure
		\end{enumerate}
	\end{exampleblock}
\end{frame}

\begin{frame}{\Ctitle}{\stitle}
	\begin{exampleblock}{Correction question 1}
		\inputpython{/home/fenarius/Travail/Cours/cpge-info/docs/itc/files/C2/fibo_rec.py}{}{\footnotesize}
	\end{exampleblock}
\end{frame}

\begin{frame}{\Ctitle}{\stitle}
	\begin{exampleblock}{Correction questions 2-3}
		\begin{center}
			\psset{levelsep=1cm,treesep=0.2cm,linecolor=OliveGreen,linewidth=0.6pt}
			\pstree{\Toval{\footnotesize fibo(5)}}{
				\pstree{\Toval{\footnotesize fibo(4)}}{
					\pstree{\Toval{\footnotesize fibo(3)}}{
						\pstree{\Toval{\footnotesize \textcolor{BrickRed}{fibo(2)}}}{\Toval{\footnotesize fibo(1)} \Toval{\footnotesize fibo(0)}}
						\Toval{\footnotesize fibo(1)}}
					\pstree{\Toval{\footnotesize \textcolor{BrickRed}{fibo(2)}}}{\Toval{\footnotesize fibo(1)} \Toval{\footnotesize fibo(0)}}
				}
				\pstree{\Toval{\footnotesize fibo(3)}}{
					\pstree{\Toval{\footnotesize \textcolor{BrickRed}{fibo(2)}}}{\Toval{\footnotesize fibo(1)} \Toval{\footnotesize fibo(0)}}
					\Toval{\footnotesize fibo(1)}}
			}
		\end{center}
		On calcule à plusieurs reprises fibo(2).
	\end{exampleblock}
\end{frame}

\makess{Programmation dynamique : formalisation}
\begin{frame}{\Ctitle}{\stitle}
	\begin{alertblock}{Tentative de définition}
		La programmation dynamique s'applique généralement à la résolution d'un problème d'optimisation qui possède les propriétés suivantes :
		\begin{enumerate}
			\item<2-> ce problème peut-être résolu à partir de problèmes similaires mais plus petits,
			\item<3-> la solution au problème initial s'obtient en combinant les solutions des problèmes.
		\end{enumerate}
		\onslide<4->{Dans le cas où les sous-problèmes se chevauchent, afin de ne pas recalculer plusieurs fois la solution à un même sous problème, on utilise la technique de mémoïsation. }\\
		\onslide<5->{En Python, la mémoïsation sera implémentée à l'aide d'un dictionnaire dont les clés sont les arguments déjà calculés des sous problèmes et les valeurs leurs réponses.}
	\end{alertblock}
\end{frame}

\end{document}
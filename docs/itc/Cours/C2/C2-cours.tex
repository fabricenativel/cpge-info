\PassOptionsToPackage{dvipsnames,table}{xcolor}
\documentclass[10pt]{beamer}
\usepackage{Cours}


\begin{document}


\newcounter{numchap}
\setcounter{numchap}{1}
\newcounter{numframe}
\setcounter{numframe}{0}
\newcommand{\mframe}[1]{\frametitle{#1} \addtocounter{numframe}{1}}
\newcommand{\cnum}{\fbox{\textcolor{yellow}{\textbf{C\thenumchap}}}~}
\newcommand{\makess}[1]{\section{#1} \label{ss\thesection}}
\newcommand{\stitle}{\textcolor{yellow}{\textbf{\thesection. \nameref{ss\thesection}}}}

\definecolor{codebg}{gray}{0.90}
\definecolor{grispale}{gray}{0.95}
\definecolor{fluo}{rgb}{1,0.96,0.62}
\newminted[langageC]{c}{linenos=true,escapeinside=||,highlightcolor=fluo,tabsize=2,breaklines=true}
\newminted[codepython]{python}{linenos=true,escapeinside=||,highlightcolor=fluo,tabsize=2,breaklines=true}
% Inclusion complète (ou partiel en indiquant premiere et dernière ligne) d'un fichier C
\newcommand{\inputC}[3]{\begin{mdframed}[backgroundcolor=codebg] \inputminted[breaklines=true,fontsize=#3,linenos=true,highlightcolor=fluo,tabsize=2,highlightlines={#2}]{c}{#1} \end{mdframed}}
\newcommand{\inputpartC}[5]{\begin{mdframed}[backgroundcolor=codebg] \inputminted[breaklines=true,fontsize=#3,linenos=true,highlightcolor=fluo,tabsize=2,highlightlines={#2},firstline=#4,lastline=#5,firstnumber=1]{c}{#1} \end{mdframed}}
\newcommand{\inputpython}[3]{\begin{mdframed}[backgroundcolor=codebg] \inputminted[breaklines=true,fontsize=#3,linenos=true,highlightcolor=fluo,tabsize=2,highlightlines={#2}]{python}{#1} \end{mdframed}}
\newcommand{\inputpartOCaml}[5]{\begin{mdframed}[backgroundcolor=codebg] \inputminted[breaklines=true,fontsize=#3,linenos=true,highlightcolor=fluo,tabsize=2,highlightlines={#2},firstline=#4,lastline=#5,firstnumber=1]{OCaml}{#1} \end{mdframed}}
\BeforeBeginEnvironment{minted}{\begin{mdframed}[backgroundcolor=codebg]}
\AfterEndEnvironment{minted}{\end{mdframed}}
\newcommand{\kw}[1]{\textcolor{blue}{\tt #1}}

\newtcolorbox{rcadre}[4]{halign=center,colback={#1},colframe={#2},width={#3cm},height={#4cm},valign=center,boxrule=1pt,left=0pt,right=0pt}
\newtcolorbox{cadre}[4]{halign=center,colback={#1},colframe={#2},arc=0mm,width={#3cm},height={#4cm},valign=center,boxrule=1pt,left=0pt,right=0pt}
\newcommand{\myem}[1]{\colorbox{fluo}{#1}}
\mdfsetup{skipabove=1pt,skipbelow=-2pt}



% Noeud dans un cadre pour les arbres
\newcommand{\noeud}[2]{\Tr{\fbox{\textcolor{#1}{\tt #2}}}}

\newcommand{\htmlmode}{\lstset{language=html,numbers=left, tabsize=4, frame=single, breaklines=true, keywordstyle=\ttfamily, basicstyle=\small,
   numberstyle=\tiny\ttfamily, framexleftmargin=0mm, backgroundcolor=\color{grispale}, xleftmargin=12mm,showstringspaces=false}}
\newcommand{\pythonmode}{\lstset{
   language=python,
   linewidth=\linewidth,
   numbers=left,
   tabsize=4,
   frame=single,
   breaklines=true,
   keywordstyle=\ttfamily\color{blue},
   basicstyle=\small,
   numberstyle=\tiny\ttfamily,
   framexleftmargin=-2mm,
   numbersep=-0.5mm,
   backgroundcolor=\color{codebg},
   xleftmargin=-1mm, 
   showstringspaces=false,
   commentstyle=\color{gray},
   stringstyle=\color{OliveGreen},
   emph={turtle,Screen,Turtle},
   emphstyle=\color{RawSienna},
   morekeywords={setheading,goto,backward,forward,left,right,pendown,penup,pensize,color,speed,hideturtle,showturtle,forward}}
   }
   \newcommand{\Cmode}{\lstset{
      language=[ANSI]C,
      linewidth=\linewidth,
      numbers=left,
      tabsize=4,
      frame=single,
      breaklines=true,
      keywordstyle=\ttfamily\color{blue},
      basicstyle=\small,
      numberstyle=\tiny\ttfamily,
      framexleftmargin=0mm,
      numbersep=2mm,
      backgroundcolor=\color{codebg},
      xleftmargin=0mm, 
      showstringspaces=false,
      commentstyle=\color{gray},
      stringstyle=\color{OliveGreen},
      emphstyle=\color{RawSienna},
      escapechar=\|,
      morekeywords={}}
      }
\newcommand{\bashmode}{\lstset{language=bash,numbers=left, tabsize=2, frame=single, breaklines=true, basicstyle=\ttfamily,
   numberstyle=\tiny\ttfamily, framexleftmargin=0mm, backgroundcolor=\color{grispale}, xleftmargin=12mm, showstringspaces=false}}
\newcommand{\exomode}{\lstset{language=python,numbers=left, tabsize=2, frame=single, breaklines=true, basicstyle=\ttfamily,
   numberstyle=\tiny\ttfamily, framexleftmargin=13mm, xleftmargin=12mm, basicstyle=\small, showstringspaces=false}}
   
   
  
%tei pour placer les images
%tei{nom de l’image}{échelle de l’image}{sens}{texte a positionner}
%sens ="1" (droite) ou "2" (gauche)
\newlength{\ltxt}
\newcommand{\tei}[4]{
\setlength{\ltxt}{\linewidth}
\setbox0=\hbox{\includegraphics[scale=#2]{#1}}
\addtolength{\ltxt}{-\wd0}
\addtolength{\ltxt}{-10pt}
\ifthenelse{\equal{#3}{1}}{
\begin{minipage}{\wd0}
\includegraphics[scale=#2]{#1}
\end{minipage}
\hfill
\begin{minipage}{\ltxt}
#4
\end{minipage}
}{
\begin{minipage}{\ltxt}
#4
\end{minipage}
\hfill
\begin{minipage}{\wd0}
\includegraphics[scale=#2]{#1}
\end{minipage}
}
}

%Juxtaposition d'une image pspciture et de texte 
%#1: = code pstricks de l'image
%#2: largeur de l'image
%#3: hauteur de l'image
%#4: Texte à écrire
\newcommand{\ptp}[4]{
\setlength{\ltxt}{\linewidth}
\addtolength{\ltxt}{-#2 cm}
\addtolength{\ltxt}{-0.1 cm}
\begin{minipage}[b][#3 cm][t]{\ltxt}
#4
\end{minipage}\hfill
\begin{minipage}[b][#3 cm][c]{#2 cm}
#1
\end{minipage}\par
}



%Macros pour les graphiques
\psset{linewidth=0.5\pslinewidth,PointSymbol=x}
\setlength{\fboxrule}{0.5pt}
\newcounter{tempangle}

%Marque la longueur du segment d'extrémité  #1 et  #2 avec la valeur #3, #4 est la distance par rapport au segment (en %age de la valeur de celui ci) et #5 l'orientation du marquage : +90 ou -90
\newcommand{\afflong}[5]{
\pstRotation[RotAngle=#4,PointSymbol=none,PointName=none]{#1}{#2}[X] 
\pstHomO[PointSymbol=none,PointName=none,HomCoef=#5]{#1}{X}[Y]
\pstTranslation[PointSymbol=none,PointName=none]{#1}{#2}{Y}[Z]
 \ncline{|<->|,linewidth=0.25\pslinewidth}{Y}{Z} \ncput*[nrot=:U]{\footnotesize{#3}}
}
\newcommand{\afflongb}[3]{
\ncline{|<->|,linewidth=0}{#1}{#2} \naput*[nrot=:U]{\footnotesize{#3}}
}

%Construis le point #4 situé à #2 cm du point #1 avant un angle #3 par rapport à l'horizontale. #5 = liste de paramètre
\newcommand{\lsegment}[5]{\pstGeonode[PointSymbol=none,PointName=none](0,0){O'}(#2,0){I'} \pstTranslation[PointSymbol=none,PointName=none]{O'}{I'}{#1}[J'] \pstRotation[RotAngle=#3,PointSymbol=x,#5]{#1}{J'}[#4]}
\newcommand{\tsegment}[5]{\pstGeonode[PointSymbol=none,PointName=none](0,0){O'}(#2,0){I'} \pstTranslation[PointSymbol=none,PointName=none]{O'}{I'}{#1}[J'] \pstRotation[RotAngle=#3,PointSymbol=x,#5]{#1}{J'}[#4] \pstLineAB{#4}{#1}}

%Construis le point #4 situé à #3 cm du point #1 et faisant un angle de  90° avec la droite (#1,#2) #5 = liste de paramètre
\newcommand{\psegment}[5]{
\pstGeonode[PointSymbol=none,PointName=none](0,0){O'}(#3,0){I'}
 \pstTranslation[PointSymbol=none,PointName=none]{O'}{I'}{#1}[J']
 \pstInterLC[PointSymbol=none,PointName=none]{#1}{#2}{#1}{J'}{M1}{M2} \pstRotation[RotAngle=-90,PointSymbol=x,#5]{#1}{M1}[#4]
  }
  
%Construis le point #4 situé à #3 cm du point #1 et faisant un angle de  #5° avec la droite (#1,#2) #6 = liste de paramètre
\newcommand{\mlogo}[6]{
\pstGeonode[PointSymbol=none,PointName=none](0,0){O'}(#3,0){I'}
 \pstTranslation[PointSymbol=none,PointName=none]{O'}{I'}{#1}[J']
 \pstInterLC[PointSymbol=none,PointName=none]{#1}{#2}{#1}{J'}{M1}{M2} \pstRotation[RotAngle=#5,PointSymbol=x,#6]{#1}{M2}[#4]
  }

% Construis un triangle avec #1=liste des 3 sommets séparés par des virgules, #2=liste des 3 longueurs séparés par des virgules, #3 et #4 : paramètre d'affichage des 2e et 3 points et #5 : inclinaison par rapport à l'horizontale
%autre macro identique mais sans tracer les segments joignant les sommets
\noexpandarg
\newcommand{\Triangleccc}[5]{
\StrBefore{#1}{,}[\pointA]
\StrBetween[1,2]{#1}{,}{,}[\pointB]
\StrBehind[2]{#1}{,}[\pointC]
\StrBefore{#2}{,}[\coteA]
\StrBetween[1,2]{#2}{,}{,}[\coteB]
\StrBehind[2]{#2}{,}[\coteC]
\tsegment{\pointA}{\coteA}{#5}{\pointB}{#3} 
\lsegment{\pointA}{\coteB}{0}{Z1}{PointSymbol=none, PointName=none}
\lsegment{\pointB}{\coteC}{0}{Z2}{PointSymbol=none, PointName=none}
\pstInterCC{\pointA}{Z1}{\pointB}{Z2}{\pointC}{Z3} 
\pstLineAB{\pointA}{\pointC} \pstLineAB{\pointB}{\pointC}
\pstSymO[PointName=\pointC,#4]{C}{C}[C]
}
\noexpandarg
\newcommand{\TrianglecccP}[5]{
\StrBefore{#1}{,}[\pointA]
\StrBetween[1,2]{#1}{,}{,}[\pointB]
\StrBehind[2]{#1}{,}[\pointC]
\StrBefore{#2}{,}[\coteA]
\StrBetween[1,2]{#2}{,}{,}[\coteB]
\StrBehind[2]{#2}{,}[\coteC]
\tsegment{\pointA}{\coteA}{#5}{\pointB}{#3} 
\lsegment{\pointA}{\coteB}{0}{Z1}{PointSymbol=none, PointName=none}
\lsegment{\pointB}{\coteC}{0}{Z2}{PointSymbol=none, PointName=none}
\pstInterCC[PointNameB=none,PointSymbolB=none,#4]{\pointA}{Z1}{\pointB}{Z2}{\pointC}{Z1} 
}


% Construis un triangle avec #1=liste des 3 sommets séparés par des virgules, #2=liste formée de 2 longueurs et d'un angle séparés par des virgules, #3 et #4 : paramètre d'affichage des 2e et 3 points et #5 : inclinaison par rapport à l'horizontale
%autre macro identique mais sans tracer les segments joignant les sommets
\newcommand{\Trianglecca}[5]{
\StrBefore{#1}{,}[\pointA]
\StrBetween[1,2]{#1}{,}{,}[\pointB]
\StrBehind[2]{#1}{,}[\pointC]
\StrBefore{#2}{,}[\coteA]
\StrBetween[1,2]{#2}{,}{,}[\coteB]
\StrBehind[2]{#2}{,}[\angleA]
\tsegment{\pointA}{\coteA}{#5}{\pointB}{#3} 
\setcounter{tempangle}{#5}
\addtocounter{tempangle}{\angleA}
\tsegment{\pointA}{\coteB}{\thetempangle}{\pointC}{#4}
\pstLineAB{\pointB}{\pointC}
}
\newcommand{\TriangleccaP}[5]{
\StrBefore{#1}{,}[\pointA]
\StrBetween[1,2]{#1}{,}{,}[\pointB]
\StrBehind[2]{#1}{,}[\pointC]
\StrBefore{#2}{,}[\coteA]
\StrBetween[1,2]{#2}{,}{,}[\coteB]
\StrBehind[2]{#2}{,}[\angleA]
\lsegment{\pointA}{\coteA}{#5}{\pointB}{#3} 
\setcounter{tempangle}{#5}
\addtocounter{tempangle}{\angleA}
\lsegment{\pointA}{\coteB}{\thetempangle}{\pointC}{#4}
}

% Construis un triangle avec #1=liste des 3 sommets séparés par des virgules, #2=liste formée de 1 longueurs et de deux angle séparés par des virgules, #3 et #4 : paramètre d'affichage des 2e et 3 points et #5 : inclinaison par rapport à l'horizontale
%autre macro identique mais sans tracer les segments joignant les sommets
\newcommand{\Trianglecaa}[5]{
\StrBefore{#1}{,}[\pointA]
\StrBetween[1,2]{#1}{,}{,}[\pointB]
\StrBehind[2]{#1}{,}[\pointC]
\StrBefore{#2}{,}[\coteA]
\StrBetween[1,2]{#2}{,}{,}[\angleA]
\StrBehind[2]{#2}{,}[\angleB]
\tsegment{\pointA}{\coteA}{#5}{\pointB}{#3} 
\setcounter{tempangle}{#5}
\addtocounter{tempangle}{\angleA}
\lsegment{\pointA}{1}{\thetempangle}{Z1}{PointSymbol=none, PointName=none}
\setcounter{tempangle}{#5}
\addtocounter{tempangle}{180}
\addtocounter{tempangle}{-\angleB}
\lsegment{\pointB}{1}{\thetempangle}{Z2}{PointSymbol=none, PointName=none}
\pstInterLL[#4]{\pointA}{Z1}{\pointB}{Z2}{\pointC}
\pstLineAB{\pointA}{\pointC}
\pstLineAB{\pointB}{\pointC}
}
\newcommand{\TrianglecaaP}[5]{
\StrBefore{#1}{,}[\pointA]
\StrBetween[1,2]{#1}{,}{,}[\pointB]
\StrBehind[2]{#1}{,}[\pointC]
\StrBefore{#2}{,}[\coteA]
\StrBetween[1,2]{#2}{,}{,}[\angleA]
\StrBehind[2]{#2}{,}[\angleB]
\lsegment{\pointA}{\coteA}{#5}{\pointB}{#3} 
\setcounter{tempangle}{#5}
\addtocounter{tempangle}{\angleA}
\lsegment{\pointA}{1}{\thetempangle}{Z1}{PointSymbol=none, PointName=none}
\setcounter{tempangle}{#5}
\addtocounter{tempangle}{180}
\addtocounter{tempangle}{-\angleB}
\lsegment{\pointB}{1}{\thetempangle}{Z2}{PointSymbol=none, PointName=none}
\pstInterLL[#4]{\pointA}{Z1}{\pointB}{Z2}{\pointC}
}

%Construction d'un cercle de centre #1 et de rayon #2 (en cm)
\newcommand{\Cercle}[2]{
\lsegment{#1}{#2}{0}{Z1}{PointSymbol=none, PointName=none}
\pstCircleOA{#1}{Z1}
}

%construction d'un parallélogramme #1 = liste des sommets, #2 = liste contenant les longueurs de 2 côtés consécutifs et leurs angles;  #3, #4 et #5 : paramètre d'affichage des sommets #6 inclinaison par rapport à l'horizontale 
% meme macro sans le tracé des segements
\newcommand{\Para}[6]{
\StrBefore{#1}{,}[\pointA]
\StrBetween[1,2]{#1}{,}{,}[\pointB]
\StrBetween[2,3]{#1}{,}{,}[\pointC]
\StrBehind[3]{#1}{,}[\pointD]
\StrBefore{#2}{,}[\longueur]
\StrBetween[1,2]{#2}{,}{,}[\largeur]
\StrBehind[2]{#2}{,}[\angle]
\tsegment{\pointA}{\longueur}{#6}{\pointB}{#3} 
\setcounter{tempangle}{#6}
\addtocounter{tempangle}{\angle}
\tsegment{\pointA}{\largeur}{\thetempangle}{\pointD}{#5}
\pstMiddleAB[PointName=none,PointSymbol=none]{\pointB}{\pointD}{Z1}
\pstSymO[#4]{Z1}{\pointA}[\pointC]
\pstLineAB{\pointB}{\pointC}
\pstLineAB{\pointC}{\pointD}
}
\newcommand{\ParaP}[6]{
\StrBefore{#1}{,}[\pointA]
\StrBetween[1,2]{#1}{,}{,}[\pointB]
\StrBetween[2,3]{#1}{,}{,}[\pointC]
\StrBehind[3]{#1}{,}[\pointD]
\StrBefore{#2}{,}[\longueur]
\StrBetween[1,2]{#2}{,}{,}[\largeur]
\StrBehind[2]{#2}{,}[\angle]
\lsegment{\pointA}{\longueur}{#6}{\pointB}{#3} 
\setcounter{tempangle}{#6}
\addtocounter{tempangle}{\angle}
\lsegment{\pointA}{\largeur}{\thetempangle}{\pointD}{#5}
\pstMiddleAB[PointName=none,PointSymbol=none]{\pointB}{\pointD}{Z1}
\pstSymO[#4]{Z1}{\pointA}[\pointC]
}


%construction d'un cerf-volant #1 = liste des sommets, #2 = liste contenant les longueurs de 2 côtés consécutifs et leurs angles;  #3, #4 et #5 : paramètre d'affichage des sommets #6 inclinaison par rapport à l'horizontale 
% meme macro sans le tracé des segements
\newcommand{\CerfVolant}[6]{
\StrBefore{#1}{,}[\pointA]
\StrBetween[1,2]{#1}{,}{,}[\pointB]
\StrBetween[2,3]{#1}{,}{,}[\pointC]
\StrBehind[3]{#1}{,}[\pointD]
\StrBefore{#2}{,}[\longueur]
\StrBetween[1,2]{#2}{,}{,}[\largeur]
\StrBehind[2]{#2}{,}[\angle]
\tsegment{\pointA}{\longueur}{#6}{\pointB}{#3} 
\setcounter{tempangle}{#6}
\addtocounter{tempangle}{\angle}
\tsegment{\pointA}{\largeur}{\thetempangle}{\pointD}{#5}
\pstOrtSym[#4]{\pointB}{\pointD}{\pointA}[\pointC]
\pstLineAB{\pointB}{\pointC}
\pstLineAB{\pointC}{\pointD}
}

%construction d'un quadrilatère quelconque #1 = liste des sommets, #2 = liste contenant les longueurs des 4 côtés et l'angle entre 2 cotés consécutifs  #3, #4 et #5 : paramètre d'affichage des sommets #6 inclinaison par rapport à l'horizontale 
% meme macro sans le tracé des segements
\newcommand{\Quadri}[6]{
\StrBefore{#1}{,}[\pointA]
\StrBetween[1,2]{#1}{,}{,}[\pointB]
\StrBetween[2,3]{#1}{,}{,}[\pointC]
\StrBehind[3]{#1}{,}[\pointD]
\StrBefore{#2}{,}[\coteA]
\StrBetween[1,2]{#2}{,}{,}[\coteB]
\StrBetween[2,3]{#2}{,}{,}[\coteC]
\StrBetween[3,4]{#2}{,}{,}[\coteD]
\StrBehind[4]{#2}{,}[\angle]
\tsegment{\pointA}{\coteA}{#6}{\pointB}{#3} 
\setcounter{tempangle}{#6}
\addtocounter{tempangle}{\angle}
\tsegment{\pointA}{\coteD}{\thetempangle}{\pointD}{#5}
\lsegment{\pointB}{\coteB}{0}{Z1}{PointSymbol=none, PointName=none}
\lsegment{\pointD}{\coteC}{0}{Z2}{PointSymbol=none, PointName=none}
\pstInterCC[PointNameA=none,PointSymbolA=none,#4]{\pointB}{Z1}{\pointD}{Z2}{Z3}{\pointC} 
\pstLineAB{\pointB}{\pointC}
\pstLineAB{\pointC}{\pointD}
}


% Définition des colonnes centrées ou à droite pour tabularx
\newcolumntype{Y}{>{\centering\arraybackslash}X}
\newcolumntype{Z}{>{\flushright\arraybackslash}X}

%Les pointillés à remplir par les élèves
\newcommand{\po}[1]{\makebox[#1 cm]{\dotfill}}
\newcommand{\lpo}[1][3]{%
\multido{}{#1}{\makebox[\linewidth]{\dotfill}
}}

%Liste des pictogrammes utilisés sur la fiche d'exercice ou d'activités
\newcommand{\bombe}{\faBomb}
\newcommand{\livre}{\faBook}
\newcommand{\calculatrice}{\faCalculator}
\newcommand{\oral}{\faCommentO}
\newcommand{\surfeuille}{\faEdit}
\newcommand{\ordinateur}{\faLaptop}
\newcommand{\ordi}{\faDesktop}
\newcommand{\ciseaux}{\faScissors}
\newcommand{\danger}{\faExclamationTriangle}
\newcommand{\out}{\faSignOut}
\newcommand{\cadeau}{\faGift}
\newcommand{\flash}{\faBolt}
\newcommand{\lumiere}{\faLightbulb}
\newcommand{\compas}{\dsmathematical}
\newcommand{\calcullitteral}{\faTimesCircleO}
\newcommand{\raisonnement}{\faCogs}
\newcommand{\recherche}{\faSearch}
\newcommand{\rappel}{\faHistory}
\newcommand{\video}{\faFilm}
\newcommand{\capacite}{\faPuzzlePiece}
\newcommand{\aide}{\faLifeRing}
\newcommand{\loin}{\faExternalLink}
\newcommand{\groupe}{\faUsers}
\newcommand{\bac}{\faGraduationCap}
\newcommand{\histoire}{\faUniversity}
\newcommand{\coeur}{\faSave}
\newcommand{\python}{\faPython}
\newcommand{\os}{\faMicrochip}
\newcommand{\rd}{\faCubes}
\newcommand{\data}{\faColumns}
\newcommand{\web}{\faCode}
\newcommand{\prog}{\faFile}
\newcommand{\algo}{\faCogs}
\newcommand{\important}{\faExclamationCircle}
\newcommand{\maths}{\faTimesCircle}
% Traitement des données en tables
\newcommand{\tables}{\faColumns}
% Types construits
\newcommand{\construits}{\faCubes}
% Type et valeurs de base
\newcommand{\debase}{{\footnotesize \faCube}}
% Systèmes d'exploitation
\newcommand{\linux}{\faLinux}
\newcommand{\sd}{\faProjectDiagram}
\newcommand{\bd}{\faDatabase}

%Les ensembles de nombres
\renewcommand{\N}{\mathbb{N}}
\newcommand{\D}{\mathbb{D}}
\newcommand{\Z}{\mathbb{Z}}
\newcommand{\Q}{\mathbb{Q}}
\newcommand{\R}{\mathbb{R}}
\newcommand{\C}{\mathbb{C}}

%Ecriture des vecteurs
\newcommand{\vect}[1]{\vbox{\halign{##\cr 
  \tiny\rightarrowfill\cr\noalign{\nointerlineskip\vskip1pt} 
  $#1\mskip2mu$\cr}}}


%Compteur activités/exos et question et mise en forme titre et questions
\newcounter{numact}
\setcounter{numact}{1}
\newcounter{numseance}
\setcounter{numseance}{1}
\newcounter{numexo}
\setcounter{numexo}{0}
\newcounter{numprojet}
\setcounter{numprojet}{0}
\newcounter{numquestion}
\newcommand{\espace}[1]{\rule[-1ex]{0pt}{#1 cm}}
\newcommand{\Quest}[3]{
\addtocounter{numquestion}{1}
\begin{tabularx}{\textwidth}{X|m{1cm}|}
\cline{2-2}
\textbf{\sffamily{\alph{numquestion})}} #1 & \dots / #2 \\
\hline 
\multicolumn{2}{|l|}{\espace{#3}} \\
\hline
\end{tabularx}
}
\newcommand{\QuestR}[3]{
\addtocounter{numquestion}{1}
\begin{tabularx}{\textwidth}{X|m{1cm}|}
\cline{2-2}
\textbf{\sffamily{\alph{numquestion})}} #1 & \dots / #2 \\
\hline 
\multicolumn{2}{|l|}{\cor{#3}} \\
\hline
\end{tabularx}
}
\newcommand{\Pre}{{\sc nsi} 1\textsuperscript{e}}
\newcommand{\Term}{{\sc nsi} Terminale}
\newcommand{\Sec}{2\textsuperscript{e}}
\newcommand{\Exo}[2]{ \addtocounter{numexo}{1} \ding{113} \textbf{\sffamily{Exercice \thenumexo}} : \textit{#1} \hfill #2  \setcounter{numquestion}{0}}
\newcommand{\Projet}[1]{ \addtocounter{numprojet}{1} \ding{118} \textbf{\sffamily{Projet \thenumprojet}} : \textit{#1}}
\newcommand{\ExoD}[2]{ \addtocounter{numexo}{1} \ding{113} \textbf{\sffamily{Exercice \thenumexo}}  \textit{(#1 pts)} \hfill #2  \setcounter{numquestion}{0}}
\newcommand{\ExoB}[2]{ \addtocounter{numexo}{1} \ding{113} \textbf{\sffamily{Exercice \thenumexo}}  \textit{(Bonus de +#1 pts maximum)} \hfill #2  \setcounter{numquestion}{0}}
\newcommand{\Act}[2]{ \ding{113} \textbf{\sffamily{Activité \thenumact}} : \textit{#1} \hfill #2  \addtocounter{numact}{1} \setcounter{numquestion}{0}}
\newcommand{\Seance}{ \rule{1.5cm}{0.5pt}\raisebox{-3pt}{\framebox[4cm]{\textbf{\sffamily{Séance \thenumseance}}}}\hrulefill  \\
  \addtocounter{numseance}{1}}
\newcommand{\Acti}[2]{ {\footnotesize \ding{117}} \textbf{\sffamily{Activité \thenumact}} : \textit{#1} \hfill #2  \addtocounter{numact}{1} \setcounter{numquestion}{0}}
\newcommand{\titre}[1]{\begin{Large}\textbf{\ding{118}}\end{Large} \begin{large}\textbf{ #1}\end{large} \vspace{0.2cm}}
\newcommand{\QListe}[1][0]{
\ifthenelse{#1=0}
{\begin{enumerate}[partopsep=0pt,topsep=0pt,parsep=0pt,itemsep=0pt,label=\textbf{\sffamily{\arabic*.}},series=question]}
{\begin{enumerate}[resume*=question]}}
\newcommand{\SQListe}[1][0]{
\ifthenelse{#1=0}
{\begin{enumerate}[partopsep=0pt,topsep=0pt,parsep=0pt,itemsep=0pt,label=\textbf{\sffamily{\alph*)}},series=squestion]}
{\begin{enumerate}[resume*=squestion]}}
\newcommand{\SQListeL}[1][0]{
\ifthenelse{#1=0}
{\begin{enumerate*}[partopsep=0pt,topsep=0pt,parsep=0pt,itemsep=0pt,label=\textbf{\sffamily{\alph*)}},series=squestion]}
{\begin{enumerate*}[resume*=squestion]}}
\newcommand{\FinListe}{\end{enumerate}}
\newcommand{\FinListeL}{\end{enumerate*}}

%Mise en forme de la correction
\newcommand{\cor}[1]{\par \textcolor{OliveGreen}{#1}}
\newcommand{\br}[1]{\cor{\textbf{#1}}}
\newcommand{\tcor}[1]{\begin{tcolorbox}[width=0.92\textwidth,colback={white},colbacktitle=white,coltitle=OliveGreen,colframe=green!75!black,boxrule=0.2mm]   
\cor{#1}
\end{tcolorbox}
}
\newcommand{\rc}[1]{\textcolor{OliveGreen}{#1}}
\newcommand{\pmc}[1]{\textcolor{blue}{\tt #1}}
\newcommand{\tmc}[1]{\textcolor{RawSienna}{\tt #1}}


%Référence aux exercices par leur numéro
\newcommand{\refexo}[1]{
\refstepcounter{numexo}
\addtocounter{numexo}{-1}
\label{#1}}

%Séparation entre deux activités
\newcommand{\separateur}{\begin{center}
\rule{1.5cm}{0.5pt}\raisebox{-3pt}{\ding{117}}\rule{1.5cm}{0.5pt}  \vspace{0.2cm}
\end{center}}

%Entête et pied de page
\newcommand{\snt}[1]{\lhead{\textbf{SNT -- La photographie numérique} \rhead{\textit{Lycée Nord}}}}
\newcommand{\Activites}[2]{\lhead{\textbf{{\sc #1}}}
\rhead{Activités -- \textbf{#2}}
\cfoot{}}
\newcommand{\Exos}[2]{\lhead{\textbf{Fiche d'exercices: {\sc #1}}}
\rhead{Niveau: \textbf{#2}}
\cfoot{}}
\newcommand{\Devoir}[2]{\lhead{\textbf{Devoir de mathématiques : {\sc #1}}}
\rhead{\textbf{#2}} \setlength{\fboxsep}{8pt}
\begin{center}
%Titre de la fiche
\fbox{\parbox[b][1cm][t]{0.3\textwidth}{Nom : \hfill \po{3} \par \vfill Prénom : \hfill \po{3}} } \hfill 
\fbox{\parbox[b][1cm][t]{0.6\textwidth}{Note : \po{1} / 20} }
\end{center} \cfoot{}}

%Devoir programmation en NSI (pas à rendre sur papier)
\newcommand{\PNSI}[2]{\lhead{\textbf{Devoir de {\sc nsi} : \textsf{ #1}}}
\rhead{\textbf{#2}} \setlength{\fboxsep}{8pt}
\begin{tcolorbox}[title=\textcolor{black}{\danger\; A lire attentivement},colbacktitle=lightgray]
{\begin{enumerate}
\item Rendre tous vous programmes en les envoyant par mail à l'adresse {\tt fnativel2@ac-reunion.fr}, en précisant bien dans le sujet vos noms et prénoms
\item Un programme qui fonctionne mal ou pas du tout peut rapporter des points
\item Les bonnes pratiques de programmation (clarté et lisiblité du code) rapportent des points
\end{enumerate}
}
\end{tcolorbox}
 \cfoot{}}


%Devoir de NSI
\newcommand{\DNSI}[2]{\lhead{\textbf{Devoir de {\sc nsi} : \textsf{ #1}}}
\rhead{\textbf{#2}} \setlength{\fboxsep}{8pt}
\begin{center}
%Titre de la fiche
\fbox{\parbox[b][1cm][t]{0.3\textwidth}{Nom : \hfill \po{3} \par \vfill Prénom : \hfill \po{3}} } \hfill 
\fbox{\parbox[b][1cm][t]{0.6\textwidth}{Note : \po{1} / 10} }
\end{center} \cfoot{}}

\newcommand{\DevoirNSI}[2]{\lhead{\textbf{Devoir de {\sc nsi} : {\sc #1}}}
\rhead{\textbf{#2}} \setlength{\fboxsep}{8pt}
\cfoot{}}

%La définition de la commande QCM pour auto-multiple-choice
%En premier argument le sujet du qcm, deuxième argument : la classe, 3e : la durée prévue et #4 : présence ou non de questions avec plusieurs bonnes réponses
\newcommand{\QCM}[4]{
{\large \textbf{\ding{52} QCM : #1}} -- Durée : \textbf{#3 min} \hfill {\large Note : \dots/10} 
\hrule \vspace{0.1cm}\namefield{}
Nom :  \textbf{\textbf{\nom{}}} \qquad \qquad Prénom :  \textbf{\prenom{}}  \hfill Classe: \textbf{#2}
\vspace{0.2cm}
\hrule  
\begin{itemize}[itemsep=0pt]
\item[-] \textit{Une bonne réponse vaut un point, une absence de réponse n'enlève pas de point. }
\item[\danger] \textit{Une mauvaise réponse enlève un point.}
\ifthenelse{#4=1}{\item[-] \textit{Les questions marquées du symbole \multiSymbole{} peuvent avoir plusieurs bonnes réponses possibles.}}{}
\end{itemize}
}
\newcommand{\DevoirC}[2]{
\renewcommand{\footrulewidth}{0.5pt}
\lhead{\textbf{Devoir de mathématiques : {\sc #1}}}
\rhead{\textbf{#2}} \setlength{\fboxsep}{8pt}
\fbox{\parbox[b][0.4cm][t]{0.955\textwidth}{Nom : \po{5} \hfill Prénom : \po{5} \hfill Classe: \textbf{1}\textsuperscript{$\dots$}} } 
\rfoot{\thepage} \cfoot{} \lfoot{Lycée Nord}}
\newcommand{\DevoirInfo}[2]{\lhead{\textbf{Evaluation : {\sc #1}}}
\rhead{\textbf{#2}} \setlength{\fboxsep}{8pt}
 \cfoot{}}
\newcommand{\DM}[2]{\lhead{\textbf{Devoir maison à rendre le #1}} \rhead{\textbf{#2}}}

%Macros permettant l'affichage des touches de la calculatrice
%Touches classiques : #1 = 0 fond blanc pour les nombres et #1= 1gris pour les opérations et entrer, second paramètre=contenu
%Si #2=1 touche arrondi avec fond gris
\newcommand{\TCalc}[2]{
\setlength{\fboxsep}{0.1pt}
\ifthenelse{#1=0}
{\psframebox[fillstyle=solid, fillcolor=white]{\parbox[c][0.25cm][c]{0.6cm}{\centering #2}}}
{\ifthenelse{#1=1}
{\psframebox[fillstyle=solid, fillcolor=lightgray]{\parbox[c][0.25cm][c]{0.6cm}{\centering #2}}}
{\psframebox[framearc=.5,fillstyle=solid, fillcolor=white]{\parbox[c][0.25cm][c]{0.6cm}{\centering #2}}}
}}
\newcommand{\Talpha}{\psdblframebox[fillstyle=solid, fillcolor=white]{\hspace{-0.05cm}\parbox[c][0.25cm][c]{0.65cm}{\centering \scriptsize{alpha}}} \;}
\newcommand{\Tsec}{\psdblframebox[fillstyle=solid, fillcolor=white]{\parbox[c][0.25cm][c]{0.6cm}{\centering \scriptsize 2nde}} \;}
\newcommand{\Tfx}{\psdblframebox[fillstyle=solid, fillcolor=white]{\parbox[c][0.25cm][c]{0.6cm}{\centering \scriptsize $f(x)$}} \;}
\newcommand{\Tvar}{\psframebox[framearc=.5,fillstyle=solid, fillcolor=white]{\hspace{-0.22cm} \parbox[c][0.25cm][c]{0.82cm}{$\scriptscriptstyle{X,T,\theta,n}$}}}
\newcommand{\Tgraphe}{\psdblframebox[fillstyle=solid, fillcolor=white]{\hspace{-0.08cm}\parbox[c][0.25cm][c]{0.68cm}{\centering \tiny{graphe}}} \;}
\newcommand{\Tfen}{\psdblframebox[fillstyle=solid, fillcolor=white]{\hspace{-0.08cm}\parbox[c][0.25cm][c]{0.68cm}{\centering \tiny{fenêtre}}} \;}
\newcommand{\Ttrace}{\psdblframebox[fillstyle=solid, fillcolor=white]{\parbox[c][0.25cm][c]{0.6cm}{\centering \scriptsize{trace}}} \;}

% Macroi pour l'affichage  d'un entier n dans  une base b
\newcommand{\base}[2]{ \overline{#1}^{#2}}
% Intervalle d'entiers
\newcommand{\intN}[2]{\llbracket #1; #2 \rrbracket}}

% Numéro et titre de chapitre
\setcounter{numchap}{2}
\newcommand{\Ctitle}{\cnum Programmation dynamique}

\makess{Les dictionnaires de Python}
\begin{frame}{\Ctitle}{\stitle}
	\begin{alertblock}{Les dictionnaires de Python}
		\begin{itemize}
			\item<1-> Les \textcolor{blue}{dictionnaires} de Python permettent de stocker des données sous forme de tableau associant une clé à une valeur : \vspace{0.2cm} \\
				\begin{tabularx}{0.8\textwidth}{l|Y|Y|Y|Y|Y|}
					\cline{2-6}
					\textcolor{blue}{Valeurs}                   & {\tt v1}                       & {\tt v2}                       & {\tt v3}                       & {\tt v4}                       & {\tt \dots}                    \\
					\cline{2-6}
					\multicolumn{1}{c}{$\uparrow$}              & \multicolumn{1}{c}{$\uparrow$} & \multicolumn{1}{c}{$\uparrow$} & \multicolumn{1}{c}{$\uparrow$} & \multicolumn{1}{c}{$\uparrow$} & \multicolumn{1}{c}{$\uparrow$} \\
					\cline{2-6}
					\multicolumn{1}{c|}{\textcolor{blue}{Clés}} & {\tt c1 }                      & {\tt c2}                       & {\tt c3}                       & {\tt c4}                       & {\tt \dots}                    \\
					\cline{2-6}
				\end{tabularx}
			\item<2-> Un dictionnaire se note entre accolades : \kw{\{} et \kw{\}}
			\item<3-> Les paires clés/valeurs sont séparés par des virgules \kw{,}
			\item<4-> Le caractère \kw{:} sépare une clé de la valeur associée.
		\end{itemize}
	\end{alertblock}
	\begin{exampleblock}{Exemples}
		\begin{itemize}
			\item<5-> Un dictionnaire contenant des objets et leurs prix :\\
				\onslide<6-> {\tt \footnotesize prix = \{ "verre":12 , "tasse" : 8, "assiette" : 16\} }
			\item<7-> Un dictionnaire traduisant des couleurs du français vers l'anglais \\
				\onslide<8-> {\tt \footnotesize couleurs = \{ "vert":"green" , "bleu" : "blue", "rouge" : "red" \} }
		\end{itemize}
	\end{exampleblock}
\end{frame}

% Opérations sur un dictionnaire
\begin{frame}{\Ctitle}{\stitle}
	\begin{alertblock}{Opérations sur un dictionnaire}
		\begin{itemize}
			\item<1-> On accède aux éléments d'un dictionnaire avec la syntaxe \textcolor{blue}{\tt nom\_dictionnaire[cle]}\\
				\onslide<2->\textcolor{gray}{\footnotesize {\tt prix = \{ "verre":12 , "tasse" : 8, "assiette" : 16, "plat" : 30 \} } \\
					Par exemple, {\tt prix["verre"]} contient 12}
			\item<3-> On peut ajouter une clé à un dictionnaire existant en effectuant une affectation \textcolor{blue}{\tt nom\_dictionnaire[nouvelle\_cle]=nouvelle\_valeur} \\
				\onslide<4->\textcolor{gray}{\footnotesize On ajoute un nouvel objet avec son prix : \\
				{\tt prix["couteau"]=20}
				}
			\item<5-> On peut modifier la valeur associée à une clé avec une affectation \textcolor{blue}{\tt nom\_dictionnaire[cle]=nouvelle\_valeur}\\
				\onslide<6->\textcolor{gray}{\footnotesize Le pris d'une tasse passe à 10 : \\
				{\tt prix["tasse"]=10}
				}
		\end{itemize}
	\end{alertblock}
\end{frame}

\begin{frame}{\Ctitle}{\stitle}
	\begin{block}{Présence dans un dictionnaire}
		\begin{itemize}
			\item<1-> Attention, essayer d'accéder à une clé qui n'est pas dans un dictionnaire renvoie une erreur !\\
				\onslide<2->\textcolor{gray}{\footnotesize Il n'y a pas de clé {\tt "fourchette"} dans le dictionnaire prix, donc \textcolor{blue}{\tt prix["fourchette"]} renvoie une erreur ({\tt \textcolor{red}{KeyError}}).}
			\item<3-> On teste la présence d'une clé dans un dictionnaire avec \textcolor{blue}{\tt cle in nom\_dictionnaire}\\
				\onslide<4->\textcolor{gray}{\footnotesize la fourchette n'est pas dans le dictionnaire prix \\
					Le test \textcolor{blue}{\tt fourchette in prix} renvoie \textcolor{blue}{\tt False}\\}
				\onslide<5->\textcolor{BrickRed}{\footnotesize \important} \textcolor{BrickRed}{Ce test d'appartenance s'effectue en temps constant (indépendant de la taille du dictionnaire)}
			\item<5-> On peut supprimer une clé existante dans un dictionnaire avec \textcolor{blue}{\tt del nom\_dictionnaire[cle]}\\
				\onslide<6->\textcolor{gray}{\footnotesize On supprimer le couteau : \\
					\textcolor{blue}{\tt del prix["couteau"]}
				}
		\end{itemize}
	\end{block}
\end{frame}


% Opérations sur un dictionnaire
\begin{frame}{\Ctitle}{\stitle}
	\begin{alertblock}{Parcours d'un dictionnaire}
		\begin{itemize}
			\item<1-> Le parcours par clé s'effectue directement avec \textcolor{blue}{\tt for cle in nom\_dictionnaire}\\
				\onslide<2->\textcolor{gray}{{\footnotesize \tt prix = \{ "verre":12 , "tasse" : 8, "assiette" : 16, "plat" : 30 \} } \\
				Par exemple, {\tt for objet in prix} permettra à la variable {\tt objet} de prendre successivement les valeurs des clés : {\tt "verre", "tasse", "assiette"} et {\tt "plat"}.}
			\item<3-> Le parcours par valeur s'effectue en ajoutant \textcolor{blue}{\tt .values()} au nom du dictionnaire : \textcolor{blue}{\tt for valeur in nom\_dictionnaire.values() \\}
				\onslide<4->\textcolor{gray}{\footnotesize
				Par exemple, {\tt for p in prix.values()} permettra à la variable {\tt p} de prendre successivement les valeurs du dictionnaire : {\tt 12, 8 , 16} et {\tt 30}.
				}
		\end{itemize}
	\end{alertblock}
\end{frame}

\begin{frame}{\Ctitle}{\stitle}
	\begin{exampleblock}{Exemple}
		On dispose d'une liste de nombres entiers et on veut obtenir le nombre d'occurence du (ou des) entiers(s) les plus fréquents dans cette liste. Par exemple si la liste est {\tt [1,7,1,3,4,1,3,4,3,1,5,108,2,3]} alors la réponse est 4, car les entiers les plus fréquents sont 1 et 3 qui apparaissent tous les deux à 4 reprises.
		\begin{enumerate}
			\item<1-> Proposer une solution qui pour chaque élément de la liste calcule son nombre d'apparitions à l'aide d'une fonction {\tt compte\_occurence}
			\item<2-> Proposer une solution utilisant un dictionnaire dont les clés sont les entiers présents dans la liste et les valeurs leurs nombre d'apparitions
			\item<3-> Commenter l'efficacité de ces deux solutions.
		\end{enumerate}
	\end{exampleblock}
\end{frame}

\begin{frame}{\Ctitle}{\stitle}
	\begin{exampleblock}{Correction question 1}
		\inputpython{/home/fenarius/Travail/Cours/cpge-info/docs/itc/files/C2/plus_frequent1.py}{}{\footnotesize}
	\end{exampleblock}
\end{frame}

\begin{frame}{\Ctitle}{\stitle}
	\begin{exampleblock}{Correction question 2}
		\inputpython{/home/fenarius/Travail/Cours/cpge-info/docs/itc/files/C2/plus_frequent2.py}{}{\footnotesize}
	\end{exampleblock}
\end{frame}

\begin{frame}{\Ctitle}{\stitle}
	\begin{exampleblock}{Correction question 2}
		\inputpython{/home/fenarius/Travail/Cours/cpge-info/docs/itc/files/C2/plus_frequent2.py}{}{\footnotesize}
	\end{exampleblock}
\end{frame}


\begin{frame}{\Ctitle}{\stitle}
	\begin{exampleblock}{Correction question 3}
		La solution avec les dictionnaires est bien plus efficace car on effectue un seul parcours de la liste et que le test d'appartenance au dictionnaire est une opération élémentaire (temps constant en moyenne).

	\end{exampleblock}
\end{frame}

\makess{Table de hachage}
\begin{frame}{\Ctitle}{\stitle}
	\begin{block}{Implémentation des dictionnaires}
		\begin{itemize}
			\item<1-> On crée un tableau $T$ de liste de longueur $N$ (donc indicé par les entiers $\intN{0}{N-1})$.
			\item<2-> Une \textcolor{blue}{fonction de hachage} $h$ transforme les clés en entier. Les clés doivent donc être \textcolor{blue}{non mutables} (ce qui exclu les listes). Ces entiers sont ramenés dans l'intervalle $\intN{0}{N-1}$ à l'aide d'un modulo.
			\item<3-> Chaque paire de clé/valeur $(c,v)$ est stockée dans le tableau $T$ à l'indice $h(c)$ (modulo $N$)
			\item<4-> Le cas où dont deux clés différentes $c1$ et $c2$ produisent le même indice s'appelle une \textcolor{blue}{collision}.
		\end{itemize}
	\end{block}
\end{frame}



\begin{frame}{\Ctitle}{\stitle}
	\begin{block}{Visualisation}
		{\tt \footnotesize prix = \{ "verre":12 , "tasse" : 8, "assiette" : 16, "bol" : 10\} } \\ \vspace{0.2cm}
		\begin{tabularx}{\textwidth}{X|c|X}
			\cline{2-2}
			{\rnode{verre}{\begin{cadre}{codebg}{blue}{2.2}{0.4}{\footnotesize "verre"}\end{cadre}}} & 0               &                                            \\
			\cline{2-2}
			                                                                                         & \rnode{i1}{1}   & \quad \quad \rnode{v1}{\tt [("Verre",12)]} \\
			\cline{2-2}
			                                                                                         & \rnode{i2}{2}   &                                            \\
			\cline{2-2}
			                                                                                         & \vdots          &                                            \\
			\cline{2-2}
			                                                                                         & \rnode{i42}{42} &                                            \\
			\cline{2-2}
			                                                                                         & \vdots          &                                            \\
			\cline{2-2}
			                                                                                         & $N$-1           &                                            \\
			\cline{2-2}
		\end{tabularx}
		\ncline[nodesepB=0.45,offsetA=0.1,offsetB=-0.05,linewidth=0.8pt,linecolor=brown]{->}{verre}{i1} \naput[nrot=:U,labelsep=0.05]{\textcolor{brown}{\footnotesize hash}}
		\ncline[nodesepA=0.3]{o->}{i1}{v1}
	\end{block}
\end{frame}

\begin{frame}{\Ctitle}{\stitle}
	\begin{block}{Visualisation}
		{\tt \footnotesize prix = \{ "verre":12 , "tasse" : 8, "assiette" : 16, "bol" : 10\} } \\ \vspace{0.2cm}
		\begin{tabularx}{\textwidth}{X|c|X}
			\cline{2-2}
			{\rnode{verre}{\begin{cadre}{codebg}{blue}{2.2}{0.4}{\footnotesize "verre"}\end{cadre}}}       & 0               &                                               \\
			\cline{2-2}
			                                                                                               & \rnode{i1}{1}   & \quad \quad \rnode{v1}{\tt [("Verre",12)]}    \\
			\cline{2-2}
			                                                                                               & \rnode{i2}{2}   & \quad \quad \rnode{v2}{\tt [("assiette",16)]} \\
			\cline{2-2}
			{\rnode{tasse}{\begin{cadre}{codebg}{blue}{2.2}{0.4}{\footnotesize "tasse"}\end{cadre}}}       & \vdots          &                                               \\
			\cline{2-2}
			                                                                                               & \rnode{i42}{42} & \quad \quad \rnode{v42}{\tt [("tasse",8)]}    \\
			\cline{2-2}
			{\rnode{assiette}{\begin{cadre}{codebg}{blue}{2.2}{0.4}{\footnotesize "assiette"}\end{cadre}}} & \vdots          &                                               \\
			\cline{2-2}
			                                                                                               & $N$-1           &                                               \\
			\cline{2-2}
		\end{tabularx}
		\ncline[nodesepB=0.45,offsetA=0.1,offsetB=-0.05,linewidth=0.8pt,linecolor=brown]{->}{verre}{i1} \naput[nrot=:U,labelsep=0.05]{\textcolor{brown}{\footnotesize hash}}
		\ncline[nodesepB=0.42,nodesepA=0.13,offsetA=-0.35,offsetB=0.15,linewidth=0.8pt,linecolor=brown]{->}{assiette}{i2}
		\ncline[nodesepB=0.42,offsetA=0.1,offsetB=-0.05,linewidth=0.8pt,linecolor=brown]{->}{tasse}{i42}
		\ncline[nodesepA=0.3]{o->}{i1}{v1}
		\ncline[nodesepA=0.3]{o->}{i2}{v2}
		\ncline[nodesepA=0.3]{o->}{i42}{v42}
	\end{block}
\end{frame}

\begin{frame}{\Ctitle}{\stitle}
	\begin{block}{Visualisation d'une collision}
		{\tt \footnotesize prix = \{ "verre":12 , "tasse" : 8, "assiette" : 16, "bol" : 10\} } \\ \vspace{0.2cm}
		\begin{tabularx}{\textwidth}{X|c|X}
			\cline{2-2}
			{\rnode{verre}{\begin{cadre}{codebg}{blue}{2.2}{0.4}{\footnotesize "verre"}\end{cadre}}}       & 0                                                                        &                                                       \\
			\cline{2-2}
			                                                                                               & \rnode{i1}{1}                                                            & \quad \quad \rnode{v1}{\tt [("Verre",12)]}            \\
			\cline{2-2}
			                                                                                               & \rnode{i2}{2}                                                            & \quad \quad \rnode{v2}{\tt [("assiette",16)]}         \\
			\cline{2-2}
			{\rnode{tasse}{\begin{cadre}{codebg}{blue}{2.2}{0.4}{\footnotesize "tasse"}\end{cadre}}}       & \vdots                                                                   &                                                       \\
			\cline{2-2}
			                                                                                               & \rnode[linecolor=BrickRed,linewidth=0.02]{i42}{\textcolor{BrickRed}{42}} & \quad \quad \rnode{v42}{\tt [("tasse",8),("bol",10)]} \\
			\cline{2-2}
			{\rnode{assiette}{\begin{cadre}{codebg}{blue}{2.2}{0.4}{\footnotesize "assiette"}\end{cadre}}} & \vdots                                                                   &                                                       \\
			\cline{2-2}
			{\rnode{bol}{\begin{cadre}{codebg}{blue}{2.2}{0.4}{\footnotesize "bol"}\end{cadre}}}           & $N$-1                                                                    &                                                       \\
			\cline{2-2}
		\end{tabularx}
		\ncline[nodesepB=0.45,offsetA=0.1,offsetB=-0.05,linewidth=0.8pt,linecolor=brown]{->}{verre}{i1} \naput[nrot=:U,labelsep=0.05]{\textcolor{brown}{\footnotesize hash}}
		\ncline[nodesepB=0.42,nodesepA=0.13,offsetA=-0.35,offsetB=0.15,linewidth=0.8pt,linecolor=brown]{->}{assiette}{i2}
		\ncline[nodesepB=0.42,offsetA=0.1,offsetB=-0.05,linewidth=0.8pt,linecolor=brown]{->}{tasse}{i42}
		\ncline[nodesepB=0.42,offsetA=-0.2,nodesepA=0.04,offsetB=-0.05,linewidth=0.8pt,linecolor=brown]{->}{bol}{i42}
		\ncline[nodesepA=0.3]{o->}{i1}{v1}
		\ncline[nodesepA=0.3]{o->}{i2}{v2}
		\ncline[nodesepA=0.3]{o->}{i42}{v42} \rput(7.1,1.65){\textcolor{BrickRed}{\scriptsize Collision}}
	\end{block}
\end{frame}


\begin{frame}{\Ctitle}{\stitle}
	\begin{block}{Visualisation d'une collision}
		{\tt \footnotesize prix = \{ "verre":12 , "tasse" : 8, "assiette" : 16, "bol" : 10\} } \\ \vspace{0.2cm}
		\begin{tabularx}{\textwidth}{X|c|X}
			\cline{2-2}
			{\rnode{verre}{\begin{cadre}{codebg}{blue}{2.2}{0.4}{\footnotesize "verre"}\end{cadre}}}       & 0                                                                        &                                                       \\
			\cline{2-2}
			                                                                                               & \rnode{i1}{1}                                                            & \quad \quad \rnode{v1}{\tt [("Verre",12)]}            \\
			\cline{2-2}
			                                                                                               & \rnode{i2}{2}                                                            & \quad \quad \rnode{v2}{\tt [("assiette",16)]}         \\
			\cline{2-2}
			{\rnode{tasse}{\begin{cadre}{codebg}{blue}{2.2}{0.4}{\footnotesize "tasse"}\end{cadre}}}       & \vdots                                                                   &                                                       \\
			\cline{2-2}
			                                                                                               & \rnode[linecolor=BrickRed,linewidth=0.02]{i42}{\textcolor{BrickRed}{42}} & \quad \quad \rnode{v42}{\tt [("tasse",8),("bol",10)]} \\
			\cline{2-2}
			{\rnode{assiette}{\begin{cadre}{codebg}{blue}{2.2}{0.4}{\footnotesize "assiette"}\end{cadre}}} & \vdots                                                                   &                                                       \\
			\cline{2-2}
			{\rnode{bol}{\begin{cadre}{codebg}{blue}{2.2}{0.4}{\footnotesize "bol"}\end{cadre}}}           & $N$-1                                                                    &                                                       \\
			\cline{2-2}
		\end{tabularx}
		\ncline[nodesepB=0.45,offsetA=0.1,offsetB=-0.05,linewidth=0.8pt,linecolor=brown]{->}{verre}{i1} \naput[nrot=:U,labelsep=0.05]{\textcolor{brown}{\footnotesize hash}}
		\ncline[nodesepB=0.42,nodesepA=0.13,offsetA=-0.35,offsetB=0.15,linewidth=0.8pt,linecolor=brown]{->}{assiette}{i2}
		\ncline[nodesepB=0.42,offsetA=0.1,offsetB=-0.05,linewidth=0.8pt,linecolor=brown]{->}{tasse}{i42}
		\ncline[nodesepB=0.42,offsetA=-0.2,nodesepA=0.04,offsetB=-0.05,linewidth=0.8pt,linecolor=brown]{->}{bol}{i42}
		\ncline[nodesepA=0.3]{o->}{i1}{v1}
		\ncline[nodesepA=0.3]{o->}{i2}{v2}
		\ncline[nodesepA=0.3]{o->}{i42}{v42} \rput(7.1,1.85){\textcolor{BrickRed}{\scriptsize Collision}}
		Pour rechercher si une clé est présente dans le dictionnaire il suffit de calculer son \textit{hash} et de regarder à l'indice correspondant dans le tableau.
	\end{block}
\end{frame}

\makess{Mémoïsation}
\begin{frame}{\Ctitle}{\stitle}
	\begin{exampleblock}{Exemple}
		\begin{enumerate}
			\item<1-> Ecrire une fonction récursive qui prend en argument un entier $n$ et renvoie le $n$ième terme de la suite de Fibonacci défini par :
				$\left\{ \begin{array}{lll}
						f_0   & = & 0,                                                  \\
						f_1   & = & 1,                                                  \\
						f_{n} & = & f_{n-1}+f_{n-2} \mathrm{\ \ pour\ tout\ \ } n\geq2.\end{array} \right.$
			\item<2-> Tracer le graphe des appels récursifs de cette fonction pour $n=5$
			\item<3-> Commenter
		\end{enumerate}
	\end{exampleblock}
\end{frame}

\begin{frame}{\Ctitle}{\stitle}
	\begin{exampleblock}{Correction question 1}
		\inputpython{/home/fenarius/Travail/Cours/cpge-info/docs/itc/files/C2/fibo_rec.py}{}{\footnotesize}
	\end{exampleblock}
\end{frame}

\begin{frame}{\Ctitle}{\stitle}
	\begin{exampleblock}{Correction questions 2}
		\begin{center}
			\psset{levelsep=1cm,treesep=0.2cm,linecolor=OliveGreen,linewidth=0.6pt}
			\pstree{\Toval{\tiny \tt fibo(5)}}{
				\pstree{\Toval{\tiny \tt fibo(4)}}{
					\pstree{\Toval{\tiny \tt  fibo(3)}}{
						\pstree{\Toval{\textcolor{BrickRed}{\tt \tiny fibo(2)}}}{\Toval{\tt \tiny fibo(1)} \Toval{\tiny \tt  fibo(0)}}
						\Toval{\tiny \tt  fibo(1)}}
					\pstree{\Toval{\textcolor{BrickRed}{\tt \tiny fibo(2)}}}{\Toval{\tiny  \tt fibo(1)} \Toval{\tiny \tt  fibo(0)}}
				}
				\pstree{\Toval{\tiny \tt fibo(3)}}{
					\pstree{\Toval{\textcolor{BrickRed}{\tt \tiny fibo(2)}}}{\Toval{\tiny \tt  fibo(1)} \Toval{\tiny \tt  fibo(0)}}
					\Toval{\tiny \tt  fibo(1)}}
			}
		\end{center}
	\end{exampleblock}
\end{frame}


\begin{frame}{\Ctitle}{\stitle}
	\begin{exampleblock}{Correction questions 3}
		\begin{center}
			\psset{levelsep=1cm,treesep=0.2cm,linecolor=OliveGreen,linewidth=0.6pt}
			\pstree{\Toval{\tiny \tt fibo(5)}}{
				\pstree{\Toval{\tiny \tt fibo(4)}}{
					\pstree{\Toval{\tiny \tt  fibo(3)}}{
						\pstree{\Toval{\textcolor{BrickRed}{\tt \tiny fibo(2)}}}{\Toval{\tt \tiny fibo(1)} \Toval{\tiny \tt  fibo(0)}}
						\Toval{\tiny \tt  fibo(1)}}
					\pstree{\Toval{\textcolor{BrickRed}{\tt \tiny fibo(2)}}}{\Toval{\tiny  \tt fibo(1)} \Toval{\tiny \tt  fibo(0)}}
				}
				\pstree{\Toval{\tiny \tt fibo(3)}}{
					\pstree{\Toval{\textcolor{BrickRed}{\tt \tiny fibo(2)}}}{\Toval{\tiny \tt  fibo(1)} \Toval{\tiny \tt  fibo(0)}}
					\Toval{\tiny \tt  fibo(1)}}
			}
		\end{center} \medskip
		{\small On calcule à plusieurs reprises les \textit{mêmes valeurs}, ici par exemple \textcolor{BrickRed}{\tt fibo(2)} est calculé à trois reprises.}
	\end{exampleblock}
\end{frame}

\begin{frame}{\Ctitle}{\stitle}
	\begin{alertblock}{Mémoïsation}
		\begin{itemize}
			\item<1-> La \textcolor{blue}{mémoïsation} consiste à stocker dans une structure de données les valeurs renvoyées par une fonction afin de ne pas les recalculer lors des appels identiques suivant.\\
			\item<2-> En Python, on utilise un dictionnaire dont les clés sont les arguments de la fonction et les valeurs les résultats de la fonction.
		\end{itemize}
	\end{alertblock}
	\begin{exampleblock}{Exemple}
		\onslide<3->{Par exemple, si on stocke dans un dictionnaire la valeur de {\tt fibo(2)} (clé : 2, valeur: 1), on n'a plus besoin de la recalculer lors des futurs appels.}
	\end{exampleblock}
\end{frame}

\begin{frame}[fragile]{\Ctitle}{\stitle}
	\begin{exampleblock}{Fibonnaci avec mémoïsation}
		\inputpython{/home/fenarius/Travail/Cours/cpge-info/docs/itc/files/C2/fibo_rec_memo.py}{}{\footnotesize}
	\end{exampleblock}
\end{frame}

\begin{frame}[fragile]{\Ctitle}{\stitle}
	\begin{block}{Remarque}
		En python, la mémoïsation peut-être effectuée de façon automatique à l'aide du décorateur \kw{lru\_cache} du module \kw{ funtools}. Après importation, on écrira simplement \kw{@lru\_cache} avant la définition de la fonction dont on veut mémoïser les appels.
	\end{block}
	\begin{exampleblock}{Fibonacci mémoïsation automatique}
		\inputpython{/home/fenarius/Travail/Cours/cpge-info/docs/itc/files/C2/fibo_rec_memo2.py}{}{\footnotesize}
	\end{exampleblock}
\end{frame}

\makess{Programmation dynamique : exemple introductif}
\begin{frame}{\Ctitle}{\stitle}
	\begin{exampleblock}{Position du problème}
		\onslide<1->{\small On considère une barre de métal de longueur entière 12 et pouvant être découpée en morceaux de longueurs entières ayant chacun prix comme indiqué ci-dessous :
			\begin{center}
				\begin{tabular}{|l|p{0.2cm}|p{0.2cm}|p{0.2cm}|p{0.2cm}|p{0.2cm}|p{0.2cm}|p{0.2cm}|p{0.2cm}|p{0.2cm}|p{0.2cm}|p{0.2cm}|p{0.2cm}|}
					\hline
					longueur & 1 & 2 & 3 & 4 & 5  & 6  & 7  & 8  & 9  & 10 & 11 & 12 \\
					\hline
					prix     & 2 & 4 & 7 & 8 & 12 & 14 & 18 & 23 & 24 & 25 & 26 & 31 \\
					\hline
				\end{tabular}
			\end{center}}
		\onslide<2->{\small Le prix de vente des différents morceaux varie donc suivant la découpe utilisée, par exemples :
			la découpe $(2, 4, 6)$ a un prix de vente de $4+8+14=26$, tandis que la découpe $(7, 5)$ a un prix de vente de $18+12=30$\\}
		\onslide<3->\textcolor{blue}{\small Le but du problème est de trouver la valeur maximale des découpes possibles.\\}
		\onslide<4->{\small On note $N$ la longueur de la barre, $(v_i)_{0\leq i \leq N}$, la valeur maximale de la découpe d'une barre de taille $i$ et $(p_i)_{0 \leq i \leq N}$ le prix d'un morceaux de longueur $i$.}
		\begin{enumerate}
			\item<5-> Donner les valeurs de $v_0$ et $v_1$.
			\item<6-> Etablir une relation de récurrence liant les $(v_i)_{0\leq i \leq N}$.
			\item<7-> En déduire une fonction Python récursive calculant la valeur de la découpe maximale.
			\item<8-> Utiliser la mémoïsation dans cette fonction.
		\end{enumerate}
	\end{exampleblock}
\end{frame}

\begin{frame}{\Ctitle}{\stitle}
	\begin{exampleblock}{Résolution}
		\begin{enumerate}
			\item<1-> \textcolor{OliveGreen}{$v_0=0$ et $v_1=1$}
			\item<2-> \textcolor{OliveGreen}{En supposant qu'on connaisse les valeurs maximales de découpe pour les tailles inférieurs $n$, la découpe maximale pour la taille $n$ s'en déduit en prenant le maximum parmi les découpes maximales d'une barre de longueur $k \leq n-1$  et d'un morceau de taille $n-k$, c'est à dire :
					$v_n = \max\left\{ v_k + p_{n-k},  0 \leq k \leq n-1\right\}$}
			\item<3-> \textcolor{OliveGreen}{Programme Python :}
				\inputpython{/home/fenarius/Travail/Cours/cpge-info/docs/itc/files/C2/decoupe.py}{}{\scriptsize}
				\onslide<4->\textcolor{OliveGreen}{Le programme affiche 32.}
		\end{enumerate}
	\end{exampleblock}
\end{frame}


\begin{frame}{\Ctitle}{\stitle}
	\begin{exampleblock}{Version avec mémoïsation}
		\inputpython{/home/fenarius/Travail/Cours/cpge-info/docs/itc/files/C2/decoupe_memo.py}{}{\scriptsize}
	\end{exampleblock}
\end{frame}

\begin{frame}{\Ctitle}{\stitle}
	\begin{exampleblock}{Calcul de bas en haut (\textit{bottom up})}
		\onslide<1->{La mémoïsation construit la solution de façon "descendante", on lance les appels récursif sur les plus grandes valeurs de taille de la barre. Une autre stratégie dite \textcolor{blue}{ascendante} ou \textcolor{blue}{de bas en haut (\textit{bottom up})} consiste à construire la solution en partant des instances les plus petites du problème.\\}
		\onslide<2->{Pour la découpe de la barre on part donc des valeurs connues $v_0$ et $v_1$ et on construit $v_2$ puis $v_3$, en utilisant la relation de récurrence $v_n = \max\left\{ v_k + p_{n-k},  0 \leq k \leq n-1\right\}$}\\
		\onslide<3->{Ce qui se traduit en Python par une solution \textcolor{blue}{itérative} :}
		\onslide<4->{\inputpython{/home/fenarius/Travail/Cours/cpge-info/docs/itc/files/C2/decoupe_iter.py}{}{\footnotesize}}
	\end{exampleblock}
\end{frame}

\begin{frame}{\Ctitle}{\stitle}
	\begin{exampleblock}{Construction d'une solution}
		On a pour le moment déterminé la valeur maximale de la découpe, mais pas la découpe elle-même. D'autre part, plusieurs découpes différentes peuvent avoir cette même valeur maximale. Pour rechercher \textit{une} découpe de valeur maximale, on peut par exemple :
		\begin{itemize}
			\item<1-> construire le tableau $(v_k)_{0\leq k \leq\N}$  et l'utiliser afin d'en déduire la découpe. \\
				\onslide<2->{\textcolor{gray}{\small Par exemple, si $v_{12} = v_8 + p_4$, cela signifie que pour avoir la valeur maximale de la découpe d'une barre de taille 12, une possibilité est d'utiliser une découpe  maximale d'une barre de taille 8 et un morceau de taille 4. En remontant ainsi de proche en proche, on obtient une découpe maximale possible}}
			\item<3-> Modifier notre fonction afin qu'elle renvoie la découpe maximale et non pas la valeur de cette découpe. \\
		\end{itemize}
		\onslide<4->{Ces deux possibilités seront abordées en TP.}
	\end{exampleblock}
\end{frame}


\makess{Programmation dynamique}
\begin{frame}{\Ctitle}{\stitle}
	\begin{alertblock}{Principes généraux}
		La programmation dynamique s'applique généralement à la résolution d'un problème d'optimisation vérifiant les conditions suivantes :
		\begin{enumerate}
			\item<2-> \textcolor{blue}{Sous structure optimale} : ce problème peut-être résolu à partir de problèmes similaires mais plus petits \\
				\onslide<3->\textcolor{gray}{\small La découpe maximale d'une barre de taille $N$ s'obtient comme découpe maximale  d'une barre de taille strictement inférieure $k$ et d'un morceau de taille $N-k$.}
				\item<4->\textcolor{blue}{Chevauchement de sous problème} : une solution récursive produit des appels identiques. Pour pallier ce problème, on utilise la mémoïsation dans les solutions récursives.\\
				\onslide<5->\textcolor{gray}{\small Pour rechercher la découpe maximale d'un barre de taille 5, on est amené à chercher celle d'une barre de taille 4,3,2,1. Et pour chercher celle d'une barre de taille 4, on fera de nouveau appel à celle d'une barre de taille 3,2,1 ...}
		\end{enumerate}
		\onslide<6->{\textcolor{BrickRed}{\small \important} L'étape cruciale est de déterminer les relations de récurrence entre les différentes instances du problème. Les différentes méthodes d'implémentation relèvent du choix du programmeur.}
	\end{alertblock}
\end{frame}

\makess{Rendu de monnaie}
\begin{frame}{\Ctitle}{\stitle}
	\begin{exampleblock}{Position du problème}
		On dispose d'un \textit{système monétaire} c'est à dire d'un ensemble de valeurs possibles pour les pièces et les billets. Le problème du rendu de monnaie consiste à déterminer le nombre minimal de pièces à utiliser pour former une somme donnée. \\
		\onslide<2->{Par exemple si le système monétaire est $\{ 1, 3, 4, 5, 10 \}$ et la somme 17,}
		\onslide<3->{alors on peut utiliser au minimum 3 pièces ($10+4+3$).}\\
		\onslide<4->{\textcolor{blue}{\small \rappel \; Rappel : }{\textcolor{gray}{ l'algorithme glouton qui consiste à rendre à tout moment la pièce de plus forte valeur possible ne fournit pas toujours la solution optimale. Ici, on obtiendrait $10, 5, 1, 1$ et donc 4 pièces.}}}
		\begin{enumerate}
			\item<5-> Ecrire une relation de récurrence entre les différentes instances du problème en donnant les solutions des cas de base.
			\item<6-> Ecrire un programme python permettant de répondre au problème.
		\end{enumerate}
	\end{exampleblock}
\end{frame}

\begin{frame}{\Ctitle}{\stitle}
	\begin{exampleblock}{Résolution}
		\begin{enumerate}
			\item<1-> \textcolor{OliveGreen}{On note $(p_i)_{0 \leq i \leq n}$ les valeurs des pièces rangées dans l'ordre croissant, $S$ la somme à rendre et $m(S)$ le nombre minimal de pièce pour rendre la somme~$S$. \\}
			\onslide<2->\textcolor{OliveGreen}{$\left\{ \begin{array}{lll}
				m(0) & = & 0,                                                  \\
				m(S) & = & min\left\{ 1 + m(S-pi),\ 0 < pi \leq S \right\} \mathrm{\ \ pour\ tout\ \ } S > 0.\end{array} \right.$}\medskip \\
			\onslide<3->\textcolor{BrickRed}{\small \danger \;}\textcolor{OliveGreen}{Si le problème n'a pas de solution, on se retrouve à chercher le minimum d'un ensemble vide, on traduira ce cas de figure dans Python en renvoyant la valeur \kw{inf} du module \kw{math}}
		\end{enumerate}
	\end{exampleblock}
\end{frame}

\begin{frame}{\Ctitle}{\stitle}
	\begin{exampleblock}{Résolution}
		\begin{enumerate}
			\addtocounter{enumi}{1}
			\item Programme Python (avec mémoïsation automatique)
			{\inputpython{/home/fenarius/Travail/Cours/cpge-info/docs/itc/files/C2/rendu_monnaie.py}{}{\footnotesize}}
		\end{enumerate}
	\end{exampleblock}
\end{frame}

\end{document}
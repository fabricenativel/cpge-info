\PassOptionsToPackage{dvipsnames,table}{xcolor}
\documentclass[10pt]{beamer}
\usepackage{Cours}

\begin{document}

\input{\detokenize{/home/fenarius/Travail/Cours/cpge-info/latex//MacrosCours.tex}}

% Numéro et titre de chapitre
\setcounter{numchap}{0}
\newcommand{\Ctitle}{\cnum Un peu de Python}

% Types de bases
\makess{Types de base}
\begin{frame}{\Ctitle}{\stitle}
	\begin{alertblock}{Types de base}
		\begin{tabularx}{\linewidth}{|l|c|>{\footnotesize}X|}
			\hline
			Type & Opérations & Commentaires \\
			\hline
			\kw{int} & \kw{+}, \kw{-}, \kw{*}, \kw{//}, \kw{\%} & Entiers signés ou non signés. Taille dynamique limitée par la mémoire\\
			\hline
			& & \  \newline \ \newline	\\
			\hline
			&&  \  \newline \  \\
			\hline
			 &  & \  \newline \  \\
			\hline
		\end{tabularx}
		\vspace{1cm}
	\end{alertblock}
\end{frame}

% Types de bases
\begin{frame}{\Ctitle}{\stitle}
	\begin{alertblock}{Types de base}
		\begin{tabularx}{\linewidth}{|l|c|>{\footnotesize}X|}
			\hline
			Type & Opérations & Commentaires \\
			\hline
			\kw{int} & \kw{+}, \kw{-}, \kw{*}, \kw{//}, \kw{\%} & Entiers signés ou non signés. Taille dynamique  limitée par la mémoire\\ 
			\hline
			\kw{float} & \kw{+}, \kw{-}, \kw{*}, \kw{/} & 	Représentation des nombres en virgule flottante (norme ieee754 : mantisse sur 53 bits, exposant sur 11 bits). Fonctions élémentaires dans \kw{math.h}\\
			\hline
			&&  \  \newline \  \\
			\hline
			 &  & \  \newline \  \\
			\hline
		\end{tabularx}
		\vspace{1cm}
	\end{alertblock}
\end{frame}

% Types de bases
\begin{frame}{\Ctitle}{\stitle}
	\begin{alertblock}{Types de base}
		\begin{tabularx}{\linewidth}{|l|c|>{\footnotesize}X|}
			\hline
			Type & Opérations & Commentaires \\
			\hline
			\kw{int} & \kw{+}, \kw{-}, \kw{*}, \kw{//}, \kw{\%} & Entiers signés ou non signés. Taille dynamique limitée par la mémoire\\ 
			\hline
			\kw{float} & \kw{+}, \kw{-}, \kw{*}, \kw{/} & 	Représentation des nombres en virgule flottante (norme ieee754 : mantisse sur 53 bits, exposant sur 11 bits). Fonctions élémentaires dans \kw{math}\\
			\hline
			\kw{bool} & \kw{or}  \kw{and}, \kw{not}, \textcolor{gray}{\tt all}, \textcolor{gray}{\tt any}  &  Evaluations paresseuses des expressions. \\
			\hline
			 &  & \  \newline \  \\
			\hline
		\end{tabularx}
		\vspace{1cm}
	\end{alertblock}
\end{frame}

% Types de bases
\begin{frame}{\Ctitle}{\stitle}
	\begin{alertblock}{Types de base}
		\begin{tabularx}{\linewidth}{|l|c|>{\footnotesize}X|}
			\hline
			Type & Opérations & Commentaires \\
			\hline
			\kw{int} & \kw{+}, \kw{-}, \kw{*}, \kw{//}, \kw{\%} & Entiers signés ou non signés. Taille dynamique limitée par la mémoire\\ 
			\hline
			\kw{float} & \kw{+}, \kw{-}, \kw{*}, \kw{/} & 	Représentation des nombres en virgule flottante (norme ieee754 : mantisse sur 53 bits, exposant sur 11 bits). Fonctions élémentaires dans \kw{math}\\
			\hline
			\kw{bool} & \kw{or}  \kw{and}, \kw{not}, \textcolor{gray}{\tt all}, \textcolor{gray}{\tt any}  &  Evaluations paresseuses des expressions. \\
			\hline
			\kw{str} &  \kw{ +, *} & Noté entre quotes (\kw{'}) ou guillemets (\kw{"}). Longueur avec \kw{len}\\
			\hline
		\end{tabularx}
		\vspace{1cm}
	\end{alertblock}
\end{frame}


% Fonction
\makess{Fonctions}
\begin{frame}[fragile]{\Ctitle}{\stitle}
	\begin{alertblock}{Définir une fonction en Python}
		Pour définir une fonction en Python :
		\begin{itemize}
		\item<2-> qui ne renvoie pas de valeur :\begin{codepython}
def <nom_fonction>(<arguments>):
	<instruction>
		\end{codepython}
		\item<2-> qui renvoie une valeur : \begin{codepython}
def <nom_fonction>(<arguments>):
	<instruction>
	return <resultat>
	\end{codepython}
\end{itemize}
	\end{alertblock}
\end{frame}

% Instructions conditionnelles
\makess{Instructions conditionnelles}
\begin{frame}[fragile]{\Ctitle}{\stitle}
	\begin{alertblock}{Instructions conditionnelles}
		\begin{itemize}
			\item<1-> Sans clause \kw{else}
			\begin{codepython}
if <condition>:
	<instructions>
			\end{codepython}
			Exécute les {\tt <instructions>} si la {\tt condition} est vérifiée.
			\item<2-> Avec clause \kw{else}
			\begin{codepython}
if <condition>:
	<instructions1>
else:
	<instructions2>
			\end{codepython}
		Cela permet d'exécuter les {\tt <instructions1>} si la {\tt condition} est vérifiée, sinon on exécute les {\tt <instructions2>}.
		\end{itemize}
	\end{alertblock}
\end{frame}

% boucle while
\makess{Boucles}
\begin{frame}[fragile]{\Ctitle}{\stitle}
	\begin{alertblock}{Boucles {\tt while}}
		\begin{itemize}
			\item<2-> La syntaxe d'une boucle \textcolor{red}{\tt while}  en Python est :
				\begin{codepython}
while <condition>:
	<instruction>
			\end{codepython}
			      Cela permet d'exécuter les {\tt <instructions>} tant que la {\tt <condition>} est  vérifiée.
			\item<3->  On ne sait pas a priori combien de fois cette boucle sera exécutée (et elle peut même être infinie), on dit que c'est une boucle \textcolor{blue}{non bornée}.
		\end{itemize}
	\end{alertblock}
\end{frame}

% boucle for
\begin{frame}[fragile]{\Ctitle}{\stitle}
	\begin{alertblock}{Boucles {\tt for}}
		\begin{itemize}
			\item<2-> Les instructions :
			      \begin{codepython}
	for <variable> in range(<entier>):
		 <instructions>
	\end{codepython}
			      créent une variable parcourant les entiers de 0 à {\tt <entier>} (exclu).
			\item<3-> Les {\tt <instructions>} indentées qui suivent seront exécutées pour chaque valeur prise par la variable.
			\item<4-> La boucle {\tt for} permet donc de répéter un nombre prédéfini de fois des instructions, on dit que c'est une boucle bornée.
		\end{itemize}
	\end{alertblock}
\end{frame}

% Exemples
\begin{frame}[fragile]{\Ctitle}{\stitle}
	\begin{exampleblock}{Exemples}
		\begin{itemize}
			\item<1-> Ecrire un programme Python permettant de calculer le {\sc pgcd} $d$ de deux entiers naturels $a$ et $b$ en utilisant l'algorithme d'Euclide.
			\item<2-> Ecrire un programme Python permettant de vérifier la conjecture de Collatz pour les entiers inférieurs à un entier $N$ donné. Le programme affichera aussi la valeur maximale atteinte durant les itérations. Par exemple pour $N=1000$, le programme affiche : \\
			{\tt Conjecture vérifiée, maximum atteint \numprint{250504}}. 
			\item<3-> Ecrire un programme Python permettant de simuler une marche aléatoire dans le dans le plan (déplacement aléatoire équibrobable dans les 4 directions cardinales). On pourra visualiser les déplacements grâce à la librairie \kw{turtle} et se déplacer jusqu'à sortir d'un cercle de rayon donné.
		\end{itemize}
	\end{exampleblock}
\end{frame}

% Exemples
\begin{frame}[fragile]{\Ctitle}{\stitle}
	\begin{exampleblock}{Correction exemple 1}
		\begin{itemize}
			\item<1-> Version itérative
		\inputpython{/home/fenarius/Travail/Cours/cpge-info/docs/itc/files/C0/pgcd.py}{}{\small}
			\item<2-> Version récursive
			\inputpython{/home/fenarius/Travail/Cours/cpge-info/docs/itc/files/C0/pgcd_rec.py}{}{\small}
		\end{itemize}
	\end{exampleblock}
\end{frame}


% Définition des listes
\makess{Les listes}
\begin{frame}[fragile]{\Ctitle}{\stitle}
	\begin{center}
		\begin{alertblock}{Les listes de Python}
			\begin{itemize}
				\item<1-> Les listes de Python sont des structures contenant  zéro, une ou plusieurs valeurs (pas forcément du mêmte type).
				\item<2-> Une liste se note entre crochets : \kw{[} et \kw{]}
				\item<3-> Les éléments sont séparés par des virgules
				\item<4-> Les éléments d'une liste sont repérés par leur position dans la liste, on dit leur \textcolor{blue}{indice}. Attention, la numérotation commence à zéro.
				\item<6-> On peut accéder à un élément en indiquant le nom de la liste puis  l'indice de cet élément entre crochet
				\item<7-> L'erreur {\tt IndexError} indique qu'on tente d'accéder à un indice qui n'existe pas.
				\item<8-> La longueur d'une liste (ie. son nombre d'éléments) s'obtient à l'aide de la fonction \kw{len}.
			\end{itemize}
		\end{alertblock}
	\end{center}
\end{frame}

% Manipulation des listes
\begin{frame}[fragile]{\Ctitle}{\stitle}
	\begin{alertblock}{Opérations sur les listes}
		Les opérations suivantes permettent de manipuler les listes (ajout, suppression, insertion d'éléments). On fera bien attention à la syntaxe on met le nom de la liste suivi d'un point suivi de l'opération à effectuer (voir exemples)
		\begin{itemize}
			\item<1-> \textcolor{blue}{\tt append} : permet d'ajouter un élément à la fin d'une liste. Par exemple : {\tt ma\_liste.append(elt)} va ajouter {\tt elt} à la fin de {\tt ma\_liste}.
			\item<2-> \textcolor{blue}{\tt pop} permet de récupérer un élement de la liste tout en le supprimant de la liste. Par exemple {\tt elt=ma\_liste.pop(2)} va mettre dans {\tt elt} {\tt ma\_liste[2]} et dans le même temps supprimer cet élément de la liste.
			\item<3-> \textcolor{blue}{\tt remove} permet de supprimer un élément d'une liste. Par exemple : {\tt ma\_liste.remove(elt)} va enlever {\tt elt} de {\tt ma\_liste}.
			\item<4-> \textcolor{blue}{\tt insert} permet d'insérer un élément à un indice donnée. Par exemple : {\tt ma\_liste.insert(indice,elt)} va insérer {\tt elt} dans {\tt ma\_liste} à l'index {\tt indice}.
		\end{itemize}
	\end{alertblock}
\end{frame}


% Génération de listes
\begin{frame}[fragile]{\Ctitle}{\stitle}
	\begin{alertblock}{Création de listes}
		On peut créer des listes de diverses façons en Python :
		\begin{itemize}
			\item<2-> \textcolor{red}{Par ajout succesif d'élement} on part alors d'une liste (éventuellement vide) et on ajoute chaque élément à l'aide d'instruction \textcolor{blue}{\tt append}.
			\item<3-> \textcolor{red}{Par répétition du même élément} on utilise alors le caractère \textcolor{blue}{\tt *} pour indiquer le nombre de répétitions. \\
			      \onslide<4-> {Par exemple : \textcolor{blue}{\tt hesitation = ["euh"]*4}}
			\item<6->	 \textcolor{red}{Par compréhension}, c'est à dire en indiquant la définition des éléments qui composent la liste. \\
			      \onslide<7-> {Par exemple la liste {\tt puissances2 = [1, 2, 4, 8, 16, 32, 64, 128]} est constitué des huits premières puissances de 2} \\
			      \onslide<8-> {Elle contient donc $2^0, 2^1, 2^2, \dots 2^7$, ce qui se traduit en Python par :}\\
			      \onslide<9-> \textcolor{blue}{\tt puissances2 = [2**k for k in range(8)]}
		\end{itemize}
	\end{alertblock}
\end{frame}


\begin{frame}[fragile]{\Ctitle}{\stitle}
	\begin{exampleblock}{Exemple}
		\begin{itemize}
			\item<1-> Crible d'Eratosthène
			\item<5-> Trie par insertion
		\end{itemize}
	\end{exampleblock}
\end{frame}


\end{document}

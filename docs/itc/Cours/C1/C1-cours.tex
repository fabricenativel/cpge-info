\PassOptionsToPackage{dvipsnames,table}{xcolor}
\documentclass[10pt]{beamer}
\usepackage{Cours}


\begin{document}

\input{\detokenize{/home/fenarius/Travail/Cours/cpge-info/latex//MacrosCours.tex}}

% Numéro et titre de chapitre
\setcounter{numchap}{1}
\newcommand{\Ctitle}{\cnum Introduction aux bases de données}

\makess{Généralités}
\begin{frame}{\Ctitle}{\stitle}
	\begin{alertblock}{Définitions}
		\begin{itemize}
			\item<1-> Une \textcolor{blue}{table} est un tableau à deux dimensions, les colonnes sont appelés \textcolor{blue}{attributs} et les lignes \textcolor{blue}{enregistrements}.
			\item<2-> Une \textcolor{blue}{base de données} est un ensemble de tables.
			\item<3-> Le \textcolor{blue}{domaine} d'un attribut est l'ensemble des valeurs que peut prendre cet attribut.
			\item<4-> Une \textcolor{blue}{clé primaire} est  un ensemble minimal d'attributs permettant d'identifier de façon unique chaque enregistrement.\\
				\onslide<5->\textcolor{gray}{\small On utilisera souvent des clés primaire ayant un seul attribut}
			\item<6-> \textcolor{blue}{\sc sql} \textit{(Structured Query Language)} est un langage de requête permettant d'interagir avec une base de données et d'y récupérer des informations.
		\end{itemize}
	\end{alertblock}
\end{frame}

\begin{frame}{\Ctitle}{\stitle}
	\begin{exampleblock}{Exemple}
		Extrait de la table {\tt Medals} de la base de données {\tt olympics1976-2008.db} :\\
		\begin{tabular}{|>{\tiny}c|>{\tiny}c|>{\tiny}c|>{\tiny}c|>{\tiny}c|>{\tiny}c|>{\tiny}c|>{\tiny}c|>{\tiny}c|>{\tiny}c|>{\tiny}c|>{\tiny}c|}
			\hline
			Id    & City        & Year & Sport     & Discipline & Event           & Athlete          & Gender & Country       & Medal  \\
			\hline
			286   & Montreal    & 1976 & Athletics & Athletics  & 110m hurdles    & DRUT, Guy        & Men    & France        & Gold   \\
			\hline
			194   & Montreal    & 1976 & Athletics & Athletics  & 100m            & BORZOV, Valery   & Men    & Soviet Union  & Bronze \\
			\hline
			13810 & Beijing     & 2008 & Athletics & Athletics  & decathlon       & CLAY, Bryan      & Men    & United States & Gold   \\
			\hline
			3455  & Los Angeles & 1984 & Fencing   & Fencing    & épée individual & BOISSE, Philippe & Men    & France        & Gold   \\
			\hline
		\end{tabular}
	\end{exampleblock}
\end{frame}

\makess{Langage SQL}
\begin{frame}{\Ctitle}{\stitle}
	\begin{exampleblock}{Exemple}
		Extrait de la table {\tt Medals} de la base de données {\tt olympics1976-2008.db} :\\
		\begin{tabular}{|>{\tiny}c|>{\tiny\columncolor{Apricot}}c|>{\tiny}c|>{\tiny}c|>{\tiny}c|>{\tiny}c|>{\tiny}c|>{\tiny}c|>{\tiny}c|>{\tiny}c|>{\tiny}c|>{\tiny}c|}
			\hline
			Id    & City        & Year & Sport     & Discipline & Event           & Athlete          & Gender & Country       & Medal  \\
			\hline
			286   & Montreal    & 1976 & Athletics & Athletics  & 110m hurdles    & DRUT, Guy        & Men    & France        & Gold   \\
			\hline
			194   & Montreal    & 1976 & Athletics & Athletics  & 100m            & BORZOV, Valery   & Men    & Soviet Union  & Bronze \\
			\hline
			13810 & Beijing     & 2008 & Athletics & Athletics  & decathlon       & CLAY, Bryan      & Men    & United States & Gold   \\
			\hline
			3455  & Los Angeles & 1984 & Fencing   & Fencing    & épée individual & BOISSE, Philippe & Men    & France        & Gold   \\
			\hline
		\end{tabular}
		\begin{itemize}
			\item<1->\textcolor{BrickRed}{{\tt City} est un attribut, son domaine est l'ensemble des noms de villes ayant accueilli les {\sc jo} sur la période 1976-2008.}
		\end{itemize}
	\end{exampleblock}
\end{frame}

\begin{frame}{\Ctitle}{\stitle}
	\begin{exampleblock}{Exemple}
		Extrait de la table {\tt Medals} de la base de données {\tt olympics1976-2008.db} :\\
		\begin{tabular}{|>{\tiny}c|>{\tiny}c|>{\tiny}c|>{\tiny}c|>{\tiny}c|>{\tiny}c|>{\tiny}c|>{\tiny}c|>{\tiny}c|>{\tiny}c|>{\tiny}c|>{\tiny}c|}
			\hline
			Id                     & City        & Year & Sport     & Discipline & Event           & Athlete          & Gender & Country       & Medal  \\
			\hline
			\rowcolor{Apricot} 286 & Montreal    & 1976 & Athletics & Athletics  & 110m hurdles    & DRUT, Guy        & Men    & France        & Gold   \\
			\hline
			194                    & Montreal    & 1976 & Athletics & Athletics  & 100m            & BORZOV, Valery   & Men    & Soviet Union  & Bronze \\
			\hline
			13810                  & Beijing     & 2008 & Athletics & Athletics  & decathlon       & CLAY, Bryan      & Men    & United States & Gold   \\
			\hline
			3455                   & Los Angeles & 1984 & Fencing   & Fencing    & épée individual & BOISSE, Philippe & Men    & France        & Gold   \\
			\hline
		\end{tabular}
		\begin{itemize}
			\item<1->{\tt City} est un attribut, son domaine est l'ensemble des noms de villes ayant accueilli les {sc jo} sur la période 1976-2008.
			\item<1->\textcolor{BrickRed}{Exemple d'enregistrement}
			\item
		\end{itemize}
	\end{exampleblock}
\end{frame}

\begin{frame}{\Ctitle}{\stitle}
	\begin{exampleblock}{Exemple}
		Extrait de la table {\tt Medals} de la base de données {\tt olympics1976-2008.db} :\\
		\begin{tabular}{|>{\tiny}c|>{\tiny}c|>{\tiny}c|>{\tiny}c|>{\tiny}c|>{\tiny}c|>{\tiny}c|>{\tiny}c|>{\tiny}c|>{\tiny}c|>{\tiny}c|>{\tiny}c|}
			\hline
			\cellcolor{Apricot}{Id} & City        & Year & Sport     & Discipline & Event           & Athlete          & Gender & Country       & Medal  \\
			\hline
			286                     & Montreal    & 1976 & Athletics & Athletics  & 110m hurdles    & DRUT, Guy        & Men    & France        & Gold   \\
			\hline
			194                     & Montreal    & 1976 & Athletics & Athletics  & 100m            & BORZOV, Valery   & Men    & Soviet Union  & Bronze \\
			\hline
			13810                   & Beijing     & 2008 & Athletics & Athletics  & decathlon       & CLAY, Bryan      & Men    & United States & Gold   \\
			\hline
			3455                    & Los Angeles & 1984 & Fencing   & Fencing    & épée individual & BOISSE, Philippe & Men    & France        & Gold   \\
			\hline
		\end{tabular}
		\begin{itemize}
			\item<1->{\tt City} est un attribut, son domaine est l'ensemble des noms de villes ayant accueilli les {\sc jo} sur la période 1976-2008.
			\item<1->{Exemple d'enregistrement}
			\item<1->\textcolor{BrickRed}{Id est un numéro unique pour chaque enregistrement, et peut donc servir de clé primaire.}
		\end{itemize}
	\end{exampleblock}
\end{frame}

\begin{frame}{\Ctitle}{\stitle}
	\begin{alertblock}{Schéma d'une table}
		Le schéma d'une table est la liste de ses attributs avec leur domaine. On souligne le (ou les) attributs formant la clé primaire.
	\end{alertblock}
	\begin{exampleblock}{Exemple}
		\onslide<2->{Le schéma relationnel de la table {\tt Medals} peut s'écrire :} \\
		\onslide<4->  \textbf{Medals} {\tt (\underline{id} : {\sc int},  {\tt City} : {\sc text}, {\tt Year} : {\sc INT}, {\tt Sport} : {\sc text},\dots)}
	\end{exampleblock}
\end{frame}


% Premiers pas en SQL
\begin{frame}{\Ctitle}{\stitle}
	\begin{alertblock}{Premiers pas en SQL}
		\begin{itemize}
			\item<1-> Pour récupérer la totalité des champs d'une table {\tt table} on utilise la syntaxe : \\
				\onslide<2->{\textcolor{blue}{\sc select} * \textcolor{blue}{\sc from} {\tt table}} \\
			\item<3-> Pour récupérer simplement les champs {\tt champ1, champ2,...} on utilise : \\
				\onslide<4->{\textcolor{blue}{\sc select} {\tt champ1, champ2,...}  \textcolor{blue}{\sc from} {\tt table}}
		\end{itemize}
	\end{alertblock}
	\begin{exampleblock}{Exemples}
		\begin{itemize}
			\item<5->{\mintinline[keywordcase=upper]{sql}{select City, Oyear FROM Medals} renvoie une table à deux colonnes : les villes olympiques et l'année.}\\
				\onslide<6->{\textcolor{BrickRed}{\small \important} Pour chaque enregistrement on affiche la ville et l'année donc on obtient des répétitions : \\
				\begin{tabular}{|>{\footnotesize}c|>{\footnotesize}c|}
					\hline
					City     & Oyear \\
					\hline
					Montreal & 1976  \\
					\hline
					Montreal & 1976  \\
					\hline
					\dots    & \dots \\
				\end{tabular}}
		\end{itemize}
	\end{exampleblock}
\end{frame}

% Premiers pas en SQL : clause WHERE
\begin{frame}[fragile]{\Ctitle}{\stitle}
	\begin{alertblock}{Clause {\sc where}}
		Une instruction \textcolor{blue}{\sc select} peut être suivie d'une clause \textcolor{blue}{\sc where} qui permet de rechercher les enregistrements correspondants à certains conditions. Ces conditions s'expriment à l'aide des opérateurs suivant :
		\begin{itemize}
			\item<2-> Comparaison : {\tt \textcolor{blue}{=}, \textcolor{blue}{<}, \textcolor{blue}{>}, \textcolor{blue}{<=}, \textcolor{blue}{>=},}  {\tt \textcolor{blue}{<>}} (différent)  et \textcolor{blue}{{\sc between}} (entre)
			\item<3-> Logique : \textcolor{blue}{\tt and}, \textcolor{blue}{\tt or} et \textcolor{blue}{\tt not}
				\item<4->\textcolor{gray}{Modèle de chaines de caractères : \textcolor{blue}{\sc like} où \textcolor{blue}{\tt \%} désigne n'importe quel suite de caractères et \textcolor{blue}{\tt \_} un unique caractère}
		\end{itemize}
	\end{alertblock}
	\begin{exampleblock}{Exemples}
		\onslide<5->{Pour chercher dans la table les champions olympiques français des {\sc JO} de 1980 \\}
		\onslide<6->\mintinline[keywordcase=upper,fontsize=\small,breaklines=true]{sql}{SELECT Athlete FROM Medals WHERE Country="France" and Medal="Gold" and Oyear="1980"}
	\end{exampleblock}
\end{frame}

% Premiers pas en SQL : Classement des résultats
\begin{frame}{\Ctitle}{\stitle}
	\begin{alertblock}{Clause {\sc order by}}
		Une instruction \textcolor{blue}{\sc select} peut être suivie d'une clause \textcolor{blue}{\sc order by} qui permet de classer les enregistrements selon un ou plusieurs champs. Cette clause est elle même suivie de :
		\begin{itemize}
			\item<2-> \textcolor{blue}{\sc asc} pour indiquer un classement par ordre croissant
			\item<3-> \textcolor{blue}{\sc desc} pour indiquer un classement par ordre décroissant
		\end{itemize}
		\onslide<4->{La valeur par défaut est {\sc asc}}
	\end{alertblock}
	\begin{exampleblock}{Exemples}
		\begin{itemize}
			\item<5-> Pour classer par ordre alphabétique les noms vainqueurs du 100 m aux JO :\\
				\onslide<6->\mintinline[keywordcase=upper,fontsize=\small,breaklines=true]{sql}{SELECT Athlete FROM Medals WHERE Event="100m" and Medal="Gold" ORDER BY Athlete}
		\end{itemize}
	\end{exampleblock}
\end{frame}

% Premiers pas en SQL : Clause distinct
\begin{frame}{\Ctitle}{\stitle}
	\begin{alertblock}{Clause {\sc distinct} et {\sc limit}}
		\begin{itemize}
			\item<2-> Une instruction \textcolor{blue}{\sc select} peut être \textit{directement} suivie d'une clause \textcolor{blue}{\sc distinct} {\tt champ} qui indique que {\tt champ} ne doit apparaître qu'une fois dans les résultats
			\item<3-> Une instruction \textcolor{blue}{\sc select} peut être suivie d'une clause \textcolor{blue}{\sc limit} qui indique le nombre maximal d'enregistrement à renvoyer. Cette clause est particulièrement utile en relation avec \textcolor{blue}{\sc order by}.
		\end{itemize}
	\end{alertblock}
	\begin{exampleblock}{Exemples}
		\begin{itemize}
			\item<5-> Pour afficher les villes olympiques  sans répétitions :\\
				\onslide<6->\mintinline[keywordcase=upper,fontsize=\small,breaklines=true]{sql}{SELECT DISTINCT City FROM Medals}
			\item<7-> Pour afficher les trois derniers champions olympiques du décathlon\\
				\onslide<8->\mintinline[keywordcase=upper,fontsize=\small,breaklines=true]{sql}{SELECT Athlete FROM Medals WHERE Event="decathlon" and Medal="Gold" ORDER BY OYear DESC LIMIT 3}
		\end{itemize}
	\end{exampleblock}
\end{frame}


\begin{frame}{\Ctitle}{\stitle}
	\begin{alertblock}{Renommage}
		\begin{itemize}
			\item<1-> On peut faire des calculs dans les requêtes
			\item<2-> On peut renommer une colonne avec \kw{AS}
			\item<3-> Cela est particulièrement utile pour y faire référence ensuite
		\end{itemize}
	\end{alertblock}
	\begin{exampleblock}{Exemples}
		\onslide<4-> On calcule l'ancienneté de chaque ville olympique et on les affiche par ordre croissant.
		\onslide<5->\mintinline[keywordcase=upper,fontsize=\small,breaklines=true]{sql}{SELECT DISTINCT City, 2023-Oyear AS Ancienneté FROM Medals ORDER BY Ancienneté DESC}
	\end{exampleblock}
\end{frame}

% Premiers pas en SQL : agrégation et GROUP BY
\begin{frame}{\Ctitle}{\stitle}
	\begin{alertblock}{Agrégation}
		Le langage {\sc sql} offre des opérateurs appelés \textcolor{blue}{fonction d'agrégation} permettant de calculer une valeur à partir d'un ensemble d'enregistrement :
		\begin{itemize}
			\item<2-> {\sc min} pour obtenir le minimum (d'un champ sur un ensemble d'enregistrement)
			\item<3-> {\sc max} pour obtenir le max
			\item<4-> {\sc sum} pour obtenir la somme
			\item<5-> {\sc avg} pour obtenir le minimum
			\item<6-> {\sc count} pour compter le nombre d'enregistrement
		\end{itemize}
	\end{alertblock}
	\begin{exampleblock}{Exemples}
		\begin{itemize}
			\item<7-> Pour compter le nombre de médailles de bronze Française en 2008 :\\
				\onslide<8->\mintinline[keywordcase=upper,fontsize=\small,breaklines=true]{sql}{SELECT COUNT(*) FROM Medals WHERE Medal="Bronze" and Country="France" and Oyear=2008}
		\end{itemize}
	\end{exampleblock}
\end{frame}

\begin{frame}{\Ctitle}{\stitle}
	\begin{alertblock}{Clause {\sc group by}}
		\begin{itemize}
			\item<1-> On peut regrouper les résultats pour un attribut donné à l'aide de \textcolor{blue}{\sc group by}.
			\item<2-> Un seul résultat sera affiché pour chaque valeur possible de l'attribut.
			\item<3-> Les fonctions d'agrégation dans le \textcolor{blue}{\sc select} s'appliquent alors à chaque groupe.
		\end{itemize}
	\end{alertblock}
	\begin{exampleblock}{Exemples}
		\onslide<4->Pour afficher le nombre total de médailles par pays\\
		\onslide<5->\mintinline[keywordcase=upper,fontsize=\small,breaklines=true]{sql}{SELECT Country, COUNT(*) AS total FROM Medals GROUP BY Country}
	\end{exampleblock}
\end{frame}

\begin{frame}{\Ctitle}{\stitle}
	\begin{alertblock}{Clause {\sc having}}
		\begin{itemize}
			\item<1-> Une clause {\sc group by} peut être complété par une clause {\sc having} qui indique une condition sur les groupes à afficher.
			\item<2-> \textcolor{BrickRed}{\small \important} Ne pas confondre :
			\begin{itemize}
				\item<3-> {\sc where} qui donne une condition sur les \textit{enregistrements} à afficher.
				\item<4-> {\sc having} qui est utilisé à la suite de {\sc group by} pour donner une condition sur les groupes à afficher.
			\end{itemize}
		\end{itemize}
	\end{alertblock}
	\begin{exampleblock}{Exemples}
		\onslide<5->{Pour afficher les pays ayant eu au total plus de 100 médailles\\}
		\onslide<6->\mintinline[keywordcase=upper,fontsize=\small,breaklines=true]{sql}{SELECT Country, COUNT(*) AS total FROM Medals GROUP BY Country HAVING total>100}
	\end{exampleblock}
\end{frame}


\end{document}

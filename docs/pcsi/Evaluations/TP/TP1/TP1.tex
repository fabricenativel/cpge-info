\PassOptionsToPackage{dvipsnames,table}{xcolor}
\documentclass[11pt,a4paper]{article}

\usepackage{Act}

\begin{document}
\input{\detokenize{/home/fenarius/Travail/Cours/cpge-info/latex/Macros.tex}}
\ModeExercice
\setboolean{corrige}{false}
\TPnote{1}{Programmation en Python}{10}
%Nom de la première activité
\setcounter{Exercise}{0}


\begin{Exercise}[title={Calculer une somme}]
	\Question{Ecrire en Python, une fonction {\tt somme} qui prend en entrée un entier {\tt n} et calcule la somme des entiers de 1 à {\tt n} qui se terminent par 3 et sont divisibles par 7. Par exemple, {\tt somme(200)}} doit renvoyer {\tt 196} car {\tt 63} et {\tt 133} sont les seul entier entre 1 et 200 se terminant par 3 et divisible par 7 et leur somme vaut {\tt 196}.
	\reponse{6}{2}
    \ifcorrige 
    \corpartC{somme.py}{}{}{1}{6}
    \fi
	\Question{Quelle est la valeur de {\tt somme(1000000)} ?}
	\reponse{0}{1}
    \tcor{On obtient la valeur {\tt 7142542855}}
\end{Exercise}

\begin{Exercise}[title={Chaine de caractères}]
	\Question{Ecrire en Python, une fonction {\tt occurrence} qui prend en entrée une chaine de caractère {\tt chaine} et un caractère {\tt c} et renvoie le nombre de fois où {\tt c} apparaît dans {\tt chaine}. Par exemples :
	\begin{itemize}
	\item {\tt occurrence("mercredi","e")} renvoie 2 puisqu'il  la lettre {\tt e} apparaît deux fois dans {\tt mercredi},
	\item {\tt occurrence("Python","e")} renvoie 0 car il n'y a pas de {\tt e} dans Python.
	\end{itemize}
	}
	\phantom{}\reponse{5}{3}
	\ifcorrige 
    \corpartC{fact.py}{}{}{7}{12}
    \fi
	\Question{Ecrire en Python, une fonction {\tt factorielle} qui prend en entrée un entier {\tt n} et renvoie {\tt n!} = {\tt n}$\times \dots \times 1$.} Par exemple {\tt factorielle(4)} renvoie $4 \times 3 \times 2 \times 1 = 24$.
	\reponse{5}{3}
	\ifcorrige 
    \corpartC{fact.py}{}{}{1}{5}
    \fi
	\Question{On rappelle qu'en Python, on peut convertir un entier en chaine de caractère avec {\tt str}, par exemple {\tt str(42)}} renvoie la chaine de caractères {\tt "42"}. Déterminer le nombre de 1 dans l'écriture décimale de factorielle de 100. 
	\reponse{0}{1}
	\tcor{On convertit {\tt factorielle(100)} en chaine de caractères avec {\tt str} puis on utilise la question 1 en cherchant le caractère {\tt "1"}, on obtient 15.}
\end{Exercise}

\end{document}
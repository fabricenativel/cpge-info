\PassOptionsToPackage{dvipsnames,table}{xcolor}
\documentclass[10pt]{beamer}
\usepackage{Cours}

\begin{document}

\input{\detokenize{/home/fenarius/Travail/Cours/cpge-info/latex/MacrosCours.tex}}

% Numéro et titre de chapitre
\setcounter{numchap}{9}
\newcommand{\Ctitle}{{Présentation de la \textcolor{yellow}{\textbf{MP2I}}}}
\newcommand{\SPATH}{/home/fenarius/Travail/Cours/cpge-info/docs/mp2i/files/C\thenumchap/}

% Définition d'une structure de données
\begin{frame}{\Ctitle}
	\setlength{\shadowsize}{1pt}
		\begin{tabularx}{0.9\textwidth}{Y}
			\rnode{lg}{\psshadowbox{\makebox[9cm]{
            \begin{tabularx}{\linewidth}{Y}
                \leavevmode\onslide<2->{\textcolor{BrickRed}{\textbf{Terminale générale}}}\\
                \leavevmode\onslide<3->{Maths + autres ES scientifiques}\\
                \leavevmode\onslide<4->{\textcolor{gray}{Maths expertes fortement recommandées}}
            \end{tabularx}
            }} } \vspace{0.5cm}           \\
			\\
			\rnode{cpge}{\psshadowbox{\makebox[9cm]{
                \begin{tabular}{l}
                \leavevmode\onslide<6->{\textcolor{BrickRed}{\textbf{1\textsuperscript{e} année CPGE scientifiques}} : {MPSI PTSI PCSI \textbf
                {MP2I}}} \\
                \leavevmode\onslide<7->{\textcolor{BrickRed}{\textbf{2\textsuperscript{e} année CPGE scientifiques}} : {MP PSI PC \textbf{MPI}}
                \end{tabular}
            }}}}  \vspace{0.9cm}    \\
            \rnode{ge}{\psshadowbox{\makebox[9cm]{
            \begin{tabularx}{\linewidth}{Y}
                \leavevmode\onslide<9->{\textcolor{BrickRed}{\textbf{Grandes Ecoles}}}\\
                \leavevmode\onslide<9->{X, ENS, Mines, Centrales \dots}\\
            \end{tabularx}
            }}} 
		\end{tabularx}
		\onslide<5->{\ncline[doubleline=true,doublesep=3pt,doublecolor=blue,linecolor=blue,linewidth=0.5pt,arrowsize=10pt,arrowinset=0.2,arrowlength=1.2]{->}{lg}{cpge} \naput{\textit{Parcoursup}}}
        \onslide<8->{\ncline[doubleline=true,doublesep=3pt,doublecolor=blue,linecolor=blue,linewidth=0.5pt,arrowsize=10pt,arrowinset=0.2,arrowlength=1.2]{->}{cpge}{ge} \naput{\textit{Concours}}}
\end{frame}

\begin{frame}{\Ctitle}
    \begin{center}
        \textcolor{BrickRed}{\textbf{\large Horaires}}
    \end{center}
    \begin{itemize}
        \item<1-> \textbf{Toute l'année}
        \begin{itemize}
            \item<2-> Maths : 12 h
            \item<3-> Physique : 6,5 h
            \item<4-> Français : 2h
            \item<5-> LV1 : 2h
            \item<6-> EPS : 2h
            \item<6-> LV2 (facultative) : 2h
        \end{itemize}
        \item<7-> \textbf{Premier semestre}
        \begin{itemize}
            \item<8-> Informatique : 4h
            \item<9-> SI : 2h
        \end{itemize}
        \item<10-> \textbf{Second semestre}
        \begin{itemize}
        \item<11-> Option 1 : Informatique  6h
        \item<12-> Option 2 : ITC  2h, SI 4h, Chimie 2h  
        \end{itemize}
    \end{itemize}
\end{frame}

\begin{frame}{\Ctitle}
    \begin{center}
        \textcolor{BrickRed}{\textbf{\large Programme d'informatique}}
    \end{center}
    \begin{itemize}
        \item<1-> Méthodes de programmation
        \item<2-> Récursivité et induction
        \item<3-> Algorithmique
        \item<4-> Gestion des ressources de la machine
        \item<5-> Logique
        \item<6-> Base de données
        \item<7-> Langage C
        \item<8-> Langage OCaml 
    \end{itemize}
\end{frame}


\end{document}
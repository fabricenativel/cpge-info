\documentclass[11pt,a4paper]{article}

\usepackage{Act}

\begin{document}
\input{\detokenize{/home/fenarius/Travail/Cours/cpge-info/latex/Macros.tex}}


\ModeExercice
\newcounter{numtd}
\setcounter{numtd}{3}
\newcommand{\SPATH}{/home/fenarius/Travail/Cours/cpge-info/docs/mp2i/files/C\thenumtd}
\TD{\thenumtd}{Pointeurs}

\setcounter{Exercise}{0}
\begin{Exercise}[title={Pointeurs}]\\
	On considère le programme suivant :
	\begin{langageC*}{fontsize=\small, tabsize=0}
		int a = 4;
		int b = 1;
		int c = 2;
		int* p;
		int* q;
		p = &a;
		q = &c;
		*p = *q + 1;
		p = q;
		q = &b;
		*p = *p - *q;
		*q = *q + 1;
		*p *= *q;
	\end{langageC*}
	Compléter le tableau suivant afin de donner l'état des variables au cours de l'exécution du programme :
	\begin{center}
		\def\arraystretch{1.2}
		\setlength\tabcolsep{0.5cm}
		\begin{tabular}{l|c|c|c|c|c}
			                    & a & b & c & p   & q \\
			\hline
			initialisation      & 4 & 1 & 2 & ?   & ? \\
			{\tt p = \&a;}      & 4 & 1 & 2 & \&a & ? \\
			{\tt q = \&c;}      &   &   &   &         \\
			{\tt *p = *q + 1;}  &   &   &   &         \\
			{\tt p = q;}        &   &   &   &         \\
			{\tt q = \&b;}      &   &   &   &         \\
			{\tt *p = *p - *q;} &   &   &   &         \\
			{\tt *q = *q + 1;}  &   &   &   &         \\
			{\tt *p *= *q;}     &   &   &   &         \\
		\end{tabular}
	\end{center}
\end{Exercise}

\begin{Exercise}[title={{\tt printf} et {\tt scanf}}]
	\Question{Ecrire l'instruction permettant d'afficher une variable {\tt n} de type entier avec {\tt printf}}
	\Question{Ecrire l'instruction permettant de saisir au clavier une variable {\tt n} de type entier avec {\tt scanf}}
	\Question{Expliquer la différence entre le mode de passage de {\tt n} dans ces deux fonctions}
\end{Exercise}

\begin{Exercise}[title={ Pointeurs}] \\
	On considère le programme suivant :
	\inputC{/home/fenarius/Travail/Cours/cpge-info/docs/mp2i/files/C3/pointeur.c}{\small}
	\Question{Ce programme est-il correct ?}
	\Question{Proposer une correction.}
\end{Exercise}
\begin{Exercise}[title={Incrémenter une variable}]\\
	La fonction suivante doit incrémenter la variable {\tt n} donnée en argument :
	\begin{langageC*}{fontsize=\small, tabsize=0}
		void incremente(int x) {
			    x = x + 1;
			}
	\end{langageC*}
	\Question{Commenter}
	\Question{Proposer une correction.}
\end{Exercise}

\begin{Exercise}[title={Fonction modifiant un paramètre}]\\
Ecrire en C une fonction {\tt inverse} qui prend en argument un pointeur vers un booléen, ne renvoie rien et inverse la valeur de ce booléen ({\tt true} devient {\tt false} et inversement).
\end{Exercise}

\begin{Exercise}[title={Renvoyer un tableau}]
	\inputpartC{ret_local.c}{}{\small}{3}{12}
	\Question{Lors de la compilation de la fonction ci-dessus, on obtient l'avertissement (\textit{warning}) suivant : \og{} \textit{function returns address of local variable}\fg{}. Expliquer cet avertissement.}
    \Question{Dans quelle partie de la mémoire est stockée le tableau {\tt tab\_entiers} défini à la ligne 6 ?}
	\Question{Remplacer la ligne 6 par une allocation sur le tas.}
\end{Exercise}



\begin{Exercise}[title = {Deux plus grandes valeurs}]\\
    On souhaite écrire une fonction en C qui prend en argument un tableau d'entiers (de taille $n \geqslant 2$) et renvoie les deux plus grandes valeurs de ce tableau.
    \Question{Proposer une solution utilisant un type structuré que l'on définira et donner alors la signature de la fonction.}
    \Question{Proposer une solution avec une fonction ne renvoyant rien mais modifiant deux paramètres passés par adresse. Donner la signature de la fonction dans ce cas.}
    \Question{Ecrire les deux implémentations.}
\end{Exercise}

\begin{Exercise}[title = {Passer un pointeur}] \\
	On considère le programme suivant :
	\inputpartC{\SPATH/ppointeur.c}{}{}{1}{16}
	Ce programme produit un \textit{warning} à la compilation : \textit{ parameter ‘p’ set but not used} et une erreur de segmentation à l'exécution.
	\Question{Expliquer ces deux résultats}
	\Question{Proposer une correction afin que {\tt init\_pointer} soit conforme à sa spécification.}
\end{Exercise}

\end{document}
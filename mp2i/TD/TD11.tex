\documentclass[11pt,a4paper]{article}

\usepackage{Act}

\begin{document}
\input{\detokenize{/home/fenarius/Travail/Cours/cpge-info/latex/Macros.tex}}
\ModeExercice
\TD{11}{Modèle relationnel, sql}

\setcounter{Exercise}{0}

\begin{Exercise}[title = {Une seule table}]
\Question{Un élève de {\sc cpge} souhaite modéliser son travail à faire sous forme d'une liste de tâches. Une tâche peut être de plusieurs types (exercice, {\sc dm}, révisions, \dots), a une description, un état (à faire, en cours ou terminée), une date de création, une date limite d'achèvement et est liée à une matière (maths, physique, \dots). Donner le schéma relationnel de cette table.} 
\Question{Ecrire les requêtes {\sc sql} permettant :}
\subQuestion{De lister les tâches en cours.}
\subQuestion{D'obtenir toutes les tâches liées à la matière informatique.}
\subQuestion{De lister toutes les tâches créees depuis au moins une semaine et pas encore commencées.}
\subQuestion{De lister les trois tâches pour lesquelles il reste le moins de temps.}
\subQuestion{D'obtenir la tâche dont la date d'achèvement est la plus éloignée dans le temps.}
\end{Exercise}

\begin{Exercise}[title = {Avec plusieurs tables}]
\Question{Un restaurant souhaite modéliser ses réservations. Proposer un schéma relationnel basé sur trois tables : les clients, les tables du restaurant et les réservations. Une réservation est faite pour un client, à une table et pour un jour et une heure donnée. Préciser les clés primaires et étrangères de votre schéma relationnel.}    
\Question{Ecrire les requêtes {\sc sql} permettant :}
\subQuestion{D'obtenir le nombre de client de ce restaurant.}
\subQuestion{D'obtenir les dates de toutes les réservations faires par un client spécifique.}
\subQuestion{D'obtenir la liste des tables non encore réservés à une date donnée}
\subQuestion{De lister les trois tables qui ont été le plus réservé.}
\subQuestion{De lister les cliens n'ayant jamais fait de réservation.}
\end{Exercise}

\begin{Exercise}[title = {Quelques requêtes {\sc sql}}]\\
On considère une base de données des pays et villes du monde, constituée de :
\begin{itemize}
    \item[-] La table \textbf{Pays} (\underline{codepays}, nom, continent, superficie, population, capitale, monnaie, pib)
    \item[-] La table \textbf{Villes} (\underline{codeville}, nom, population, \#codepays, longitude, latitude)
\end{itemize}
Le champ codepays de la table Ville fait référence à la clé primaire de la table Pays. Pour chaque question, écrire une requête {\sc sql} permettant d'obtenir le résultat :
\Question{La superficie et la population de la France.}
\Question{Le nombre de pays.}
\Question{La liste alphabétique des pays situés sur le continent américain.}
\Question{Les dix pays les plus peuplés au monde}
\Question{Le nom de la capitale du Togo.}
\Question{La liste des pays dont la capitale est située dans l'hemisphère nord.}
\Question{La liste des pays d'Europe avec leur capitale triée par superficie décroissante.}
\end{Exercise}

\begin{Exercise}[title = {Résultats de requêtes}]\\
On considère le schéma relationnel suivant pour une bibliothèque :
\begin{itemize}
    \item[-] \textbf{Livre} (\underline{isbn}, titre, \#id\_auteur, année, stock)
    \item[-] \textbf{Auteur} (\underline{id\_auteur}, nom, prenom)
    \item[-] \textbf{Client} (\underline{id\_client}, nom, prenom, adresse, email, telephone)
    \item[-] \textbf{Emprunt} (\underline{id\_emprunt}, \#id\_client, \#isbn, date, retour)
\end{itemize}
\Question{Expliquer ce que renvoie chacune des requêtes suivantes :}
\begin{enumerate}
\item[]
\begin{minted}{sql}
SELECT COUNT(*)
FROM Emprunt
WHERE Emprunt.retour IS NULL;
\end{minted}
\item[]
\begin{minted}{sql}
SELECT Livre.titre, Emprunt.date, Emprunt.retour
FROM Livre
JOIN Emprunt ON Livre.isbn = Emprunt.isbn
WHERE Emprunt.id_client = 42;
\end{minted}
\item[]
\begin{minted}{sql}
SELECT Auteur.nom, Auteur.prenom, COUNT(Livre.isbn)
FROM Auteur
JOIN Livre ON Auteur.id_auteur = Livre.id_auteur
GROUP BY Auteur.id_Auteur;    
\end{minted}
\end{enumerate}
\Question{Ecrire les requêtes permettant de : }
\subQuestion{Lister les livres dont le stock est inférieur ou égal à 1.}
\subQuestion{Lister les titres des livres de J. Verne par ordre alphabétique}
\subQuestion{Lister les noms et email des clients ayant des livres en retard à rendre.}
\subQuestion{Obtenir le nom de l'auteur ayant écrit le plus de livres.}
\end{Exercise}
\end{document}
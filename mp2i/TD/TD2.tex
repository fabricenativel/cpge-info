\documentclass[11pt,a4paper]{article}

\usepackage{Act}

\begin{document}
\input{\detokenize{/home/fenarius/Travail/Cours/cpge-info/latex/Macros.tex}}
\ModeExercice
\TD{2}{Validation et tests}

\begin{Exercise}[title={Spécifications}]\\
Proposer un nom, une spécification, des préconditions et un jeu de tests pour les fonctions suivantes :
\Question{\
\begin{langageC}
bool fonction1(int a, int b, int c)
	{return (a==b) || (b==c) || (a==c);}
\end{langageC}
}
\Question{\ 
\begin{langageC}
float fonction2(float x, float y)
	{return 1/(x*x+y*y);}
\end{langageC}
}
\Question{\
\begin{langageC}
int fonction3(float a, float b) {
	if (a<b)
	{return a;}
	else 
	{return b;}
	}
\end{langageC}
}
\end{Exercise}

\begin{Exercise}[title={Fonction mystère}]\\
On considère la fonction {\tt mystere} suivante :
	\begin{langageC}
bool mystere(int n) {
	int d=2;
	while (d*d<=n)
	{	if (n%d==0)
			{return false;}
		d=d+1;}
	return true;
}
\end{langageC}
\Question{Nommer cette fonction et en donner une spécification.}
\Question{Tracer son graphe de flot de contrôle.}
\Question{Proposer un jeu de tests permettant de couvrir tous les arcs.}
\end{Exercise}

\begin{Exercise}[title={nombre de jours dans un mois}]
	\Question{Ecrire une fonction {\tt nb\_jours} qui prend en argument un entier {\tt mois} et un entier {\tt annee} et qui renvoie le nombre de jours de ce mois. Par exemple, {\tt nb\_jours(5,1970)} doit renvoyer le nombres de jours du mois de mai 1970. On pourra utiliser sans la réécrire la fonction {\tt bissextile} vue en cours.}
	\Question{Proposer des préconditions pour cette fonction.}
	\Question{Proposer un jeu de tests pour cette fonction.}
\end{Exercise}

\begin{Exercise}[title={Triangles}]
\Question{Ecrire une fonction {\tt triangle} qui prend en argument trois entiers et renvoie :
\begin{itemize}
	\item 0 si les trois entiers ne sont pas les côtés d'un triangle
	\item 1 si les trois entiers sont les côtés d'un triangle scalène
	\item 2 si les trois entiers sont les côtés d'un triangle isocèle non rectangle
	\item 3 si les trois entiers sont les côtés d'un triangle équilatéral
	\item 4 si les trois entiers sont les côtés d'un triangle rectangle
\end{itemize}
\Question{Tracer le graphe de flot de contrôle de cette fonction.}
\Question{Proposer un jeu de tests pour cette fonction.}
}
\end{Exercise}

\end{document}
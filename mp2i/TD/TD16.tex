\documentclass[11pt,a4paper]{article}

\usepackage{Act}

\begin{document}
\input{\detokenize{/home/fenarius/Travail/Cours/cpge-info/latex/Macros.tex}}
\ModeExercice
\TD{16}{Logique}
\newcommand{\SPATH}{/home/fenarius/Travail/Cours/cpge-info/docs/mp2i/files/C14/}

\newcommand{\non}{\neg}
\newcommand{\et}{\wedge}
\newcommand{\ou}{\vee}
\newcommand{\imp}{\to}
\newcommand{\eq}{\leftrightarrow}
\newcommand{\lnode}[1]{\TCircle{$#1$}}
\psset{treesep=0.5cm,levelsep=0.8cm}

\setcounter{Exercise}{0}


\begin{Exercise}[title = {formules logiques}] \\
	Les expressions suivantes sont-elles des fes formules logique sur l'ensemble de propositions $V = \{p, q, r\}$ ? (on ne s'autorise pas dans cet exercice les simplfications d'écriture)
	\Question{ $(p) \ou \non q$}
	\Question{ $((p \et q) \ou (\top \et r))$}
	\Question{ $ ( p \ou r) \et (\non q))$}
	\Question{ $ (\non (p \ou q)) \et \non r)$}
	\Question{ $ ((s \et t) \ou (p \et q))$}
\end{Exercise}

\begin{Exercise}[title = {Représentation arborescente}]
	\Question{Représenter les arbres syntaxiques des formules logiques suivantes }
	\subQuestion{ $(p \ou q) \imp (r \et s)$}
	\subQuestion{ $ (\non p \ou \non q) \eq (r \imp s)$}
	\Question{Ecrire les formules formules logiques dont les arbres sont syntaxique sont}
	\subQuestion{
		\pstree{\lnode{\et}}{
			\pstree{\lnode{\ou}}{\lnode{p} \pstree{\lnode{\non}}{\lnode{q}}}
			\pstree{\lnode{\imp}}{ \pstree{\lnode{\non}}{\lnode{r}} \lnode{s} } }
	}
	\subQuestion{
		\pstree{\lnode{\eq}}{
			\pstree{\lnode{\et}}{
				\lnode{r}
				\pstree{\lnode{\non}}{\lnode{q}}
			}
			\pstree{\lnode{\ou}}{
				\pstree{\lnode{\imp}}{\lnode{p} \lnode{q}}
				\pstree{\lnode{\eq}}{\lnode{r} \lnode{s}}
			}
		}
	}
\end{Exercise}

\begin{Exercise}[title = {valeurs de vérité}]
	\Question{On considère la formule logique $ P = ((p \imp q) \ou (\non p \et q)) \et (p \ou q)$, déterminer la valeur de vérité de $P$ pour la valuation $\varphi(p) = F$ et $\varphi(q) = F$.}
	\Question{Dresser la table de vérité de $P$.}
\end{Exercise}

\begin{Exercise}[title = {tautologies}]\\
	Montrer que les formules logiques suivantes sont des tautologies :
	\Question{$(((x \imp y) \et x) \imp y)$}
	\Question{$((x \ou y) \eq  \neg(\neg x \et \neg y))$}
\end{Exercise}

\begin{Exercise}[title = {Où on trouve $\eq$, $\equiv$ et $\Leftrightarrow$}]\\
	Soient $F$ et $G$ deux formules logiques, montrer que $ F \equiv G$ si et seulement si $F \eq G$ est une tautologie.
\end{Exercise}

\begin{Exercise}[title = {connecteur de Sheffer}]\\
	On définit le connecteur de Sheffer (ou connecteur d'incompatibilité) par $x \uparrow y = \neg x \ou \neg y$.
	\Question{Dresser la table de vérité du connecteur de Sheffer}
	\Question{Donner une equivalence logique de ce connecteur utilisant $\neg$ et $\et$.}
	\Question{Vérifier que $\neg x \equiv x \uparrow x$}
	\Question{En déduire une équivalence logique de $x \et y$, $x \ou y$, $x \imp y$ qui n'utilise que le connecteur de Sheffer.}
	\Question{Démontrer par induction que tout formule logique peut s'écrire en utilisant uniquement le connecteur de Sheffer.}
	\Question{Donner une équivalence logique de $x \ou (\neg y \et z)$ utilisant uniquement le connecteur de Sheffer.}

\end{Exercise}

\begin{Exercise}[title = {qui prend un dessert ?}] \\
	Trois personnes $A$, $B$ et $C$ mangent ensemble. On sait que :
	\begin{itemize}
		\item Si $A$ prend un dessert alors $B$ aussi
		\item $B$ ou $C$ prennent un dessert mais pas les deux
		\item $A$ ou $C$ prend un dessert
		\item si $C$ prend un dessert alors $A$ aussi
	\end{itemize}
	Déterminer qui prend un dessert en utilisant une table de vérité.
\end{Exercise}

\begin{Exercise}[title={Ethnologie}, origin={\bac {\sc ccinp} 2017}]\\
	Imaginez-vous ethnologue. Vous étudiez une peuplade primitive qui présente un comportement manichéen extrême : lorsque plusieurs personnes participent à une même conversation sur un sujet donné, elles vont toutes avoir le même comportement manichéen tant que la conversation reste sur le même sujet, c’est-à-dire que toutes les affirmations seront soit des vérités, soit des mensonges. Par contre, si le sujet de la conversation change, la nature des affirmations, soit mensonge, soit vérité, peut changer,	mais toutes les affirmations seront de la même nature tant que le sujet ne changera pas à nouveau.	
    
    Pour être autorisé à séjourner dans cette peuplade, vous devez respecter cette règle. Vous participez à	une conversation avec trois de leurs membres que nous appellerons $X$, $Y$ et $Z$. Ceux-ci vous indiquent	comment rejoindre leur village. Si vous n’arrivez pas à le rejoindre, vous ne serez pas autorisé à y	séjourner.	Le premier sujet abordé est la région dans laquelle se trouve le village :
    \begin{itemize}
    \item[] $X$ indique : \og{}\textit{Le village se trouve dans la vallée}\fg{} ;	
    \item[] $Z$ réplique : \og{}\textit{Non, il ne s’y trouve pas}\fg{} ;	
    \item[] $X$ reprend : \og{}\textit{Ou alors dans les collines}\fg{}.	
    \end{itemize}
    Nous noterons $V$ et $C$ les variables propositionnelles associées à la région dans laquelle se trouve le	village.	Nous noterons $X_1$ et $Z_1$ les formules propositionnelles correspondant aux affirmations de $X$ et de $Z$	sur le premier sujet.	
    
    Puis, le second sujet est abordé : le chemin qui permet de rejoindre le village dans la région concernée. 
    \begin{itemize}
    \item[] $X$ dit : \og{}\textit{Le chemin de gauche conduit au village}\fg{} ;	
    \item[] $Z$ répond : \og{}\textit{Tu as raison} \fg{} ;
    \item[]	$X$ complète : \og{}\textit{Le chemin de droite y conduit aussi}\fg{} ;
    \item[]	$Y$ affirme : \og{}\textit{Si le chemin du milieu y conduit, alors celui de droite n’y conduit pas}\fg{} ; 
    \item[] $Z$ indique : \og{}\textit{Celui du milieu n’y conduit pas}\fg{}.
    \end{itemize}
     Nous noterons $G$, $M$, $D$ les variables propositionnelles correspondant respectivement au fait que le chemin de gauche, du milieu et de droite, conduit au village. Nous noterons $X_2$ , $Y_2$ et $Z_2$ les formules propositionnelles correspondant aux affirmations de $X$, de $Y$ et de $Z$ sur le second sujet.
     \Question{Représenter le comportement manichéen des interlocuteurs dans le premier sujet abordé sous la forme d'une formule du calcul des propositions dépendant des formules $X_1$ et $Z_1$.}
     \Question{Représenter les informations données par les participants sous la forme de deux formules du calcul des propositions $X_1$ et $Z_1$ dépendant des variables $V$ et $C$.}
     \Question{En utilisant la résolution avec les propriétés des opérateurs booléens et les formules de De Morgan en calcul des propostions, déterminer dans quelle région vous devez vous rendre pour rejoindre le village.}
     \Question{Représenter le comportement manichéen des interlocuteurs dans le second sujet abordé sous la forme d'une formule du calcul des propositions dépendant des formules $X_2$, $Y_2$ et $Z_2$.}
     \Question{Représenter les informations données par les participants sous la forme de trois formules du calcul des propositions $X_2, Y_2$ et $Z_2$ dépendant des variables $G$, $M$ et $D$.}
     \Question{En utilisant la résolution avec les tables de vérité en calcul des propositions, déterminer quel chemin vous devez suivre pour rejoindre le village.}
    \Question{En admettant que les trois participants aient menti, pouviez-vous prendre d'autres chemins ? Si oui, le ou lesquels ?}
\end{Exercise}

\begin{Exercise}[title = {footballeur}, origin={\bac {\sc ccinp} sujet zéro}]\\
	Un footballeur affirme à la presse :
	\begin{enumerate}
		\item Le jour où je marque un but, je suis content et je fais la fête.
		\item Le jour où mon équipe gagne, ou bien je suis content, ou bien je fais la fête ou les deux.
		\item Le jour où mon équipe perd, ou bien je ne suis pas content, ou bien j'ai marqué un but ou les deux
		\item Le jour où je ne marque pas et je fais la fête, je suis content
		\item Aujourd'hui, je ne suis pas content
	\end{enumerate}

	\Question{Définir les variables propositionnelles nécessaires à la modélisation de ce problème.}
	\Question{Modéliser chacune des assertions à l'aide de formules propositionnelles.}
	\Question{Mettre chacune de ces formules en forme normale conjonctive}
	\Question{On souhaite savoir si le joueur a marqué, si son équipe a gagné et s'il a fait la fête. Donner la formule $P$ permettant de répondre à ces questions. A quel problème classique est-on confronté ?}
	\Question{En appliquant l'algorithme de Quine, trouver une valuation qui rend $P$ vraie.}
\end{Exercise}


\end{document}
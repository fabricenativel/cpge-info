\documentclass[11pt,a4paper]{article}

\usepackage{Act}

\begin{document}
\input{\detokenize{/home/fenarius/Travail/Cours/cpge-info/latex/Macros.tex}}
\ModeExercice
\TD{0}{Systèmes}

%Nom de la première activité
\setcounter{Exercise}{0}
\begin{Exercise}[title={Arborescence}]\\
On donne ci-dessous l'arborescence du répertoire personnel de l'utilisateur {\tt John} : 
\begin{center}
	\pstree[levelsep=1.5cm]{\noeud{John}}{
						\pstree[levelsep=1.2cm]{\noeud{Public}}{\Tr{\tt img1.jpg} \Tr{\tt img2.jpg} \Tr{\tt img3.jpg}}
						\noeud{Téléchargements}
						\pstree[levelsep=1.2cm]{\noeud{Documents}}{\Tr{\tt secret.txt} \Tr{\tt .abc}}
						\noeud{Vidéos}
						\pstree[levelsep=1.2cm]{\noeud{Travail}}{\noeud{\footnotesize Cours} \noeud{\footnotesize TD} \noeud{\footnotesize TP}}}
\end{center}
\Question{En supposant que le répertoire personnel de John était initialement vide, donner une suite de commande ayant permis de construire cette arborescence de dossier.}
\Question{On suppose maintenant qu'on se trouve dans le répertoire {\tt Documents}. Donner l'effet des commandes suivantes (exécutées dans l'ordre)}
\subQuestion{\tt touch new.txt}
\subQuestion{\tt ls }
\subQuestion{\tt mkdir Todo}
\subQuestion{\tt cd Todo}
\subQuestion{\tt cp \~{}/Public/img1-cp.jpg .}
\subQuestion{\tt ln \~{}/Public/img2.jpg img1-ln.jpg}
\subQuestion{\tt ln -s \~{}/Public/img3.jpg img3-ls.jpg}
\Question{En supposant qu'on se retrouve dans le dossier {\tt Cours}, donner les chemins relatifs pour accéder aux dossiers suivants :}
\subQuestion{\tt Travail}
\subQuestion{\tt Documents}
\Question{On suppose maintenant que dans le dossier {\tt Cours}, se trouvent (entre autres) des fichiers portant l'extension {\tt .pdf}. Ecrire les commandes permettant de créer un dossier {\tt Sauvegarde} dans le répertoire personnel de John et d'y copier tous ces fichiers.}
\end{Exercise}

\begin{Exercise}[title={Droits sur les fichiers}]\\
On donne ci-dessous le résultat de la commande {\tt ls -l -i} dans un répertoire : 
\begin{minted}{shell}
9726850 -r-xr-xr-x 2 neo geek 5 juil.  9 10:16 fic1.txt
9726851 -rw-rw-r-- 1 neo geek 0 août  26  2021 fic2.txt
9726849 -rw-rw-r-- 1 neo geek 0 août  26  2021 fic3.txt
9726850 -r-xr-xr-x 2 neo geek 5 juil.  9 10:16 fic4.txt
9750174 lrwxrwxrwx 1 neo geek 8 juil. 10 17:32 fic5.txt -> fic3.txt
\end{minted}
\Question{Qui est le propriétaire de ces fichiers ? Dans quel groupe se trouve-t-il ?}
\Question{Quelle est la signification de {\tt r}, {\tt w} et {\tt x} ?}
\Question{Traduire en notation octale les droits sur chacun des fichiers}
\Question{Repérer les liens physiques et les liens symboliques}
\Question{Ecrire les commandes permettant de :}
\subQuestion{rajouter le droit d'écriture pour le propriétaire sur {\tt fic1.txt}}
\subQuestion{retirer le droit d'exécution à tous le monde sur {\tt fic4.txt}}
\subQuestion{changer les droits sur {\tt fic2.txt} en {\tt r-x\;r-{}-\;-{}-{}-}}
\subQuestion{changer les droits sur {\tt fic3.txt} en {\tt rwx\;r-{}x\;-{}-x}}
\end{Exercise}

\begin{Exercise}[title={Motifs glob}]\\
Pour chacun des motifs ci-dessous, donner une suite de caractères de longueur au moins 2 reconnue par le motif 
\Question{*pdf}
\Question{fic?.txt}
\Question{*.???}
\Question{doc202[12345].txt}
\end{Exercise}

\end{document}
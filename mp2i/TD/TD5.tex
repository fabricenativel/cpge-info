\documentclass[11pt,a4paper]{article}

\usepackage{Act}

\begin{document}
\input{\detokenize{/home/fenarius/Travail/Cours/cpge-info/latex/Macros.tex}}
\ModeExercice
\TD{5}{Récursivité}
\newcommand{\SPATH}{/home/fenarius/Travail/Cours/cpge-info/docs/mp2i/files/C5/}


\psset{treesep=0.5cm,levelsep=0.8cm}

\setcounter{Exercise}{0}

\begin{Exercise}[title = {Dessin récursif}] \\
	On souhaite écrire une fonction qui affiche à l'écran le dessin suivant :
\begin{verbatim}
    *
    **
    ***
    ****
\end{verbatim}
	Où le nombre de lignes affichées est le paramètre {\tt n} de la fonction.
	\Question{Ecrire en pseudo langage une version itérative de cette fonction.}
	\Question{Ecrire en pseudo langage une version récursive de cette fonction.}
\end{Exercise}

\begin{Exercise}[title={Exponentiation rapide version récursive}]
	\Question{Ecrire en pseudo langage l'algorithme d"exponentiation rapide récursif vu en cours.}
	\Question{Prouver que cet algorithme termine.}
	\Question{Prouver qu'il est totalement correct.}
\end{Exercise}

\begin{Exercise}[title = {Somme des éléments d'un tableau}]
	\Question{Ecrire un algorithme récursif permettant de calculer la somme des éléments d'un tableau.}
	\Question{Prouver que cet algorithme est totalement correcte.}
\end{Exercise}

\begin{Exercise}[title = {Calcul du \sc{pgcd}}]
	\Question{Ecrire une version récursive de l'algorithme d'Euclide de calcul du \sc{pgcd}.}
	\Question{Prouver que cet algorithme termine.}
	\Question{Prouver qu'il est totalement correcte.}
\end{Exercise}



\begin{Exercise}[title = {Palindrome}]
	\Question{Ecrire une version récursive d'un algorithme permettant de vérifier qu'une chaine de caractère est un palindrome.}
	\Question{Prouver que cet algorithme termine.}
	\Question{Prouver qu'il est totalement correcte.}
\end{Exercise}

\begin{Exercise}[title = {Coefficient du binôme}]
	\Question{Rappeler la relation de récurrence liant les coefficients du binôme $\binom{n}{k}$.}
	\Question{Donner les valeurs de $\binom{n}{0}$ et $\binom{n}{n}$.}
	\Question{En déduire un algorithme récursif permettant de calculer $\binom{n}{k}$.}
	\Question{Tracer l'arbre des appels récursifs pour le calcul de $\binom{4}{3}$. Que peut-on en déduire ?}
\end{Exercise}

\begin{Exercise}[title={Correction de l'exponentiation rapide itérative}]\\
	On rappelle l'algorithme d'exponentiation rapide en version itérative vu en cours :\\
	\begin{algorithm}[H]
		\DontPrintSemicolon
		\caption{Version itérative de l'exponentiation rapide}
		\Entree{$a \in \R^{*+}, n \in \N$}
		\Sortie{$a^n$}
		\everypar={\footnotesize \textcolor{gray}{\nl}}
		$p \leftarrow 1$\;
		\Tq{$n \neq 0$}{
			\Si{$n$ est impair}
			{$p \leftarrow p\times a$ \;}
			$a \leftarrow a*a$ \;
			$n \leftarrow \lfloor\frac{n}{2}\rfloor$ \;
		}
		\Return $p$
	\end{algorithm}
	\Question{Prouver que cet algorithme termine.}
	\Question{Prouver que cet algorithme est totalement correct.\\
	\aide \; On pourra prouver l'invariant suivant : $p \times a_0^{n_0} = a^n$ où $a_0$ (resp. $n_0$) désigne la valeur initiale de $a$ (resp $n$).}
\end{Exercise}


\end{document}
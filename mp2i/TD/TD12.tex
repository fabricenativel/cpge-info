\documentclass[11pt,a4paper]{article}

\usepackage{Act}

\begin{document}
\input{\detokenize{/home/fenarius/Travail/Cours/cpge-info/latex/Macros.tex}}
\ModeExercice
\TD{12}{Arbres binaires}
\newcommand{\SPATH}{/home/fenarius/Travail/Cours/cpge-info/docs/mp2i/files/C12/}
\newcommand{\CR}[1]{\TCircle[radius=0.25]{#1}}
\newcommand{\CN}[1]{\Tdia[fillstyle=solid,fillcolor=gray!20]{#1}}
\newcommand{\Tri}[1]{\Ttri[linestyle=dashed]{#1}}
\setcounter{Exercise}{0}
\psset{arrows=->,treesep=0.8cm,levelsep=0.8cm, radius=0.3cm}

\begin{Exercise}[title = {Représentation d'arbres binaires}]
    \Question{Dessiner tous les arbres binaires ayant 3 noeuds.}
    \Question{Dessiner tous les arbres binaires ayant 4 noeuds.}
    \Question{Dessiner un arbre binaire ayant 8 noeuds et de hauteur maximale (resp. minimale).}
\end{Exercise}

\begin{Exercise}[title = {Représentation en C}]\\
    On rappelle qu'on a défini en C, un arbre binaire (avec des étiquettes entières) par :
    \inputpartOCaml{\SPATH/arbres_binaires.c}{}{}{4}{11}
\Question{Rappeler la définition de la hauteur d'un arbre binaire et écrire une fonction de prototype \\
\mintinline{c}{int hauteur(ab arbrebinaire)}} qui renvoie la hauteur de l'arbre donné en argument.
\Question{On rappelle que dans cette implémentation, l'espace nécessaire au stockage des noeuds est alloué dynamiquement à l'aide d'instructions {\tt malloc}. Ecrire une fonction de prototype \\
\mintinline{c}{void libere(ab* arbrebinaire)} qui détruit l'arbre binaire donné en paramètre, en libérant l'espace alloué par ses noeuds. A la fin de l'appel {\tt ab} vaut {\tt NULL}.} 
\end{Exercise}


\begin{Exercise}[title = {Représentation en OCaml}]\\
    \label{abcaml}
    On rappelle qu'on a défini en OCaml un arbre binaire (avec des étiquettes entières) par :
    \inputpartOCaml{\SPATH/arbres_binaires_int.ml}{}{}{1}{3}
  \Question{Dessiner l'arbre représenté par :
  \inputpartOCaml{\SPATH/arbres_binaires_int.ml}{}{}{57}{65}
  }
  \Question{Donner sa taille et sa hauteur}
  \Question{S'agit-il d'un arbre binaire de recherche ? Justifier}
  \Question{Donner la représentation en OCaml de l'arbre :\\
  \pstree[arrows=->,treesep=0.8cm,levelsep=1cm]{\CR{$7$}}
					{
						\pstree{\CR{$2$}}{
							\Tn{}
							\CR{$5$}
						}
						\pstree{\CR{$7$}}{
							\CR{$1$}
							\Tn{}
						}
					}
  }
\end{Exercise}



\begin{Exercise}[title = {Un peu de dénombrement}]\\
    On note $T_n$ le nombre d'arbres binaires à $n$ noeuds.
    \Question{Donner $T_0$ et déterminer une relation de récurrence liant les $(T_k)_{0 \leqslant k \leqslant n}$ \\
    {\small \aide}\ Utiliser la définition par récurrence des arbres binaires.}
    \Question{Vérifier que $T_5 = 42$.}
    \\
    Le nombre de Catalan d'indice $n$ est défini par : $$ C_n = \dfrac{1}{n+1} \binom{2n}{n}$$ et on prouve que $T_n = C_n$.
    
\end{Exercise}

\begin{Exercise}[title = {Parcours d'un arbre binaire}] \vspace{0.2cm}\\ 
    \begin{tabularx}{\textwidth}{Y|Y|Y}
        \pstree{\TCircle{\tt 29}}
        {\pstree{\TCircle{\tt 24}}
        {\pstree{\TCircle{\tt 14}}
        { \Tn{} 
        \pstree{\TCircle{\tt 16}}
        { \Tn{} 
        \TCircle{\tt 20} 
        }}\TCircle{\tt 28} 
        }\pstree{\TCircle{\tt 31}}
        { \Tn{} 
        \TCircle{\tt 30} 
         
          }} &
          \pstree{\TCircle{22}}
{\pstree{\TCircle{20}}
{\pstree{\TCircle{14}}
{ \Tn{} 
\TCircle{18} 
}\TCircle{21} 
}\pstree{\TCircle{31}}
{ \Tn{} 
\pstree{\TCircle{24}}
{ 
\TCircle{27} 
\Tn{} 
} 
 \Tn{} }}
          & 
          \pstree{\TCircle{26}}
{\pstree{\TCircle{16}}
{\pstree{\TCircle{12}}
{ \Tn{} 
\TCircle{15} 
}\pstree{\TCircle{27}}
{\TCircle{17} 
\TCircle{24} 
}}\TCircle{30} 
}
          \\
          $T_1$ & $T_2$ & $T_3$ \\
    \end{tabularx}
\Question{Pour chacun des trois arbres binaires ci-dessus, donner l'ordre des noeuds lors d'un parcours prefixe, infixe et suffixe.}
\Question{Lequel de ces arbres binaires est un {\sc abr} ? Justifier}
\end{Exercise}

\begin{Exercise}[title = {Un peu de complexité}]\\
On considère la fonction OCaml suivante qui prend en argument un arbre binaire tel que défini par le type de l'exercice \ref{abcaml}
\inputpartOCaml{\SPATH/arbres_binaires_int.ml}{}{}{45}{47}
\Question{Ecrire une spécification et donner un nom plus approprié à la fonction {\tt mystère}.}
\Question{Rappeler la complexité de l'opérateur {\tt @} et en déduire celle de la fonction {\tt mystere}}
\Question{Proposer une version de cette fonction ayant une complexité linéaire en fonction du nombre de noeuds de l'arbre.\\
{\small \aide\;} Utiliser une fonction auxiliaire avec un accumulateur.
}
\end{Exercise}


\begin{Exercise}[title = {Reconstruction d'un arbre à partir de parcours}]

\Question{Est-il possible de reconstruire de façon unique un arbre binaire à partir de son parcours préfixe et de son parcours postfixe ?}
\Question{Quel est l'arbre binaire dont le parcours préfixe est {\tt 16; 6; 2; 8; 9; 12; 24; 19; 26; } et le parcours infixe {\tt 2; 6; 8; 9; 12; 16; 19; 24; 26;}}

\end{Exercise}


\begin{Exercise}[title = {Arbre binaire de recherche}]

\Question{Rappeler la caractérisation d'un arbre binaire de recherche par le parcours infixe}
\Question{En utilisant le parcours infixe, écrire une fonction {\tt est\_abr} de signature {\tt ab -> bool} qui indique si l'arbre donné en paramètre est un {\sc abr} ou non. Donner sa complexité, on supposera qu'on dispose déjà de la fonction qui renvoie le parcours infixe de l'arbre et que cette fonction a une complexité en $O(n)$.}
\Question{On propose la solution suivante pour déterminer si un arbre est un {\sc abr} :
\inputpartOCaml{\SPATH/est_abr.ml}{}{}{88}{92}
Selon vous, quel est le rôle des fonctions {\tt plus\_petit} et {\tt plus\_grand} ? Ecrire ces fonctions.
}
\Question{La solution précédente est-elle correcte ? Quelle est sa complexité ?}
\Question{Pour tester si un arbre est un {\sc abr}, on propose de parcourir l'arbre en donnant l'intervalle de valeurs dans lequel doit se trouver les éléments. Initialement l'intervalle est celui des entiers représentables (qu'on peut obtenir avec \mintinline{ocaml}{Int.min_int} et \mintinline{ocaml}{Int.max_int}), puis à chaque fois qu'on descend à gauche (resp. à droite) on met à jour la borne droite (resp gauche) de l'intervalle. Ecrire cette nouvelle méthode pour tester si un arbre est un {\sc abr}.}
\Question{Quelle est la complexité de cette fonction ?}

\end{Exercise}

\begin{Exercise}[title = {Opérations sur un {\sc abr}}]
    \Question{Rappeler l'algorithme d'insertion d'un élément dans un arbre binaire de recherche}
    \Question{Détailler l'insertion de la valeur 17 dans l'arbre ci-dessous (en ayant vérifié qu'il s'agit bien d'un {\sc abr})
    \pstree{\TCircle{\tt 29}}
        {\pstree{\TCircle{\tt 24}}
        {\pstree{\TCircle{\tt 14}}
        { \Tn{} 
        \pstree{\TCircle{\tt 16}}
        { \Tn{} 
        \TCircle{\tt 20} 
        }}\TCircle{\tt 28} 
        }\pstree{\TCircle{\tt 31}}
        { \Tn{} 
        \TCircle{\tt 35} 
         
          }}
    }
    \Question{Ecrire une fonction {\tt extraire\_min} en OCaml de signaure {\tt abr -> int * abr}, qui renvoie un couple composé du minimum d'un arbre binaire de recherche et de cet arbre privé du noeud de valeur minimale. On gère le cas de l'arbre vide avec {\tt failwith}}
    \Question{Afin de supprimer une valeur dans un {\sc abr}, on propose de procéder de la façon suivante : si l'arbre est vide la suppression est impossible, sinon on descend dans le sous arbre gauche (resp. droit) si la racine est strictement plus grande (resp. petite) que la valeur a supprimé. Si la racine est égale à la valeur à supprimer alors on la remplace par le minimum du sous arbre droit que l'on supprime. Détailler le fonctionnement de cette méthode sur l'arbre suivant d'où on veut supprimer la valeur 2 : \\
    \pstree{\TCircle{9}}
{\pstree{\TCircle{2}}
{\TCircle{0} 
\pstree{\TCircle{6}}
{\TCircle{5} 
\pstree{\TCircle{8}}
{ \Tn{} 
\TCircle{7} 
 
 \Tn{} }}}\pstree{\TCircle{12}}
{\TCircle{10} 
\TCircle{13} 
}} 
    }
    \Question{Donner une implémentation en OCaml sous la forme d'une fonction {\tt supprime} de signature {\tt abr -> int -> abr}}
\end{Exercise}

\begin{Exercise}[title = {Equilibrage d'un arbre rouge-noir}]

\Question{Pour insérer un noeud dans un arbre rouge-noir, on commence par utiliser l'algorithme d'insertion usuel dans un {\sc abr} et on attribut au nouveau noeud la couleur \textit{rouge}. Quel est alors le seul conflit possible ? (on appellera un tel conflit un \textit{conflit rouge-rouge}).}
\Question{Si le conflit rouge-rouge se situe à la racine, donner une méthode simple pour le résoudre.}
\Question{Si le conflit n'est pas situé à la racine, justifier qu'on se trouve dans l'un des quatres cas suivants où les noeuds rouges sont représentés dans un cercle et les noeuds noirs dans un losange grisé:}
\begin{center}
\begin{tabularx}{0.8\textwidth}{Y|Y}
    \pstree[arrows=->,treesep=1cm,levelsep=1cm]{\CN{$z$}}{
				\pstree{\CR{$y$}}{
					\pstree{\CR{$x$}}
					{
						\Tri{$t_1$}
						\Tri{$t_2$}
					}
                    \Tri{$t_3$}
				}
				\Tri{$t_4$}
			} 
    
    & 
    \pstree[arrows=->,treesep=1cm,levelsep=1cm]{\CN{$x$}}{
        \Tri{$t_1$}			
    \pstree{\CR{$y$}}{
                    \Tri{$t_2$}
					\pstree{\CR{$z$}}
					{
						\Tri{$t_3$}
						\Tri{$t_4$}
					}
				}
				
			} 
    \\
    \hline 
    & \\
    \pstree[arrows=->,treesep=1cm,levelsep=1cm]{\CN{$z$}}{
				\pstree{\CR{$x$}}{
					\Tri{$t_1$}
					\pstree{\CR{$y$}}
					{
						\Tri{$t_2$}
						\Tri{$t_3$}
					}
				}
				\Tri{$t_4$}
			} & 
            \pstree[arrows=->,treesep=1cm,levelsep=1cm]{\CN{$x$}}{
                \Tri{$t_1$}	
            \pstree{\CR{$z$}}{
                \pstree{\CR{$y$}}
                {
                    \Tri{$t_2$}
                    \Tri{$t_3$}
                }		
            \Tri{$t_4$}
					
				}
				
			}
            \\
   
\end{tabularx}
\end{center}
\Question{Montrer qu'en effectuant une ou plusieurs rotations, ces arbres se ramènent à }
\begin{center}
\pstree[arrows=->,treesep=1cm,levelsep=1cm]{\CN{$y$}}{
            \pstree{\CR{$x$}}{
                {
                    \Tri{$t_1$}
                    \Tri{$t_2$}
                }}	
                \pstree{\CR{$z$}}{
                    {
                        \Tri{$t_3$}
                        \Tri{$t_4$}
                    }		
					
				}
				
			}
\end{center}
\end{Exercise}

\begin{Exercise}[title = {Tri par tas}]
    \Question{Rappeler la définiton d'un tas binaire}
    \Question{Rappeler les relations liant l'indice d'un parent à ceux de ses fils (lorsqu'ils existent) lorsqu'on représente un tas binaire par un tableau}
    \Question{Rappeler le principe de l'insertion d'un nouvel élément dans un tas binaire}
    \Question{Rappeler le principe de la suppression de l'élément minimal d'un un tas binaire}
    \Question{Détailler les étapes et le fonctionnement de l'algorithme du tri par tas pour trier le tableau {\tt [7; 15; 3; 4; 19; 11]}}
    \Question{On rappelle ci-dessous la définition du type représentant un tas binaire en OCaml  : \\
    \mintinline{ocaml}{type 'a heap = {mutable size : int; data : 'a array};;}\\
    Ecrire une fonction {\tt insere} de signature {\tt 'a -> 'a heap -> unit} qui permet d'insérer une nouvelle valeur dans un tas binaire.}
\end{Exercise}

\end{document}
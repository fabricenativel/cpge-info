\PassOptionsToPackage{dvipsnames,table}{xcolor}
\documentclass[11pt,a4paper]{article}

\usepackage{Act}


\begin{document}
\input{\detokenize{/home/fenarius/Travail/Cours/cpge-info/latex/Macros.tex}}
\newcommand{\SPATH}{/home/fenarius/Travail/Cours/cpge-info/docs/mp2i/files/}

\ModeExercice

\lhead{{\sc mpi --} {\bf Préparation aux oraux}}
\rhead{2024}
\setboolean{corrige}{true}

\begin{center}
    \textbf{\Large Sujet B}
\end{center}


\setcounter{Exercise}{0}


\newcommand{\cn}[1]{\TCircle{\tt #1}}
\psset{arrows=->,treesep=1cm,levelsep=1cm}

\begin{Exercise*}[title = type A]\\
    On rappelle que le principe de la compression {\sc lzw} est d'attribuer un code aux préfixes rencontrés lors de la lecture du texte à compresser de façon à disposer d'un code compact si ce code se présente à nouveau. Initialement, seuls les codes de l'alphabet (usuellement les caractères {\sc ascii}) sont présents dans la table de codage. \\
    Ici, on s'intéresse à des chaines de caractères représentant des séquences génétique codés sur l'alphabet {\tt \{A, C, G, T\}}, par souci de simplicité on attribue initialement les codes {\tt A} $\rightarrow 0$, {\tt C} $\rightarrow 1$, {\tt G} $\rightarrow 2$  et {\tt T} $\rightarrow 3$. Et on souhaite compresser la séquence {\tt ATGTGATGTCCT}.
    Le début de l'algorithme va donc consister à attribuer un nouveau code au premier préfixe non encore codé qui apparaît lors de la lecture du texte. Et donc, ici, on attribue le code {\tt 4} au prefixe {\tt AT} et on emmettra le code de {\tt A} c'est à dire 0.
    \Question{Poursuivre le déroulement de cet algorithme en complétant le tableau suivant : \\
    \begin{tabular}{|l|l|l|}
        \hline
        Position dans le texte & Code émis & Nouveau préfixe ajouté \\
        \hline
        {\tt \underline{A}TGTGATGTCCT} & 0 & {\tt AT } $\rightarrow 4$ \\
    \end{tabular}\\
    }
    \tcor{
        \begin{tabular}{|l|l|l|}
        \hline
        Position dans le texte & Code émis & Nouveau préfixe ajouté \\
        \hline
        {\tt \underline{A}TGTGATGTCCT} & 0 & {\tt AT } $\rightarrow 4$ \\
        \hline
        {\tt A\underline{T}GTGATGTCCT} & 3 & {\tt TG } $\rightarrow 5$ \\
        \hline
        {\tt AT\underline{G}TGATGTCCT} & 2 & {\tt GT } $\rightarrow 6$ \\
         \hline
        {\tt ATG\underline{TG}ATGTCCT} & 5 & {\tt TGA } $\rightarrow 7$ \\
         \hline
        {\tt ATGTG\underline{AT}GTCCT} & 4 & {\tt ATT } $\rightarrow 8$ \\
        \hline
        {\tt ATGTGAT\underline{GT}CCT} & 6 & {\tt GTT } $\rightarrow 9$ \\
        \hline
        {\tt ATGTGATGT\underline{C}CT} & 1 & {\tt CC } $\rightarrow 10$ \\
        \hline
        {\tt ATGTGATGTC\underline{C}T} & 1 & {\tt CT } $\rightarrow 11$ \\
        \hline
        {\tt ATGTGATGTCC\underline{T}} & 3 &   \\
        \hline
    \end{tabular}\\
    On obtient donc la suite de code : {\tt [0; 3; 2; 5; 4; 6; 1; 1; 3]}}
    \Question{Quel est le taux de compression obtenu en supposant qu'un octet permet de représenter chaque code ?}
    \tcor{12 sur 9}
    \Question{Décompresser le texte $T$ codé par suite de codes {\tt [3; 1; 4; 6; 5; 2; 0; 2 ]} sur ce même alphabet en expliquant comment est reconstruit le dictionnaire de décompression.}
    \tcor{$T = $ {\tt TCTCTCTCTGAG}}
    \Question{Rappeler rapidement le principe  de l'algorithme de compression de Huffman et compresser le texte $T$ à l'aide de cet algorithme.}
\end{Exercise*}


\begin{Exercise*}[title = {type B}]
    \Question{Rappeler la défintion d'un tas binaire (min)}
	\Question{On suppose qu'un tas est représenté par un tableau $t = (t_0,\dots t_{n-1})$. Lorsqu'ils existent quels sont les indices des fils de $t_i$ ?}
	\Question{Quel est l'indice (lorsqu'il existe) du père de $t_i$ ?}
	\Question{Vérifier que l'arbre binaire suivant possède bien la structure de tas et donner sa représentation sous forme de tableau: \label{tas}
		\begin{center}
			\pstree[arrows=->,treesep=1cm,levelsep=1cm]{\TCircle{3}}
			{\pstree{\TCircle{6}}
				{\pstree{\TCircle{10}}
					{\TCircle{17} \TCircle{11}}
					\pstree{\TCircle{9}}
					{\TCircle{15}
						\TCircle{18}
					}}\pstree{\TCircle{5}}
				{\TCircle{7}
					\TCircle{13}
				}}
		\end{center}
	}
    \Question{Rappeler le principe d'insertion d'un nouvel élément dans un tas binaire puis détailler les étapes de l'insertion de 4 dans le tas représenté à la question \ref{tas}.}
    \Question{Rappeler le principe d'extraction du minimum d'un tas binaire, puis détailler les étapes de l'extraction du minimum du tas représenté à la question \ref{tas}.}
    \Question{Dans la suite, on utilisera la structure de données suivante permettant de représenter un tas en OCaml :
        \inputpartOCaml{tas.ml}{}{}{3}{3}
    Quel est le rôle du champ {\tt size} ? Pourquoi est-il déclaré en {\tt mutable} ? 
    }
    \Question{Ecrire une fonction qui renvoie un tas de taille maximale {\tt 1000} et qui contient des valeurs entières.}
    \Question{Ecrire la fonction d'insertion d'un élément dans un tas binaire contenant des entiers. On renvoie {\tt true} si l'insertion est possible (c'est-à-dire que le tas n'est pas plein) et {\tt false} sinon. La signature de cette fonction est donc {\tt insertion: 'a -> 'a heap -> bool}}
\end{Exercise*}
\end{document}
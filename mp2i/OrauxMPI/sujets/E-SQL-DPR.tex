\PassOptionsToPackage{dvipsnames,table}{xcolor}
\documentclass[11pt,a4paper]{article}

\usepackage{Act}


\begin{document}
\input{\detokenize{/home/fenarius/Travail/Cours/cpge-info/latex/Macros.tex}}
\newcommand{\SPATH}{/home/fenarius/Travail/Cours/cpge-info/docs/mp2i/files/}

\ModeExercice

\lhead{{\sc mpi --} {\bf Préparation aux oraux}}
\rhead{2024}
\setboolean{corrige}{false}
\begin{center}
	\textbf{\Large Sujet E}
\end{center}


\setcounter{Exercise}{0}

\newcommand{\cn}[1]{\TCircle{\tt #1}}
\psset{arrows=->,treesep=1cm,levelsep=1cm}

\begin{Exercise*}[title = type A]\\
	On s'intéresse dans cette partie à un site Internet d'échange de supports de cours entre enseignants  de {\sc mp2i/mpi}. Chaque personne désirant proposer ou récupérer du contenu doit commencer par se créer un compte sur ce site et peut ensuite accéder à du contenu ou en proposer.

	Ce site repose sur une base de données contenant en particulier une table, nommée {\tt ressources}. Elle possède un enregistrement par document téléversé sur le site. Ses attributs sont :
	\begin{itemize}
		\item {\tt id}, un identifiant numérique, unique pour chaque ressource ;
		\item {\tt owner}, le pseudo de la personne ayant créé la ressource ;
		\item {\tt annee}, l'année de publication de la ressource ;
		\item {\tt titre}, une chaine de caractères décrivant la ressource ;
		\item {\tt type}, chaine de caractères pouvant être cours, ds, tp ou td.
	\end{itemize}

	Voici un extrait de cette table :

	\medskip

	\begin{center}
		\begin{tabular}{|l|l|c|c|r|}
			\hline
			{\tt id} & {\tt owner} & {\tt annee} & {\tt titre}               & {\tt type} \\
			\hline
			4        & dknuth      & 2020        & Machine à décalage        & cours      \\
			13       & alovelace   & 2022        & Intelligence artificielle & td         \\
			$\cdots$ & $\cdots$    & $\cdots$    & $\cdots$                  & $\cdots$   \\
			\hline
		\end{tabular}
	\end{center}


	\Question{Écrire une requête SQL permettant de connaitre tous les titres des ressources déposées par \og{}{\tt jclarke}\fg{} classées par année de publication croissante.}
	\ifcorrige
		\begin{minted}{sql}
SELECT titre 
FROM ressources 
WHERE owner="jclarke"
ORDER BY annee ASC;
\end{minted}
	\fi
	\Question{Écrire une requête SQL permettant de connaitre le nombre total de ressources de type cours présentes sur le site.}
	\ifcorrige
		\begin{minted}{sql}
SELECT COUNT(*)
FROM ressources
WHERE type = "cours";
\end{minted}
	\fi
	\Question{Que fait la requête suivante : ?
		\begin{minted}{sql}
SELECT R.owner
FROM Ressources AS R
WHERE R.type = 'td'
GROUP BY R.owner
ORDER BY COUNT(*) DESC
LIMIT 3
\end{minted}
	}
	\tcor{Cette requête permet d'afficher les trois premiers propriétaires de ressources ayant posté le plus de ressources de type {\tt td}}

	\NRet
	Cette base de données contient également une table {\tt utilisateurs} qui contient les informations sur les utilisateurs du site. Elle possède un enregistrement par utilisateur. Ses attributs sont :
	\begin{itemize}
		\item {\tt nom}, le nom de l'utilisateur (clé primaire) ;
		\item {\tt mdp}, le mot de passe de l'utilisateur.
		\item {\tt email}, l'adresse email de l'utilisateur.
		\item {\tt naissance}, l'année de naissance de l'utilisateur.
	\end{itemize}
	Voici un extrait de cette table :
	\medskip
	\begin{center}
		\begin{tabular}{|l|l|l|l|}
			\hline
			{\tt nom} & {\tt mdp} & {\tt email}        & {\tt naissance } \\
			\hline
			dknuth    & chepas123 & dknuth@bigboss.com & 1938             \\
			\hline
			\dots     & \dots     & \dots              & \dots            \\
		\end{tabular}
	\end{center}
	L'attribut {\tt owner} de la table {\tt ressources} est une clé étrangère qui référence l'attribut {\tt nom} de la table {\tt utilisateurs}.
	\Ret
	\Question{Ecrire une requête SQL permettant de lister toutes les ressources déposées par des utilisateurs nés après 2000.}
	\ifcorrige
		\begin{minted}{sql}
SELECT * FROM ressources
JOIN utilisateurs
ON ressources.owner = utilisateurs.nom
WHERE utilisateurs.naissance>2000
\end{minted}
	\fi
	\Question{Ecrire une requête SQL permettant de lister tous les utilisateurs n'ayant déposé aucune ressource.}
	\ifcorrige
		\begin{minted}{sql}
SELECT nom FROM utilisateurs
EXCEPT 
SELECT owner FROM ressources
\end{minted}
	\fi
\end{Exercise*}

\begin{Exercise*}[title = {type B}]\\
	On considère un tableau d'entiers $t$ de taille $n \neq  0$ et on s'intéresse au problème de la recherche de la somme maximale d'une tranche de  $t$ (c'est à dire d'éléments contigus de $t$).\\
	Par exemple, si $t = \{2, -7, -5, 4, -1, 10, -4, 9, -2\}$ alors la somme maximale d'une tranche est $18$ (obtenue en prenant la tranche $\{4, -1, 10, -4, 9\}$). On notera $(t_i)_{0  \leqslant i \leqslant n-1}$ les éléments de $t$ et $S_{i,j} = t_i+ \dots +t_j$ ($0 \leqslant i \leqslant j \leqslant n-1$) la somme de la tranche comprises entre les indices $i$ (inclus) et $j$ (inclus).\\
	Les fonctions demandées dans cet exercice doivent être écrites en langage C. Un fichier contenant le code compagnon de cet exercice est à télécharger à l'adresse {\tt https://fabricenativel.github.io/cpge-info/oraux/}, il contient une fonction de prototype \mintinline{c}{int max3(int a, int b, int c)} qui renvoie le maximum des trois entiers {\tt a}, {\tt b} et {\tt c} ainsi qu'une fonction {\tt main} dans laquelle est définie le tableau donné en exemple ci-dessus.
	\Question{Ecrire une fonction de signature \mintinline{c}{int tranchemax_naif(int tab[], int taille)} qui résoud ce problème en calculant de proche en proche les  $S_{i,j}$ grâce aux relations $S_{i,i} =t_i$ et $S_{i,j} = S_{i,j-1}+ t_j$ et en renvoyant leur maximum.}
	\ifcorrige
	\inputpartC{tmax.c}{}{}{4}{21}
	\fi
	\Question{Quelle est la complexité de cette fonction ?}
	\tcor{On utilise donc deux boucles imbriquées dans laquelle on effectue uniquement des opérations élémentaires, la complexité est donc  $\mathcal{O}(n^2)$.}
	
	\NRet \smallskip
	On veut maintenant implémenter une méthode \textit{diviser pour régner}, pour cela, on note $k = \PE{\dfrac{n}{2}}$ et :
	\begin{itemize}
		\item On sépare $t$ en deux sous tableaux $t_g = \{t_0, \dots, t_{k-1}\}$ et $t_d = \{ t_{k+1}, \dots t_{n-1} \}$ (on remarquera bien que $t_k$ n'appartient à aucun de ces deux sous tableaux.)
		\item On recherche récursivement les tranches maximales de ces deux sous tableaux ainsi que la valeur maximale d'une tranche contenant $t_k$ (qu'on pourra obtenir en considérant le maximum des tranches se terminant sur $t_k$ et celui des tranches commençant en $t_k$.).
		\item on prend le maximum des trois valeurs obtenues.
	\end{itemize}

	\Ret \smallskip
	\Question{Ecrire une fonction de signature \mintinline{c}{int max_tranchek(int tab[], int start, int k, int end)} qui renvoie dans le tableau {\tt tab} entre les indices {\tt start} (inclus) et {\tt end} (exclu) le maximum d'une tranche contenant l'élémént {\tt tab[k]}}.
	\ifcorrige
	\inputpartC{tmax.c}{}{}{32}{59}
	\fi
    \Question{Ecrire une fonction de signature \mintinline{c}{int tranchemax_dpr(int tab[], int size, int start, int end)} qui implémente la méthode diviser pour régner décrite ci-dessus.
}
\ifcorrige
	\inputpartC{tmax.c}{}{}{61}{82}
	\fi
	\Question{Quelle est la complexité de la méthode diviser pour régner ?}
	\tcor{Pour résoudre un problème de taille $n$, on doit en résoudre deux de tailles $\lfloor \frac{n}{2} \rfloor$ et rechercher le maximum des tranches contenant l'élément $t_k$. Cette opération a une complexité linéaire, on a donc :
				$\left\{
					\begin{array}{lll}
						C(0)  & \in & O(1)          \\
						C(2n) & =   & 2C(n)  + O(n) \\
					\end{array}
					\right.$\\
				La résolution de cette équation de complexité donne une complexité en  $O(n\,\log n)$.}
\end{Exercise*}


\end{document}
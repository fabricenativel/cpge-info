\documentclass[11pt,a4paper]{article}

\usepackage{Act}

\begin{document}
\input{\detokenize{/home/fenarius/Travail/Cours/cpge-info/latex/Macros.tex}}
\ModeExercice
\Colles{S17/S18/S19}{Analyse d'algorithmes}

\setcounter{Exercise}{0}

\begin{Exercise}[title = {Calcul des termes de la suite de Fibonnacci}]\\
On rappelle que la suite de Fibonacci $(F_n)_{n \in \N}$ est définie par $F_0=1$, $F_1 = 1$ et $\forall n\geq 2, F_n = F_{n-1} + F_{n-2}$ (chaque terme est la somme des deux précédents)

\Question{Donner un algorithme récursif naïf permettant de calculer le nième terme de cette suite.}
\tcor{Pour calculer $F_n$, on calcule récursivement $F_{n-1}$ et $F_{n-1}$ et on les ajoute.}
\Question{En proposer une implémentation en OCaml.}
\tcor{\inputpartOCaml{fibo.ml}{}{}{1}{2}}
\Question{On note $K_n$ le nombre d'appels récursifs nécessaires au calcul de $F_n$. Donner $K_0$, $K_1$ et la relation de récurrence vérifiée par $(K_n)_{(n \in \N)}$}
\tcor{
    $K_0 = 1$, $K_1 =1$ et $K_n = 1 + K_{n-1} + K_{n-2}$
}
\Question{Etablir que $K_n = 2F_n - 1$.}
\tcor{
    On effectue une preuve par récurrence de $\mathcal{P}(n)$ = \og{ $K_n = 2F_n - 1$} \fg{}
    \begin{itemize}
        \item[\textbullet] $2F_0 - 1 = 1$ et $K_0 = 1$ donc $\mathcal{P}(0)$ est vraie et de même $\mathcal{P}(1)$ est vraie.
        \item[\textbullet] Soit $n \in \N$ tel que $\mathcal{P}(n)$ et $\mathcal{P}(n-1)$ soient vraies, montrons $\mathcal{P}(n+1)$ vraie. \\
        $K_{n+1} = 1 + K_{n} + K_{n-1}$ et par hypothèse de récurrence : \\
        $K_{n+1} = 1 + 2F_n - 1 + 2F_{n-1} - 1$ \\
        $K_{n+1} = F_n - 1$
    \end{itemize}
}
\Question{En déduire la complexité de l'algorithme récursif naïf.}
\tcor{
    En notant $\varphi = \dfrac{1+\sqrt{5}}{2}$, on sait que $F_n$ est un grand $O$ de $\varphi^n$. Donc l'algorithme récursif naif a une complexité exponentielle ($\varphi >1$).
}
\Question{Donner un algorithme de complexité linéaire permettant de calculer $F_n$.}
\tcor{
    On peut calculer les termes de proches en proches de façon itérative.
}
\Question{En fournir une implémentation en langage C.}
\tcor{\inputpartC{fibo.c}{}{}{3}{15}}
\end{Exercise}

\begin{Exercise}[title = {Suite \og \textit{lock and say} \fg}]\\
    La suite de Conway ou suite \og \textit{lock and say} \fg{} (regarder et dire) a pour premier terme $u_1=1$, puis chaque terme se détermine en énonçant les chiffres du terme précédent. Ainsi, \\
    $u_2 = 11$ (car le terme précédent contient une fois le chiffre 1) \\
    $u_3 = 21$ (car le terme précédent contient 2 fois le chiffre 1) \\
    $u_4 = 1211$ (car le terme précédent contient une fois le chiffre 2 puis 1 fois le chiffre 1)

    \Question{Déterminer les termes $u_5$, $u_6$, $u_7$ et $u_8$.}
    \tcor{
        $u_5 = 111221$ \\
        $u_6 = 312211$ \\
        $u_7 = 13112221$ \\
        $u_8 = 1113213211$ 
    }
    \Question{Proposer une conjecture sur les chiffres pouvant intervenir dans un terme de la suite.}
    \tcor{On conjecture que la suite ne contient que les chiffres 1, 2 ou 3.}
    \Question{Prouver cette conjecture 
    \tcor{On effectue une preuve par récurrence sur $\N^*$ de la propriété $\mathcal{P}(n)=$\textit{\og{} "Le nième terme de la suite ne contient que les chiffres 1,2 ou 3 \fg{}}
    \begin{itemize}
        \item $\mathcal{P}(1)$ est vraie puisque $u_1=1$.
        \item Soit $n \in \N$ tel que $\mathcal{P}(n)$ est vraie, montrons qu'alors $P(n+1)$ est vraie. Par hypothèse de récurrence, $u_n$ ne contient que des 1, des 2 ou des 3. La seule façon pour qu'un autre chiffre $k \in \N$, ($k\neq 1, 2 \text{ ou } 3$) apparaisse dans $P(n+1)$ est donc d'avoir $k$ chiffres consécutifs identiques dans $u_n$. Or les chiffres de $u_{n+1}$ apparaissent par pairs constitué d'un nombre d'apparitions et d'un chiffre, qu'on note $n_1, c_1, n_2, c_2, \dots n_p,c_p$. Les chiffres de deux pairs consécutives sont différents car sinon ils seraient regroupés dans la même pair, donc $c_i \neq c_{i+1}$ pour $i \in \intN{0}{p-1}$.  Donc cette suite ne peut contenir plus de 3 chiffres consécutifs égaux.
    \end{itemize}}
    {\small \aide}\; On pourra faire une preuve par récurrence et raisonner par l'absurde.}
    \Question{En utilisant le résultat établi à la question précédente, proposer une fonction en OCaml {\tt int list -> int list} qui prend en argument un terme de la suite de Conway (sous la forme de la liste de ses chiffres) et renvoie le terme suivant (toujours sous la forme de la liste de ses chiffres)} \\
    \tcor{
        D'après la résultat précédent on ne peut avoir plus de 3 chiffres consécutifs égaux, on peut donc se limiter à une correspondance de motifs sur au maximum 3 chiffres :
        \inputpartOCaml{lockandsay.ml}{}{}{1}{8}
    }
    {\small \aide} \; Utiliser  une correspondance de motif sur les chiffres par groupe de 3.
    \Question{Ecrire une fonction en OCaml utilisant la fonction précédente et calculant le nième terme de la suite de Conway.}
    \tcor{        \inputpartOCaml{lockandsay.ml}{}{}{9}{12}
    }
\end{Exercise}

\end{document}
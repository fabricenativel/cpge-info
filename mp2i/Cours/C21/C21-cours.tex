\PassOptionsToPackage{dvipsnames,table}{xcolor}
\documentclass[10pt]{beamer}
\usepackage{Cours}

\begin{document}


\newcounter{numchap}
\setcounter{numchap}{1}
\newcounter{numframe}
\setcounter{numframe}{0}
\newcommand{\mframe}[1]{\frametitle{#1} \addtocounter{numframe}{1}}
\newcommand{\cnum}{\fbox{\textcolor{yellow}{\textbf{C\thenumchap}}}~}
\newcommand{\makess}[1]{\section{#1} \label{ss\thesection}}
\newcommand{\stitle}{\textcolor{yellow}{\textbf{\thesection. \nameref{ss\thesection}}}}

\definecolor{codebg}{gray}{0.90}
\definecolor{grispale}{gray}{0.95}
\definecolor{fluo}{rgb}{1,0.96,0.62}
\newminted[langageC]{c}{linenos=true,escapeinside=||,highlightcolor=fluo,tabsize=2,breaklines=true}
\newminted[codepython]{python}{linenos=true,escapeinside=||,highlightcolor=fluo,tabsize=2,breaklines=true}
% Inclusion complète (ou partiel en indiquant premiere et dernière ligne) d'un fichier C
\newcommand{\inputC}[3]{\begin{mdframed}[backgroundcolor=codebg] \inputminted[breaklines=true,fontsize=#3,linenos=true,highlightcolor=fluo,tabsize=2,highlightlines={#2}]{c}{#1} \end{mdframed}}
\newcommand{\inputpartC}[5]{\begin{mdframed}[backgroundcolor=codebg] \inputminted[breaklines=true,fontsize=#3,linenos=true,highlightcolor=fluo,tabsize=2,highlightlines={#2},firstline=#4,lastline=#5,firstnumber=1]{c}{#1} \end{mdframed}}
\newcommand{\inputpython}[3]{\begin{mdframed}[backgroundcolor=codebg] \inputminted[breaklines=true,fontsize=#3,linenos=true,highlightcolor=fluo,tabsize=2,highlightlines={#2}]{python}{#1} \end{mdframed}}
\newcommand{\inputpartOCaml}[5]{\begin{mdframed}[backgroundcolor=codebg] \inputminted[breaklines=true,fontsize=#3,linenos=true,highlightcolor=fluo,tabsize=2,highlightlines={#2},firstline=#4,lastline=#5,firstnumber=1]{OCaml}{#1} \end{mdframed}}
\BeforeBeginEnvironment{minted}{\begin{mdframed}[backgroundcolor=codebg]}
\AfterEndEnvironment{minted}{\end{mdframed}}
\newcommand{\kw}[1]{\textcolor{blue}{\tt #1}}

\newtcolorbox{rcadre}[4]{halign=center,colback={#1},colframe={#2},width={#3cm},height={#4cm},valign=center,boxrule=1pt,left=0pt,right=0pt}
\newtcolorbox{cadre}[4]{halign=center,colback={#1},colframe={#2},arc=0mm,width={#3cm},height={#4cm},valign=center,boxrule=1pt,left=0pt,right=0pt}
\newcommand{\myem}[1]{\colorbox{fluo}{#1}}
\mdfsetup{skipabove=1pt,skipbelow=-2pt}



% Noeud dans un cadre pour les arbres
\newcommand{\noeud}[2]{\Tr{\fbox{\textcolor{#1}{\tt #2}}}}

\newcommand{\htmlmode}{\lstset{language=html,numbers=left, tabsize=4, frame=single, breaklines=true, keywordstyle=\ttfamily, basicstyle=\small,
   numberstyle=\tiny\ttfamily, framexleftmargin=0mm, backgroundcolor=\color{grispale}, xleftmargin=12mm,showstringspaces=false}}
\newcommand{\pythonmode}{\lstset{
   language=python,
   linewidth=\linewidth,
   numbers=left,
   tabsize=4,
   frame=single,
   breaklines=true,
   keywordstyle=\ttfamily\color{blue},
   basicstyle=\small,
   numberstyle=\tiny\ttfamily,
   framexleftmargin=-2mm,
   numbersep=-0.5mm,
   backgroundcolor=\color{codebg},
   xleftmargin=-1mm, 
   showstringspaces=false,
   commentstyle=\color{gray},
   stringstyle=\color{OliveGreen},
   emph={turtle,Screen,Turtle},
   emphstyle=\color{RawSienna},
   morekeywords={setheading,goto,backward,forward,left,right,pendown,penup,pensize,color,speed,hideturtle,showturtle,forward}}
   }
   \newcommand{\Cmode}{\lstset{
      language=[ANSI]C,
      linewidth=\linewidth,
      numbers=left,
      tabsize=4,
      frame=single,
      breaklines=true,
      keywordstyle=\ttfamily\color{blue},
      basicstyle=\small,
      numberstyle=\tiny\ttfamily,
      framexleftmargin=0mm,
      numbersep=2mm,
      backgroundcolor=\color{codebg},
      xleftmargin=0mm, 
      showstringspaces=false,
      commentstyle=\color{gray},
      stringstyle=\color{OliveGreen},
      emphstyle=\color{RawSienna},
      escapechar=\|,
      morekeywords={}}
      }
\newcommand{\bashmode}{\lstset{language=bash,numbers=left, tabsize=2, frame=single, breaklines=true, basicstyle=\ttfamily,
   numberstyle=\tiny\ttfamily, framexleftmargin=0mm, backgroundcolor=\color{grispale}, xleftmargin=12mm, showstringspaces=false}}
\newcommand{\exomode}{\lstset{language=python,numbers=left, tabsize=2, frame=single, breaklines=true, basicstyle=\ttfamily,
   numberstyle=\tiny\ttfamily, framexleftmargin=13mm, xleftmargin=12mm, basicstyle=\small, showstringspaces=false}}
   
   
  
%tei pour placer les images
%tei{nom de l’image}{échelle de l’image}{sens}{texte a positionner}
%sens ="1" (droite) ou "2" (gauche)
\newlength{\ltxt}
\newcommand{\tei}[4]{
\setlength{\ltxt}{\linewidth}
\setbox0=\hbox{\includegraphics[scale=#2]{#1}}
\addtolength{\ltxt}{-\wd0}
\addtolength{\ltxt}{-10pt}
\ifthenelse{\equal{#3}{1}}{
\begin{minipage}{\wd0}
\includegraphics[scale=#2]{#1}
\end{minipage}
\hfill
\begin{minipage}{\ltxt}
#4
\end{minipage}
}{
\begin{minipage}{\ltxt}
#4
\end{minipage}
\hfill
\begin{minipage}{\wd0}
\includegraphics[scale=#2]{#1}
\end{minipage}
}
}

%Juxtaposition d'une image pspciture et de texte 
%#1: = code pstricks de l'image
%#2: largeur de l'image
%#3: hauteur de l'image
%#4: Texte à écrire
\newcommand{\ptp}[4]{
\setlength{\ltxt}{\linewidth}
\addtolength{\ltxt}{-#2 cm}
\addtolength{\ltxt}{-0.1 cm}
\begin{minipage}[b][#3 cm][t]{\ltxt}
#4
\end{minipage}\hfill
\begin{minipage}[b][#3 cm][c]{#2 cm}
#1
\end{minipage}\par
}



%Macros pour les graphiques
\psset{linewidth=0.5\pslinewidth,PointSymbol=x}
\setlength{\fboxrule}{0.5pt}
\newcounter{tempangle}

%Marque la longueur du segment d'extrémité  #1 et  #2 avec la valeur #3, #4 est la distance par rapport au segment (en %age de la valeur de celui ci) et #5 l'orientation du marquage : +90 ou -90
\newcommand{\afflong}[5]{
\pstRotation[RotAngle=#4,PointSymbol=none,PointName=none]{#1}{#2}[X] 
\pstHomO[PointSymbol=none,PointName=none,HomCoef=#5]{#1}{X}[Y]
\pstTranslation[PointSymbol=none,PointName=none]{#1}{#2}{Y}[Z]
 \ncline{|<->|,linewidth=0.25\pslinewidth}{Y}{Z} \ncput*[nrot=:U]{\footnotesize{#3}}
}
\newcommand{\afflongb}[3]{
\ncline{|<->|,linewidth=0}{#1}{#2} \naput*[nrot=:U]{\footnotesize{#3}}
}

%Construis le point #4 situé à #2 cm du point #1 avant un angle #3 par rapport à l'horizontale. #5 = liste de paramètre
\newcommand{\lsegment}[5]{\pstGeonode[PointSymbol=none,PointName=none](0,0){O'}(#2,0){I'} \pstTranslation[PointSymbol=none,PointName=none]{O'}{I'}{#1}[J'] \pstRotation[RotAngle=#3,PointSymbol=x,#5]{#1}{J'}[#4]}
\newcommand{\tsegment}[5]{\pstGeonode[PointSymbol=none,PointName=none](0,0){O'}(#2,0){I'} \pstTranslation[PointSymbol=none,PointName=none]{O'}{I'}{#1}[J'] \pstRotation[RotAngle=#3,PointSymbol=x,#5]{#1}{J'}[#4] \pstLineAB{#4}{#1}}

%Construis le point #4 situé à #3 cm du point #1 et faisant un angle de  90° avec la droite (#1,#2) #5 = liste de paramètre
\newcommand{\psegment}[5]{
\pstGeonode[PointSymbol=none,PointName=none](0,0){O'}(#3,0){I'}
 \pstTranslation[PointSymbol=none,PointName=none]{O'}{I'}{#1}[J']
 \pstInterLC[PointSymbol=none,PointName=none]{#1}{#2}{#1}{J'}{M1}{M2} \pstRotation[RotAngle=-90,PointSymbol=x,#5]{#1}{M1}[#4]
  }
  
%Construis le point #4 situé à #3 cm du point #1 et faisant un angle de  #5° avec la droite (#1,#2) #6 = liste de paramètre
\newcommand{\mlogo}[6]{
\pstGeonode[PointSymbol=none,PointName=none](0,0){O'}(#3,0){I'}
 \pstTranslation[PointSymbol=none,PointName=none]{O'}{I'}{#1}[J']
 \pstInterLC[PointSymbol=none,PointName=none]{#1}{#2}{#1}{J'}{M1}{M2} \pstRotation[RotAngle=#5,PointSymbol=x,#6]{#1}{M2}[#4]
  }

% Construis un triangle avec #1=liste des 3 sommets séparés par des virgules, #2=liste des 3 longueurs séparés par des virgules, #3 et #4 : paramètre d'affichage des 2e et 3 points et #5 : inclinaison par rapport à l'horizontale
%autre macro identique mais sans tracer les segments joignant les sommets
\noexpandarg
\newcommand{\Triangleccc}[5]{
\StrBefore{#1}{,}[\pointA]
\StrBetween[1,2]{#1}{,}{,}[\pointB]
\StrBehind[2]{#1}{,}[\pointC]
\StrBefore{#2}{,}[\coteA]
\StrBetween[1,2]{#2}{,}{,}[\coteB]
\StrBehind[2]{#2}{,}[\coteC]
\tsegment{\pointA}{\coteA}{#5}{\pointB}{#3} 
\lsegment{\pointA}{\coteB}{0}{Z1}{PointSymbol=none, PointName=none}
\lsegment{\pointB}{\coteC}{0}{Z2}{PointSymbol=none, PointName=none}
\pstInterCC{\pointA}{Z1}{\pointB}{Z2}{\pointC}{Z3} 
\pstLineAB{\pointA}{\pointC} \pstLineAB{\pointB}{\pointC}
\pstSymO[PointName=\pointC,#4]{C}{C}[C]
}
\noexpandarg
\newcommand{\TrianglecccP}[5]{
\StrBefore{#1}{,}[\pointA]
\StrBetween[1,2]{#1}{,}{,}[\pointB]
\StrBehind[2]{#1}{,}[\pointC]
\StrBefore{#2}{,}[\coteA]
\StrBetween[1,2]{#2}{,}{,}[\coteB]
\StrBehind[2]{#2}{,}[\coteC]
\tsegment{\pointA}{\coteA}{#5}{\pointB}{#3} 
\lsegment{\pointA}{\coteB}{0}{Z1}{PointSymbol=none, PointName=none}
\lsegment{\pointB}{\coteC}{0}{Z2}{PointSymbol=none, PointName=none}
\pstInterCC[PointNameB=none,PointSymbolB=none,#4]{\pointA}{Z1}{\pointB}{Z2}{\pointC}{Z1} 
}


% Construis un triangle avec #1=liste des 3 sommets séparés par des virgules, #2=liste formée de 2 longueurs et d'un angle séparés par des virgules, #3 et #4 : paramètre d'affichage des 2e et 3 points et #5 : inclinaison par rapport à l'horizontale
%autre macro identique mais sans tracer les segments joignant les sommets
\newcommand{\Trianglecca}[5]{
\StrBefore{#1}{,}[\pointA]
\StrBetween[1,2]{#1}{,}{,}[\pointB]
\StrBehind[2]{#1}{,}[\pointC]
\StrBefore{#2}{,}[\coteA]
\StrBetween[1,2]{#2}{,}{,}[\coteB]
\StrBehind[2]{#2}{,}[\angleA]
\tsegment{\pointA}{\coteA}{#5}{\pointB}{#3} 
\setcounter{tempangle}{#5}
\addtocounter{tempangle}{\angleA}
\tsegment{\pointA}{\coteB}{\thetempangle}{\pointC}{#4}
\pstLineAB{\pointB}{\pointC}
}
\newcommand{\TriangleccaP}[5]{
\StrBefore{#1}{,}[\pointA]
\StrBetween[1,2]{#1}{,}{,}[\pointB]
\StrBehind[2]{#1}{,}[\pointC]
\StrBefore{#2}{,}[\coteA]
\StrBetween[1,2]{#2}{,}{,}[\coteB]
\StrBehind[2]{#2}{,}[\angleA]
\lsegment{\pointA}{\coteA}{#5}{\pointB}{#3} 
\setcounter{tempangle}{#5}
\addtocounter{tempangle}{\angleA}
\lsegment{\pointA}{\coteB}{\thetempangle}{\pointC}{#4}
}

% Construis un triangle avec #1=liste des 3 sommets séparés par des virgules, #2=liste formée de 1 longueurs et de deux angle séparés par des virgules, #3 et #4 : paramètre d'affichage des 2e et 3 points et #5 : inclinaison par rapport à l'horizontale
%autre macro identique mais sans tracer les segments joignant les sommets
\newcommand{\Trianglecaa}[5]{
\StrBefore{#1}{,}[\pointA]
\StrBetween[1,2]{#1}{,}{,}[\pointB]
\StrBehind[2]{#1}{,}[\pointC]
\StrBefore{#2}{,}[\coteA]
\StrBetween[1,2]{#2}{,}{,}[\angleA]
\StrBehind[2]{#2}{,}[\angleB]
\tsegment{\pointA}{\coteA}{#5}{\pointB}{#3} 
\setcounter{tempangle}{#5}
\addtocounter{tempangle}{\angleA}
\lsegment{\pointA}{1}{\thetempangle}{Z1}{PointSymbol=none, PointName=none}
\setcounter{tempangle}{#5}
\addtocounter{tempangle}{180}
\addtocounter{tempangle}{-\angleB}
\lsegment{\pointB}{1}{\thetempangle}{Z2}{PointSymbol=none, PointName=none}
\pstInterLL[#4]{\pointA}{Z1}{\pointB}{Z2}{\pointC}
\pstLineAB{\pointA}{\pointC}
\pstLineAB{\pointB}{\pointC}
}
\newcommand{\TrianglecaaP}[5]{
\StrBefore{#1}{,}[\pointA]
\StrBetween[1,2]{#1}{,}{,}[\pointB]
\StrBehind[2]{#1}{,}[\pointC]
\StrBefore{#2}{,}[\coteA]
\StrBetween[1,2]{#2}{,}{,}[\angleA]
\StrBehind[2]{#2}{,}[\angleB]
\lsegment{\pointA}{\coteA}{#5}{\pointB}{#3} 
\setcounter{tempangle}{#5}
\addtocounter{tempangle}{\angleA}
\lsegment{\pointA}{1}{\thetempangle}{Z1}{PointSymbol=none, PointName=none}
\setcounter{tempangle}{#5}
\addtocounter{tempangle}{180}
\addtocounter{tempangle}{-\angleB}
\lsegment{\pointB}{1}{\thetempangle}{Z2}{PointSymbol=none, PointName=none}
\pstInterLL[#4]{\pointA}{Z1}{\pointB}{Z2}{\pointC}
}

%Construction d'un cercle de centre #1 et de rayon #2 (en cm)
\newcommand{\Cercle}[2]{
\lsegment{#1}{#2}{0}{Z1}{PointSymbol=none, PointName=none}
\pstCircleOA{#1}{Z1}
}

%construction d'un parallélogramme #1 = liste des sommets, #2 = liste contenant les longueurs de 2 côtés consécutifs et leurs angles;  #3, #4 et #5 : paramètre d'affichage des sommets #6 inclinaison par rapport à l'horizontale 
% meme macro sans le tracé des segements
\newcommand{\Para}[6]{
\StrBefore{#1}{,}[\pointA]
\StrBetween[1,2]{#1}{,}{,}[\pointB]
\StrBetween[2,3]{#1}{,}{,}[\pointC]
\StrBehind[3]{#1}{,}[\pointD]
\StrBefore{#2}{,}[\longueur]
\StrBetween[1,2]{#2}{,}{,}[\largeur]
\StrBehind[2]{#2}{,}[\angle]
\tsegment{\pointA}{\longueur}{#6}{\pointB}{#3} 
\setcounter{tempangle}{#6}
\addtocounter{tempangle}{\angle}
\tsegment{\pointA}{\largeur}{\thetempangle}{\pointD}{#5}
\pstMiddleAB[PointName=none,PointSymbol=none]{\pointB}{\pointD}{Z1}
\pstSymO[#4]{Z1}{\pointA}[\pointC]
\pstLineAB{\pointB}{\pointC}
\pstLineAB{\pointC}{\pointD}
}
\newcommand{\ParaP}[6]{
\StrBefore{#1}{,}[\pointA]
\StrBetween[1,2]{#1}{,}{,}[\pointB]
\StrBetween[2,3]{#1}{,}{,}[\pointC]
\StrBehind[3]{#1}{,}[\pointD]
\StrBefore{#2}{,}[\longueur]
\StrBetween[1,2]{#2}{,}{,}[\largeur]
\StrBehind[2]{#2}{,}[\angle]
\lsegment{\pointA}{\longueur}{#6}{\pointB}{#3} 
\setcounter{tempangle}{#6}
\addtocounter{tempangle}{\angle}
\lsegment{\pointA}{\largeur}{\thetempangle}{\pointD}{#5}
\pstMiddleAB[PointName=none,PointSymbol=none]{\pointB}{\pointD}{Z1}
\pstSymO[#4]{Z1}{\pointA}[\pointC]
}


%construction d'un cerf-volant #1 = liste des sommets, #2 = liste contenant les longueurs de 2 côtés consécutifs et leurs angles;  #3, #4 et #5 : paramètre d'affichage des sommets #6 inclinaison par rapport à l'horizontale 
% meme macro sans le tracé des segements
\newcommand{\CerfVolant}[6]{
\StrBefore{#1}{,}[\pointA]
\StrBetween[1,2]{#1}{,}{,}[\pointB]
\StrBetween[2,3]{#1}{,}{,}[\pointC]
\StrBehind[3]{#1}{,}[\pointD]
\StrBefore{#2}{,}[\longueur]
\StrBetween[1,2]{#2}{,}{,}[\largeur]
\StrBehind[2]{#2}{,}[\angle]
\tsegment{\pointA}{\longueur}{#6}{\pointB}{#3} 
\setcounter{tempangle}{#6}
\addtocounter{tempangle}{\angle}
\tsegment{\pointA}{\largeur}{\thetempangle}{\pointD}{#5}
\pstOrtSym[#4]{\pointB}{\pointD}{\pointA}[\pointC]
\pstLineAB{\pointB}{\pointC}
\pstLineAB{\pointC}{\pointD}
}

%construction d'un quadrilatère quelconque #1 = liste des sommets, #2 = liste contenant les longueurs des 4 côtés et l'angle entre 2 cotés consécutifs  #3, #4 et #5 : paramètre d'affichage des sommets #6 inclinaison par rapport à l'horizontale 
% meme macro sans le tracé des segements
\newcommand{\Quadri}[6]{
\StrBefore{#1}{,}[\pointA]
\StrBetween[1,2]{#1}{,}{,}[\pointB]
\StrBetween[2,3]{#1}{,}{,}[\pointC]
\StrBehind[3]{#1}{,}[\pointD]
\StrBefore{#2}{,}[\coteA]
\StrBetween[1,2]{#2}{,}{,}[\coteB]
\StrBetween[2,3]{#2}{,}{,}[\coteC]
\StrBetween[3,4]{#2}{,}{,}[\coteD]
\StrBehind[4]{#2}{,}[\angle]
\tsegment{\pointA}{\coteA}{#6}{\pointB}{#3} 
\setcounter{tempangle}{#6}
\addtocounter{tempangle}{\angle}
\tsegment{\pointA}{\coteD}{\thetempangle}{\pointD}{#5}
\lsegment{\pointB}{\coteB}{0}{Z1}{PointSymbol=none, PointName=none}
\lsegment{\pointD}{\coteC}{0}{Z2}{PointSymbol=none, PointName=none}
\pstInterCC[PointNameA=none,PointSymbolA=none,#4]{\pointB}{Z1}{\pointD}{Z2}{Z3}{\pointC} 
\pstLineAB{\pointB}{\pointC}
\pstLineAB{\pointC}{\pointD}
}


% Définition des colonnes centrées ou à droite pour tabularx
\newcolumntype{Y}{>{\centering\arraybackslash}X}
\newcolumntype{Z}{>{\flushright\arraybackslash}X}

%Les pointillés à remplir par les élèves
\newcommand{\po}[1]{\makebox[#1 cm]{\dotfill}}
\newcommand{\lpo}[1][3]{%
\multido{}{#1}{\makebox[\linewidth]{\dotfill}
}}

%Liste des pictogrammes utilisés sur la fiche d'exercice ou d'activités
\newcommand{\bombe}{\faBomb}
\newcommand{\livre}{\faBook}
\newcommand{\calculatrice}{\faCalculator}
\newcommand{\oral}{\faCommentO}
\newcommand{\surfeuille}{\faEdit}
\newcommand{\ordinateur}{\faLaptop}
\newcommand{\ordi}{\faDesktop}
\newcommand{\ciseaux}{\faScissors}
\newcommand{\danger}{\faExclamationTriangle}
\newcommand{\out}{\faSignOut}
\newcommand{\cadeau}{\faGift}
\newcommand{\flash}{\faBolt}
\newcommand{\lumiere}{\faLightbulb}
\newcommand{\compas}{\dsmathematical}
\newcommand{\calcullitteral}{\faTimesCircleO}
\newcommand{\raisonnement}{\faCogs}
\newcommand{\recherche}{\faSearch}
\newcommand{\rappel}{\faHistory}
\newcommand{\video}{\faFilm}
\newcommand{\capacite}{\faPuzzlePiece}
\newcommand{\aide}{\faLifeRing}
\newcommand{\loin}{\faExternalLink}
\newcommand{\groupe}{\faUsers}
\newcommand{\bac}{\faGraduationCap}
\newcommand{\histoire}{\faUniversity}
\newcommand{\coeur}{\faSave}
\newcommand{\python}{\faPython}
\newcommand{\os}{\faMicrochip}
\newcommand{\rd}{\faCubes}
\newcommand{\data}{\faColumns}
\newcommand{\web}{\faCode}
\newcommand{\prog}{\faFile}
\newcommand{\algo}{\faCogs}
\newcommand{\important}{\faExclamationCircle}
\newcommand{\maths}{\faTimesCircle}
% Traitement des données en tables
\newcommand{\tables}{\faColumns}
% Types construits
\newcommand{\construits}{\faCubes}
% Type et valeurs de base
\newcommand{\debase}{{\footnotesize \faCube}}
% Systèmes d'exploitation
\newcommand{\linux}{\faLinux}
\newcommand{\sd}{\faProjectDiagram}
\newcommand{\bd}{\faDatabase}

%Les ensembles de nombres
\renewcommand{\N}{\mathbb{N}}
\newcommand{\D}{\mathbb{D}}
\newcommand{\Z}{\mathbb{Z}}
\newcommand{\Q}{\mathbb{Q}}
\newcommand{\R}{\mathbb{R}}
\newcommand{\C}{\mathbb{C}}

%Ecriture des vecteurs
\newcommand{\vect}[1]{\vbox{\halign{##\cr 
  \tiny\rightarrowfill\cr\noalign{\nointerlineskip\vskip1pt} 
  $#1\mskip2mu$\cr}}}


%Compteur activités/exos et question et mise en forme titre et questions
\newcounter{numact}
\setcounter{numact}{1}
\newcounter{numseance}
\setcounter{numseance}{1}
\newcounter{numexo}
\setcounter{numexo}{0}
\newcounter{numprojet}
\setcounter{numprojet}{0}
\newcounter{numquestion}
\newcommand{\espace}[1]{\rule[-1ex]{0pt}{#1 cm}}
\newcommand{\Quest}[3]{
\addtocounter{numquestion}{1}
\begin{tabularx}{\textwidth}{X|m{1cm}|}
\cline{2-2}
\textbf{\sffamily{\alph{numquestion})}} #1 & \dots / #2 \\
\hline 
\multicolumn{2}{|l|}{\espace{#3}} \\
\hline
\end{tabularx}
}
\newcommand{\QuestR}[3]{
\addtocounter{numquestion}{1}
\begin{tabularx}{\textwidth}{X|m{1cm}|}
\cline{2-2}
\textbf{\sffamily{\alph{numquestion})}} #1 & \dots / #2 \\
\hline 
\multicolumn{2}{|l|}{\cor{#3}} \\
\hline
\end{tabularx}
}
\newcommand{\Pre}{{\sc nsi} 1\textsuperscript{e}}
\newcommand{\Term}{{\sc nsi} Terminale}
\newcommand{\Sec}{2\textsuperscript{e}}
\newcommand{\Exo}[2]{ \addtocounter{numexo}{1} \ding{113} \textbf{\sffamily{Exercice \thenumexo}} : \textit{#1} \hfill #2  \setcounter{numquestion}{0}}
\newcommand{\Projet}[1]{ \addtocounter{numprojet}{1} \ding{118} \textbf{\sffamily{Projet \thenumprojet}} : \textit{#1}}
\newcommand{\ExoD}[2]{ \addtocounter{numexo}{1} \ding{113} \textbf{\sffamily{Exercice \thenumexo}}  \textit{(#1 pts)} \hfill #2  \setcounter{numquestion}{0}}
\newcommand{\ExoB}[2]{ \addtocounter{numexo}{1} \ding{113} \textbf{\sffamily{Exercice \thenumexo}}  \textit{(Bonus de +#1 pts maximum)} \hfill #2  \setcounter{numquestion}{0}}
\newcommand{\Act}[2]{ \ding{113} \textbf{\sffamily{Activité \thenumact}} : \textit{#1} \hfill #2  \addtocounter{numact}{1} \setcounter{numquestion}{0}}
\newcommand{\Seance}{ \rule{1.5cm}{0.5pt}\raisebox{-3pt}{\framebox[4cm]{\textbf{\sffamily{Séance \thenumseance}}}}\hrulefill  \\
  \addtocounter{numseance}{1}}
\newcommand{\Acti}[2]{ {\footnotesize \ding{117}} \textbf{\sffamily{Activité \thenumact}} : \textit{#1} \hfill #2  \addtocounter{numact}{1} \setcounter{numquestion}{0}}
\newcommand{\titre}[1]{\begin{Large}\textbf{\ding{118}}\end{Large} \begin{large}\textbf{ #1}\end{large} \vspace{0.2cm}}
\newcommand{\QListe}[1][0]{
\ifthenelse{#1=0}
{\begin{enumerate}[partopsep=0pt,topsep=0pt,parsep=0pt,itemsep=0pt,label=\textbf{\sffamily{\arabic*.}},series=question]}
{\begin{enumerate}[resume*=question]}}
\newcommand{\SQListe}[1][0]{
\ifthenelse{#1=0}
{\begin{enumerate}[partopsep=0pt,topsep=0pt,parsep=0pt,itemsep=0pt,label=\textbf{\sffamily{\alph*)}},series=squestion]}
{\begin{enumerate}[resume*=squestion]}}
\newcommand{\SQListeL}[1][0]{
\ifthenelse{#1=0}
{\begin{enumerate*}[partopsep=0pt,topsep=0pt,parsep=0pt,itemsep=0pt,label=\textbf{\sffamily{\alph*)}},series=squestion]}
{\begin{enumerate*}[resume*=squestion]}}
\newcommand{\FinListe}{\end{enumerate}}
\newcommand{\FinListeL}{\end{enumerate*}}

%Mise en forme de la correction
\newcommand{\cor}[1]{\par \textcolor{OliveGreen}{#1}}
\newcommand{\br}[1]{\cor{\textbf{#1}}}
\newcommand{\tcor}[1]{\begin{tcolorbox}[width=0.92\textwidth,colback={white},colbacktitle=white,coltitle=OliveGreen,colframe=green!75!black,boxrule=0.2mm]   
\cor{#1}
\end{tcolorbox}
}
\newcommand{\rc}[1]{\textcolor{OliveGreen}{#1}}
\newcommand{\pmc}[1]{\textcolor{blue}{\tt #1}}
\newcommand{\tmc}[1]{\textcolor{RawSienna}{\tt #1}}


%Référence aux exercices par leur numéro
\newcommand{\refexo}[1]{
\refstepcounter{numexo}
\addtocounter{numexo}{-1}
\label{#1}}

%Séparation entre deux activités
\newcommand{\separateur}{\begin{center}
\rule{1.5cm}{0.5pt}\raisebox{-3pt}{\ding{117}}\rule{1.5cm}{0.5pt}  \vspace{0.2cm}
\end{center}}

%Entête et pied de page
\newcommand{\snt}[1]{\lhead{\textbf{SNT -- La photographie numérique} \rhead{\textit{Lycée Nord}}}}
\newcommand{\Activites}[2]{\lhead{\textbf{{\sc #1}}}
\rhead{Activités -- \textbf{#2}}
\cfoot{}}
\newcommand{\Exos}[2]{\lhead{\textbf{Fiche d'exercices: {\sc #1}}}
\rhead{Niveau: \textbf{#2}}
\cfoot{}}
\newcommand{\Devoir}[2]{\lhead{\textbf{Devoir de mathématiques : {\sc #1}}}
\rhead{\textbf{#2}} \setlength{\fboxsep}{8pt}
\begin{center}
%Titre de la fiche
\fbox{\parbox[b][1cm][t]{0.3\textwidth}{Nom : \hfill \po{3} \par \vfill Prénom : \hfill \po{3}} } \hfill 
\fbox{\parbox[b][1cm][t]{0.6\textwidth}{Note : \po{1} / 20} }
\end{center} \cfoot{}}

%Devoir programmation en NSI (pas à rendre sur papier)
\newcommand{\PNSI}[2]{\lhead{\textbf{Devoir de {\sc nsi} : \textsf{ #1}}}
\rhead{\textbf{#2}} \setlength{\fboxsep}{8pt}
\begin{tcolorbox}[title=\textcolor{black}{\danger\; A lire attentivement},colbacktitle=lightgray]
{\begin{enumerate}
\item Rendre tous vous programmes en les envoyant par mail à l'adresse {\tt fnativel2@ac-reunion.fr}, en précisant bien dans le sujet vos noms et prénoms
\item Un programme qui fonctionne mal ou pas du tout peut rapporter des points
\item Les bonnes pratiques de programmation (clarté et lisiblité du code) rapportent des points
\end{enumerate}
}
\end{tcolorbox}
 \cfoot{}}


%Devoir de NSI
\newcommand{\DNSI}[2]{\lhead{\textbf{Devoir de {\sc nsi} : \textsf{ #1}}}
\rhead{\textbf{#2}} \setlength{\fboxsep}{8pt}
\begin{center}
%Titre de la fiche
\fbox{\parbox[b][1cm][t]{0.3\textwidth}{Nom : \hfill \po{3} \par \vfill Prénom : \hfill \po{3}} } \hfill 
\fbox{\parbox[b][1cm][t]{0.6\textwidth}{Note : \po{1} / 10} }
\end{center} \cfoot{}}

\newcommand{\DevoirNSI}[2]{\lhead{\textbf{Devoir de {\sc nsi} : {\sc #1}}}
\rhead{\textbf{#2}} \setlength{\fboxsep}{8pt}
\cfoot{}}

%La définition de la commande QCM pour auto-multiple-choice
%En premier argument le sujet du qcm, deuxième argument : la classe, 3e : la durée prévue et #4 : présence ou non de questions avec plusieurs bonnes réponses
\newcommand{\QCM}[4]{
{\large \textbf{\ding{52} QCM : #1}} -- Durée : \textbf{#3 min} \hfill {\large Note : \dots/10} 
\hrule \vspace{0.1cm}\namefield{}
Nom :  \textbf{\textbf{\nom{}}} \qquad \qquad Prénom :  \textbf{\prenom{}}  \hfill Classe: \textbf{#2}
\vspace{0.2cm}
\hrule  
\begin{itemize}[itemsep=0pt]
\item[-] \textit{Une bonne réponse vaut un point, une absence de réponse n'enlève pas de point. }
\item[\danger] \textit{Une mauvaise réponse enlève un point.}
\ifthenelse{#4=1}{\item[-] \textit{Les questions marquées du symbole \multiSymbole{} peuvent avoir plusieurs bonnes réponses possibles.}}{}
\end{itemize}
}
\newcommand{\DevoirC}[2]{
\renewcommand{\footrulewidth}{0.5pt}
\lhead{\textbf{Devoir de mathématiques : {\sc #1}}}
\rhead{\textbf{#2}} \setlength{\fboxsep}{8pt}
\fbox{\parbox[b][0.4cm][t]{0.955\textwidth}{Nom : \po{5} \hfill Prénom : \po{5} \hfill Classe: \textbf{1}\textsuperscript{$\dots$}} } 
\rfoot{\thepage} \cfoot{} \lfoot{Lycée Nord}}
\newcommand{\DevoirInfo}[2]{\lhead{\textbf{Evaluation : {\sc #1}}}
\rhead{\textbf{#2}} \setlength{\fboxsep}{8pt}
 \cfoot{}}
\newcommand{\DM}[2]{\lhead{\textbf{Devoir maison à rendre le #1}} \rhead{\textbf{#2}}}

%Macros permettant l'affichage des touches de la calculatrice
%Touches classiques : #1 = 0 fond blanc pour les nombres et #1= 1gris pour les opérations et entrer, second paramètre=contenu
%Si #2=1 touche arrondi avec fond gris
\newcommand{\TCalc}[2]{
\setlength{\fboxsep}{0.1pt}
\ifthenelse{#1=0}
{\psframebox[fillstyle=solid, fillcolor=white]{\parbox[c][0.25cm][c]{0.6cm}{\centering #2}}}
{\ifthenelse{#1=1}
{\psframebox[fillstyle=solid, fillcolor=lightgray]{\parbox[c][0.25cm][c]{0.6cm}{\centering #2}}}
{\psframebox[framearc=.5,fillstyle=solid, fillcolor=white]{\parbox[c][0.25cm][c]{0.6cm}{\centering #2}}}
}}
\newcommand{\Talpha}{\psdblframebox[fillstyle=solid, fillcolor=white]{\hspace{-0.05cm}\parbox[c][0.25cm][c]{0.65cm}{\centering \scriptsize{alpha}}} \;}
\newcommand{\Tsec}{\psdblframebox[fillstyle=solid, fillcolor=white]{\parbox[c][0.25cm][c]{0.6cm}{\centering \scriptsize 2nde}} \;}
\newcommand{\Tfx}{\psdblframebox[fillstyle=solid, fillcolor=white]{\parbox[c][0.25cm][c]{0.6cm}{\centering \scriptsize $f(x)$}} \;}
\newcommand{\Tvar}{\psframebox[framearc=.5,fillstyle=solid, fillcolor=white]{\hspace{-0.22cm} \parbox[c][0.25cm][c]{0.82cm}{$\scriptscriptstyle{X,T,\theta,n}$}}}
\newcommand{\Tgraphe}{\psdblframebox[fillstyle=solid, fillcolor=white]{\hspace{-0.08cm}\parbox[c][0.25cm][c]{0.68cm}{\centering \tiny{graphe}}} \;}
\newcommand{\Tfen}{\psdblframebox[fillstyle=solid, fillcolor=white]{\hspace{-0.08cm}\parbox[c][0.25cm][c]{0.68cm}{\centering \tiny{fenêtre}}} \;}
\newcommand{\Ttrace}{\psdblframebox[fillstyle=solid, fillcolor=white]{\parbox[c][0.25cm][c]{0.6cm}{\centering \scriptsize{trace}}} \;}

% Macroi pour l'affichage  d'un entier n dans  une base b
\newcommand{\base}[2]{ \overline{#1}^{#2}}
% Intervalle d'entiers
\newcommand{\intN}[2]{\llbracket #1; #2 \rrbracket}}

% Numéro et titre de chapitre
\setcounter{numchap}{21}
\newcommand{\Ctitle}{\cnum {Compléments sur les arbres}}
\newcommand{\SPATH}{/home/fenarius/Travail/Cours/cpge-info/docs/mp2i/files/C\thenumchap/}
\newcommand{\NR}[1]{\TCircle[fillstyle=solid,fillcolor=red!20!white,linecolor=red,radius=0.25cm]{#1}}
\newcommand{\NN}[1]{\TCircle[fillstyle=solid,fillcolor=black!20!white,linecolor=black,radius=0.25cm]{#1}}
\psset{arrows=->,treesep=0.8cm,levelsep=0.8cm, radius=0.3cm}
\makess{Rappel}
\begin{frame}[fragile]{\Ctitle}{\stitle}
	\begin{block}{\textcolor{lightgray}{\small \rappel} Arbres binaires de recherche}
		\begin{enumerate}
			\item<1-> Rappeler les relations entre la hauteur $h$ et la taille $n$ d'un arbre binaire.
			\item<2-> Rappeler la définition d'un arbre binaire de recherche ({\sc abr}).
			\item<3-> L'arbre ci-dessous est-il un {\sc abr} ?
				\begin{center}
					\pstree{\TCircle{\tt 28}}
					{\pstree{\TCircle{\tt 16}}
						{\pstree{\TCircle{\tt 9}}
							{ \TCircle{7}
								\pstree{\TCircle{\tt 11}}
								{ \Tn{}
									\TCircle{\tt 13}
								}}\TCircle{\tt 21}
						}\pstree{\TCircle{\tt 37}}
						{
							\TCircle{\tt 34}
							\Tn{}
						}}
				\end{center}
			\item<4-> Quelle est le nombre maximal de comparaison lors de la recherche d'un élément dans cet arbre ?
			\item<5-> Construire un {\sc abr} contenant les valeurs $2, 9, 10, 17$ et $21$ et de hauteur minimale. Même question avec la hauteur maximale.
		\end{enumerate}
	\end{block}
\end{frame}


\begin{frame}[fragile]{\Ctitle}{\stitle}
	\begin{block}{Complexité}
		La complexité des opérations d'insertion et de recherche dans un {\sc abr} est majorée par la hauteur $h$ de l'arbre. \onslide<2->\textcolor{gray}{ On descend d'un niveau dans l'arbre à chaque comparaison et la profondeur d'un noeud est inférieure à $h$.}\\
		\onslide<3->{Or on sait que $ h+1 \leqslant n \leqslant 2^{h+1}-1$, et les deux bornes sont atteintes}
		\begin{itemize}
			\item<4-> Dans le cas d'un peigne ($n=h+1$) les opérations seront en $\mathcal{O}(n)$.
			\item<5-> Dans le cas d'un arbre complet ($n=2^{h+1}-1$), les opérations seront en $\mathcal{O}(\log(n))$.
		\end{itemize}
	\end{block}
	\onslide<6->{
		\begin{alertblock}{Définition}
			Soit $S$, un ensemble d'abres binaires. On dit que les arbres de $S$ sont \textcolor{blue}{équilibrés} s'il existe une constante $C$ telle que, pour tout arbre $s \in S$ :
			$$ h(s) \leqslant C \log(n(s))$$
		\end{alertblock}}
\end{frame}


\begin{frame}[fragile]{\Ctitle}{\stitle}
	\begin{block}{Rotation d'un {\sc abr}}
		On considère l'{\sc abr} suivant où $u$ et $v$ sont les étiquettes des noeuds représentés et $t_1$, $t_2$, $t_3$ des arbres binaires :
		\onslide<2->{\begin{center}\pstree[arrows=->,treesep=0.5cm,levelsep=0.7cm]{\TCircle[radius=0.25cm]{$v$}}{
					\pstree{\TCircle[radius=0.25cm]{$u$}}{
						\Tr{$t_1$}
						\Tr{$t_2$}}
					\Tr{$t_3$}
				}
			\end{center}}
		\onslide<3->{La \textcolor{blue}{rotation droite} de cet arbre, consiste à réorganiser les noeuds \textit{en conservant la propriété d'{\sc abr}} de la façon suivante :}
		\onslide<4->{
			\begin{center}
				\pstree[arrows=->,treesep=0.5cm,levelsep=0.7cm]{\TCircle[radius=0.25cm]{$u$}}{
					\Tr{$t_1$}
					\pstree{\TCircle[radius=0.25cm]{$v$}}{
						\Tr{$t_2$}
						\Tr{$t_3$}
					}
				}
			\end{center}}
		\onslide<5->{De façon symétrique, la \textcolor{blue}{rotation gauche} consiste en partant de cet arbre à revenir à l'arbre initial.}
	\end{block}
\end{frame}

\begin{frame}[fragile]{\Ctitle}{\stitle}
	\begin{exampleblock}{Exemple}
		On considère l'arbre binaire suivant :
		\begin{center}\pstree[arrows=->,treesep=0.5cm,levelsep=0.7cm]{\TCircle[radius=0.25cm]{$7$}}{
				\pstree{\TCircle[radius=0.25cm]{$3$}}{
					\TCircle[radius=0.25cm]{$2$}
					\pstree{\Tcircle[radius=0.25cm]{$5$}}
					{
						\TCircle[radius=0.25cm]{$4$}
						\TCircle[radius=0.25cm]{$6$}
					}
				}
				\TCircle[radius=0.25cm]{$9$}
			}
		\end{center}
		\begin{enumerate}
			\item<2-> Vérifier qu'il s'agit d'un {\sc abr}
			\item<3-> Montrer qu'un utilisant des rotations, on peut transformer cet arbre en un arbre binaire parfait.
		\end{enumerate}
	\end{exampleblock}
\end{frame}

\begin{frame}[fragile]{\Ctitle}{\stitle}
	\begin{exampleblock}{Correction}
		\begin{center}
			\pstree[arrows=->,treesep=0.7cm,levelsep=1cm]{\TCircle[radius=0.25cm]{$7$}}{
				\pstree{\TCircle[linecolor=BrickRed,linewidth=1pt,radius=0.25cm]{\textcolor{BrickRed}{$3$}} \nput[labelsep=1 pt]{0}{\pssucc}{\textcolor{BrickRed}{$u$}}}{
					\TCircle[name=D,radius=0.25cm]{$\ 2\ $}
					\ncbox[linecolor=gray,nodesep=0.1,boxsize=0.3,linestyle=dashed]{D}{D}
					\nbput[labelsep=0]{\textcolor{gray}{$\scriptstyle t_1$}}
					\pstree{\Tcircle[linecolor=BrickRed,linewidth=1pt,radius=0.25cm]{\textcolor{BrickRed}{$5$}} \nput[labelsep=1 pt]{0}{\pssucc}{\textcolor{BrickRed}{$v$}}}
					{
						\TCircle[name=Q,radius=0.25cm]{$4$}
						\ncbox[linecolor=gray,nodesep=0.1,boxsize=0.3,linestyle=dashed]{Q}{Q}
						\nbput[labelsep=0]{\textcolor{gray}{$\scriptstyle t_2$}}
						\TCircle[name=S,radius=0.25cm]{$6$}
						\ncbox[linecolor=gray,nodesep=0.1,boxsize=0.3,linestyle=dashed]{S}{S}
						\nbput[labelsep=0]{\textcolor{gray}{$\scriptstyle t_3$}}
					}
				}
				\TCircle[radius=0.25cm]{$9$}
			} \hspace{1.5cm}
			\onslide<3->{
				\pstree[arrows=->,treesep=0.7cm,levelsep=1cm]{\TCircle[radius=0.25cm]{$7$}}{
					\pstree{\TCircle[linecolor=BrickRed,linewidth=1pt,radius=0.25cm]{\textcolor{BrickRed}{$5$}} \nput[labelsep=1 pt]{0}{\pssucc}{\textcolor{BrickRed}{$v$}}}{
						\pstree{\Tcircle[linecolor=BrickRed,linewidth=1pt,radius=0.25cm]{\textcolor{BrickRed}{$3$}} \nput[labelsep=1 pt]{0}{\pssucc}{\textcolor{BrickRed}{$u$}}}
						{
							\TCircle[name=D,radius=0.25cm]{$2$}
							\ncbox[linecolor=gray,nodesep=0.1,boxsize=0.3,linestyle=dashed]{D}{D}
							\nbput[labelsep=0]{\textcolor{gray}{$\scriptstyle t_1$}}
							\TCircle[name=Q,radius=0.25cm]{$4$}
							\ncbox[linecolor=gray,nodesep=0.1,boxsize=0.3,linestyle=dashed]{Q}{Q}
							\nbput[labelsep=0]{\textcolor{gray}{$\scriptstyle t_2$}}
						}
						\TCircle[name=S,radius=0.25cm]{$6$}
						\ncbox[linecolor=gray,nodesep=0.1,boxsize=0.3,linestyle=dashed]{S}{S}
						\nbput[labelsep=0]{\textcolor{gray}{$\scriptstyle t_3$}}
					}
					\TCircle[radius=0.25cm]{$9$}
				}} \vspace{0.2cm}\\

			\vspace{1cm}
		\end{center}
	\end{exampleblock}
\end{frame}



\begin{frame}{\Ctitle}{\stitle}
	\begin{exampleblock}{Correction}
		\begin{center}
			\pstree[arrows=->,treesep=0.7cm,levelsep=1cm]{\TCircle[linecolor=BrickRed,linewidth=1pt,radius=0.25cm]{\textcolor{BrickRed}{$7$}} \nput[labelsep=1 pt]{0}{\pssucc}{\textcolor{BrickRed}{$u$}}}{
				\pstree{\TCircle[linecolor=BrickRed,linewidth=1pt,radius=0.25cm]{\textcolor{BrickRed}{$5$}} \nput[labelsep=1 pt]{0}{\pssucc}{\textcolor{BrickRed}{$v$}}}{
					\pstree{\Tcircle[radius=0.25cm]{$3$}}
					{
						\TCircle[name=D,radius=0.25cm]{$2$}
						\TCircle[name=Q,radius=0.25cm]{$4$}

					}
					\psframe[linecolor=gray,linestyle=dashed](0.9,0.25)(-1.1,-1.3)
					\nbput[labelsep=1.6cm]{\textcolor{gray}{$\scriptstyle t_1$}}
					\TCircle[name=S,radius=0.25cm]{$6$}
					\ncbox[linecolor=gray,nodesep=0.1,boxsize=0.3,linestyle=dashed]{S}{S}
					\nbput[labelsep=0]{\textcolor{gray}{$\scriptstyle t_2$}}
				}
				\TCircle[name=N,radius=0.25cm]{$9$}
				\ncbox[linecolor=gray,nodesep=0.1,boxsize=0.3,linestyle=dashed]{N}{N}
				\nbput[labelsep=0]{\textcolor{gray}{$\scriptstyle t_3$}}
			} \hspace{1.5cm}
			\onslide<2->
			{
				\pstree[arrows=->,treesep=0.7cm,levelsep=1cm]{\TCircle[radius=0.25cm]{$5$}}
				{
					\pstree[treesep=0.7cm,levelsep=1cm]{\TCircle[radius=0.25cm]{$3$}}{
						\TCircle[radius=0.25cm]{$2$}
						\TCircle[radius=0.25cm]{$4$}
					}
					\pstree[treesep=0.7cm,levelsep=1cm]{\TCircle[radius=0.25cm]{$7$}}{
						\TCircle[radius=0.25cm]{$6$}
						\TCircle[radius=0.25cm]{$9$}
					}
				}
			} \vspace{0.5cm}
		\end{center}
	\end{exampleblock}
	\onslide<3->{
		\begin{block}{Equilibrage d'un abre binaire}
			Les rotations droite et gauche sont les opérations permettant de maintenir un certain équilibre dans un {\sc abr}. Et donc de \textcolor{blue}{garantir une complexité logarithmique} des opérations usuelles. Parmi les nombreuses possibilités d'{\sc abr} équilibrés, nous allons détailler les \textcolor{BrickRed}{arbres rouge-noir}.
		\end{block}
	}
\end{frame}

\makess{Arbres rouge-noir}
\begin{frame}[fragile]{\Ctitle}{\stitle}
	\begin{alertblock}{Définition des arbres rouge-noir}
		Un \textcolor{blue}{arbre rouge-noir} $t$ est un {\sc abr} (\textcolor{OliveGreen}{\ding{182}}), dans lequel chaque noeud porte une information de couleur (rouge ou noir), et ayant les deux propriétés suivantes :
		\begin{itemize}
			\item<1-> le père d'un noeud rouge est noir (\textcolor{OliveGreen}{\ding{183}}),
			\item<2-> le nombre de noeuds noirs le long d'un chemin de la racine à un sous arbre vide est toujours le même (\textcolor{OliveGreen}{\ding{184}}), on appellera \textcolor{blue}{hauteur noire} de $t$ et on notera $b(t)$ cette quantité .
		\end{itemize}
	\end{alertblock}
	\onslide<3->{
		\begin{exampleblock}{Exemples}
			\renewcommand{\arraystretch}{1.1}
			\begin{tabularx}{\textwidth}{Y|Y|Y}
				\pstree[arrows=->,treesep=0.5cm,levelsep=0.7cm]{\NN{$8$}}{
					\pstree{\NR{$3$}}{
						\NN{$1$}
						\pstree{\NN{$6$}}
						{
							\NR{$5$}
							\NR{$7$}
						}
					}
					\NN{$9$}
				}                                                                                                                                &
				\pstree[arrows=->,treesep=0.5cm,levelsep=0.7cm]{\NN{$5$}}
				{
					\pstree[treesep=0.5cm,levelsep=0.7cm]{\NR{$2$}}{
						\NN{$1$}
						\pstree{\NR{$3$}}
						{
							\Tn{}
							\NN{$4$}

						}
					}
					\pstree[treesep=0.5cm,levelsep=0.7cm]{\NR{$8$}}{
						\NN{$7$}
						\Tn{}
					}
				}
				                                                                                                                                 &
				\pstree[arrows=->,treesep=0.5cm,levelsep=0.7cm]{\NN{$7$}}
				{
					\pstree[treesep=0.5cm,levelsep=0.7cm]{\NN{$2$}}{
						\Tn{}
						\Tn{}
					}
					\pstree[treesep=0.5cm,levelsep=0.7cm]{\NR{$8$}}{
						\NN{$6$}
						\NN{9}
					}
				}
				\\
				\leavevmode\onslide<3->{\textcolor{OliveGreen}{\small \ding{182} \faCheck \quad \ding{183} \faCheck  \quad \ding{184} \faCheck}} & \leavevmode\onslide<4->{\textcolor{OliveGreen}{\small \ding{182} \faCheck \quad  \ding{183}} \textcolor{BrickRed}{\small \faTimesCircle} \textcolor{OliveGreen}{\small \quad \ding{184} \faCheck}} & \leavevmode\onslide<5->{\textcolor{OliveGreen}{\small \ding{182}}  \textcolor{BrickRed}{\small \faTimesCircle} \textcolor{OliveGreen}{\small \quad\ding{183} \faTimesCircle \quad \ding{184} \faCheck}} \\
			\end{tabularx}

		\end{exampleblock}
	}
\end{frame}

\begin{frame}[fragile]{\Ctitle}{\stitle}
	\begin{block}{Proriété d'équilibre des arbres rouge-noirs}
		Pour tout arbre rouge noir $t$ :
		\begin{itemize}
			\item<2-> $h(t) \leqslant 2b(t)$
			\item<3-> $2^{b(t)} \leqslant n(t) +1$
		\end{itemize}
		\onslide<4->{Conséquence : les arbres rouge-noir forment un ensemble d'arbres équilibrés.}
	\end{block}
	\onslide<5->{\begin{block}{Implémentation}
			Une implémentation en OCaml sera vue en TP, les opérations d'insertion et de suppression sont difficiles et reposent sur les rotations droite et gauche des {\sc abr}.
		\end{block}}
\end{frame}

\makess{Arbres binaires stricts}
\begin{frame}[fragile]{\Ctitle}{\stitle}
	\begin{block}{Définition}
		Un arbre binaire est \textcolor{blue}{strict} si tous ses noeuds ont soit 0, soit 2 fils.
		Un noeud est donc soit une feuille (pas de fils), soit un noeud interne (deux fils).
	\end{block}
	\onslide<2->{
		\begin{exampleblock}{Exemples}
			Les arbres de codage de Huffman sont des arbres binaires stricts. Qu'on peut représenter en OCaml par le type :
			\inputpartOCaml{\SPATH/abs.ml}{}{}{1}{3}
			Ici les noeuds internes ne portent pas d'information.
		\end{exampleblock}}
\end{frame}

\begin{frame}[fragile]{\Ctitle}{\stitle}
	\begin{exampleblock}{Exercice}
		\begin{enumerate}
			\item Donner une définition inductive des arbres binaires stricts.
			\item Ecrire en OCaml la fonction renvoyant la taille d'un arbre binaire strict.
			\item On note $n$ le nombre de noeuds internes d'un arbres binaire strict et $f$ son nombre de feuilles. Montrer que $f = n +1$. \\
			      \textcolor{OliveGreen}{\small \aide \;} On pourra raisonner par récurrence sur la taille de l'arbre ou dénombrer de deux façon différentes les noeuds ayant un père.

		\end{enumerate}
	\end{exampleblock}
\end{frame}

\makess{Arbres}
\begin{frame}[fragile]{\Ctitle}{\stitle}
	\begin{alertblock}{Définition}
		Un \textcolor{blue}{arbre} est un ensemble de $n \geqslant 1$ noeuds structurés de la manière suivante :
		\begin{itemize}
			\item un noeud particulier $r$ est appelé la \textit{racine} de l'arbre,
			\item les $n-1$ autres noeuds sont partitionnés en $k \geqslant 0$ sous ensembles disjoints qui forment autant d'arbres, appelés \textit{sous-arbres} de $r$,
			\item la racine $r$ est relié à la racine de chacun de ces sous-arbres par une arête.
		\end{itemize}
	\end{alertblock}
	\begin{exampleblock}{Exemple}
		\begin{center}
			\pstree{\TCircle{A}}{
				\TCircle{B}
				\pstree{\TCircle{C}}{
					\TCircle{E}
					\TCircle{F}
				}
				\pstree{\TCircle{D}}{
					\TCircle{G}
					\TCircle{H}
					\TCircle{I}
				}
				\pstree{\TCircle{J}}{
					\TCircle{K}
				}
			}
		\end{center}
	\end{exampleblock}
\end{frame}

\begin{frame}[fragile]{\Ctitle}{\stitle}
	\begin{block}{Remarques}
		\begin{itemize}
		\item La séquence des sous arbres d'un noeud est appelée \textcolor{blue}{forêt}.
		\item Un arbre réduit à un seul noeud est appelé \textcolor{blue}{feuille}.
		\item \textcolor{BrickRed}{\small \danger \;} Un arbre binaire n'est \textcolor{BrickRed}{\textit{pas}} un arbre. En effet :
		\begin{itemize}
		\item  un arbre binaire peut être vide (et pas un arbre);
		\item dans un arbre binaire on distingue le fils gauche du fils droit, c'est à dire que les deux arbres binaires ci-dessous sont différents :
			\begin{center}
				\pstree{\TCircle{A}}{\pstree{\TCircle{B}}{\Tr{$\varnothing$} \Tr{$\varnothing$}} \Tr{$\varnothing$}} \qquad \pstree{\TCircle{A}}{\Tr{$\varnothing$} \pstree{\TCircle{B}}{\Tr{$\varnothing$} \Tr{$\varnothing$}}}
			\end{center}
			Alors que le seul \textit{arbre} ayant deux noeuds est :
			\begin{center}\pstree{\TCircle{A}}{\TCircle{B}}\end{center}
		\end{itemize}
		\end{itemize}
	\end{block}
\end{frame}

\makess{Représentation en machine}
\begin{frame}[fragile]{\Ctitle}{\stitle}
	\begin{block}{Représentation en OCaml}
		\inputpartOCaml{\SPATH/arbres.ml}{}{}{1}{2}
	\end{block}
	\begin{exampleblock}{Exercice}
		Ecrire une fonction {\tt size : 'a tree -> int} qui renvoie le nombre de noeuds d'un arbre. \\
		\textcolor{OliveGreen}{\small \aide \;} On pourra utiliser deux fonctions \textit{mutuellement récursive} qui renvoient respectivement la taille d'un arbre et celle d'une forêt.
	\end{exampleblock}
\end{frame}

\begin{frame}[fragile]{\Ctitle}{\stitle}
	\begin{block}{Représentation en C}
		\begin{pspicture}(-5,2.8)(5,6.2)
				\rput(0,6){\Circlenode{A}{A}}
				\rput(-1,5){\Circlenode{B}{B}}
				\rput(1,5){\Circlenode{C}{C}}
				\rput(-1,4){\Circlenode{D}{D}}
				\rput(0,4){\Circlenode{E}{E}}
				\rput(1,4){\Circlenode{F}{F}}
				\rput(2,4){\Circlenode{G}{G}}
				\rput(1,3){\Circlenode{J}{J}}
				\rput(-1.5,3){\Circlenode{H}{H}}
				\rput(-0.5,3){\Circlenode{I}{I}}
				\ncline{-}{A}{B}
				\ncline{-}{A}{C}
				\ncline{-}{B}{D}
				\ncline{-}{C}{E}
				\ncline{-}{C}{F}
				\ncline{-}{C}{G}
				\ncline{-}{D}{H}
				\ncline{-}{D}{I}
				\ncline{-}{F}{J}
		\end{pspicture}\\
		\onslide<2->{Principe : utiliser un pointeur vers le premier fils et un pointeur vers le frère suivant. En anglais, \textcolor{blue}{\sc lcrs} : \textit{left child right sibling}.}
	\end{block}
\end{frame}

\begin{frame}[fragile]{\Ctitle}{\stitle}
	\begin{block}{Représentation en C}
		En notant en :
		\begin{itemize}
		\item  \textcolor{BrickRed}{$\longrightarrow$} le pointeur vers le  fils gauche,
		\item  \textcolor{blue}{$\longrightarrow$} le pointeur vers le frère suivant,
		\item $\bot$ le pointeur {\sc null}.
		\end{itemize}
		\begin{center}
		\begin{pspicture}(-5,2)(5,6.2)
				\rput(0,6){\Circlenode{A}{A}} \rput(0.7,6){\rnode{NA}{$\bot$}}
				\rput(-1.5,5){\Circlenode{B}{B}}
				\rput(1,5){\Circlenode{C}{C}} \rput(1.7,5){\rnode{NC}{$\bot$}}
				\rput(-1.5,4){\Circlenode{D}{D}} \rput(-0.8,4){\rnode{ND}{$\bot$}}
				\rput(0,4){\Circlenode{E}{E}} \rput(0,3.3){\rnode{VE}{$\bot$}}
				\rput(1,4){\Circlenode{F}{F}} 
				\rput(2,4){\Circlenode{G}{G}} \rput(2.7,4){\rnode{NG}{$\bot$}} \rput(2,3.3){\rnode{VG}{$\bot$}}
				\rput(1,3){\Circlenode{J}{J}} \rput(1.7,3){\rnode{NJ}{$\bot$}} \rput(1,2.3){\rnode{VJ}{$\bot$}}
				\rput(-2,3){\Circlenode{H}{H}}
				\rput(-2,2.3){\rnode{VH}{$\bot$}} 
				\rput(-1,3){\Circlenode{I}{I}} \rput(-0.3,3){\rnode{NI}{$\bot$}} \rput(-1,2.3){\rnode{VI}{$\bot$}} 
				\ncline[linecolor=BrickRed,linewidth=0.03]{->}{A}{B}
				\ncline[linecolor=blue, linewidth=0.03]{->}{A}{NA}
				\ncline[linecolor=blue, linewidth=0.03]{->}{C}{NC}
				\ncline[linecolor=blue, linewidth=0.03]{->}{D}{ND}
				\ncline[linecolor=blue, linewidth=0.03]{->}{G}{NG}
				\ncline[linecolor=blue, linewidth=0.03]{->}{J}{NJ}
				\ncline[linecolor=BrickRed, linewidth=0.03]{->}{H}{VH}
				\ncline[linecolor=BrickRed, linewidth=0.03]{->}{I}{VI}
				\ncline[linecolor=BrickRed, linewidth=0.03]{->}{J}{VJ}
				\ncline[linecolor=BrickRed, linewidth=0.03]{->}{E}{VE}
				\ncline[linecolor=BrickRed, linewidth=0.03]{->}{G}{VG}
				\ncline[linecolor=blue, linewidth=0.03]{->}{I}{NI}
				\ncline[linecolor=blue, linewidth=0.03]{->}{B}{C}
				\ncline[linecolor=blue, linewidth=0.03]{->}{H}{I}
				\ncline[linecolor=blue, linewidth=0.03]{->}{E}{F}
				\ncline[linecolor=blue, linewidth=0.03]{->}{F}{G}
				\ncline[linecolor=BrickRed,linewidth=0.03]{->}{B}{D}
				\ncline[linecolor=BrickRed,linewidth=0.03]{->}{C}{E}
				\ncline[linecolor=BrickRed,linewidth=0.03]{->}{D}{H}
				\ncline[linecolor=BrickRed,linewidth=0.03]{->}{F}{J}
		\end{pspicture}
	\end{center}
	\end{block}
\end{frame}


\begin{frame}[fragile]{\Ctitle}{\stitle}
	\begin{block}{Représentation en C}
	\inputpartC{\SPATH/arbres.c}{}{\small}{6}{13}	
	\end{block}
	\onslide<2->{
	\begin{exampleblock}{Exercice}
		\begin{enumerate}
		\item Ecrire une fonction de signature \mintinline{c}{tree create_tree(int value)} qui renvoie un arbre réduit à un noeud de valeur {\tt value}.
		\item Ecrire une fonction de signature \mintinline{c}{int size(tree t)} qui renvoie le nombre de noeuds d'un arbre. 
		\end{enumerate}
	\end{exampleblock}}
\end{frame}

\makess{Conversion entre arbre et arbre binaire}
\begin{frame}[fragile]{\Ctitle}{\stitle}
	\begin{block}{Remarque}
		\begin{tabularx}{\textwidth}{X|X}
			Arbres & Arbres binaires \\
		\inputpartC{\SPATH/arbres.c}{}{\footnotesize}{6}{13} &  \inputpartC{\SPATH/abr.c}{}{\footnotesize}{4}{11} \\
		\end{tabularx}
		\onslide<2->{On constate que les types représentant les arbres et les arbres binaires sont identiques, il y a un \textit{isomorphisme naturel} entre les arbres et les arbres binaires. On peut convertir un arbre en arbre binaire et inversement}
	\end{block}
\end{frame}

\begin{frame}[fragile]{\Ctitle}{\stitle}
	\begin{exampleblock}{Exercice}
		Donner la représentation \textit{\sc lcrs} de l'arbre binaire suivant (on n'a pas fait figuré les sous arbres vides) puis sa représentation sous la forme d'arbre généralisé :\smallskip\\
		\begin{center}
			\begin{pspicture}(-5,1.3)(5,6.2)
			\rput(0,6){\Circlenode{A}{A}}
			\rput(-1,5){\Circlenode{B}{B}}
			\rput(-4,4){\Circlenode{C}{C}}
			\rput(2,4){\Circlenode{D}{D}}
			\rput(1,3){\Circlenode{G}{G}}
			\rput(0,2){\Circlenode{L}{L}}
			\rput(3,3){\Circlenode{H}{H}}
			\rput(4,2){\Circlenode{M}{M}}
			\rput(-5,3){\Circlenode{E}{E}}
			\rput(-4,2){\Circlenode{I}{I}}
			\rput(-2,3){\Circlenode{F}{F}}
			\rput(-3,2){\Circlenode{J}{J}}
			\rput(-1,2){\Circlenode{K}{K}}
			\ncline{->}{A}{B}
			\ncline{->}{B}{C}
			\ncline{->}{B}{D}
			\ncline{->}{C}{E}
			\ncline{->}{C}{F}
			\ncline{->}{E}{I}
			\ncline{->}{F}{J}
			\ncline{->}{F}{K}
			\ncline{->}{D}{G}
			\ncline{->}{D}{H}
			\ncline{->}{G}{L}
			\ncline{->}{H}{M}
			\end{pspicture}
		\end{center}
	\end{exampleblock}
\end{frame}


\begin{frame}[fragile]{\Ctitle}{\stitle}
	\begin{exampleblock}{Exercice}
		Représentation \textit{\sc lcrs}  :\smallskip\\
		\begin{center}
			\begin{pspicture}(-5,1.3)(5,6.2)
			\rput(0,6){\Circlenode{A}{A}}
			\rput(-1,5){\Circlenode{B}{B}}
			\rput(-4,4){\Circlenode{C}{C}}
			\rput(2,4){\Circlenode{D}{D}}
			\rput(1,3){\Circlenode{G}{G}}
			\rput(0,2){\Circlenode{L}{L}}
			\rput(3,3){\Circlenode{H}{H}}
			\rput(4,2){\Circlenode{M}{M}}
			\rput(-5,3){\Circlenode{E}{E}}
			\rput(-4,2){\Circlenode{I}{I}}
			\rput(-2,3){\Circlenode{F}{F}}
			\rput(-3,2){\Circlenode{J}{J}}
			\rput(-1,2){\Circlenode{K}{K}}
			\ncline[linecolor=BrickRed, linewidth=0.03]{->}{A}{B}
			\ncline[linecolor=BrickRed, linewidth=0.03]{->}{B}{C}
			\ncline[linecolor=blue, linewidth=0.03]{->}{B}{D}
			\ncline[linecolor=BrickRed, linewidth=0.03]{->}{C}{E}
			\ncline[linecolor=blue, linewidth=0.03]{->}{C}{F}
			\ncline[linecolor=blue, linewidth=0.03]{->}{E}{I}
			\ncline[linecolor=BrickRed, linewidth=0.03]{->}{F}{J}
			\ncline[linecolor=blue, linewidth=0.03]{->}{F}{K}
			\ncline[linecolor=BrickRed, linewidth=0.03]{->}{D}{G}
			\ncline[linecolor=blue, linewidth=0.03]{->}{D}{H}
			\ncline[linecolor=BrickRed, linewidth=0.03]{->}{G}{L}
			\ncline[linecolor=blue, linewidth=0.03]{->}{H}{M}
			\end{pspicture}
		\end{center}
	\end{exampleblock}
\end{frame}


\begin{frame}[fragile]{\Ctitle}{\stitle}
	\begin{exampleblock}{Exercice}
		Conversion en arbre généralisé  :\smallskip\\
		\begin{center}
			\begin{pspicture}(-5,1.3)(5,6.2)
			\rput(0,6){\Circlenode{A}{A}}
			\rput(-1,4.5){\Circlenode{B}{B}}
			\rput(-4,3.5){\Circlenode{C}{C}}
			\rput(2,4.5){\Circlenode{D}{D}}
			\rput(2,3.5){\Circlenode{G}{G}}
			\rput(2,2.5){\Circlenode{L}{L}}
			\rput(3,4.5){\Circlenode{H}{H}}
			\rput(4,4.5){\Circlenode{M}{M}}
			\rput(-5,2.5){\Circlenode{E}{E}}
			\rput(-4,2.5){\Circlenode{I}{I}}
			\rput(-2,3.5){\Circlenode{F}{F}}
			\rput(-2,2.5){\Circlenode{J}{J}}
			\rput(-1,3.5){\Circlenode{K}{K}}
			\ncline{->}{A}{B}
			\ncline{->}{A}{D}
			\ncline{->}{A}{H}
			\ncline{->}{A}{M}
			\ncline{->}{B}{C}
			\ncline{->}{B}{F}
			\ncline{->}{B}{K}
			\ncline{->}{C}{E}
			\ncline{->}{C}{I}
			\ncline{->}{F}{J}
			\ncline{->}{D}{G}
			\ncline{->}{G}{L}
			\end{pspicture}
		\end{center}
	\end{exampleblock}
\end{frame}

\begin{frame}[fragile]{\Ctitle}{\stitle}
	\begin{exampleblock}{Exercice}
		Donner la représentation \textit{\sc lcrs} de l'arbre  suivant  puis sa représentation sous la forme d'arbre binaire :\smallskip\\
		\begin{center}
			\begin{pspicture}(-5,1.3)(5,6.2)
			\rput(0,6){\Circlenode{A}{A}}
			\rput(-2,5){\Circlenode{B}{B}}
			\rput(2,5){\Circlenode{C}{C}}
			\rput(-1,4){\Circlenode{E}{E}}
			\rput(2,4){\Circlenode{F}{F}}
			\rput(1,3){\Circlenode{J}{J}}
			\rput(3,3){\Circlenode{K}{K}}
			\rput(-3,4){\Circlenode{D}{D}}
			\rput(-2,3){\Circlenode{G}{G}}
			\rput(-1,3){\Circlenode{H}{H}}
			\rput(-0,3){\Circlenode{I}{I}}
			\ncline{->}{A}{B}
			\ncline{->}{A}{C}
			\ncline{->}{B}{D}
			\ncline{->}{C}{F}
			\ncline{->}{E}{G}
			\ncline{->}{E}{H}
			\ncline{->}{E}{I}
			\ncline{->}{B}{E}
			\ncline{->}{F}{J}
			\ncline{->}{F}{K}
			\end{pspicture}
		\end{center}
	\end{exampleblock}
\end{frame}



\begin{frame}[fragile]{\Ctitle}{\stitle}
	\begin{exampleblock}{Exercice}
		Représentation \textit{\sc lcrs}  :\smallskip\\
		\begin{center}
			\begin{pspicture}(-5,1.3)(5,6.2)
			\rput(0,6){\Circlenode{A}{A}}
			\rput(-2,5){\Circlenode{B}{B}}
			\rput(2,5){\Circlenode{C}{C}}
			\rput(-1,4){\Circlenode{E}{E}}
			\rput(2,4){\Circlenode{F}{F}}
			\rput(1,3){\Circlenode{J}{J}}
			\rput(3,3){\Circlenode{K}{K}}
			\rput(-3,4){\Circlenode{D}{D}}
			\rput(-2,3){\Circlenode{G}{G}}
			\rput(-1,3){\Circlenode{H}{H}}
			\rput(-0,3){\Circlenode{I}{I}}
			\ncline[linecolor=BrickRed, linewidth=0.03]{->}{A}{B}
			\ncline[linecolor=blue, linewidth=0.03]{->}{B}{C}
			\ncline[linecolor=BrickRed, linewidth=0.03]{->}{B}{D}
			\ncline[linecolor=BrickRed, linewidth=0.03]{->}{C}{F}
			\ncline[linecolor=BrickRed, linewidth=0.03]{->}{E}{G}
			\ncline[linecolor=blue, linewidth=0.03]{->}{D}{E}
			\ncline[linecolor=blue, linewidth=0.03]{->}{G}{H}
			\ncline[linecolor=blue, linewidth=0.03]{->}{H}{I}
			\ncline[linecolor=BrickRed, linewidth=0.03]{->}{F}{J}
			\ncline[linecolor=blue, linewidth=0.03]{->}{J}{K}
			\end{pspicture}
		\end{center}
	\end{exampleblock}
\end{frame}



\begin{frame}[fragile]{\Ctitle}{\stitle}
	\begin{exampleblock}{Exercice}
		Conversion en arbre binaire :\smallskip\\
		\begin{center}
			\begin{pspicture}(-5,-0.2)(5,6.2)
			\rput(0,6){\Circlenode{A}{A}}
			\rput(-2,5){\Circlenode{B}{B}}
			\rput(1,4){\Circlenode{C}{C}}
			\rput(-4,4){\Circlenode{D}{D}}
			\rput(-3,3){\Circlenode{E}{E}}
			\rput(0,3){\Circlenode{F}{F}}
			\rput(-1,2){\Circlenode{J}{J}}
			\rput(0,1){\Circlenode{K}{K}}
			\rput(-4,2){\Circlenode{G}{G}}
			\rput(-3,1){\Circlenode{H}{H}}
			\rput(-2,0){\Circlenode{I}{I}}
			\ncline[linecolor=BrickRed, linewidth=0.03]{->}{A}{B}
			\ncline[linecolor=blue, linewidth=0.03]{->}{B}{C}
			\ncline[linecolor=BrickRed, linewidth=0.03]{->}{B}{D}
			\ncline[linecolor=BrickRed, linewidth=0.03]{->}{C}{F}
			\ncline[linecolor=BrickRed, linewidth=0.03]{->}{E}{G}
			\ncline[linecolor=blue, linewidth=0.03]{->}{D}{E}
			\ncline[linecolor=blue, linewidth=0.03]{->}{G}{H}
			\ncline[linecolor=blue, linewidth=0.03]{->}{H}{I}
			\ncline[linecolor=BrickRed, linewidth=0.03]{->}{F}{J}
			\ncline[linecolor=blue, linewidth=0.03]{->}{J}{K}
			\end{pspicture}
		\end{center}
	\end{exampleblock}
\end{frame}

\makess{Parcours d'un arbre généralisé}
\begin{frame}[fragile]{\Ctitle}{\stitle}
	\begin{block}{Parcours}
	La notion de parcours \textit{infixe} n'a plus de sens cependant, pour un arbre généralisé, on peut définir les parcours suivants :
		\begin{itemize}
			\item<2-> \textcolor{blue}{prefixe} : on visite le noeud avant ses fils,
			\item<2-> \textcolor{blue}{suffixe} : on visite le noeud après ses fils,
			\item<2-> \textcolor{blue}{en largeur} : on visite les noeuds par niveau.
		\end{itemize}	
	\end{block}
	\begin{exampleblock}{Exemples}
	Donner les parcours prefixe, suffixe et en largeur de l'arbre généralisé 
		\begin{center}
			\begin{pspicture}(-5,2.8)(5,6.2)
			\rput(0,6){\Circlenode{A}{A}}
			\rput(-2,5.2){\Circlenode{B}{B}}
			\rput(2,5.2){\Circlenode{C}{C}}
			\rput(-1,4.4){\Circlenode{E}{E}}
			\rput(2,4.4){\Circlenode{F}{F}}
			\rput(1,3.6){\Circlenode{J}{J}}
			\rput(3,3.6){\Circlenode{K}{K}}
			\rput(-3,4.4){\Circlenode{D}{D}}
			\rput(-2,3.6){\Circlenode{G}{G}}
			\rput(-1,3.6){\Circlenode{H}{H}}
			\rput(-0,3.6){\Circlenode{I}{I}}
			\ncline{->}{A}{B}
			\ncline{->}{A}{C}
			\ncline{->}{B}{D}
			\ncline{->}{C}{F}
			\ncline{->}{E}{G}
			\ncline{->}{E}{H}
			\ncline{->}{E}{I}
			\ncline{->}{B}{E}
			\ncline{->}{F}{J}
			\ncline{->}{F}{K}
			\end{pspicture}
		\end{center}
	\end{exampleblock}
\end{frame}

\makess{Point de vue des graphes}
\begin{frame}[fragile]{\Ctitle}{\stitle}
	\begin{block}{Remarques}
	Un arbre généralisé est un graphe $(S,A)$ non orienté, connexe sans cycle.
	\begin{itemize}
		\item<2-> $S$ est l'ensemble des noeuds de l'arbre,
		\item<3-> $A$ est l'ensemble des arêtes de l'arbre, chaque arête est de la forme $(x,y)$ où $x$ est le père de $y$.
		\item<4-> \textcolor{blue}{Connexité} : il existe un chemin entre deux noeuds quelconques de l'arbre., $\forall (x,y) \in S^2, x \neq y$, il existe $x_1, \dots x_n$ tels que $\{x,x_1\}, \{x_1,x_2\}, \dots, \{x_{n-1},y\} \in A$.
		\item \textcolor{blue}{Acyclique}: il n'existe pas de cycle dans l'arbre, c'est à dire qu'il n'existe pas de suite de noeuds $x_1, \dots, x_n$ tels que $\{x_1,x_2\}, \{x_2,x_3\}, \dots, \{x_{n-1},x_n\}, \{x_n,x_1\} \in A$.
	\end{itemize}
	\textcolor{BrickRed}{\small \danger \;} On peut choisir n'importe quel noeud comme racine.
	\end{block}
\end{frame}


\makess{Sérialisation d'un arbre}
\begin{frame}[fragile]{\Ctitle}{\stitle}
	\begin{block}{Définition}
		La \textcolor{blue}{sérialisation} d'un arbre est une représentation de l'arbre sous forme de chaîne de caractères (à des fins de sauvegarde, transmission, reconstruction \dots). \\
		\onslide<2->{Un unique parcours (prefixe, infixe, suffixe ou en largeur) \textit{ne permet pas} de sérialiser un arbre, en effet des arbres différents peuvent avoir le même parcours.} \\
		\onslide<3->{Les solutions suivantes sont envisageables dans le cas d'un \textcolor{BrickRed}{un arbre binaire} :}
		\begin{itemize}
			\item<4-> stocker les parcours infixe et préfixe (mais cela impose de stocker les étiquettes en double);
			\item<5-> utiliser un parcours en largeur avec un marqueur spécial indiquant un fils absent;
			\item<6-> utiliser un parcours prefixe avec un marquer pour les fils absents
		\end{itemize}
	\end{block}
\end{frame}



\begin{frame}[fragile]{\Ctitle}{\stitle}
	\begin{exampleblock}{Exemple}
		\begin{center}		
			\pstree{\TCircle{9}}
		{\pstree{\TCircle{2}}
			{\TCircle{0}
				\pstree{\TCircle{6}}
				{\TCircle{5}
					\pstree{\TCircle{8}}
					{ 
						\TCircle{7}

						\Tn{} }}}\pstree{\TCircle{12}}
			{\TCircle{10}
				\TCircle{13}
			}}
		\end{center}
	\begin{itemize}
		\item Parcours préfixe et infixe :\\ \onslide<2->{\textcolor{OliveGreen}{{\tt 9 2 0 6 5 8 7 12 10 13 }} et \textcolor{OliveGreen}{\tt 0 2 5 6 7 8 9 10 12 13}}
		\item Parcours en largeur avec le marqueur {\tt \#} pour fils absent :\\ \onslide<3->{\textcolor{OliveGreen}{9 2 12 0 6 10 13 \# \# 5 8 \# \# \# \# \# \# 7}}
		\item Parcours préfixe avec le marqueur {\tt \#} pour fils absent : \\ \onslide<4->{\textcolor{OliveGreen}{9 2 0 \# \# 6 5 \# \# 8 7 \# 12 10 \# \# 13 \# \#}}
	\end{itemize}
	\end{exampleblock}

\end{frame}

\end{document}
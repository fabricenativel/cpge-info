\PassOptionsToPackage{dvipsnames,table}{xcolor}
\documentclass[10pt]{beamer}
\usepackage{Cours}

\begin{document}

\input{\detokenize{/home/fenarius/Travail/Cours/cpge-info/latex/MacrosCours.tex}}

% Numéro et titre de chapitre
\setcounter{numchap}{21}
\newcommand{\Ctitle}{\cnum {Compléments sur les arbres}}
\newcommand{\SPATH}{/home/fenarius/Travail/Cours/cpge-info/docs/mp2i/files/C\thenumchap/}
\newcommand{\NR}[1]{\TCircle[fillstyle=solid,fillcolor=red!20!white,linecolor=red,radius=0.25cm]{#1}}
\newcommand{\NN}[1]{\TCircle[fillstyle=solid,fillcolor=black!20!white,linecolor=black,radius=0.25cm]{#1}}
\psset{arrows=->,treesep=0.8cm,levelsep=0.8cm, radius=0.3cm}
\makess{Rappel}
\begin{frame}[fragile]{\Ctitle}{\stitle}
	\begin{block}{\textcolor{lightgray}{\small \rappel} Arbres binaires de recherche}
	\begin{enumerate}
	\item<1-> Rappeler les relations entre la hauteur $h$ et la taille $n$ d'un arbre binaire.
	\item<2-> Rappeler la définition d'un arbre binaire de recherche ({\sc abr}).
	\item<3-> L'arbre ci-dessous est-il un {\sc abr} ?
	\begin{center}
			\pstree{\TCircle{\tt 28}}
			{\pstree{\TCircle{\tt 16}}
				{\pstree{\TCircle{\tt 9}}
					{ \TCircle{7}
						\pstree{\TCircle{\tt 11}}
						{ \Tn{}
							\TCircle{\tt 13}
						}}\TCircle{\tt 21}
				}\pstree{\TCircle{\tt 37}}
				{
					\TCircle{\tt 34}
					\Tn{}
				}}
		\end{center}
	\item<4-> Quelle est le nombre maximal de comparaison lors de la recherche d'un élément dans cet arbre ?
	\item<5-> Construire un {\sc abr} contenant les valeurs $2, 9, 10, 17$ et $21$ et de hauteur minimale. Même question avec la hauteur maximale.
	\end{enumerate}
	\end{block}
\end{frame}


\begin{frame}[fragile]{\Ctitle}{\stitle}
	\begin{block}{Complexité}
		La complexité des opérations d'insertion et de recherche dans un {\sc abr} est majorée par la hauteur $h$ de l'arbre. \onslide<2->\textcolor{gray}{ On descend d'un niveau dans l'arbre à chaque comparaison et la profondeur d'un noeud est inférieure à $h$.}\\
		\onslide<3->{Or on sait que $ h+1 \leqslant n \leqslant 2^{h+1}-1$, et les deux bornes sont atteintes}
		\begin{itemize}
			\item<4-> Dans le cas d'un peigne ($n=h+1$) les opérations seront en $\mathcal{O}(n)$.
			\item<5-> Dans le cas d'un arbre complet ($n=2^{h+1}-1$), les opérations seront en $\mathcal{O}(\log(n))$.
		\end{itemize}
	\end{block}
	\onslide<6->{
		\begin{alertblock}{Définition}
			Soit $S$, un ensemble d'abres binaires. On dit que les arbres de $S$ sont \textcolor{blue}{équilibrés} s'il existe une constante $C$ telle que, pour tout arbre $s \in S$ :
			$$ h(s) \leqslant C \log(n(s))$$
		\end{alertblock}}
\end{frame}


\begin{frame}[fragile]{\Ctitle}{\stitle}
	\begin{block}{Rotation d'un {\sc abr}}
		On considère l'{\sc abr} suivant où $u$ et $v$ sont les étiquettes des noeuds représentés et $t_1$, $t_2$, $t_3$ des arbres binaires :
		\onslide<2->{\begin{center}\pstree[arrows=->,treesep=0.5cm,levelsep=0.7cm]{\TCircle[radius=0.25cm]{$v$}}{
					\pstree{\TCircle[radius=0.25cm]{$u$}}{
						\Tr{$t_1$}
						\Tr{$t_2$}}
					\Tr{$t_3$}
				}
			\end{center}}
		\onslide<3->{La \textcolor{blue}{rotation droite} de cet arbre, consiste à réorganiser les noeuds \textit{en conservant la propriété d'{\sc abr}} de la façon suivante :}
		\onslide<4->{
			\begin{center}
				\pstree[arrows=->,treesep=0.5cm,levelsep=0.7cm]{\TCircle[radius=0.25cm]{$u$}}{
					\Tr{$t_1$}
					\pstree{\TCircle[radius=0.25cm]{$v$}}{
						\Tr{$t_2$}
						\Tr{$t_3$}
					}
				}
			\end{center}}
		\onslide<5->{De façon symétrique, la \textcolor{blue}{rotation gauche} consiste en partant de cet arbre à revenir à l'arbre initial.}
	\end{block}
\end{frame}

\begin{frame}[fragile]{\Ctitle}{\stitle}
	\begin{exampleblock}{Exemple}
		On considère l'arbre binaire suivant :
		\begin{center}\pstree[arrows=->,treesep=0.5cm,levelsep=0.7cm]{\TCircle[radius=0.25cm]{$7$}}{
				\pstree{\TCircle[radius=0.25cm]{$3$}}{
					\TCircle[radius=0.25cm]{$2$}
					\pstree{\Tcircle[radius=0.25cm]{$5$}}
					{
						\TCircle[radius=0.25cm]{$4$}
						\TCircle[radius=0.25cm]{$6$}
					}
				}
				\TCircle[radius=0.25cm]{$9$}
			}
		\end{center}
		\begin{enumerate}
			\item<2-> Vérifier qu'il s'agit d'un {\sc abr}
			\item<3-> Montrer qu'un utilisant des rotations, on peut transformer cet arbre en un arbre binaire parfait.
		\end{enumerate}
	\end{exampleblock}
\end{frame}

\begin{frame}[fragile]{\Ctitle}{\stitle}
	\begin{exampleblock}{Correction}
		\begin{center}
			\pstree[arrows=->,treesep=0.7cm,levelsep=1cm]{\TCircle[radius=0.25cm]{$7$}}{
				\pstree{\TCircle[linecolor=BrickRed,linewidth=1pt,radius=0.25cm]{\textcolor{BrickRed}{$3$}} \nput[labelsep=1 pt]{0}{\pssucc}{\textcolor{BrickRed}{$u$}}}{
					\TCircle[name=D,radius=0.25cm]{$\ 2\ $}
					\ncbox[linecolor=gray,nodesep=0.1,boxsize=0.3,linestyle=dashed]{D}{D}
					\nbput[labelsep=0]{\textcolor{gray}{$\scriptstyle t_1$}}
					\pstree{\Tcircle[linecolor=BrickRed,linewidth=1pt,radius=0.25cm]{\textcolor{BrickRed}{$5$}} \nput[labelsep=1 pt]{0}{\pssucc}{\textcolor{BrickRed}{$v$}}}
					{
						\TCircle[name=Q,radius=0.25cm]{$4$}
						\ncbox[linecolor=gray,nodesep=0.1,boxsize=0.3,linestyle=dashed]{Q}{Q}
						\nbput[labelsep=0]{\textcolor{gray}{$\scriptstyle t_2$}}
						\TCircle[name=S,radius=0.25cm]{$6$}
						\ncbox[linecolor=gray,nodesep=0.1,boxsize=0.3,linestyle=dashed]{S}{S}
						\nbput[labelsep=0]{\textcolor{gray}{$\scriptstyle t_3$}}
					}
				}
				\TCircle[radius=0.25cm]{$9$}
			} \hspace{1.5cm}
			\onslide<3->{
				\pstree[arrows=->,treesep=0.7cm,levelsep=1cm]{\TCircle[radius=0.25cm]{$7$}}{
					\pstree{\TCircle[linecolor=BrickRed,linewidth=1pt,radius=0.25cm]{\textcolor{BrickRed}{$5$}} \nput[labelsep=1 pt]{0}{\pssucc}{\textcolor{BrickRed}{$v$}}}{
						\pstree{\Tcircle[linecolor=BrickRed,linewidth=1pt,radius=0.25cm]{\textcolor{BrickRed}{$3$}} \nput[labelsep=1 pt]{0}{\pssucc}{\textcolor{BrickRed}{$u$}}}
						{
							\TCircle[name=D,radius=0.25cm]{$2$}
							\ncbox[linecolor=gray,nodesep=0.1,boxsize=0.3,linestyle=dashed]{D}{D}
							\nbput[labelsep=0]{\textcolor{gray}{$\scriptstyle t_1$}}
							\TCircle[name=Q,radius=0.25cm]{$4$}
							\ncbox[linecolor=gray,nodesep=0.1,boxsize=0.3,linestyle=dashed]{Q}{Q}
							\nbput[labelsep=0]{\textcolor{gray}{$\scriptstyle t_2$}}
						}
						\TCircle[name=S,radius=0.25cm]{$6$}
						\ncbox[linecolor=gray,nodesep=0.1,boxsize=0.3,linestyle=dashed]{S}{S}
						\nbput[labelsep=0]{\textcolor{gray}{$\scriptstyle t_3$}}
					}
					\TCircle[radius=0.25cm]{$9$}
				}} \vspace{0.2cm}\\

			\vspace{1cm}
		\end{center}
	\end{exampleblock}
\end{frame}



\begin{frame}{\Ctitle}{\stitle}
	\begin{exampleblock}{Correction}
		\begin{center}
			\pstree[arrows=->,treesep=0.7cm,levelsep=1cm]{\TCircle[linecolor=BrickRed,linewidth=1pt,radius=0.25cm]{\textcolor{BrickRed}{$7$}} \nput[labelsep=1 pt]{0}{\pssucc}{\textcolor{BrickRed}{$u$}}}{
				\pstree{\TCircle[linecolor=BrickRed,linewidth=1pt,radius=0.25cm]{\textcolor{BrickRed}{$5$}} \nput[labelsep=1 pt]{0}{\pssucc}{\textcolor{BrickRed}{$v$}}}{
					\pstree{\Tcircle[radius=0.25cm]{$3$}}
					{
						\TCircle[name=D,radius=0.25cm]{$2$}
						\TCircle[name=Q,radius=0.25cm]{$4$}

					}
					\psframe[linecolor=gray,linestyle=dashed](0.9,0.25)(-1.1,-1.3)
					\nbput[labelsep=1.6cm]{\textcolor{gray}{$\scriptstyle t_1$}}
					\TCircle[name=S,radius=0.25cm]{$6$}
					\ncbox[linecolor=gray,nodesep=0.1,boxsize=0.3,linestyle=dashed]{S}{S}
					\nbput[labelsep=0]{\textcolor{gray}{$\scriptstyle t_2$}}
				}
				\TCircle[name=N,radius=0.25cm]{$9$}
				\ncbox[linecolor=gray,nodesep=0.1,boxsize=0.3,linestyle=dashed]{N}{N}
				\nbput[labelsep=0]{\textcolor{gray}{$\scriptstyle t_3$}}
			} \hspace{1.5cm}
			\onslide<2->
			{
				\pstree[arrows=->,treesep=0.7cm,levelsep=1cm]{\TCircle[radius=0.25cm]{$5$}}
				{
					\pstree[treesep=0.7cm,levelsep=1cm]{\TCircle[radius=0.25cm]{$3$}}{
						\TCircle[radius=0.25cm]{$2$}
						\TCircle[radius=0.25cm]{$4$}
					}
					\pstree[treesep=0.7cm,levelsep=1cm]{\TCircle[radius=0.25cm]{$7$}}{
						\TCircle[radius=0.25cm]{$6$}
						\TCircle[radius=0.25cm]{$9$}
					}
				}
			} \vspace{0.5cm}
		\end{center}
	\end{exampleblock}
	\onslide<3->{
		\begin{block}{Equilibrage d'un abre binaire}
			Les rotations droite et gauche sont les opérations permettant de maintenir un certain équilibre dans un {\sc abr}. Et donc de \textcolor{blue}{garantir une complexité logarithmique} des opérations usuelles. Parmi les nombreuses possibilités d'{\sc abr} équilibrés, nous allons détailler les \textcolor{BrickRed}{arbres rouge-noir}.
		\end{block}
	}
\end{frame}

\makess{Arbres rouge-noir}
\begin{frame}[fragile]{\Ctitle}{\stitle}
	\begin{alertblock}{Définition des arbres rouge-noir}
		Un \textcolor{blue}{arbre rouge-noir} $t$ est un {\sc abr} (\textcolor{OliveGreen}{\ding{182}}), dans lequel chaque noeud porte une information de couleur (rouge ou noir), et ayant les deux propriétés suivantes :
		\begin{itemize}
			\item<1-> le père d'un noeud rouge est noir (\textcolor{OliveGreen}{\ding{183}}),
			\item<2-> le nombre de noeuds noirs le long d'un chemin de la racine à un sous arbre vide est toujours le même (\textcolor{OliveGreen}{\ding{184}}), on appellera \textcolor{blue}{hauteur noire} de $t$ et on notera $b(t)$ cette quantité .
		\end{itemize}
	\end{alertblock}
	\onslide<3->{
		\begin{exampleblock}{Exemples}
			\renewcommand{\arraystretch}{1.1}
			\begin{tabularx}{\textwidth}{Y|Y|Y}
				\pstree[arrows=->,treesep=0.5cm,levelsep=0.7cm]{\NN{$8$}}{
					\pstree{\NR{$3$}}{
						\NN{$1$}
						\pstree{\NN{$6$}}
						{
							\NR{$5$}
							\NR{$7$}
						}
					}
					\NN{$9$}
				}                                                                                                                                &
				\pstree[arrows=->,treesep=0.5cm,levelsep=0.7cm]{\NN{$5$}}
				{
					\pstree[treesep=0.5cm,levelsep=0.7cm]{\NR{$2$}}{
						\NN{$1$}
						\pstree{\NR{$3$}}
						{
							\Tn{}
							\NN{$4$}

						}
					}
					\pstree[treesep=0.5cm,levelsep=0.7cm]{\NR{$8$}}{
						\NN{$7$}
						\Tn{}
					}
				}
				                                                                                                                                 &
				\pstree[arrows=->,treesep=0.5cm,levelsep=0.7cm]{\NN{$7$}}
				{
					\pstree[treesep=0.5cm,levelsep=0.7cm]{\NN{$2$}}{
						\Tn{}
						\Tn{}
					}
					\pstree[treesep=0.5cm,levelsep=0.7cm]{\NR{$8$}}{
						\NN{$6$}
						\NN{9}
					}
				}
				\\
				\leavevmode\onslide<3->{\textcolor{OliveGreen}{\small \ding{182} \faCheck \quad \ding{183} \faCheck  \quad \ding{184} \faCheck}} & \leavevmode\onslide<4->{\textcolor{OliveGreen}{\small \ding{182} \faCheck \quad  \ding{183}} \textcolor{BrickRed}{\small \faTimesCircle} \textcolor{OliveGreen}{\small \quad \ding{184} \faCheck}} & \leavevmode\onslide<5->{\textcolor{OliveGreen}{\small \ding{182}}  \textcolor{BrickRed}{\small \faTimesCircle} \textcolor{OliveGreen}{\small \quad\ding{183} \faTimesCircle \quad \ding{184} \faCheck}} \\
			\end{tabularx}

		\end{exampleblock}
	}
\end{frame}

\begin{frame}[fragile]{\Ctitle}{\stitle}
	\begin{block}{Proriété d'équilibre des arbres rouge-noirs}
		Pour tout arbre rouge noir $t$ :
		\begin{itemize}
			\item<2-> $h(t) \leqslant 2b(t)$
			\item<3-> $2^{b(t)} \leqslant n(t) +1$
		\end{itemize}
		\onslide<4->{Conséquence : les arbres rouge-noir forment un ensemble d'arbres équilibrés.}
	\end{block}
	\onslide<5->{\begin{block}{Implémentation}
			Une implémentation en OCaml sera vue en TP, les opérations d'insertion et de suppression sont difficiles et reposent sur les rotations droite et gauche des {\sc abr}.
		\end{block}}
\end{frame}


\end{document}
\PassOptionsToPackage{dvipsnames,table}{xcolor}
\documentclass[10pt]{beamer}
\usepackage{Cours}

\begin{document}

\input{\detokenize{/home/fenarius/Travail/Cours/cpge-info/latex/MacrosCours.tex}}

% Numéro et titre de chapitre
\setcounter{numchap}{21}
\newcommand{\Ctitle}{\cnum {Compléments sur les arbres}}
\newcommand{\SPATH}{/home/fenarius/Travail/Cours/cpge-info/docs/mp2i/files/C\thenumchap/}
\newcommand{\NR}[1]{\TCircle[fillstyle=solid,fillcolor=red!20!white,linecolor=red,radius=0.25cm]{#1}}
\newcommand{\NN}[1]{\TCircle[fillstyle=solid,fillcolor=black!20!white,linecolor=black,radius=0.25cm]{#1}}
\psset{arrows=->,treesep=0.8cm,levelsep=0.8cm, radius=0.3cm}
\makess{Rappel}
\begin{frame}[fragile]{\Ctitle}{\stitle}
	\begin{block}{\textcolor{lightgray}{\small \rappel} Arbres binaires de recherche}
		\begin{enumerate}
			\item<1-> Rappeler les relations entre la hauteur $h$ et la taille $n$ d'un arbre binaire.
			\item<2-> Rappeler la définition d'un arbre binaire de recherche ({\sc abr}).
			\item<3-> L'arbre ci-dessous est-il un {\sc abr} ?
				\begin{center}
					\pstree{\TCircle{\tt 28}}
					{\pstree{\TCircle{\tt 16}}
						{\pstree{\TCircle{\tt 9}}
							{ \TCircle{7}
								\pstree{\TCircle{\tt 11}}
								{ \Tn{}
									\TCircle{\tt 13}
								}}\TCircle{\tt 21}
						}\pstree{\TCircle{\tt 37}}
						{
							\TCircle{\tt 34}
							\Tn{}
						}}
				\end{center}
			\item<4-> Quelle est le nombre maximal de comparaison lors de la recherche d'un élément dans cet arbre ?
			\item<5-> Construire un {\sc abr} contenant les valeurs $2, 9, 10, 17$ et $21$ et de hauteur minimale. Même question avec la hauteur maximale.
		\end{enumerate}
	\end{block}
\end{frame}


\begin{frame}[fragile]{\Ctitle}{\stitle}
	\begin{block}{Complexité}
		La complexité des opérations d'insertion et de recherche dans un {\sc abr} est majorée par la hauteur $h$ de l'arbre. \onslide<2->\textcolor{gray}{ On descend d'un niveau dans l'arbre à chaque comparaison et la profondeur d'un noeud est inférieure à $h$.}\\
		\onslide<3->{Or on sait que $ h+1 \leqslant n \leqslant 2^{h+1}-1$, et les deux bornes sont atteintes}
		\begin{itemize}
			\item<4-> Dans le cas d'un peigne ($n=h+1$) les opérations seront en $\mathcal{O}(n)$.
			\item<5-> Dans le cas d'un arbre complet ($n=2^{h+1}-1$), les opérations seront en $\mathcal{O}(\log(n))$.
		\end{itemize}
	\end{block}
	\onslide<6->{
		\begin{alertblock}{Définition}
			Soit $S$, un ensemble d'abres binaires. On dit que les arbres de $S$ sont \textcolor{blue}{équilibrés} s'il existe une constante $C$ telle que, pour tout arbre $s \in S$ :
			$$ h(s) \leqslant C \log(n(s))$$
		\end{alertblock}}
\end{frame}


\begin{frame}[fragile]{\Ctitle}{\stitle}
	\begin{block}{Rotation d'un {\sc abr}}
		On considère l'{\sc abr} suivant où $u$ et $v$ sont les étiquettes des noeuds représentés et $t_1$, $t_2$, $t_3$ des arbres binaires :
		\onslide<2->{\begin{center}\pstree[arrows=->,treesep=0.5cm,levelsep=0.7cm]{\TCircle[radius=0.25cm]{$v$}}{
					\pstree{\TCircle[radius=0.25cm]{$u$}}{
						\Tr{$t_1$}
						\Tr{$t_2$}}
					\Tr{$t_3$}
				}
			\end{center}}
		\onslide<3->{La \textcolor{blue}{rotation droite} de cet arbre, consiste à réorganiser les noeuds \textit{en conservant la propriété d'{\sc abr}} de la façon suivante :}
		\onslide<4->{
			\begin{center}
				\pstree[arrows=->,treesep=0.5cm,levelsep=0.7cm]{\TCircle[radius=0.25cm]{$u$}}{
					\Tr{$t_1$}
					\pstree{\TCircle[radius=0.25cm]{$v$}}{
						\Tr{$t_2$}
						\Tr{$t_3$}
					}
				}
			\end{center}}
		\onslide<5->{De façon symétrique, la \textcolor{blue}{rotation gauche} consiste en partant de cet arbre à revenir à l'arbre initial.}
	\end{block}
\end{frame}

\begin{frame}[fragile]{\Ctitle}{\stitle}
	\begin{exampleblock}{Exemple}
		On considère l'arbre binaire suivant :
		\begin{center}\pstree[arrows=->,treesep=0.5cm,levelsep=0.7cm]{\TCircle[radius=0.25cm]{$7$}}{
				\pstree{\TCircle[radius=0.25cm]{$3$}}{
					\TCircle[radius=0.25cm]{$2$}
					\pstree{\Tcircle[radius=0.25cm]{$5$}}
					{
						\TCircle[radius=0.25cm]{$4$}
						\TCircle[radius=0.25cm]{$6$}
					}
				}
				\TCircle[radius=0.25cm]{$9$}
			}
		\end{center}
		\begin{enumerate}
			\item<2-> Vérifier qu'il s'agit d'un {\sc abr}
			\item<3-> Montrer qu'un utilisant des rotations, on peut transformer cet arbre en un arbre binaire parfait.
		\end{enumerate}
	\end{exampleblock}
\end{frame}

\begin{frame}[fragile]{\Ctitle}{\stitle}
	\begin{exampleblock}{Correction}
		\begin{center}
			\pstree[arrows=->,treesep=0.7cm,levelsep=1cm]{\TCircle[radius=0.25cm]{$7$}}{
				\pstree{\TCircle[linecolor=BrickRed,linewidth=1pt,radius=0.25cm]{\textcolor{BrickRed}{$3$}} \nput[labelsep=1 pt]{0}{\pssucc}{\textcolor{BrickRed}{$u$}}}{
					\TCircle[name=D,radius=0.25cm]{$\ 2\ $}
					\ncbox[linecolor=gray,nodesep=0.1,boxsize=0.3,linestyle=dashed]{D}{D}
					\nbput[labelsep=0]{\textcolor{gray}{$\scriptstyle t_1$}}
					\pstree{\Tcircle[linecolor=BrickRed,linewidth=1pt,radius=0.25cm]{\textcolor{BrickRed}{$5$}} \nput[labelsep=1 pt]{0}{\pssucc}{\textcolor{BrickRed}{$v$}}}
					{
						\TCircle[name=Q,radius=0.25cm]{$4$}
						\ncbox[linecolor=gray,nodesep=0.1,boxsize=0.3,linestyle=dashed]{Q}{Q}
						\nbput[labelsep=0]{\textcolor{gray}{$\scriptstyle t_2$}}
						\TCircle[name=S,radius=0.25cm]{$6$}
						\ncbox[linecolor=gray,nodesep=0.1,boxsize=0.3,linestyle=dashed]{S}{S}
						\nbput[labelsep=0]{\textcolor{gray}{$\scriptstyle t_3$}}
					}
				}
				\TCircle[radius=0.25cm]{$9$}
			} \hspace{1.5cm}
			\onslide<3->{
				\pstree[arrows=->,treesep=0.7cm,levelsep=1cm]{\TCircle[radius=0.25cm]{$7$}}{
					\pstree{\TCircle[linecolor=BrickRed,linewidth=1pt,radius=0.25cm]{\textcolor{BrickRed}{$5$}} \nput[labelsep=1 pt]{0}{\pssucc}{\textcolor{BrickRed}{$v$}}}{
						\pstree{\Tcircle[linecolor=BrickRed,linewidth=1pt,radius=0.25cm]{\textcolor{BrickRed}{$3$}} \nput[labelsep=1 pt]{0}{\pssucc}{\textcolor{BrickRed}{$u$}}}
						{
							\TCircle[name=D,radius=0.25cm]{$2$}
							\ncbox[linecolor=gray,nodesep=0.1,boxsize=0.3,linestyle=dashed]{D}{D}
							\nbput[labelsep=0]{\textcolor{gray}{$\scriptstyle t_1$}}
							\TCircle[name=Q,radius=0.25cm]{$4$}
							\ncbox[linecolor=gray,nodesep=0.1,boxsize=0.3,linestyle=dashed]{Q}{Q}
							\nbput[labelsep=0]{\textcolor{gray}{$\scriptstyle t_2$}}
						}
						\TCircle[name=S,radius=0.25cm]{$6$}
						\ncbox[linecolor=gray,nodesep=0.1,boxsize=0.3,linestyle=dashed]{S}{S}
						\nbput[labelsep=0]{\textcolor{gray}{$\scriptstyle t_3$}}
					}
					\TCircle[radius=0.25cm]{$9$}
				}} \vspace{0.2cm}\\

			\vspace{1cm}
		\end{center}
	\end{exampleblock}
\end{frame}



\begin{frame}{\Ctitle}{\stitle}
	\begin{exampleblock}{Correction}
		\begin{center}
			\pstree[arrows=->,treesep=0.7cm,levelsep=1cm]{\TCircle[linecolor=BrickRed,linewidth=1pt,radius=0.25cm]{\textcolor{BrickRed}{$7$}} \nput[labelsep=1 pt]{0}{\pssucc}{\textcolor{BrickRed}{$u$}}}{
				\pstree{\TCircle[linecolor=BrickRed,linewidth=1pt,radius=0.25cm]{\textcolor{BrickRed}{$5$}} \nput[labelsep=1 pt]{0}{\pssucc}{\textcolor{BrickRed}{$v$}}}{
					\pstree{\Tcircle[radius=0.25cm]{$3$}}
					{
						\TCircle[name=D,radius=0.25cm]{$2$}
						\TCircle[name=Q,radius=0.25cm]{$4$}

					}
					\psframe[linecolor=gray,linestyle=dashed](0.9,0.25)(-1.1,-1.3)
					\nbput[labelsep=1.6cm]{\textcolor{gray}{$\scriptstyle t_1$}}
					\TCircle[name=S,radius=0.25cm]{$6$}
					\ncbox[linecolor=gray,nodesep=0.1,boxsize=0.3,linestyle=dashed]{S}{S}
					\nbput[labelsep=0]{\textcolor{gray}{$\scriptstyle t_2$}}
				}
				\TCircle[name=N,radius=0.25cm]{$9$}
				\ncbox[linecolor=gray,nodesep=0.1,boxsize=0.3,linestyle=dashed]{N}{N}
				\nbput[labelsep=0]{\textcolor{gray}{$\scriptstyle t_3$}}
			} \hspace{1.5cm}
			\onslide<2->
			{
				\pstree[arrows=->,treesep=0.7cm,levelsep=1cm]{\TCircle[radius=0.25cm]{$5$}}
				{
					\pstree[treesep=0.7cm,levelsep=1cm]{\TCircle[radius=0.25cm]{$3$}}{
						\TCircle[radius=0.25cm]{$2$}
						\TCircle[radius=0.25cm]{$4$}
					}
					\pstree[treesep=0.7cm,levelsep=1cm]{\TCircle[radius=0.25cm]{$7$}}{
						\TCircle[radius=0.25cm]{$6$}
						\TCircle[radius=0.25cm]{$9$}
					}
				}
			} \vspace{0.5cm}
		\end{center}
	\end{exampleblock}
	\onslide<3->{
		\begin{block}{Equilibrage d'un abre binaire}
			Les rotations droite et gauche sont les opérations permettant de maintenir un certain équilibre dans un {\sc abr}. Et donc de \textcolor{blue}{garantir une complexité logarithmique} des opérations usuelles. Parmi les nombreuses possibilités d'{\sc abr} équilibrés, nous allons détailler les \textcolor{BrickRed}{arbres rouge-noir}.
		\end{block}
	}
\end{frame}

\makess{Arbres rouge-noir}
\begin{frame}[fragile]{\Ctitle}{\stitle}
	\begin{alertblock}{Définition des arbres rouge-noir}
		Un \textcolor{blue}{arbre rouge-noir} $t$ est un {\sc abr} (\textcolor{OliveGreen}{\ding{182}}), dans lequel chaque noeud porte une information de couleur (rouge ou noir), et ayant les deux propriétés suivantes :
		\begin{itemize}
			\item<1-> le père d'un noeud rouge est noir (\textcolor{OliveGreen}{\ding{183}}),
			\item<2-> le nombre de noeuds noirs le long d'un chemin de la racine à un sous arbre vide est toujours le même (\textcolor{OliveGreen}{\ding{184}}), on appellera \textcolor{blue}{hauteur noire} de $t$ et on notera $b(t)$ cette quantité .
		\end{itemize}
	\end{alertblock}
	\onslide<3->{
		\begin{exampleblock}{Exemples}
			\renewcommand{\arraystretch}{1.1}
			\begin{tabularx}{\textwidth}{Y|Y|Y}
				\pstree[arrows=->,treesep=0.5cm,levelsep=0.7cm]{\NN{$8$}}{
					\pstree{\NR{$3$}}{
						\NN{$1$}
						\pstree{\NN{$6$}}
						{
							\NR{$5$}
							\NR{$7$}
						}
					}
					\NN{$9$}
				}                                                                                                                                &
				\pstree[arrows=->,treesep=0.5cm,levelsep=0.7cm]{\NN{$5$}}
				{
					\pstree[treesep=0.5cm,levelsep=0.7cm]{\NR{$2$}}{
						\NN{$1$}
						\pstree{\NR{$3$}}
						{
							\Tn{}
							\NN{$4$}

						}
					}
					\pstree[treesep=0.5cm,levelsep=0.7cm]{\NR{$8$}}{
						\NN{$7$}
						\Tn{}
					}
				}
				                                                                                                                                 &
				\pstree[arrows=->,treesep=0.5cm,levelsep=0.7cm]{\NN{$7$}}
				{
					\pstree[treesep=0.5cm,levelsep=0.7cm]{\NN{$2$}}{
						\Tn{}
						\Tn{}
					}
					\pstree[treesep=0.5cm,levelsep=0.7cm]{\NR{$8$}}{
						\NN{$6$}
						\NN{9}
					}
				}
				\\
				\leavevmode\onslide<3->{\textcolor{OliveGreen}{\small \ding{182} \faCheck \quad \ding{183} \faCheck  \quad \ding{184} \faCheck}} & \leavevmode\onslide<4->{\textcolor{OliveGreen}{\small \ding{182} \faCheck \quad  \ding{183}} \textcolor{BrickRed}{\small \faTimesCircle} \textcolor{OliveGreen}{\small \quad \ding{184} \faCheck}} & \leavevmode\onslide<5->{\textcolor{OliveGreen}{\small \ding{182}}  \textcolor{BrickRed}{\small \faTimesCircle} \textcolor{OliveGreen}{\small \quad\ding{183} \faTimesCircle \quad \ding{184} \faCheck}} \\
			\end{tabularx}

		\end{exampleblock}
	}
\end{frame}

\begin{frame}[fragile]{\Ctitle}{\stitle}
	\begin{block}{Proriété d'équilibre des arbres rouge-noirs}
		Pour tout arbre rouge noir $t$ :
		\begin{itemize}
			\item<2-> $h(t) \leqslant 2b(t)$
			\item<3-> $2^{b(t)} \leqslant n(t) +1$
		\end{itemize}
		\onslide<4->{Conséquence : les arbres rouge-noir forment un ensemble d'arbres équilibrés.}
	\end{block}
	\onslide<5->{\begin{block}{Implémentation}
			Une implémentation en OCaml sera vue en TP, les opérations d'insertion et de suppression sont difficiles et reposent sur les rotations droite et gauche des {\sc abr}.
		\end{block}}
\end{frame}

\makess{Arbres binaires stricts}
\begin{frame}[fragile]{\Ctitle}{\stitle}
	\begin{block}{Définition}
		Un arbre binaire est \textcolor{blue}{strict} si tous ses noeuds ont soit 0, soit 2 fils.
		Un noeud est donc soit une feuille (pas de fils), soit un noeud interne (deux fils).
	\end{block}
	\onslide<2->{
		\begin{exampleblock}{Exemples}
			Les arbres de codage de Huffman sont des arbres binaires stricts. Qu'on peut représenter en OCaml par le type :
			\inputpartOCaml{\SPATH/abs.ml}{}{}{1}{3}
			Ici les noeuds internes ne portent pas d'information.
		\end{exampleblock}}
\end{frame}

\begin{frame}[fragile]{\Ctitle}{\stitle}
	\begin{exampleblock}{Exercice}
		\begin{enumerate}
			\item Donner une définition inductive des arbres binaires stricts.
			\item Ecrire en OCaml la fonction renvoyant la taille d'un arbre binaire strict.
			\item On note $n$ le nombre de noeuds internes d'un arbres binaire strict et $f$ son nombre de feuilles. Montrer que $f = n +1$. \\
			      \textcolor{OliveGreen}{\small \aide \;} On pourra raisonner par récurrence sur la taille de l'arbre ou dénombrer de deux façon différentes les noeuds ayant un père.

		\end{enumerate}
	\end{exampleblock}
\end{frame}

\makess{Arbres}
\begin{frame}[fragile]{\Ctitle}{\stitle}
	\begin{alertblock}{Définition}
		Un \textcolor{blue}{arbre} est un ensemble de $n \geqslant 1$ noeuds structurés de la manière suivante :
		\begin{itemize}
			\item un noeud particulier $r$ est appelé la \textit{racine} de l'arbre,
			\item les $n-1$ autres noeuds sont partitionnés en $k \geqslant 0$ sous ensembles disjoints qui forment autant d'arbres, appelés \textit{sous-arbres} de $r$,
			\item la racine $r$ est relié à la racine de chacun de ces sous-arbres par une arête.
		\end{itemize}
	\end{alertblock}
	\begin{exampleblock}{Exemple}
		\begin{center}
			\pstree{\TCircle{A}}{
				\TCircle{B}
				\pstree{\TCircle{C}}{
					\TCircle{E}
					\TCircle{F}
				}
				\pstree{\TCircle{D}}{
					\TCircle{G}
					\TCircle{H}
					\TCircle{I}
				}
				\pstree{\TCircle{J}}{
					\TCircle{K}
				}
			}
		\end{center}
	\end{exampleblock}
\end{frame}

\begin{frame}[fragile]{\Ctitle}{\stitle}
	\begin{block}{Remarques}
		\begin{itemize}
		\item La séquence des sous arbres d'un noeud est appelée \textcolor{blue}{forêt}.
		\item Un arbre réduit à un seul noeud est appelé \textcolor{blue}{feuille}.
		\item \textcolor{BrickRed}{\small \danger \;} Un arbre binaire n'est \textcolor{BrickRed}{\textit{pas}} un arbre. En effet :
		\begin{itemize}
		\item  un arbre binaire peut être vide (et pas un arbre);
		\item dans un arbre binaire on distingue le fils gauche du fils droit, c'est à dire que les deux arbres binaires ci-dessous sont différents :
			\begin{center}
				\pstree{\TCircle{A}}{\pstree{\TCircle{B}}{\Tr{$\varnothing$} \Tr{$\varnothing$}} \Tr{$\varnothing$}} \qquad \pstree{\TCircle{A}}{\Tr{$\varnothing$} \pstree{\TCircle{B}}{\Tr{$\varnothing$} \Tr{$\varnothing$}}}
			\end{center}
			Alors que le seul \textit{arbre} ayant deux noeuds est :
			\begin{center}\pstree{\TCircle{A}}{\TCircle{B}}\end{center}
		\end{itemize}
		\end{itemize}
	\end{block}
\end{frame}

\makess{Représentation en machine}
\begin{frame}[fragile]{\Ctitle}{\stitle}
	\begin{block}{Représentation en OCaml}
		\inputpartOCaml{\SPATH/arbres.ml}{}{}{1}{2}
	\end{block}
	\begin{exampleblock}{Exercice}
		Ecrire une fonction {\tt size : 'a tree -> int} qui renvoie le nombre de noeuds d'un arbre. \\
		\textcolor{OliveGreen}{\small \aide \;} On pourra utiliser deux fonctions \textit{mutuellement récursive} qui renvoient respectivement la taille d'un arbre et celle d'une forêt.
	\end{exampleblock}
\end{frame}

\begin{frame}[fragile]{\Ctitle}{\stitle}
	\begin{block}{Représentation en C}
		\begin{pspicture}(-5,2.8)(5,6.2)
				\rput(0,6){\Circlenode{A}{A}}
				\rput(-1,5){\Circlenode{B}{B}}
				\rput(1,5){\Circlenode{C}{C}}
				\rput(-1,4){\Circlenode{D}{D}}
				\rput(0,4){\Circlenode{E}{E}}
				\rput(1,4){\Circlenode{F}{F}}
				\rput(2,4){\Circlenode{G}{G}}
				\rput(1,3){\Circlenode{J}{J}}
				\rput(-1.5,3){\Circlenode{H}{H}}
				\rput(-0.5,3){\Circlenode{I}{I}}
				\ncline{-}{A}{B}
				\ncline{-}{A}{C}
				\ncline{-}{B}{D}
				\ncline{-}{C}{E}
				\ncline{-}{C}{F}
				\ncline{-}{C}{G}
				\ncline{-}{D}{H}
				\ncline{-}{D}{I}
				\ncline{-}{F}{J}
		\end{pspicture}\\
		\onslide<2->{Principe : utiliser un pointeur vers le premier fils et un pointeur vers le frère suivant. En anglais, \textcolor{blue}{\sc lcrs} : \textit{left child right sibling}.}
	\end{block}
\end{frame}

\begin{frame}[fragile]{\Ctitle}{\stitle}
	\begin{block}{Représentation en C}
		En notant en :
		\begin{itemize}
		\item  \textcolor{BrickRed}{$\longrightarrow$} le pointeur vers le  fils gauche,
		\item  \textcolor{blue}{$\longrightarrow$} le pointeur vers le frère suivant,
		\item $\bot$ le pointeur {\sc null}.
		\end{itemize}
		\begin{center}
		\begin{pspicture}(-5,2)(5,6.2)
				\rput(0,6){\Circlenode{A}{A}} \rput(0.7,6){\rnode{NA}{$\bot$}}
				\rput(-1.5,5){\Circlenode{B}{B}}
				\rput(1,5){\Circlenode{C}{C}} \rput(1.7,5){\rnode{NC}{$\bot$}}
				\rput(-1.5,4){\Circlenode{D}{D}} \rput(-0.8,4){\rnode{ND}{$\bot$}}
				\rput(0,4){\Circlenode{E}{E}} \rput(0,3.3){\rnode{VE}{$\bot$}}
				\rput(1,4){\Circlenode{F}{F}} 
				\rput(2,4){\Circlenode{G}{G}} \rput(2.7,4){\rnode{NG}{$\bot$}} \rput(2,3.3){\rnode{VG}{$\bot$}}
				\rput(1,3){\Circlenode{J}{J}} \rput(1.7,3){\rnode{NJ}{$\bot$}} \rput(1,2.3){\rnode{VJ}{$\bot$}}
				\rput(-2,3){\Circlenode{H}{H}}
				\rput(-2,2.3){\rnode{VH}{$\bot$}} 
				\rput(-1,3){\Circlenode{I}{I}} \rput(-0.3,3){\rnode{NI}{$\bot$}} \rput(-1,2.3){\rnode{VI}{$\bot$}} 
				\ncline[linecolor=BrickRed,linewidth=0.03]{->}{A}{B}
				\ncline[linecolor=blue, linewidth=0.03]{->}{A}{NA}
				\ncline[linecolor=blue, linewidth=0.03]{->}{C}{NC}
				\ncline[linecolor=blue, linewidth=0.03]{->}{D}{ND}
				\ncline[linecolor=blue, linewidth=0.03]{->}{G}{NG}
				\ncline[linecolor=blue, linewidth=0.03]{->}{J}{NJ}
				\ncline[linecolor=BrickRed, linewidth=0.03]{->}{H}{VH}
				\ncline[linecolor=BrickRed, linewidth=0.03]{->}{I}{VI}
				\ncline[linecolor=BrickRed, linewidth=0.03]{->}{J}{VJ}
				\ncline[linecolor=BrickRed, linewidth=0.03]{->}{E}{VE}
				\ncline[linecolor=BrickRed, linewidth=0.03]{->}{G}{VG}
				\ncline[linecolor=blue, linewidth=0.03]{->}{I}{NI}
				\ncline[linecolor=blue, linewidth=0.03]{->}{B}{C}
				\ncline[linecolor=blue, linewidth=0.03]{->}{H}{I}
				\ncline[linecolor=blue, linewidth=0.03]{->}{E}{F}
				\ncline[linecolor=blue, linewidth=0.03]{->}{F}{G}
				\ncline[linecolor=BrickRed,linewidth=0.03]{->}{B}{D}
				\ncline[linecolor=BrickRed,linewidth=0.03]{->}{C}{E}
				\ncline[linecolor=BrickRed,linewidth=0.03]{->}{D}{H}
				\ncline[linecolor=BrickRed,linewidth=0.03]{->}{F}{J}
		\end{pspicture}
	\end{center}
	\end{block}
\end{frame}


\begin{frame}[fragile]{\Ctitle}{\stitle}
	\begin{block}{Représentation en C}
	\inputpartC{\SPATH/arbres.c}{}{\small}{6}{13}	
	\end{block}
	\onslide<2->{
	\begin{exampleblock}{Exercice}
		\begin{enumerate}
		\item Ecrire une fonction de signature \mintinline{c}{tree create_tree(int value)} qui renvoie un arbre réduit à un noeud de valeur {\tt value}.
		\item Ecrire une fonction de signature \mintinline{c}{int size(tree t)} qui renvoie le nombre de noeuds d'un arbre. 
		\end{enumerate}
	\end{exampleblock}}
\end{frame}

\makess{Conversion entre arbre et arbre binaire}
\begin{frame}[fragile]{\Ctitle}{\stitle}
	\begin{block}{Remarque}
		\begin{tabularx}{\textwidth}{X|X}
			Arbres & Arbres binaires \\
		\inputpartC{\SPATH/arbres.c}{}{\footnotesize}{6}{13} &  \inputpartC{\SPATH/abr.c}{}{\footnotesize}{4}{11} \\
		\end{tabularx}
		\onslide<2->{On constate que les types représentant les arbres et les arbres binaires sont identiques, il y a un \textit{isomorphisme naturel} entre les arbres et les arbres binaires. On peut convertir un arbre en arbre binaire et inversement}
	\end{block}
\end{frame}

\begin{frame}[fragile]{\Ctitle}{\stitle}
	\begin{exampleblock}{Exercice}
		Donner la représentation \textit{\sc lcrs} de l'arbre binaire suivant (on n'a pas fait figuré les sous arbres vides) puis sa représentation sous la forme d'arbre généralisé :\smallskip\\
		\begin{center}
			\begin{pspicture}(-5,1.3)(5,6.2)
			\rput(0,6){\Circlenode{A}{A}}
			\rput(-1,5){\Circlenode{B}{B}}
			\rput(-4,4){\Circlenode{C}{C}}
			\rput(2,4){\Circlenode{D}{D}}
			\rput(1,3){\Circlenode{G}{G}}
			\rput(0,2){\Circlenode{L}{L}}
			\rput(3,3){\Circlenode{H}{H}}
			\rput(4,2){\Circlenode{M}{M}}
			\rput(-5,3){\Circlenode{E}{E}}
			\rput(-4,2){\Circlenode{I}{I}}
			\rput(-2,3){\Circlenode{F}{F}}
			\rput(-3,2){\Circlenode{J}{J}}
			\rput(-1,2){\Circlenode{K}{K}}
			\ncline{->}{A}{B}
			\ncline{->}{B}{C}
			\ncline{->}{B}{D}
			\ncline{->}{C}{E}
			\ncline{->}{C}{F}
			\ncline{->}{E}{I}
			\ncline{->}{F}{J}
			\ncline{->}{F}{K}
			\ncline{->}{D}{G}
			\ncline{->}{D}{H}
			\ncline{->}{G}{L}
			\ncline{->}{H}{M}
			\end{pspicture}
		\end{center}
	\end{exampleblock}
\end{frame}


\begin{frame}[fragile]{\Ctitle}{\stitle}
	\begin{exampleblock}{Exercice}
		Représentation \textit{\sc lcrs}  :\smallskip\\
		\begin{center}
			\begin{pspicture}(-5,1.3)(5,6.2)
			\rput(0,6){\Circlenode{A}{A}}
			\rput(-1,5){\Circlenode{B}{B}}
			\rput(-4,4){\Circlenode{C}{C}}
			\rput(2,4){\Circlenode{D}{D}}
			\rput(1,3){\Circlenode{G}{G}}
			\rput(0,2){\Circlenode{L}{L}}
			\rput(3,3){\Circlenode{H}{H}}
			\rput(4,2){\Circlenode{M}{M}}
			\rput(-5,3){\Circlenode{E}{E}}
			\rput(-4,2){\Circlenode{I}{I}}
			\rput(-2,3){\Circlenode{F}{F}}
			\rput(-3,2){\Circlenode{J}{J}}
			\rput(-1,2){\Circlenode{K}{K}}
			\ncline[linecolor=BrickRed, linewidth=0.03]{->}{A}{B}
			\ncline[linecolor=BrickRed, linewidth=0.03]{->}{B}{C}
			\ncline[linecolor=blue, linewidth=0.03]{->}{B}{D}
			\ncline[linecolor=BrickRed, linewidth=0.03]{->}{C}{E}
			\ncline[linecolor=blue, linewidth=0.03]{->}{C}{F}
			\ncline[linecolor=blue, linewidth=0.03]{->}{E}{I}
			\ncline[linecolor=BrickRed, linewidth=0.03]{->}{F}{J}
			\ncline[linecolor=blue, linewidth=0.03]{->}{F}{K}
			\ncline[linecolor=BrickRed, linewidth=0.03]{->}{D}{G}
			\ncline[linecolor=blue, linewidth=0.03]{->}{D}{H}
			\ncline[linecolor=BrickRed, linewidth=0.03]{->}{G}{L}
			\ncline[linecolor=blue, linewidth=0.03]{->}{H}{M}
			\end{pspicture}
		\end{center}
	\end{exampleblock}
\end{frame}


\begin{frame}[fragile]{\Ctitle}{\stitle}
	\begin{exampleblock}{Exercice}
		Conversion en arbre généralisé  :\smallskip\\
		\begin{center}
			\begin{pspicture}(-5,1.3)(5,6.2)
			\rput(0,6){\Circlenode{A}{A}}
			\rput(-1,4.5){\Circlenode{B}{B}}
			\rput(-4,3.5){\Circlenode{C}{C}}
			\rput(2,4.5){\Circlenode{D}{D}}
			\rput(2,3.5){\Circlenode{G}{G}}
			\rput(2,2.5){\Circlenode{L}{L}}
			\rput(3,4.5){\Circlenode{H}{H}}
			\rput(4,4.5){\Circlenode{M}{M}}
			\rput(-5,2.5){\Circlenode{E}{E}}
			\rput(-4,2.5){\Circlenode{I}{I}}
			\rput(-2,3.5){\Circlenode{F}{F}}
			\rput(-2,2.5){\Circlenode{J}{J}}
			\rput(-1,3.5){\Circlenode{K}{K}}
			\ncline{->}{A}{B}
			\ncline{->}{A}{D}
			\ncline{->}{A}{H}
			\ncline{->}{A}{M}
			\ncline{->}{B}{C}
			\ncline{->}{B}{F}
			\ncline{->}{B}{K}
			\ncline{->}{C}{E}
			\ncline{->}{C}{I}
			\ncline{->}{F}{J}
			\ncline{->}{D}{G}
			\ncline{->}{G}{L}
			\end{pspicture}
		\end{center}
	\end{exampleblock}
\end{frame}

\begin{frame}[fragile]{\Ctitle}{\stitle}
	\begin{exampleblock}{Exercice}
		Donner la représentation \textit{\sc lcrs} de l'arbre  suivant  puis sa représentation sous la forme d'arbre binaire :\smallskip\\
		\begin{center}
			\begin{pspicture}(-5,1.3)(5,6.2)
			\rput(0,6){\Circlenode{A}{A}}
			\rput(-2,5){\Circlenode{B}{B}}
			\rput(2,5){\Circlenode{C}{C}}
			\rput(-1,4){\Circlenode{E}{E}}
			\rput(2,4){\Circlenode{F}{F}}
			\rput(1,3){\Circlenode{J}{J}}
			\rput(3,3){\Circlenode{K}{K}}
			\rput(-3,4){\Circlenode{D}{D}}
			\rput(-2,3){\Circlenode{G}{G}}
			\rput(-1,3){\Circlenode{H}{H}}
			\rput(-0,3){\Circlenode{I}{I}}
			\ncline{->}{A}{B}
			\ncline{->}{A}{C}
			\ncline{->}{B}{D}
			\ncline{->}{C}{F}
			\ncline{->}{E}{G}
			\ncline{->}{E}{H}
			\ncline{->}{E}{I}
			\ncline{->}{B}{E}
			\ncline{->}{F}{J}
			\ncline{->}{F}{K}
			\end{pspicture}
		\end{center}
	\end{exampleblock}
\end{frame}



\begin{frame}[fragile]{\Ctitle}{\stitle}
	\begin{exampleblock}{Exercice}
		Représentation \textit{\sc lcrs}  :\smallskip\\
		\begin{center}
			\begin{pspicture}(-5,1.3)(5,6.2)
			\rput(0,6){\Circlenode{A}{A}}
			\rput(-2,5){\Circlenode{B}{B}}
			\rput(2,5){\Circlenode{C}{C}}
			\rput(-1,4){\Circlenode{E}{E}}
			\rput(2,4){\Circlenode{F}{F}}
			\rput(1,3){\Circlenode{J}{J}}
			\rput(3,3){\Circlenode{K}{K}}
			\rput(-3,4){\Circlenode{D}{D}}
			\rput(-2,3){\Circlenode{G}{G}}
			\rput(-1,3){\Circlenode{H}{H}}
			\rput(-0,3){\Circlenode{I}{I}}
			\ncline[linecolor=BrickRed, linewidth=0.03]{->}{A}{B}
			\ncline[linecolor=blue, linewidth=0.03]{->}{B}{C}
			\ncline[linecolor=BrickRed, linewidth=0.03]{->}{B}{D}
			\ncline[linecolor=BrickRed, linewidth=0.03]{->}{C}{F}
			\ncline[linecolor=BrickRed, linewidth=0.03]{->}{E}{G}
			\ncline[linecolor=blue, linewidth=0.03]{->}{D}{E}
			\ncline[linecolor=blue, linewidth=0.03]{->}{G}{H}
			\ncline[linecolor=blue, linewidth=0.03]{->}{H}{I}
			\ncline[linecolor=BrickRed, linewidth=0.03]{->}{F}{J}
			\ncline[linecolor=blue, linewidth=0.03]{->}{J}{K}
			\end{pspicture}
		\end{center}
	\end{exampleblock}
\end{frame}



\begin{frame}[fragile]{\Ctitle}{\stitle}
	\begin{exampleblock}{Exercice}
		Conversion en arbre binaire :\smallskip\\
		\begin{center}
			\begin{pspicture}(-5,-0.2)(5,6.2)
			\rput(0,6){\Circlenode{A}{A}}
			\rput(-2,5){\Circlenode{B}{B}}
			\rput(1,4){\Circlenode{C}{C}}
			\rput(-4,4){\Circlenode{D}{D}}
			\rput(-3,3){\Circlenode{E}{E}}
			\rput(0,3){\Circlenode{F}{F}}
			\rput(-1,2){\Circlenode{J}{J}}
			\rput(0,1){\Circlenode{K}{K}}
			\rput(-4,2){\Circlenode{G}{G}}
			\rput(-3,1){\Circlenode{H}{H}}
			\rput(-2,0){\Circlenode{I}{I}}
			\ncline[linecolor=BrickRed, linewidth=0.03]{->}{A}{B}
			\ncline[linecolor=blue, linewidth=0.03]{->}{B}{C}
			\ncline[linecolor=BrickRed, linewidth=0.03]{->}{B}{D}
			\ncline[linecolor=BrickRed, linewidth=0.03]{->}{C}{F}
			\ncline[linecolor=BrickRed, linewidth=0.03]{->}{E}{G}
			\ncline[linecolor=blue, linewidth=0.03]{->}{D}{E}
			\ncline[linecolor=blue, linewidth=0.03]{->}{G}{H}
			\ncline[linecolor=blue, linewidth=0.03]{->}{H}{I}
			\ncline[linecolor=BrickRed, linewidth=0.03]{->}{F}{J}
			\ncline[linecolor=blue, linewidth=0.03]{->}{J}{K}
			\end{pspicture}
		\end{center}
	\end{exampleblock}
\end{frame}

\makess{Parcours d'un arbre généralisé}
\begin{frame}[fragile]{\Ctitle}{\stitle}
	\begin{block}{Parcours}
	La notion de parcours \textit{infixe} n'a plus de sens cependant, pour un arbre généralisé, on peut définir les parcours suivants :
		\begin{itemize}
			\item<2-> \textcolor{blue}{prefixe} : on visite le noeud avant ses fils,
			\item<2-> \textcolor{blue}{suffixe} : on visite le noeud après ses fils,
			\item<2-> \textcolor{blue}{en largeur} : on visite les noeuds par niveau.
		\end{itemize}	
	\end{block}
	\begin{exampleblock}{Exemples}
	Donner les parcours prefixe, suffixe et en largeur de l'arbre généralisé 
		\begin{center}
			\begin{pspicture}(-5,2.8)(5,6.2)
			\rput(0,6){\Circlenode{A}{A}}
			\rput(-2,5.2){\Circlenode{B}{B}}
			\rput(2,5.2){\Circlenode{C}{C}}
			\rput(-1,4.4){\Circlenode{E}{E}}
			\rput(2,4.4){\Circlenode{F}{F}}
			\rput(1,3.6){\Circlenode{J}{J}}
			\rput(3,3.6){\Circlenode{K}{K}}
			\rput(-3,4.4){\Circlenode{D}{D}}
			\rput(-2,3.6){\Circlenode{G}{G}}
			\rput(-1,3.6){\Circlenode{H}{H}}
			\rput(-0,3.6){\Circlenode{I}{I}}
			\ncline{->}{A}{B}
			\ncline{->}{A}{C}
			\ncline{->}{B}{D}
			\ncline{->}{C}{F}
			\ncline{->}{E}{G}
			\ncline{->}{E}{H}
			\ncline{->}{E}{I}
			\ncline{->}{B}{E}
			\ncline{->}{F}{J}
			\ncline{->}{F}{K}
			\end{pspicture}
		\end{center}
	\end{exampleblock}
\end{frame}

\makess{Point de vue des graphes}
\begin{frame}[fragile]{\Ctitle}{\stitle}
	\begin{block}{Remarques}
	Un arbre généralisé est un graphe $(S,A)$ non orienté, connexe sans cycle.
	\begin{itemize}
		\item<2-> $S$ est l'ensemble des noeuds de l'arbre,
		\item<3-> $A$ est l'ensemble des arêtes de l'arbre, chaque arête est de la forme $(x,y)$ où $x$ est le père de $y$.
		\item<4-> \textcolor{blue}{Connexité} : il existe un chemin entre deux noeuds quelconques de l'arbre., $\forall (x,y) \in S^2, x \neq y$, il existe $x_1, \dots x_n$ tels que $\{x,x_1\}, \{x_1,x_2\}, \dots, \{x_{n-1},y\} \in A$.
		\item \textcolor{blue}{Acyclique}: il n'existe pas de cycle dans l'arbre, c'est à dire qu'il n'existe pas de suite de noeuds $x_1, \dots, x_n$ tels que $\{x_1,x_2\}, \{x_2,x_3\}, \dots, \{x_{n-1},x_n\}, \{x_n,x_1\} \in A$.
	\end{itemize}
	\textcolor{BrickRed}{\small \danger \;} On peut choisir n'importe quel noeud comme racine.
	\end{block}
\end{frame}


\makess{Sérialisation d'un arbre}
\begin{frame}[fragile]{\Ctitle}{\stitle}
	\begin{block}{Définition}
		La \textcolor{blue}{sérialisation} d'un arbre est une représentation de l'arbre sous forme de chaîne de caractères (à des fins de sauvegarde, transmission, reconstruction \dots). \\
		\onslide<2->{Un unique parcours (prefixe, infixe, suffixe ou en largeur) \textit{ne permet pas} de sérialiser un arbre, en effet des arbres différents peuvent avoir le même parcours.} \\
		\onslide<3->{Les solutions suivantes sont envisageables dans le cas d'un \textcolor{BrickRed}{un arbre binaire} :}
		\begin{itemize}
			\item<4-> stocker les parcours infixe et préfixe (mais cela impose de stocker les étiquettes en double);
			\item<5-> utiliser un parcours en largeur avec un marqueur spécial indiquant un fils absent;
			\item<6-> utiliser un parcours prefixe avec un marquer pour les fils absents
		\end{itemize}
	\end{block}
\end{frame}



\begin{frame}[fragile]{\Ctitle}{\stitle}
	\begin{exampleblock}{Exemple}
		\begin{center}		
			\pstree{\TCircle{9}}
		{\pstree{\TCircle{2}}
			{\TCircle{0}
				\pstree{\TCircle{6}}
				{\TCircle{5}
					\pstree{\TCircle{8}}
					{ 
						\TCircle{7}

						\Tn{} }}}\pstree{\TCircle{12}}
			{\TCircle{10}
				\TCircle{13}
			}}
		\end{center}
	\begin{itemize}
		\item Parcours préfixe et infixe :\\ \onslide<2->{\textcolor{OliveGreen}{{\tt 9 2 0 6 5 8 7 12 10 13 }} et \textcolor{OliveGreen}{\tt 0 2 5 6 7 8 9 10 12 13}}
		\item Parcours en largeur avec le marqueur {\tt \#} pour fils absent :\\ \onslide<3->{\textcolor{OliveGreen}{9 2 12 0 6 10 13 \# \# 5 8 \# \# \# \# \# \# 7}}
		\item Parcours préfixe avec le marqueur {\tt \#} pour fils absent : \\ \onslide<4->{\textcolor{OliveGreen}{9 2 0 \# \# 6 5 \# \# 8 7 \# 12 10 \# \# 13 \# \#}}
	\end{itemize}
	\end{exampleblock}

\end{frame}

\end{document}
\PassOptionsToPackage{dvipsnames,table}{xcolor}
\documentclass[10pt,french]{beamer}
\usepackage{Cours}

\begin{document}


\newcounter{numchap}
\setcounter{numchap}{1}
\newcounter{numframe}
\setcounter{numframe}{0}
\newcommand{\mframe}[1]{\frametitle{#1} \addtocounter{numframe}{1}}
\newcommand{\cnum}{\fbox{\textcolor{yellow}{\textbf{C\thenumchap}}}~}
\newcommand{\makess}[1]{\section{#1} \label{ss\thesection}}
\newcommand{\stitle}{\textcolor{yellow}{\textbf{\thesection. \nameref{ss\thesection}}}}

\definecolor{codebg}{gray}{0.90}
\definecolor{grispale}{gray}{0.95}
\definecolor{fluo}{rgb}{1,0.96,0.62}
\newminted[langageC]{c}{linenos=true,escapeinside=||,highlightcolor=fluo,tabsize=2,breaklines=true}
\newminted[codepython]{python}{linenos=true,escapeinside=||,highlightcolor=fluo,tabsize=2,breaklines=true}
% Inclusion complète (ou partiel en indiquant premiere et dernière ligne) d'un fichier C
\newcommand{\inputC}[3]{\begin{mdframed}[backgroundcolor=codebg] \inputminted[breaklines=true,fontsize=#3,linenos=true,highlightcolor=fluo,tabsize=2,highlightlines={#2}]{c}{#1} \end{mdframed}}
\newcommand{\inputpartC}[5]{\begin{mdframed}[backgroundcolor=codebg] \inputminted[breaklines=true,fontsize=#3,linenos=true,highlightcolor=fluo,tabsize=2,highlightlines={#2},firstline=#4,lastline=#5,firstnumber=1]{c}{#1} \end{mdframed}}
\newcommand{\inputpython}[3]{\begin{mdframed}[backgroundcolor=codebg] \inputminted[breaklines=true,fontsize=#3,linenos=true,highlightcolor=fluo,tabsize=2,highlightlines={#2}]{python}{#1} \end{mdframed}}
\newcommand{\inputpartOCaml}[5]{\begin{mdframed}[backgroundcolor=codebg] \inputminted[breaklines=true,fontsize=#3,linenos=true,highlightcolor=fluo,tabsize=2,highlightlines={#2},firstline=#4,lastline=#5,firstnumber=1]{OCaml}{#1} \end{mdframed}}
\BeforeBeginEnvironment{minted}{\begin{mdframed}[backgroundcolor=codebg]}
\AfterEndEnvironment{minted}{\end{mdframed}}
\newcommand{\kw}[1]{\textcolor{blue}{\tt #1}}

\newtcolorbox{rcadre}[4]{halign=center,colback={#1},colframe={#2},width={#3cm},height={#4cm},valign=center,boxrule=1pt,left=0pt,right=0pt}
\newtcolorbox{cadre}[4]{halign=center,colback={#1},colframe={#2},arc=0mm,width={#3cm},height={#4cm},valign=center,boxrule=1pt,left=0pt,right=0pt}
\newcommand{\myem}[1]{\colorbox{fluo}{#1}}
\mdfsetup{skipabove=1pt,skipbelow=-2pt}



% Noeud dans un cadre pour les arbres
\newcommand{\noeud}[2]{\Tr{\fbox{\textcolor{#1}{\tt #2}}}}

\newcommand{\htmlmode}{\lstset{language=html,numbers=left, tabsize=4, frame=single, breaklines=true, keywordstyle=\ttfamily, basicstyle=\small,
   numberstyle=\tiny\ttfamily, framexleftmargin=0mm, backgroundcolor=\color{grispale}, xleftmargin=12mm,showstringspaces=false}}
\newcommand{\pythonmode}{\lstset{
   language=python,
   linewidth=\linewidth,
   numbers=left,
   tabsize=4,
   frame=single,
   breaklines=true,
   keywordstyle=\ttfamily\color{blue},
   basicstyle=\small,
   numberstyle=\tiny\ttfamily,
   framexleftmargin=-2mm,
   numbersep=-0.5mm,
   backgroundcolor=\color{codebg},
   xleftmargin=-1mm, 
   showstringspaces=false,
   commentstyle=\color{gray},
   stringstyle=\color{OliveGreen},
   emph={turtle,Screen,Turtle},
   emphstyle=\color{RawSienna},
   morekeywords={setheading,goto,backward,forward,left,right,pendown,penup,pensize,color,speed,hideturtle,showturtle,forward}}
   }
   \newcommand{\Cmode}{\lstset{
      language=[ANSI]C,
      linewidth=\linewidth,
      numbers=left,
      tabsize=4,
      frame=single,
      breaklines=true,
      keywordstyle=\ttfamily\color{blue},
      basicstyle=\small,
      numberstyle=\tiny\ttfamily,
      framexleftmargin=0mm,
      numbersep=2mm,
      backgroundcolor=\color{codebg},
      xleftmargin=0mm, 
      showstringspaces=false,
      commentstyle=\color{gray},
      stringstyle=\color{OliveGreen},
      emphstyle=\color{RawSienna},
      escapechar=\|,
      morekeywords={}}
      }
\newcommand{\bashmode}{\lstset{language=bash,numbers=left, tabsize=2, frame=single, breaklines=true, basicstyle=\ttfamily,
   numberstyle=\tiny\ttfamily, framexleftmargin=0mm, backgroundcolor=\color{grispale}, xleftmargin=12mm, showstringspaces=false}}
\newcommand{\exomode}{\lstset{language=python,numbers=left, tabsize=2, frame=single, breaklines=true, basicstyle=\ttfamily,
   numberstyle=\tiny\ttfamily, framexleftmargin=13mm, xleftmargin=12mm, basicstyle=\small, showstringspaces=false}}
   
   
  
%tei pour placer les images
%tei{nom de l’image}{échelle de l’image}{sens}{texte a positionner}
%sens ="1" (droite) ou "2" (gauche)
\newlength{\ltxt}
\newcommand{\tei}[4]{
\setlength{\ltxt}{\linewidth}
\setbox0=\hbox{\includegraphics[scale=#2]{#1}}
\addtolength{\ltxt}{-\wd0}
\addtolength{\ltxt}{-10pt}
\ifthenelse{\equal{#3}{1}}{
\begin{minipage}{\wd0}
\includegraphics[scale=#2]{#1}
\end{minipage}
\hfill
\begin{minipage}{\ltxt}
#4
\end{minipage}
}{
\begin{minipage}{\ltxt}
#4
\end{minipage}
\hfill
\begin{minipage}{\wd0}
\includegraphics[scale=#2]{#1}
\end{minipage}
}
}

%Juxtaposition d'une image pspciture et de texte 
%#1: = code pstricks de l'image
%#2: largeur de l'image
%#3: hauteur de l'image
%#4: Texte à écrire
\newcommand{\ptp}[4]{
\setlength{\ltxt}{\linewidth}
\addtolength{\ltxt}{-#2 cm}
\addtolength{\ltxt}{-0.1 cm}
\begin{minipage}[b][#3 cm][t]{\ltxt}
#4
\end{minipage}\hfill
\begin{minipage}[b][#3 cm][c]{#2 cm}
#1
\end{minipage}\par
}



%Macros pour les graphiques
\psset{linewidth=0.5\pslinewidth,PointSymbol=x}
\setlength{\fboxrule}{0.5pt}
\newcounter{tempangle}

%Marque la longueur du segment d'extrémité  #1 et  #2 avec la valeur #3, #4 est la distance par rapport au segment (en %age de la valeur de celui ci) et #5 l'orientation du marquage : +90 ou -90
\newcommand{\afflong}[5]{
\pstRotation[RotAngle=#4,PointSymbol=none,PointName=none]{#1}{#2}[X] 
\pstHomO[PointSymbol=none,PointName=none,HomCoef=#5]{#1}{X}[Y]
\pstTranslation[PointSymbol=none,PointName=none]{#1}{#2}{Y}[Z]
 \ncline{|<->|,linewidth=0.25\pslinewidth}{Y}{Z} \ncput*[nrot=:U]{\footnotesize{#3}}
}
\newcommand{\afflongb}[3]{
\ncline{|<->|,linewidth=0}{#1}{#2} \naput*[nrot=:U]{\footnotesize{#3}}
}

%Construis le point #4 situé à #2 cm du point #1 avant un angle #3 par rapport à l'horizontale. #5 = liste de paramètre
\newcommand{\lsegment}[5]{\pstGeonode[PointSymbol=none,PointName=none](0,0){O'}(#2,0){I'} \pstTranslation[PointSymbol=none,PointName=none]{O'}{I'}{#1}[J'] \pstRotation[RotAngle=#3,PointSymbol=x,#5]{#1}{J'}[#4]}
\newcommand{\tsegment}[5]{\pstGeonode[PointSymbol=none,PointName=none](0,0){O'}(#2,0){I'} \pstTranslation[PointSymbol=none,PointName=none]{O'}{I'}{#1}[J'] \pstRotation[RotAngle=#3,PointSymbol=x,#5]{#1}{J'}[#4] \pstLineAB{#4}{#1}}

%Construis le point #4 situé à #3 cm du point #1 et faisant un angle de  90° avec la droite (#1,#2) #5 = liste de paramètre
\newcommand{\psegment}[5]{
\pstGeonode[PointSymbol=none,PointName=none](0,0){O'}(#3,0){I'}
 \pstTranslation[PointSymbol=none,PointName=none]{O'}{I'}{#1}[J']
 \pstInterLC[PointSymbol=none,PointName=none]{#1}{#2}{#1}{J'}{M1}{M2} \pstRotation[RotAngle=-90,PointSymbol=x,#5]{#1}{M1}[#4]
  }
  
%Construis le point #4 situé à #3 cm du point #1 et faisant un angle de  #5° avec la droite (#1,#2) #6 = liste de paramètre
\newcommand{\mlogo}[6]{
\pstGeonode[PointSymbol=none,PointName=none](0,0){O'}(#3,0){I'}
 \pstTranslation[PointSymbol=none,PointName=none]{O'}{I'}{#1}[J']
 \pstInterLC[PointSymbol=none,PointName=none]{#1}{#2}{#1}{J'}{M1}{M2} \pstRotation[RotAngle=#5,PointSymbol=x,#6]{#1}{M2}[#4]
  }

% Construis un triangle avec #1=liste des 3 sommets séparés par des virgules, #2=liste des 3 longueurs séparés par des virgules, #3 et #4 : paramètre d'affichage des 2e et 3 points et #5 : inclinaison par rapport à l'horizontale
%autre macro identique mais sans tracer les segments joignant les sommets
\noexpandarg
\newcommand{\Triangleccc}[5]{
\StrBefore{#1}{,}[\pointA]
\StrBetween[1,2]{#1}{,}{,}[\pointB]
\StrBehind[2]{#1}{,}[\pointC]
\StrBefore{#2}{,}[\coteA]
\StrBetween[1,2]{#2}{,}{,}[\coteB]
\StrBehind[2]{#2}{,}[\coteC]
\tsegment{\pointA}{\coteA}{#5}{\pointB}{#3} 
\lsegment{\pointA}{\coteB}{0}{Z1}{PointSymbol=none, PointName=none}
\lsegment{\pointB}{\coteC}{0}{Z2}{PointSymbol=none, PointName=none}
\pstInterCC{\pointA}{Z1}{\pointB}{Z2}{\pointC}{Z3} 
\pstLineAB{\pointA}{\pointC} \pstLineAB{\pointB}{\pointC}
\pstSymO[PointName=\pointC,#4]{C}{C}[C]
}
\noexpandarg
\newcommand{\TrianglecccP}[5]{
\StrBefore{#1}{,}[\pointA]
\StrBetween[1,2]{#1}{,}{,}[\pointB]
\StrBehind[2]{#1}{,}[\pointC]
\StrBefore{#2}{,}[\coteA]
\StrBetween[1,2]{#2}{,}{,}[\coteB]
\StrBehind[2]{#2}{,}[\coteC]
\tsegment{\pointA}{\coteA}{#5}{\pointB}{#3} 
\lsegment{\pointA}{\coteB}{0}{Z1}{PointSymbol=none, PointName=none}
\lsegment{\pointB}{\coteC}{0}{Z2}{PointSymbol=none, PointName=none}
\pstInterCC[PointNameB=none,PointSymbolB=none,#4]{\pointA}{Z1}{\pointB}{Z2}{\pointC}{Z1} 
}


% Construis un triangle avec #1=liste des 3 sommets séparés par des virgules, #2=liste formée de 2 longueurs et d'un angle séparés par des virgules, #3 et #4 : paramètre d'affichage des 2e et 3 points et #5 : inclinaison par rapport à l'horizontale
%autre macro identique mais sans tracer les segments joignant les sommets
\newcommand{\Trianglecca}[5]{
\StrBefore{#1}{,}[\pointA]
\StrBetween[1,2]{#1}{,}{,}[\pointB]
\StrBehind[2]{#1}{,}[\pointC]
\StrBefore{#2}{,}[\coteA]
\StrBetween[1,2]{#2}{,}{,}[\coteB]
\StrBehind[2]{#2}{,}[\angleA]
\tsegment{\pointA}{\coteA}{#5}{\pointB}{#3} 
\setcounter{tempangle}{#5}
\addtocounter{tempangle}{\angleA}
\tsegment{\pointA}{\coteB}{\thetempangle}{\pointC}{#4}
\pstLineAB{\pointB}{\pointC}
}
\newcommand{\TriangleccaP}[5]{
\StrBefore{#1}{,}[\pointA]
\StrBetween[1,2]{#1}{,}{,}[\pointB]
\StrBehind[2]{#1}{,}[\pointC]
\StrBefore{#2}{,}[\coteA]
\StrBetween[1,2]{#2}{,}{,}[\coteB]
\StrBehind[2]{#2}{,}[\angleA]
\lsegment{\pointA}{\coteA}{#5}{\pointB}{#3} 
\setcounter{tempangle}{#5}
\addtocounter{tempangle}{\angleA}
\lsegment{\pointA}{\coteB}{\thetempangle}{\pointC}{#4}
}

% Construis un triangle avec #1=liste des 3 sommets séparés par des virgules, #2=liste formée de 1 longueurs et de deux angle séparés par des virgules, #3 et #4 : paramètre d'affichage des 2e et 3 points et #5 : inclinaison par rapport à l'horizontale
%autre macro identique mais sans tracer les segments joignant les sommets
\newcommand{\Trianglecaa}[5]{
\StrBefore{#1}{,}[\pointA]
\StrBetween[1,2]{#1}{,}{,}[\pointB]
\StrBehind[2]{#1}{,}[\pointC]
\StrBefore{#2}{,}[\coteA]
\StrBetween[1,2]{#2}{,}{,}[\angleA]
\StrBehind[2]{#2}{,}[\angleB]
\tsegment{\pointA}{\coteA}{#5}{\pointB}{#3} 
\setcounter{tempangle}{#5}
\addtocounter{tempangle}{\angleA}
\lsegment{\pointA}{1}{\thetempangle}{Z1}{PointSymbol=none, PointName=none}
\setcounter{tempangle}{#5}
\addtocounter{tempangle}{180}
\addtocounter{tempangle}{-\angleB}
\lsegment{\pointB}{1}{\thetempangle}{Z2}{PointSymbol=none, PointName=none}
\pstInterLL[#4]{\pointA}{Z1}{\pointB}{Z2}{\pointC}
\pstLineAB{\pointA}{\pointC}
\pstLineAB{\pointB}{\pointC}
}
\newcommand{\TrianglecaaP}[5]{
\StrBefore{#1}{,}[\pointA]
\StrBetween[1,2]{#1}{,}{,}[\pointB]
\StrBehind[2]{#1}{,}[\pointC]
\StrBefore{#2}{,}[\coteA]
\StrBetween[1,2]{#2}{,}{,}[\angleA]
\StrBehind[2]{#2}{,}[\angleB]
\lsegment{\pointA}{\coteA}{#5}{\pointB}{#3} 
\setcounter{tempangle}{#5}
\addtocounter{tempangle}{\angleA}
\lsegment{\pointA}{1}{\thetempangle}{Z1}{PointSymbol=none, PointName=none}
\setcounter{tempangle}{#5}
\addtocounter{tempangle}{180}
\addtocounter{tempangle}{-\angleB}
\lsegment{\pointB}{1}{\thetempangle}{Z2}{PointSymbol=none, PointName=none}
\pstInterLL[#4]{\pointA}{Z1}{\pointB}{Z2}{\pointC}
}

%Construction d'un cercle de centre #1 et de rayon #2 (en cm)
\newcommand{\Cercle}[2]{
\lsegment{#1}{#2}{0}{Z1}{PointSymbol=none, PointName=none}
\pstCircleOA{#1}{Z1}
}

%construction d'un parallélogramme #1 = liste des sommets, #2 = liste contenant les longueurs de 2 côtés consécutifs et leurs angles;  #3, #4 et #5 : paramètre d'affichage des sommets #6 inclinaison par rapport à l'horizontale 
% meme macro sans le tracé des segements
\newcommand{\Para}[6]{
\StrBefore{#1}{,}[\pointA]
\StrBetween[1,2]{#1}{,}{,}[\pointB]
\StrBetween[2,3]{#1}{,}{,}[\pointC]
\StrBehind[3]{#1}{,}[\pointD]
\StrBefore{#2}{,}[\longueur]
\StrBetween[1,2]{#2}{,}{,}[\largeur]
\StrBehind[2]{#2}{,}[\angle]
\tsegment{\pointA}{\longueur}{#6}{\pointB}{#3} 
\setcounter{tempangle}{#6}
\addtocounter{tempangle}{\angle}
\tsegment{\pointA}{\largeur}{\thetempangle}{\pointD}{#5}
\pstMiddleAB[PointName=none,PointSymbol=none]{\pointB}{\pointD}{Z1}
\pstSymO[#4]{Z1}{\pointA}[\pointC]
\pstLineAB{\pointB}{\pointC}
\pstLineAB{\pointC}{\pointD}
}
\newcommand{\ParaP}[6]{
\StrBefore{#1}{,}[\pointA]
\StrBetween[1,2]{#1}{,}{,}[\pointB]
\StrBetween[2,3]{#1}{,}{,}[\pointC]
\StrBehind[3]{#1}{,}[\pointD]
\StrBefore{#2}{,}[\longueur]
\StrBetween[1,2]{#2}{,}{,}[\largeur]
\StrBehind[2]{#2}{,}[\angle]
\lsegment{\pointA}{\longueur}{#6}{\pointB}{#3} 
\setcounter{tempangle}{#6}
\addtocounter{tempangle}{\angle}
\lsegment{\pointA}{\largeur}{\thetempangle}{\pointD}{#5}
\pstMiddleAB[PointName=none,PointSymbol=none]{\pointB}{\pointD}{Z1}
\pstSymO[#4]{Z1}{\pointA}[\pointC]
}


%construction d'un cerf-volant #1 = liste des sommets, #2 = liste contenant les longueurs de 2 côtés consécutifs et leurs angles;  #3, #4 et #5 : paramètre d'affichage des sommets #6 inclinaison par rapport à l'horizontale 
% meme macro sans le tracé des segements
\newcommand{\CerfVolant}[6]{
\StrBefore{#1}{,}[\pointA]
\StrBetween[1,2]{#1}{,}{,}[\pointB]
\StrBetween[2,3]{#1}{,}{,}[\pointC]
\StrBehind[3]{#1}{,}[\pointD]
\StrBefore{#2}{,}[\longueur]
\StrBetween[1,2]{#2}{,}{,}[\largeur]
\StrBehind[2]{#2}{,}[\angle]
\tsegment{\pointA}{\longueur}{#6}{\pointB}{#3} 
\setcounter{tempangle}{#6}
\addtocounter{tempangle}{\angle}
\tsegment{\pointA}{\largeur}{\thetempangle}{\pointD}{#5}
\pstOrtSym[#4]{\pointB}{\pointD}{\pointA}[\pointC]
\pstLineAB{\pointB}{\pointC}
\pstLineAB{\pointC}{\pointD}
}

%construction d'un quadrilatère quelconque #1 = liste des sommets, #2 = liste contenant les longueurs des 4 côtés et l'angle entre 2 cotés consécutifs  #3, #4 et #5 : paramètre d'affichage des sommets #6 inclinaison par rapport à l'horizontale 
% meme macro sans le tracé des segements
\newcommand{\Quadri}[6]{
\StrBefore{#1}{,}[\pointA]
\StrBetween[1,2]{#1}{,}{,}[\pointB]
\StrBetween[2,3]{#1}{,}{,}[\pointC]
\StrBehind[3]{#1}{,}[\pointD]
\StrBefore{#2}{,}[\coteA]
\StrBetween[1,2]{#2}{,}{,}[\coteB]
\StrBetween[2,3]{#2}{,}{,}[\coteC]
\StrBetween[3,4]{#2}{,}{,}[\coteD]
\StrBehind[4]{#2}{,}[\angle]
\tsegment{\pointA}{\coteA}{#6}{\pointB}{#3} 
\setcounter{tempangle}{#6}
\addtocounter{tempangle}{\angle}
\tsegment{\pointA}{\coteD}{\thetempangle}{\pointD}{#5}
\lsegment{\pointB}{\coteB}{0}{Z1}{PointSymbol=none, PointName=none}
\lsegment{\pointD}{\coteC}{0}{Z2}{PointSymbol=none, PointName=none}
\pstInterCC[PointNameA=none,PointSymbolA=none,#4]{\pointB}{Z1}{\pointD}{Z2}{Z3}{\pointC} 
\pstLineAB{\pointB}{\pointC}
\pstLineAB{\pointC}{\pointD}
}


% Définition des colonnes centrées ou à droite pour tabularx
\newcolumntype{Y}{>{\centering\arraybackslash}X}
\newcolumntype{Z}{>{\flushright\arraybackslash}X}

%Les pointillés à remplir par les élèves
\newcommand{\po}[1]{\makebox[#1 cm]{\dotfill}}
\newcommand{\lpo}[1][3]{%
\multido{}{#1}{\makebox[\linewidth]{\dotfill}
}}

%Liste des pictogrammes utilisés sur la fiche d'exercice ou d'activités
\newcommand{\bombe}{\faBomb}
\newcommand{\livre}{\faBook}
\newcommand{\calculatrice}{\faCalculator}
\newcommand{\oral}{\faCommentO}
\newcommand{\surfeuille}{\faEdit}
\newcommand{\ordinateur}{\faLaptop}
\newcommand{\ordi}{\faDesktop}
\newcommand{\ciseaux}{\faScissors}
\newcommand{\danger}{\faExclamationTriangle}
\newcommand{\out}{\faSignOut}
\newcommand{\cadeau}{\faGift}
\newcommand{\flash}{\faBolt}
\newcommand{\lumiere}{\faLightbulb}
\newcommand{\compas}{\dsmathematical}
\newcommand{\calcullitteral}{\faTimesCircleO}
\newcommand{\raisonnement}{\faCogs}
\newcommand{\recherche}{\faSearch}
\newcommand{\rappel}{\faHistory}
\newcommand{\video}{\faFilm}
\newcommand{\capacite}{\faPuzzlePiece}
\newcommand{\aide}{\faLifeRing}
\newcommand{\loin}{\faExternalLink}
\newcommand{\groupe}{\faUsers}
\newcommand{\bac}{\faGraduationCap}
\newcommand{\histoire}{\faUniversity}
\newcommand{\coeur}{\faSave}
\newcommand{\python}{\faPython}
\newcommand{\os}{\faMicrochip}
\newcommand{\rd}{\faCubes}
\newcommand{\data}{\faColumns}
\newcommand{\web}{\faCode}
\newcommand{\prog}{\faFile}
\newcommand{\algo}{\faCogs}
\newcommand{\important}{\faExclamationCircle}
\newcommand{\maths}{\faTimesCircle}
% Traitement des données en tables
\newcommand{\tables}{\faColumns}
% Types construits
\newcommand{\construits}{\faCubes}
% Type et valeurs de base
\newcommand{\debase}{{\footnotesize \faCube}}
% Systèmes d'exploitation
\newcommand{\linux}{\faLinux}
\newcommand{\sd}{\faProjectDiagram}
\newcommand{\bd}{\faDatabase}

%Les ensembles de nombres
\renewcommand{\N}{\mathbb{N}}
\newcommand{\D}{\mathbb{D}}
\newcommand{\Z}{\mathbb{Z}}
\newcommand{\Q}{\mathbb{Q}}
\newcommand{\R}{\mathbb{R}}
\newcommand{\C}{\mathbb{C}}

%Ecriture des vecteurs
\newcommand{\vect}[1]{\vbox{\halign{##\cr 
  \tiny\rightarrowfill\cr\noalign{\nointerlineskip\vskip1pt} 
  $#1\mskip2mu$\cr}}}


%Compteur activités/exos et question et mise en forme titre et questions
\newcounter{numact}
\setcounter{numact}{1}
\newcounter{numseance}
\setcounter{numseance}{1}
\newcounter{numexo}
\setcounter{numexo}{0}
\newcounter{numprojet}
\setcounter{numprojet}{0}
\newcounter{numquestion}
\newcommand{\espace}[1]{\rule[-1ex]{0pt}{#1 cm}}
\newcommand{\Quest}[3]{
\addtocounter{numquestion}{1}
\begin{tabularx}{\textwidth}{X|m{1cm}|}
\cline{2-2}
\textbf{\sffamily{\alph{numquestion})}} #1 & \dots / #2 \\
\hline 
\multicolumn{2}{|l|}{\espace{#3}} \\
\hline
\end{tabularx}
}
\newcommand{\QuestR}[3]{
\addtocounter{numquestion}{1}
\begin{tabularx}{\textwidth}{X|m{1cm}|}
\cline{2-2}
\textbf{\sffamily{\alph{numquestion})}} #1 & \dots / #2 \\
\hline 
\multicolumn{2}{|l|}{\cor{#3}} \\
\hline
\end{tabularx}
}
\newcommand{\Pre}{{\sc nsi} 1\textsuperscript{e}}
\newcommand{\Term}{{\sc nsi} Terminale}
\newcommand{\Sec}{2\textsuperscript{e}}
\newcommand{\Exo}[2]{ \addtocounter{numexo}{1} \ding{113} \textbf{\sffamily{Exercice \thenumexo}} : \textit{#1} \hfill #2  \setcounter{numquestion}{0}}
\newcommand{\Projet}[1]{ \addtocounter{numprojet}{1} \ding{118} \textbf{\sffamily{Projet \thenumprojet}} : \textit{#1}}
\newcommand{\ExoD}[2]{ \addtocounter{numexo}{1} \ding{113} \textbf{\sffamily{Exercice \thenumexo}}  \textit{(#1 pts)} \hfill #2  \setcounter{numquestion}{0}}
\newcommand{\ExoB}[2]{ \addtocounter{numexo}{1} \ding{113} \textbf{\sffamily{Exercice \thenumexo}}  \textit{(Bonus de +#1 pts maximum)} \hfill #2  \setcounter{numquestion}{0}}
\newcommand{\Act}[2]{ \ding{113} \textbf{\sffamily{Activité \thenumact}} : \textit{#1} \hfill #2  \addtocounter{numact}{1} \setcounter{numquestion}{0}}
\newcommand{\Seance}{ \rule{1.5cm}{0.5pt}\raisebox{-3pt}{\framebox[4cm]{\textbf{\sffamily{Séance \thenumseance}}}}\hrulefill  \\
  \addtocounter{numseance}{1}}
\newcommand{\Acti}[2]{ {\footnotesize \ding{117}} \textbf{\sffamily{Activité \thenumact}} : \textit{#1} \hfill #2  \addtocounter{numact}{1} \setcounter{numquestion}{0}}
\newcommand{\titre}[1]{\begin{Large}\textbf{\ding{118}}\end{Large} \begin{large}\textbf{ #1}\end{large} \vspace{0.2cm}}
\newcommand{\QListe}[1][0]{
\ifthenelse{#1=0}
{\begin{enumerate}[partopsep=0pt,topsep=0pt,parsep=0pt,itemsep=0pt,label=\textbf{\sffamily{\arabic*.}},series=question]}
{\begin{enumerate}[resume*=question]}}
\newcommand{\SQListe}[1][0]{
\ifthenelse{#1=0}
{\begin{enumerate}[partopsep=0pt,topsep=0pt,parsep=0pt,itemsep=0pt,label=\textbf{\sffamily{\alph*)}},series=squestion]}
{\begin{enumerate}[resume*=squestion]}}
\newcommand{\SQListeL}[1][0]{
\ifthenelse{#1=0}
{\begin{enumerate*}[partopsep=0pt,topsep=0pt,parsep=0pt,itemsep=0pt,label=\textbf{\sffamily{\alph*)}},series=squestion]}
{\begin{enumerate*}[resume*=squestion]}}
\newcommand{\FinListe}{\end{enumerate}}
\newcommand{\FinListeL}{\end{enumerate*}}

%Mise en forme de la correction
\newcommand{\cor}[1]{\par \textcolor{OliveGreen}{#1}}
\newcommand{\br}[1]{\cor{\textbf{#1}}}
\newcommand{\tcor}[1]{\begin{tcolorbox}[width=0.92\textwidth,colback={white},colbacktitle=white,coltitle=OliveGreen,colframe=green!75!black,boxrule=0.2mm]   
\cor{#1}
\end{tcolorbox}
}
\newcommand{\rc}[1]{\textcolor{OliveGreen}{#1}}
\newcommand{\pmc}[1]{\textcolor{blue}{\tt #1}}
\newcommand{\tmc}[1]{\textcolor{RawSienna}{\tt #1}}


%Référence aux exercices par leur numéro
\newcommand{\refexo}[1]{
\refstepcounter{numexo}
\addtocounter{numexo}{-1}
\label{#1}}

%Séparation entre deux activités
\newcommand{\separateur}{\begin{center}
\rule{1.5cm}{0.5pt}\raisebox{-3pt}{\ding{117}}\rule{1.5cm}{0.5pt}  \vspace{0.2cm}
\end{center}}

%Entête et pied de page
\newcommand{\snt}[1]{\lhead{\textbf{SNT -- La photographie numérique} \rhead{\textit{Lycée Nord}}}}
\newcommand{\Activites}[2]{\lhead{\textbf{{\sc #1}}}
\rhead{Activités -- \textbf{#2}}
\cfoot{}}
\newcommand{\Exos}[2]{\lhead{\textbf{Fiche d'exercices: {\sc #1}}}
\rhead{Niveau: \textbf{#2}}
\cfoot{}}
\newcommand{\Devoir}[2]{\lhead{\textbf{Devoir de mathématiques : {\sc #1}}}
\rhead{\textbf{#2}} \setlength{\fboxsep}{8pt}
\begin{center}
%Titre de la fiche
\fbox{\parbox[b][1cm][t]{0.3\textwidth}{Nom : \hfill \po{3} \par \vfill Prénom : \hfill \po{3}} } \hfill 
\fbox{\parbox[b][1cm][t]{0.6\textwidth}{Note : \po{1} / 20} }
\end{center} \cfoot{}}

%Devoir programmation en NSI (pas à rendre sur papier)
\newcommand{\PNSI}[2]{\lhead{\textbf{Devoir de {\sc nsi} : \textsf{ #1}}}
\rhead{\textbf{#2}} \setlength{\fboxsep}{8pt}
\begin{tcolorbox}[title=\textcolor{black}{\danger\; A lire attentivement},colbacktitle=lightgray]
{\begin{enumerate}
\item Rendre tous vous programmes en les envoyant par mail à l'adresse {\tt fnativel2@ac-reunion.fr}, en précisant bien dans le sujet vos noms et prénoms
\item Un programme qui fonctionne mal ou pas du tout peut rapporter des points
\item Les bonnes pratiques de programmation (clarté et lisiblité du code) rapportent des points
\end{enumerate}
}
\end{tcolorbox}
 \cfoot{}}


%Devoir de NSI
\newcommand{\DNSI}[2]{\lhead{\textbf{Devoir de {\sc nsi} : \textsf{ #1}}}
\rhead{\textbf{#2}} \setlength{\fboxsep}{8pt}
\begin{center}
%Titre de la fiche
\fbox{\parbox[b][1cm][t]{0.3\textwidth}{Nom : \hfill \po{3} \par \vfill Prénom : \hfill \po{3}} } \hfill 
\fbox{\parbox[b][1cm][t]{0.6\textwidth}{Note : \po{1} / 10} }
\end{center} \cfoot{}}

\newcommand{\DevoirNSI}[2]{\lhead{\textbf{Devoir de {\sc nsi} : {\sc #1}}}
\rhead{\textbf{#2}} \setlength{\fboxsep}{8pt}
\cfoot{}}

%La définition de la commande QCM pour auto-multiple-choice
%En premier argument le sujet du qcm, deuxième argument : la classe, 3e : la durée prévue et #4 : présence ou non de questions avec plusieurs bonnes réponses
\newcommand{\QCM}[4]{
{\large \textbf{\ding{52} QCM : #1}} -- Durée : \textbf{#3 min} \hfill {\large Note : \dots/10} 
\hrule \vspace{0.1cm}\namefield{}
Nom :  \textbf{\textbf{\nom{}}} \qquad \qquad Prénom :  \textbf{\prenom{}}  \hfill Classe: \textbf{#2}
\vspace{0.2cm}
\hrule  
\begin{itemize}[itemsep=0pt]
\item[-] \textit{Une bonne réponse vaut un point, une absence de réponse n'enlève pas de point. }
\item[\danger] \textit{Une mauvaise réponse enlève un point.}
\ifthenelse{#4=1}{\item[-] \textit{Les questions marquées du symbole \multiSymbole{} peuvent avoir plusieurs bonnes réponses possibles.}}{}
\end{itemize}
}
\newcommand{\DevoirC}[2]{
\renewcommand{\footrulewidth}{0.5pt}
\lhead{\textbf{Devoir de mathématiques : {\sc #1}}}
\rhead{\textbf{#2}} \setlength{\fboxsep}{8pt}
\fbox{\parbox[b][0.4cm][t]{0.955\textwidth}{Nom : \po{5} \hfill Prénom : \po{5} \hfill Classe: \textbf{1}\textsuperscript{$\dots$}} } 
\rfoot{\thepage} \cfoot{} \lfoot{Lycée Nord}}
\newcommand{\DevoirInfo}[2]{\lhead{\textbf{Evaluation : {\sc #1}}}
\rhead{\textbf{#2}} \setlength{\fboxsep}{8pt}
 \cfoot{}}
\newcommand{\DM}[2]{\lhead{\textbf{Devoir maison à rendre le #1}} \rhead{\textbf{#2}}}

%Macros permettant l'affichage des touches de la calculatrice
%Touches classiques : #1 = 0 fond blanc pour les nombres et #1= 1gris pour les opérations et entrer, second paramètre=contenu
%Si #2=1 touche arrondi avec fond gris
\newcommand{\TCalc}[2]{
\setlength{\fboxsep}{0.1pt}
\ifthenelse{#1=0}
{\psframebox[fillstyle=solid, fillcolor=white]{\parbox[c][0.25cm][c]{0.6cm}{\centering #2}}}
{\ifthenelse{#1=1}
{\psframebox[fillstyle=solid, fillcolor=lightgray]{\parbox[c][0.25cm][c]{0.6cm}{\centering #2}}}
{\psframebox[framearc=.5,fillstyle=solid, fillcolor=white]{\parbox[c][0.25cm][c]{0.6cm}{\centering #2}}}
}}
\newcommand{\Talpha}{\psdblframebox[fillstyle=solid, fillcolor=white]{\hspace{-0.05cm}\parbox[c][0.25cm][c]{0.65cm}{\centering \scriptsize{alpha}}} \;}
\newcommand{\Tsec}{\psdblframebox[fillstyle=solid, fillcolor=white]{\parbox[c][0.25cm][c]{0.6cm}{\centering \scriptsize 2nde}} \;}
\newcommand{\Tfx}{\psdblframebox[fillstyle=solid, fillcolor=white]{\parbox[c][0.25cm][c]{0.6cm}{\centering \scriptsize $f(x)$}} \;}
\newcommand{\Tvar}{\psframebox[framearc=.5,fillstyle=solid, fillcolor=white]{\hspace{-0.22cm} \parbox[c][0.25cm][c]{0.82cm}{$\scriptscriptstyle{X,T,\theta,n}$}}}
\newcommand{\Tgraphe}{\psdblframebox[fillstyle=solid, fillcolor=white]{\hspace{-0.08cm}\parbox[c][0.25cm][c]{0.68cm}{\centering \tiny{graphe}}} \;}
\newcommand{\Tfen}{\psdblframebox[fillstyle=solid, fillcolor=white]{\hspace{-0.08cm}\parbox[c][0.25cm][c]{0.68cm}{\centering \tiny{fenêtre}}} \;}
\newcommand{\Ttrace}{\psdblframebox[fillstyle=solid, fillcolor=white]{\parbox[c][0.25cm][c]{0.6cm}{\centering \scriptsize{trace}}} \;}

% Macroi pour l'affichage  d'un entier n dans  une base b
\newcommand{\base}[2]{ \overline{#1}^{#2}}
% Intervalle d'entiers
\newcommand{\intN}[2]{\llbracket #1; #2 \rrbracket}}

% Numéro et titre de chapitre
\setcounter{numchap}{2}
\newcommand{\Ctitle}{\cnum {Discipline de programmation}}
\newcommand{\SPATH}{/home/fenarius/Travail/Cours/cpge-info/docs/mp2i/files/C\thenumchap/}

\makess{Algorithme, programme, paradigme}
\begin{frame}{\Ctitle}{\stitle}
	\begin{block}{Définitions}
		\begin{itemize}
			\item<1-> Un \textcolor{blue}{algorithme} est une suite d'instructions et d'opérations permettant de résoudre un problème. \\
			\onslide<2->{\textcolor{gray}{Par exemple pour résoudre le problème du tri d'une liste de valeurs, on peut utiliser l'algorithme du tri par insertion. Cela consiste à prendre une à une chaque valeur et à l'insérer au bon emplacement dans une liste initialement vide.}}
			\item<3-> Un \textcolor{blue}{programme} est la traduction d'un algorithme dans un langage informatique. On dit qu'un programme est la mise en oeuvre d'un algorithme.\\
			\onslide<4->{\textcolor{gray}{L'algorithme du tri par insertion peut être écrit en Python, en C, \dots}}
			\item<5-> Un \textcolor{blue}{paradigme} de programmation est une façon de d'approcher un problème et d'en concevoir et  modéliser une solution.\\
			\onslide<6->{\textcolor{gray}{Le langage C est une illustration du paradigme de programmation impérative. Le langage OCaml nous permettra d'illustrer le paradigme fonctionnel}}
		\end{itemize}
	\end{block}
\end{frame}

\makess{Spécification}
\begin{frame}{\Ctitle}{\stitle}
	\begin{exampleblock}{Exemple introductif}
		On considère le programme C suivant :
		\inputpartC{/home/fenarius/Travail/Cours/cpge-info/docs/mp2i/files/C2/get_max.c}{}{\small}{3}{10}
	\begin{enumerate}
		\item<1-> Quel est le résultat renvoyé si le tableau fourni en argument contient dans cet ordre les valeurs : {\tt 12, 18, 11, 9, 10} ?
		\item<2-> Même question avec le tableau  {\tt 12, 18, 11, 18, 10}
		\item<3-> Même question avec le tableau {\tt -12, -15, -7}
		\item<4-> Même question si le tableau est vide (c'est-à-dire que {\tt size =0})
	\end{enumerate}
	\end{exampleblock}
\end{frame}

\begin{frame}{\Ctitle}{\stitle}
	\begin{exampleblock}{Correction}
		\begin{enumerate}
			\item<1-> \textcolor{OliveGreen} {La fonction renvoie 1 (indice de l'élément maximal du tableau)}
			\item<2-> \textcolor{OliveGreen}{la fonction renvoie 3, c'est l'indice de la dernière occurence du maximum des éléments du tablea }
			\item<3-> \textcolor{OliveGreen}{C'est un comportement indéfini, la variable {\tt imax} n'est pas initialisée.}
			\item<4-> \textcolor{OliveGreen}{Une nouvelle fois, le comportement est indéfini car on renvoie la variable {\tt imax} qui n'a jamais été initialisée.}
		\end{enumerate}
	\end{exampleblock}
	\begin{alertblock}{Définition}
		\onslide<5->{Ecrire la \textcolor{blue}{spécification} d'une fonction c'est donné une description formelle et détaillée de ses caractéristiques. En particulier :}
		\begin{itemize}
			\item<6-> les entrées admissibles : types et valeurs possibles des arguments (\textcolor{blue}{préconditions}), 
			\item<7-> ce que renvoie la fonction  et les  les effets de bords (\textit{side effects}) éventuels : modification des arguments, affichage, \dots . Ce sont les (\textcolor{blue}{postconditions}).
		\end{itemize}
	\end{alertblock}
\end{frame}

\begin{frame}{\Ctitle}{\stitle}
\begin{block}{Remarques}
\begin{itemize}
\item<1-> La spécification est souvent donnée en commentaire dans le code source. \\
\onslide<2-> \textcolor{gray}{Les commentaires en C s'écrivent entre \kw{/*} et \kw{*/} et en OCaml entre \kw{(*} et \kw{*)}}
\item<3-> La vérification des préconditions peut s'effectuer à l'aide d'instructions \kw{assert}. Elles sont de la forme \kw{assert (condition)} en C, et nécessitent d'importer \kw{assert.h}. Si la condition échoue le programme s'arrête et affiche une erreur, c'est ce qu'on appelle de la programmation défensive (anticipation des erreurs).
\end{itemize}
\end{block}
\begin{exampleblock}{Exemples}
\onslide<4->{Ecrire en C, une fonction qui :}
\begin{itemize}
	\item<5-> Accepte en argument un tablau \textit{non vide} d'entiers.
	\item<6-> Renvoie l'indice de la première occurence du maximum des éléments de ce tableau.
\end{itemize}
\end{exampleblock}
\end{frame}


\begin{frame}{\Ctitle}{\stitle}
	\begin{exampleblock}{Correction}
		\inputpartC{/home/fenarius/Travail/Cours/cpge-info/docs/mp2i/files/C2/get_max_ok.c}{}{\small}{2}{14}
	\end{exampleblock}
\end{frame}

\makess{Validation, test}
\begin{frame}{\Ctitle}{\stitle}
	\begin{block}{Jeu de tests}
		\onslide<1->Le comportement correct d'une fonction peut être "validé" (mais pas prouvé), par l'utilisation d'un \textcolor{blue}{jeu de test}. c'est-à-dire un ensemble de couple d'entrées du programme et de sorties attendues. \\
		\onslide<2->\textcolor{gray}{Dans la fonction précédente, on pourrait tester (entre autres) des cas limites (\textit{edge cases}), comme par exemple un tableau à un seul élément, ou vide ou des situations ou le maximum se trouve en première ou dernière position du tableau.}
	\end{block}
	\begin{exampleblock}{Exemple}
		\begin{enumerate}
		\item<3-> Ecrire une fonction {\tt contient\_double} qui prend en argument un tableau d'entiers et renvoie {\tt true} si ce tableau contient deux éléments consécutifs égaux.
		\item<4-> Proposer un jeu de tests pour cette fonction.
		\end{enumerate}
	\end{exampleblock}
\end{frame}


\begin{frame}{\Ctitle}{\stitle}
	\begin{exampleblock}{Correction}
		\begin{enumerate}
		\item<1-> \ \\ \inputpartC{/home/fenarius/Travail/Cours/cpge-info/docs/mp2i/files/C2/contient_double.c}{}{\small}{4}{8}
		\item<2-> On pourrait tester les cas suivants :
		\begin{itemize}
			\item<3-> Tableau vide ou à un seul élément
			\item<4-> Même élément mais non consécutifs
			\item<5-> Elément présent en plus de deux exemplaires consécutifs
			\item<6-> Présence du double en tout début ou toute fin de tableau
		\end{itemize}
		\end{enumerate}
	\end{exampleblock}
\end{frame}

\makess{Graphe de flot de contôle}
\begin{frame}{\Ctitle}{\stitle}
	\begin{exampleblock}{Exemple introductif}
		Ecrire la fonction {\tt duree\_vol} qui prend en argument un entier positif {\tt n} et renvoie le nombre d'itérations nécessaires avant d'atteindre 1 en prenant cet entier comme valeur initiale dans la suite de syracuse. On rappelle que : \\
		$s_{n+1} = \left\{ \begin{array}{ll} \dfrac{s_n}{2} & \mathrm{\ si\ } n \mathrm{\ est \ paire} \\ 3s_n+1 & \mathrm{\ sinon} \end{array}\right.$
		\onslide<2->\inputpartC{/home/fenarius/Travail/Cours/cpge-info/docs/mp2i/files/C2/syracuse.c}{}{\small}{3}{11}
	\end{exampleblock}
\end{frame}



\begin{frame}{\Ctitle}{\stitle}
	\begin{exampleblock}{Représentation des exécutions possibles}
		\vspace{6.2cm}
		\rput(5,6){\circlenode[linecolor=red]{E}{\textcolor{red}{E}}}
		\rput(5,5){\rnode{I}{\psframebox{\tt dvol=0}}}
		\rput(5,4){\ovalnode[linecolor=blue]{W}{\kw{n==1}}}
		\rput(7,4){\circlenode[linecolor=red]{S}{\textcolor{red}{S}}}
		\ncline{->}{W}{S} \naput[labelsep=1pt]{\small \textcolor{OliveGreen}{V}}
		\ncline{->}{E}{I}
		\ncline{->}{I}{W}
		\rput(5,3){\ovalnode[linecolor=blue]{T}{\kw {n\%2==0}}}
		\rput(5,1){\rnode{D}{\psframebox{\tt dvol++}}}
		\ncline{->}{W}{T} \naput[labelsep=1pt]{\small \textcolor{OrangeRed}{F}}
		\rput(3,2){\rnode{PA}{\psframebox{\tt n=n/2}}}
		\rput(7,2){\rnode{IM}{\psframebox{\tt n=3n+1}}}
		\ncline{->}{T}{PA} \naput[labelsep=1pt]{\small \textcolor{OliveGreen}{V}}
		\ncline{->}{T}{IM} \nbput[labelsep=1pt]{\small \textcolor{OrangeRed}{F}}
		\ncline{->}{PA}{D}
		\ncline{<-}{D}{IM}
		\ncbar[angle=180,arm=2.5cm]{->}{D}{W}
	\end{exampleblock}
\end{frame}

\begin{frame}{\Ctitle}{\stitle}
	\begin{block}{Définitions}
		Le \textcolor{blue}{graphe de flot de contrôle} représente les exécutions possibles d'un programme.
		\begin{itemize}
			\item<1-> Les sommets \kw{E} et \kw{S} représentent l'entrée et la sortie
			\item<2-> Les autres noeuds sont les blocs d'instructions
			\item<3-> On dessine un arc entre deux noeuds {\tt A} et {\tt B} lorsque l'exécution de {\tt B} peut suivre celle de {\tt A}
			\item<4-> Les noeuds quittant les instructions conditionnelle sont étiquetés par {\tt V} ou {\tt F}.
			\item<5-> On dira qu'un chemin dans le graphe est \textcolor{blue}{faisable} lorsqu'il existe des valeurs d'entrées pour lesquels l'exécution passe par tous les noeuds de ce chemin.
		\end{itemize}
	\end{block}
	\begin{alertblock}{Définition}
		\onslide<6->{On dit qu'un jeu de test couvre tous les sommets (resp. tous les arcs) lorsque son exécution permet de passer par tous les noeuds (resp. tous les arcs)}
	\end{alertblock}
\end{frame}

\begin{frame}{\Ctitle}{\stitle}
	\begin{exampleblock}{Exemple}
		\begin{enumerate}
			\item<1-> Montrer que Le jeu de test \kw{\{n=1, n=8\}} ne couvre ni tous les arcs, ni tous les sommets. \\
			\onslide<2->\textcolor{OliveGreen}{On ne passe jamais par le noeud représentant l'instruction {\tt n=3*n+1}, en effet pour {\tt n=1} on atteint directement la sortie {\tt S} et pour {\tt n=8}, tous les valeurs de la suite sont paires.}
			\item<3-> Proposer un jeu de test permettant de couvrir tous les arcs. \\
			\onslide<4->\textcolor{OliveGreen}{On peut ajouter par exemple le test \kw{n=5}}
		\end{enumerate}
	\end{exampleblock}
	\begin{exampleblock}{Exercice}
		\begin{enumerate}
			\item<5-> Ecrire une fonction prenant en argument un flottant {\tt x} et un entier positif ou nul {\tt n} et qui renvoie {\tt x} puissance {\tt n}.
			\item<6-> Tracer son graphe de flot de contrôle.
			\item<7-> Proposer un jeu de test.
		\end{enumerate}
	\end{exampleblock}
\end{frame}



\begin{frame}{\Ctitle}{\stitle}
	\begin{exampleblock}{Correction} 
		\begin{tabularx}{\textwidth}{XX}
		\begin{minipage}{0.6\textwidth}
			\vspace{0.5cm}
			\inputpartC{/home/fenarius/Travail/Cours/cpge-info/docs/mp2i/files/C2/puissance.c}{}{\small}{5}{10}
			\vspace{0.8cm}
		\end{minipage} &
	\end{tabularx}
	\end{exampleblock}
	\end{frame}


	\begin{frame}{\Ctitle}{\stitle}
		\begin{exampleblock}{Correction} 
			\begin{tabularx}{\textwidth}{XX}
			\begin{minipage}{0.6\textwidth}
				\vspace{0.5cm}
				\inputpartC{/home/fenarius/Travail/Cours/cpge-info/docs/mp2i/files/C2/puissance.c}{}{\small}{5}{10}
				\vspace{0.8cm}
			\end{minipage} &
			\rput(3,1.8){\circlenode[linecolor=red]{E}{\textcolor{red}{E}}}
			\rput(3,0.8){\rnode{I}{\psframebox{\begin{tabular}{>{\tt}l}  xn=1.0 \\ i=1 \end{tabular}}}}
			\rput(3,-0.2){\ovalnode[linecolor=blue]{W}{\kw{i<=n}}}
			\rput(5,-0.2){\circlenode[linecolor=red]{S}{\textcolor{red}{S}}}
			\rput(3,-1.4){\rnode{D}{\psframebox{\begin{tabular}{>{\tt}l}  xn=xn*x \\ i=i+1 \end{tabular}}}}
			\ncline{->}{E}{I}
			\ncline{->}{I}{W}
			\ncline{->}{W}{S} \naput[labelsep=1pt]{\small \textcolor{OrangeRed}{F}}
			\ncline{->}{W}{D} \naput[labelsep=1pt]{\small \textcolor{OliveGreen}{V}}
			\ncbar[angle=180]{->}{D}{W}
		\end{tabularx}
		\end{exampleblock}
		\end{frame}

\begin{frame}{\Ctitle}{\stitle}
\begin{exampleblock}{Correction} 
	\begin{tabularx}{\textwidth}{XX}
	\begin{minipage}{0.6\textwidth}
		\vspace{0.5cm}
		\inputpartC{/home/fenarius/Travail/Cours/cpge-info/docs/mp2i/files/C2/puissance.c}{}{\small}{5}{10}
		\vspace{0.8cm}
	\end{minipage} &
	\rput(3,1.8){\circlenode[linecolor=red]{E}{\textcolor{red}{E}}}
	\rput(3,0.8){\rnode{I}{\psframebox{\begin{tabular}{>{\tt}l}  xn=1.0 \\ i=1 \end{tabular}}}}
	\rput(3,-0.2){\ovalnode[linecolor=blue]{W}{\kw{i<=n}}}
	\rput(5,-0.2){\circlenode[linecolor=red]{S}{\textcolor{red}{S}}}
	\rput(3,-1.4){\rnode{D}{\psframebox{\begin{tabular}{>{\tt}l}  xn=xn*x \\ i=i+1 \end{tabular}}}}
	\ncline{->}{E}{I}
	\ncline{->}{I}{W}
	\ncline{->}{W}{S} \naput[labelsep=1pt]{\small \textcolor{OrangeRed}{F}}
	\ncline{->}{W}{D} \naput[labelsep=1pt]{\small \textcolor{OliveGreen}{V}}
	\ncbar[angle=180]{->}{D}{W}
\end{tabularx}
On peut proposer le jeu de test suivant :
\begin{itemize}
  \item<1-> \kw{x=2.0, n=0}
  \item<2-> \kw{x=2.0, n=3}
\end{itemize}
\end{exampleblock}
\end{frame}

\makess{Test exhaustif d'une conditionnelle}
\begin{frame}{\Ctitle}{\stitle}
	\begin{block}{Remarque}
		Lorsqu'une instruction conditionnelle comporte des conjonctions ou des disjonctions, on peut formuler des tests qui prévoient toutes les possibilités de la satisfaire.
	\end{block} 
	\begin{exampleblock}{Exemple}
		Une année est bissextile si elle est divisible par 4 mais pas par 100 ou s'il est divisible par 400.
		\begin{enumerate}
			\item<1-> Ecrire une fonction bissextile qui prend en argument un entier positif {\tt annee} et renvoie \kw{true} si l'année est bissextile et \kw{false} sinon.
			\item<2-> Proposer un jeu de test pour cette fonction qui prévoit toutes les possibilités de satisfaire la condition d'être bissextile.
		\end{enumerate}
	\end{exampleblock}
\end{frame}

\begin{frame}[fragile]{\Ctitle}{\stitle}
\begin{exampleblock}{Correction}
	\begin{enumerate}
	\item<1-> \ \\
	\begin{langageC}
bool bissextile(int annee) {
	return ((annee%4==0 && annee%100!=0) || (annee%400==0))
}
	\end{langageC}
	\item<2-> On peut proposer les tests suivants :
	\begin{itemize}
		\item<3-> \kw{n=2004} qui permet de valider la première condition
		\item<4-> \kw{n=2000} qui permet de valider la deuxième
	\end{itemize}
\end{enumerate}
\end{exampleblock}
\end{frame}

\makess{Terminaison d'un algorithme}
\begin{frame}[fragile]{\Ctitle}{\stitle}
	\begin{block}{Exemple introductif}
	\SetAlFnt{\small}
	\setlength{\algomargin}{8pt}
	\begin{algorithm}[H]
		\DontPrintSemicolon
		\caption{Multiplier sans utiliser {\tt *}}
		\Entree{$n \in \N, m \in \N$}
		\Sortie{$nm$}
		\everypar={\footnotesize \textcolor{gray}{\nl}}
		$r \leftarrow 0$\;
		\Tq{$m >0$}{
		$m \leftarrow m-1$ \;
		$r \leftarrow r+n$ \;
		}
		\Return $r$
	  \end{algorithm}
	  \begin{enumerate}
	  \item<2-> Donner une implémentation en C de cet algorithme.
	  \item<3-> Montrer que dans la boucle "tant que":
	  \begin{itemize}
		\item<3-> $m$ ne prend que des valeurs entières
		\item<4-> $m$ prend des valeurs positives
		\item<5-> les valeurs prises par $m$ sont \textit{strictement} décroissantes.
	  \end{itemize}
	  \end{enumerate}
	\end{block}
\end{frame}

\begin{frame}[fragile]{\Ctitle}{\stitle}
	\begin{block}{Définitions}
    \begin{itemize}
        \item<2-> On dit qu'un algorithme \textcolor{blue}{termine} lorsqu'il renvoie un résultat en un nombre fini d'étapes quels que soient les valeurs des entrées.
        \item<3-> Un \textcolor{blue}{variant de boucle} est une quantité :
            \begin{enumerate}
            \item<4-> à valeurs entières,
            \item<5-> positives,
            \item<6-> qui décroît \textit{strictement} à chaque passage dans la boucle.
            \end{enumerate}
		\end{itemize}
	\end{block}
	\onslide<7->
	{\begin{alertblock}{\textcolor{yellow}{\important \;} Propriété}
		Si une boucle admet un variant, alors cette boucle termine.
	\end{alertblock}}
	\onslide<8->
	{
		\begin{exampleblock}{Exemple}
			$m$ est un variant de boucle de l'algorithme de multiplication ci-dessus et donc cet algorithme termine.
		\end{exampleblock}
	}
\end{frame}

\begin{frame}[fragile]{\Ctitle}{\stitle}
	\begin{exampleblock}{Exemple}
	\SetAlFnt{\small}
	\setlength{\algomargin}{8pt}
	\begin{algorithm}[H]
		\DontPrintSemicolon
		\caption{Nombre de chiffres en base 10}
		\Entree{$n \in \N$}
		\Sortie{$p$ nombre de chiffres de $n$ dans son écriture en base 10.}
		\everypar={\footnotesize \textcolor{gray}{\nl}}
		\Si{$n=0$}{\Return 1}
		$p \leftarrow 0$\;
		\Tq{$n >0$}{
		$p \leftarrow p+1$ \;
		$n \leftarrow \lfloor \frac{n}{10} \rfloor $ \;
		}
		\Return $p$
	  \end{algorithm}
	  \begin{enumerate}
	  \item<2-> Donner une implémentation en C de cet algorithme.
	  \item<3-> Prouver la terminaison de cet algorithme.
	  \end{enumerate}
	\end{exampleblock}
\end{frame}

\begin{frame}{\Ctitle}{\stitle}
	\begin{exampleblock}{Implémentation en C}
		\inputpartC{\SPATH/nbchiffres.c}{}{\small}{4}{18}
	\end{exampleblock}
\end{frame}

\begin{frame}{\Ctitle}{\stitle}
	\begin{exampleblock}{Preuve de terminaison}
		Montrons que $n$ est un variant de la boucle {\tt tant que} de l'algorithme :
		\begin{enumerate}
		\item<1-> $n$ ne prend que des valeurs entières, en effet, $n \in \N$  en entrée et $\PE{\dfrac{n}{10}}$ est entier.
		\item<2-> $n$ est positif  par condition d'entrée dans la boucle. 
		\item<3-> $n$ décroît strictement à chaque passage dans la boucle car comme $n > 0$, $\PE{\dfrac{n}{10}} < n$
		\end{enumerate}
	\end{exampleblock}
\end{frame}

\begin{frame}{\Ctitle}{\stitle}
	\begin{exampleblock}{Exercice}
		\begin{enumerate}
		\item<1-> Ecrire un algorithme qui prend en entrée un entier $n>1$ et renvoie le premier diviseur strictement supérieur à 1 de cet entier. Par exemple pour $n=7$ l'algorithme renvoie 7 et pour $n=15$, l'algorithme renvoie $3$.
		\item<2-> Donner une implémentation en C de cet algorithme sous la forme d'une fonction dont on précisera soigneusement la spécification.
		\item<3-> Prouver la terminaison de cet algorithme.
		\end{enumerate}
	\end{exampleblock}
\end{frame}

\makess{Correction d'un algorithme}
\begin{frame}[fragile]{\Ctitle}{\stitle}
	\begin{block}{Exemple introductif}
	\SetAlFnt{\small}
	\setlength{\algomargin}{8pt}
	\begin{algorithm}[H]
		\DontPrintSemicolon
		\caption{Multiplier sans utiliser {\tt *}}
		\Entree{$n \in \N, m \in \N$}
		\Sortie{$nm$}
		\everypar={\footnotesize \textcolor{gray}{\nl}}
		$r \leftarrow 0$\;
		\Tq{$m >0$}{
		$m \leftarrow m-1$ \;
		$r \leftarrow r+n$ \;
		}
		\Return $r$
	  \end{algorithm}
	  On note $m_0$ la valeur initiale de $m$ et on considère la propriété suivante suivante noté $I$ : \og{} $r = (m_0-m)n$ \fg{}. Montrer que cette propriété est vraie :
	  \begin{enumerate}
	  \item<2-> avant d'entrée dans la boucle,
	  \item<3-> qu'elle reste vraie à chaque tour de boucle.
	  \end{enumerate}
	  \onslide<4->{Que peut-on en conclure ?}
	\end{block}
\end{frame}

\begin{frame}[fragile]{\Ctitle}{\stitle}
	\begin{block}{Définitions}
    \begin{itemize}
        \item<1-> Un \textcolor{blue}{invariant de boucle} est une propriété qui :
            \begin{itemize}
            \item<2-> est vraie à l'entrée dans la boucle (\textcolor{blue}{initialisation}),
            \item<3-> reste vraie à chaque itération si elle l'était à l'itération précédente (\textcolor{blue}{conservation}).
            \end{itemize}
		\item<4-> Un algorithme est dit \textcolor{blue}{partiellement correct} lorsqu'il renvoie la réponse attendue quand il se termine.
		\item<5-> Un algorithme est dit \textcolor{blue}{totalement correct} lorsqu'il est partiellement correcte et que sa terminaison est prouvée.
		\end{itemize}
	\end{block}
	\onslide<7->
	{\begin{alertblock}{\textcolor{yellow}{\important \;} Prouver la correction d'un algorithme}
		On utilise un invariant de boucle qui en sortie de boucle fournit une propriété permettant de montrer la correction.\\
		\textcolor{gray}{La méthode est similaire à une récurrence mathématique.}
	\end{alertblock}}
\end{frame}


\begin{frame}[fragile]{\Ctitle}{\stitle}
	\begin{exampleblock}{Exemple}
	\SetAlFnt{\small}
	\setlength{\algomargin}{8pt}
	\begin{algorithm}[H]
		\DontPrintSemicolon
		\caption{Nombre de chiffres en base 10}
		\Entree{$n \in \N$}
		\Sortie{$p$ nombre de chiffres de $n$ dans son écriture en base 10.}
		\everypar={\footnotesize \textcolor{gray}{\nl}}
		\Si{$n=0$}{\Return 1}
		$p \leftarrow 0$\;
		\Tq{$n >0$}{
		$p \leftarrow p+1$ \;
		$n \leftarrow \lfloor \frac{n}{10} \rfloor $ \;
		}
		\Return $p$
	  \end{algorithm}
	  Prouver que cet algorithme est correct.
	\end{exampleblock}
\end{frame}

\begin{frame}[fragile]{\Ctitle}{\stitle}
\begin{exampleblock}{Correction}
	\textcolor{gray}{Idée : à chaque tour de boucle, on incrémente $p$ et $n$ perd un chiffre. La somme de $p$ et du nombre de chiffres de $n$ est donc constante. C'est l'invariant de boucle ! \\ \medskip}
	\onslide<2->{Le cas $n=0$ est trivial, on note $n_0$ la valeur initiale de $n$ et on suppose donc $n_0>0$,  on considère la propriété $I$ : \og{} $c(n_0) = p + c(n)$ \fg{}. Où $c(m)$ vaut 0 si $m=0$ et le nombre de chiffres de $m$ en base 10 sinon.}
	\begin{itemize}
	\item<3-> initialisation : $I$ est vraie avant d'entrer dans la boucle puisque $p=0$ et $n=n_0$.
	\item<4-> conservation : on suppose $I$ vraie du début d'un tour de boucle et  on note $n'$ (resp. $p'$) la valeur de $n$ (resp $p'$) après ce tour de boucle  \\
	\onslide<5->{$p' + c(n') = p + 1 + c\left(\PE{\dfrac{n}{10}}\right)$ \\}
	\onslide<6->{$p' + c(n') = p + 1 + c\left(n\right) - 1$ \\}
	\onslide<7->{$p' + c(n') = c(n_0)$, par $I$ est vraie à l'entrée de la boucle.}
	\end{itemize}
	\onslide<8->{En sortie de boucle, on a  $n'=0$ et donc $p = c(n_0)$ et l'algorithme est correct.}
\end{exampleblock}
\end{frame}

\begin{frame}[fragile]{\Ctitle}{\stitle}
	\begin{exampleblock}{Exemple}
	\SetAlFnt{\small}
	\setlength{\algomargin}{8pt}
	\begin{algorithm}[H]
		\DontPrintSemicolon
		\caption{Premier diviseur}
		\Entree{$n \in \N, n > 1$}
		\Sortie{$d$ premier diviseur de $n$ strictement supérieur à $1$.}
		\everypar={\footnotesize \textcolor{gray}{\nl}}
		$d \leftarrow 2$\;
		\Tq{$n \mod d \neq 0$}{
		$d \leftarrow d+1$ \;
		}
		\Return $d$
	  \end{algorithm}
	  Prouver que cet algorithme est correct.
	\end{exampleblock}
\end{frame}

\end{document}


\documentclass[11pt,a4paper]{article}

\usepackage{Act}

\begin{document}
\input{\detokenize{/home/fenarius/Travail/Cours/cpge-info/latex/Macros.tex}}
\ModeExercice
\IC{1}{Récursivité}
\newcommand{\SPATH}{\FPATH Evaluations/IC/IC2/}

\setcounter{Exercise}{0}
\begin{Exercise}[title={Compte à rebours}]
	\Question{Rappeler la définition d'une fonction récursive}
    \tcor{Une fonctioin récursive est une fonction qui s'appelle elle-même.}
    \Question{Ecrire en C, une fonction itérative {\tt compte\_rebours} qui prend en argument un entier {\tt n} ne renvoie rien et affiche les entiers de {\tt n} à {\tt 0} puis {\tt "Partez !"}}
    \inputpartC{compte_rebours.c}{}{}{3}{10}
    \Question{Ecrire une version récursive de cette fonction qu'on appelera {\tt compte\_rebours\_rec}}
	\inputpartC{compte_rebours.c}{}{}{12}{23}
\end{Exercise}

\begin{Exercise}[title = {un peu de OCaml}]\\
	On donne la définition de la fonction {\tt mystere} en OCaml :
	\inputpartOCaml{\SPATH ex2.ml}{}{}{1}{2}
	\Question{Quel type est automatiquement inféré pour {\tt n} ? Pourquoi ?}
	\tcor{{\tt n} est de type entier car la division {\tt /} s'effectue forcément entre deux entiers (pour les flottants la division est {\tt /.}).}
	\Question{Donner les résultat des appels suivants : {\tt mystere 7}, {\tt mystere 42}, {\tt mystere 666}, {\tt mystere 2023} en complétant le tableau ci-dessous} \\
	\begin{tabularx}{\linewidth}{|X|}
		\hline
		\renewcommand{\arraystretch}{2}
		\begin{tabular}{c|p{1cm}|p{1cm}|p{1cm}|p{1cm}}
			\hline
			{\tt n} & 7 & 42 & 666 & 2023 \\
			\hline
			{\tt mystere n} & \cor{1} & \cor{2} & \cor{3} & \cor{4}  \\
			\hline
		\end{tabular} \\
		\hline
	\end{tabularx}
	\Question{Proposer une spécification et un nom plus adapté pour cette fonction.}
	\tcor{Cette fonction calcule le nombre de chiffres en base 10 de l'entier positif {\tt n} donné en argument, on peut la renommer {\tt nb\_chiffres}}
\end{Exercise}

\end{document}

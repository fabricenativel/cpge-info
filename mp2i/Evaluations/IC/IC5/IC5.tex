\PassOptionsToPackage{dvipsnames,table}{xcolor}
\documentclass[11pt,a4paper]{article}

\usepackage{Act}

\begin{document}
\input{\detokenize{/home/fenarius/Travail/Cours/cpge-info/latex/Macros.tex}}
\ModeExercice
\IC{5}{Récursivité}
\setboolean{corrige}{false}


\setcounter{Exercise}{0}

\begin{Exercise}[title={Compte à rebours}]
	\Question{Rappeler la définition d'une fonction récursive}
    \reponse{1}{1}
    \tcor{Une fonction récursive est une fonction qui s'appelle elle-même.}
    \Question{Ecrire en C, une fonction itérative {\tt compte\_rebours} qui prend en argument un entier {\tt n} ne renvoie rien et affiche les entiers de {\tt n} à {\tt 0} puis {\tt "Partez !"}}
    \reponse{5}{2}
    \ifcorrige
    \inputpartC{rebours.c}{}{\small}{3}{10}
    \fi
    \Question{Ecrire une version récursive de cette fonction qu'on appelera {\tt compte\_rebours\_rec}}
	\reponse{5}{3}
    \ifcorrige
    \inputpartC{rebours.c}{}{\small}{12}{23}
    \fi
\end{Exercise}

\begin{Exercise}[title = {un peu de OCaml}]\\
	On donne la définition de la fonction {\tt mystere} en OCaml :
	\inputpartOCaml{ex2.ml}{}{}{1}{2}
	\Question{Quel type est automatiquement inféré pour {\tt n} ? Pourquoi ?}
	\reponse{1}{1}
    \tcor{Le type de {\tt n} est {\tt int} car il est utilisé avec l'opérateur de comparaison {\tt <} avec un entier puis avec l'addition {\tt +}.}
	\Question{Donner les résultat des appels suivants : {\tt mystere 7}, {\tt mystere 42}, {\tt mystere 666}, {\tt mystere 2023} en complétant le tableau ci-dessous} \points{2}\\
	\begin{tabularx}{\linewidth}{|X|}
		\hline
		\renewcommand{\arraystretch}{2}
		\begin{tabular}{c|p{1cm}|p{1cm}|p{1cm}|p{1cm}}
			\hline
			{\tt n} & 7 & 42 & 666 & 2025 \\
			\hline
			{\tt mystere n} & \comp{1} & \comp{2} & \comp{3} & \comp{4}  \\
			\hline
		\end{tabular} \\
		\hline
	\end{tabularx}
	\Question{Proposer une spécification et un nom plus adapté pour cette fonction.}
	\reponse{1}{1}
    \tcor{Cette fonction calcule le nombre de chiffres d'un entier positif, on pourrait l'appeler {\tt nb\_chiffres} ou {\tt nombre\_de\_chiffres}.}
\end{Exercise}

\end{document}

\end{document}

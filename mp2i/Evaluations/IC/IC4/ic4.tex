\documentclass[11pt,a4paper]{article}

\usepackage{Act}

\begin{document}
\input{\detokenize{/home/fenarius/Travail/Cours/cpge-info/latex/Macros.tex}}
\ModeExercice
\IC{4}{Relations d'ordre}
\newcommand{\SPATH}{\FPATH Evaluations/IC/IC4/}

\setcounter{Exercise}{0}
\begin{Exercise}[title = {Caractérisation des ordres bien fondés}]
\Question{Rappeler la définition d'un ordre bien bien fondé.}\\
\begin{tabularx}{\linewidth}{|X|}
    \hline
    \dotfill \\ 
    \dotfill \\ 
    \dotfill \\ 
    \hline
\end{tabularx}
\Question{Soit $(E,\preccurlyeq)$ un ensemble ordonné, prouver qu'un $\preccurlyeq$ est bien fondé si et seulement si toute partie non vide de $(E,\preccurlyeq)$ admet un élément minimal\\}
\begin{tabularx}{\linewidth}{|X|}
    \hline
    \dotfill \\ 
    \dotfill \\ 
    \dotfill \\
    \dotfill \\ 
    \dotfill \\ 
    \dotfill \\
    \dotfill \\ 
    \dotfill \\ 
    \dotfill \\
    \dotfill \\ 
    \dotfill \\ 
    \dotfill \\
    \dotfill \\ 
    \dotfill \\ 
    \dotfill \\
    \dotfill \\ 
    \dotfill \\
    \hline
\end{tabularx}
\end{Exercise}
\begin{Exercise}[title = {Terminaison d'une fonction}]
\Question{Ecrire en OCaml la fonction {\tt fusion int list -> int list -> int list} qui prend en argument deux listes d'entiers triées et renvoie  leur fusion triée. Par exemple {\tt fusion [1; 4; 7; 9; 10] [2; 3; 8; 15]} renvoie {\tt [1; 2; 3; 4; 7; 8; 9; 10; 15]}}\\
\begin{tabularx}{\linewidth}{|X|}
\hline
\dotfill \\
\dotfill \\ 
\dotfill \\ 
\dotfill \\
\dotfill \\ 
\dotfill \\ 
\dotfill \\
\dotfill \\ 
\hline
\end{tabularx}
\Question{En utilisant un variant sur $(\N^2, \preccurlyeq_p)$ où $\preccurlyeq_P$ désigne l'ordre produit sur $\N^2$, prouver la terminaison de cette fonction}\\
\begin{tabularx}{\linewidth}{|X|}
    \hline
    \dotfill \\
    \dotfill \\ 
    \dotfill \\ 
    \dotfill \\
    \dotfill \\ 
    \dotfill \\ 
    \dotfill \\
    \dotfill \\
    \dotfill \\ 
    \dotfill \\ 
    \dotfill \\
    \hline
    \end{tabularx}
\end{Exercise}
\end{document}

\documentclass[11pt,a4paper]{article}

\usepackage{Act}

\begin{document}
\input{\detokenize{/home/fenarius/Travail/Cours/cpge-info/latex/Macros.tex}}
\ModeExercice
\IC{1}{Représentation des entiers}

\setcounter{Exercise}{0}
\begin{Exercise}[title={Conversions}]\\
	Commpléter le tableau de conversion suivant :
	\renewcommand{\arraystretch}{2}
	\begin{center}
		\begin{tabularx}{0.8\textwidth}{|Y|Y|Y|}
			\hline
			Décimal                 & Binaire                            & Hexadécimal            \\
			\hline
			$\base{205}{10}$        & \cor{$\base{1100\,1101}{2}$}       & \cor{$\base{CD}{16}$}  \\
			\hline
			\cor{$\base{327}{10}$}  & $\base{1\,0100\,0111}{2}$          & \cor{$\base{147}{16}$} \\
			\hline
			\cor{$\base{1068}{10}$} & \cor{$\base{100\,0010\,1100}{2}$}  & $\base{42C}{16}$       \\
			\hline
			\cor{$\base{912}{10}$}  & $\base{11\,1001\,0000}{2}$         & \cor{$\base{390}{16}$} \\
			\hline
			$\base{2654}{10}$       & \cor{$\base{1010\,0101\,1110}{2}$} & \cor{$\base{A5E}{16}$} \\
			\hline
		\end{tabularx}
	\end{center}
\end{Exercise}

\begin{Exercise}[title={Complément à deux}]\\
	Dans cet exercice, on suppose que les nombres entiers sont représentés en complément à deux sur 10 bits.
	\Question{Quelle est l'ensemble des nombres représentables ?  \\
		\renewcommand{\arraystretch}{2}
		\begin{tabularx}{\linewidth}{|X|}
			\hline
			\cor{L'ensemble des nombres représentables est $\intN{-2^{9}}{2^9-1}$, c'est à dire $\intN{-512}{511}$} \\
			\hline
		\end{tabularx}
	}
	\Question{Donner la représentation de $\base{-421}{10}$\\
		\renewcommand{\arraystretch}{2}
		\begin{tabularx}{\linewidth}{|X|}
			\hline
			\cor{On calcule celle de $421$ \textit{sur 10 bits}, on inverse tous les bits, on ajoute 1, on obtient : $\base{10\,0101\,1011}{2}$} \\
			\hline
		\end{tabularx}
	}
	\Question{Donner la représentation de $\base{-59}{10}$\\
		\renewcommand{\arraystretch}{2}
		\begin{tabularx}{\linewidth}{|X|}
			\hline
			\cor{De la même façon, on obtient : $\base{11\,1100\,0101}{2}$} \\
			\hline
		\end{tabularx}
	}
\end{Exercise}

\begin{Exercise}[title={Un programme en C}]\\
	Le programme C ci dessous compile correctement (et ne produit aucun \textit{warning} avec l'option {\tt -Wall}) quel sera le résultat de son exécution ? Commenter et justifier.
	\inputpartC{\spath{IC}{1}/ic1.c}{}{\small}{1}{10}
	\renewcommand{\arraystretch}{2}
	\begin{tabularx}{\linewidth}{|X|}
		\hline
		\cor{La variable {\tt i} déclarée dans la boucle est de type {\tt uint8\_t}, par conséquent elle prend ses valeurs dans $\intN{0}{255}$ et un dépassement de capacité n'est pas un comportement indéfini, on effectue simplement les calculs modulo 256.
		\newline
		Lorsque la variable {\tt i} atteint zéro, l'opération {\tt i = i - 1} donne {\tt i=255} (puisque $-1 \mod 256 = 255$).
		La condition de sortie de boucle est {\tt i<0} et elle n'est donc jamais réalisée, par conséquent, ce programme affiche les entiers de 10 à 0 puis boucle indéfiniment en affichant les entiers de 255 à 0.
		} \\
		\hline
	\end{tabularx}

\end{Exercise}


\end{document}

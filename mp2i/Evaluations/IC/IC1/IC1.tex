\documentclass[11pt,a4paper]{article}

\usepackage{Act}

\begin{document}
\input{\detokenize{/home/fenarius/Travail/Cours/cpge-info/latex/Macros.tex}}
\ModeExercice
\IC{1}{Représentation des entiers}

\setcounter{Exercise}{0}
\begin{Exercise}[title={Conversions}]\\
	Commpléter le tableau de conversion suivant :
	\renewcommand{\arraystretch}{2}
	\begin{center}
	\begin{tabularx}{0.8\textwidth}{|Y|Y|Y|}
		\hline
		Décimal          & Binaire               & Hexadécimal      \\
		\hline
		$\base{205}{10}$ & \dotfill          & \dotfill           \\
		\hline
		\dotfill            & $\base{1\,0100\,0111}{2}$ & \dotfill            \\
		\hline
		\dotfill            & \dotfill                 & $\base{42C}{16}$ \\
		\hline    
        \dotfill            & $\base{11\,1001\,0000}{2}$ & \dotfill            \\
        \hline
        $\base{2654}{10}$ & \dotfill                 & \dotfill            \\
        \hline 
	\end{tabularx}
\end{center}
\end{Exercise}

\begin{Exercise}[title={Complément à deux}]\\
	Dans cet exercice, on suppose que les nombres entiers sont représentés en complément à deux sur 10 bits.
	\Question{Quelle est l'ensemble des nombres représentables ?  \\
	\renewcommand{\arraystretch}{2}
	\begin{tabularx}{\linewidth}{|X|}
		\hline
		\dotfill \\ 
		\hline
	\end{tabularx}
	}
	\Question{Donner la représentation de $\base{-421}{10}$\\
	\renewcommand{\arraystretch}{2}
	\begin{tabularx}{\linewidth}{|X|}
		\hline
		\dotfill \\ 
		\hline
	\end{tabularx}
	}
	\Question{Donner la représentation de $\base{-59}{10}$\\
	\renewcommand{\arraystretch}{2}
	\begin{tabularx}{\linewidth}{|X|}
		\hline
		\dotfill \\ 
		\hline
	\end{tabularx}
	}
\end{Exercise}

\begin{Exercise}[title={Un programme en C}]\\
Le programme C ci dessous compile correctement (et ne produit aucun \textit{warning} avec l'option {\tt -Wall}) quel sera le résultat de son exécution ? Commenter et justifier.
\inputpartC{\spath{IC}{1}/ic1.c}{}{\small}{1}{10}
\renewcommand{\arraystretch}{2}
	\begin{tabularx}{\linewidth}{|X|}
		\hline
		\dotfill \\ 
		\dotfill \\ 
		\dotfill \\ 
		\dotfill \\ 
		\dotfill \\ 
		\hline
	\end{tabularx}
	
\end{Exercise}


\end{document}

\documentclass[11pt,a4paper]{article}

\usepackage{Act}

\begin{document}
\input{\detokenize{/home/fenarius/Travail/Cours/cpge-info/latex/Macros.tex}}
\ModeExercice
\IC{3}{Terminaison}
\newcommand{\SPATH}{\FPATH Evaluations/IC/IC3/}

\setcounter{Exercise}{0}
\begin{Exercise}[title={}]\\
	On considère la fonction \mintinline{c}{mystere} suivante en C :
    \inputpartC{premier_diviseur.c}{}{}{6}{11}

\Question{Donner la valeur renvoyée par \mintinline{c}{mystere(42)}}\\
\renewcommand{\arraystretch}{1.6}
\begin{tabularx}{\linewidth}{|X|}
    \hline
    \dotfill \\ 
    \hline
\end{tabularx}
\Question{Proposer une spécification aussi précise que possible pour cette fonction }\\
\begin{tabularx}{\linewidth}{|X|}
    \hline
    \dotfill \\ 
    \dotfill \\ 
    \hline
\end{tabularx}
\Question{Prouver la terminaison de cette fonction.}\\
\begin{tabularx}{\linewidth}{|X|}
    \hline
    \dotfill \\ 
    \dotfill \\ 
    \dotfill \\ 
    \dotfill \\ 
    \hline
\end{tabularx}
\Question{Ecrire une version récursive de cette fonction en OCaml.}\\
\begin{tabularx}{\linewidth}{|X|}
    \hline
    \dotfill \\ 
    \dotfill \\ 
    \dotfill \\ 
    \dotfill \\ 
    \hline
\end{tabularx}
\end{Exercise}

\begin{Exercise}[title = {Terminaison d'une fonction récursive}]\\
    Montrer la terminaison de la fonction ci-dessous pour $n \in \N^*$ :
    \inputpartOCaml{mystere.ml}{}{}{1}{4}
    \renewcommand{\arraystretch}{1.6}
    \begin{tabularx}{\linewidth}{|X|}
        \hline
        \dotfill \\ 
        \dotfill \\ 
        \dotfill \\ 
        \dotfill \\ 
        \dotfill \\ 
        \dotfill \\ 
        \hline
    \end{tabularx}
\end{Exercise}
\end{document}

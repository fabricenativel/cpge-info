\PassOptionsToPackage{dvipsnames,table}{xcolor}
\documentclass[11pt,a4paper]{article}

\usepackage{Act}
\begin{document}
\input{\detokenize{/home/fenarius/Travail/Cours/cpge-info/latex/Macros.tex}}
\ModeExercice
\setboolean{corrige}{false}
\IC{3}{Pointeurs}
\setcounter{Exercise}{0}

\begin{Exercise}[title = pointeurs] \\
    Le programme suivant produit un \textit{warning} à la compilation \og{}\textit{‘p’ is used uninitialized}\fg{} et une erreur de segmentation à l'exécution.
    \inputpartC{p1.c}{}{}{1}{9}
    
    \Question{Proposer une correction afin d'initialiser le pointeur {\tt p} vers une adresse se trouvant dans le tas avant de l'utiliser, quel {\tt include} est alors nécessaire ?}
    \reponse{1}{2}
    \tcor{On remplace la ligne 6 par \mintinline{c}{int *p = malloc(sizeof(int));} et on ajoute \mintinline{c}{#include <stdlib.h>} en début de fichier.}
    \Question{On suppose le pointeur {\tt p} correctement initialisé, quel affichage produit le programme ?}
    \reponse{0}{1}
    \tcor{Le programme affiche : \mintinline{c}{Valeur pointé par p :  42}.}
    \Question{Quel est le rôle de l'opérateur {\tt \&} en C ? Que contient {\tt \&n} ?}
    \reponse{1}{2}
    \tcor{Cette variable contient l'adresse mémoire de la variable {\tt n}.}
    \Question{Les variables {\tt \&n} et {\tt p} sont-elles égales ?}
    \reponse{2}{2}
    \tcor{Non, elles ne sont pas égales car {\tt p} est un pointeur vers une adresse mémoire dans le tas alors que {\tt n} est une variable locale dans la pile. Par contre le contenu de ces deux adresses est le même (l'entier 42).}
\end{Exercise}

\begin{Exercise}[title = pointeurs] \\
    Le programme suivant ne produit aucun \textit{warning} à la compilation et s'exécute sans erreur.
    \inputpartC{p2.c}{}{}{1}{9}
    \Question{Quel est l'affichage produit par le programme ?}
    \reponse{0}{1}
    \tcor{Le programme affiche : \mintinline{c}{Valeur de n = 2025}.}
    \Question{Les variables {\tt \&n} et {\tt p} sont-elles égales ?}
    \reponse{1}{2}
    \tcor{Oui, elles sont égales car {\tt p} est un pointeur vers l'adresse mémoire de la variable {\tt n}.}
\end{Exercise}



\end{document}
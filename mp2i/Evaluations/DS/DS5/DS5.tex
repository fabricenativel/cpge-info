\PassOptionsToPackage{dvipsnames,table}{xcolor}
\documentclass[11pt,a4paper]{article}

\usepackage{DS}

\begin{document}

\input{\detokenize{/home/fenarius/Travail/Cours/cpge-info/latex/Macros.tex}}
\ModeConcours
\DS{MP2I}{5}{Avril 2025}

\setboolean{corrige}{false}
\newcommand{\SPATH}{/home/fenarius/Travail/Cours/cpge-info/docs/mp2i/files/C12}

\newcommand{\maillon}[3]{
	\begin{tabular}{|p{0.2cm}|p{0.2cm}|}
		\hline
		\rnode{#2}{#1} & \rnode{#3}{\phantom{$e_0$}} \\
		\hline
	\end{tabular}
}

\alertbox{\danger}{Consignes}{
	\begin{itemize}
		\item[\textbullet] Les programmes demandés doivent être écrits en C ou en OCaml. Dans le cas du C, on suppose que les librairies standards usuelles ({\tt <stdio.h>}, {\tt <stdlib.h>}, {\tt <stdbool.h>}, {\tt <stdassert.h>}, \dots) sont déjà importées.
		\item[\textbullet] On pourra toujours librement utiliser une fonction demandée à une question précédente même si cette question n'a pas été traitée.
		\item[\textbullet] Veillez à présenter vos idées et vos réponses partielles même si vous ne trouvez pas la solution complète à une question.
		\item[\textbullet] La clarté et la lisibilité de la rédaction et des programmes sont des éléments de notation.
	\end{itemize}
}

\psset{arrows=->,treesep=0.8cm,levelsep=0.8cm, radius=0.3cm}
\begin{Exercise}[title={Questions de cours : terminaison}]\\
	On considère la fonction d'Ackerman $a : \N \times \N \mapsto \N$ définie par : \\
	$\left\{
		\begin{array}{lll}
			a(0, m) & = & m + 1                                             \\
			a(n, 0) & = & a(n-1,1) \text{ si } n>0                          \\
			a(n,m)  & = & a(n-1, a(n, m-1)) \text{ si } n>0 \text{ et } m>0 \\
		\end{array}
		\right.$
	\Question{Calculer $a(1,2)$}
  \tcor{$\begin{array}{lcl}
    a(1,2) & = & a(0, a(1, 1)) \\
           & = & a(0, a(0, a(1, 0))) \\
           & = & a(0, a(0, a(0, 1))) \\
           & = & a(0, a(0, 2)) \\
           & = & a(0, 3) \\
           & = & 4
  \end{array}$}
	\Question{Ecrire en OCaml une fonction {\tt ack: int -> int -> int} qui prend en argument deux entiers positifs $n$ et $m$ et renvoie $a(n,m)$.}
  \ifcorrige
  \corpartOCaml{ack.ml}{}{}{1}{5}
  \fi
	\Question{Prouver la terminaison de la fonction {\tt ack} on précisera soigneusement le variant et la relation d'ordre bien fondée utilisée.}
  \tcor{On considère l'ordre lexicographique sur $N^2$ noté $\preccurlyeq_L$, montrons que $(n,m)$ est un variant .
  \begin{itemize}
    \item si $m=0$ alors on effectue un appel récursif avec $(n-1,1)$ et comme $(n-1,1) \preccurlyeq_L (n,0)$ la quantité $(n,m)$ décroit strictement.
    \item sinon, on effectue un premier appel récursif avec $(n, m-1)$ et comme $(n, m-1) \preccurlyeq_L (n,m)$ la quantité $(n,m)$ décroit strictement. Le second appel récursif s'effectue avec $(n-1, a(n,m-1))$ qui est strictement inférieur à $(n,m)$. 
  \end{itemize}
  Dans tous les cas $(n,m)$ décroit strictement à chaque appel récursif, c'est donc un variant et puisque $(N^2,\preccurlyeq)$ est un ensemble bien fondé, la fonction {\tt ack} termine.}
\end{Exercise}

\begin{Exercise}[title={Questions de cours : preuve par induction structurelle}]
	\Question{Donner la définition inductive des arbres binaires sur un ensemble d'étiquettes $E$.}
  \tcor{
    On définit inductivement l'ensemble des arbres binaires sur un ensemble d'étiquettes $E$ avec :
    \begin{itemize}
    \item L'ensemble d'axiomes $X_0 = \{\varnothing\}$ où $\varnothing$ est l'arbre vide.
    \item La règle d'inférence d'arité 2 : $ r : (g,d) \rightarrow (g,e,d)$ où $e \in E$.
    \end{itemize}
  }
	\Question{Ecrire en OCaml un type {\tt arbrebin} représentant un arbre binaire en utilisant un type paramétré {\tt 'a} pour l'ensemble des étiquettes.}
  \ifcorrige
  \corpartOCaml{ack.ml}{}{}{7}{9}
  \fi
	\Question{Rappeler la définition de la hauteur et de la taille d'un arbre binaire.}
  \tcor{
    \begin{itemize}
			\item[\textbullet] Le nombre de noeuds d'un arbre binaire $A$, noté $n(A)$, se définit récursivement par :\\
				$ \left\{
					\begin{array}{llll}
						n(A) & = & 0               & \text{si $A$ est vide}  \\
						n(A) & = & 1 + n(g) + n(d) & \text{si $A = r(g,d)$} \\
					\end{array}
					\right.
				$
			\item[\textbullet] La hauteur d'un arbre binaire $A$, noté $h(A)$, se définit récursivement par : \\
				$\left\{
					\begin{array}{llll}
						h(A) & = & -1                  & \text{si $A$ est vide}  \\
						h(A) & = & 1 + \max(h(g),h(d)) & \text{si $A = r(g,d)$} \\
					\end{array}
					\right.
				$
		\end{itemize}
  }
  \Question{Prouver \textit{par induction structurelle} que la taille d'un arbre binaire de hauteur $h$ est inférieure ou égale à $2^{h+1}-1$.}
  \tcor{Soit $A$ un arbre binaire de hauteur $n$ et de taille $h$ on note $\mathcal{P}(A)$ la propriété $n \leqslant 2^{h+1}-1$. Montrons par induction structurelle que $P$ est vraie pour tous les arbres binaires.
  \begin{itemize}
    \item $\mathcal{P}$ est vraie pour tout axiome, en effet il y a un seul axiome, l'arbre vide qui par définition est de taille 0 et de hauteur $-1$ et on a bien $2^{-1 + 1} -1 \geqslant 0 $.
    \item Supposons que $\mathcal{P}$ est vraie pour deux arbres $g$ et $d$ et montrons qu'alors $\mathcal{P}(r(g,d))$ est vraie, c'est à dire la conservation de $\mathcal{P}$ par application de l'unique règle d'inférence. \\
    $\begin{array}{lcll}
      n & = & n(g) + n(d) + 1 & \text{par définition de la taille de }r(g,d)   \\
      n & \leqslant & 2^{h(g) + 1} -1 + 2^{h(g) + 1} -1 + 1  & \text{car $P(g)$ et $P(d)$ sont vraies} \\
      n & \leqslant & 2^{\max(h(g),h(d)) + 1} -1 + 2^{\max(h(g),h(d)) + 1} -1 + 1  &  \\
      n & \leqslant & 2^{\max(h(g),h(d)) + 2} - 1 &  \\
      n & \leqslant & 2^{h+1} -1
    \end{array}$
    Donc par induction structurelle,  $\mathcal{P}$ est vraie pour tout arbre binaire.
  \end{itemize}

  }
\end{Exercise}


\begin{Exercise}[title = {Recherche des $k$ premiers maximums d'une liste}] \\
	Les fonctions demandées dans cet exercice doivent être écrites en langage OCaml.\smallskip \\
	On s'intéresse au problème de la recherche des $k$ premiers maximums d'une liste de $n$ entiers. Dans toute la suite de l'exercice on supposera que la liste est \textit{non vide} : $n>0$ et qu'on extrait moins de maximums qu'il n'y a d'éléments dans la liste c'est à dire que $k \leqslant n$. On cherche donc à écrire une fonction {\tt kmax : int list -> int -> int list}  qui renvoie la liste --\textit{triée dans l'ordre décroissant}-- des {\tt k} premiers maximums de la liste donnée en argument . Par exemples :
	\begin{itemize}
		\item {\tt kmax [2; 5; 1; 8; 3; 0; 4] 2} renvoie {\tt [8, 5]}
		\item {\tt kmax [7; 8; 8; 1; 6; 3; 2; 9] 3} renvoie {\tt [9; 8; 8]}
		\item {\tt kmax [1; 0; 1; 2; 4]} 4 renvoie {\tt [4; 2; 1; 1]}
	\end{itemize}
	Les trois parties sont indépendantes et dans chacune d'elle on propose un algorithme différent afin d'écrire la fonction {\tt kmax}. \medskip

	\ExePart[name = {Résolution par recherche successive des maximums}]
	\Question{Ecrire une fonction {\tt max\_reste : int list -> int * int list} qui prend en argument une liste  et renvoie le couple composé du maximum de cette liste et de cette liste privée d'un de ses maximums. Par exemple {\tt max\_reste [2; 6; 4; 6; 5]} renvoie {\tt (6, [2; 4; 6; 5])}. On pourra procéder par correspondance de motif et traiter le cas de la liste vide par un {\tt failwith}}.
  \ifcorrige
  \corpartOCaml{kmax.ml}{}{}{3}{7}
  \fi
	\Question{Donner en la justifiant brièvement la complexité de la fonction {\tt max\_reste}.}
  \tcor{Le fonction {\tt max\_reste} a une complexité linéaire en fonction de la taille $n$ de la liste car elle effectue $n-1$ appels récursifs et chacun de ces appels n'effectue que des opérations élémentaires.}
	\Question{Ecrire une première version de la fonction {\tt kmax} qui procède par extraction successive des {\tt k} premiers maximums en utilisant la fonction {\tt max\_reste}. On procèdera de façon récursive sans utiliser les aspects impératifs de OCaml.}
  \ifcorrige
  \corpartOCaml{kmax.ml}{}{}{9}{14}
  \fi
	\Question{Quelle est la complexité de cette version de la fonction {\tt kmax} en fonction du nombre $k$ de maximums à extraire et de la longueur $n$ de la liste ?}\medskip 
  \tcor{La fonction {\tt kmax} effectue un appel récursif à {\tt max\_reste} pour chacun des $k$ maximums à extraire et comme {\tt max\_reste} a une complexité en $\mathcal{O}(n)$, cette version de {\tt kmax} a une complexité en $\mathcal{O}(kn)$.}
	\ExePart[name = {Résolution par un tri}]
	\Question{Ecrire une fonction {\tt kpremiers int list -> int -> int list} qui prend en argument une liste {\tt lst} et un entier {\tt k} et renvoie la liste composée des {\tt k} premiers éléments de {\tt lst}. Par exemple {\tt kpremiers [2; 7; 1; 8; 5] 3} renvoie la liste {\tt [2; 7; 1]}. On procédera de façon récursive sans utiliser les apsects impératifs de OCaml.}
  \ifcorrige
  \corpartOCaml{kmax.ml}{}{}{16}{20}
  \fi
	\Question{On propose d'écrire la fonction {\tt kmax} en triant la liste  par ordre décroissant puis en prenant ses $k$ premiers éléments. En supposant que l'algorithme de tri utilisé a une complexité en $\mathcal{O}(n\log n)$, donner la complexité de ce nouvel algorithme en fonction du nombre $k$ de maximums à extraire et de la longueur $n$ de la liste (on ne demande \textit{pas} de le programmer).}\medskip
  \tcor{La fonction {\tt kmax} effectue un appel à la fonction de tri qui a une complexité en $\mathcal{O}(n\log n)$ et ensuite elle effectue un appel à la fonction {\tt kpremiers} qui a une complexité en $\mathcal{O}(k)$. Donc la complexité de cette version de {\tt kmax} est en $\mathcal{O}(n\log n + k)$ et comme $k <n$ la complexité est en $\mathcal{O}(n\log n)$.}
	\ExePart[name = {Résolution en utilisant un tas}]\\
	Dans la suite on suppose que la structure de données de tas d'entiers (type {\tt int}), est \textit{déjà implémentée} par un type {\tt tas} sur lequel on dispose des fonctions suivantes :
	\begin{itemize}
		\item {\tt cree\_tas : int -> tas} qui prend en argument un entier {\tt cap} et renvoie un tas binaire vide de capacité maximale {\tt cap}.
		\item {\tt donne\_taille : tas -> int} qui prend en argument un tas et renvoie sa taille (le nombre d'éléments actuellement stocké dans le tas).
		\item {\tt insere : int -> tas -> unit} qui insère une nouvelle valeur dans le tas. Cette fonction échoue lorsque le tas est plein.
		\item {\tt donne\_min : tas -> int} qui renvoie la valeur minimale contenu dans le tas  \textit{sans modifier le tas}. Cette fonction échoue lorsque le tas est vide.
		\item {\tt extrait\_min : tas -> int} qui renvoie, \textit{en le supprimant du tas} le minimum du tas. Cette fonction échoue lorsque le tas est vide.
	\end{itemize}
	\Question{Rappeler en les justifiant, les complexités des opérations {\tt insere} et {\tt extrait\_min} en fonction de la taille du tas notée {\tt k} si on suppose que l'implémentation de la structure de tas est réalisée grâce à un tableau.}\medskip
	\tcor{Un tas est un arbre binaire complet et à chaque étape d'une insertion ou d'une extraction, on remonte (ou on descend) d'un niveau dans cet arbre, la complexité de ces opérations est donc en $O(h)$ où $h$ est la  hauteur de l'arbre or l'arbre étant complet $\mathcal{O}(h)=\mathcal{O}(\log k)$ où $k$ est la taille de l'arbre. En conclusion, {\tt insere} et {\tt extrait\_min} ont une complexité en $\mathcal{O}(\log k)$}

	Afin d'extraire les {\tt k} premiers éléments d'une liste de taille {\tt n}, on propose créer un tas de taille {\tt k} puis  de parcourir récursivement la liste, pour chaque élément rencontré :
	\begin{itemize}
		\item si le tas n'est pas plein  on y insère l'élément.
		\item sinon,  on compare l'élément avec le minimum du tas, s'il est plus grand on extrait le minimum du tas et on insère l'élément dans le tas.
	\end{itemize}
	Par exemple, si on veut extraire les 3 premiers maximums de la liste {\tt [4; 6; 2; 8; 3; 7; 1; 9; 5]}, après l'insertion des trois premiers éléments, le tas est : \\
	\pstree[arrows=->,treesep=0.8cm,levelsep=1cm]{\Tcircle{2}}{\Tcircle{6} \Tcircle{4}} \\
	A l'étape suivante, 8 étant plus grand que 2 (le minimum du tas), on extrait 2 du tas et on y insère 8 ce qui donne : \\
	\pstree[arrows=->,treesep=0.8cm,levelsep=1cm]{\Tcircle{4}}{\Tcircle{6} \Tn{}} \quad \pstree[arrows=->,treesep=0.8cm,levelsep=1cm]{\Tcircle{4}}{\Tcircle{6} \Tcircle{8}}  \\
	\psset{arrows=->,treesep=0.8cm,levelsep=1cm}
	\Question{Poursuivre le déroulement de cet algorithme en faisant figurer comme ci-dessus les étapes de l'évolution du tas.}
	\tcor{
		\begin{itemize}
			\item On traite 3, comme $3<4$ pas de modification \pstree{\Tcircle{4}}{\Tcircle{6} \Tcircle{8}}
			\item On traite 7, comme $7>4$, on extrait 4 et on insère 7 \pstree{\Tcircle{6}}{\Tcircle{8} \Tcircle{7}}
			\item On traite 1, comme $1<6$, pas de modification \pstree{\Tcircle{6}}{\Tcircle{8} \Tcircle{7}}
			\item On traite 9, comme $9>6$, on extrait 6 et on insère 9 \pstree{\Tcircle{7}}{\Tcircle{8} \Tcircle{9}}
			\item On traite 5, comme $5<7$, pas de modification \pstree{\Tcircle{7}}{\Tcircle{8} \Tcircle{9}}
		\end{itemize}}
	\Question{Donner une implémentation de la fonction {\tt kmax} utilisant ce nouvel algorithme. On rappelle qu'on pourra utiliser les fonctions de manipulation de la structure de {\tt tas} données en début de partie. Comme précédemment, on procédera par récurrence sans utiliser les aspects impératifs de OCaml.}
  \ifcorrige
  \corpartOCaml{kmax.ml}{}{}{82}{95}
  \fi
	\Question{Donner en la justifiant la complexité de ce nouvel algorithme en fonction de $k$ et $n$.}
	\tcor{Les opérations d'insertion et d'extraction dans le tas sont toutes en $O(\log k)$ car le tas est de taille $k$. On effectue ces opérations au plus $n$  fois (une fois pour chaque élément du tableau) et donc la complexité de ce nouvel algorithme est $O(n \log k)$}
\end{Exercise}


\begin{Exercise}[title={Saut de valeur maximale}, origin={\bac {\sc capes nsi 2023}}]\\
	Les fonctions demandées dans cet exercice sont à écrire en langage C. \smallskip \\
	Dans un tableau de flottants (type {\tt double} du langage C) {\tt tab} de taille {\tt n}, on appelle \textit{saut} un couple $(i,j)$ avec $0 \leq i \leq j <$ {\tt n  } et la \textit{valeur} d'un saut est la valeur {\tt tab[j]-tab[i]}. Le but de l'exercice est de rechercher la valeur maximale d'un saut dans un tableau. Par exemple, dans le tableau {\tt \{ 2.0, 0.2, 3.0, 5.3, 2.0\}}, la valeur maximale d'un saut est {\tt 5.3 - 0.2 = 5.1}, cette valeur est obtenue en considérant le saut {\tt (1,3)}.\medskip

  \ExePart[name = {Questions préliminaires et résolution naïve}]
	\Question{Ecrire une fonction de signature \mintinline{c}{double valeur(double tab[], int i, int j, int n)
  } qui prend en argument un tableau (de {\tt double}) {\tt tab} de taile {\tt n} ainsi que deux indices {\tt i} et {\tt j} et renvoie la valeur du saut {\tt (i,j)}. On vérifiera les préconditions sur {\tt i} et {\tt j} à l'aide d'instructions {\tt assert}. Par exemple  si le tableau {\tt tab} est {\tt \{2.0; 0.2; 3.0; 5.3; 2.0\}} alors {\tt valeur(tab, 0, 2, 5)} renvoie {\tt 1.0} (car {\tt tab[2]-tab[0] = 1.0}).}
  \ifcorrige
  \corpartC{sautmax.c}{}{}{7}{11}
  \fi
	\Question{Donner un exemple de tableau avec exactement deux sauts de valeur maximale et préciser ces sauts.}
	\tcor{La liste {\tt [2, 6, 1, 5]} possède deux sauts de valeurs maximale : {\tt (0,1)} et {\tt (2,3)} (ces deux sauts ont une valeur de 4)}
	\Question{À l'aide d'un contre-exemple, montrer qu'on ne peut pas se contenter de chercher le minimum et le maximum d'un tableau pour trouver un saut de valeur maximale.}
	\tcor{Dans la liste {\tt [2, 6, 1, 5]} le minimum est à l'indice 2 (c'est 1) et le maximum à l'indice 1 (c'est 3) et comme le minimum est après le maximum ce n'est pas le saut maximal.}
	\Question{Écrire une fonction {\tt sautmax\_naif} qui renvoie un saut de valeur maximale dans un tableau de taille {\tt n} en testant tous les couples $(i,j)$ tels que $0 \leq i \leq j <$ {\tt n}.\label{naif}}
  \ifcorrige
  \corpartC{sautmax.c}{}{}{7}{11}
  \fi
	\Question{Quelle est la complexité de la fonction {\tt sautmax\_naif} en fonction de la taille $n$ du tableau ?}\medskip
  \tcor{Il y a deux boucles {\tt for} imbriquées et les deux sont exécutés au plus {\tt n fois}, les opérations à l'intérieur de ces boucles sont toutes des opérations élémentaires, donc la complexité est en $\mathcal{O}(n^2)$}
  \Question{Au lieu de rechercher la valeur maximale d'un saut noté $v_m$, on peut vouloir récupérer deux indices $(i_m,j_m)$ tels que le saut $(i_m,j_m)$ soit de valeur $v_m$. Pour cela, on propose d'écrire une fonction {\tt indices\_sautmax} qui prendra en plus du tableau et de sa taille  deux pointeurs vers des entiers {\tt im} et {\tt jm} qui seront modifiés par la fonction afin qu'ils contienne après appel les indices d'un saut de valeur maximale. La signature de la fonction est alors :\\ \mintinline{c}{void indices_sautmax(double tab[], int n, int *im, int *jm)}, c'est à dire qu'elle ne renvoie rien mais modifiera le contenu des pointeurs {\tt im} et {\tt jm}. Ecrire cette fonction.}
  \ifcorrige
  \corpartC{sautmax.c}{}{}{32}{49}
  \fi
  \ExePart[name = {Résolution avec une méthode diviser pour régner}]\\
	On propose maintenant d'utiliser une méthode diviser pour régner afin de calculer la valeur maximale d'un saut. On note $n$ la taille du tableau {\tt t} et $p = \lfloor \frac{n}{2} \rfloor$ (où $\lfloor \; \rfloor$ désigne la partie entière). On souhaite calculer :
	\begin{itemize}
		\item $(i_g,j_g)$ un saut de valeur maximale lorsque $j_g < p$ (c'est à dire un saut maximal dans la moitié gauche)
		\item $(i_d,j_d)$ un saut de valeur maximale lorsque $id \geqslant p$ (c'est à dire un saut maximal dans la moitié gauche)
		\item $(i_m,j_m)$ un saut de valeur maximal lorsque $i_m < p < j_m$ (c'est à dire un saut maximal dont le premier indice est dans la moitié gauche et le second dans la moitié droite)
	\end{itemize}

	\Question{Justifier qu'un saut de valeur maximale du tableau {\tt t} est nécessairement un des trois ci-dessus. On pourra faire un schéma pour illustrer le raisonnement.}
	\tcor{Un saut de valeur maximal $(i,j)$ du tableau {\tt t} est tel que $i \leqslant j$ trois situations sont donc possibles en fonction de la position relative de $i,j$ et $p$ :
		\begin{itemize}
			\item $i\leqslant j<p$ et donc le saut maximal $(i,j)$ se situe dans la moité gauche du tableau.\smallskip \\
      \psline{|-|}(0,0)(10,0)
      \uput[d](0,0){$0$}
      \uput[d](10,0){$n$}
      \uput[d](5,0.65){$p$}
      \psdot(5,0)
      % Draw and label i and j in the left half
      \uput[d](2,0.7){$i$}
      \psdot(2,0)
      \uput[d](4,0.7){$j$}
      \psdot(4,0)
			\item $p \leqslant i \leqslant j$ et donc le saut maximal $(i,j)$ se situe dans la moité droite du tableau.\smallskip \\
      \psline{|-|}(0,0)(10,0)
      \uput[d](0,0){$0$}
      \uput[d](10,0){$n$}
      \uput[d](5,0.65){$p$}
      \psdot(5,0)
      % Draw and label i and j in the left half
      \uput[d](7,0.7){$i$}
      \psdot(7,0)
      \uput[d](9.5,0.7){$j$}
      \psdot(9.5,0)
			\item $i < p < j$ c'est à dire que le saut \og{} traverse \fg{} le milieu du tableau.\smallskip \\
			\psline{|-|}(0,0)(10,0)
      \uput[d](0,0){$0$}
      \uput[d](10,0){$n$}
      \uput[d](5,0.65){$p$}
      \psdot(5,0)
      % Draw and label i and j in the left half
      \uput[d](3,0.7){$i$}
      \psdot(3,0)
      \uput[d](6.5,0.7){$j$}
      \psdot(6.5,0)
		\end{itemize}
		Ce qui correspond bien aux trois situations décrites ci-dessus.
	}
	\Question{Justifier que $i_m$ est nécessairement l'indice d'une valeur minimale dans la moitié gauche de {\tt t} (on admettra que de même $j_m$ est nécessairement l'indice d'une valeur maximale dans la moitié droite de {\tt t}).}
	\tcor{On raisonne par l'absurde, si tel n'était pas le cas, on aurait une valeur d'indice $q$ dans la moitié gauche strictement inférieur à $t[i_m]$ et donc le saut $(q,j_m)$ aurait une valeur supérieure au saut $(i_m,j_m)$ ce qui contredit que $(i_m,j_m)$ est un saut de valeur maximale.}
  \Question{Ecrire une fonction de signature \mintinline{c}{double min(double tab[], int a, int b)} qui prend en argument un tableau {\tt tab}, ainsi que deux entiers {\tt a} et {\tt b} (avec {\tt a<=b}) et renvoie l'indice d'un minimum de {\tt tab} entre les deux indices {\tt a} (inclus) et {\tt b} (exclu).\\On supposera dans la suite \textit{déjà écrite} une fonction qui {\tt max} qui prend les mêmes arguments et renvoie l'indice d'un maximum du sous tableau {\tt \{tab[a],...,tab[b-1]\}} }
  \ifcorrige
  \corpartC{sautmax.c}{}{}{64}{76}
  \fi
	\Question{Ecrire une fonction de signature \mintinline{c}{double sautmax_dpr(double tab[], int a, int b)} qui prend en argument un tableau {\tt tab},  ainsi que deux entiers {\tt a} et {\tt b} (avec {\tt a<=b}) et renvoie la valeur d'un saut maximale dans {\tt tab} entre les deux indices {\tt a} (inclus) et {\tt b} (exclu).\textit{Cette fonction doit être récursive et utiliser la méthode diviser pour régner}. On pourra supposer déjà écrite une fonction {\tt max3} qui renvoie le maximum des trois {\tt double} donnés en argument.
	}
  \ifcorrige
  \corpartC{sautmax.c}{}{}{92}{107}
  \fi
  \Question{Déterminer la complexité de la fonction {\tt sautmax\_dpr}.}
  \tcor{
    On obtient l'équation de compexité $C(2n) = 2 C(n) + \mathcal{O}(n)$ en effet on résout deux sous problèmes de taille deux fois plus petite et on doit ensuite calculer un minimum et un maximum d'une liste de taille $n$ et ces opérations sont en $\mathcal{O}(n)$. On suppose sans perdre de généralité que $n=2^k$ et on note $u_k = C(2^k)/2^k$, en divisant l'équation de complexité par $2^{k+1}$ on obtient alors : \\
    $u_{k+1} \leqslant u_k + \dfrac{M2^k}{2^{k+1}}$ donc,
    $u_{k+1} \leqslant u_k + M\times \dfrac{1}{2}$ et en sommant pour $i=0$ à $k$ on obtient : \\
    $u_{k} \leqslant u_0 + M' k$\\
    $C(n) \leqslant n \, \left( C(1) + M' \log n \right)$\\
    Et donc $C(n) \in \mathcal{O}(n \log n)$.
  }
  \ExePart[name = {Résolution par programmation dynamique}]\\
    On cherche maintenant à résoudre ce problème par programmation dynamique, et on adopte les notations suivantes :
    \begin{itemize}
    \item $t$ est un tableau de taille $n$ contenant des flottants 
    \item $t[i]$ est l'élément d'indice $i$ ($0\leqslant i < n$) de $t$,
    \item pour $0<k \leqslant n$, $t_k$ est le sous tableau $t[0], \dots, t[k-1]$
    \item $m_k$ est l'indice d'un minimum de $t_k$ pour $0<k<n$.
    \item $(i_k,j_k)$ est un saut de valeur maximale dans $t_k$ pour $0<k<n$.
    \end{itemize}
    \Question{Donner les valeurs de $i_1$, $j_1$ et $m_1$}
    \tcor{Comme le tableau ne contient qu'un seul élément (celui d'indice 0), $i_1 = 0, j_1=0$ et $m_1=0$.}
    \Question{Donner la relation de récurrence liant $m_{k+1}$, $m_k$ et $t[k+1]$.}
    \tcor{Si $t[k+1]<m_k$ alors $m_{k+1} = t[k+1]$ sinon $m_{k+1} = m_k$.}
    \Question{Justifier que la relation suivante est correcte :\\
    $(i_{k+1},j_{k+1}) = \left\{ \begin{array}{ll} (i_k,j_k) \text{ si } t[k]-t[m_k]<t[j_k]-t[i_k] \\ (m_k,k) \text{ sinon} \end{array}\right.$
    }
    \tcor{Le saut de valeur maximal ayant $k$ comme second indice est $(m_k, k)$ et vaut $t[k]-t[m_k]$. En effet le second indice étant fixé on l'obtient en prenant le miminum du sous tableau $t_k$. On doit donc comparer ce nouveau saut avec le le saut maximal du sous tableau $t_k$. Soit il est plus petit c'est à dire $t[k]-t[m_k] < t[j_k] - t[i_k]$ et donc $(i_{k+1},j_{k+1}) = (i_k,j_k)$ ou bien il est plus grand et donc $(i_{k+1},j_{k+1}) = (m_k,k)$.}
    \Question{Ecrire une fonction de signature \mintinline{c}{double sautmax_dyn(double tab[], int n)} qui prend en argument un tableau et sa taille et renvoie la valeur maximale d'un saut de ce tableau. On procèdera \textit{de façon ascendante} en utilisant les relations de récurrences de la question précédente et en calculant successivement les valeurs maximales de saut dans les sous tableaux $t_1$ puis $t_2$,  $\dots$ jusqu'à $t_n$.}
    \ifcorrige
    \corpartC{sautmax.c}{}{}{109}{130}
    \fi
    \Question{Déterminer la complexité de la fonction {\tt sautmax\_dyn}}
    \tcor{La fonction parcourt le tableau à l'aide d'une boucle {\tt for} qui ne contient que des opérations élémentaires, la complexité est donc linéaire en fonction de la taille du tableau.}
\end{Exercise}


\end{document}
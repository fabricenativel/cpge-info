
\documentclass[11pt,a4paper]{article}
\usepackage{Act}
\begin{document}
\input{\detokenize{/home/fenarius/Travail/Cours/cpge-info/latex/Macros.tex}}
\RDS{1}{20/09/2025}

\begin{tcolorbox}[enhanced,width=\textwidth,center upper,fontupper=\bfseries,drop shadow southwest,sharp corners]
{\sc \large Astruc} Alexandre
\end{tcolorbox}
\medskip
\begin{tabularx}{\textwidth}{p{5cm}X}
	\alertbox{\faAward}{Note}{
		\begin{itemize}[leftmargin=0pt]
			\item[\textbullet] Note : \textbf{\large 11.7}
			\item[\textbullet] Rang : \textbf{7}
			\item[\textbullet] Traité : 87 \%
		\end{itemize}
	} &
	\alertbox{\faChartLine}{Statistiques des notes}{
		\begin{pspicture}(0,-0.1)(16,1.45)
			\psset{xunit=1,fillstyle=solid}
		   \savedata{\data}[11.7 11.2 18.9 8.4 8.4 10.9 15.7 5.8 8.0 13.5 17.8 16.7 6.0 18.6 9.5 11.6]
		   \rput{-90}(0,0.9){\psBoxplot[barwidth=1.1cm,yunit=0.5,fillcolor=gray,linewidth=1pt]{\data}}
		   \psaxes[yAxis=false,dx=1cm,Dx=2,labelsep=1pt,linecolor=gray,xlabelFontSize=\scriptstyle](0,0)(10.1,4)
		   \psdot[dotsize=8pt,dotstyle=diamond,linecolor=black,fillstyle=solid,fillcolor=white,linewidth=1pt](5.85,0.85)
           \psdot[dotsize=6pt,dotstyle=x,linecolor=black,linewidth=3pt](6.021875,0.85)
		   \end{pspicture}
	} \\
    
\end{tabularx}\\
\begin{tabularx}{\textwidth}{X}
\alertbox{\faComment}{Commentaire}
{
	L’ensemble est correct. Dans les preuves de correction, utilise des « primes » pour indiquer les valeurs avant et après un passage dans une boucle. Par exemple, $n$ la valeur en entrant dans la boucle et $n’$ la valeur après. Il faut revoir les {\tt struct} et les pointeurs.
}
\end{tabularx}
\medskip
    \ding{113} \textbf{\sffamily{Résultats par thème}} \medskip \\
    \renewcommand{\arraystretch}{1.2}
    \begin{tabular}{|l|r|r|}
    \cline{2-3}
    \multicolumn{1}{l|}{} & \multicolumn{1}{|c|}{Points} & \multicolumn{1}{|c|}{Traitées} \\
    \hline
    {Discipline de programmation} & 80\% \;{\small (08/10)} & 100\% \;{\small (2/2)} \\ \hline {Programmation de base en C} & 44\% \;{\small (47/105)} & 72\% \;{\small (8/11)} \\ \hline {Preuve de correction} & 67\% \;{\small (37/55)} & 100\% \;{\small (4/4)} \\ \hline {Preuve de terminaison} & 62\% \;{\small (22/35)} & 100\% \;{\small (3/3)} \\ \hline {Comprendre un algorithme} & 100\% \;{\small (15/15)} & 100\% \;{\small (3/3)} \\ \hline \end{tabular} \\\\\medskip \\
    \ding{113} \textbf{\sffamily{Résultats par exercice}} \medskip \\
    \renewcommand{\arraystretch}{1.2}
    \begin{tabular}{|l|r|r|}
    \cline{2-3}
    \multicolumn{1}{l|}{} & \multicolumn{1}{|c|}{Points} & \multicolumn{1}{|c|}{Traitées} \\
    \hline
    Exercice {1} & 85\% \;{\small (34/40)} & 100\% \;{\small (4/4)} \\ \hline Exercice {2} & 50\% \;{\small (40/80)} & 87\% \;{\small (7/8)} \\ \hline Exercice {3} & 60\% \;{\small (21/35)} & 100\% \;{\small (5/5)} \\ \hline Exercice {4} & 52\% \;{\small (34/65)} & 66\% \;{\small (4/6)} \\ \hline \end{tabular} \\\\\pagebreak
\begin{tcolorbox}[enhanced,width=\textwidth,center upper,fontupper=\bfseries,drop shadow southwest,sharp corners]
{\sc \large Berfeuil} Rohan
\end{tcolorbox}
\medskip
\begin{tabularx}{\textwidth}{p{5cm}X}
	\alertbox{\faAward}{Note}{
		\begin{itemize}[leftmargin=0pt]
			\item[\textbullet] Note : \textbf{\large 11.2}
			\item[\textbullet] Rang : \textbf{9}
			\item[\textbullet] Traité : 78 \%
		\end{itemize}
	} &
	\alertbox{\faChartLine}{Statistiques des notes}{
		\begin{pspicture}(0,-0.1)(16,1.45)
			\psset{xunit=1,fillstyle=solid}
		   \savedata{\data}[11.7 11.2 18.9 8.4 8.4 10.9 15.7 5.8 8.0 13.5 17.8 16.7 6.0 18.6 9.5 11.6]
		   \rput{-90}(0,0.9){\psBoxplot[barwidth=1.1cm,yunit=0.5,fillcolor=gray,linewidth=1pt]{\data}}
		   \psaxes[yAxis=false,dx=1cm,Dx=2,labelsep=1pt,linecolor=gray,xlabelFontSize=\scriptstyle](0,0)(10.1,4)
		   \psdot[dotsize=8pt,dotstyle=diamond,linecolor=black,fillstyle=solid,fillcolor=white,linewidth=1pt](5.6,0.85)
           \psdot[dotsize=6pt,dotstyle=x,linecolor=black,linewidth=3pt](6.021875,0.85)
		   \end{pspicture}
	} \\
    
\end{tabularx}\\
\begin{tabularx}{\textwidth}{X}
\alertbox{\faComment}{Commentaire}
{
	L’ensemble est correct mais la notion d’invariant semble parfois mal comprise. C’est une propriété (par exemple une égalité) qui dépend des variables du programme. Il faut aussi revoir la façon dont sont représentées les chaines de caractères en C.
}
\end{tabularx}
\medskip
    \ding{113} \textbf{\sffamily{Résultats par thème}} \medskip \\
    \renewcommand{\arraystretch}{1.2}
    \begin{tabular}{|l|r|r|}
    \cline{2-3}
    \multicolumn{1}{l|}{} & \multicolumn{1}{|c|}{Points} & \multicolumn{1}{|c|}{Traitées} \\
    \hline
    {Discipline de programmation} & 100\% \;{\small (10/10)} & 100\% \;{\small (2/2)} \\ \hline {Programmation de base en C} & 30\% \;{\small (32/105)} & 63\% \;{\small (7/11)} \\ \hline {Preuve de correction} & 61\% \;{\small (34/55)} & 75\% \;{\small (3/4)} \\ \hline {Preuve de terminaison} & 91\% \;{\small (32/35)} & 100\% \;{\small (3/3)} \\ \hline {Comprendre un algorithme} & 100\% \;{\small (15/15)} & 100\% \;{\small (3/3)} \\ \hline \end{tabular} \\\\\medskip \\
    \ding{113} \textbf{\sffamily{Résultats par exercice}} \medskip \\
    \renewcommand{\arraystretch}{1.2}
    \begin{tabular}{|l|r|r|}
    \cline{2-3}
    \multicolumn{1}{l|}{} & \multicolumn{1}{|c|}{Points} & \multicolumn{1}{|c|}{Traitées} \\
    \hline
    Exercice {1} & 77\% \;{\small (31/40)} & 100\% \;{\small (4/4)} \\ \hline Exercice {2} & 42\% \;{\small (34/80)} & 75\% \;{\small (6/8)} \\ \hline Exercice {3} & 94\% \;{\small (33/35)} & 100\% \;{\small (5/5)} \\ \hline Exercice {4} & 38\% \;{\small (25/65)} & 50\% \;{\small (3/6)} \\ \hline \end{tabular} \\\\\pagebreak
\begin{tcolorbox}[enhanced,width=\textwidth,center upper,fontupper=\bfseries,drop shadow southwest,sharp corners]
{\sc \large Body} Timothée
\end{tcolorbox}
\medskip
\begin{tabularx}{\textwidth}{p{5cm}X}
	\alertbox{\faAward}{Note}{
		\begin{itemize}[leftmargin=0pt]
			\item[\textbullet] Note : \textbf{\large 18.9}
			\item[\textbullet] Rang : \textbf{1}
			\item[\textbullet] Traité : 100 \%
		\end{itemize}
	} &
	\alertbox{\faChartLine}{Statistiques des notes}{
		\begin{pspicture}(0,-0.1)(16,1.45)
			\psset{xunit=1,fillstyle=solid}
		   \savedata{\data}[11.7 11.2 18.9 8.4 8.4 10.9 15.7 5.8 8.0 13.5 17.8 16.7 6.0 18.6 9.5 11.6]
		   \rput{-90}(0,0.9){\psBoxplot[barwidth=1.1cm,yunit=0.5,fillcolor=gray,linewidth=1pt]{\data}}
		   \psaxes[yAxis=false,dx=1cm,Dx=2,labelsep=1pt,linecolor=gray,xlabelFontSize=\scriptstyle](0,0)(10.1,4)
		   \psdot[dotsize=8pt,dotstyle=diamond,linecolor=black,fillstyle=solid,fillcolor=white,linewidth=1pt](9.45,0.85)
           \psdot[dotsize=6pt,dotstyle=x,linecolor=black,linewidth=3pt](6.021875,0.85)
		   \end{pspicture}
	} \\
    
\end{tabularx}\\
\begin{tabularx}{\textwidth}{X}
\alertbox{\faComment}{Commentaire}
{
	Excellent travail, bravo !
}
\end{tabularx}
\medskip
    \ding{113} \textbf{\sffamily{Résultats par thème}} \medskip \\
    \renewcommand{\arraystretch}{1.2}
    \begin{tabular}{|l|r|r|}
    \cline{2-3}
    \multicolumn{1}{l|}{} & \multicolumn{1}{|c|}{Points} & \multicolumn{1}{|c|}{Traitées} \\
    \hline
    {Discipline de programmation} & 100\% \;{\small (10/10)} & 100\% \;{\small (2/2)} \\ \hline {Programmation de base en C} & 100\% \;{\small (105/105)} & 100\% \;{\small (11/11)} \\ \hline {Preuve de correction} & 78\% \;{\small (43/55)} & 100\% \;{\small (4/4)} \\ \hline {Preuve de terminaison} & 100\% \;{\small (35/35)} & 100\% \;{\small (3/3)} \\ \hline {Comprendre un algorithme} & 100\% \;{\small (15/15)} & 100\% \;{\small (3/3)} \\ \hline \end{tabular} \\\\\medskip \\
    \ding{113} \textbf{\sffamily{Résultats par exercice}} \medskip \\
    \renewcommand{\arraystretch}{1.2}
    \begin{tabular}{|l|r|r|}
    \cline{2-3}
    \multicolumn{1}{l|}{} & \multicolumn{1}{|c|}{Points} & \multicolumn{1}{|c|}{Traitées} \\
    \hline
    Exercice {1} & 100\% \;{\small (40/40)} & 100\% \;{\small (4/4)} \\ \hline Exercice {2} & 85\% \;{\small (68/80)} & 100\% \;{\small (8/8)} \\ \hline Exercice {3} & 100\% \;{\small (35/35)} & 100\% \;{\small (5/5)} \\ \hline Exercice {4} & 100\% \;{\small (65/65)} & 100\% \;{\small (6/6)} \\ \hline \end{tabular} \\\\\pagebreak
\begin{tcolorbox}[enhanced,width=\textwidth,center upper,fontupper=\bfseries,drop shadow southwest,sharp corners]
{\sc \large Boucher} Mathis
\end{tcolorbox}
\medskip
\begin{tabularx}{\textwidth}{p{5cm}X}
	\alertbox{\faAward}{Note}{
		\begin{itemize}[leftmargin=0pt]
			\item[\textbullet] Note : \textbf{\large 8.4}
			\item[\textbullet] Rang : \textbf{12}
			\item[\textbullet] Traité : 65 \%
		\end{itemize}
	} &
	\alertbox{\faChartLine}{Statistiques des notes}{
		\begin{pspicture}(0,-0.1)(16,1.45)
			\psset{xunit=1,fillstyle=solid}
		   \savedata{\data}[11.7 11.2 18.9 8.4 8.4 10.9 15.7 5.8 8.0 13.5 17.8 16.7 6.0 18.6 9.5 11.6]
		   \rput{-90}(0,0.9){\psBoxplot[barwidth=1.1cm,yunit=0.5,fillcolor=gray,linewidth=1pt]{\data}}
		   \psaxes[yAxis=false,dx=1cm,Dx=2,labelsep=1pt,linecolor=gray,xlabelFontSize=\scriptstyle](0,0)(10.1,4)
		   \psdot[dotsize=8pt,dotstyle=diamond,linecolor=black,fillstyle=solid,fillcolor=white,linewidth=1pt](4.2,0.85)
           \psdot[dotsize=6pt,dotstyle=x,linecolor=black,linewidth=3pt](6.021875,0.85)
		   \end{pspicture}
	} \\
    
\end{tabularx}\\
\begin{tabularx}{\textwidth}{X}
\alertbox{\faComment}{Commentaire}
{
	C’est correct dans l’ensemble, mais il faut gagner en rapidité pour traiter plus de questions. Tu as perdu beaucoup de points sur des questions faciles (notamment l’écriture de fonctions en C)
}
\end{tabularx}
\medskip
    \ding{113} \textbf{\sffamily{Résultats par thème}} \medskip \\
    \renewcommand{\arraystretch}{1.2}
    \begin{tabular}{|l|r|r|}
    \cline{2-3}
    \multicolumn{1}{l|}{} & \multicolumn{1}{|c|}{Points} & \multicolumn{1}{|c|}{Traitées} \\
    \hline
    {Discipline de programmation} & 40\% \;{\small (04/10)} & 50\% \;{\small (1/2)} \\ \hline {Programmation de base en C} & 33\% \;{\small (35/105)} & 54\% \;{\small (6/11)} \\ \hline {Preuve de correction} & 23\% \;{\small (13/55)} & 50\% \;{\small (2/4)} \\ \hline {Preuve de terminaison} & 71\% \;{\small (25/35)} & 100\% \;{\small (3/3)} \\ \hline {Comprendre un algorithme} & 100\% \;{\small (15/15)} & 100\% \;{\small (3/3)} \\ \hline \end{tabular} \\\\\medskip \\
    \ding{113} \textbf{\sffamily{Résultats par exercice}} \medskip \\
    \renewcommand{\arraystretch}{1.2}
    \begin{tabular}{|l|r|r|}
    \cline{2-3}
    \multicolumn{1}{l|}{} & \multicolumn{1}{|c|}{Points} & \multicolumn{1}{|c|}{Traitées} \\
    \hline
    Exercice {1} & 70\% \;{\small (28/40)} & 100\% \;{\small (4/4)} \\ \hline Exercice {2} & 52\% \;{\small (42/80)} & 75\% \;{\small (6/8)} \\ \hline Exercice {3} & 62\% \;{\small (22/35)} & 100\% \;{\small (5/5)} \\ \hline Exercice {4} & 0\% \;{\small (00/65)} & 0\% \;{\small (0/6)} \\ \hline \end{tabular} \\\\\pagebreak
\begin{tcolorbox}[enhanced,width=\textwidth,center upper,fontupper=\bfseries,drop shadow southwest,sharp corners]
{\sc \large Chane-lock} Maxime
\end{tcolorbox}
\medskip
\begin{tabularx}{\textwidth}{p{5cm}X}
	\alertbox{\faAward}{Note}{
		\begin{itemize}[leftmargin=0pt]
			\item[\textbullet] Note : \textbf{\large 8.4}
			\item[\textbullet] Rang : \textbf{13}
			\item[\textbullet] Traité : 83 \%
		\end{itemize}
	} &
	\alertbox{\faChartLine}{Statistiques des notes}{
		\begin{pspicture}(0,-0.1)(16,1.45)
			\psset{xunit=1,fillstyle=solid}
		   \savedata{\data}[11.7 11.2 18.9 8.4 8.4 10.9 15.7 5.8 8.0 13.5 17.8 16.7 6.0 18.6 9.5 11.6]
		   \rput{-90}(0,0.9){\psBoxplot[barwidth=1.1cm,yunit=0.5,fillcolor=gray,linewidth=1pt]{\data}}
		   \psaxes[yAxis=false,dx=1cm,Dx=2,labelsep=1pt,linecolor=gray,xlabelFontSize=\scriptstyle](0,0)(10.1,4)
		   \psdot[dotsize=8pt,dotstyle=diamond,linecolor=black,fillstyle=solid,fillcolor=white,linewidth=1pt](4.2,0.85)
           \psdot[dotsize=6pt,dotstyle=x,linecolor=black,linewidth=3pt](6.021875,0.85)
		   \end{pspicture}
	} \\
    
\end{tabularx}\\
\begin{tabularx}{\textwidth}{X}
\alertbox{\faComment}{Commentaire}
{
	Le cours est flou, il faut revoir d’urgence la notion d’invariant : c’est une propriété (par exemple une égalité) qui dépend des variables du programme.
}
\end{tabularx}
\medskip
    \ding{113} \textbf{\sffamily{Résultats par thème}} \medskip \\
    \renewcommand{\arraystretch}{1.2}
    \begin{tabular}{|l|r|r|}
    \cline{2-3}
    \multicolumn{1}{l|}{} & \multicolumn{1}{|c|}{Points} & \multicolumn{1}{|c|}{Traitées} \\
    \hline
    {Discipline de programmation} & 100\% \;{\small (10/10)} & 100\% \;{\small (2/2)} \\ \hline {Programmation de base en C} & 37\% \;{\small (39/105)} & 63\% \;{\small (7/11)} \\ \hline {Preuve de correction} & 30\% \;{\small (17/55)} & 100\% \;{\small (4/4)} \\ \hline {Preuve de terminaison} & 37\% \;{\small (13/35)} & 100\% \;{\small (3/3)} \\ \hline {Comprendre un algorithme} & 86\% \;{\small (13/15)} & 100\% \;{\small (3/3)} \\ \hline \end{tabular} \\\\\medskip \\
    \ding{113} \textbf{\sffamily{Résultats par exercice}} \medskip \\
    \renewcommand{\arraystretch}{1.2}
    \begin{tabular}{|l|r|r|}
    \cline{2-3}
    \multicolumn{1}{l|}{} & \multicolumn{1}{|c|}{Points} & \multicolumn{1}{|c|}{Traitées} \\
    \hline
    Exercice {1} & 55\% \;{\small (22/40)} & 100\% \;{\small (4/4)} \\ \hline Exercice {2} & 43\% \;{\small (35/80)} & 87\% \;{\small (7/8)} \\ \hline Exercice {3} & 54\% \;{\small (19/35)} & 100\% \;{\small (5/5)} \\ \hline Exercice {4} & 24\% \;{\small (16/65)} & 50\% \;{\small (3/6)} \\ \hline \end{tabular} \\\\\pagebreak
\begin{tcolorbox}[enhanced,width=\textwidth,center upper,fontupper=\bfseries,drop shadow southwest,sharp corners]
{\sc \large Courounadin-Mouny} Maxence
\end{tcolorbox}
\medskip
\begin{tabularx}{\textwidth}{p{5cm}X}
	\alertbox{\faAward}{Note}{
		\begin{itemize}[leftmargin=0pt]
			\item[\textbullet] Note : \textbf{\large 10.9}
			\item[\textbullet] Rang : \textbf{10}
			\item[\textbullet] Traité : 96 \%
		\end{itemize}
	} &
	\alertbox{\faChartLine}{Statistiques des notes}{
		\begin{pspicture}(0,-0.1)(16,1.45)
			\psset{xunit=1,fillstyle=solid}
		   \savedata{\data}[11.7 11.2 18.9 8.4 8.4 10.9 15.7 5.8 8.0 13.5 17.8 16.7 6.0 18.6 9.5 11.6]
		   \rput{-90}(0,0.9){\psBoxplot[barwidth=1.1cm,yunit=0.5,fillcolor=gray,linewidth=1pt]{\data}}
		   \psaxes[yAxis=false,dx=1cm,Dx=2,labelsep=1pt,linecolor=gray,xlabelFontSize=\scriptstyle](0,0)(10.1,4)
		   \psdot[dotsize=8pt,dotstyle=diamond,linecolor=black,fillstyle=solid,fillcolor=white,linewidth=1pt](5.45,0.85)
           \psdot[dotsize=6pt,dotstyle=x,linecolor=black,linewidth=3pt](6.021875,0.85)
		   \end{pspicture}
	} \\
    
\end{tabularx}\\
\begin{tabularx}{\textwidth}{X}
\alertbox{\faComment}{Commentaire}
{
	L’ensemble est correct. Essaye d’aller plus vite pour traiter le maximum de questions faciles. Attention à la confusion entre {\tt sizeof} (qui renvoie la taille en mémoire d’un objet) et {\tt strlen} (fonction de la librairie {\tt <string.h>} qui renvoie la longueur d’une chaine de caractères).
}
\end{tabularx}
\medskip
    \ding{113} \textbf{\sffamily{Résultats par thème}} \medskip \\
    \renewcommand{\arraystretch}{1.2}
    \begin{tabular}{|l|r|r|}
    \cline{2-3}
    \multicolumn{1}{l|}{} & \multicolumn{1}{|c|}{Points} & \multicolumn{1}{|c|}{Traitées} \\
    \hline
    {Discipline de programmation} & 90\% \;{\small (09/10)} & 100\% \;{\small (2/2)} \\ \hline {Programmation de base en C} & 53\% \;{\small (56/105)} & 100\% \;{\small (11/11)} \\ \hline {Preuve de correction} & 29\% \;{\small (16/55)} & 75\% \;{\small (3/4)} \\ \hline {Preuve de terminaison} & 71\% \;{\small (25/35)} & 100\% \;{\small (3/3)} \\ \hline {Comprendre un algorithme} & 93\% \;{\small (14/15)} & 100\% \;{\small (3/3)} \\ \hline \end{tabular} \\\\\medskip \\
    \ding{113} \textbf{\sffamily{Résultats par exercice}} \medskip \\
    \renewcommand{\arraystretch}{1.2}
    \begin{tabular}{|l|r|r|}
    \cline{2-3}
    \multicolumn{1}{l|}{} & \multicolumn{1}{|c|}{Points} & \multicolumn{1}{|c|}{Traitées} \\
    \hline
    Exercice {1} & 77\% \;{\small (31/40)} & 100\% \;{\small (4/4)} \\ \hline Exercice {2} & 60\% \;{\small (48/80)} & 100\% \;{\small (8/8)} \\ \hline Exercice {3} & 60\% \;{\small (21/35)} & 100\% \;{\small (5/5)} \\ \hline Exercice {4} & 30\% \;{\small (20/65)} & 83\% \;{\small (5/6)} \\ \hline \end{tabular} \\\\\pagebreak
\begin{tcolorbox}[enhanced,width=\textwidth,center upper,fontupper=\bfseries,drop shadow southwest,sharp corners]
{\sc \large Dominguez} Raphaël
\end{tcolorbox}
\medskip
\begin{tabularx}{\textwidth}{p{5cm}X}
	\alertbox{\faAward}{Note}{
		\begin{itemize}[leftmargin=0pt]
			\item[\textbullet] Note : \textbf{\large 15.7}
			\item[\textbullet] Rang : \textbf{5}
			\item[\textbullet] Traité : 96 \%
		\end{itemize}
	} &
	\alertbox{\faChartLine}{Statistiques des notes}{
		\begin{pspicture}(0,-0.1)(16,1.45)
			\psset{xunit=1,fillstyle=solid}
		   \savedata{\data}[11.7 11.2 18.9 8.4 8.4 10.9 15.7 5.8 8.0 13.5 17.8 16.7 6.0 18.6 9.5 11.6]
		   \rput{-90}(0,0.9){\psBoxplot[barwidth=1.1cm,yunit=0.5,fillcolor=gray,linewidth=1pt]{\data}}
		   \psaxes[yAxis=false,dx=1cm,Dx=2,labelsep=1pt,linecolor=gray,xlabelFontSize=\scriptstyle](0,0)(10.1,4)
		   \psdot[dotsize=8pt,dotstyle=diamond,linecolor=black,fillstyle=solid,fillcolor=white,linewidth=1pt](7.85,0.85)
           \psdot[dotsize=6pt,dotstyle=x,linecolor=black,linewidth=3pt](6.021875,0.85)
		   \end{pspicture}
	} \\
    
\end{tabularx}\\
\begin{tabularx}{\textwidth}{X}
\alertbox{\faComment}{Commentaire}
{
	C’est bien, il faut pratiquer davantage la programmation afin d’accumuler moins de fautes de syntaxe du C dans les codes de tes fonctions
}
\end{tabularx}
\medskip
    \ding{113} \textbf{\sffamily{Résultats par thème}} \medskip \\
    \renewcommand{\arraystretch}{1.2}
    \begin{tabular}{|l|r|r|}
    \cline{2-3}
    \multicolumn{1}{l|}{} & \multicolumn{1}{|c|}{Points} & \multicolumn{1}{|c|}{Traitées} \\
    \hline
    {Discipline de programmation} & 90\% \;{\small (09/10)} & 100\% \;{\small (2/2)} \\ \hline {Programmation de base en C} & 60\% \;{\small (64/105)} & 90\% \;{\small (10/11)} \\ \hline {Preuve de correction} & 94\% \;{\small (52/55)} & 100\% \;{\small (4/4)} \\ \hline {Preuve de terminaison} & 94\% \;{\small (33/35)} & 100\% \;{\small (3/3)} \\ \hline {Comprendre un algorithme} & 100\% \;{\small (15/15)} & 100\% \;{\small (3/3)} \\ \hline \end{tabular} \\\\\medskip \\
    \ding{113} \textbf{\sffamily{Résultats par exercice}} \medskip \\
    \renewcommand{\arraystretch}{1.2}
    \begin{tabular}{|l|r|r|}
    \cline{2-3}
    \multicolumn{1}{l|}{} & \multicolumn{1}{|c|}{Points} & \multicolumn{1}{|c|}{Traitées} \\
    \hline
    Exercice {1} & 92\% \;{\small (37/40)} & 100\% \;{\small (4/4)} \\ \hline Exercice {2} & 85\% \;{\small (68/80)} & 100\% \;{\small (8/8)} \\ \hline Exercice {3} & 94\% \;{\small (33/35)} & 100\% \;{\small (5/5)} \\ \hline Exercice {4} & 53\% \;{\small (35/65)} & 83\% \;{\small (5/6)} \\ \hline \end{tabular} \\\\\pagebreak
\begin{tcolorbox}[enhanced,width=\textwidth,center upper,fontupper=\bfseries,drop shadow southwest,sharp corners]
{\sc \large Garbal} Alizée
\end{tcolorbox}
\medskip
\begin{tabularx}{\textwidth}{p{5cm}X}
	\alertbox{\faAward}{Note}{
		\begin{itemize}[leftmargin=0pt]
			\item[\textbullet] Note : \textbf{\large 5.8}
			\item[\textbullet] Rang : \textbf{16}
			\item[\textbullet] Traité : 83 \%
		\end{itemize}
	} &
	\alertbox{\faChartLine}{Statistiques des notes}{
		\begin{pspicture}(0,-0.1)(16,1.45)
			\psset{xunit=1,fillstyle=solid}
		   \savedata{\data}[11.7 11.2 18.9 8.4 8.4 10.9 15.7 5.8 8.0 13.5 17.8 16.7 6.0 18.6 9.5 11.6]
		   \rput{-90}(0,0.9){\psBoxplot[barwidth=1.1cm,yunit=0.5,fillcolor=gray,linewidth=1pt]{\data}}
		   \psaxes[yAxis=false,dx=1cm,Dx=2,labelsep=1pt,linecolor=gray,xlabelFontSize=\scriptstyle](0,0)(10.1,4)
		   \psdot[dotsize=8pt,dotstyle=diamond,linecolor=black,fillstyle=solid,fillcolor=white,linewidth=1pt](2.9,0.85)
           \psdot[dotsize=6pt,dotstyle=x,linecolor=black,linewidth=3pt](6.021875,0.85)
		   \end{pspicture}
	} \\
    
\end{tabularx}\\
\begin{tabularx}{\textwidth}{X}
\alertbox{\faComment}{Commentaire}
{
	Les notions de variant et d’invariants ne sont pas acquises, il faut revoir le cours et les exemples. Pratique davantage le C à la maison afin d’aller plus vite et d’acquérir des automatismes
}
\end{tabularx}
\medskip
    \ding{113} \textbf{\sffamily{Résultats par thème}} \medskip \\
    \renewcommand{\arraystretch}{1.2}
    \begin{tabular}{|l|r|r|}
    \cline{2-3}
    \multicolumn{1}{l|}{} & \multicolumn{1}{|c|}{Points} & \multicolumn{1}{|c|}{Traitées} \\
    \hline
    {Discipline de programmation} & 80\% \;{\small (08/10)} & 100\% \;{\small (2/2)} \\ \hline {Programmation de base en C} & 27\% \;{\small (29/105)} & 72\% \;{\small (8/11)} \\ \hline {Preuve de correction} & 3\% \;{\small (02/55)} & 75\% \;{\small (3/4)} \\ \hline {Preuve de terminaison} & 31\% \;{\small (11/35)} & 100\% \;{\small (3/3)} \\ \hline {Comprendre un algorithme} & 93\% \;{\small (14/15)} & 100\% \;{\small (3/3)} \\ \hline \end{tabular} \\\\\medskip \\
    \ding{113} \textbf{\sffamily{Résultats par exercice}} \medskip \\
    \renewcommand{\arraystretch}{1.2}
    \begin{tabular}{|l|r|r|}
    \cline{2-3}
    \multicolumn{1}{l|}{} & \multicolumn{1}{|c|}{Points} & \multicolumn{1}{|c|}{Traitées} \\
    \hline
    Exercice {1} & 47\% \;{\small (19/40)} & 100\% \;{\small (4/4)} \\ \hline Exercice {2} & 22\% \;{\small (18/80)} & 100\% \;{\small (8/8)} \\ \hline Exercice {3} & 37\% \;{\small (13/35)} & 100\% \;{\small (5/5)} \\ \hline Exercice {4} & 21\% \;{\small (14/65)} & 33\% \;{\small (2/6)} \\ \hline \end{tabular} \\\\\pagebreak
\begin{tcolorbox}[enhanced,width=\textwidth,center upper,fontupper=\bfseries,drop shadow southwest,sharp corners]
{\sc \large Hoarau} Alessandro
\end{tcolorbox}
\medskip
\begin{tabularx}{\textwidth}{p{5cm}X}
	\alertbox{\faAward}{Note}{
		\begin{itemize}[leftmargin=0pt]
			\item[\textbullet] Note : \textbf{\large 8.0}
			\item[\textbullet] Rang : \textbf{14}
			\item[\textbullet] Traité : 83 \%
		\end{itemize}
	} &
	\alertbox{\faChartLine}{Statistiques des notes}{
		\begin{pspicture}(0,-0.1)(16,1.45)
			\psset{xunit=1,fillstyle=solid}
		   \savedata{\data}[11.7 11.2 18.9 8.4 8.4 10.9 15.7 5.8 8.0 13.5 17.8 16.7 6.0 18.6 9.5 11.6]
		   \rput{-90}(0,0.9){\psBoxplot[barwidth=1.1cm,yunit=0.5,fillcolor=gray,linewidth=1pt]{\data}}
		   \psaxes[yAxis=false,dx=1cm,Dx=2,labelsep=1pt,linecolor=gray,xlabelFontSize=\scriptstyle](0,0)(10.1,4)
		   \psdot[dotsize=8pt,dotstyle=diamond,linecolor=black,fillstyle=solid,fillcolor=white,linewidth=1pt](4.0,0.85)
           \psdot[dotsize=6pt,dotstyle=x,linecolor=black,linewidth=3pt](6.021875,0.85)
		   \end{pspicture}
	} \\
    
\end{tabularx}\\
\begin{tabularx}{\textwidth}{X}
\alertbox{\faComment}{Commentaire}
{
	C’est correct pour la partie programmation en C, par cours les notions de variant et d’invariant sont encore mal comprises. Il faut retravailler cette partie du cours.
}
\end{tabularx}
\medskip
    \ding{113} \textbf{\sffamily{Résultats par thème}} \medskip \\
    \renewcommand{\arraystretch}{1.2}
    \begin{tabular}{|l|r|r|}
    \cline{2-3}
    \multicolumn{1}{l|}{} & \multicolumn{1}{|c|}{Points} & \multicolumn{1}{|c|}{Traitées} \\
    \hline
    {Discipline de programmation} & 40\% \;{\small (04/10)} & 100\% \;{\small (2/2)} \\ \hline {Programmation de base en C} & 47\% \;{\small (50/105)} & 81\% \;{\small (9/11)} \\ \hline {Preuve de correction} & 16\% \;{\small (09/55)} & 50\% \;{\small (2/4)} \\ \hline {Preuve de terminaison} & 28\% \;{\small (10/35)} & 100\% \;{\small (3/3)} \\ \hline {Comprendre un algorithme} & 100\% \;{\small (15/15)} & 100\% \;{\small (3/3)} \\ \hline \end{tabular} \\\\\medskip \\
    \ding{113} \textbf{\sffamily{Résultats par exercice}} \medskip \\
    \renewcommand{\arraystretch}{1.2}
    \begin{tabular}{|l|r|r|}
    \cline{2-3}
    \multicolumn{1}{l|}{} & \multicolumn{1}{|c|}{Points} & \multicolumn{1}{|c|}{Traitées} \\
    \hline
    Exercice {1} & 47\% \;{\small (19/40)} & 100\% \;{\small (4/4)} \\ \hline Exercice {2} & 51\% \;{\small (41/80)} & 87\% \;{\small (7/8)} \\ \hline Exercice {3} & 62\% \;{\small (22/35)} & 100\% \;{\small (5/5)} \\ \hline Exercice {4} & 9\% \;{\small (06/65)} & 50\% \;{\small (3/6)} \\ \hline \end{tabular} \\\\\pagebreak
\begin{tcolorbox}[enhanced,width=\textwidth,center upper,fontupper=\bfseries,drop shadow southwest,sharp corners]
{\sc \large Mahomed Issop} Jérémy
\end{tcolorbox}
\medskip
\begin{tabularx}{\textwidth}{p{5cm}X}
	\alertbox{\faAward}{Note}{
		\begin{itemize}[leftmargin=0pt]
			\item[\textbullet] Note : \textbf{\large 13.5}
			\item[\textbullet] Rang : \textbf{6}
			\item[\textbullet] Traité : 83 \%
		\end{itemize}
	} &
	\alertbox{\faChartLine}{Statistiques des notes}{
		\begin{pspicture}(0,-0.1)(16,1.45)
			\psset{xunit=1,fillstyle=solid}
		   \savedata{\data}[11.7 11.2 18.9 8.4 8.4 10.9 15.7 5.8 8.0 13.5 17.8 16.7 6.0 18.6 9.5 11.6]
		   \rput{-90}(0,0.9){\psBoxplot[barwidth=1.1cm,yunit=0.5,fillcolor=gray,linewidth=1pt]{\data}}
		   \psaxes[yAxis=false,dx=1cm,Dx=2,labelsep=1pt,linecolor=gray,xlabelFontSize=\scriptstyle](0,0)(10.1,4)
		   \psdot[dotsize=8pt,dotstyle=diamond,linecolor=black,fillstyle=solid,fillcolor=white,linewidth=1pt](6.75,0.85)
           \psdot[dotsize=6pt,dotstyle=x,linecolor=black,linewidth=3pt](6.021875,0.85)
		   \end{pspicture}
	} \\
    
\end{tabularx}\\
\begin{tabularx}{\textwidth}{X}
\alertbox{\faComment}{Commentaire}
{
	Très bon travail, tu as simplement manqué de temps, il faut travailler cet aspect des choses en essayant d’aller plus vite sur les questions faciles.
}
\end{tabularx}
\medskip
    \ding{113} \textbf{\sffamily{Résultats par thème}} \medskip \\
    \renewcommand{\arraystretch}{1.2}
    \begin{tabular}{|l|r|r|}
    \cline{2-3}
    \multicolumn{1}{l|}{} & \multicolumn{1}{|c|}{Points} & \multicolumn{1}{|c|}{Traitées} \\
    \hline
    {Discipline de programmation} & 100\% \;{\small (10/10)} & 100\% \;{\small (2/2)} \\ \hline {Programmation de base en C} & 52\% \;{\small (55/105)} & 72\% \;{\small (8/11)} \\ \hline {Preuve de correction} & 61\% \;{\small (34/55)} & 75\% \;{\small (3/4)} \\ \hline {Preuve de terminaison} & 100\% \;{\small (35/35)} & 100\% \;{\small (3/3)} \\ \hline {Comprendre un algorithme} & 100\% \;{\small (15/15)} & 100\% \;{\small (3/3)} \\ \hline \end{tabular} \\\\\medskip \\
    \ding{113} \textbf{\sffamily{Résultats par exercice}} \medskip \\
    \renewcommand{\arraystretch}{1.2}
    \begin{tabular}{|l|r|r|}
    \cline{2-3}
    \multicolumn{1}{l|}{} & \multicolumn{1}{|c|}{Points} & \multicolumn{1}{|c|}{Traitées} \\
    \hline
    Exercice {1} & 100\% \;{\small (40/40)} & 100\% \;{\small (4/4)} \\ \hline Exercice {2} & 41\% \;{\small (33/80)} & 62\% \;{\small (5/8)} \\ \hline Exercice {3} & 100\% \;{\small (35/35)} & 100\% \;{\small (5/5)} \\ \hline Exercice {4} & 63\% \;{\small (41/65)} & 83\% \;{\small (5/6)} \\ \hline \end{tabular} \\\\\pagebreak
\begin{tcolorbox}[enhanced,width=\textwidth,center upper,fontupper=\bfseries,drop shadow southwest,sharp corners]
{\sc \large Mamodhoussen} Djavad
\end{tcolorbox}
\medskip
\begin{tabularx}{\textwidth}{p{5cm}X}
	\alertbox{\faAward}{Note}{
		\begin{itemize}[leftmargin=0pt]
			\item[\textbullet] Note : \textbf{\large 17.8}
			\item[\textbullet] Rang : \textbf{3}
			\item[\textbullet] Traité : 100 \%
		\end{itemize}
	} &
	\alertbox{\faChartLine}{Statistiques des notes}{
		\begin{pspicture}(0,-0.1)(16,1.45)
			\psset{xunit=1,fillstyle=solid}
		   \savedata{\data}[11.7 11.2 18.9 8.4 8.4 10.9 15.7 5.8 8.0 13.5 17.8 16.7 6.0 18.6 9.5 11.6]
		   \rput{-90}(0,0.9){\psBoxplot[barwidth=1.1cm,yunit=0.5,fillcolor=gray,linewidth=1pt]{\data}}
		   \psaxes[yAxis=false,dx=1cm,Dx=2,labelsep=1pt,linecolor=gray,xlabelFontSize=\scriptstyle](0,0)(10.1,4)
		   \psdot[dotsize=8pt,dotstyle=diamond,linecolor=black,fillstyle=solid,fillcolor=white,linewidth=1pt](8.9,0.85)
           \psdot[dotsize=6pt,dotstyle=x,linecolor=black,linewidth=3pt](6.021875,0.85)
		   \end{pspicture}
	} \\
    
\end{tabularx}\\
\begin{tabularx}{\textwidth}{X}
\alertbox{\faComment}{Commentaire}
{
	Excellent travail, bravo ! Attention à différencier les écritures mathématiques de celles du C. Si $n$ et $d$ sont entiers (avec $d$ non nul) alors $n/d$ est un rationnel si tu parles de {\tt n/d} en C (qui est entier) il faut le dire ou utiliser la notation mathématique appropriée.
}
\end{tabularx}
\medskip
    \ding{113} \textbf{\sffamily{Résultats par thème}} \medskip \\
    \renewcommand{\arraystretch}{1.2}
    \begin{tabular}{|l|r|r|}
    \cline{2-3}
    \multicolumn{1}{l|}{} & \multicolumn{1}{|c|}{Points} & \multicolumn{1}{|c|}{Traitées} \\
    \hline
    {Discipline de programmation} & 100\% \;{\small (10/10)} & 100\% \;{\small (2/2)} \\ \hline {Programmation de base en C} & 77\% \;{\small (81/105)} & 100\% \;{\small (11/11)} \\ \hline {Preuve de correction} & 100\% \;{\small (55/55)} & 100\% \;{\small (4/4)} \\ \hline {Preuve de terminaison} & 100\% \;{\small (35/35)} & 100\% \;{\small (3/3)} \\ \hline {Comprendre un algorithme} & 100\% \;{\small (15/15)} & 100\% \;{\small (3/3)} \\ \hline \end{tabular} \\\\\medskip \\
    \ding{113} \textbf{\sffamily{Résultats par exercice}} \medskip \\
    \renewcommand{\arraystretch}{1.2}
    \begin{tabular}{|l|r|r|}
    \cline{2-3}
    \multicolumn{1}{l|}{} & \multicolumn{1}{|c|}{Points} & \multicolumn{1}{|c|}{Traitées} \\
    \hline
    Exercice {1} & 100\% \;{\small (40/40)} & 100\% \;{\small (4/4)} \\ \hline Exercice {2} & 87\% \;{\small (70/80)} & 100\% \;{\small (8/8)} \\ \hline Exercice {3} & 94\% \;{\small (33/35)} & 100\% \;{\small (5/5)} \\ \hline Exercice {4} & 81\% \;{\small (53/65)} & 100\% \;{\small (6/6)} \\ \hline \end{tabular} \\\\\pagebreak
\begin{tcolorbox}[enhanced,width=\textwidth,center upper,fontupper=\bfseries,drop shadow southwest,sharp corners]
{\sc \large Morel} Lucas
\end{tcolorbox}
\medskip
\begin{tabularx}{\textwidth}{p{5cm}X}
	\alertbox{\faAward}{Note}{
		\begin{itemize}[leftmargin=0pt]
			\item[\textbullet] Note : \textbf{\large 16.7}
			\item[\textbullet] Rang : \textbf{4}
			\item[\textbullet] Traité : 96 \%
		\end{itemize}
	} &
	\alertbox{\faChartLine}{Statistiques des notes}{
		\begin{pspicture}(0,-0.1)(16,1.45)
			\psset{xunit=1,fillstyle=solid}
		   \savedata{\data}[11.7 11.2 18.9 8.4 8.4 10.9 15.7 5.8 8.0 13.5 17.8 16.7 6.0 18.6 9.5 11.6]
		   \rput{-90}(0,0.9){\psBoxplot[barwidth=1.1cm,yunit=0.5,fillcolor=gray,linewidth=1pt]{\data}}
		   \psaxes[yAxis=false,dx=1cm,Dx=2,labelsep=1pt,linecolor=gray,xlabelFontSize=\scriptstyle](0,0)(10.1,4)
		   \psdot[dotsize=8pt,dotstyle=diamond,linecolor=black,fillstyle=solid,fillcolor=white,linewidth=1pt](8.35,0.85)
           \psdot[dotsize=6pt,dotstyle=x,linecolor=black,linewidth=3pt](6.021875,0.85)
		   \end{pspicture}
	} \\
    
\end{tabularx}\\
\begin{tabularx}{\textwidth}{X}
\alertbox{\faComment}{Commentaire}
{
	Très bon travail. Prends le temps d’écrire (et de tester) la fonction qui renvoie les deux premiers minimums, dans le cas où on rencontre un élément inférieur à {\tt min1}, on doit modifier {\tt min1} et {\tt min2}.
}
\end{tabularx}
\medskip
    \ding{113} \textbf{\sffamily{Résultats par thème}} \medskip \\
    \renewcommand{\arraystretch}{1.2}
    \begin{tabular}{|l|r|r|}
    \cline{2-3}
    \multicolumn{1}{l|}{} & \multicolumn{1}{|c|}{Points} & \multicolumn{1}{|c|}{Traitées} \\
    \hline
    {Discipline de programmation} & 100\% \;{\small (10/10)} & 100\% \;{\small (2/2)} \\ \hline {Programmation de base en C} & 88\% \;{\small (93/105)} & 100\% \;{\small (11/11)} \\ \hline {Preuve de correction} & 56\% \;{\small (31/55)} & 75\% \;{\small (3/4)} \\ \hline {Preuve de terminaison} & 100\% \;{\small (35/35)} & 100\% \;{\small (3/3)} \\ \hline {Comprendre un algorithme} & 100\% \;{\small (15/15)} & 100\% \;{\small (3/3)} \\ \hline \end{tabular} \\\\\medskip \\
    \ding{113} \textbf{\sffamily{Résultats par exercice}} \medskip \\
    \renewcommand{\arraystretch}{1.2}
    \begin{tabular}{|l|r|r|}
    \cline{2-3}
    \multicolumn{1}{l|}{} & \multicolumn{1}{|c|}{Points} & \multicolumn{1}{|c|}{Traitées} \\
    \hline
    Exercice {1} & 100\% \;{\small (40/40)} & 100\% \;{\small (4/4)} \\ \hline Exercice {2} & 76\% \;{\small (61/80)} & 87\% \;{\small (7/8)} \\ \hline Exercice {3} & 94\% \;{\small (33/35)} & 100\% \;{\small (5/5)} \\ \hline Exercice {4} & 76\% \;{\small (50/65)} & 100\% \;{\small (6/6)} \\ \hline \end{tabular} \\\\\pagebreak
\begin{tcolorbox}[enhanced,width=\textwidth,center upper,fontupper=\bfseries,drop shadow southwest,sharp corners]
{\sc \large Payet} Arnaud
\end{tcolorbox}
\medskip
\begin{tabularx}{\textwidth}{p{5cm}X}
	\alertbox{\faAward}{Note}{
		\begin{itemize}[leftmargin=0pt]
			\item[\textbullet] Note : \textbf{\large 6.0}
			\item[\textbullet] Rang : \textbf{15}
			\item[\textbullet] Traité : 74 \%
		\end{itemize}
	} &
	\alertbox{\faChartLine}{Statistiques des notes}{
		\begin{pspicture}(0,-0.1)(16,1.45)
			\psset{xunit=1,fillstyle=solid}
		   \savedata{\data}[11.7 11.2 18.9 8.4 8.4 10.9 15.7 5.8 8.0 13.5 17.8 16.7 6.0 18.6 9.5 11.6]
		   \rput{-90}(0,0.9){\psBoxplot[barwidth=1.1cm,yunit=0.5,fillcolor=gray,linewidth=1pt]{\data}}
		   \psaxes[yAxis=false,dx=1cm,Dx=2,labelsep=1pt,linecolor=gray,xlabelFontSize=\scriptstyle](0,0)(10.1,4)
		   \psdot[dotsize=8pt,dotstyle=diamond,linecolor=black,fillstyle=solid,fillcolor=white,linewidth=1pt](3.0,0.85)
           \psdot[dotsize=6pt,dotstyle=x,linecolor=black,linewidth=3pt](6.021875,0.85)
		   \end{pspicture}
	} \\
    
\end{tabularx}\\
\begin{tabularx}{\textwidth}{X}
\alertbox{\faComment}{Commentaire}
{
	Il y a de bons éléments mais les notions de terminaison et de correction ne sont pas acquises, il faut les retravailler et refaire seul les exemples du cours.
}
\end{tabularx}
\medskip
    \ding{113} \textbf{\sffamily{Résultats par thème}} \medskip \\
    \renewcommand{\arraystretch}{1.2}
    \begin{tabular}{|l|r|r|}
    \cline{2-3}
    \multicolumn{1}{l|}{} & \multicolumn{1}{|c|}{Points} & \multicolumn{1}{|c|}{Traitées} \\
    \hline
    {Discipline de programmation} & 0\% \;{\small (00/10)} & 50\% \;{\small (1/2)} \\ \hline {Programmation de base en C} & 34\% \;{\small (36/105)} & 63\% \;{\small (7/11)} \\ \hline {Preuve de correction} & 16\% \;{\small (09/55)} & 75\% \;{\small (3/4)} \\ \hline {Preuve de terminaison} & 22\% \;{\small (08/35)} & 100\% \;{\small (3/3)} \\ \hline {Comprendre un algorithme} & 86\% \;{\small (13/15)} & 100\% \;{\small (3/3)} \\ \hline \end{tabular} \\\\\medskip \\
    \ding{113} \textbf{\sffamily{Résultats par exercice}} \medskip \\
    \renewcommand{\arraystretch}{1.2}
    \begin{tabular}{|l|r|r|}
    \cline{2-3}
    \multicolumn{1}{l|}{} & \multicolumn{1}{|c|}{Points} & \multicolumn{1}{|c|}{Traitées} \\
    \hline
    Exercice {1} & 47\% \;{\small (19/40)} & 100\% \;{\small (4/4)} \\ \hline Exercice {2} & 48\% \;{\small (39/80)} & 100\% \;{\small (8/8)} \\ \hline Exercice {3} & 22\% \;{\small (08/35)} & 100\% \;{\small (5/5)} \\ \hline Exercice {4} & 0\% \;{\small (00/65)} & 0\% \;{\small (0/6)} \\ \hline \end{tabular} \\\\\pagebreak
\begin{tcolorbox}[enhanced,width=\textwidth,center upper,fontupper=\bfseries,drop shadow southwest,sharp corners]
{\sc \large Randriamiarivola Korodo} Lionel
\end{tcolorbox}
\medskip
\begin{tabularx}{\textwidth}{p{5cm}X}
	\alertbox{\faAward}{Note}{
		\begin{itemize}[leftmargin=0pt]
			\item[\textbullet] Note : \textbf{\large 18.6}
			\item[\textbullet] Rang : \textbf{2}
			\item[\textbullet] Traité : 100 \%
		\end{itemize}
	} &
	\alertbox{\faChartLine}{Statistiques des notes}{
		\begin{pspicture}(0,-0.1)(16,1.45)
			\psset{xunit=1,fillstyle=solid}
		   \savedata{\data}[11.7 11.2 18.9 8.4 8.4 10.9 15.7 5.8 8.0 13.5 17.8 16.7 6.0 18.6 9.5 11.6]
		   \rput{-90}(0,0.9){\psBoxplot[barwidth=1.1cm,yunit=0.5,fillcolor=gray,linewidth=1pt]{\data}}
		   \psaxes[yAxis=false,dx=1cm,Dx=2,labelsep=1pt,linecolor=gray,xlabelFontSize=\scriptstyle](0,0)(10.1,4)
		   \psdot[dotsize=8pt,dotstyle=diamond,linecolor=black,fillstyle=solid,fillcolor=white,linewidth=1pt](9.3,0.85)
           \psdot[dotsize=6pt,dotstyle=x,linecolor=black,linewidth=3pt](6.021875,0.85)
		   \end{pspicture}
	} \\
    
\end{tabularx}\\
\begin{tabularx}{\textwidth}{X}
\alertbox{\faComment}{Commentaire}
{
	Excellent travail ! Bravo
}
\end{tabularx}
\medskip
    \ding{113} \textbf{\sffamily{Résultats par thème}} \medskip \\
    \renewcommand{\arraystretch}{1.2}
    \begin{tabular}{|l|r|r|}
    \cline{2-3}
    \multicolumn{1}{l|}{} & \multicolumn{1}{|c|}{Points} & \multicolumn{1}{|c|}{Traitées} \\
    \hline
    {Discipline de programmation} & 90\% \;{\small (09/10)} & 100\% \;{\small (2/2)} \\ \hline {Programmation de base en C} & 98\% \;{\small (103/105)} & 100\% \;{\small (11/11)} \\ \hline {Preuve de correction} & 78\% \;{\small (43/55)} & 100\% \;{\small (4/4)} \\ \hline {Preuve de terminaison} & 100\% \;{\small (35/35)} & 100\% \;{\small (3/3)} \\ \hline {Comprendre un algorithme} & 100\% \;{\small (15/15)} & 100\% \;{\small (3/3)} \\ \hline \end{tabular} \\\\\medskip \\
    \ding{113} \textbf{\sffamily{Résultats par exercice}} \medskip \\
    \renewcommand{\arraystretch}{1.2}
    \begin{tabular}{|l|r|r|}
    \cline{2-3}
    \multicolumn{1}{l|}{} & \multicolumn{1}{|c|}{Points} & \multicolumn{1}{|c|}{Traitées} \\
    \hline
    Exercice {1} & 100\% \;{\small (40/40)} & 100\% \;{\small (4/4)} \\ \hline Exercice {2} & 85\% \;{\small (68/80)} & 100\% \;{\small (8/8)} \\ \hline Exercice {3} & 91\% \;{\small (32/35)} & 100\% \;{\small (5/5)} \\ \hline Exercice {4} & 100\% \;{\small (65/65)} & 100\% \;{\small (6/6)} \\ \hline \end{tabular} \\\\\pagebreak
\begin{tcolorbox}[enhanced,width=\textwidth,center upper,fontupper=\bfseries,drop shadow southwest,sharp corners]
{\sc \large Rasolofotsara} Ando
\end{tcolorbox}
\medskip
\begin{tabularx}{\textwidth}{p{5cm}X}
	\alertbox{\faAward}{Note}{
		\begin{itemize}[leftmargin=0pt]
			\item[\textbullet] Note : \textbf{\large 9.5}
			\item[\textbullet] Rang : \textbf{11}
			\item[\textbullet] Traité : 91 \%
		\end{itemize}
	} &
	\alertbox{\faChartLine}{Statistiques des notes}{
		\begin{pspicture}(0,-0.1)(16,1.45)
			\psset{xunit=1,fillstyle=solid}
		   \savedata{\data}[11.7 11.2 18.9 8.4 8.4 10.9 15.7 5.8 8.0 13.5 17.8 16.7 6.0 18.6 9.5 11.6]
		   \rput{-90}(0,0.9){\psBoxplot[barwidth=1.1cm,yunit=0.5,fillcolor=gray,linewidth=1pt]{\data}}
		   \psaxes[yAxis=false,dx=1cm,Dx=2,labelsep=1pt,linecolor=gray,xlabelFontSize=\scriptstyle](0,0)(10.1,4)
		   \psdot[dotsize=8pt,dotstyle=diamond,linecolor=black,fillstyle=solid,fillcolor=white,linewidth=1pt](4.75,0.85)
           \psdot[dotsize=6pt,dotstyle=x,linecolor=black,linewidth=3pt](6.021875,0.85)
		   \end{pspicture}
	} \\
    
\end{tabularx}\\
\begin{tabularx}{\textwidth}{X}
\alertbox{\faComment}{Commentaire}
{
	L’ensemble est correct, mais il faut revoir les preuves de terminaison et de correction. Tu devrais aussi refaire à la maison l’exercice sur les deux premiers minimums en écrivant et testant les fonctions demandées.
}
\end{tabularx}
\medskip
    \ding{113} \textbf{\sffamily{Résultats par thème}} \medskip \\
    \renewcommand{\arraystretch}{1.2}
    \begin{tabular}{|l|r|r|}
    \cline{2-3}
    \multicolumn{1}{l|}{} & \multicolumn{1}{|c|}{Points} & \multicolumn{1}{|c|}{Traitées} \\
    \hline
    {Discipline de programmation} & 100\% \;{\small (10/10)} & 100\% \;{\small (2/2)} \\ \hline {Programmation de base en C} & 44\% \;{\small (47/105)} & 90\% \;{\small (10/11)} \\ \hline {Preuve de correction} & 10\% \;{\small (06/55)} & 75\% \;{\small (3/4)} \\ \hline {Preuve de terminaison} & 77\% \;{\small (27/35)} & 100\% \;{\small (3/3)} \\ \hline {Comprendre un algorithme} & 100\% \;{\small (15/15)} & 100\% \;{\small (3/3)} \\ \hline \end{tabular} \\\\\medskip \\
    \ding{113} \textbf{\sffamily{Résultats par exercice}} \medskip \\
    \renewcommand{\arraystretch}{1.2}
    \begin{tabular}{|l|r|r|}
    \cline{2-3}
    \multicolumn{1}{l|}{} & \multicolumn{1}{|c|}{Points} & \multicolumn{1}{|c|}{Traitées} \\
    \hline
    Exercice {1} & 77\% \;{\small (31/40)} & 100\% \;{\small (4/4)} \\ \hline Exercice {2} & 38\% \;{\small (31/80)} & 75\% \;{\small (6/8)} \\ \hline Exercice {3} & 77\% \;{\small (27/35)} & 100\% \;{\small (5/5)} \\ \hline Exercice {4} & 24\% \;{\small (16/65)} & 100\% \;{\small (6/6)} \\ \hline \end{tabular} \\\\\pagebreak
\begin{tcolorbox}[enhanced,width=\textwidth,center upper,fontupper=\bfseries,drop shadow southwest,sharp corners]
{\sc \large Silotia} Donovan
\end{tcolorbox}
\medskip
\begin{tabularx}{\textwidth}{p{5cm}X}
	\alertbox{\faAward}{Note}{
		\begin{itemize}[leftmargin=0pt]
			\item[\textbullet] Note : \textbf{\large 11.6}
			\item[\textbullet] Rang : \textbf{8}
			\item[\textbullet] Traité : 78 \%
		\end{itemize}
	} &
	\alertbox{\faChartLine}{Statistiques des notes}{
		\begin{pspicture}(0,-0.1)(16,1.45)
			\psset{xunit=1,fillstyle=solid}
		   \savedata{\data}[11.7 11.2 18.9 8.4 8.4 10.9 15.7 5.8 8.0 13.5 17.8 16.7 6.0 18.6 9.5 11.6]
		   \rput{-90}(0,0.9){\psBoxplot[barwidth=1.1cm,yunit=0.5,fillcolor=gray,linewidth=1pt]{\data}}
		   \psaxes[yAxis=false,dx=1cm,Dx=2,labelsep=1pt,linecolor=gray,xlabelFontSize=\scriptstyle](0,0)(10.1,4)
		   \psdot[dotsize=8pt,dotstyle=diamond,linecolor=black,fillstyle=solid,fillcolor=white,linewidth=1pt](5.8,0.85)
           \psdot[dotsize=6pt,dotstyle=x,linecolor=black,linewidth=3pt](6.021875,0.85)
		   \end{pspicture}
	} \\
    
\end{tabularx}\\
\begin{tabularx}{\textwidth}{X}
\alertbox{\faComment}{Commentaire}
{
	Bon travail dans l’ensemble, il faut travailler la rapidité pour traiter plus de questions.
}
\end{tabularx}
\medskip
    \ding{113} \textbf{\sffamily{Résultats par thème}} \medskip \\
    \renewcommand{\arraystretch}{1.2}
    \begin{tabular}{|l|r|r|}
    \cline{2-3}
    \multicolumn{1}{l|}{} & \multicolumn{1}{|c|}{Points} & \multicolumn{1}{|c|}{Traitées} \\
    \hline
    {Discipline de programmation} & 70\% \;{\small (07/10)} & 100\% \;{\small (2/2)} \\ \hline {Programmation de base en C} & 46\% \;{\small (49/105)} & 72\% \;{\small (8/11)} \\ \hline {Preuve de correction} & 40\% \;{\small (22/55)} & 50\% \;{\small (2/4)} \\ \hline {Preuve de terminaison} & 100\% \;{\small (35/35)} & 100\% \;{\small (3/3)} \\ \hline {Comprendre un algorithme} & 100\% \;{\small (15/15)} & 100\% \;{\small (3/3)} \\ \hline \end{tabular} \\\\\medskip \\
    \ding{113} \textbf{\sffamily{Résultats par exercice}} \medskip \\
    \renewcommand{\arraystretch}{1.2}
    \begin{tabular}{|l|r|r|}
    \cline{2-3}
    \multicolumn{1}{l|}{} & \multicolumn{1}{|c|}{Points} & \multicolumn{1}{|c|}{Traitées} \\
    \hline
    Exercice {1} & 92\% \;{\small (37/40)} & 100\% \;{\small (4/4)} \\ \hline Exercice {2} & 63\% \;{\small (51/80)} & 87\% \;{\small (7/8)} \\ \hline Exercice {3} & 94\% \;{\small (33/35)} & 100\% \;{\small (5/5)} \\ \hline Exercice {4} & 10\% \;{\small (07/65)} & 33\% \;{\small (2/6)} \\ \hline \end{tabular} \\\\\pagebreak\end{document}
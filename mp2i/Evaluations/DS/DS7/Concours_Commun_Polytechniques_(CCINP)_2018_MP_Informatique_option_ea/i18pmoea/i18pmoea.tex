\documentclass[11pt,a4paper]{article}\nofiles
\usepackage[utf8]{inputenc}
\usepackage{amsfonts,amsmath,amssymb,enumitem}
\usepackage[francais]{babel}
\textheight245mm
\textwidth170mm
\hoffset-21mm
\voffset-15mm
\newcommand{\N}{\mathbb{N}}
\newcommand{\R}{\mathbb{R}}
\newcommand{\C}{\mathbb{C}}
\newcommand{\VV}{\mathbb{V}}
\newcommand{\EE}{\mathbb{E}}
\newcommand{\PP}{\mathbb{P}}
\parindent0pt
\begin{document}
\begin{center}
{\bf \Large CCP 2018 - Option informatique}
\end{center}

\section*{I. Logique et calcul des propositions}
Vous avez \'et\'e s\'electionn\'e(e) pour participer au jeu ``Cherchez les Cl\'es du Paradis (CCP)". 
Le jeu se d\'eroule en trois \'epreuves, au cours desquelles vous devez collecter des cl\'es vertes. 
A l'issue de chacune d'entre elles, vous passez \`a l'\'epreuve suivante en cas de succ\`es et \^etes \'elimin\'e(e) en cas d'\'echec.

\subsection*{I.1 Premi\`ere \'epreuve}
Jean-Pierre Pendule, le c\'el\`ebre animateur, vous accueille pour la premi\`ere \'epreuve. 
Il vous explique la r\`egle du jeu. 
Devant vous, deux bo\^\i tes et sur chacune d'entre elles une inscription. 
Chacune des bo\^\i tes contient soit une cl\'e verte, soit une cl\'e rouge. 
Vous devez choisir l'une des bo\^\i tes~: si le r\'esultat est une cl\'e rouge, alors vous quittez le jeu, si c'est une cl\'e verte vous \^etes qualifi\'e(e) pour l'\'epreuve suivante.

Jean-Pierre Pendule d\'evoile les inscriptions sur chacune des bo\^\i tes et vous affirme qu'elles sont soit vraies toutes les deux, soit fausses toutes les deux~:
\begin{itemize}
\item sur la bo\^\i te $1$, il est \'ecrit ``Une au moins des deux bo\^\i tes contient une cl\'e verte"~;
\item  sur la bo\^\i te $2$, il est \'ecrit ``Il y a une cl\'e rouge dans l'autre bo\^\i te".
\end{itemize}
Dans toute cette partie, on note $P_i$ la proposition affirmant qu'il y a une cl\'e verte dans la bo\^\i te $i$.
\begin{enumerate}
\item[{\bf Q.1}] Donner une formule de la logique des propositions repr\'esentant la phrase \'ecrite sur la bo\^\i te 1.
\item[{\bf Q.2}] Donner ude m\^eme ne formule de la logique des propositions pour l'inscription de la bo\^\i te 2.
\item[{\bf Q.3}] Donner une formule repr\'esentant l'affirmation de l'animateur. Simplifier cette formule de sorte \`a n'obtenir qu'une seule occurrence de chaque $P_i$.
\item[{\bf Q.4}] Quel choix devez-vous faire pour continuer le jeu \`a coup s\^ur~?
\end{enumerate}

\subsection*{I.2 Deuxi\`eme \'epreuve}
Bravo, vous avez obtenu la premi\`ere cl\'e verte. 
Jean-PierrePendule vous f\'elicit\'e et vous annonce que cette premi\`ere \'epreuve n'\'etait qu'une mise en jambe. 
Avec les m\^emes r\`egles du jeu, l'animateur vous propose alors deux nouvelles bo\^\i tes portant les inscriptions suivantes~:
\begin{itemize}
\item sur la bo\^\i te $1$, il est \'ecrit ``Il y a une cl\'e rouge dans cette bo\^\i te, ou bien il y a une cl\'e verte dans la bo\^\i te 2"~;
\item  sur la bo\^\i te $2$, il est \'ecrit ``Il y a une cl\'e verte dans la bo\^\i te 1".
\end{itemize}
\begin{enumerate}
\item[{\bf Q.5}] Donner une formule de la logique des propositions pour chaque affirmation.
\item[{\bf Q.6}] Sachant qu'encore une fois les deux affirmations sont soit vraies toutes les deux, soit fausses toutes les deux, donner le contenu de chaqu bo\^\i te. En d\'eduire votre choix pour remporter la deuxi\`eme cl\'e verte.
\end{enumerate}

\subsection*{I.3 Troisi\`eme \'epreuve}
Le suspens est \`a son comble, vous voici arriv\'e(e) \`a la derni\`ere \'epreuve. 
A votre grande surprise, Jean-Pierre Pendule vous d\'evoile une troisi\`eme bo\^\i te et vous explique les r\`egles du jeu. 
Dans une des bo\^\i tes se cache la cl\'e qui vous permet de remporter la victoire finale. 
Dans une autre bo\^\i te se cache une cl\'e rouge qui vous fait tout perdre. 
La derni\`ere bo\^\i te est vide. 
Encore une fois, chacune des bo\^\i tes porte une inscription~:
\begin{itemize}
\item sur la bo\^\i te $1$, il est \'ecrit ``La bo\^\i te 3 est vide"~;
\item  sur la bo\^\i te $2$, il est \'ecrit ``La cl\'e rouge est dans la bo\^\i te 1"~;
\item  sur la bo\^\i te $3$, il est \'ecrit ``Cette bo\^\i te est vide".
\end{itemize}
L'animateur affirme que l'inscription port\'ee sur la bo\^\i te contenant la cl\'e verte est vraie, celle port\'ee par la bo\^\i te contenant la cl\'e rouge est fausse. 
L'inscription affich\'ee sur la bo\^\i te vide est aussi vraie.
\begin{enumerate}
\item[{\bf Q.7}] Donner une formule de la logique des propositions pour chaque inscription.
\item[{\bf Q.8}] Donner une formule de la logique des propositions synth\'etisant l'information que vous apport\'ee l'animateur.
\item[{\bf Q.9}] En supposant que la cl\'e verte est dans la bo\^\i te 2, montrer par l'absurde que l'on aboutit \`a une incoh\'erence.
\item[{\bf Q.10}] Donner alors la composition des trois bo\^\i tes.
\end{enumerate}

\section*{II. Automates}
\textit{Le sujet pr\'esent\'e ici n'est pas exactement celui d'origine. Celui-ci comportait de nombreuses fautes d'\'enonc\'e et des notions \'etaient utilis\'ees sans \^etre d\'efinies. Les questions restent cependant les m\^emes (et elles me semblent souvent mal pos\'ees).}\\

\textbf{Dans ce probl\`eme, on travaillera exclusivement avec des automates d\'eterministes et complets que l'on appellera simplement automates.}\\

Un langage est r\'egulier si et seulement si il est accept\'e par un automate.
Cependant, plusieurs automates peuvent accepter le m\^eme langage. 
L'objectif de cette partie est de montrer que tout langage reconnaissable $L$ est reconnu par un unique (au renommage pr\`es des \'etats) automate, tel que tout automate reconnaissant $L$ a au moins autant d'\'etats que lui. 
Cet unique automate est appel\'e automate minimal reconnaissant $L$.

\subsection*{II.1 D\'efinitions}
Un \textbf{automate} $A$ est un quintuplet $A=(Q,\Sigma,q_0,F,\delta)$, avec~:
\begin{itemize}
\item $Q$ un ensemble fini non vide d'\'etats,
\item $\Sigma$ un alphabet (ensemble fini non vide),
\item $q_0$ est l'\'etat initial,
\item $F\subset Q$ un ensemble d'\'etats finaux,
\item $\delta : Q\times \Sigma \to Q$ une application de transition d\'efinie sur $Q\times \Sigma$ tout entier.
\end{itemize}
Il s'agit, comme \'evoqu\'e plus haut, de la notion d'automate d\'eterministe et complet.
\vskip3mm

Soit $\Sigma$ un alphabet. $\Sigma^*$ est l'ensemble des mots construits \`a partir de $\Sigma$, $\Lambda$ est le mot vide et $|u|_a$ est le nombre d'occurrences de la lettre $a\in~\Sigma$ dans le mot $u$.
\vskip3mm

Soit $L\subset \Sigma^*$ un langage et $u\in \Sigma^*$. Le \textbf{r\'esiduel \`a gauche de $L$ par rapport \`a $u$} est le langage
\[u^{-1}L=\{v\in \Sigma^*,\ uv\in L\}\]
\begin{enumerate}
\item[{\bf Q.11}] Pour $L=\{ab,ba,aab\}$, donner le r\'esiduel de $L$ par rapport \`a $a$.
\end{enumerate}
D\'efinissons alors une relation $\sim_L$ sur $\Sigma^*$, dite congruence de N\'erode, de la mani\`ere suivante~: pour tous $u,v\in \Sigma^*$,
\[u\sim_L v\iff u^{-1}L=v^{-1}L\]
\begin{enumerate}
\item[{\bf Q.12}] Montrer que $\sim_L$ est une relation d'\'equivalence et que $\forall u,v,w\in \Sigma^*,\ (u\sim_Lv)\Rightarrow (uw\sim_L vw)$.
\item[{\bf Q.13}] Posons $L=\{u\in \Sigma^*,\ |u|_a=0\ \textrm{mod }\ 3\}$ le langage bas\'e sur $\Sigma=\{a,b\}$, compos\'e des mots de $\Sigma^*$ ayant un nombre de $a$ multiple de $3$. Pour chacun des cas suivants, d\'eterminer si les deux mots sont \'equivalents par $\sim_L$~:
\begin{enumerate}
\item[(i)] $b$ et $ab$,
\item[(ii)] $aba$ et $bab$,
\item[(iii)] $abbaba$ et $aaa$.
\end{enumerate}
\end{enumerate}
Soit $A=(Q,\Sigma,q_0,F,\delta)$ un automate (au sens d\'efini plus haut). La fonction de transition $\delta^*$ \'etendue aux mots est d\'efinie de mani\`ere r\'ecursive par
\begin{itemize}
\item $\forall q\in Q,\ \delta^*(q,\Lambda)=q$
\item $\forall a\in \Sigma,\ \forall m\in \Sigma^*,\ \forall q\in Q,\ \delta^*(q,m.a)=\delta(\delta^*(q,m),a)$
\end{itemize}
Pour $q\in Q$ et $G\subset Q$, on note
\[q^{-1}G=\{u\in \Sigma^*,\ \delta^*(q,u)\in G\}\]
$q^{-1}G$ est donc l'ensemble des mots qui correspondent \`a des chemins d\'ebutant en $q$ et aboutissant dans un \'etat de $G$. En particulier, $q^{-1}F$ est l'ensemble menant de $q$ \`a un \'etat terminal. On dira que $q^{-1}F$ est l'ensemble des mots \textbf{accept\'e par l'automate depuis $q$}.\\

Une relation d'\'equivalence $\sim_A$ est \'egalement d\'efinie sur $Q$ par
\[p\sim_A q\iff p^{-1}F=q^{-1}F\]
L'automate $A$ accepte le langage $L=q_0^{-1}F$. Si $q\in Q$ et $u\in \Sigma^*$ sont tels que $\delta^*(q_0,u)=q$ alors on peut montrer que $q^{-1}F=u^{-1}L$. 
Ainsi, la congruence de N\'erode peut \^etre utilis\'ee pour d\'efinir un automate particulier appel\'e automate minimal de $L$.\\

Soit $L$ un langage. L'\textbf{automate minimal} de $L$ est d\'efini par le quintuplet $A_L=(Q_L,\Sigma,q_L,F_L,\delta_L)$ avec
\begin{itemize}
\item $Q_L=\{u^{-1}L,\ u\in \Sigma^*\}$
\item $q_L=\Lambda^{-1}L=L$
\item $F_L=\{u^{-1}L,\ u\in L\}=\{q\in Q_L,\ \Lambda \in q\}$
\item $\forall q\in Q_L,\ \forall a\in \Sigma,\ \delta_L(q,a)=a^{-1}q$
\end{itemize}

\begin{enumerate}
\item[{\bf Q.14}] Montrer que si $L$ est un langage r\'egulier alors $Q_L$ est fini.\\
Dans ce cas, l'automate minimal est alors bien un automate au sens de notre \'enonc\'e (ensemble fini d'\'etat, caract\`ere d\'eterministe et complet).
\end{enumerate}

Un automate $A=(Q,\Sigma,q_0,F,\delta)$ est \textbf{accessible} si pour tout $q\in Q$, il existe $u\in \Sigma^*$ tel que $\delta^*(q_0,u)=q$.\\

Un automate $A=(Q,\Sigma,q_0,F,\delta)$ est \textbf{r\'eduit} si pour tous $p,q\in Q$, $(p^{-1}F=q^{-1}F)\Rightarrow p=q$. $A$ est donc r\'eduit si les langages accept\'es depuis deux \'etats distincts sont distincts, ou encore si chaque classe d'\'equivalence pour la relation $\sim_A$ sur $Q$ est un singleton.\\

Pour $L\subset \Sigma^*$ rationnel, il est possible de montrer que l'automate minimal $A_L$ de $L$ est accessible et r\'eduit. Le paragraphe suivant s'int\'eresse \`a la construction de l'automate minimal du langage reconnu par un automate fini $A$ donn\'e en exploitant cette propri\'et\'e.

\subsection*{II.2 Construction de l'automate minimal}
Soit $A=(Q,\Sigma,q_0,F,\delta)$ un automate fini acceptant le langage $L\subset \Sigma^*$. 
Trouver l'automate minimal $A_L$ de $L$ revient \`a trouver un automate fini accessible et r\'eduit \'equivalent \`a $A$ (c'est \`a dire reconnaissant le m\^eme langage).\\

Pour touver un automate accessible, il suffit par exemple de visiter les \'etats qui peuvent \^etre atteints par $A$ depuis $q_0$ et d'\'eliminer les autres \'etats. 
Il reste donc \`a trouver une m\'ethode pour rendre $A$ r\'eduit. 
Par d\'efinition de la relation $\sim_A$ sur $Q$, $A$ est r\'eduit si pour tout couple $(p,q)\in Q^2$ avec $p\neq q$, $p\not \sim_A q$. 
Notons que $p\not \sim_A q$ a lieu s'il existe $u\in \Sigma^*$ tel que $\delta^*(p,u)\in F$ et $\delta^*(q,u)\notin F$ ou $\delta^*(q,v)\in F$ et $\delta^*(p,u)\notin F$. 
On dit alors que $u$ \textbf{distingue} $p$ et $q$ ou que le couple $(u,v)$ est \textbf{distingu\'e} par $u$.\\

L'algorithme qui suit est un algorithme de r\'eduction du langage associ\'e \`a $A$ utilisant ces notions. Dans la suite, $N_k$ d\'esigne l'ensemble des couples d'\'etats de $Q$ qui sont distingu\'es par un mot de longueur $k$ et qui ne sont distingu\'es par aucun autre mot strictement plus court.
\vskip3mm
\hrule
\vskip1mm
\textbf{Algorithme} : recherche des \'etats \'equivalents
\vskip1mm
\hrule

\textbf{Entr\'ee} : un automate d\'eterministe $A=(Q,\Sigma,q_0,F,\delta)$\\
\textbf{Sortie} : les ensembles d'\'etats \'equivalents\\

$k\leftarrow 0$\\
*********Initialisation************\\
$N_0\leftarrow \emptyset$\\
\textbf{pour} tous $p\in F$ et $q\in Q\setminus F$, \textbf{faire}\par
\hskip7mm ***La paire $(p,q)$ est distingu\'ee***\par
\hskip7mm $N_0\leftarrow N_0\cup\{(p,q)\}$.\\
\textbf{Tant que} $N_k\neq \emptyset$, \textbf{faire}\par
\hskip7mm ****Construction de $N_{k+1}$****\par
\hskip7mm $N_{k+1}\leftarrow \emptyset$\par
\hskip7mm \textbf{pour} chaque paire $(p,q)\in N_k$, \textbf{faire}\par
\hskip14mm \textbf{pour} chaque $a\in \Sigma$, \textbf{faire}\par
\hskip21mm \textbf{pour} chaque $(r,s)\in Q^2$ tel que $\delta(r,a)=p$ et $\delta(s,a)=q$, \textbf{faire}\par
\hskip28mm \textbf{si} $(r,s)\notin \bigcup_{i=0}^kN_i$, \textbf{alors}\par
\hskip35mm $N_{k+1}\leftarrow N_{k+1}\cup \{(r,s),(s,r)\}$\par
\hskip7mm $k\leftarrow k+1$
\vskip1mm
\hrule

\begin{enumerate}
\item[{\bf Q.15}] Montrer pourquoi, dans la phase d'initialisation, la paire $(p,q)$ est distingu\'ee.
\item[{\bf Q.16}] Montrer pourquoi, si $N_i=\emptyset$, alors $\forall j>i,\ N_j=\emptyset$.
\end{enumerate}
Soit alors $A=(Q,\Sigma,q_0,F,\delta)$ l'automate fini suivant
\begin{itemize}
\item $Q=\{1,2,3,4,5,6\}$
\item $\Sigma=\{a,b\}$
\item $q_0=\{4\}$
\item $F=\{2,5\}$
\item $\delta$ est repr\'esent\'e par la table $\begin{array}{c|c|c|c|c|c|c}&1&2&3&4&5&6\\\hline
a&2&4&6&5&1&6\\\hline
b&1&3&2&1&6&2\end{array}$.
\end{itemize}
\begin{enumerate} 
\item[{\bf Q.17}] Repr\'esenter graphiquement l'automate $A$. La repr\'esentation suivra les contraintes suivantes
\begin{itemize}
\item les \'etats sont des cercles, le nom de l'\'etat est \'ecrit \`a l'int\'erieur du cercle
\item les transitions sont repr\'esent\'ees par des fl\`eches partant de l'\'etat de d\'epart et pointant sur l'\'etat d'arriv\'ee~; le symbole d\'efinissant la transition est indiqu\'e au milieu de la fl\`eche
\item l'\'etat initial est signal\'e par une fl\`eche sans \'etiquette pointant sur cet \'etat
\item les \'etats finaux sont entour\'es d'un deuxi\`eme cercle, externe au premier
\end{itemize}
\item[{\bf Q.18}] Appliquer l'algorithme pour trouver l'ensemble des \'etats \'equivalents. 
Pour chaque it\'eration $k$, la trace de l'algorithme sera donn\'ee par une matrice $T$ de taille $|Q|\times |Q|$, avec $T(i,j)$ valant la longueur d'un chemin, s'il existe, qui distingue le couple $(i,j)$ ($T(i,j)$ vide sinon). 
En d\'eduire les classes d'\'equivalence des \'etats de $A$.
\end{enumerate}
Un th\'eor\`eme, non d\'etaill\'e ici, permet alors de projeter $A$ sur $A_L$ et de pr\'eciser \'etats et transitions de cet automate minimal. 
Il permet en particulier de d\'efinir les \'etats de $A_L$ comme \'etant les classes d'\'equivalence issues de l'algorithme pr\'ec\'edent. 
Il permet \'egalement d'affirmer que si un \'etat de $A_L$ correspond \`a une classe d'\'equivalence $[q]_A$ pour la relation $\sim_A$, alors la lecture d'un symbole $a\in \Sigma$ depuis cet \'etat dans $A_L$ conduit \`a l'\'etat correspondant \`a la classe $[\delta(q,a)]_A$.
\begin{enumerate}
\item[{\bf Q.19}] Repr\'esenter graphiquement l'automate minimal de la question pr\'ec\'edente, avec ses \'etats et ses transitions.
\end{enumerate}


\section*{III. Algorithmique et programmation}
Nous proposons dans cette partie d'\'etudier une m\'ethode de compression de donn\'ees. 
L'algorithme propos\'e ici impl\'emente plusieurs couches d'arrangement de donn\'ees et de compressions successives, utilis\'ees dans l'ordre suivant pour la compression et l'ordre inverse pour la d\'ecompression~:
\begin{itemize}
\item[(i)] Transformation de Burrows-Wheeler,
\item[(ii)] Codage par plages (RLE),
\item[(iii)] Codage de Huffman
\end{itemize}

Soit $\Sigma$ un alphabet de symboles, de cardinal $|\Sigma|=h$. 
On munit $\Sigma$ d'une relation d'ordre $\leq$. 
$\Sigma^k$ est l'ensemble des mots de longueur $k$ construits \`a partir de $\Sigma$. 
$\Sigma^k$ est muni de la relation d'ordre lexicographique induite par la relation d'ordre $\leq$.

Pour $\mu\in \Sigma^k$, on note $|\mu|=k$ la taille de $\mu$ et $|\mu|_a$ le nombre d'occurrences de $a\in \Sigma$ dans $\mu$.\\

Dans toute cette partie, lorsqu'il s'agira de coder une fonction \texttt{Ocaml}, un mot $\mu \in \Sigma^k$ sera repr\'esent\'e par une liste de caract\`eres (\texttt{char list}).\\

Les paragraphes suivants \'etudient les algorithmes et propri\'et\'es de ces phases, chacun d'entre eux pouvant \^etre abord\'e \textbf{de mani\`ere ind\'ependante}.

\subsection*{III.1 Transformation de Burrows-Wheeler (BWT)}
Soit $\mu\in \Sigma^k$ un mot. La transformation BWT r\'ealise une permutation des symboles de $\mu$ de sorte que les symboles identiques sont regroup\'es dans de longues s\'equences. 
Cette transformation n'effectue pas de compression, mais pr\'epare donc \`a une compression plus efficace.

Dans la suite, nous \'etudions le codage et le d\'ecodage d'un mot transform\'e par cette op\'eration.\\

\textbf{Phase de codage}\\
On rajoute \`a la fin de $\mu$ un marqueur de fin (par convention not\'e $|$, inf\'erieur par $\leq$ \`a tous les autres symboles de $\Sigma$). 
Dans toute la suite, $\hat{\mu}$ d\'esigne le mot auquel on a ajout\'e le symbole $|$. 
\begin{enumerate}
\item[{\bf Q.20}] Pour $\mu=turlututu$, construire une matrice $M$ dont les lignes sont les diff\'erentes permutations circulaires successives du mot $\hat{\mu}$. Les permutations seront ici envisag\'ees par d\'ecalage \`a droite des caract\`eres.
\item[{\bf Q.21}] Ecrire une fonction r\'ecursive \texttt{circulaire : 'a list $\to$ 'a list} qui r\'ealise une permutation \`a droite d'un mot $\mu$ donn\'e en entr\'ee.
\item[{\bf Q.22}] Ecrire une fonction \texttt{matrice\_mot : 'a list $\to$ 'a list list} qui construit la matrice $M$ \`a partir d'un mot pass\'e en entr\'ee. 
La valeur de retour est une liste de liste de symboles (une liste de mots). Cette fonction utilisera la fonction \texttt{circulaire}.
\end{enumerate}
Une permutation des lignes de $M$ est alors effectu\'ee, de sorte \`a classer les lignes par ordre lexicographique. On note $M'=P\cdot M$ la matrice obtenue, $P$ \'etant la matrice de permutation.
\begin{enumerate}
\item[{\bf Q.23}] Donner les matrices $P$ et $M'$ dans le cas du mot $\hat{\mu}$, pour $\mu=turlututu$.
\end{enumerate}
 Pour construire la matrice de permutation $P$, il faut trier la liste des mots d\'efinissant $M$. La m\'ethode de tri choisie ici est le tri par insertion.
\begin{enumerate}
\item[{\bf Q.24}] Ecrire une fonction r\'ecursive \texttt{tri : 'a list $\to$ 'a list} qui r\'ealise le tri par insertion d'une liste d'\'el\'ements.
\item[{\bf Q.25}] En d\'eduire une fonction \texttt{matrice\_mot\_tri\'ee : 'a list $\to$ 'a list list} qui construit $M'$ \`a partir de $M$.
\item[{\bf Q.26}] Pour $\mu\in \Sigma^k$, donner le nombre de comparaisons de symboles n\'ecessaires au pire des cas, pour trier deux permutations circulaires du mot $\mu$.
\item[{\bf Q.27}] En d\'eduire la complexit\'e dans le pire des cas pour le tri des $k$ permutations circulaires d'un mot $\mu\in \Sigma^k$ (exprim\'ee en nombre de comparaisons de symboles).
\end{enumerate}
La transformation BWT consiste alors \`a coder le mot $\mu$ par la derni\`ere colonne de la matrice $M'$ obtenue \`a partir de $\hat{\mu}$. On note $BWT(\mu)$ ce codage.
\begin{enumerate} 
\item[{\bf Q.28}] Ecrire une fonction \texttt{codageBWT : char list $\to$ char list} qui encode un mot pass\'e en entr\'ee. On utilisera une fonction r\'ecursive pemettant de r\'ecup\'erer le dernier symbole de chacun des mots de $M'$. Donner le codage du mot $\mu=turlututu$.
\end{enumerate}

\textbf{Phase de d\'ecodage}\\
Pour d\'ecoder un mot cod\'e par BWT, il est n\'ecessaire de reconstruire it\'erativement $M'$ \`a partir de la seule donn\'ee du mot code $BWT(\mu)$. 
Par construction, $BWT(\mu)$ est la derni\`ere colonne de $M'$.

On pose ici comme exemple $BWT(\mu)=edngvnea|$.
\begin{enumerate}
\item[{\bf Q.29}] Construire, \`a partir de la seule donn\'ee de $BWT(\mu)$, la premi\`ere colonne de $M'$. Justifier le principe de construction.
\end{enumerate}
La derni\`ere et la premi\`ere colonne de $M'$ donnent alors tous les sous-mots de longueur $2$ du mot $\mu$.
\begin{enumerate}
\item[{\bf Q.30}] Proposer un algorithme permettant d'obtenir la deuxi\`eme colonne de $M'$. Donner cette deuxi\`eme colonne pour $BWT(\mu)=edngvnea|$.
\item[{\bf Q.31}] On dispose \`a l'it\'eration $n$ des $(n-1)$ premi\`eres colonnes de $M'$ et de sa derni\`ere colonne. Proposer un principe algorithmique permettant de construire la $n$-i\`eme colonne de $M'$.
\item[{\bf Q.32}] En d\'eduire un algorithme it\'eratif permettant de reconstruire $M'$.
\item[{\bf Q.33}] Quel d\'ecodage obtient-on pour le mot $BWT(\mu)$ propos\'e~?
\end{enumerate}

\subsection*{III.2 Codage par plages RLE (Informatique pour tous)}
Le codage RLE (Run Length Coding), ou codage par plages, est une m\'ethode de compression dont le principe est de remplacer dans une cha\^\i ne de symboles une sous-cha\^\i ne de symboles identiques par le couple constitu\'e du nombre de symboles identiques et du symbole lui m\^eme. 
Par exemple, la cha\^\i ne "aaababb" est compress\'ee en $[(3,`a`),(1,`b`),(1,`a`),(2,`b`)]$.
\begin{enumerate}
\item[{\bf Q.34}] Proposer un type naturel Python pour la compression RLE, qui permet de repr\'esenter le r\'esultat comme indiqu\'e pr\'ec\'edemment.
\end{enumerate}

\textbf{Phase de codage}\\
On s'int\'eresse tout d'abord au codage RLE d'un mot.
\begin{enumerate}
\item[{\bf Q.35}] Ecrire une fonction it\'erative en Python \texttt{def RLE(mot):} qui code un mot pass\'e en entr\'ee par codage RLE.
\end{enumerate}
\textbf{Phase de codage}\\
On s'int\'eresse maintenant au d\'ecodage d'une liste.
\begin{enumerate}
\item[{\bf Q.36}] Ecrire une fonction it\'erative en Python \texttt{def decodeRLE(codeRLE):} qui d\'ecode une liste \texttt{codeRLE} issue du codage RLE d'un mot.
\end{enumerate}

\subsection*{III.3 Codage de Huffman (Informatique pour tous)}
\textit{Le texte qui suit n'est pas tout \`a fait celui d'origine car des notions n'\'etaient soit pas d\'efinies soit inutiles dans le cadre du sujet.}\\

La derni\`ere \'etape de l'algorithme propos\'e impl\'emente le codage de Huffman, qui utilise la structure d'arbre binaire. 
Le principe est de coder un symbole de mani\`ere d'autant plus courte que son nombre d'occurrences est \'elev\'e. 
Les arbres consid\'er\'es sont \'etiquet\'es. Plus pr\'ecis\'ement, un noeud est soit une feuille soit un noeud interne \`a deux fils.
\begin{itemize}
\item Chaque feuille est \'etiquet\'ee par un couple form\'e d'un caract\`ere et d'un entier naturel. Ce dernier s'appelle le poids de la feuille (du noeud \'egal \`a cette feuille).
\item Chaque noeud interne est \'etiquet\'e par un entier naturel qu'on appelle aussi le poids du noeud.
\end{itemize}
Si $t$ est un noeud, on note $n_t$ son poids. Ce poids se confond avec l'\'etiquette du noeud dans le cs d'un noeud interne.\\
L'arbre de Huffman se construit \`a l'aide de l'algorithme suivant.\\

\hrule \vskip1mm
\textbf{Algorithme} : Codage de Huffman
\vskip1mm
\hrule
\vskip1mm
\textbf{Entr\'ee} : $\mu$ un mot de taille $|\mu|$.\\
\textbf{Sortie} : \texttt{Huffman}($\mu$) le codage de Huffman de $\mu$.\\
\textbf{pour} $a\in \Sigma$, \textbf{faire}\par
\hspace{7mm} \textbf{si} $|\mu|_a>0$ \textbf{alors}\par
\hspace{14mm} cr\'eer une feuille d'\'etiquette $(a,|\mu|_a)$.\\
$\mathcal{L} \leftarrow$ liste des noeuds cr\'e\'es dans l'ordre croissant des poids\\
$\mathcal{A} \leftarrow$ liste vide\\
\textbf{tant que} $(longueur (\mathcal{L})+longueur (\mathcal{A})>1)$ \textbf{faire}\par
\hspace{7mm} \begin{minipage}{12cm}$(g,d)\leftarrow$ deux noeuds de plus faible poids parmi les 2 premiers noeuds de $\mathcal{L}$ et les 2 premiers noeuds de $\mathcal{A}$\end{minipage}\par
\hspace{7mm} \begin{minipage}{12cm} Cr\'eer un noeud $t$ d'\'etiquette $n_t=n_g+n_d$\ et tel que $gauche(t)=g$ et $droite(t)= d$\end{minipage}\par
%\hspace{7mm} Coder la branche de $t$ \`a $g$ par $0$\par
%\hspace{7mm} Coder la branche de $t$ \`a $d$ par $1$\par
\hspace{7mm} Ins\'erer $t$ \`a la fin de $\mathcal{A}$\par
\hspace{7mm} Retirer $g$ et $d$ de $\mathcal{L}$ ou $\mathcal{A}$ (selon leur provenance)\\
\texttt{Huffman}($\mu$) $ \leftarrow \mathcal{A}$.\\
\vskip1mm
\hrule
\begin{enumerate}
\item[{\bf Q.37}] Construire le codage de Huffman du mot $\mu=turlututu$ en utilisant l'algorithme $2$. Vous expliciterez par des dessins d'arbres chacune des \'etapes de construction de \texttt{Huffman}($\mu$).
\item[{\bf Q.38}] Quelle est la forme de l'arbre de Huffman dans un mot o\`u tous les symboles ont le m\^eme nombre d'occurrences~?
\end{enumerate}
\end{document}
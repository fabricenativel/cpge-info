
\documentclass[11pt,a4paper]{article}
\usepackage{Act}
\begin{document}
\input{\detokenize{/home/fenarius/Travail/Cours/cpge-info/latex/Macros.tex}}
\RDS{2}{08/11/2025}

\begin{tcolorbox}[enhanced,width=\textwidth,center upper,fontupper=\bfseries,drop shadow southwest,sharp corners]
{\sc \large Astruc} Alexandre
\end{tcolorbox}
\medskip
\begin{tabularx}{\textwidth}{p{5cm}X}
	\alertbox{\faAward}{Note}{
		\begin{itemize}[leftmargin=0pt]
			\item[\textbullet] Note : \textbf{\large 11.9}
			\item[\textbullet] Rang : \textbf{9}
			\item[\textbullet] Traité : 100 \%
		\end{itemize}
	} &
	\alertbox{\faChartLine}{Statistiques des notes}{
        \psset{xunit=1cm, yunit=1cm,fillstyle=solid}
		\begin{pspicture}(0,-0.1)(16,1.45)
		   \savedata{\data}[11.9 15.4 18.4 11.8 9.0 11.1 16.1 4.6 8.2 15.0 18.0 15.5 17.5 10.0 15.4]
		   \rput{-90}(0,0.9){\psBoxplot[barwidth=1.1cm,yunit=0.5,fillcolor=gray,linewidth=1pt]{\data}}
		   \psaxes[yAxis=false,dx=1cm,Dx=2,labelsep=1pt,linecolor=gray,xlabelFontSize=\scriptstyle](0,0)(10.1,4)
		   \psdot[dotsize=8pt,dotstyle=diamond,linecolor=black,fillstyle=solid,fillcolor=white,linewidth=1pt](5.95,0.85)
           \psdot[dotsize=6pt,dotstyle=x,linecolor=black,linewidth=3pt](6.596666666666667,0.85)
		   \end{pspicture}
	} \\
    
\end{tabularx}\\
\begin{tabularx}{\textwidth}{X}
\alertbox{\faComment}{Commentaire}
{
	Il faut impérativement faire des efforts pour l’écriture et la présentation. Tu risques de perdre des points précieux aux concours sinon.
Le mécanisme permettant de renvoyer « un tableau » en C n’est pas compris. On doit allouer l’espace sur le tas avec malloc et renvoyer un pointeur. Revois le cours et les exercices sur le modèle mémoire du langage C.
}
\end{tabularx}
\medskip
    \ding{113} \textbf{\sffamily{Résultats par thème}} \medskip \\
    \renewcommand{\arraystretch}{1.2}
    \begin{tabular}{|l|r|r|}
    \cline{2-3}
    \multicolumn{1}{l|}{} & \multicolumn{1}{|c|}{Points} & \multicolumn{1}{|c|}{Traitées} \\
    \hline
    {Preuve de correction} & 100\% \;{\small (15/15)} & 100\% \;{\small (1/1)} \\ \hline {Types structurés en C et pointeurs} & 40\% \;{\small (06/15)} & 100\% \;{\small (1/1)} \\ \hline {Comprendre un algorithme} & 100\% \;{\small (05/5)} & 100\% \;{\small (1/1)} \\ \hline {Programmation de base en C} & 41\% \;{\small (29/70)} & 100\% \;{\small (8/8)} \\ \hline {Représentation des entiers} & 46\% \;{\small (23/50)} & 100\% \;{\small (8/8)} \\ \hline {Base d'OCaml (fonctionnel)} & 74\% \;{\small (63/85)} & 100\% \;{\small (11/11)} \\ \hline {Preuve de terminaison} & 100\% \;{\small (05/5)} & 100\% \;{\small (1/1)} \\ \hline \end{tabular} \\\\\medskip \\
    \ding{113} \textbf{\sffamily{Résultats par exercice}} \medskip \\
    \renewcommand{\arraystretch}{1.2}
    \begin{tabular}{|l|r|r|}
    \cline{2-3}
    \multicolumn{1}{l|}{} & \multicolumn{1}{|c|}{Points} & \multicolumn{1}{|c|}{Traitées} \\
    \hline
    Exercice {1} & 75\% \;{\small (30/40)} & 100\% \;{\small (5/5)} \\ \hline Exercice {2} & 81\% \;{\small (61/75)} & 100\% \;{\small (10/10)} \\ \hline Exercice {3} & 48\% \;{\small (24/50)} & 100\% \;{\small (5/5)} \\ \hline Exercice {4} & 38\% \;{\small (31/80)} & 100\% \;{\small (11/11)} \\ \hline \end{tabular} \\\\
    \vspace{0.5cm}\\
    \ding{113} \textbf{\sffamily{Historique des notes}} \medskip \\
    \psset{xunit=1.4cm, yunit=0.2cm}
    \begin{pspicture}(-1,-1)(12,22)

% --- Axe X : numéros de devoir ---
\psaxes[
    Dx=1,
    Dy=2,
    Ox=0,
    Oy=0,
    labels=all,
    ticks=all,
](0,0)(0,0)(7,20)


\listplot[plotstyle=line,showpoints=true,linecolor=black,linewidth=0.7pt, dotstyle=diamond*, dotsize=0.2]{%
    1 11.7
2 11.9
}

% Minimum de la classe
\listplot[plotstyle=line,showpoints=true,linecolor=gray,linewidth=0.7pt, dotstyle=|, dotangle=90, dotsize=0.15, linestyle=dotted]{%
    1 5.8
2 4.6
}

% Maximum de la classe
\listplot[plotstyle=line,showpoints=true,linecolor=gray,linewidth=0.7pt, dotstyle=|, dotangle=90, dotsize=0.15, linestyle=dotted]{%
    1 18.9
2 18.4
}

% Moyenne de la classe
\listplot[plotstyle=line,showpoints=true,linecolor=gray,linewidth=0.7pt, dotstyle=x, dotsize = 0.15, linestyle=dashed]{%
    1 12.4466666666667
2 13.1933333333333
}
\end{pspicture}
\pagebreak
\begin{tcolorbox}[enhanced,width=\textwidth,center upper,fontupper=\bfseries,drop shadow southwest,sharp corners]
{\sc \large Berfeuil} Rohan
\end{tcolorbox}
\medskip
\begin{tabularx}{\textwidth}{p{5cm}X}
	\alertbox{\faAward}{Note}{
		\begin{itemize}[leftmargin=0pt]
			\item[\textbullet] Note : \textbf{\large 15.4}
			\item[\textbullet] Rang : \textbf{6}
			\item[\textbullet] Traité : 87 \%
		\end{itemize}
	} &
	\alertbox{\faChartLine}{Statistiques des notes}{
        \psset{xunit=1cm, yunit=1cm,fillstyle=solid}
		\begin{pspicture}(0,-0.1)(16,1.45)
		   \savedata{\data}[11.9 15.4 18.4 11.8 9.0 11.1 16.1 4.6 8.2 15.0 18.0 15.5 17.5 10.0 15.4]
		   \rput{-90}(0,0.9){\psBoxplot[barwidth=1.1cm,yunit=0.5,fillcolor=gray,linewidth=1pt]{\data}}
		   \psaxes[yAxis=false,dx=1cm,Dx=2,labelsep=1pt,linecolor=gray,xlabelFontSize=\scriptstyle](0,0)(10.1,4)
		   \psdot[dotsize=8pt,dotstyle=diamond,linecolor=black,fillstyle=solid,fillcolor=white,linewidth=1pt](7.7,0.85)
           \psdot[dotsize=6pt,dotstyle=x,linecolor=black,linewidth=3pt](6.596666666666667,0.85)
		   \end{pspicture}
	} \\
    
\end{tabularx}\\
\begin{tabularx}{\textwidth}{X}
\alertbox{\faComment}{Commentaire}
{
	Très bon devoir, c’est bien il faut continuer ainsi !
}
\end{tabularx}
\medskip
    \ding{113} \textbf{\sffamily{Résultats par thème}} \medskip \\
    \renewcommand{\arraystretch}{1.2}
    \begin{tabular}{|l|r|r|}
    \cline{2-3}
    \multicolumn{1}{l|}{} & \multicolumn{1}{|c|}{Points} & \multicolumn{1}{|c|}{Traitées} \\
    \hline
    {Preuve de correction} & 100\% \;{\small (15/15)} & 100\% \;{\small (1/1)} \\ \hline {Types structurés en C et pointeurs} & 0\% \;{\small (00/15)} & 0\% \;{\small (0/1)} \\ \hline {Comprendre un algorithme} & 100\% \;{\small (05/5)} & 100\% \;{\small (1/1)} \\ \hline {Programmation de base en C} & 80\% \;{\small (56/70)} & 87\% \;{\small (7/8)} \\ \hline {Représentation des entiers} & 57\% \;{\small (29/50)} & 75\% \;{\small (6/8)} \\ \hline {Base d'OCaml (fonctionnel)} & 92\% \;{\small (79/85)} & 100\% \;{\small (11/11)} \\ \hline {Preuve de terminaison} & 100\% \;{\small (05/5)} & 100\% \;{\small (1/1)} \\ \hline \end{tabular} \\\\\medskip \\
    \ding{113} \textbf{\sffamily{Résultats par exercice}} \medskip \\
    \renewcommand{\arraystretch}{1.2}
    \begin{tabular}{|l|r|r|}
    \cline{2-3}
    \multicolumn{1}{l|}{} & \multicolumn{1}{|c|}{Points} & \multicolumn{1}{|c|}{Traitées} \\
    \hline
    Exercice {1} & 85\% \;{\small (34/40)} & 100\% \;{\small (5/5)} \\ \hline Exercice {2} & 100\% \;{\small (75/75)} & 100\% \;{\small (10/10)} \\ \hline Exercice {3} & 70\% \;{\small (35/50)} & 80\% \;{\small (4/5)} \\ \hline Exercice {4} & 56\% \;{\small (45/80)} & 72\% \;{\small (8/11)} \\ \hline \end{tabular} \\\\
    \vspace{0.5cm}\\
    \ding{113} \textbf{\sffamily{Historique des notes}} \medskip \\
    \psset{xunit=1.4cm, yunit=0.2cm}
    \begin{pspicture}(-1,-1)(12,22)

% --- Axe X : numéros de devoir ---
\psaxes[
    Dx=1,
    Dy=2,
    Ox=0,
    Oy=0,
    labels=all,
    ticks=all,
](0,0)(0,0)(7,20)


\listplot[plotstyle=line,showpoints=true,linecolor=black,linewidth=0.7pt, dotstyle=diamond*, dotsize=0.2]{%
    1 11.2
2 15.4
}

% Minimum de la classe
\listplot[plotstyle=line,showpoints=true,linecolor=gray,linewidth=0.7pt, dotstyle=|, dotangle=90, dotsize=0.15, linestyle=dotted]{%
    1 5.8
2 4.6
}

% Maximum de la classe
\listplot[plotstyle=line,showpoints=true,linecolor=gray,linewidth=0.7pt, dotstyle=|, dotangle=90, dotsize=0.15, linestyle=dotted]{%
    1 18.9
2 18.4
}

% Moyenne de la classe
\listplot[plotstyle=line,showpoints=true,linecolor=gray,linewidth=0.7pt, dotstyle=x, dotsize = 0.15, linestyle=dashed]{%
    1 12.4466666666667
2 13.1933333333333
}
\end{pspicture}
\pagebreak
\begin{tcolorbox}[enhanced,width=\textwidth,center upper,fontupper=\bfseries,drop shadow southwest,sharp corners]
{\sc \large Body} Timothée
\end{tcolorbox}
\medskip
\begin{tabularx}{\textwidth}{p{5cm}X}
	\alertbox{\faAward}{Note}{
		\begin{itemize}[leftmargin=0pt]
			\item[\textbullet] Note : \textbf{\large 18.4}
			\item[\textbullet] Rang : \textbf{1}
			\item[\textbullet] Traité : 100 \%
		\end{itemize}
	} &
	\alertbox{\faChartLine}{Statistiques des notes}{
        \psset{xunit=1cm, yunit=1cm,fillstyle=solid}
		\begin{pspicture}(0,-0.1)(16,1.45)
		   \savedata{\data}[11.9 15.4 18.4 11.8 9.0 11.1 16.1 4.6 8.2 15.0 18.0 15.5 17.5 10.0 15.4]
		   \rput{-90}(0,0.9){\psBoxplot[barwidth=1.1cm,yunit=0.5,fillcolor=gray,linewidth=1pt]{\data}}
		   \psaxes[yAxis=false,dx=1cm,Dx=2,labelsep=1pt,linecolor=gray,xlabelFontSize=\scriptstyle](0,0)(10.1,4)
		   \psdot[dotsize=8pt,dotstyle=diamond,linecolor=black,fillstyle=solid,fillcolor=white,linewidth=1pt](9.2,0.85)
           \psdot[dotsize=6pt,dotstyle=x,linecolor=black,linewidth=3pt](6.596666666666667,0.85)
		   \end{pspicture}
	} \\
    
\end{tabularx}\\
\begin{tabularx}{\textwidth}{X}
\alertbox{\faComment}{Commentaire}
{
	Excellent travail, continue sur cette lancée. 
}
\end{tabularx}
\medskip
    \ding{113} \textbf{\sffamily{Résultats par thème}} \medskip \\
    \renewcommand{\arraystretch}{1.2}
    \begin{tabular}{|l|r|r|}
    \cline{2-3}
    \multicolumn{1}{l|}{} & \multicolumn{1}{|c|}{Points} & \multicolumn{1}{|c|}{Traitées} \\
    \hline
    {Preuve de correction} & 100\% \;{\small (15/15)} & 100\% \;{\small (1/1)} \\ \hline {Types structurés en C et pointeurs} & 100\% \;{\small (15/15)} & 100\% \;{\small (1/1)} \\ \hline {Comprendre un algorithme} & 100\% \;{\small (05/5)} & 100\% \;{\small (1/1)} \\ \hline {Programmation de base en C} & 85\% \;{\small (60/70)} & 100\% \;{\small (8/8)} \\ \hline {Représentation des entiers} & 92\% \;{\small (46/50)} & 100\% \;{\small (8/8)} \\ \hline {Base d'OCaml (fonctionnel)} & 94\% \;{\small (80/85)} & 100\% \;{\small (11/11)} \\ \hline {Preuve de terminaison} & 100\% \;{\small (05/5)} & 100\% \;{\small (1/1)} \\ \hline \end{tabular} \\\\\medskip \\
    \ding{113} \textbf{\sffamily{Résultats par exercice}} \medskip \\
    \renewcommand{\arraystretch}{1.2}
    \begin{tabular}{|l|r|r|}
    \cline{2-3}
    \multicolumn{1}{l|}{} & \multicolumn{1}{|c|}{Points} & \multicolumn{1}{|c|}{Traitées} \\
    \hline
    Exercice {1} & 100\% \;{\small (40/40)} & 100\% \;{\small (5/5)} \\ \hline Exercice {2} & 93\% \;{\small (70/75)} & 100\% \;{\small (10/10)} \\ \hline Exercice {3} & 100\% \;{\small (50/50)} & 100\% \;{\small (5/5)} \\ \hline Exercice {4} & 82\% \;{\small (66/80)} & 100\% \;{\small (11/11)} \\ \hline \end{tabular} \\\\
    \vspace{0.5cm}\\
    \ding{113} \textbf{\sffamily{Historique des notes}} \medskip \\
    \psset{xunit=1.4cm, yunit=0.2cm}
    \begin{pspicture}(-1,-1)(12,22)

% --- Axe X : numéros de devoir ---
\psaxes[
    Dx=1,
    Dy=2,
    Ox=0,
    Oy=0,
    labels=all,
    ticks=all,
](0,0)(0,0)(7,20)


\listplot[plotstyle=line,showpoints=true,linecolor=black,linewidth=0.7pt, dotstyle=diamond*, dotsize=0.2]{%
    1 18.9
2 18.4
}

% Minimum de la classe
\listplot[plotstyle=line,showpoints=true,linecolor=gray,linewidth=0.7pt, dotstyle=|, dotangle=90, dotsize=0.15, linestyle=dotted]{%
    1 5.8
2 4.6
}

% Maximum de la classe
\listplot[plotstyle=line,showpoints=true,linecolor=gray,linewidth=0.7pt, dotstyle=|, dotangle=90, dotsize=0.15, linestyle=dotted]{%
    1 18.9
2 18.4
}

% Moyenne de la classe
\listplot[plotstyle=line,showpoints=true,linecolor=gray,linewidth=0.7pt, dotstyle=x, dotsize = 0.15, linestyle=dashed]{%
    1 12.4466666666667
2 13.1933333333333
}
\end{pspicture}
\pagebreak
\begin{tcolorbox}[enhanced,width=\textwidth,center upper,fontupper=\bfseries,drop shadow southwest,sharp corners]
{\sc \large Boucher} Mathis
\end{tcolorbox}
\medskip
\begin{tabularx}{\textwidth}{p{5cm}X}
	\alertbox{\faAward}{Note}{
		\begin{itemize}[leftmargin=0pt]
			\item[\textbullet] Note : \textbf{\large 11.8}
			\item[\textbullet] Rang : \textbf{10}
			\item[\textbullet] Traité : 84 \%
		\end{itemize}
	} &
	\alertbox{\faChartLine}{Statistiques des notes}{
        \psset{xunit=1cm, yunit=1cm,fillstyle=solid}
		\begin{pspicture}(0,-0.1)(16,1.45)
		   \savedata{\data}[11.9 15.4 18.4 11.8 9.0 11.1 16.1 4.6 8.2 15.0 18.0 15.5 17.5 10.0 15.4]
		   \rput{-90}(0,0.9){\psBoxplot[barwidth=1.1cm,yunit=0.5,fillcolor=gray,linewidth=1pt]{\data}}
		   \psaxes[yAxis=false,dx=1cm,Dx=2,labelsep=1pt,linecolor=gray,xlabelFontSize=\scriptstyle](0,0)(10.1,4)
		   \psdot[dotsize=8pt,dotstyle=diamond,linecolor=black,fillstyle=solid,fillcolor=white,linewidth=1pt](5.9,0.85)
           \psdot[dotsize=6pt,dotstyle=x,linecolor=black,linewidth=3pt](6.596666666666667,0.85)
		   \end{pspicture}
	} \\
    
\end{tabularx}\\
\begin{tabularx}{\textwidth}{X}
\alertbox{\faComment}{Commentaire}
{
	L’ensemble est correct, pour la preuve de correction attention à bien spécifier qu’on utiliser l’invariant à la sortie de boucle. Dans le modèle mémoire du C une variable locale occupe un espace dans la pile.
}
\end{tabularx}
\medskip
    \ding{113} \textbf{\sffamily{Résultats par thème}} \medskip \\
    \renewcommand{\arraystretch}{1.2}
    \begin{tabular}{|l|r|r|}
    \cline{2-3}
    \multicolumn{1}{l|}{} & \multicolumn{1}{|c|}{Points} & \multicolumn{1}{|c|}{Traitées} \\
    \hline
    {Preuve de correction} & 80\% \;{\small (12/15)} & 100\% \;{\small (1/1)} \\ \hline {Types structurés en C et pointeurs} & 0\% \;{\small (00/15)} & 100\% \;{\small (1/1)} \\ \hline {Comprendre un algorithme} & 100\% \;{\small (05/5)} & 100\% \;{\small (1/1)} \\ \hline {Programmation de base en C} & 37\% \;{\small (26/70)} & 62\% \;{\small (5/8)} \\ \hline {Représentation des entiers} & 42\% \;{\small (21/50)} & 75\% \;{\small (6/8)} \\ \hline {Base d'OCaml (fonctionnel)} & 89\% \;{\small (76/85)} & 100\% \;{\small (11/11)} \\ \hline {Preuve de terminaison} & 100\% \;{\small (05/5)} & 100\% \;{\small (1/1)} \\ \hline \end{tabular} \\\\\medskip \\
    \ding{113} \textbf{\sffamily{Résultats par exercice}} \medskip \\
    \renewcommand{\arraystretch}{1.2}
    \begin{tabular}{|l|r|r|}
    \cline{2-3}
    \multicolumn{1}{l|}{} & \multicolumn{1}{|c|}{Points} & \multicolumn{1}{|c|}{Traitées} \\
    \hline
    Exercice {1} & 72\% \;{\small (29/40)} & 100\% \;{\small (5/5)} \\ \hline Exercice {2} & 98\% \;{\small (74/75)} & 100\% \;{\small (10/10)} \\ \hline Exercice {3} & 42\% \;{\small (21/50)} & 100\% \;{\small (5/5)} \\ \hline Exercice {4} & 26\% \;{\small (21/80)} & 54\% \;{\small (6/11)} \\ \hline \end{tabular} \\\\
    \vspace{0.5cm}\\
    \ding{113} \textbf{\sffamily{Historique des notes}} \medskip \\
    \psset{xunit=1.4cm, yunit=0.2cm}
    \begin{pspicture}(-1,-1)(12,22)

% --- Axe X : numéros de devoir ---
\psaxes[
    Dx=1,
    Dy=2,
    Ox=0,
    Oy=0,
    labels=all,
    ticks=all,
](0,0)(0,0)(7,20)


\listplot[plotstyle=line,showpoints=true,linecolor=black,linewidth=0.7pt, dotstyle=diamond*, dotsize=0.2]{%
    1 8.4
2 11.8
}

% Minimum de la classe
\listplot[plotstyle=line,showpoints=true,linecolor=gray,linewidth=0.7pt, dotstyle=|, dotangle=90, dotsize=0.15, linestyle=dotted]{%
    1 5.8
2 4.6
}

% Maximum de la classe
\listplot[plotstyle=line,showpoints=true,linecolor=gray,linewidth=0.7pt, dotstyle=|, dotangle=90, dotsize=0.15, linestyle=dotted]{%
    1 18.9
2 18.4
}

% Moyenne de la classe
\listplot[plotstyle=line,showpoints=true,linecolor=gray,linewidth=0.7pt, dotstyle=x, dotsize = 0.15, linestyle=dashed]{%
    1 12.4466666666667
2 13.1933333333333
}
\end{pspicture}
\pagebreak
\begin{tcolorbox}[enhanced,width=\textwidth,center upper,fontupper=\bfseries,drop shadow southwest,sharp corners]
{\sc \large Chane-lock} Maxime
\end{tcolorbox}
\medskip
\begin{tabularx}{\textwidth}{p{5cm}X}
	\alertbox{\faAward}{Note}{
		\begin{itemize}[leftmargin=0pt]
			\item[\textbullet] Note : \textbf{\large 9.0}
			\item[\textbullet] Rang : \textbf{13}
			\item[\textbullet] Traité : 65 \%
		\end{itemize}
	} &
	\alertbox{\faChartLine}{Statistiques des notes}{
        \psset{xunit=1cm, yunit=1cm,fillstyle=solid}
		\begin{pspicture}(0,-0.1)(16,1.45)
		   \savedata{\data}[11.9 15.4 18.4 11.8 9.0 11.1 16.1 4.6 8.2 15.0 18.0 15.5 17.5 10.0 15.4]
		   \rput{-90}(0,0.9){\psBoxplot[barwidth=1.1cm,yunit=0.5,fillcolor=gray,linewidth=1pt]{\data}}
		   \psaxes[yAxis=false,dx=1cm,Dx=2,labelsep=1pt,linecolor=gray,xlabelFontSize=\scriptstyle](0,0)(10.1,4)
		   \psdot[dotsize=8pt,dotstyle=diamond,linecolor=black,fillstyle=solid,fillcolor=white,linewidth=1pt](4.5,0.85)
           \psdot[dotsize=6pt,dotstyle=x,linecolor=black,linewidth=3pt](6.596666666666667,0.85)
		   \end{pspicture}
	} \\
    
\end{tabularx}\\
\begin{tabularx}{\textwidth}{X}
\alertbox{\faComment}{Commentaire}
{
	Dans la preuve de correction, les cas pair et impair ne sont pas correctement distingués, de plus c’est l’utilisation de l’invariant en sortie qui prouve la correction. En Ocaml tu confonds les types char et str.
Tu sembles avoir compris le problème du renvoie d’adresses de variables locales mais le malloc est à revoir.
Il faut essayer de gagner en rapidité pour traiter plus de questions
}
\end{tabularx}
\medskip
    \ding{113} \textbf{\sffamily{Résultats par thème}} \medskip \\
    \renewcommand{\arraystretch}{1.2}
    \begin{tabular}{|l|r|r|}
    \cline{2-3}
    \multicolumn{1}{l|}{} & \multicolumn{1}{|c|}{Points} & \multicolumn{1}{|c|}{Traitées} \\
    \hline
    {Preuve de correction} & 60\% \;{\small (09/15)} & 100\% \;{\small (1/1)} \\ \hline {Types structurés en C et pointeurs} & 20\% \;{\small (03/15)} & 100\% \;{\small (1/1)} \\ \hline {Comprendre un algorithme} & 60\% \;{\small (03/5)} & 100\% \;{\small (1/1)} \\ \hline {Programmation de base en C} & 40\% \;{\small (28/70)} & 62\% \;{\small (5/8)} \\ \hline {Représentation des entiers} & 0\% \;{\small (00/50)} & 0\% \;{\small (0/8)} \\ \hline {Base d'OCaml (fonctionnel)} & 72\% \;{\small (62/85)} & 100\% \;{\small (11/11)} \\ \hline {Preuve de terminaison} & 100\% \;{\small (05/5)} & 100\% \;{\small (1/1)} \\ \hline \end{tabular} \\\\\medskip \\
    \ding{113} \textbf{\sffamily{Résultats par exercice}} \medskip \\
    \renewcommand{\arraystretch}{1.2}
    \begin{tabular}{|l|r|r|}
    \cline{2-3}
    \multicolumn{1}{l|}{} & \multicolumn{1}{|c|}{Points} & \multicolumn{1}{|c|}{Traitées} \\
    \hline
    Exercice {1} & 57\% \;{\small (23/40)} & 100\% \;{\small (5/5)} \\ \hline Exercice {2} & 80\% \;{\small (60/75)} & 100\% \;{\small (10/10)} \\ \hline Exercice {3} & 54\% \;{\small (27/50)} & 100\% \;{\small (5/5)} \\ \hline Exercice {4} & 0\% \;{\small (00/80)} & 0\% \;{\small (0/11)} \\ \hline \end{tabular} \\\\
    \vspace{0.5cm}\\
    \ding{113} \textbf{\sffamily{Historique des notes}} \medskip \\
    \psset{xunit=1.4cm, yunit=0.2cm}
    \begin{pspicture}(-1,-1)(12,22)

% --- Axe X : numéros de devoir ---
\psaxes[
    Dx=1,
    Dy=2,
    Ox=0,
    Oy=0,
    labels=all,
    ticks=all,
](0,0)(0,0)(7,20)


\listplot[plotstyle=line,showpoints=true,linecolor=black,linewidth=0.7pt, dotstyle=diamond*, dotsize=0.2]{%
    1 8.4
2 9
}

% Minimum de la classe
\listplot[plotstyle=line,showpoints=true,linecolor=gray,linewidth=0.7pt, dotstyle=|, dotangle=90, dotsize=0.15, linestyle=dotted]{%
    1 5.8
2 4.6
}

% Maximum de la classe
\listplot[plotstyle=line,showpoints=true,linecolor=gray,linewidth=0.7pt, dotstyle=|, dotangle=90, dotsize=0.15, linestyle=dotted]{%
    1 18.9
2 18.4
}

% Moyenne de la classe
\listplot[plotstyle=line,showpoints=true,linecolor=gray,linewidth=0.7pt, dotstyle=x, dotsize = 0.15, linestyle=dashed]{%
    1 12.4466666666667
2 13.1933333333333
}
\end{pspicture}
\pagebreak
\begin{tcolorbox}[enhanced,width=\textwidth,center upper,fontupper=\bfseries,drop shadow southwest,sharp corners]
{\sc \large Courounadin-Mouny} Maxence
\end{tcolorbox}
\medskip
\begin{tabularx}{\textwidth}{p{5cm}X}
	\alertbox{\faAward}{Note}{
		\begin{itemize}[leftmargin=0pt]
			\item[\textbullet] Note : \textbf{\large 11.1}
			\item[\textbullet] Rang : \textbf{11}
			\item[\textbullet] Traité : 90 \%
		\end{itemize}
	} &
	\alertbox{\faChartLine}{Statistiques des notes}{
        \psset{xunit=1cm, yunit=1cm,fillstyle=solid}
		\begin{pspicture}(0,-0.1)(16,1.45)
		   \savedata{\data}[11.9 15.4 18.4 11.8 9.0 11.1 16.1 4.6 8.2 15.0 18.0 15.5 17.5 10.0 15.4]
		   \rput{-90}(0,0.9){\psBoxplot[barwidth=1.1cm,yunit=0.5,fillcolor=gray,linewidth=1pt]{\data}}
		   \psaxes[yAxis=false,dx=1cm,Dx=2,labelsep=1pt,linecolor=gray,xlabelFontSize=\scriptstyle](0,0)(10.1,4)
		   \psdot[dotsize=8pt,dotstyle=diamond,linecolor=black,fillstyle=solid,fillcolor=white,linewidth=1pt](5.55,0.85)
           \psdot[dotsize=6pt,dotstyle=x,linecolor=black,linewidth=3pt](6.596666666666667,0.85)
		   \end{pspicture}
	} \\
    
\end{tabularx}\\
\begin{tabularx}{\textwidth}{X}
\alertbox{\faComment}{Commentaire}
{
	Le modèle mémoire du C est à revoir (piles, tas et allocation avec malloc). Dans ta preuve de correction tu dois utiliser l’invariant en sortie de boucle. Les dépassements de capacités sur les entiers non signés ne sont pas UB.
}
\end{tabularx}
\medskip
    \ding{113} \textbf{\sffamily{Résultats par thème}} \medskip \\
    \renewcommand{\arraystretch}{1.2}
    \begin{tabular}{|l|r|r|}
    \cline{2-3}
    \multicolumn{1}{l|}{} & \multicolumn{1}{|c|}{Points} & \multicolumn{1}{|c|}{Traitées} \\
    \hline
    {Preuve de correction} & 80\% \;{\small (12/15)} & 100\% \;{\small (1/1)} \\ \hline {Types structurés en C et pointeurs} & 20\% \;{\small (03/15)} & 100\% \;{\small (1/1)} \\ \hline {Comprendre un algorithme} & 100\% \;{\small (05/5)} & 100\% \;{\small (1/1)} \\ \hline {Programmation de base en C} & 47\% \;{\small (33/70)} & 87\% \;{\small (7/8)} \\ \hline {Représentation des entiers} & 20\% \;{\small (10/50)} & 75\% \;{\small (6/8)} \\ \hline {Base d'OCaml (fonctionnel)} & 81\% \;{\small (69/85)} & 100\% \;{\small (11/11)} \\ \hline {Preuve de terminaison} & 80\% \;{\small (04/5)} & 100\% \;{\small (1/1)} \\ \hline \end{tabular} \\\\\medskip \\
    \ding{113} \textbf{\sffamily{Résultats par exercice}} \medskip \\
    \renewcommand{\arraystretch}{1.2}
    \begin{tabular}{|l|r|r|}
    \cline{2-3}
    \multicolumn{1}{l|}{} & \multicolumn{1}{|c|}{Points} & \multicolumn{1}{|c|}{Traitées} \\
    \hline
    Exercice {1} & 70\% \;{\small (28/40)} & 100\% \;{\small (5/5)} \\ \hline Exercice {2} & 89\% \;{\small (67/75)} & 100\% \;{\small (10/10)} \\ \hline Exercice {3} & 30\% \;{\small (15/50)} & 100\% \;{\small (5/5)} \\ \hline Exercice {4} & 32\% \;{\small (26/80)} & 72\% \;{\small (8/11)} \\ \hline \end{tabular} \\\\
    \vspace{0.5cm}\\
    \ding{113} \textbf{\sffamily{Historique des notes}} \medskip \\
    \psset{xunit=1.4cm, yunit=0.2cm}
    \begin{pspicture}(-1,-1)(12,22)

% --- Axe X : numéros de devoir ---
\psaxes[
    Dx=1,
    Dy=2,
    Ox=0,
    Oy=0,
    labels=all,
    ticks=all,
](0,0)(0,0)(7,20)


\listplot[plotstyle=line,showpoints=true,linecolor=black,linewidth=0.7pt, dotstyle=diamond*, dotsize=0.2]{%
    1 10.9
2 11.1
}

% Minimum de la classe
\listplot[plotstyle=line,showpoints=true,linecolor=gray,linewidth=0.7pt, dotstyle=|, dotangle=90, dotsize=0.15, linestyle=dotted]{%
    1 5.8
2 4.6
}

% Maximum de la classe
\listplot[plotstyle=line,showpoints=true,linecolor=gray,linewidth=0.7pt, dotstyle=|, dotangle=90, dotsize=0.15, linestyle=dotted]{%
    1 18.9
2 18.4
}

% Moyenne de la classe
\listplot[plotstyle=line,showpoints=true,linecolor=gray,linewidth=0.7pt, dotstyle=x, dotsize = 0.15, linestyle=dashed]{%
    1 12.4466666666667
2 13.1933333333333
}
\end{pspicture}
\pagebreak
\begin{tcolorbox}[enhanced,width=\textwidth,center upper,fontupper=\bfseries,drop shadow southwest,sharp corners]
{\sc \large Dominguez} Raphaël
\end{tcolorbox}
\medskip
\begin{tabularx}{\textwidth}{p{5cm}X}
	\alertbox{\faAward}{Note}{
		\begin{itemize}[leftmargin=0pt]
			\item[\textbullet] Note : \textbf{\large 16.1}
			\item[\textbullet] Rang : \textbf{4}
			\item[\textbullet] Traité : 100 \%
		\end{itemize}
	} &
	\alertbox{\faChartLine}{Statistiques des notes}{
        \psset{xunit=1cm, yunit=1cm,fillstyle=solid}
		\begin{pspicture}(0,-0.1)(16,1.45)
		   \savedata{\data}[11.9 15.4 18.4 11.8 9.0 11.1 16.1 4.6 8.2 15.0 18.0 15.5 17.5 10.0 15.4]
		   \rput{-90}(0,0.9){\psBoxplot[barwidth=1.1cm,yunit=0.5,fillcolor=gray,linewidth=1pt]{\data}}
		   \psaxes[yAxis=false,dx=1cm,Dx=2,labelsep=1pt,linecolor=gray,xlabelFontSize=\scriptstyle](0,0)(10.1,4)
		   \psdot[dotsize=8pt,dotstyle=diamond,linecolor=black,fillstyle=solid,fillcolor=white,linewidth=1pt](8.05,0.85)
           \psdot[dotsize=6pt,dotstyle=x,linecolor=black,linewidth=3pt](6.596666666666667,0.85)
		   \end{pspicture}
	} \\
    
\end{tabularx}\\
\begin{tabularx}{\textwidth}{X}
\alertbox{\faComment}{Commentaire}
{
	Très bon devoir, tu as fais des progrès en langage C, c’est très bien. Le Ocaml est encore à travailler.
}
\end{tabularx}
\medskip
    \ding{113} \textbf{\sffamily{Résultats par thème}} \medskip \\
    \renewcommand{\arraystretch}{1.2}
    \begin{tabular}{|l|r|r|}
    \cline{2-3}
    \multicolumn{1}{l|}{} & \multicolumn{1}{|c|}{Points} & \multicolumn{1}{|c|}{Traitées} \\
    \hline
    {Preuve de correction} & 100\% \;{\small (15/15)} & 100\% \;{\small (1/1)} \\ \hline {Types structurés en C et pointeurs} & 100\% \;{\small (15/15)} & 100\% \;{\small (1/1)} \\ \hline {Comprendre un algorithme} & 100\% \;{\small (05/5)} & 100\% \;{\small (1/1)} \\ \hline {Programmation de base en C} & 81\% \;{\small (57/70)} & 100\% \;{\small (8/8)} \\ \hline {Représentation des entiers} & 100\% \;{\small (50/50)} & 100\% \;{\small (8/8)} \\ \hline {Base d'OCaml (fonctionnel)} & 61\% \;{\small (52/85)} & 100\% \;{\small (11/11)} \\ \hline {Preuve de terminaison} & 60\% \;{\small (03/5)} & 100\% \;{\small (1/1)} \\ \hline \end{tabular} \\\\\medskip \\
    \ding{113} \textbf{\sffamily{Résultats par exercice}} \medskip \\
    \renewcommand{\arraystretch}{1.2}
    \begin{tabular}{|l|r|r|}
    \cline{2-3}
    \multicolumn{1}{l|}{} & \multicolumn{1}{|c|}{Points} & \multicolumn{1}{|c|}{Traitées} \\
    \hline
    Exercice {1} & 87\% \;{\small (35/40)} & 100\% \;{\small (5/5)} \\ \hline Exercice {2} & 58\% \;{\small (44/75)} & 100\% \;{\small (10/10)} \\ \hline Exercice {3} & 76\% \;{\small (38/50)} & 100\% \;{\small (5/5)} \\ \hline Exercice {4} & 100\% \;{\small (80/80)} & 100\% \;{\small (11/11)} \\ \hline \end{tabular} \\\\
    \vspace{0.5cm}\\
    \ding{113} \textbf{\sffamily{Historique des notes}} \medskip \\
    \psset{xunit=1.4cm, yunit=0.2cm}
    \begin{pspicture}(-1,-1)(12,22)

% --- Axe X : numéros de devoir ---
\psaxes[
    Dx=1,
    Dy=2,
    Ox=0,
    Oy=0,
    labels=all,
    ticks=all,
](0,0)(0,0)(7,20)


\listplot[plotstyle=line,showpoints=true,linecolor=black,linewidth=0.7pt, dotstyle=diamond*, dotsize=0.2]{%
    1 15.7
2 16.1
}

% Minimum de la classe
\listplot[plotstyle=line,showpoints=true,linecolor=gray,linewidth=0.7pt, dotstyle=|, dotangle=90, dotsize=0.15, linestyle=dotted]{%
    1 5.8
2 4.6
}

% Maximum de la classe
\listplot[plotstyle=line,showpoints=true,linecolor=gray,linewidth=0.7pt, dotstyle=|, dotangle=90, dotsize=0.15, linestyle=dotted]{%
    1 18.9
2 18.4
}

% Moyenne de la classe
\listplot[plotstyle=line,showpoints=true,linecolor=gray,linewidth=0.7pt, dotstyle=x, dotsize = 0.15, linestyle=dashed]{%
    1 12.4466666666667
2 13.1933333333333
}
\end{pspicture}
\pagebreak
\begin{tcolorbox}[enhanced,width=\textwidth,center upper,fontupper=\bfseries,drop shadow southwest,sharp corners]
{\sc \large Garbal} Alizée
\end{tcolorbox}
\medskip
\begin{tabularx}{\textwidth}{p{5cm}X}
	\alertbox{\faAward}{Note}{
		\begin{itemize}[leftmargin=0pt]
			\item[\textbullet] Note : \textbf{\large 4.6}
			\item[\textbullet] Rang : \textbf{15}
			\item[\textbullet] Traité : 74 \%
		\end{itemize}
	} &
	\alertbox{\faChartLine}{Statistiques des notes}{
        \psset{xunit=1cm, yunit=1cm,fillstyle=solid}
		\begin{pspicture}(0,-0.1)(16,1.45)
		   \savedata{\data}[11.9 15.4 18.4 11.8 9.0 11.1 16.1 4.6 8.2 15.0 18.0 15.5 17.5 10.0 15.4]
		   \rput{-90}(0,0.9){\psBoxplot[barwidth=1.1cm,yunit=0.5,fillcolor=gray,linewidth=1pt]{\data}}
		   \psaxes[yAxis=false,dx=1cm,Dx=2,labelsep=1pt,linecolor=gray,xlabelFontSize=\scriptstyle](0,0)(10.1,4)
		   \psdot[dotsize=8pt,dotstyle=diamond,linecolor=black,fillstyle=solid,fillcolor=white,linewidth=1pt](2.3,0.85)
           \psdot[dotsize=6pt,dotstyle=x,linecolor=black,linewidth=3pt](6.596666666666667,0.85)
		   \end{pspicture}
	} \\
    
\end{tabularx}\\
\begin{tabularx}{\textwidth}{X}
\alertbox{\faComment}{Commentaire}
{
	C’est nettement insuffisant, il faut notamment absolument progresser sur la programmation en langage C et comprendre le principe d’une preuve de correction ou de terminaison.
}
\end{tabularx}
\medskip
    \ding{113} \textbf{\sffamily{Résultats par thème}} \medskip \\
    \renewcommand{\arraystretch}{1.2}
    \begin{tabular}{|l|r|r|}
    \cline{2-3}
    \multicolumn{1}{l|}{} & \multicolumn{1}{|c|}{Points} & \multicolumn{1}{|c|}{Traitées} \\
    \hline
    {Preuve de correction} & 0\% \;{\small (00/15)} & 100\% \;{\small (1/1)} \\ \hline {Types structurés en C et pointeurs} & 0\% \;{\small (00/15)} & 0\% \;{\small (0/1)} \\ \hline {Comprendre un algorithme} & 0\% \;{\small (00/5)} & 100\% \;{\small (1/1)} \\ \hline {Programmation de base en C} & 4\% \;{\small (03/70)} & 50\% \;{\small (4/8)} \\ \hline {Représentation des entiers} & 16\% \;{\small (08/50)} & 62\% \;{\small (5/8)} \\ \hline {Base d'OCaml (fonctionnel)} & 52\% \;{\small (45/85)} & 100\% \;{\small (11/11)} \\ \hline {Preuve de terminaison} & 0\% \;{\small (00/5)} & 100\% \;{\small (1/1)} \\ \hline \end{tabular} \\\\\medskip \\
    \ding{113} \textbf{\sffamily{Résultats par exercice}} \medskip \\
    \renewcommand{\arraystretch}{1.2}
    \begin{tabular}{|l|r|r|}
    \cline{2-3}
    \multicolumn{1}{l|}{} & \multicolumn{1}{|c|}{Points} & \multicolumn{1}{|c|}{Traitées} \\
    \hline
    Exercice {1} & 7\% \;{\small (03/40)} & 100\% \;{\small (5/5)} \\ \hline Exercice {2} & 60\% \;{\small (45/75)} & 100\% \;{\small (10/10)} \\ \hline Exercice {3} & 0\% \;{\small (00/50)} & 60\% \;{\small (3/5)} \\ \hline Exercice {4} & 10\% \;{\small (08/80)} & 45\% \;{\small (5/11)} \\ \hline \end{tabular} \\\\
    \vspace{0.5cm}\\
    \ding{113} \textbf{\sffamily{Historique des notes}} \medskip \\
    \psset{xunit=1.4cm, yunit=0.2cm}
    \begin{pspicture}(-1,-1)(12,22)

% --- Axe X : numéros de devoir ---
\psaxes[
    Dx=1,
    Dy=2,
    Ox=0,
    Oy=0,
    labels=all,
    ticks=all,
](0,0)(0,0)(7,20)


\listplot[plotstyle=line,showpoints=true,linecolor=black,linewidth=0.7pt, dotstyle=diamond*, dotsize=0.2]{%
    1 5.8
2 4.6
}

% Minimum de la classe
\listplot[plotstyle=line,showpoints=true,linecolor=gray,linewidth=0.7pt, dotstyle=|, dotangle=90, dotsize=0.15, linestyle=dotted]{%
    1 5.8
2 4.6
}

% Maximum de la classe
\listplot[plotstyle=line,showpoints=true,linecolor=gray,linewidth=0.7pt, dotstyle=|, dotangle=90, dotsize=0.15, linestyle=dotted]{%
    1 18.9
2 18.4
}

% Moyenne de la classe
\listplot[plotstyle=line,showpoints=true,linecolor=gray,linewidth=0.7pt, dotstyle=x, dotsize = 0.15, linestyle=dashed]{%
    1 12.4466666666667
2 13.1933333333333
}
\end{pspicture}
\pagebreak
\begin{tcolorbox}[enhanced,width=\textwidth,center upper,fontupper=\bfseries,drop shadow southwest,sharp corners]
{\sc \large Hoarau} Alessandro
\end{tcolorbox}
\medskip
\begin{tabularx}{\textwidth}{p{5cm}X}
	\alertbox{\faAward}{Note}{
		\begin{itemize}[leftmargin=0pt]
			\item[\textbullet] Note : \textbf{\large 8.2}
			\item[\textbullet] Rang : \textbf{14}
			\item[\textbullet] Traité : 55 \%
		\end{itemize}
	} &
	\alertbox{\faChartLine}{Statistiques des notes}{
        \psset{xunit=1cm, yunit=1cm,fillstyle=solid}
		\begin{pspicture}(0,-0.1)(16,1.45)
		   \savedata{\data}[11.9 15.4 18.4 11.8 9.0 11.1 16.1 4.6 8.2 15.0 18.0 15.5 17.5 10.0 15.4]
		   \rput{-90}(0,0.9){\psBoxplot[barwidth=1.1cm,yunit=0.5,fillcolor=gray,linewidth=1pt]{\data}}
		   \psaxes[yAxis=false,dx=1cm,Dx=2,labelsep=1pt,linecolor=gray,xlabelFontSize=\scriptstyle](0,0)(10.1,4)
		   \psdot[dotsize=8pt,dotstyle=diamond,linecolor=black,fillstyle=solid,fillcolor=white,linewidth=1pt](4.1,0.85)
           \psdot[dotsize=6pt,dotstyle=x,linecolor=black,linewidth=3pt](6.596666666666667,0.85)
		   \end{pspicture}
	} \\
    
\end{tabularx}\\
\begin{tabularx}{\textwidth}{X}
\alertbox{\faComment}{Commentaire}
{
	Il faut revoir les preuves de corrections et de terminaison. Les bases du C semblent acquises mais le modèle mémoire et les pointeurs sont encore à travailler.
}
\end{tabularx}
\medskip
    \ding{113} \textbf{\sffamily{Résultats par thème}} \medskip \\
    \renewcommand{\arraystretch}{1.2}
    \begin{tabular}{|l|r|r|}
    \cline{2-3}
    \multicolumn{1}{l|}{} & \multicolumn{1}{|c|}{Points} & \multicolumn{1}{|c|}{Traitées} \\
    \hline
    {Preuve de correction} & 20\% \;{\small (03/15)} & 100\% \;{\small (1/1)} \\ \hline {Types structurés en C et pointeurs} & 0\% \;{\small (00/15)} & 0\% \;{\small (0/1)} \\ \hline {Comprendre un algorithme} & 100\% \;{\small (05/5)} & 100\% \;{\small (1/1)} \\ \hline {Programmation de base en C} & 27\% \;{\small (19/70)} & 37\% \;{\small (3/8)} \\ \hline {Représentation des entiers} & 0\% \;{\small (00/50)} & 0\% \;{\small (0/8)} \\ \hline {Base d'OCaml (fonctionnel)} & 83\% \;{\small (71/85)} & 100\% \;{\small (11/11)} \\ \hline {Preuve de terminaison} & 60\% \;{\small (03/5)} & 100\% \;{\small (1/1)} \\ \hline \end{tabular} \\\\\medskip \\
    \ding{113} \textbf{\sffamily{Résultats par exercice}} \medskip \\
    \renewcommand{\arraystretch}{1.2}
    \begin{tabular}{|l|r|r|}
    \cline{2-3}
    \multicolumn{1}{l|}{} & \multicolumn{1}{|c|}{Points} & \multicolumn{1}{|c|}{Traitées} \\
    \hline
    Exercice {1} & 40\% \;{\small (16/40)} & 100\% \;{\small (5/5)} \\ \hline Exercice {2} & 94\% \;{\small (71/75)} & 100\% \;{\small (10/10)} \\ \hline Exercice {3} & 28\% \;{\small (14/50)} & 40\% \;{\small (2/5)} \\ \hline Exercice {4} & 0\% \;{\small (00/80)} & 0\% \;{\small (0/11)} \\ \hline \end{tabular} \\\\
    \vspace{0.5cm}\\
    \ding{113} \textbf{\sffamily{Historique des notes}} \medskip \\
    \psset{xunit=1.4cm, yunit=0.2cm}
    \begin{pspicture}(-1,-1)(12,22)

% --- Axe X : numéros de devoir ---
\psaxes[
    Dx=1,
    Dy=2,
    Ox=0,
    Oy=0,
    labels=all,
    ticks=all,
](0,0)(0,0)(7,20)


\listplot[plotstyle=line,showpoints=true,linecolor=black,linewidth=0.7pt, dotstyle=diamond*, dotsize=0.2]{%
    1 8
2 8.2
}

% Minimum de la classe
\listplot[plotstyle=line,showpoints=true,linecolor=gray,linewidth=0.7pt, dotstyle=|, dotangle=90, dotsize=0.15, linestyle=dotted]{%
    1 5.8
2 4.6
}

% Maximum de la classe
\listplot[plotstyle=line,showpoints=true,linecolor=gray,linewidth=0.7pt, dotstyle=|, dotangle=90, dotsize=0.15, linestyle=dotted]{%
    1 18.9
2 18.4
}

% Moyenne de la classe
\listplot[plotstyle=line,showpoints=true,linecolor=gray,linewidth=0.7pt, dotstyle=x, dotsize = 0.15, linestyle=dashed]{%
    1 12.4466666666667
2 13.1933333333333
}
\end{pspicture}
\pagebreak
\begin{tcolorbox}[enhanced,width=\textwidth,center upper,fontupper=\bfseries,drop shadow southwest,sharp corners]
{\sc \large Mahomed Issop} Jérémy
\end{tcolorbox}
\medskip
\begin{tabularx}{\textwidth}{p{5cm}X}
	\alertbox{\faAward}{Note}{
		\begin{itemize}[leftmargin=0pt]
			\item[\textbullet] Note : \textbf{\large 15.0}
			\item[\textbullet] Rang : \textbf{8}
			\item[\textbullet] Traité : 87 \%
		\end{itemize}
	} &
	\alertbox{\faChartLine}{Statistiques des notes}{
        \psset{xunit=1cm, yunit=1cm,fillstyle=solid}
		\begin{pspicture}(0,-0.1)(16,1.45)
		   \savedata{\data}[11.9 15.4 18.4 11.8 9.0 11.1 16.1 4.6 8.2 15.0 18.0 15.5 17.5 10.0 15.4]
		   \rput{-90}(0,0.9){\psBoxplot[barwidth=1.1cm,yunit=0.5,fillcolor=gray,linewidth=1pt]{\data}}
		   \psaxes[yAxis=false,dx=1cm,Dx=2,labelsep=1pt,linecolor=gray,xlabelFontSize=\scriptstyle](0,0)(10.1,4)
		   \psdot[dotsize=8pt,dotstyle=diamond,linecolor=black,fillstyle=solid,fillcolor=white,linewidth=1pt](7.5,0.85)
           \psdot[dotsize=6pt,dotstyle=x,linecolor=black,linewidth=3pt](6.596666666666667,0.85)
		   \end{pspicture}
	} \\
    
\end{tabularx}\\
\begin{tabularx}{\textwidth}{X}
\alertbox{\faComment}{Commentaire}
{
	Très bon travail dans l’ensemble, revois le modèle mémoire du langage C et notamment le problème lié au fait de renvoyer l’adresse d’une variable locale à  une fonction.
}
\end{tabularx}
\medskip
    \ding{113} \textbf{\sffamily{Résultats par thème}} \medskip \\
    \renewcommand{\arraystretch}{1.2}
    \begin{tabular}{|l|r|r|}
    \cline{2-3}
    \multicolumn{1}{l|}{} & \multicolumn{1}{|c|}{Points} & \multicolumn{1}{|c|}{Traitées} \\
    \hline
    {Preuve de correction} & 100\% \;{\small (15/15)} & 100\% \;{\small (1/1)} \\ \hline {Types structurés en C et pointeurs} & 100\% \;{\small (15/15)} & 100\% \;{\small (1/1)} \\ \hline {Comprendre un algorithme} & 100\% \;{\small (05/5)} & 100\% \;{\small (1/1)} \\ \hline {Programmation de base en C} & 47\% \;{\small (33/70)} & 75\% \;{\small (6/8)} \\ \hline {Représentation des entiers} & 60\% \;{\small (30/50)} & 75\% \;{\small (6/8)} \\ \hline {Base d'OCaml (fonctionnel)} & 95\% \;{\small (81/85)} & 100\% \;{\small (11/11)} \\ \hline {Preuve de terminaison} & 100\% \;{\small (05/5)} & 100\% \;{\small (1/1)} \\ \hline \end{tabular} \\\\\medskip \\
    \ding{113} \textbf{\sffamily{Résultats par exercice}} \medskip \\
    \renewcommand{\arraystretch}{1.2}
    \begin{tabular}{|l|r|r|}
    \cline{2-3}
    \multicolumn{1}{l|}{} & \multicolumn{1}{|c|}{Points} & \multicolumn{1}{|c|}{Traitées} \\
    \hline
    Exercice {1} & 95\% \;{\small (38/40)} & 100\% \;{\small (5/5)} \\ \hline Exercice {2} & 97\% \;{\small (73/75)} & 100\% \;{\small (10/10)} \\ \hline Exercice {3} & 66\% \;{\small (33/50)} & 100\% \;{\small (5/5)} \\ \hline Exercice {4} & 50\% \;{\small (40/80)} & 63\% \;{\small (7/11)} \\ \hline \end{tabular} \\\\
    \vspace{0.5cm}\\
    \ding{113} \textbf{\sffamily{Historique des notes}} \medskip \\
    \psset{xunit=1.4cm, yunit=0.2cm}
    \begin{pspicture}(-1,-1)(12,22)

% --- Axe X : numéros de devoir ---
\psaxes[
    Dx=1,
    Dy=2,
    Ox=0,
    Oy=0,
    labels=all,
    ticks=all,
](0,0)(0,0)(7,20)


\listplot[plotstyle=line,showpoints=true,linecolor=black,linewidth=0.7pt, dotstyle=diamond*, dotsize=0.2]{%
    1 13.5
2 15
}

% Minimum de la classe
\listplot[plotstyle=line,showpoints=true,linecolor=gray,linewidth=0.7pt, dotstyle=|, dotangle=90, dotsize=0.15, linestyle=dotted]{%
    1 5.8
2 4.6
}

% Maximum de la classe
\listplot[plotstyle=line,showpoints=true,linecolor=gray,linewidth=0.7pt, dotstyle=|, dotangle=90, dotsize=0.15, linestyle=dotted]{%
    1 18.9
2 18.4
}

% Moyenne de la classe
\listplot[plotstyle=line,showpoints=true,linecolor=gray,linewidth=0.7pt, dotstyle=x, dotsize = 0.15, linestyle=dashed]{%
    1 12.4466666666667
2 13.1933333333333
}
\end{pspicture}
\pagebreak
\begin{tcolorbox}[enhanced,width=\textwidth,center upper,fontupper=\bfseries,drop shadow southwest,sharp corners]
{\sc \large Mamodhoussen} Djavad
\end{tcolorbox}
\medskip
\begin{tabularx}{\textwidth}{p{5cm}X}
	\alertbox{\faAward}{Note}{
		\begin{itemize}[leftmargin=0pt]
			\item[\textbullet] Note : \textbf{\large 18.0}
			\item[\textbullet] Rang : \textbf{2}
			\item[\textbullet] Traité : 100 \%
		\end{itemize}
	} &
	\alertbox{\faChartLine}{Statistiques des notes}{
        \psset{xunit=1cm, yunit=1cm,fillstyle=solid}
		\begin{pspicture}(0,-0.1)(16,1.45)
		   \savedata{\data}[11.9 15.4 18.4 11.8 9.0 11.1 16.1 4.6 8.2 15.0 18.0 15.5 17.5 10.0 15.4]
		   \rput{-90}(0,0.9){\psBoxplot[barwidth=1.1cm,yunit=0.5,fillcolor=gray,linewidth=1pt]{\data}}
		   \psaxes[yAxis=false,dx=1cm,Dx=2,labelsep=1pt,linecolor=gray,xlabelFontSize=\scriptstyle](0,0)(10.1,4)
		   \psdot[dotsize=8pt,dotstyle=diamond,linecolor=black,fillstyle=solid,fillcolor=white,linewidth=1pt](9.0,0.85)
           \psdot[dotsize=6pt,dotstyle=x,linecolor=black,linewidth=3pt](6.596666666666667,0.85)
		   \end{pspicture}
	} \\
    
\end{tabularx}\\
\begin{tabularx}{\textwidth}{X}
\alertbox{\faComment}{Commentaire}
{
	Excellent travail, bravo, il faut continuer sur ainsi.
}
\end{tabularx}
\medskip
    \ding{113} \textbf{\sffamily{Résultats par thème}} \medskip \\
    \renewcommand{\arraystretch}{1.2}
    \begin{tabular}{|l|r|r|}
    \cline{2-3}
    \multicolumn{1}{l|}{} & \multicolumn{1}{|c|}{Points} & \multicolumn{1}{|c|}{Traitées} \\
    \hline
    {Preuve de correction} & 100\% \;{\small (15/15)} & 100\% \;{\small (1/1)} \\ \hline {Types structurés en C et pointeurs} & 100\% \;{\small (15/15)} & 100\% \;{\small (1/1)} \\ \hline {Comprendre un algorithme} & 100\% \;{\small (05/5)} & 100\% \;{\small (1/1)} \\ \hline {Programmation de base en C} & 77\% \;{\small (54/70)} & 100\% \;{\small (8/8)} \\ \hline {Représentation des entiers} & 84\% \;{\small (42/50)} & 100\% \;{\small (8/8)} \\ \hline {Base d'OCaml (fonctionnel)} & 100\% \;{\small (85/85)} & 100\% \;{\small (11/11)} \\ \hline {Preuve de terminaison} & 100\% \;{\small (05/5)} & 100\% \;{\small (1/1)} \\ \hline \end{tabular} \\\\\medskip \\
    \ding{113} \textbf{\sffamily{Résultats par exercice}} \medskip \\
    \renewcommand{\arraystretch}{1.2}
    \begin{tabular}{|l|r|r|}
    \cline{2-3}
    \multicolumn{1}{l|}{} & \multicolumn{1}{|c|}{Points} & \multicolumn{1}{|c|}{Traitées} \\
    \hline
    Exercice {1} & 100\% \;{\small (40/40)} & 100\% \;{\small (5/5)} \\ \hline Exercice {2} & 100\% \;{\small (75/75)} & 100\% \;{\small (10/10)} \\ \hline Exercice {3} & 88\% \;{\small (44/50)} & 100\% \;{\small (5/5)} \\ \hline Exercice {4} & 77\% \;{\small (62/80)} & 100\% \;{\small (11/11)} \\ \hline \end{tabular} \\\\
    \vspace{0.5cm}\\
    \ding{113} \textbf{\sffamily{Historique des notes}} \medskip \\
    \psset{xunit=1.4cm, yunit=0.2cm}
    \begin{pspicture}(-1,-1)(12,22)

% --- Axe X : numéros de devoir ---
\psaxes[
    Dx=1,
    Dy=2,
    Ox=0,
    Oy=0,
    labels=all,
    ticks=all,
](0,0)(0,0)(7,20)


\listplot[plotstyle=line,showpoints=true,linecolor=black,linewidth=0.7pt, dotstyle=diamond*, dotsize=0.2]{%
    1 17.8
2 18
}

% Minimum de la classe
\listplot[plotstyle=line,showpoints=true,linecolor=gray,linewidth=0.7pt, dotstyle=|, dotangle=90, dotsize=0.15, linestyle=dotted]{%
    1 5.8
2 4.6
}

% Maximum de la classe
\listplot[plotstyle=line,showpoints=true,linecolor=gray,linewidth=0.7pt, dotstyle=|, dotangle=90, dotsize=0.15, linestyle=dotted]{%
    1 18.9
2 18.4
}

% Moyenne de la classe
\listplot[plotstyle=line,showpoints=true,linecolor=gray,linewidth=0.7pt, dotstyle=x, dotsize = 0.15, linestyle=dashed]{%
    1 12.4466666666667
2 13.1933333333333
}
\end{pspicture}
\pagebreak
\begin{tcolorbox}[enhanced,width=\textwidth,center upper,fontupper=\bfseries,drop shadow southwest,sharp corners]
{\sc \large Morel} Lucas
\end{tcolorbox}
\medskip
\begin{tabularx}{\textwidth}{p{5cm}X}
	\alertbox{\faAward}{Note}{
		\begin{itemize}[leftmargin=0pt]
			\item[\textbullet] Note : \textbf{\large 15.5}
			\item[\textbullet] Rang : \textbf{5}
			\item[\textbullet] Traité : 94 \%
		\end{itemize}
	} &
	\alertbox{\faChartLine}{Statistiques des notes}{
        \psset{xunit=1cm, yunit=1cm,fillstyle=solid}
		\begin{pspicture}(0,-0.1)(16,1.45)
		   \savedata{\data}[11.9 15.4 18.4 11.8 9.0 11.1 16.1 4.6 8.2 15.0 18.0 15.5 17.5 10.0 15.4]
		   \rput{-90}(0,0.9){\psBoxplot[barwidth=1.1cm,yunit=0.5,fillcolor=gray,linewidth=1pt]{\data}}
		   \psaxes[yAxis=false,dx=1cm,Dx=2,labelsep=1pt,linecolor=gray,xlabelFontSize=\scriptstyle](0,0)(10.1,4)
		   \psdot[dotsize=8pt,dotstyle=diamond,linecolor=black,fillstyle=solid,fillcolor=white,linewidth=1pt](7.75,0.85)
           \psdot[dotsize=6pt,dotstyle=x,linecolor=black,linewidth=3pt](6.596666666666667,0.85)
		   \end{pspicture}
	} \\
    
\end{tabularx}\\
\begin{tabularx}{\textwidth}{X}
\alertbox{\faComment}{Commentaire}
{
	Bon devoir, c’est très bien mais le modèle mémoire du C, les pointeurs et  les allocations mémoire doivent être encore travaillés, tu devrais par exemple prendre le temps de coder les questions de l’exercice 3 ou refaire un exercice similaire vu en TP.
}
\end{tabularx}
\medskip
    \ding{113} \textbf{\sffamily{Résultats par thème}} \medskip \\
    \renewcommand{\arraystretch}{1.2}
    \begin{tabular}{|l|r|r|}
    \cline{2-3}
    \multicolumn{1}{l|}{} & \multicolumn{1}{|c|}{Points} & \multicolumn{1}{|c|}{Traitées} \\
    \hline
    {Preuve de correction} & 80\% \;{\small (12/15)} & 100\% \;{\small (1/1)} \\ \hline {Types structurés en C et pointeurs} & 20\% \;{\small (03/15)} & 100\% \;{\small (1/1)} \\ \hline {Comprendre un algorithme} & 80\% \;{\small (04/5)} & 100\% \;{\small (1/1)} \\ \hline {Programmation de base en C} & 58\% \;{\small (41/70)} & 75\% \;{\small (6/8)} \\ \hline {Représentation des entiers} & 80\% \;{\small (40/50)} & 100\% \;{\small (8/8)} \\ \hline {Base d'OCaml (fonctionnel)} & 100\% \;{\small (85/85)} & 100\% \;{\small (11/11)} \\ \hline {Preuve de terminaison} & 100\% \;{\small (05/5)} & 100\% \;{\small (1/1)} \\ \hline \end{tabular} \\\\\medskip \\
    \ding{113} \textbf{\sffamily{Résultats par exercice}} \medskip \\
    \renewcommand{\arraystretch}{1.2}
    \begin{tabular}{|l|r|r|}
    \cline{2-3}
    \multicolumn{1}{l|}{} & \multicolumn{1}{|c|}{Points} & \multicolumn{1}{|c|}{Traitées} \\
    \hline
    Exercice {1} & 85\% \;{\small (34/40)} & 100\% \;{\small (5/5)} \\ \hline Exercice {2} & 100\% \;{\small (75/75)} & 100\% \;{\small (10/10)} \\ \hline Exercice {3} & 62\% \;{\small (31/50)} & 100\% \;{\small (5/5)} \\ \hline Exercice {4} & 62\% \;{\small (50/80)} & 81\% \;{\small (9/11)} \\ \hline \end{tabular} \\\\
    \vspace{0.5cm}\\
    \ding{113} \textbf{\sffamily{Historique des notes}} \medskip \\
    \psset{xunit=1.4cm, yunit=0.2cm}
    \begin{pspicture}(-1,-1)(12,22)

% --- Axe X : numéros de devoir ---
\psaxes[
    Dx=1,
    Dy=2,
    Ox=0,
    Oy=0,
    labels=all,
    ticks=all,
](0,0)(0,0)(7,20)


\listplot[plotstyle=line,showpoints=true,linecolor=black,linewidth=0.7pt, dotstyle=diamond*, dotsize=0.2]{%
    1 16.7
2 15.5
}

% Minimum de la classe
\listplot[plotstyle=line,showpoints=true,linecolor=gray,linewidth=0.7pt, dotstyle=|, dotangle=90, dotsize=0.15, linestyle=dotted]{%
    1 5.8
2 4.6
}

% Maximum de la classe
\listplot[plotstyle=line,showpoints=true,linecolor=gray,linewidth=0.7pt, dotstyle=|, dotangle=90, dotsize=0.15, linestyle=dotted]{%
    1 18.9
2 18.4
}

% Moyenne de la classe
\listplot[plotstyle=line,showpoints=true,linecolor=gray,linewidth=0.7pt, dotstyle=x, dotsize = 0.15, linestyle=dashed]{%
    1 12.4466666666667
2 13.1933333333333
}
\end{pspicture}
\pagebreak
\begin{tcolorbox}[enhanced,width=\textwidth,center upper,fontupper=\bfseries,drop shadow southwest,sharp corners]
{\sc \large Randriamiarivola Korodo} Lionel
\end{tcolorbox}
\medskip
\begin{tabularx}{\textwidth}{p{5cm}X}
	\alertbox{\faAward}{Note}{
		\begin{itemize}[leftmargin=0pt]
			\item[\textbullet] Note : \textbf{\large 17.5}
			\item[\textbullet] Rang : \textbf{3}
			\item[\textbullet] Traité : 100 \%
		\end{itemize}
	} &
	\alertbox{\faChartLine}{Statistiques des notes}{
        \psset{xunit=1cm, yunit=1cm,fillstyle=solid}
		\begin{pspicture}(0,-0.1)(16,1.45)
		   \savedata{\data}[11.9 15.4 18.4 11.8 9.0 11.1 16.1 4.6 8.2 15.0 18.0 15.5 17.5 10.0 15.4]
		   \rput{-90}(0,0.9){\psBoxplot[barwidth=1.1cm,yunit=0.5,fillcolor=gray,linewidth=1pt]{\data}}
		   \psaxes[yAxis=false,dx=1cm,Dx=2,labelsep=1pt,linecolor=gray,xlabelFontSize=\scriptstyle](0,0)(10.1,4)
		   \psdot[dotsize=8pt,dotstyle=diamond,linecolor=black,fillstyle=solid,fillcolor=white,linewidth=1pt](8.75,0.85)
           \psdot[dotsize=6pt,dotstyle=x,linecolor=black,linewidth=3pt](6.596666666666667,0.85)
		   \end{pspicture}
	} \\
    
\end{tabularx}\\
\begin{tabularx}{\textwidth}{X}
\alertbox{\faComment}{Commentaire}
{
	Bon devoir, tous les exercices sont bien traités.  Il faut revoir l’utilisation du in dans les expressions de Ocaml
}
\end{tabularx}
\medskip
    \ding{113} \textbf{\sffamily{Résultats par thème}} \medskip \\
    \renewcommand{\arraystretch}{1.2}
    \begin{tabular}{|l|r|r|}
    \cline{2-3}
    \multicolumn{1}{l|}{} & \multicolumn{1}{|c|}{Points} & \multicolumn{1}{|c|}{Traitées} \\
    \hline
    {Preuve de correction} & 100\% \;{\small (15/15)} & 100\% \;{\small (1/1)} \\ \hline {Types structurés en C et pointeurs} & 100\% \;{\small (15/15)} & 100\% \;{\small (1/1)} \\ \hline {Comprendre un algorithme} & 100\% \;{\small (05/5)} & 100\% \;{\small (1/1)} \\ \hline {Programmation de base en C} & 85\% \;{\small (60/70)} & 100\% \;{\small (8/8)} \\ \hline {Représentation des entiers} & 100\% \;{\small (50/50)} & 100\% \;{\small (8/8)} \\ \hline {Base d'OCaml (fonctionnel)} & 76\% \;{\small (65/85)} & 100\% \;{\small (11/11)} \\ \hline {Preuve de terminaison} & 80\% \;{\small (04/5)} & 100\% \;{\small (1/1)} \\ \hline \end{tabular} \\\\\medskip \\
    \ding{113} \textbf{\sffamily{Résultats par exercice}} \medskip \\
    \renewcommand{\arraystretch}{1.2}
    \begin{tabular}{|l|r|r|}
    \cline{2-3}
    \multicolumn{1}{l|}{} & \multicolumn{1}{|c|}{Points} & \multicolumn{1}{|c|}{Traitées} \\
    \hline
    Exercice {1} & 97\% \;{\small (39/40)} & 100\% \;{\small (5/5)} \\ \hline Exercice {2} & 73\% \;{\small (55/75)} & 100\% \;{\small (10/10)} \\ \hline Exercice {3} & 100\% \;{\small (50/50)} & 100\% \;{\small (5/5)} \\ \hline Exercice {4} & 87\% \;{\small (70/80)} & 100\% \;{\small (11/11)} \\ \hline \end{tabular} \\\\
    \vspace{0.5cm}\\
    \ding{113} \textbf{\sffamily{Historique des notes}} \medskip \\
    \psset{xunit=1.4cm, yunit=0.2cm}
    \begin{pspicture}(-1,-1)(12,22)

% --- Axe X : numéros de devoir ---
\psaxes[
    Dx=1,
    Dy=2,
    Ox=0,
    Oy=0,
    labels=all,
    ticks=all,
](0,0)(0,0)(7,20)


\listplot[plotstyle=line,showpoints=true,linecolor=black,linewidth=0.7pt, dotstyle=diamond*, dotsize=0.2]{%
    1 18.6
2 17.5
}

% Minimum de la classe
\listplot[plotstyle=line,showpoints=true,linecolor=gray,linewidth=0.7pt, dotstyle=|, dotangle=90, dotsize=0.15, linestyle=dotted]{%
    1 5.8
2 4.6
}

% Maximum de la classe
\listplot[plotstyle=line,showpoints=true,linecolor=gray,linewidth=0.7pt, dotstyle=|, dotangle=90, dotsize=0.15, linestyle=dotted]{%
    1 18.9
2 18.4
}

% Moyenne de la classe
\listplot[plotstyle=line,showpoints=true,linecolor=gray,linewidth=0.7pt, dotstyle=x, dotsize = 0.15, linestyle=dashed]{%
    1 12.4466666666667
2 13.1933333333333
}
\end{pspicture}
\pagebreak
\begin{tcolorbox}[enhanced,width=\textwidth,center upper,fontupper=\bfseries,drop shadow southwest,sharp corners]
{\sc \large Rasolofotsara} Ando
\end{tcolorbox}
\medskip
\begin{tabularx}{\textwidth}{p{5cm}X}
	\alertbox{\faAward}{Note}{
		\begin{itemize}[leftmargin=0pt]
			\item[\textbullet] Note : \textbf{\large 10.0}
			\item[\textbullet] Rang : \textbf{12}
			\item[\textbullet] Traité : 77 \%
		\end{itemize}
	} &
	\alertbox{\faChartLine}{Statistiques des notes}{
        \psset{xunit=1cm, yunit=1cm,fillstyle=solid}
		\begin{pspicture}(0,-0.1)(16,1.45)
		   \savedata{\data}[11.9 15.4 18.4 11.8 9.0 11.1 16.1 4.6 8.2 15.0 18.0 15.5 17.5 10.0 15.4]
		   \rput{-90}(0,0.9){\psBoxplot[barwidth=1.1cm,yunit=0.5,fillcolor=gray,linewidth=1pt]{\data}}
		   \psaxes[yAxis=false,dx=1cm,Dx=2,labelsep=1pt,linecolor=gray,xlabelFontSize=\scriptstyle](0,0)(10.1,4)
		   \psdot[dotsize=8pt,dotstyle=diamond,linecolor=black,fillstyle=solid,fillcolor=white,linewidth=1pt](5.0,0.85)
           \psdot[dotsize=6pt,dotstyle=x,linecolor=black,linewidth=3pt](6.596666666666667,0.85)
		   \end{pspicture}
	} \\
    
\end{tabularx}\\
\begin{tabularx}{\textwidth}{X}
\alertbox{\faComment}{Commentaire}
{
	Tu es en progrès, il faut continuer à travailler. Tu devrais refaire l’exercice 3 en codant et en testant tes fonctions afin de bien assimiler les struct et les pointeurs.
}
\end{tabularx}
\medskip
    \ding{113} \textbf{\sffamily{Résultats par thème}} \medskip \\
    \renewcommand{\arraystretch}{1.2}
    \begin{tabular}{|l|r|r|}
    \cline{2-3}
    \multicolumn{1}{l|}{} & \multicolumn{1}{|c|}{Points} & \multicolumn{1}{|c|}{Traitées} \\
    \hline
    {Preuve de correction} & 100\% \;{\small (15/15)} & 100\% \;{\small (1/1)} \\ \hline {Types structurés en C et pointeurs} & 0\% \;{\small (00/15)} & 0\% \;{\small (0/1)} \\ \hline {Comprendre un algorithme} & 100\% \;{\small (05/5)} & 100\% \;{\small (1/1)} \\ \hline {Programmation de base en C} & 21\% \;{\small (15/70)} & 50\% \;{\small (4/8)} \\ \hline {Représentation des entiers} & 44\% \;{\small (22/50)} & 75\% \;{\small (6/8)} \\ \hline {Base d'OCaml (fonctionnel)} & 71\% \;{\small (61/85)} & 100\% \;{\small (11/11)} \\ \hline {Preuve de terminaison} & 100\% \;{\small (05/5)} & 100\% \;{\small (1/1)} \\ \hline \end{tabular} \\\\\medskip \\
    \ding{113} \textbf{\sffamily{Résultats par exercice}} \medskip \\
    \renewcommand{\arraystretch}{1.2}
    \begin{tabular}{|l|r|r|}
    \cline{2-3}
    \multicolumn{1}{l|}{} & \multicolumn{1}{|c|}{Points} & \multicolumn{1}{|c|}{Traitées} \\
    \hline
    Exercice {1} & 70\% \;{\small (28/40)} & 100\% \;{\small (5/5)} \\ \hline Exercice {2} & 81\% \;{\small (61/75)} & 100\% \;{\small (10/10)} \\ \hline Exercice {3} & 24\% \;{\small (12/50)} & 60\% \;{\small (3/5)} \\ \hline Exercice {4} & 27\% \;{\small (22/80)} & 54\% \;{\small (6/11)} \\ \hline \end{tabular} \\\\
    \vspace{0.5cm}\\
    \ding{113} \textbf{\sffamily{Historique des notes}} \medskip \\
    \psset{xunit=1.4cm, yunit=0.2cm}
    \begin{pspicture}(-1,-1)(12,22)

% --- Axe X : numéros de devoir ---
\psaxes[
    Dx=1,
    Dy=2,
    Ox=0,
    Oy=0,
    labels=all,
    ticks=all,
](0,0)(0,0)(7,20)


\listplot[plotstyle=line,showpoints=true,linecolor=black,linewidth=0.7pt, dotstyle=diamond*, dotsize=0.2]{%
    1 9.5
2 10
}

% Minimum de la classe
\listplot[plotstyle=line,showpoints=true,linecolor=gray,linewidth=0.7pt, dotstyle=|, dotangle=90, dotsize=0.15, linestyle=dotted]{%
    1 5.8
2 4.6
}

% Maximum de la classe
\listplot[plotstyle=line,showpoints=true,linecolor=gray,linewidth=0.7pt, dotstyle=|, dotangle=90, dotsize=0.15, linestyle=dotted]{%
    1 18.9
2 18.4
}

% Moyenne de la classe
\listplot[plotstyle=line,showpoints=true,linecolor=gray,linewidth=0.7pt, dotstyle=x, dotsize = 0.15, linestyle=dashed]{%
    1 12.4466666666667
2 13.1933333333333
}
\end{pspicture}
\pagebreak
\begin{tcolorbox}[enhanced,width=\textwidth,center upper,fontupper=\bfseries,drop shadow southwest,sharp corners]
{\sc \large Silotia} Donovan
\end{tcolorbox}
\medskip
\begin{tabularx}{\textwidth}{p{5cm}X}
	\alertbox{\faAward}{Note}{
		\begin{itemize}[leftmargin=0pt]
			\item[\textbullet] Note : \textbf{\large 15.4}
			\item[\textbullet] Rang : \textbf{7}
			\item[\textbullet] Traité : 97 \%
		\end{itemize}
	} &
	\alertbox{\faChartLine}{Statistiques des notes}{
        \psset{xunit=1cm, yunit=1cm,fillstyle=solid}
		\begin{pspicture}(0,-0.1)(16,1.45)
		   \savedata{\data}[11.9 15.4 18.4 11.8 9.0 11.1 16.1 4.6 8.2 15.0 18.0 15.5 17.5 10.0 15.4]
		   \rput{-90}(0,0.9){\psBoxplot[barwidth=1.1cm,yunit=0.5,fillcolor=gray,linewidth=1pt]{\data}}
		   \psaxes[yAxis=false,dx=1cm,Dx=2,labelsep=1pt,linecolor=gray,xlabelFontSize=\scriptstyle](0,0)(10.1,4)
		   \psdot[dotsize=8pt,dotstyle=diamond,linecolor=black,fillstyle=solid,fillcolor=white,linewidth=1pt](7.7,0.85)
           \psdot[dotsize=6pt,dotstyle=x,linecolor=black,linewidth=3pt](6.596666666666667,0.85)
		   \end{pspicture}
	} \\
    
\end{tabularx}\\
\begin{tabularx}{\textwidth}{X}
\alertbox{\faComment}{Commentaire}
{
	Très bon travail, tu es en progrès, il faut continuer ainsi. Le modèle mémoire du C est encore à travailler, je te conseille de refaire en les codant vraiment les fonctions de l’exercice 3
}
\end{tabularx}
\medskip
    \ding{113} \textbf{\sffamily{Résultats par thème}} \medskip \\
    \renewcommand{\arraystretch}{1.2}
    \begin{tabular}{|l|r|r|}
    \cline{2-3}
    \multicolumn{1}{l|}{} & \multicolumn{1}{|c|}{Points} & \multicolumn{1}{|c|}{Traitées} \\
    \hline
    {Preuve de correction} & 80\% \;{\small (12/15)} & 100\% \;{\small (1/1)} \\ \hline {Types structurés en C et pointeurs} & 80\% \;{\small (12/15)} & 100\% \;{\small (1/1)} \\ \hline {Comprendre un algorithme} & 100\% \;{\small (05/5)} & 100\% \;{\small (1/1)} \\ \hline {Programmation de base en C} & 51\% \;{\small (36/70)} & 87\% \;{\small (7/8)} \\ \hline {Représentation des entiers} & 100\% \;{\small (50/50)} & 100\% \;{\small (8/8)} \\ \hline {Base d'OCaml (fonctionnel)} & 83\% \;{\small (71/85)} & 100\% \;{\small (11/11)} \\ \hline {Preuve de terminaison} & 60\% \;{\small (03/5)} & 100\% \;{\small (1/1)} \\ \hline \end{tabular} \\\\\medskip \\
    \ding{113} \textbf{\sffamily{Résultats par exercice}} \medskip \\
    \renewcommand{\arraystretch}{1.2}
    \begin{tabular}{|l|r|r|}
    \cline{2-3}
    \multicolumn{1}{l|}{} & \multicolumn{1}{|c|}{Points} & \multicolumn{1}{|c|}{Traitées} \\
    \hline
    Exercice {1} & 60\% \;{\small (24/40)} & 100\% \;{\small (5/5)} \\ \hline Exercice {2} & 94\% \;{\small (71/75)} & 100\% \;{\small (10/10)} \\ \hline Exercice {3} & 68\% \;{\small (34/50)} & 100\% \;{\small (5/5)} \\ \hline Exercice {4} & 75\% \;{\small (60/80)} & 90\% \;{\small (10/11)} \\ \hline \end{tabular} \\\\
    \vspace{0.5cm}\\
    \ding{113} \textbf{\sffamily{Historique des notes}} \medskip \\
    \psset{xunit=1.4cm, yunit=0.2cm}
    \begin{pspicture}(-1,-1)(12,22)

% --- Axe X : numéros de devoir ---
\psaxes[
    Dx=1,
    Dy=2,
    Ox=0,
    Oy=0,
    labels=all,
    ticks=all,
](0,0)(0,0)(7,20)


\listplot[plotstyle=line,showpoints=true,linecolor=black,linewidth=0.7pt, dotstyle=diamond*, dotsize=0.2]{%
    1 11.6
2 15.4
}

% Minimum de la classe
\listplot[plotstyle=line,showpoints=true,linecolor=gray,linewidth=0.7pt, dotstyle=|, dotangle=90, dotsize=0.15, linestyle=dotted]{%
    1 5.8
2 4.6
}

% Maximum de la classe
\listplot[plotstyle=line,showpoints=true,linecolor=gray,linewidth=0.7pt, dotstyle=|, dotangle=90, dotsize=0.15, linestyle=dotted]{%
    1 18.9
2 18.4
}

% Moyenne de la classe
\listplot[plotstyle=line,showpoints=true,linecolor=gray,linewidth=0.7pt, dotstyle=x, dotsize = 0.15, linestyle=dashed]{%
    1 12.4466666666667
2 13.1933333333333
}
\end{pspicture}
\pagebreak\end{document}
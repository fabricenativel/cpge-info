\PassOptionsToPackage{dvipsnames,table}{xcolor}
\documentclass[10pt]{beamer}
\usepackage{Cours}

\begin{document}

\input{\detokenize{/home/fenarius/Travail/Cours/cpge-info/latex//MacrosCours.tex}}

% Numéro et titre de chapitre
\setcounter{numchap}{0}
\newcommand{\Ctitle}{\cnum Un peu de Python}

\makess{Généralités}
\begin{frame}{\Ctitle}{\stitle}
	\begin{alertblock}{Bonnes pratiques de programmation}
		\begin{itemize}
			\item<1-> Les \textit{commentaires} s'écrivent en faisant commencer la ligne par le caractères \kw{\#}
			\item<2-> Les noms de variables et de fonction doivent être explicites. 
			\item<3-> L'instruction \kw{assert <condition>} permet de vérifier que \kw{<condition>} est vérifiée avant de continuer l'exécution du programme. On peut ainsi tester des fonctions ou vérifier des \textit{préconditions} sur  des arguments.
		\end{itemize}
	\end{alertblock}
\end{frame}

\begin{frame}[fragile]{\Ctitle}{\stitle}
	\begin{alertblock}{Utilisation de librairies}
		\begin{itemize}
			\item<1-> On peut importer la totalité de la librairie \kw{<lib>} à l'aide de \mintinline{python}{import <lib>}. Dans ce cas les fonctions de cette librairie doivent être utilisées en les faisant précéder du nom de la librairie
			\item<2-> Cet import peut se faire en donnant un \textit{alias} : \mintinline{python}{import <lib> as <alias>}
			\item<3-> Pour importer simple la fonction \kw{<fonc>} de la librairie \kw{<lib>}, on utilise \mintinline{python}{from <lib> import <fonc>}. Le nom de la fonction est alors utilisé directement.
		\end{itemize}
	\end{alertblock}
	\begin{exampleblock}{Exemple}
			\onslide<4->\begin{codepython}
import random
de = random.randint(1,6)
\end{codepython}
			\onslide<5->\begin{codepython}
from random import randint
de = randint(1,6)
\end{codepython}
	\end{exampleblock}
\end{frame}


% Types de bases
\makess{Types de base}
\begin{frame}{\Ctitle}{\stitle}
	\begin{alertblock}{Types de base}
		\begin{tabularx}{\linewidth}{|l|c|>{\footnotesize}X|}
			\hline
			Type & Opérations & Commentaires \\
			\hline
			\kw{int} & \kw{+}, \kw{-}, \kw{*}, \kw{//}, \kw{\%} & Entiers signés ou non signés. Taille dynamique limitée par la mémoire\\
			\hline
			& & \  \newline \ \newline	\\
			\hline
			&&  \  \newline \  \\
			\hline
		\end{tabularx}
		\vspace{1cm}
	\end{alertblock}
\end{frame}

% Types de bases
\begin{frame}{\Ctitle}{\stitle}
	\begin{alertblock}{Types de base}
		\begin{tabularx}{\linewidth}{|l|c|>{\footnotesize}X|}
			\hline
			Type & Opérations & Commentaires \\
			\hline
			\kw{int} & \kw{+}, \kw{-}, \kw{*}, \kw{//}, \kw{\%} & Entiers signés ou non signés. Taille dynamique  limitée par la mémoire\\ 
			\hline
			\kw{float} & \kw{+}, \kw{-}, \kw{*}, \kw{/}, \kw{**} & 	Représentation des nombres en virgule flottante (norme ieee754 : mantisse sur 53 bits, exposant sur 11 bits). Fonctions élémentaires dans \kw{math.h}\\
			\hline
			&&  \  \newline \  \\
			\hline
		\end{tabularx}
		\vspace{1cm}
	\end{alertblock}
\end{frame}

% Types de bases
\begin{frame}{\Ctitle}{\stitle}
	\begin{alertblock}{Types de base}
		\begin{tabularx}{\linewidth}{|l|c|>{\footnotesize}X|}
			\hline
			Type & Opérations & Commentaires \\
			\hline
			\kw{int} & \kw{+}, \kw{-}, \kw{*}, \kw{//}, \kw{\%} & Entiers signés ou non signés. Taille dynamique limitée par la mémoire\\ 
			\hline
			\kw{float} & \kw{+}, \kw{-}, \kw{*}, \kw{/} & 	Représentation des nombres en virgule flottante (norme ieee754 : mantisse sur 53 bits, exposant sur 11 bits). Fonctions élémentaires dans \kw{math}\\
			\hline
			\kw{bool} & \kw{or}  \kw{and}, \kw{not}, \textcolor{gray}{\tt all}, \textcolor{gray}{\tt any}  &  Evaluations paresseuses des expressions. \\
			\hline
		\end{tabularx}
		\vspace{1cm}
	\end{alertblock}
\end{frame}



% Fonction
\makess{Fonctions}
\begin{frame}[fragile]{\Ctitle}{\stitle}
	\begin{alertblock}{Définir une fonction en Python}
		Pour définir une fonction en Python :
		\begin{itemize}
		\item<2-> qui ne renvoie pas de valeur :\begin{codepython}
def <nom_fonction>(<arguments>):
	<instruction>
		\end{codepython}
		\item<2-> qui renvoie une valeur : \begin{codepython}
def <nom_fonction>(<arguments>):
	<instruction>
	return <resultat>
	\end{codepython}
\end{itemize}
	\end{alertblock}
\end{frame}

% Instructions conditionnelles
\makess{Instructions conditionnelles}
\begin{frame}[fragile]{\Ctitle}{\stitle}
	\begin{alertblock}{Instructions conditionnelles}
		\begin{itemize}
			\item<1-> Sans clause \kw{else}
			\begin{codepython}
if <condition>:
	<instructions>
			\end{codepython}
			Exécute les {\tt <instructions>} si la {\tt condition} est vérifiée.
			\item<2-> Avec clause \kw{else}
			\begin{codepython}
if <condition>:
	<instructions1>
else:
	<instructions2>
			\end{codepython}
		Cela permet d'exécuter les {\tt <instructions1>} si la {\tt condition} est vérifiée, sinon on exécute les {\tt <instructions2>}.
		\end{itemize}
	\end{alertblock}
\end{frame}

\begin{frame}{\Ctitle}{\stitle}
\begin{alertblock}{Opérateurs de comparaison}
	\begin{itemize}
		\item<1-> L'égalité se teste avec \kw{==}
		\item<2-> La différence avec \kw{!=}
		\item<3-> Plus grand ou égal avec \kw{>=}, plus petit ou égal avec \kw{<=}
		\item<4-> Plus grand strictement avec \kw{>}, plus petit strictement avec \kw{<}
	\end{itemize}
\end{alertblock}
\end{frame}

% boucle while
\makess{Boucles}
\begin{frame}[fragile]{\Ctitle}{\stitle}
	\begin{alertblock}{Boucles {\tt while}}
		\begin{itemize}
			\item<2-> La syntaxe d'une boucle \textcolor{red}{\tt while}  en Python est :
				\begin{codepython}
while <condition>:
	<instruction>
			\end{codepython}
			      Cela permet d'exécuter les {\tt <instructions>} tant que la {\tt <condition>} est  vérifiée.
			\item<3-> L'instruction \kw{break} permet de sortir de la boucle de façon anticipée.
 			\item<4->  On ne sait pas a priori combien de fois cette boucle sera exécutée (et elle peut même être infinie), on dit que c'est une boucle \textcolor{blue}{non bornée}.
		\end{itemize}
	\end{alertblock}
\end{frame}

% boucle for
\begin{frame}[fragile]{\Ctitle}{\stitle}
	\begin{alertblock}{Boucles {\tt for}}
		\begin{itemize}
			\item<2-> Les instructions :
			      \begin{codepython}
	for <variable> in range(<entier>):
		 <instructions>
	\end{codepython}
			      créent une variable parcourant les entiers de 0 à {\tt <entier>} (exclu).
			\item<3-> Les {\tt <instructions>} indentées qui suivent seront exécutées pour chaque valeur prise par la variable.
			\item<4-> L'instruction \kw{break} permet de sortir de la boucle de façon anticipée.
			\item<5-> La boucle {\tt for} permet donc de répéter un nombre prédéfini de fois des instructions, on dit que c'est une boucle bornée.
		\end{itemize}
	\end{alertblock}
\end{frame}


\begin{frame}[fragile]{\Ctitle}{\stitle}
	\begin{exampleblock}{Exemple 1}
		Ecrire et tester une fonction {\tt syracuse} qui prend en argument un entier naturel $n$ et renvoie $n/2$ si $n$ est pair et $3n+1$ sinon. \\
		\onslide<2->\inputpython{/home/fenarius/Travail/Cours/cpge-info/docs/itc/files/C0/syracuse.py}{}{}
	\end{exampleblock}
\end{frame}

\begin{frame}[fragile]{\Ctitle}{\stitle}
	\begin{exampleblock}{Exemple 2}
		Ecrire une fonction {\tt serie\_harmonique} qui prend en argument un entier $n$ et renvoie la somme $\displaystyle{\sum_{k=1}^n \frac{1}{k}}$
		\onslide<2->\inputpython{/home/fenarius/Travail/Cours/cpge-info/docs/itc/files/C0/harmo.py}{}{}
	\end{exampleblock}
\end{frame}


\begin{frame}[fragile]{\Ctitle}{\stitle}
	\begin{exampleblock}{Exemple 3}
		Ecrire  une fonction {\tt pgcd} qui prend en argument deux entiers naturels $a$ et $b$ et renvoie leur {\sc pgcd}. \\
		\onslide<2->{\small \textcolor{OliveGreen}{\aide} on rappelle que l'algorithme consiste --tant que $b$ n'est pas nul- à effectuer la division euclidienne de $a$ par $b$. En remplaçant à chaque étape $a$ par $b$ et $b$ par $r$.}
		\begin{itemize}
			\item \onslide<3->{Version 1 : }\onslide<5->{\textcolor{BrickRed}{iterative}}
		\onslide<3->{\inputpython{/home/fenarius/Travail/Cours/cpge-info/docs/itc/files/C0/pgcd.py}{}{\small}}
			\item \onslide<4->{Version 2 : }\onslide<5->{\textcolor{BrickRed}{récursive}}
		\onslide<4->{\inputpython{/home/fenarius/Travail/Cours/cpge-info/docs/itc/files/C0/pgcd_rec.py}{}{\small}}
		\end{itemize}
	\end{exampleblock}
\end{frame}


% Définition des listes
\makess{Les listes}
\begin{frame}[fragile]{\Ctitle}{\stitle}
	\begin{center}
		\begin{alertblock}{Les listes de Python}
			\begin{itemize}
				\item<1-> Les listes de Python sont des structures contenant  zéro, une ou plusieurs valeurs (pas forcément du mêmte type).
				\item<2-> Une liste se note entre crochets : \kw{[} et \kw{]}
				\item<3-> Les éléments sont séparés par des virgules
				\item<4-> Les éléments d'une liste sont repérés par leur position dans la liste, on dit leur \textcolor{blue}{indice}. Attention, la numérotation commence à zéro.
				\item<6-> On peut accéder à un élément en indiquant le nom de la liste puis  l'indice de cet élément entre crochet
				\item<7-> L'erreur {\tt IndexError} indique qu'on tente d'accéder à un indice qui n'existe pas.
				\item<8-> La longueur d'une liste (ie. son nombre d'éléments) s'obtient à l'aide de la fonction \kw{len}.
			\end{itemize}
		\end{alertblock}
	\end{center}
\end{frame}


	% Manipulation des listes
	\begin{frame}[fragile]{\Ctitle}{\stitle}
		\begin{alertblock}{Opérations sur les listes}
			Les opérations suivantes permettent de manipuler les listes (ajout, suppression, insertion d'éléments). On fera bien attention à la syntaxe on met le nom de la liste suivi d'un point suivi de l'opération à effectuer (voir exemples)
			\begin{itemize}
				\item<1-> \textcolor{blue}{\tt append} : permet d'ajouter un élément à la fin d'une liste. Par exemple : {\tt ma\_liste.append(elt)} va ajouter {\tt elt} à la fin de {\tt ma\_liste}.
				\item<2-> \textcolor{blue}{\tt pop} permet de récupérer un élement de la liste tout en le supprimant de la liste. Par exemple {\tt elt=ma\_liste.pop(2)} va mettre dans {\tt elt} {\tt ma\_liste[2]} et dans le même temps supprimer cet élément de la liste. \\
				\onslide<3->{\textcolor{BrickRed}{\important} On utilisera le plus souvent \kw{pop} sans argument, dans ce cas c'est le dernier élément de la liste qui est supprimé}
			\end{itemize}
		\end{alertblock}
	\end{frame}
	
\begin{frame}[fragile]{\Ctitle}{\stitle}
\begin{alertblock}{\textcolor{yellow}{\small \important} Spécificité de Python}
	Les listes de Python sont \textcolor{blue}{mutables}, c'est à dire que les modifications faites sur une liste passée en argument à une fonction sont effectivement réalisées sur la liste. Ce n'est \textcolor{blue}{pas} le cas sur les arguments de type entier ou flottants.
\end{alertblock}
\end{frame}


\begin{frame}[fragile]{\Ctitle}{\stitle}
	\begin{exampleblock}{Exemples}
	\begin{itemize}
	\item<1-> Ce programme affiche 42 car {\tt n} étant de type entier  l'opération effectuée sur {\tt n} ne se répercute pas sur l'argument de la fonction.
	\begin{codepython*}{fontsize=\small}
def carre(n):
	n = n * n

n = 42
carre(n)
print(n)
	\end{codepython*}
	\item<2-> Ce programme modifie la liste passée en argument et donc affichera {\tt [5,7]}
	\begin{codepython*}{fontsize=\small}
def ajoute(liste,valeur):
	liste.append(valeur)

liste = [5]
liste.ajoute(7)
print(liste)
			\end{codepython*}
	\end{itemize}
\end{exampleblock}
\end{frame}


% Génération de listes
\begin{frame}[fragile]{\Ctitle}{\stitle}
	\begin{alertblock}{Création de listes}
		On peut créer des listes de diverses façons en Python :
		\begin{itemize}
			\item<2-> \textcolor{red}{Par ajout succesif d'élement} on part alors d'une liste (éventuellement vide) et on ajoute chaque élément à l'aide d'instruction \textcolor{blue}{\tt append}.
			\item<3-> \textcolor{red}{Par répétition du même élément} on utilise alors le caractère \textcolor{blue}{\tt *} pour indiquer le nombre de répétitions. \\
			      \onslide<4-> {Par exemple : \textcolor{blue}{\tt hesitation = ["euh"]*4}}
			\item<6->	 \textcolor{red}{Par compréhension}, c'est à dire en indiquant la définition des éléments qui composent la liste. \\
			      \onslide<7-> {Par exemple la liste {\tt puissances2 = [1, 2, 4, 8, 16, 32, 64, 128]} est constitué des huits premières puissances de 2} \\
			      \onslide<8-> {Elle contient donc $2^0, 2^1, 2^2, \dots 2^7$, ce qui se traduit en Python par :}\\
			      \onslide<9-> \textcolor{blue}{\tt puissances2 = [2**k for k in range(8)]}
		\end{itemize}
	\end{alertblock}
\end{frame}

%Tranches
\begin{frame}[fragile]{\Ctitle}{\stitle}
	\begin{alertblock}{Tranches (\textit{slices})}
		\begin{itemize}
		\item<1->On peut extraire une tranche d'une liste en donnant entre crochets l'indice du premier élément puis l'indice du dernier (qui sera exclu) séparé par un \kw{:}.\\
		\onslide<2->\textcolor{gray}{\small Par exemple si la liste est {\tt l=[2,3,5,7,11,13,17,19]}} alors {\tt l[2:4]} est une liste qui contient {\tt [5,7]}.
		\item<3-> Si l'indice du premier est omis alors la tranche commmence à l'indice 0.\\
		\onslide<4->\textcolor{gray}{\small Avec la même liste {\tt l}, on a {\tt l[:5]} est une liste qui contient {\tt [2,3,5,7,11]}.}
		\item<5-> Si l'indice du dernier est omis alors la tranche va jusqu'à la fin de la liste.\\
		\onslide<6->\textcolor{gray}{\small Avec la même liste {\tt l}, on a {\tt l[7:]} est une liste qui contient {\tt [19]}.}
		\end{itemize}
	\end{alertblock}
\end{frame}

\makess{Chaine de caractères et tuples}
\begin{frame}[fragile]{\Ctitle}{\stitle}
	\begin{alertblock}{Tuples}
		\begin{itemize}
			\item<1-> Les \textcolor{blue}{tuples} sont le pendant non mutables des listes. Ils se notent entre parenthèses \kw{(} et \kw{)}, les éléments sont aussi séparés par des virgules.
			\item<2-> De même que pour les listes, on peut accéder à la longueur avec \kw{len}, aux éléments avec la notation crochet et le parcours avec une boucle \kw{for} est aussi possible.
			\item<3-> La modification par contre n'est pas possible
		\end{itemize}
	\end{alertblock}
	\begin{exampleblock}{Exemple}
		\begin{codepython*}{fontsize=\small}
			anniv = (31,"Janvier",1956) 
			print("Mois de naissance = ",anniv[1])
			anniv[2] = 1970 #provoque une erreur
		\end{codepython*}
	\end{exampleblock}
\end{frame}

\begin{frame}[fragile]{\Ctitle}{\stitle}
	\begin{alertblock}{Chaines de caractères}
		\begin{itemize}
		\item<1-> La notation avec les crochets permettant d'accéder aux éléments d'une liste s'utilise aussi avec les chaines de caractères. \\
			      \onslide<2-> Par exemple si \mintinline{python}{mot = "Génial"} alors \mintinline{python}{mot[2]} contient la lettre \mintinline{python}{"n"}
		\item<3-> Le parcours par élément peut aussi se faire sur une chaine de caractères. \\
			      \onslide<4-> Pour afficher chaque lettre du mot "Génial", on peut donc écrire :
			      \onslide<5->\begin{codepython}
for lettre in mot:
	print(lettre)
				  \end{codepython}
		\item<6-> Comme les tuples, les chaines de caractères sont non mutables.
		\item<7-> \textcolor{blue}{\small \important} Les variables lues au clavier (instruction \kw{input}) ou issus de la lecture d'un fichier sont des chaines de caractères. On doit les convertir dans le type approprié pour les utiliser comme nombre.
		\item<8-> La fonction \kw{split} permet de renvoyer une liste de sous chaines en utilisant le séparateur donné en argument.
		\end{itemize}
	\end{alertblock}
\end{frame}

\begin{frame}[fragile]{\Ctitle}{\stitle}
	\begin{exampleblock}{Exemple}
		Ecrire une fonction {\tt check\_date} qui prend en argument une chaine de caractères et renvoie \kw{True} si cette chaine est une date valide au format JJ/MM/AAAA et \kw{False} sinon. Pour simplifier on testera simplement que le jour est entre 1 et 31 et le mois entre 1 et 12.
		\onslide<2->{\inputpython{/home/fenarius/Travail/Cours/cpge-info/docs/itc/files/C0/check_date.py}{}{\small}}
	\end{exampleblock}
\end{frame}

% Parcours d'une liste
\begin{frame}{\Ctitle}{\stitle}
	\begin{alertblock}{Parcours d'une liste}
		On rappelle qu'une liste \textcolor{blue}{\tt L}, en Python peut se représenter par le schéma suivant : \\
		\begin{tabularx}{0.8\textwidth}{lc|Y|Y|Y|Y|Y|}
			\cline{3-7}
			\textcolor{blue}{Eléments} &\textcolor{blue}{$\blacktriangleright $}                 & {\tt L[0]}                     & {\tt L[1]}                     & {\tt L[2]}                     & {\tt L[3]}                     & {\dots}                        \\
			\cline{3-7}
			\multicolumn{1}{c}{ }       & \multicolumn{1}{c}{ } & \multicolumn{1}{c}{$\uparrow$} & \multicolumn{1}{c}{$\uparrow$} & \multicolumn{1}{c}{$\uparrow$} & \multicolumn{1}{c}{$\uparrow$} \\
			\multicolumn{1}{c}{\textcolor{blue}{Indices}} & \multicolumn{1}{c}{\textcolor{blue}{$\blacktriangleright $}} & \multicolumn{1}{c}{0}          & \multicolumn{1}{c}{1}          & \multicolumn{1}{c}{2}          & \multicolumn{1}{c}{3}          & \multicolumn{1}{c}{\dots}      \\
		\end{tabularx} \\
		On peut parcourir cette liste :
		\begin{itemize}
			\item<2-> \textcolor{red}{Par indice} (on se place sur la seconde ligne du schéma ci-dessus) et on crée une variable (un entier) qui va parcourir la liste des indices : \\
			      \textcolor{blue}{\tt for indice in range(len(L))} \\
			      Il faut alors accéder aux éléments en utilisant leurs indices.
			\item <3->\textcolor{red}{Par élément} (on se place sur la première ligne du schéma ci-dessus) et on crée une variable qui va parcourir directement la liste des éléments : \\
			      \textcolor{blue}{\tt for element in L} \\
			      La variable de parcours (ici {\tt element}) contient alors directement les éléments).
		\end{itemize}
	\end{alertblock}
\end{frame}

\begin{frame}[fragile]{\Ctitle}{\stitle}
	\begin{exampleblock}{Exemple 1}
		Ecrire une fonction {\tt est\_dans} qui prend en argument un entier {\tt n}  et une liste d'entiers {\tt l}  et renvoie {\tt True} si {\tt n} est dans {\tt l} et {\tt False} sinon. On écrira une version utilisant un parcours par valeur et une version utilisant un parcours par indice.
		\begin{itemize}
		\item<2-> Parcours par élément :
		\onslide<2->{\inputpython{/home/fenarius/Travail/Cours/cpge-info/docs/itc/files/C0/est_dans.py}{}{\small}}
		\item<3-> Parcours par indice :
		\onslide<2->{\inputpython{/home/fenarius/Travail/Cours/cpge-info/docs/itc/files/C0/est_dans_ind.py}{}{\small}}
		\end{itemize}
	\end{exampleblock}
\end{frame}

\begin{frame}[fragile]{\Ctitle}{\stitle}
	\begin{exampleblock}{Exemple 2}
		Ecrire une fonction {\tt max\_liste} qui prend en argument une liste non vide d'entiers {\tt l}  et renvoie le maximum des éléments de cette liste
		\onslide<2->{\inputpython{/home/fenarius/Travail/Cours/cpge-info/docs/itc/files/C0/max_liste.py}{}{\small}}
	\end{exampleblock}
\end{frame}



% Descripteur de fichiers
\makess{fichier}
\begin{frame}[fragile]{\Ctitle}{\stitle}
	\begin{alertblock}{Gestions des fichiers en Python}
		En python, on peut ouvrir un fichier présent sur l'ordinateur à l'aide de l'instruction \textcolor{blue}{\tt open}. Cette instruction renvoie une variable appelée \textcolor{blue}{descripteur de fichier} et prend un paramètre indiquant le mode d'ouverture du fichier :
		\begin{itemize}
			\item<2-> \textcolor{red}{"r"} (read) pour ouvrir le fichier en lecture. C'est le mode par défaut.
			\item<3-> \textcolor{red}{"w"} (write) pour ouvrir le fichier en écriture. Attention, le contenu initial du fichier est alors perdu.
			\item<4-> \textcolor{red}{"a"} (append) pour ouvrir le fichier en ajout.
		\end{itemize}
	\end{alertblock}
\end{frame}

\begin{frame}[fragile]{\Ctitle}{\stitle}
	\begin{alertblock}{Opérations sur les descripteurs de fichiers}
		Les opérations suivantes sont possibles sur un descripteur de fichier crée à l'aide de l'instruction {\tt open} :
		\begin{itemize}
			\item<2-> Lecture du contenu complet du fichier avec \textcolor{blue}{\tt read}
			\item<3-> Lecture du contenu ligne par ligne avec \textcolor{blue}{\tt readline}
			\item<4-> Ecriture avec de \textcolor{blue}{\tt write}
			\item<5-> Fermeture avec \textcolor{blue}{\tt close}
		\end{itemize}
	\end{alertblock}
	\begin{exampleblock}{Exemples}
		\onslide<7->{Ouvrir le fichier "truc.txt", lire sa première ligne puis le refermer. \\}
		\onslide<8->{\textcolor{OliveGreen}{\tt fic = open("truc.txt","r")}\\}
		\onslide<9->{\textcolor{OliveGreen}{\tt lig1 = fic.readline()}\\}
		\onslide<10->{\textcolor{OliveGreen}{\tt fic.close()}\\}
	\end{exampleblock}
\end{frame}

\makess{Exercice de synthèse}
\begin{frame}[fragile]{\Ctitle}{\stitle}
	\begin{exampleblock}{Volume maximum d'une boîte}
		Sur chaque ligne d'un fichier {\tt boites.txt} on trouve séparé par des espaces, un code à 4 lettres représentant la référence du modèle de boîte puis 3 entiers représentant les dimensions de la boîte.
		Par exemple, les premières lignes du fichiers sont :
		\begin{verbatim}
			IXRP 136 214 287
			YCEU 33 193 122
			NYFV 138 54 197
		\end{verbatim}
		\onslide<2->{Ecrire un programme python permettant de déterminer la boîte ayant le plus grand volume et donner sa référence.}
	\end{exampleblock}
\end{frame}


\end{document}

\documentclass[11pt,a4paper]{article}

\usepackage{Act}


\begin{document}
\input{\detokenize{/home/fenarius/Travail/Cours/cpge-info/latex/Macros.tex}}

% Entête de la fiche (à modifier avec la macro correspondante dans le fichier macros)
\Fiche{Bases de Python}{\sc pcsi}

\pythonmode


\begin{tcolorbox}[left=0cm,title=\bf{\faPython \; Types de bases et opérateurs associés},colbacktitle=cfond]
\begin{itemize}
 \item [\textbullet] \mintinline{python}{int} : nombres entiers de taille \textit{non limitée} \strut \\
 \begin{tabularx}{\linewidth}{|>{\tt}l|l|X|}
	\hline
	+ & addition &  \\
	\hline
	- & soustraction &  \\
	\hline
	* & multiplication & \\
	\hline
	/ & division \textit{décimale} &  ex : {\tt 13/5} vaut {\tt 2.6}. \\
	\hline
	** & exponentiation & ex : {\tt 3**4} vaut 81 \\
	\hline
	// & quotient dans la division euclidienne & ex : {\tt 13//5} vaut 2. \\
	\hline
	\% & reste dans la division euclidienne & ex : {\tt 13\%5} vaut 3. Tests de divisibilité. \\
	\hline
 \end{tabularx}
 \item [\textbullet] \mintinline{python}{float} : nombres en virgule flottante de taille limitée \\
 S'écrivent toujours avec le séparateur décimal \textbf{\tt .}, mêmes opérateurs que sur les entiers (à l'exception de {\tt //} et {\tt \%}).
 \item [\textbullet] \mintinline{python}{bool} : deux valeurs possibles {\tt True} et {\tt False} \\
 \begin{tabularx}{\linewidth}{|>{\tt}l|l|X|}
	\hline
	not & négation (unaire) & inverse la valeur de l'argument.\\
	\hline
	or & ou (binaire) & vaut {\tt True} si au moins un des arguments vaut {\tt True}. \\
	\hline
	and & et (binaire) & vaut {\tt True} si les deux arguments valent {\tt True}. \\
	\hline
 \end{tabularx}
	\item [\textbullet] \mintinline{python}{str} : chaines de caractères \\
	\begin{tabularx}{\linewidth}{|>{\tt}l|l|X|}
		\hline
		+ & concaténation & ex : {\tt "Hello"+"World"} vaut {\tt "HelloWorld"}\\
		\hline
		* & répétition & ex : {\tt "Euh"*4} vaut {\tt "EuhEuhEuhEuh"} \\
		\hline
		len & longueur & ex : {\tt len("Python")} vaut 6 \\
		\hline
		[] & ième caractères & numérotation depuis 0, ex : {\tt "Hello"[1]} vaut {\tt "e"}\\
		\hline
	 \end{tabularx}
\end{itemize}
Des conversions sont possibles entre ces divers types, par exemple, \mintinline{python}{int("34")} transforme la chaine de caractères {\tt "34"} en l'entier {\tt 34}, \mintinline{python}{float(34)} transforme l'entier 34 en flottant {\tt 34.0}.
\end{tcolorbox}

\begin{tcolorbox}[left=0cm,title=\bf{\faPython \; Comparaison},colbacktitle=cfond]
	\begin{tabularx}{\linewidth}{|>{\tt}l|l|X|}
		\hline
		> & strictement supérieur & \\
		\hline
		< & strictement inférieur & \\
		\hline
		>= & supérieur ou égal & les symboles sont dans l'ordre de leur lecture\\
		\hline
		<= & inférieur ou égal & \\
		\hline
		== & égal & {\small \danger} à ne pas confondre avec {\tt =} utilisé pour l'affectation des variables \\
		\hline
		!= & différent & \\
		\hline
	\end{tabularx}
\end{tcolorbox}

\begin{tcolorbox}[left=0cm,title=\bf{\faPython \; Instructions conditionnelles},colbacktitle=cfond]
	\begin{itemize}
		\item[\textbullet] Pour exécuter des {\tt <instructions>} si une {\tt condition} est vérifiée:
	\setlength{\multicolsep}{0pt}
	\begin{multicols}{2}
	Syntaxe :
	\begin{python} 
if <condition>:
	<instructions>
	\end{python} 
	Exemple : 
	\begin{python} 
if a!=0:
	b = 1/a
	\end{python} 
\end{multicols}
\item[\textbullet] Pour exécuter {\tt <instructions1>} si une {\tt condition} est vérifiée et {\tt <instructions2>} sinon:
\begin{multicols}{2}
	Syntaxe :
	\begin{python} 
if <condition>:
	<instructions1>
else:
	<instructions2>
	\end{python} 
	Exemple : 
	\begin{python} 
if x < y:
	minimum = x
else:
	minimum = y
	\end{python} 
\end{multicols}
\end{itemize}
Pour imbriquer plusieurs {\tt if ... else}, on utilise {\tt elif}.
\end{tcolorbox}

\begin{tcolorbox}[left=0cm,title=\bf{\faPython \; Boucles {\tt while}},colbacktitle=cfond]
	Les boucles {\tt while} répètent un bloc d'instruction tant qu'une condition est vraie.
	\setlength{\multicolsep}{0pt}
	\begin{multicols}{2}
		Syntaxe :
		\begin{python} 
while <condition>:
	<instructions>
		\end{python} 
		Exemple : 
		\begin{python}
rep = "" 
while rep!='O' and rep!='N':
	rep = input("O/N ?")
		\end{python}
	\end{multicols}
\end{tcolorbox}

\begin{tcolorbox}[left=0cm,title=\bf{\faPython \; Boucles {\tt for} avec {\tt range}},colbacktitle=cfond]
\begin{itemize}
	\item[\textbullet] L'instruction {\tt range} génère des entiers et peut prendre un, deux ou trois arguments :
	\begin{tabularx}{\linewidth}{|>{\tt}l|X|}
		\hline
		range(n) & génère les {\tt n} entiers de l'intervalle $\intN{0}{n-1}$ \\
		\hline
		range(m,n) & génère les entiers de l'intervalle $\intN{m}{n-1}$ \\
		\hline
		range(m,n,s) & génère les entiers de l'intervalle $\intN{m}{n-1} \cap \{m+ks, s \in \N\}$ \\
		\hline
	\end{tabularx}
	{\small \danger \;} Dans les trois cas, la valeur {\tt n} n'est pas prise.
	\item[\textbullet] L'instruction {\tt range} peut s'utiliser conjointement avec une boucle {\tt for} pour créer une variable qui prendra les valeurs générées par le {\tt range}, le bloc d'{\tt <instructions>} qui suit est alors exécuté pour chaque valeur de la variable.
	\setlength{\multicolsep}{0pt}
	\begin{multicols}{2}
		Syntaxe :
		\begin{python} 
for <variable> in range(...):
	<instructions1>
		\end{python} 
		Exemple : 
		\begin{python} 
for i in range(1,9):
	print(i) # va afficher 1, 2,... 8
		\end{python}
	\end{multicols}
Une boucle {\tt for} permet donc en particulier de \textit{répéter} un nombre donné de fois des instructions.
\end{itemize}
\end{tcolorbox}

\begin{tcolorbox}[left=0cm,title=\bf{\faPython \; Fonctions},colbacktitle=cfond]
\begin{itemize}
\item[\textbullet] Les fonctions sont des blocs d'instructions réutilisables (chaque appel à la fonction exécute son bloc d'instruction), leur définition commence par le mot clé {\tt def} puis ont écrit le nom de la fonction puis la liste de ses arguments entre parenthèses (séparés par des virgules).
\item[\textbullet] Une fonction peut prendre zéro, un ou plusieurs arguments.
\item[\textbullet] Une fonction \textit{peut} renvoyer un résultat à l'aide d'une instruction {\tt return}.
\item[\textbullet] Exemple : une fonction à 3 arguments et qui renvoie une valeur 
\begin{python}
def discriminant(a,b,c):
	return b**2 - 4*a*c
\end{python}
\item[\textbullet] Exemple : une fonction à un argument et qui ne renvoie rien (elle produit un affichage)
\begin{python}
def triangle(n):
	for i in range(1,n):
		print("*"*i) 
\end{python}
\end{itemize}
{\small \danger} Ne pas confondre  {\tt print} et {\tt return}. La fonction {\tt discriminant} ci-dessus \textit{renvoie} un résultat, donc on pourrait écrire {\tt d = discriminant(1,-11,30)} afin de récupérer dans {\tt d} la valeur calculée. La fonction {\tt triangle} ne renvoie rien, elle produit un affichage, il serait donc illogique d'écrire {\tt t~=~triangle(4)}. Par contre {\tt triangle(4)} fera appel à cette fonction et affichera un triangle de 3 lignes d'étoiles.
\end{tcolorbox}

\end{document}
\PassOptionsToPackage{dvipsnames,table}{xcolor}
\documentclass[10pt]{beamer}
\usepackage{Cours}

\begin{document}

\input{\detokenize{/home/fenarius/Travail/Cours/cpge-info/latex//MacrosCours.tex}}

% Numéro et titre de chapitre
\setcounter{numchap}{9}
\newcommand{\Ctitle}{\cnum Représentation des entiers}
\newcommand{\SPATH}{/home/fenarius/Travail/Cours/cpge-info/docs/itc/files/C\thenumchap}

\makess{Introduction}
\begin{frame}{\Ctitle}{\stitle}
	\begin{block}{Le problème de la représentation des données}
		\begin{itemize}
			\item<1-> La mémoire d'un ordinateur est composé de $bits$ pouvant prendre uniquement les valeurs 0 et de 1
			\item<2-> Le regroupement de 8 bits s'appelle un octet (\textit{byte} en anglais), c'est l'unité minimal de mémoire :
				$$1 \mathrm{\ octet\ }  = \underbrace{\begin{array}{|c|c|c|c|c|c|c|c|} \hline 0& 1 & 0 & 1 & 1 & 0 & 1 & 0\\ \hline \end{array}}_{8 \mathrm{\ bits\ }}$$
			\item<3-> Toutes les données doivent donc être \textcolor{blue}{représentées} en utilisant des octets.
			\item<4-> On s'intéresse ici à la représentation des entiers positifs et négatifs.
		\end{itemize}
	\end{block}
\end{frame}


\makess{Entiers positifs}
\begin{frame}{\Ctitle}{\stitle}
	\begin{block}{De la base 10 à la base 2}
		\begin{itemize}
			\item<1->Nous sommes habitués à écrire en utilisant 10 chiffres (0,1,2,3,4,5,6,7,8 et 9), chaque chiffre étant  multiplié par une puissance de 10 suivant son emplacement dans le nombre.\\
			\onslide<2->{Par exemple, pour \textcolor{blue}{$\base{1815}{10}$} :\\}
			\onslide<3->{\begin{tabular}{p{0.4cm}p{0.4cm}p{0.4cm}p{0.4cm}c}
					                    &                     &                     &                     & \\
					\textcolor{blue}{1} & \textcolor{blue}{8} & \textcolor{blue}{1} & \textcolor{blue}{5} & \\
				\end{tabular}
			}
		\end{itemize}
		\vspace{3.5cm}
	\end{block}
\end{frame}

\begin{frame}{\Ctitle}{\stitle}
	\begin{block}{De la base 10 à la base 2}
		\begin{itemize}
			\item<1-> Nous sommes habitués à écrire les entiers postifs en utilisant \textcolor{red}{10} chiffres, chaque chiffre étant  multiplié par une puissance de \textcolor{red}{10} suivant son emplacement dans le nombre.\\
				Par exemple, pour \textcolor{blue}{$\base{1815}{10}$} :\\
				\begin{tabular}{p{0.4cm}|p{0.4cm}|p{0.4cm}|p{0.4cm}c}
					\textcolor{BrickRed}{$\scriptstyle{10^3}$} & \textcolor{BrickRed}{$\scriptstyle{10^2}$} & \textcolor{BrickRed}{$\scriptstyle{10^1}$} & \textcolor{BrickRed}{$\scriptstyle{10^0}$} &                                                                                                                                                                                                                                               \\
					\cline{1-4}
					\textcolor{blue}{1}                        & \textcolor{blue}{8}                        & \textcolor{blue}{1}                        & \textcolor{blue}{5}                        & \onslide<2->{${=\textcolor{blue}{1} \times \textcolor{BrickRed}{1000} + \textcolor{blue}{8} \times \textcolor{BrickRed}{100} + \textcolor{blue}{1}\times \textcolor{BrickRed}{10}+ \textcolor{blue}{5} \times \textcolor{BrickRed}{1}=1815}$} \\
				\end{tabular}
			\item<3-> De la même façon, on pourrait utiliser simplement \textcolor{red}{2} chiffres et multiplier chaque chiffre par une puissance de \textcolor{red}{2} suivant son emplacement dans le nombre.\\
				\onslide<4-> Par exemple, pour \textcolor{blue}{$\base{11100010111}{2}$} :
				\onslide<4->{
					\begin{tabular}{p{0.4cm}p{0.4cm}p{0.4cm}p{0.4cm}p{0.4cm}p{0.4cm}p{0.4cm}p{0.4cm}p{0.4cm}p{0.4cm}p{0.4cm}c}
						                    &                     &                     &                     &                     &                     &                     &                     &                     &                     &                     & \\
						\textcolor{blue}{1} & \textcolor{blue}{1} & \textcolor{blue}{1} & \textcolor{blue}{0} & \textcolor{blue}{0} & \textcolor{blue}{0} & \textcolor{blue}{1} & \textcolor{blue}{0} & \textcolor{blue}{1} & \textcolor{blue}{1} & \textcolor{blue}{1}   \\
					\end{tabular}}
		\end{itemize}
	\end{block}
\end{frame}


\begin{frame}{\Ctitle}{\stitle}
	\begin{block}{De la base 10 à la base 2}
		\begin{itemize}
			\item Nous sommes habitués à écrire les entiers postifs en utilisant \textcolor{red}{10} chiffres, chaque chiffre étant  multiplié par une puissance de \textcolor{red}{10} suivant son emplacement dans le nombre.\\
			      Par exemple, pour \textcolor{blue}{1815} :\\
			      \begin{tabular}{p{0.4cm}|p{0.4cm}|p{0.4cm}|p{0.4cm}c}
				      \textcolor{BrickRed}{$\scriptstyle{10^3}$} & \textcolor{BrickRed}{$\scriptstyle{10^2}$} & \textcolor{BrickRed}{$\scriptstyle{10^1}$} & \textcolor{BrickRed}{$\scriptstyle{10^0}$} &                                                                                                                                                                                                                                 \\
				      \cline{1-4}
				      \textcolor{blue}{1}                        & \textcolor{blue}{8}                        & \textcolor{blue}{1}                        & \textcolor{blue}{5}                        & ${=\textcolor{blue}{1} \times \textcolor{BrickRed}{1000} + \textcolor{blue}{8} \times \textcolor{BrickRed}{100} + \textcolor{blue}{1}\times \textcolor{BrickRed}{10}+ \textcolor{blue}{5} \times \textcolor{BrickRed}{1}=1815}$ \\
			      \end{tabular}
			\item De la même façon, on pourrait utiliser simplement \textcolor{red}{2} chiffres et multiplier chaque chiffre par une puissance de \textcolor{red}{2} suivant son emplacement dans le nombre.\\
			      Par exemple, pour \textcolor{blue}{$\base{11100010111}{2}$} :
			      \begin{tabular}{p{0.4cm}|p{0.4cm}|p{0.4cm}|p{0.4cm}|p{0.4cm}|p{0.4cm}|p{0.4cm}|p{0.4cm}|p{0.4cm}|p{0.4cm}|p{0.4cm}c}
				      \textcolor{BrickRed}{$\scriptstyle{2^{10}}$} & \textcolor{BrickRed}{$\scriptstyle{2^{9}}$} & \textcolor{BrickRed}{$\scriptstyle{2^{8}}$} & \textcolor{BrickRed}{$\scriptstyle{2^{7}}$} & \textcolor{BrickRed}{$\scriptstyle{2^{6}}$} & \textcolor{BrickRed}{$\scriptstyle{2^{5}}$} & \textcolor{BrickRed}{$\scriptstyle{2^{4}}$} & \textcolor{BrickRed}{$\scriptstyle{2^{3}}$} & \textcolor{BrickRed}{$\scriptstyle{2^{2}}$} & \textcolor{BrickRed}{$\scriptstyle{2^{1}}$} & \textcolor{BrickRed}{$\scriptstyle{2^{0}}$} & \\
				      \hline
				      \textcolor{blue}{1}                          & \textcolor{blue}{1}                         & \textcolor{blue}{1}                         & \textcolor{blue}{0}                         & \textcolor{blue}{0}                         & \textcolor{blue}{0}                         & \textcolor{blue}{1}                         & \textcolor{blue}{0}                         & \textcolor{blue}{1}                         & \textcolor{blue}{1}                         & \textcolor{blue}{1}                           \\
			      \end{tabular}
			      \onslide<2-> \\ \textcolor{BrickRed}{$\scriptstyle{= 2^{10} + 2^9 + 2^8 + 2^4 + 2^2 + 2^1 + 2^0}$}
			      \onslide<3-> \\ \textcolor{BrickRed}{$\scriptstyle{= 1815}$}
		\end{itemize}
	\end{block}
\end{frame}

\begin{frame}{\Ctitle}{\stitle}
	Ce sont des cas particuliers (avec $b=10$ et $b=2$), du théorème suivant  :
	\begin{alertblock}{Décomposition en base $b$}
		Tout entier $n \in \N$ peut s'écrire sous la forme :
		$$ n = \sum_{k=0}^{p} a_k b^k$$
		avec $p \geq 0$ et $a_k \in \intN{0}{b-1}$. De plus, cette écriture est unique si $a_p \neq 0$ et s'appelle \textcolor{blue}{\textit{décomposition en base $b$ de $n$}} et on la note $n = \overline{a_p\dots a_1 a_0}^b$
	\end{alertblock}
\end{frame}


\begin{frame}{\Ctitle}{\stitle}
	\begin{exampleblock}{Exemples}
		Ecrire en base 10 les nombres ci-dessous
		\begin{itemize}
			\item<1-> $\base{10001011}{2}$
				\onslide<6->{\textcolor{OliveGreen}{$=\base{139}{10}$}}
			\item <2-> $\base{1101001011}{2}$
			      \onslide<7->{\textcolor{OliveGreen}{$=\base{843}{10}$} }
			\item<3-> $\base{421}{5}$
				\onslide<8->{\textcolor{OliveGreen}{$=\base{111}{10}$}}
			\item<4-> $\base{3EA}{16}$
				\onslide<9-> {\textcolor{OliveGreen}{$=\base{1002}{10}$}}\\
				\onslide<5->\textcolor{gray}{\small On travaille ici en base 16, donc avec 16 chiffres, les lettres majuscules de $A$ à $F$ représentent les "chiffres" 10 à 15.}  \\
		\end{itemize}
	\end{exampleblock}
\end{frame}

\begin{frame}{\Ctitle}{\stitle}
	\begin{block}{Limitations mémoire et dépassement de capacité}
		\begin{itemize}
			\item<1-> Le nombre de bits représentant un entier est limité, le plus grand nombre représentable sur $n$ bits est : \\
				$\base{1\dots1}{2} = 2^{n-1} + \dots + 1 = 2^{n}-1$
			\item<2-> Certains langages utilisent un nombre défini de bits pour représenter un entier, on peut donc avoir des problèmes de dépassement de capacités (\textit{overflow}).
			\item<3-> Python utilise des entiers dit \textit{multi-précision} dont la taille (en nombre de bits) évolue, on a donc pas le problème de dépassement de capacité.
			\item<4-> Par contre, il devient problématique d'évaluer le temps nécessaire à une opération donnée (par exemple une multiplication) sur ces entiers.
		\end{itemize}
	\end{block}
\end{frame}


\begin{frame}{\Ctitle}{\stitle}
	\begin{alertblock}{Fonction {\tt bin}}
        La fonction \texttt{bin} de Python prend en argument un nombre entier et renvoie la représentation binaire de cet entier sous la forme d'une chaîne de caractères composée de 0 et de 1 et précédée de \texttt{"0b"}.
	\end{alertblock}
    \onslide<2->
        {\begin{exampleblock}{Exemple}
        \begin{itemize}
            \item \texttt{bin(10)} renvoie \texttt{"0b1010"}
            \item \texttt{bin(255)} renvoie \texttt{"0b11111111"}
        \end{itemize}
        \end{exampleblock}}
\end{frame}

\makess{Représentation des entiers négatifs}
\begin{frame}{\Ctitle}{\stitle}
	\begin{block}{Complément à deux}
		\begin{itemize}
			\item<1-> La stratégie qui consiste à prendre \textit{un bit de signe}  et à représenter la valeur absolue de l'entier sur les autres présente deux difficultés : 0 est représenté deux fois et surtout l'addition binaire bit à bit ne fonctionne pas.
			\item<2-> La méthode utilisée est celle du complément à 2, sur $n$ bits, on compte négativement le bit de poids $2^{n-1}$ et positivement les autres. \\
				\onslide<3->{Par exemple, sur 8 bits : $\begin{array}{|c|c|c|c|c|c|c|c|} \hline \textcolor{BrickRed}{1}& 0 & 0 & 1 & 1 & 0 & 1 & 0\\ \hline \end{array} = \textcolor{BrickRed}{-2^7} + 2^4 + 2^3 + 2^1 = -101$}
			\item<3-> De façon générale, sur $n$ bits, la valeur en \textcolor{blue}{complément à deux} de la suite bits $(b_{n-1}\dots b_0)$ est :
				$$-b_{n-1}\,2^{n-1} + \sum_{k=0}^{n-2} b_k\,2^k $$
		\end{itemize}
	\end{block}
\end{frame}

\begin{frame}{\Ctitle}{\stitle}
	\begin{block}{Conséquences de la représentation en complément à 2}
		\begin{itemize}
			\item<1-> Les difficultés de la stratégie du \textit{un bit de signe} sont levées.
			\item<2-> Le plus petit petit représentable sur $n$ bits est alors $-2^{n-1}$ et le plus grand $2^{n-1}-1$
			\item<3-> Comme pour les entiers positifs, il n'y a pas de problème de dépassement de capacité.
		\end{itemize}
	\end{block}
\end{frame}

\begin{frame}{\Ctitle}{\stitle}
	\begin{block}{Méthode pratique}
		Pour obtenir la représentation en complément à deux sur $n$ bits d'un entier négatif on pourra utiliser la méthode suivante :
		\begin{enumerate}
			\item<2-> on commence par écrire la représentation binaire de la valeur absolue de ce nombre
			\item<3-> on inverse tous les bits de cette représentation
			\item<4-> on ajoute 1, sans tenir compte de la dernière retenue éventuelle
		\end{enumerate}
	\end{block}
	\begin{exampleblock}{Exemples}
		\begin{enumerate}
			\item<5-> Quel est le nombre codé en complément à 2 sur 8 bits par $\base{10110001}{2}$ ?
			\item<6-> Donner l'écriture en complément à 2 sur 8 bits de $-12$.
			\item<7-> Donner l'écriture en complément à 2 sur 8 bits de $-75$.
		\end{enumerate}
	\end{exampleblock}
\end{frame}

\begin{frame}{\Ctitle}{\stitle}
	\begin{exampleblock}{Correction}
		\begin{enumerate}
			\item<1->\textcolor{OliveGreen}{En complément à 2 sur 8 bits, $\base{10110001}{2} = -2^{7}+2^5+2^4+2^0 = -78$}
			\item<2->\textcolor{OliveGreen}{Ecriture en complément à 2 sur 8 bits de $-12$.} \\
			      \begin{tabular}{ll}
				      \onslide<2->\textcolor{OliveGreen}{\textbf{1.} On écrit $12=(8+4)$ en binaire sur 8 bits :} & \onslide<3->\textcolor{OliveGreen}{$00001100$} \\
				      \onslide<4->\textcolor{OliveGreen}{\textbf{2.} On inverser tous les bits :}                 & \onslide<5->\textcolor{OliveGreen}{$11110011$} \\
				      \onslide<6->\textcolor{OliveGreen}{\textbf{3.} On ajoute 1 :}                               & \onslide<7->\textcolor{OliveGreen}{$11110100$} \\
			      \end{tabular}
			\item<8->\textcolor{OliveGreen}{Ecriture en complément à 2 sur 8 bits de $-75$.} \\
			      \begin{tabular}{ll}
				      \onslide<9->\textcolor{OliveGreen}{\textbf{1.} On écrit $75=64+8+2+1$ en binaire sur 8 bits :} & \onslide<10->\textcolor{OliveGreen}{$01001011$}  \\
				      \onslide<11->\textcolor{OliveGreen}{\textbf{2.} On inverser tous les bits :}                    & \onslide<12->\textcolor{OliveGreen}{$10110100$} \\
				      \onslide<13->\textcolor{OliveGreen}{\textbf{3.} On ajoute 1 :}                                 & \onslide<14->\textcolor{OliveGreen}{$10110101$} \\
			      \end{tabular}
		\end{enumerate}
	\end{exampleblock}
\end{frame}


%Algorithme des divisions successives
\makess{Algorithme des divisions successives}
\begin{frame}{\Ctitle}{\stitle}
	\begin{block}{Algorithme des divisions successives}
		\begin{itemize}
			\onslide<1->{\item L'algorithme des \textcolor{blue}{divisions successives}, permet d'écrire un nombre donnée en base 10 dans n'importe quelle base $b$. Le principe est d'effectuer les divisions euclidiennes successives par $b$, les restes de ces divisions sont les chiffres du nombre dans la base $b$.}
			      \onslide<2->{\item Pour écrire $N$ en base $b$ :}
			      \begin{enumerate}
				      \item<3-> Faire la division euclidienne de $N$ par $b$, soit $Q$ le quotient et $R$ le reste. \\
				            (c'est à dire écrire $N = Q\times b + R$ avec $R<b$)
				      \item<4-> Ajouter $R$ aux chiffres de $N$ en base $b$
				      \item<5-> Si $Q=0$ s'arrêter, sinon recommencer à partir de l'étape 1 en remplaçant $N$ par $Q$.
			      \end{enumerate}
		\end{itemize}
	\end{block}
\end{frame}

%Exemple algorithme des divisions successives
\begin{frame}{\Ctitle}{\stitle}
	\begin{exampleblock}{Exemple d'utilisation de l'algorithme des divisions successives}
		Donner l'écriture en base 16 de $\base{2019}{10}$. \\ \pause
		\begin{tabular}{lllllll}
			$2019$                                & $=$               & $\onslide<3->{\textcolor{blue}{126}}$ & $\times$               & $16$               & $+$               & $\onslide<4->{\textcolor{red}{\boxed{3}}} $  \\
			$\onslide<5->{\textcolor{blue}{126}}$ & \onslide<5->{$=$} & $\onslide<6->{\textcolor{blue}{7}}$   & \onslide<5->{$\times$} & \onslide<5->{$16$} & \onslide<5->{$+$} & $\onslide<7->{\textcolor{red}{\boxed{14}}} $ \\
			$\onslide<8->{\textcolor{blue}{7}}$   & \onslide<8->{$=$} & $\onslide<9->{\textcolor{blue}{0}}$   & \onslide<8->{$\times$} & \onslide<8->{$16$} & \onslide<8->{$+$} & $\onslide<10->{\textcolor{red}{\boxed{7}}} $ \\
		\end{tabular} \\
		\onslide<14->{
			Le quotient est nul, l'algorithme s'arrête et les chiffres en base 16 sont les restes obtenus à chaque étape donc  $\base{2019}{10}=\base{7E3}{16}$ (car 14 correspond au chiffre E).}
	\end{exampleblock}
\end{frame}

%Exemple algorithme des divisions successives
\begin{frame}{\Ctitle}{\stitle}
	\begin{exampleblock}{Exemple d'utilisation de l'algorithme des divisions successives}
		Donner l'écriture en base 16 de $\base{9787}{10}$. \\ \pause
		\begin{tabular}{lllllll}
			$9787$                                & $=$                 & $\onslide<3->{\textcolor{blue}{611}}$ & $\times$                & $16$                & $+$                & $\onslide<4->{\textcolor{red}{\boxed{11}}} $ \\
			$\onslide<5->{\textcolor{blue}{611}}$ & \onslide<5->{$=$}   & $\onslide<6->{\textcolor{blue}{38}}$  & \onslide<5->{$\times$}  & \onslide<5->{$16$}  & \onslide<5->{$+$}  & $\onslide<7->{\textcolor{red}{\boxed{3}}} $  \\
			$\onslide<8->{\textcolor{blue}{38}}$  & \onslide<8->{$=$}   & $\onslide<9->{\textcolor{blue}{2}}$   & \onslide<8->{$\times$}  & \onslide<8->{$16$}  & \onslide<8->{$+$}  & $\onslide<10->{\textcolor{red}{\boxed{6}}} $ \\
			$\onslide<11->{\textcolor{blue}{2}}$  & \onslide<11->{ $=$} & $\onslide<12->{\textcolor{blue}{0}}$  & \onslide<11->{$\times$} & \onslide<11->{$16$} & \onslide<11->{$+$} & $\onslide<13->{\textcolor{red}{\boxed{2}}} $ \\
		\end{tabular} \\
		\onslide<14->{
			Le quotient est nul, l'algorithme s'arrête et les chiffres en base 16 sont les restes obtenus à chaque étape donc  $\base{9781}{10}=\base{263B}{16}$ (car 11 correspond au chiffre B).}
	\end{exampleblock}
\end{frame}


%Exemple algorithme des divisions successives (base 2)
\begin{frame}{\Ctitle}{\stitle}
	\begin{exampleblock}{Exemple d'utilisation de l'algorithme des divisions successives}
		Donner l'écriture en base 2 de $\base{786}{10}$. \\ \pause
		\begin{tabular}{lllllll}
			$786$                                 & $=$                 & $\onslide<3->{\textcolor{blue}{393}}$ & $\times$                & $2$                & $+$                & $\onslide<4->{\textcolor{red}{\boxed{0}}} $  \\
			$\onslide<5->{\textcolor{blue}{393}}$ & \onslide<5->{$=$}   & $\onslide<6->{\textcolor{blue}{196}}$ & \onslide<5->{$\times$}  & \onslide<5->{$2$}  & \onslide<5->{$+$}  & $\onslide<7->{\textcolor{red}{\boxed{1}}} $  \\
			$\onslide<8->{\textcolor{blue}{196}}$ & \onslide<8->{$=$}   & $\onslide<9->{\textcolor{blue}{98}}$  & \onslide<8->{$\times$}  & \onslide<8->{$2$}  & \onslide<8->{$+$}  & $\onslide<10->{\textcolor{red}{\boxed{0}}} $ \\
			$\onslide<11->{\textcolor{blue}{98}}$ & \onslide<11->{ $=$} & $\onslide<12->{\textcolor{blue}{49}}$ & \onslide<11->{$\times$} & \onslide<11->{$2$} & \onslide<11->{$+$} & $\onslide<13->{\textcolor{red}{\boxed{0}}} $ \\
			$\onslide<14->{\textcolor{blue}{49}}$ & \onslide<14->{ $=$} & $\onslide<15->{\textcolor{blue}{24}}$ & \onslide<14->{$\times$} & \onslide<14->{$2$} & \onslide<14->{$+$} & $\onslide<16->{\textcolor{red}{\boxed{1}}} $ \\
			$\onslide<17->{\textcolor{blue}{24}}$ & \onslide<17->{ $=$} & $\onslide<18->{\textcolor{blue}{12}}$ & \onslide<17->{$\times$} & \onslide<17->{$2$} & \onslide<17->{$+$} & $\onslide<19->{\textcolor{red}{\boxed{0}}} $ \\
			$\onslide<20->{\textcolor{blue}{12}}$ & \onslide<20->{ $=$} & $\onslide<21->{\textcolor{blue}{6}}$  & \onslide<20->{$\times$} & \onslide<20->{$2$} & \onslide<20->{$+$} & $\onslide<22->{\textcolor{red}{\boxed{0}}} $ \\
			$\onslide<23->{\textcolor{blue}{6}}$  & \onslide<23->{ $=$} & $\onslide<24->{\textcolor{blue}{3}}$  & \onslide<23->{$\times$} & \onslide<23->{$2$} & \onslide<23->{$+$} & $\onslide<25->{\textcolor{red}{\boxed{0}}} $ \\
			$\onslide<26->{\textcolor{blue}{3}}$  & \onslide<26->{ $=$} & $\onslide<27->{\textcolor{blue}{1}}$  & \onslide<26->{$\times$} & \onslide<26->{$2$} & \onslide<26->{$+$} & $\onslide<28->{\textcolor{red}{\boxed{1}}} $ \\
			$\onslide<29->{\textcolor{blue}{1}}$  & \onslide<29->{ $=$} & $\onslide<30->{\textcolor{blue}{0}}$  & \onslide<29->{$\times$} & \onslide<29->{$2$} & \onslide<29->{$+$} & $\onslide<31->{\textcolor{red}{\boxed{1}}} $ \\
		\end{tabular} \\
		\onslide<32->{
			Le quotient est nul, l'algorithme s'arrête et $\base{786}{10}=\base{1100010010}{2}$.}
	\end{exampleblock}
\end{frame}

\begin{frame}{\Ctitle}{\stitle}
	\begin{block}{Algorithme en pseudo-code}
		\SetAlFnt{\small}
	\setlength{\algomargin}{8pt}
	\begin{algorithm}[H]
		\DontPrintSemicolon
		\caption{Conversion de la base 10 vers la base b}
		\Entree{$n \in \N$ (en base 10) et $b \in \N, b\geqslant 2$.}
		\Sortie{Les chiffres $a_{p-1},\dots a_0$ de $n$ en base $b$ (donc des éléments de $\intN{0}{b-1}$)}
		\everypar={\footnotesize \textcolor{gray}{\nl}}
		\Si{$n=0$}{\Return 0}
		$p \leftarrow$ nombre de chiffres de n en base b\\
		\Pour{$i \leftarrow 0$ à $p-1$ }{
			$a_{i} \leftarrow$ reste dans la division euclidienne de $n$ par b\\
			$n \leftarrow \lfloor \dfrac{n}{b} \rfloor$\\
		}
		\Return $a_{p-1},\dots,a_0$
	  \end{algorithm}
	  \onslide<2->{Une implémentation sera vue en TP.}
	\end{block}
\end{frame}



\end{document}

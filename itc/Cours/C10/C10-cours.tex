\PassOptionsToPackage{dvipsnames,table}{xcolor}
\documentclass[10pt,french]{beamer}
\usepackage{Cours}

\begin{document}

\input{\detokenize{/home/fenarius/Travail/Cours/cpge-info/latex//MacrosCours.tex}}

% Numéro et titre de chapitre
\setcounter{numchap}{10}
\newcommand{\Ctitle}{\cnum {Terminaison et correction}}
\newcommand{\SPATH}{/home/fenarius/Travail/Cours/cpge-info/docs/itc/files/C\thenumchap/}

\makess{Exemple introductif}
\begin{frame}{\Ctitle}{\stitle}
	\begin{block}{Définitions}
		\begin{itemize}
			\item<1-> Un \textcolor{blue}{algorithme} est une suite d'instructions et d'opérations permettant de résoudre un problème. \\
				\onslide<2->{\textcolor{gray}{\small Par exemple pour résoudre le problème du tri d'une liste de valeurs, on peut utiliser l'algorithme du tri par insertion. Cela consiste à prendre une à une chaque valeur et à l'insérer au bon emplacement dans une liste initialement vide.}}
			\item<3-> Un \textcolor{blue}{programme} est la traduction d'un algorithme dans un langage informatique. On dit qu'un programme est la mise en oeuvre d'un algorithme.\\
				\onslide<4->{\textcolor{gray}{\small L'algorithme du tri par insertion peut être écrit en Python, mais aussi dans un autre langage de programmation.}}
			\item<5-> L'algorithme (la méthode) est donc \textcolor{blue}{indépendante} du langage utilisé pour le mettre en oeuvre sur un ordinateur.
			\item<6->Pour décrire l'algorithme on utilise parfois un \textit{pseudo langage}, permettant de donner la suite des instructions, tests et opérations de l'algorithme.
		\end{itemize}
	\end{block}
\end{frame}

\begin{frame}{\Ctitle}{\stitle}
	\begin{exampleblock}{Exemples}
		\SetAlFnt{\small}
		\setlength{\algomargin}{8pt}
		\begin{algorithm}[H]
			\DontPrintSemicolon
			\caption{Algorithme mystère}
			\Entree{$n \in \N, m \in \N$}
			\Sortie{$?$}
			\everypar={\footnotesize \textcolor{gray}{\nl}}
			$r \leftarrow 0$\;
			\Tq{$m >0$}{
				$m \leftarrow m-1$ \;
				$r \leftarrow r+n$ \;
			}
			\Return $r$
		\end{algorithm}
		\begin{enumerate}
			\item<2-> Faire fonctionner cet algorithme avec $n=3$ et $m=4$.
			\item<3-> Quel est le rôle de cet algorithme ? Proposer un nom pour cet algorithme.
			\item<4-> Donner une implémentation en Python de cet algorithme.
		\end{enumerate}
	\end{exampleblock}
\end{frame}

\begin{frame}[fragile]{\Ctitle}{\stitle}
	\begin{block}{Algorithmique}
		Dans l'étude des algorithmes (\textcolor{blue}{algorithmique}), on s'intéresse aux trois problèmes suivants :
		\begin{enumerate}
			\item<2-> \textcolor{blue}{terminaison} : peut-on garantir que l'algorithme se termine en un temps fini ? (ne concerne que les algorithmes récursifs ou avec des boucles non bornées)
			\item<3-> \textcolor{blue}{correction} : peut-on garantir que l'algorithme fournit la réponse attendue ?
			\item<4-> \textcolor{blue}{complexité} : evolution du temps d'exécution de l'algorithme en fonction de la taille des données (prédiction de temps de calcul, comparaisons des performances d'algorithmes résolvant le même problème.)
		\end{enumerate}
		\onslide<5->{L'algorithme étudié doit avoir une \textcolor{blue}{spécification} précise (entrées, sorties, préconditions, postconditions, effets de bord). On parle d'algorithmes (et non de programmes) car ces questions sont indépendantes de l'implémentation dans un langage de programmation quelconque.}
	\end{block}
\end{frame}

\begin{frame}[fragile]{\Ctitle}{\stitle}
	\begin{exampleblock}{Exemple}
		Dans le cas de l'algorithme permettant de multiplier deux entiers positifs $m$ et $n$ sans utiliser l'opérateur de multiplication, on veut donc :
		\begin{itemize}
		\item<2-> \textit{Prouver} que cet algorithme s'arrête après un nombre fini d'étapes  (\textcolor{blue}{terminaison}).
		\item<3-> \textit{Prouver} que le résultat renvoyé est bien $n\times m$ (\textcolor{blue}{correction}).
		\item<4-> Donner une mesure de l'évolution du nombre d'opérations réalisés par l'algorithme en fonction de la taille des entrées $n$ et $m$ (\textcolor{blue}{complexité}).
		\end{itemize}
	\end{exampleblock}
\end{frame}

\makess{Terminaison d'un algorithme}
\begin{frame}[fragile]{\Ctitle}{\stitle}
	\begin{block}{Exemple introductif}
		\SetAlFnt{\small}
		\setlength{\algomargin}{8pt}
		\begin{algorithm}[H]
			\DontPrintSemicolon
			\caption{Multiplier sans utiliser {\tt *}}
			\Entree{$n \in \N, m \in \N$}
			\Sortie{$nm$}
			\everypar={\footnotesize \textcolor{gray}{\nl}}
			$r \leftarrow 0$\;
			\Tq{$m >0$}{
				$m \leftarrow m-1$ \;
				$r \leftarrow r+n$ \;
			}
			\Return $r$
		\end{algorithm}
		\begin{enumerate}
			\item<2-> Montrer que dans la boucle "tant que":
				\begin{itemize}
					\item<3-> $m$ ne prend que des valeurs entières
					\item<4-> $m$ prend des valeurs positives
					\item<5-> les valeurs prises par $m$ sont \textbf{strictement} décroissantes.
				\end{itemize}
		\end{enumerate}
	\end{block}
\end{frame}

\begin{frame}[fragile]{\Ctitle}{\stitle}
	\begin{block}{Définitions}
		\begin{itemize}
			\item<2-> On dit qu'un algorithme \textcolor{blue}{termine} lorsqu'il renvoie un résultat en un nombre fini d'étapes quels que soient les valeurs des entrées.
			\item<3-> Un \textcolor{blue}{variant de boucle} est une quantité :
				\begin{enumerate}
					\item<4-> à valeurs entières,
					\item<5-> positives,
					\item<6-> qui décroît \textit{strictement} à chaque passage dans la boucle.
				\end{enumerate}
		\end{itemize}
	\end{block}
	\onslide<7->
	{\begin{alertblock}{\textcolor{yellow}{\important \;} Propriété}
			Si une boucle admet un variant, alors cette boucle termine.
		\end{alertblock}}
	\onslide<8->
	{
		\begin{exampleblock}{Exemple}
			$m$ est un variant de boucle de l'algorithme de multiplication ci-dessus et donc cet algorithme termine.
		\end{exampleblock}
	}
\end{frame}

\begin{frame}[fragile]{\Ctitle}{\stitle}
	\begin{exampleblock}{Exemple}
		\SetAlFnt{\small}
		\setlength{\algomargin}{8pt}
		\begin{algorithm}[H]
			\DontPrintSemicolon
			\caption{Nombre de chiffres en base 10}
			\Entree{$n \in \N$}
			\Sortie{$p$ nombre de chiffres de $n$ dans son écriture en base 10.}
			\everypar={\footnotesize \textcolor{gray}{\nl}}
			\Si{$n=0$}{\Return 1}
			$p \leftarrow 0$\;
			\Tq{$n >0$}{
				$p \leftarrow p+1$ \;
				$n \leftarrow \lfloor \frac{n}{10} \rfloor $ \;
			}
			\Return $p$
		\end{algorithm}
		\begin{enumerate}
			\item<2-> Donner une implémentation en Python de cet algorithme.
			\item<3-> Prouver la terminaison de cet algorithme.
		\end{enumerate}
	\end{exampleblock}
\end{frame}

\begin{frame}{\Ctitle}{\stitle}
	\begin{exampleblock}{Implémentation en Python}
		\inputpartPython{\SPATH/nbchiffres.py}{}{}{1}{10}
	\end{exampleblock}
\end{frame}

\begin{frame}{\Ctitle}{\stitle}
	\begin{exampleblock}{Preuve de terminaison}
		Montrons que $n$ est un variant de la boucle {\tt tant que} de l'algorithme :
		\begin{enumerate}
			\item<1-> $n$ ne prend que des valeurs entières, en effet, $n \in \N$  en entrée et $\PE{\dfrac{n}{10}}$ est entier.
			\item<2-> $n$ est positif  par condition d'entrée dans la boucle.
			\item<3-> $n$ décroît strictement à chaque passage dans la boucle car comme $n > 0$, $\PE{\dfrac{n}{10}} < n$
		\end{enumerate}
	\end{exampleblock}
\end{frame}

\begin{frame}{\Ctitle}{\stitle}
	\begin{exampleblock}{Exercice}
		\begin{enumerate}
			\item<1-> Ecrire un algorithme qui prend en entrée un entier $n>1$ et renvoie le premier diviseur strictement supérieur à 1 de cet entier. Par exemple pour $n=7$ l'algorithme renvoie 7 et pour $n=15$, l'algorithme renvoie $3$.
			\item<2-> Donner une implémentation en Python de cet algorithme sous la forme d'une fonction dont on précisera soigneusement la spécification.
			\item<3-> Prouver la terminaison de cet algorithme.
		\end{enumerate}
	\end{exampleblock}
\end{frame}

\makess{Correction d'un algorithme}
\begin{frame}[fragile]{\Ctitle}{\stitle}
	\begin{block}{Exemple introductif}
		\SetAlFnt{\small}
		\setlength{\algomargin}{8pt}
		\begin{algorithm}[H]
			\DontPrintSemicolon
			\caption{Multiplier sans utiliser {\tt *}}
			\Entree{$n \in \N, m \in \N$}
			\Sortie{$nm$}
			\everypar={\footnotesize \textcolor{gray}{\nl}}
			$r \leftarrow 0$\;
			\Tq{$m >0$}{
				$m \leftarrow m-1$ \;
				$r \leftarrow r+n$ \;
			}
			\Return $r$
		\end{algorithm}
		On note $m_0$ la valeur initiale de $m$ et on considère la propriété suivante suivante noté $I$ : \og{} $r = (m_0-m)n$ \fg{}. Montrer que cette propriété est vraie :
		\begin{enumerate}
			\item<2-> avant d'entrée dans la boucle,
			\item<3-> qu'elle reste vraie à chaque tour de boucle.
		\end{enumerate}
		\onslide<4->{Que peut-on en conclure ?}
	\end{block}
\end{frame}

\begin{frame}[fragile]{\Ctitle}{\stitle}
	\begin{block}{Définitions}
		\begin{itemize}
			\item<1-> Un \textcolor{blue}{invariant de boucle} est une propriété qui :
				\begin{itemize}
					\item<2-> est vraie à l'entrée dans la boucle (\textcolor{blue}{initialisation}),
					\item<3-> reste vraie à chaque itération si elle l'était à l'itération précédente (\textcolor{blue}{conservation}).
				\end{itemize}
			\item<4-> Un algorithme est dit \textcolor{blue}{partiellement correct} lorsqu'il renvoie la réponse attendue quand il se termine.
			\item<5-> Un algorithme est dit \textcolor{blue}{totalement correct} lorsqu'il est partiellement correcte et que sa terminaison est prouvée.
		\end{itemize}
	\end{block}
	\onslide<7->
	{\begin{alertblock}{\textcolor{yellow}{\important \;} Prouver la correction d'un algorithme}
			On utilise un invariant de boucle qui en sortie de boucle fournit une propriété permettant de montrer la correction.\\
			\textcolor{gray}{La méthode est similaire à une récurrence mathématique.}
		\end{alertblock}}
\end{frame}


\begin{frame}[fragile]{\Ctitle}{\stitle}
	\begin{exampleblock}{Exemple}
		\SetAlFnt{\small}
		\setlength{\algomargin}{8pt}
		\begin{algorithm}[H]
			\DontPrintSemicolon
			\caption{Nombre de chiffres en base 10}
			\Entree{$n \in \N$}
			\Sortie{$p$ nombre de chiffres de $n$ dans son écriture en base 10.}
			\everypar={\footnotesize \textcolor{gray}{\nl}}
			\Si{$n=0$}{\Return 1}
			$p \leftarrow 0$\;
			\Tq{$n >0$}{
				$p \leftarrow p+1$ \;
				$n \leftarrow \lfloor \frac{n}{10} \rfloor $ \;
			}
			\Return $p$
		\end{algorithm}
		Prouver que cet algorithme est correct.
	\end{exampleblock}
\end{frame}

\begin{frame}[fragile]{\Ctitle}{\stitle}
	\begin{exampleblock}{Correction}
		\textcolor{gray}{Idée : à chaque tour de boucle, on incrémente $p$ et $n$ perd un chiffre. La somme de $p$ et du nombre de chiffres de $n$ est donc constante. C'est l'invariant de boucle ! \\ \medskip}
		\onslide<2->{Le cas $n=0$ est trivial, on note $n_0$ la valeur initiale de $n$ et on suppose donc $n_0>0$,  on considère la propriété $I$ : \og{} $c(n_0) = p + c(n)$ \fg{}. Où $c(m)$ vaut 0 si $m=0$ et le nombre de chiffres de $m$ en base 10 sinon.}
		\begin{itemize}
			\item<3-> initialisation : $I$ est vraie avant d'entrer dans la boucle puisque $p=0$ et $n=n_0$.
			\item<4-> conservation : on suppose $I$ vraie du début d'un tour de boucle et  on note $n'$ (resp. $p'$) la valeur de $n$ (resp $p'$) après ce tour de boucle  \\
				\onslide<5->{$p' + c(n') = p + 1 + c\left(\PE{\dfrac{n}{10}}\right)$ \\}
				\onslide<6->{$p' + c(n') = p + 1 + c\left(n\right) - 1$ \\}
				\onslide<7->{$p' + c(n') = c(n_0)$, car $I$ est vraie à l'entrée de la boucle.}
		\end{itemize}
		\onslide<8->{En sortie de boucle, on a  $n'=0$ et donc $p = c(n_0)$ et l'algorithme est correct.}
	\end{exampleblock}
\end{frame}

\begin{frame}[fragile]{\Ctitle}{\stitle}
	\begin{exampleblock}{Exemple}
		\SetAlFnt{\small}
		\setlength{\algomargin}{8pt}
		\begin{algorithm}[H]
			\DontPrintSemicolon
			\caption{Premier diviseur}
			\Entree{$n \in \N, n > 1$}
			\Sortie{$d$ premier diviseur de $n$ strictement supérieur à $1$.}
			\everypar={\footnotesize \textcolor{gray}{\nl}}
			$d \leftarrow 2$\;
			\Tq{$n \mod d \neq 0$}{
				$d \leftarrow d+1$ \;
			}
			\Return $d$
		\end{algorithm}
		Prouver que cet algorithme est correct.
	\end{exampleblock}
\end{frame}

\makess{Exercices}
\begin{frame}[fragile]{\Ctitle}{\stitle}
	\begin{exampleblock}{Exercice 1 : fonction mystère}
		On considère la fonction Python ci-dessous :
		\inputpartPython{\SPATH/quotient.py}{}{}{1}{7}
		\begin{enumerate}
		\item<1-> Proposer un nom et une spécification pour cette fonction
		\item<2-> Prouver sa terminaison
		\item<3-> Prouver sa correction par rapport à la spécification proposée à la question 1.
	\end{enumerate}
	\end{exampleblock}
\end{frame}

\begin{frame}[fragile]{\Ctitle}{\stitle}
	\begin{exampleblock}{Exercice 2 : exponentiation rapide}
		
	\begin{tabularx}{\textwidth}{X|X}
		On donne ci-dessous l'algorithme d'exponentiation rapide en version itérative : \\
    \begin{algorithm}[H]
		\DontPrintSemicolon
		\caption{\small Exponentiation rapide}
		\Entree{$a \in \R, n \in \N$}
		\Sortie{$a^n$}
		\everypar={\footnotesize \textcolor{gray}{\nl}}
		$p \leftarrow 1$\;
		\Tq{$n \neq 0$}{
			\Si{$n$ est impair}
			{$p \leftarrow p\times a$ \;}
			$a \leftarrow a*a$ \;
			$n \leftarrow \lfloor\frac{n}{2}\rfloor$ \;
		}
		\Return $p$
	\end{algorithm} & 
	\vspace{-3.7cm}
	\begin{enumerate}
		\item Donner les valeurs prises par $a$, $n$ et $p$ lorsqu'on fait fonctionner cet algorithme avec $a=2$ et $n=13$.
		\item Donner une implémentation de cet algorithme en Python.
		\item Prouver que cet algorithme termine.
		\item Prouver qu'il est correct. \\
		\textcolor{OliveGreen}{\small En notant $a_0$ (resp. $n_0$) la valeur initiale de $a$ (resp. $n$), on pourra prouver l'invariant $p \times a^{n} = a_0^{n_0}$.}
	\end{enumerate} \\
\end{tabularx}
	\end{exampleblock}
\end{frame}

\end{document}


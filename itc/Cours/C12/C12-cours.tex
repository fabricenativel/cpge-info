\PassOptionsToPackage{dvipsnames,table}{xcolor}
\documentclass[10pt,french]{beamer}
\usepackage{Cours}

\begin{document}

\input{\detokenize{/home/fenarius/Travail/Cours/cpge-info/latex//MacrosCours.tex}}

% Numéro et titre de chapitre
\setcounter{numchap}{12}
\newcommand{\Ctitle}{\cnum {Complexité des algorithmes}}
\newcommand{\SPATH}{/home/fenarius/Travail/Cours/cpge-info/docs/itc/files/C\thenumchap/}
\newcommand{\ar}[1]{\{#1\}}
\newcommand{\normal}[1]{\circlenode{#1}{#1}}
\newcommand{\todo}[1]{\circlenode[linewidth=1pt,linecolor=red,fillcolor=fluo,fillstyle=solid]{#1}{#1}}
\newcommand{\visited}[1]{\circlenode[linewidth=1pt,linecolor=red,fillcolor=fluo,fillstyle=solid]{#1}{#1}}

\makess{Introduction}
\begin{frame}[fragile]{\Ctitle}{\stitle}
	\begin{exampleblock}{Exemple introductif}
		On considère l'algorithme suivant :
		\SetAlFnt{\small}
		\setlength{\algomargin}{8pt}
		\begin{algorithm}[H]
			\DontPrintSemicolon
			\caption{Recherche simple}
			\Entree{$a \in \N$ et un tableau d'entiers $t$ de longueur $n$}
			\Sortie{Booléen indiquant si $a \in t$.}
			\everypar={\footnotesize \textcolor{gray}{\nl}}
			\Pour{$i \leftarrow 0$ à $n-1$ }{
				\Si{$t[i]=a$}{
					\Return Vrai
				}
			}
			\Return Faux
		\end{algorithm}
		\begin{enumerate}
			\item<2-> Combien de comparaisons effectue cet algorithme dans le meilleur des cas ?
			\item<3-> Même question dans le pire des cas.
			\item<4-> Que dire du cas où on recherche un élément $a$ présent en un seul exemplaire dans le tableau $t$ en supposant les positions équiprobables ?
		\end{enumerate}
	\end{exampleblock}
\end{frame}


\begin{frame}[fragile]{\Ctitle}{\stitle}
	\begin{exampleblock}{Correction}
		\begin{enumerate}
			\item<2-> \textcolor{OliveGreen}{\small Si l'élément cherché est en première position dans le tableau on effectue une seule comparaison.}
			\item<3-> \textcolor{OliveGreen}{\small Si l'élément cherché n'est pas dans le tableau (ou qu'il y figure en dernière position) on effectue $n$ comparaison.}
			\item<4-> \textcolor{OliveGreen}{\small on note $X$ le nombre de comparaisons avant de trouver $a$, alors $p(X=k) = \dfrac{1}{n}$. Donc,\\
					$\begin{array}{lll}
							\onslide<5->{E(X) & = & \displaystyle{\sum_{k=1}^{n} k\,\dfrac{1}{n}} \\}
							\onslide<6->{E(X) & = & \dfrac{n+1}{2}}
						\end{array}$
				}
		\end{enumerate}
		\onslide<7->\textcolor{OliveGreen}{Le nombre de comparaisons varie donc avec les données du problème.}\\
		\onslide<8->{On peut cependant toujours \textit{majorer} le nombre de comparaisons, qui reste inférieur dans tous les cas à $Kn$  où $K$ est une constante et $n$ la taille du tableau.}
	\end{exampleblock}
\end{frame}

\makess{Calcul de la complexité}
\begin{frame}[fragile]{\Ctitle}{\stitle}
	\begin{block}{Calcul de complexité}
		L'exemple précédent est celui du calcul de la complexité d'un algorithme :
		\begin{itemize}
			\item<1-> on utilise \textcolor{blue}{une mesure de l'efficacité} de l'algorithme.\\
				\onslide<2-> \textcolor{gray}{Dans l'exemple c'est le nombre de tests de comparaisons effectués par l'algorithme. D'autres mesures sont possibles, notamment le nombre d'opérations élémentaires ou encore la quantité de mémoire occupée.}
			\item<3-> comme les performances de l'algorithme varient en fonction des données, on s'intéresse généralement à une simple \textcolor{blue}{majoration dans le \textit{pire des cas}}. \\
				\onslide<4-> \textcolor{gray}{Dans l'exemple bien que l'algorithme puisse parfois répondre avec une seule comparaison c'est le pire des cas qui nous intéresse.}
			\item<5-> on fournit une \textcolor{blue}{majoration asymptotique} c'est à dire à une constante multiplicative près et à partir d'un certain rang.\\
				\onslide<6-> \textcolor{gray}{Dans l'exemple précédent la majoration était $Kn$.}
		\end{itemize}
	\end{block}
\end{frame}

\begin{frame}[fragile]{\Ctitle}{\stitle}
	\begin{alertblock}{Définition}
		La \textcolor{red}{complexité} d'un algorithme est une mesure de son efficacité.
		\onslide<2-> On parle notamment de :
		\begin{itemize}
			\item<3-> \textcolor{blue}{Complexité en temps} : le nombre d'opérations \textit{élémentaires} nécessaire à l'exécution d'un algorithme.
			\item<4-> Complexité en mémoire : l'occupation mémoire en fonction de la taille des données.
		\end{itemize}
		\onslide<5->{Ces deux éléments varient en fonction de la taille et de la nature des données, on donne donc généralement une majoration dans le pire des cas.}
	\end{alertblock}
	\onslide<6->{
		\begin{block}{Remarque}
			On peut aussi parler de la \textcolor{blue}{complexité en moyenne}, qui s'intéresse au  nombre moyen d'operations effectuées par un algorithme sur un ensemble d'entrées de taille $n$.
		\end{block}}
\end{frame}


\begin{frame}[fragile]{\Ctitle}{\stitle}
	\begin{alertblock}{Calcul de la complexité temporelle}
		\begin{itemize}
			\item<1-> On considère certaines opérations comme élémentaires, leur coût est alors majoré par une constante.\\
				\onslide<2->{\textcolor{OliveGreen}{\small Par exemple les opérations arithmétiques, les tests, les affectations \dots } \\
					\textcolor{BrickRed}{\small \danger \;} En Python, certaines opérations comme par exemple le test d'appartenance à une liste avec \textcolor{blue}{\tt in} ne sont pas des opérations élémentaires.
				}

			\item<3-> On exprime le coût de l'algorithme pour une entrée de taille {\tt n} en nombre d'opérations élémentaires nécessaires à sa réalisation.\\
				\onslide<4->{
				\textcolor{OliveGreen}{\small Par exemple, la fonction {\tt f} définie par \mintinline{python}{def f(x) = return x*x + 2*x + 3} demande 5 opérations quelque soit la taille de l'entrée {\tt x}. }
				}
		\end{itemize}
	\end{alertblock}
\end{frame}

\begin{frame}[fragile]{\Ctitle}{\stitle}
    \begin{exampleblock}{Exemple}
        On considère la fonction suivante :
        \inputpartPython{somme.py}{}{}{1}{7}
        \begin{enumerate}
            \item Quelle est la complexité de la fonction suivante en nombre d'opérations élémentaires ?
            \item La complexité serait-elle la même si on remplaçait la boucle {\tt while} par un {\tt for} ?
        \end{enumerate}
    \end{exampleblock}
\end{frame}

\makess{Majorations asymptotiques}
\begin{frame}[fragile]{\Ctitle}{\stitle}
	\begin{block}{Majoration asymptotique}
		\begin{itemize}
			\item En pratique, seul une \textcolor{blue}{majoration asymptotique} du coût $C(n)$ d'un algorithme nous intéresse et pas sa détermination exacte.\\
			      \onslide<2->\textcolor{OliveGreen}{Par exemple, si le coût de l'algorithme est $C(n) = 3n + 15$ opérations élémentaires, on dira  que $C(n)$ est majoré asymptotiquement par $n$ car $C(n) < 4n$ dès que $n>15$.}
			\item<3-> L'outil mathématique associé est la notion de \textcolor{blue}{domination} d'une suite : \\
				Etant donné deux suites $(u_n)_{n \in \N}$ et $(v_n)_{n \in \N}$ à valeurs strictement positives. On dit que $(u_n)_{n \in \N}$ est dominée par $(v_n)_{n \in \N}$ lorsqu'il existe un entier $K>0$ et un rang $N \in \N$ tel que : \\ \onslide<3->{$\forall n \in \N, n>N$, on a  $u_n \leq K v_n$.\\}
				\onslide<4->{On note alors \textcolor{BrickRed}{$u = \GO(v)$} (ou encore $u \in \GO(v)$) et on dit que $u$ est un grand $\GO$ de $v$.\\}
				\onslide<5->{\og{} $u_n$ est inférieur à $v_n$ à une constante multiplicative près et pour $n$ assez grand \fg.\\}
				\onslide<6->\textcolor{OliveGreen}{Dans l'exemple précédent, $C(n)=3n+15$ est un $\GO(n)$.}
		\end{itemize}
	\end{block}
\end{frame}

\begin{frame}[fragile]{\Ctitle}{\stitle}
	\begin{exampleblock}{Exemples}
		\begin{itemize}
			\item<1-> Montrer que $(a_n)$ de terme général $10n + 3$ est un $\GO(n)$\\
				\onslide<4->{\textcolor{OliveGreen}{$10 n + 3 < 11 n$ pour $n>3$}}
			\item<2-> Montrer que $(b_n)$ de terme général $n^2 + n + 1$ est un $\GO(n^2)$\\
				\onslide<5->{\textcolor{OliveGreen}{$n^2 + n + 1 < 2 n^2$ pour $n>2$}}
			\item<3-> Déterminer un grand $\GO$ de $(c_n)$ de terme général $ 7 n + \ln(n)$\\
				\onslide<6->{\textcolor{OliveGreen}{Comme $\ln(n) < n$, $c_n<8n$ et donc $(c_n)$ est un $\GO(n)$.}}
		\end{itemize}
	\end{exampleblock}
\end{frame}

\begin{frame}[fragile]{\Ctitle}{\stitle}
	\begin{block}{Remarques}
		Ecrire $u_n = \GO(v_n)$ traduit une \textit{majoration asymptotique}, c'est à dire que \og{}$(u_n)$ est au plus de l'ordre de $(v_n)$\fg{}.\\
		\textcolor{gray}{Par exemple si $u_n = 42n+2024$,  on pourrait écrire $u_n = O(n)$ mais aussi  que $u_n = \GO(n^2)$ (ou encore $u_n = \GO(n^3)$).}\\
		On veut généralement  donner le \og{}meilleur grand $\GO$ \fg{}. Afin d'exprimer formellement cette notion, on note :
		\begin{itemize}
			\item<2-> $u_n = \Omega(v_n)$ s'il existe $K \in \R^+$ et $n_0 \in \N$ tel que pour tout $n \geqslant n_0, u_n \geqslant Kv_n$, c'est à dire que \og{}$(u_n)$ est au moins de l'ordre de $(v_n)$\fg{}\\
			\item<3-> $v_n = \Theta(v_n)$ si $u_n = \GO(v_n)$ et $v_n = \GO(u_n)$, c'est à dire que \og{}$(u_n)$ est  de l'ordre de $(v_n)$\fg{}\\
		\end{itemize}
		\onslide<4->{On utilisera principalement la notation $\GO$, en gardant à l'esprit qu'on essaye toujours d'avoir la meilleure majoration}.
	\end{block}
\end{frame}


\makess{Complexités usuelles}
\begin{frame}[fragile]{\Ctitle}{\stitle}
	\begin{alertblock}{Complexités usuelles}
		\renewcommand{\arraystretch}{1.3}
		\begin{tabularx}{\textwidth}{|c|l|X|}
			\hline
			\textcolor{blue}{Complexité}                & \textcolor{blue}{Nom}                  & \textcolor{blue}{Exemple}                                             \\
			\hline
			\leavevmode\onslide<2->{$\GO(1)$}           & \leavevmode\onslide<2->{Constant}      & \leavevmode\onslide<2->{Accéder à un élément d'un tableau}            \\
			\hline
			\leavevmode\onslide<3->{$\GO(\log(n))$}     & \leavevmode\onslide<3->{Logarithmique} & \leavevmode\onslide<3->{Recherche dichotomique dans une liste}        \\
			\hline
			\leavevmode\onslide<4->{$\GO(n)$          } & \leavevmode\onslide<4->{Linéaire     } & \leavevmode\onslide<4->{Recherche simple dans une liste}              \\
			\hline
			\leavevmode\onslide<5->{$\GO(n\log(n))$   } & \leavevmode\onslide<5->{Linéaritmique} & \leavevmode\onslide<5->{Tri fusion}                                   \\
			\hline
			\leavevmode\onslide<6->{$\GO(n^2)$        } & \leavevmode\onslide<6->{Quadratique  } & \leavevmode\onslide<6->{Tri par insertion d'une liste }               \\
			\hline
			\leavevmode\onslide<7->{$\GO(2^n)$        } & \leavevmode\onslide<7->{Exponentielle} & \leavevmode\onslide<7->{Algorithme par force brute pour le sac à dos} \\
			\hline
		\end{tabularx}
	\end{alertblock}
\end{frame}

\begin{frame}[fragile]{\Ctitle}{\stitle}
	\begin{block}{Représentation graphique}
		\begin{center}
			\includegraphics[height=6cm]{complexite.eps}
		\end{center}
	\end{block}
\end{frame}

\begin{frame}[fragile]{\Ctitle}{\stitle}
	\begin{block}{Temps de calcul effectif}
		Sur un ordinateur réalisant 100 million d'opérations par seconde, en notant \textcolor{green}{\faCheck} un temps de calcul quasi instantané et \textcolor{red}{\faTimes} un temps de calcul inaccessible :
		\renewcommand{\arraystretch}{1.4}
		\begin{tabular}{lccccc}
			\cline{2-6}
			\multicolumn{1}{c}{\leavevmode} & \leavevmode$n=10$                                    & $n=100$                                              & $n=1000$                                             & $n=10^6$                                             & $n=10^9$                                         \\
			\hline
			\leavevmode$\GO(\log(n))$       & \textcolor{green}{\faCheck}                          & \textcolor{green}{\faCheck}                          & \textcolor{green}{\faCheck}                          & \textcolor{green}{\faCheck}                          & \textcolor{green}{\faCheck}                      \\
			\hline
			\leavevmode$\GO(n)$             & \leavevmode\onslide<2->{\textcolor{green}{\faCheck}} & \leavevmode\onslide<2->{\textcolor{green}{\faCheck}} & \leavevmode\onslide<2->{\textcolor{green}{\faCheck}} & \leavevmode\onslide<2->{\textcolor{green}{\faCheck}} & \leavevmode\onslide<2->{$\simeq 10$s }           \\
			\hline
			\leavevmode$\GO(n\log(n))$      & \leavevmode\onslide<3->{\textcolor{green}{\faCheck}} & \leavevmode\onslide<3->{\textcolor{green}{\faCheck}} & \leavevmode\onslide<3->{\textcolor{green}{\faCheck}} & \leavevmode\onslide<3->{\textcolor{green}{\faCheck}} & \leavevmode\onslide<3->{$\simeq 1,5$ mn}         \\
			\hline
			\leavevmode$\GO(n^2)$           & \leavevmode\onslide<4->{\textcolor{green}{\faCheck}} & \leavevmode\onslide<4->{\textcolor{green}{\faCheck}} & \leavevmode\onslide<4->{\textcolor{green}{\faCheck}} & \leavevmode\onslide<4->{$\simeq 3$ h }               & \leavevmode\onslide<4->{$\simeq 300$ ans }       \\
			\hline
			\leavevmode$\GO(2^n)$           & \leavevmode\onslide<5->{\textcolor{green}{\faCheck}} & \leavevmode\onslide<5->{\textcolor{red}{\faTimes}}   & \leavevmode\onslide<5->{\textcolor{red}{\faTimes}}   & \leavevmode\onslide<5->\textcolor{red}{\faTimes}     & \leavevmode\onslide<5->\textcolor{red}{\faTimes} \\
			\hline
		\end{tabular}
	\end{block}
\end{frame}

\begin{frame}[fragile]{\Ctitle}{\stitle}
	\begin{exampleblock}{Exemples}
		\begin{itemize}
			\item<1-> On suppose qu'on dispose d'un algorithme de complexité linéaire travaillant sur une liste, il traite une liste de \numprint{1000} éléments en \numprint{0.015} secondes. Donner une estimation du temps de calcul pour une liste de \numprint{250000} éléments.\\
				\onslide<2-> {\textcolor{OliveGreen}{La taille des données a été multiplié par 250, la complexité étant lineaire le temps de calcul sera aussi approximativement multiplié par 250. \\}}
				\onslide<3->{\textcolor{OliveGreen}{$0.015 \times 250 = 3.75$, on peut donc prévoir un temps de calcul d'environ 3,75 secondes}}
			\item<4-> Même question pour un algorithme de complexité quadratique qui traite une liste de \numprint{1000} éléments en \numprint{0.07} secondes.\\
				\onslide<5-> {\textcolor{OliveGreen}{La taille des données a été multiplié par 250, la complexité étant quadratique le temps de calcul sera  approximativement multiplié par $250^2=62500$ \\}}
				\onslide<6->{\textcolor{OliveGreen}{$0.07 \times 62\,500 = 4375$, on peut donc prévoir un temps de calcul d'environ $4\,375$ secondes, c'est-à-dire près d'une heure et 15 minutes !}}
		\end{itemize}
	\end{exampleblock}
\end{frame}

\makess{Complexité des fonctions récursives}
\begin{frame}[fragile]{\Ctitle}{\stitle}
	\begin{block}{Equation de complexité}
		Dans le cas des fonction récursives, la complexité pour une entrée de taille $n$ s'exprime à partir de complexité pour des tailles inférieures. On est donc amené à résoudre une \textit{équation de complexité}.
	\end{block}
	\onslide<2->{
		\begin{exampleblock}{Exemple}
			Par exemple si on considère la version récursive du calcul de la somme d'une liste d'entiers en OCaml :

			Alors on a $C(n) = C(n-1) + a$, et donc $C(n)$ est arithmétique de raison $a$ et $C(n)$ est un $O(n)$.
		\end{exampleblock}}
\end{frame}

\makess{Exemples}
\begin{frame}[fragile]{\Ctitle}{\stitle}
	\begin{exampleblock}{Les tours de Hanoï}
		On rappelle que le jeu des tours de Hanoï peut être résolu de façon élégante par récursion. On note $T(n)$ le nombre de mouvement minimal nécessaire afin de résoudre Hanoï avec $n$ disques en utilisant l'algorithme récursif.
		\begin{enumerate}
			\item Déterminer $T(1)$
			\item Exprimer $T(n)$ en fonction $T(n-1)$
			\item En déduire la complexité de l'algorithme.
		\end{enumerate}
	\end{exampleblock}
\end{frame}


\end{document}


\PassOptionsToPackage{dvipsnames,table}{xcolor}
\documentclass[10pt]{beamer}
\usepackage{Cours}

\begin{document}

\input{\detokenize{/home/fenarius/Travail/Cours/cpge-info/latex//MacrosCours.tex}}

% Numéro et titre de chapitre
\setcounter{numchap}{6}
\newcommand{\Ctitle}{\cnum Récursivité}
\newcommand{\SPATH}{/home/fenarius/Travail/Cours/cpge-info/docs/itc/files/C\thenumchap}

\makess{Définition}
\begin{frame}{\Ctitle}{\stitle}
	\begin{alertblock}{Définition}
		\onslide<2->{En informatique, on dit qu'une fonction est \textcolor{red}{récursive},}
		\onslide<3->{lorsque cette fonction fait appel à elle-même.}
	\end{alertblock}
	\onslide<4->{
		\begin{block}{Remarques}
			\begin{itemize}
				\item<5-> Une fonction récursive permet donc, \textit{comme une boucle}, de répéter des instructions. Une même fonction peut donc souvent se programmer de façon \textcolor{blue}{itérative} (avec des boucles) ou de façon \textcolor{blue}{récursive} (en s'appelant elle-même).
				\item<6-> Une fonction récursive doit toujours \textcolor{blue}{contenir une condition d'arrêt}, dans le cas contraire elle s'appelle elle-même à l'infini et le programme ne se termine jamais.
				\item<7-> Les valeurs passées en paramètres lors des appels successifs doivent être différents, sinon la fonction s'exécute à l'identique à chaque appel et donc boucle à l'infini.
			\end{itemize}
		\end{block}}
\end{frame}

\makess{Exemples de fonctions récursives}
\begin{frame}{\Ctitle}{\stitle}
	\begin{exampleblock}{Exemple : compte à rebours}
		\begin{itemize}
			\item<1->Que fait la fonction suivante :
			\inputpartPython{rebours.py}{}{}{1}{4}
			\onslide<2->\textcolor{OliveGreen}{Elle prend en argument un entier {\tt n}, puis affiche un compte à rebours de {\tt n} à 1 et ensuite {\tt "Partez !"}.}
			\item<3->Proposer une version récursive de cette fonction
			\onslide<4->{
				\inputpartPython{rebours.py}{}{}{6}{11}}
		\end{itemize}
	\end{exampleblock}
\end{frame}

% Récursivité : Exemples des puissances
\begin{frame}{\Ctitle}{\stitle}
	\begin{exampleblock}{Exemple : les puissances positives}
		En mathématiques, pour un nombre quelconque $a$ et un entier positif $n$, on définit $a$ puissance $n$ par :\\
		$ a^n = a \times a \times \dots \times a $, et on convient que $a^0=1$
		\begin{itemize}
			\item<2->{Définir une fonction Python {\tt puissance} qui prend en argument {\tt a} et {\tt n} et renvoie $a^n$ en effectuant ce calcul de façon itératif}
			\item<3->{Recopier et compléter : $a^n = \dots \times a^{\dots}$}
			\item<4->{En déduire une version récursive de la fonction calculant les puissances}
		\end{itemize}
	\end{exampleblock}
\end{frame}

% Récursivité : Exemples des puissances
\begin{frame}{\Ctitle}{\stitle}
	\begin{exampleblock}{Exemple : les puissances positives}
		\begin{itemize}
			\item<1-> \textcolor{OliveGreen}{Puissance : version itérative}
				\inputpartPython{puissance.py}{}{}{1}{5}
			\item<2-> $a^n = \textcolor{OliveGreen}{a} \times a^{\textcolor{OliveGreen}{n-1}}$
			\item<3-> \textcolor{OliveGreen}{Puissance : version récursive}
				\inputpartPython{puissance.py}{}{}{7}{10}
		\end{itemize}
	\end{exampleblock}
\end{frame}

\begin{frame}{\Ctitle}{\stitle}
    \begin{exampleblock}{Exemple : maximum des éléments d'une liste non vide}
        \begin{itemize}
        \item<1-> {\small Ecrire une fonction itérative qui renvoie le maximum des éléments d'une liste}
        \onslide<2->{\inputpartPython{max.py}{}{\small}{1}{6}}
        \item<3-> {\small Proposer une version récursive de cette fonction}
        \onslide<4->{\inputpartPython{max.py}{}{\small}{9}{16}}
        \end{itemize}
    \end{exampleblock}
\end{frame}


\makess{Analyser un programme récursif}
\begin{frame}{\Ctitle}{\stitle}
	\begin{exampleblock}{Une fonction à analyser}
		\inputpartPython{mystere.py}{}{}{1}{8}
		\begin{itemize}
			\item<2-> Que fait la fonction {\tt mystere} ci-dessus ?
			\item<3-> Cette fonction est-elle programmée de façon itérative ? récursive ? Justifier.
			\item<4-> Proposer une version de cette fonction qui ne s'appelle pas elle-même.
		\end{itemize}
	\end{exampleblock}
\end{frame}

% Récursivité : Exemple du nombre d'occurence récursif - Correction
\begin{frame}{\Ctitle}{\stitle}
	\begin{exampleblock}{Exemple : une fonction à analyser}
		\begin{itemize}
			\item<2-> \textcolor{OliveGreen}{Cette fonction compte le nombre d'occurence de {\tt elt} dans {\tt liste}}
			\item<3-> \textcolor{OliveGreen}{Elle ne contient pas de boucle, elle n'est donc pas programmé de façon itérative. Par contre c'est une fonction récursive car elle fait appel à elle même.}
			\item<4-> \textcolor{OliveGreen}{Version itérative}
				\inputpartPython{mystere.py}{}{}{10}{15}
		\end{itemize}
	\end{exampleblock}
\end{frame}

% Récursivité : Remarques
\begin{frame}{\Ctitle}{\stitle}
	\begin{block}{Remarques importantes}
		\begin{itemize}
			\item<2-> On peut toujours transformer une fonction itérative en son équivalent récursif.
			\item<3-> Certains problèmes (que nous verrons en exercice) ont une solution récursive très lisible et rapide à programmer. La formulation récursive est donc parfois \og plus adaptée \fg à un problème.
			\item<4-> La programmation récursive est parfois gourmande en ressource car les appels récursifs successifs doivent parfois être conservés dans une \textcolor{blue}{pile} dont la taille est limitée.
		\end{itemize}
	\end{block}
\end{frame}

\makess{Exercice}
\begin{frame}{\Ctitle}{\stitle}
	\begin{exampleblock}{Fusion de deux listes triées}
		On souhaite écrire une fonction qui prend en entrée deux listes {\tt l1} et {\tt l2} qu'on suppose \textcolor{blue}{déjà triées} (dans l'ordre croissant) et renvoie une liste triée résultat de la fusion de {\tt l1} et {\tt l2}. Par exemples :
		\begin{itemize}
		\item<2-> Si {\tt l1 = [2, 6, 9]} et {\tt l2 = [1, 8]} alors la fonction renvoie {\tt [1, 2, 6, 8, 9]}
		\item<3-> Si {\tt l1 = [0, 5, 7, 9]} et {\tt l2 = [5, 5, 8]} alors la fonction renvoie {\tt [0, 5, 5, 5, 7, 8, 9]}
		\item<4-> Si {\tt l1 = [2, 4, 6, 7]} et {\tt l2 = []} alors la fonction renvoie {\tt [2, 4, 6, 7]}
		\end{itemize}
	\end{exampleblock}
\end{frame}

\end{document}

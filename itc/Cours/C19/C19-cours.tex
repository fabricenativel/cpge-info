\PassOptionsToPackage{dvipsnames,table}{xcolor}
\documentclass[10pt]{beamer}
\usepackage{Cours}


\begin{document}


\newcounter{numchap}
\setcounter{numchap}{1}
\newcounter{numframe}
\setcounter{numframe}{0}
\newcommand{\mframe}[1]{\frametitle{#1} \addtocounter{numframe}{1}}
\newcommand{\cnum}{\fbox{\textcolor{yellow}{\textbf{C\thenumchap}}}~}
\newcommand{\makess}[1]{\section{#1} \label{ss\thesection}}
\newcommand{\stitle}{\textcolor{yellow}{\textbf{\thesection. \nameref{ss\thesection}}}}

\definecolor{codebg}{gray}{0.90}
\definecolor{grispale}{gray}{0.95}
\definecolor{fluo}{rgb}{1,0.96,0.62}
\newminted[langageC]{c}{linenos=true,escapeinside=||,highlightcolor=fluo,tabsize=2,breaklines=true}
\newminted[codepython]{python}{linenos=true,escapeinside=||,highlightcolor=fluo,tabsize=2,breaklines=true}
% Inclusion complète (ou partiel en indiquant premiere et dernière ligne) d'un fichier C
\newcommand{\inputC}[3]{\begin{mdframed}[backgroundcolor=codebg] \inputminted[breaklines=true,fontsize=#3,linenos=true,highlightcolor=fluo,tabsize=2,highlightlines={#2}]{c}{#1} \end{mdframed}}
\newcommand{\inputpartC}[5]{\begin{mdframed}[backgroundcolor=codebg] \inputminted[breaklines=true,fontsize=#3,linenos=true,highlightcolor=fluo,tabsize=2,highlightlines={#2},firstline=#4,lastline=#5,firstnumber=1]{c}{#1} \end{mdframed}}
\newcommand{\inputpython}[3]{\begin{mdframed}[backgroundcolor=codebg] \inputminted[breaklines=true,fontsize=#3,linenos=true,highlightcolor=fluo,tabsize=2,highlightlines={#2}]{python}{#1} \end{mdframed}}
\newcommand{\inputpartOCaml}[5]{\begin{mdframed}[backgroundcolor=codebg] \inputminted[breaklines=true,fontsize=#3,linenos=true,highlightcolor=fluo,tabsize=2,highlightlines={#2},firstline=#4,lastline=#5,firstnumber=1]{OCaml}{#1} \end{mdframed}}
\BeforeBeginEnvironment{minted}{\begin{mdframed}[backgroundcolor=codebg]}
\AfterEndEnvironment{minted}{\end{mdframed}}
\newcommand{\kw}[1]{\textcolor{blue}{\tt #1}}

\newtcolorbox{rcadre}[4]{halign=center,colback={#1},colframe={#2},width={#3cm},height={#4cm},valign=center,boxrule=1pt,left=0pt,right=0pt}
\newtcolorbox{cadre}[4]{halign=center,colback={#1},colframe={#2},arc=0mm,width={#3cm},height={#4cm},valign=center,boxrule=1pt,left=0pt,right=0pt}
\newcommand{\myem}[1]{\colorbox{fluo}{#1}}
\mdfsetup{skipabove=1pt,skipbelow=-2pt}



% Noeud dans un cadre pour les arbres
\newcommand{\noeud}[2]{\Tr{\fbox{\textcolor{#1}{\tt #2}}}}

\newcommand{\htmlmode}{\lstset{language=html,numbers=left, tabsize=4, frame=single, breaklines=true, keywordstyle=\ttfamily, basicstyle=\small,
   numberstyle=\tiny\ttfamily, framexleftmargin=0mm, backgroundcolor=\color{grispale}, xleftmargin=12mm,showstringspaces=false}}
\newcommand{\pythonmode}{\lstset{
   language=python,
   linewidth=\linewidth,
   numbers=left,
   tabsize=4,
   frame=single,
   breaklines=true,
   keywordstyle=\ttfamily\color{blue},
   basicstyle=\small,
   numberstyle=\tiny\ttfamily,
   framexleftmargin=-2mm,
   numbersep=-0.5mm,
   backgroundcolor=\color{codebg},
   xleftmargin=-1mm, 
   showstringspaces=false,
   commentstyle=\color{gray},
   stringstyle=\color{OliveGreen},
   emph={turtle,Screen,Turtle},
   emphstyle=\color{RawSienna},
   morekeywords={setheading,goto,backward,forward,left,right,pendown,penup,pensize,color,speed,hideturtle,showturtle,forward}}
   }
   \newcommand{\Cmode}{\lstset{
      language=[ANSI]C,
      linewidth=\linewidth,
      numbers=left,
      tabsize=4,
      frame=single,
      breaklines=true,
      keywordstyle=\ttfamily\color{blue},
      basicstyle=\small,
      numberstyle=\tiny\ttfamily,
      framexleftmargin=0mm,
      numbersep=2mm,
      backgroundcolor=\color{codebg},
      xleftmargin=0mm, 
      showstringspaces=false,
      commentstyle=\color{gray},
      stringstyle=\color{OliveGreen},
      emphstyle=\color{RawSienna},
      escapechar=\|,
      morekeywords={}}
      }
\newcommand{\bashmode}{\lstset{language=bash,numbers=left, tabsize=2, frame=single, breaklines=true, basicstyle=\ttfamily,
   numberstyle=\tiny\ttfamily, framexleftmargin=0mm, backgroundcolor=\color{grispale}, xleftmargin=12mm, showstringspaces=false}}
\newcommand{\exomode}{\lstset{language=python,numbers=left, tabsize=2, frame=single, breaklines=true, basicstyle=\ttfamily,
   numberstyle=\tiny\ttfamily, framexleftmargin=13mm, xleftmargin=12mm, basicstyle=\small, showstringspaces=false}}
   
   
  
%tei pour placer les images
%tei{nom de l’image}{échelle de l’image}{sens}{texte a positionner}
%sens ="1" (droite) ou "2" (gauche)
\newlength{\ltxt}
\newcommand{\tei}[4]{
\setlength{\ltxt}{\linewidth}
\setbox0=\hbox{\includegraphics[scale=#2]{#1}}
\addtolength{\ltxt}{-\wd0}
\addtolength{\ltxt}{-10pt}
\ifthenelse{\equal{#3}{1}}{
\begin{minipage}{\wd0}
\includegraphics[scale=#2]{#1}
\end{minipage}
\hfill
\begin{minipage}{\ltxt}
#4
\end{minipage}
}{
\begin{minipage}{\ltxt}
#4
\end{minipage}
\hfill
\begin{minipage}{\wd0}
\includegraphics[scale=#2]{#1}
\end{minipage}
}
}

%Juxtaposition d'une image pspciture et de texte 
%#1: = code pstricks de l'image
%#2: largeur de l'image
%#3: hauteur de l'image
%#4: Texte à écrire
\newcommand{\ptp}[4]{
\setlength{\ltxt}{\linewidth}
\addtolength{\ltxt}{-#2 cm}
\addtolength{\ltxt}{-0.1 cm}
\begin{minipage}[b][#3 cm][t]{\ltxt}
#4
\end{minipage}\hfill
\begin{minipage}[b][#3 cm][c]{#2 cm}
#1
\end{minipage}\par
}



%Macros pour les graphiques
\psset{linewidth=0.5\pslinewidth,PointSymbol=x}
\setlength{\fboxrule}{0.5pt}
\newcounter{tempangle}

%Marque la longueur du segment d'extrémité  #1 et  #2 avec la valeur #3, #4 est la distance par rapport au segment (en %age de la valeur de celui ci) et #5 l'orientation du marquage : +90 ou -90
\newcommand{\afflong}[5]{
\pstRotation[RotAngle=#4,PointSymbol=none,PointName=none]{#1}{#2}[X] 
\pstHomO[PointSymbol=none,PointName=none,HomCoef=#5]{#1}{X}[Y]
\pstTranslation[PointSymbol=none,PointName=none]{#1}{#2}{Y}[Z]
 \ncline{|<->|,linewidth=0.25\pslinewidth}{Y}{Z} \ncput*[nrot=:U]{\footnotesize{#3}}
}
\newcommand{\afflongb}[3]{
\ncline{|<->|,linewidth=0}{#1}{#2} \naput*[nrot=:U]{\footnotesize{#3}}
}

%Construis le point #4 situé à #2 cm du point #1 avant un angle #3 par rapport à l'horizontale. #5 = liste de paramètre
\newcommand{\lsegment}[5]{\pstGeonode[PointSymbol=none,PointName=none](0,0){O'}(#2,0){I'} \pstTranslation[PointSymbol=none,PointName=none]{O'}{I'}{#1}[J'] \pstRotation[RotAngle=#3,PointSymbol=x,#5]{#1}{J'}[#4]}
\newcommand{\tsegment}[5]{\pstGeonode[PointSymbol=none,PointName=none](0,0){O'}(#2,0){I'} \pstTranslation[PointSymbol=none,PointName=none]{O'}{I'}{#1}[J'] \pstRotation[RotAngle=#3,PointSymbol=x,#5]{#1}{J'}[#4] \pstLineAB{#4}{#1}}

%Construis le point #4 situé à #3 cm du point #1 et faisant un angle de  90° avec la droite (#1,#2) #5 = liste de paramètre
\newcommand{\psegment}[5]{
\pstGeonode[PointSymbol=none,PointName=none](0,0){O'}(#3,0){I'}
 \pstTranslation[PointSymbol=none,PointName=none]{O'}{I'}{#1}[J']
 \pstInterLC[PointSymbol=none,PointName=none]{#1}{#2}{#1}{J'}{M1}{M2} \pstRotation[RotAngle=-90,PointSymbol=x,#5]{#1}{M1}[#4]
  }
  
%Construis le point #4 situé à #3 cm du point #1 et faisant un angle de  #5° avec la droite (#1,#2) #6 = liste de paramètre
\newcommand{\mlogo}[6]{
\pstGeonode[PointSymbol=none,PointName=none](0,0){O'}(#3,0){I'}
 \pstTranslation[PointSymbol=none,PointName=none]{O'}{I'}{#1}[J']
 \pstInterLC[PointSymbol=none,PointName=none]{#1}{#2}{#1}{J'}{M1}{M2} \pstRotation[RotAngle=#5,PointSymbol=x,#6]{#1}{M2}[#4]
  }

% Construis un triangle avec #1=liste des 3 sommets séparés par des virgules, #2=liste des 3 longueurs séparés par des virgules, #3 et #4 : paramètre d'affichage des 2e et 3 points et #5 : inclinaison par rapport à l'horizontale
%autre macro identique mais sans tracer les segments joignant les sommets
\noexpandarg
\newcommand{\Triangleccc}[5]{
\StrBefore{#1}{,}[\pointA]
\StrBetween[1,2]{#1}{,}{,}[\pointB]
\StrBehind[2]{#1}{,}[\pointC]
\StrBefore{#2}{,}[\coteA]
\StrBetween[1,2]{#2}{,}{,}[\coteB]
\StrBehind[2]{#2}{,}[\coteC]
\tsegment{\pointA}{\coteA}{#5}{\pointB}{#3} 
\lsegment{\pointA}{\coteB}{0}{Z1}{PointSymbol=none, PointName=none}
\lsegment{\pointB}{\coteC}{0}{Z2}{PointSymbol=none, PointName=none}
\pstInterCC{\pointA}{Z1}{\pointB}{Z2}{\pointC}{Z3} 
\pstLineAB{\pointA}{\pointC} \pstLineAB{\pointB}{\pointC}
\pstSymO[PointName=\pointC,#4]{C}{C}[C]
}
\noexpandarg
\newcommand{\TrianglecccP}[5]{
\StrBefore{#1}{,}[\pointA]
\StrBetween[1,2]{#1}{,}{,}[\pointB]
\StrBehind[2]{#1}{,}[\pointC]
\StrBefore{#2}{,}[\coteA]
\StrBetween[1,2]{#2}{,}{,}[\coteB]
\StrBehind[2]{#2}{,}[\coteC]
\tsegment{\pointA}{\coteA}{#5}{\pointB}{#3} 
\lsegment{\pointA}{\coteB}{0}{Z1}{PointSymbol=none, PointName=none}
\lsegment{\pointB}{\coteC}{0}{Z2}{PointSymbol=none, PointName=none}
\pstInterCC[PointNameB=none,PointSymbolB=none,#4]{\pointA}{Z1}{\pointB}{Z2}{\pointC}{Z1} 
}


% Construis un triangle avec #1=liste des 3 sommets séparés par des virgules, #2=liste formée de 2 longueurs et d'un angle séparés par des virgules, #3 et #4 : paramètre d'affichage des 2e et 3 points et #5 : inclinaison par rapport à l'horizontale
%autre macro identique mais sans tracer les segments joignant les sommets
\newcommand{\Trianglecca}[5]{
\StrBefore{#1}{,}[\pointA]
\StrBetween[1,2]{#1}{,}{,}[\pointB]
\StrBehind[2]{#1}{,}[\pointC]
\StrBefore{#2}{,}[\coteA]
\StrBetween[1,2]{#2}{,}{,}[\coteB]
\StrBehind[2]{#2}{,}[\angleA]
\tsegment{\pointA}{\coteA}{#5}{\pointB}{#3} 
\setcounter{tempangle}{#5}
\addtocounter{tempangle}{\angleA}
\tsegment{\pointA}{\coteB}{\thetempangle}{\pointC}{#4}
\pstLineAB{\pointB}{\pointC}
}
\newcommand{\TriangleccaP}[5]{
\StrBefore{#1}{,}[\pointA]
\StrBetween[1,2]{#1}{,}{,}[\pointB]
\StrBehind[2]{#1}{,}[\pointC]
\StrBefore{#2}{,}[\coteA]
\StrBetween[1,2]{#2}{,}{,}[\coteB]
\StrBehind[2]{#2}{,}[\angleA]
\lsegment{\pointA}{\coteA}{#5}{\pointB}{#3} 
\setcounter{tempangle}{#5}
\addtocounter{tempangle}{\angleA}
\lsegment{\pointA}{\coteB}{\thetempangle}{\pointC}{#4}
}

% Construis un triangle avec #1=liste des 3 sommets séparés par des virgules, #2=liste formée de 1 longueurs et de deux angle séparés par des virgules, #3 et #4 : paramètre d'affichage des 2e et 3 points et #5 : inclinaison par rapport à l'horizontale
%autre macro identique mais sans tracer les segments joignant les sommets
\newcommand{\Trianglecaa}[5]{
\StrBefore{#1}{,}[\pointA]
\StrBetween[1,2]{#1}{,}{,}[\pointB]
\StrBehind[2]{#1}{,}[\pointC]
\StrBefore{#2}{,}[\coteA]
\StrBetween[1,2]{#2}{,}{,}[\angleA]
\StrBehind[2]{#2}{,}[\angleB]
\tsegment{\pointA}{\coteA}{#5}{\pointB}{#3} 
\setcounter{tempangle}{#5}
\addtocounter{tempangle}{\angleA}
\lsegment{\pointA}{1}{\thetempangle}{Z1}{PointSymbol=none, PointName=none}
\setcounter{tempangle}{#5}
\addtocounter{tempangle}{180}
\addtocounter{tempangle}{-\angleB}
\lsegment{\pointB}{1}{\thetempangle}{Z2}{PointSymbol=none, PointName=none}
\pstInterLL[#4]{\pointA}{Z1}{\pointB}{Z2}{\pointC}
\pstLineAB{\pointA}{\pointC}
\pstLineAB{\pointB}{\pointC}
}
\newcommand{\TrianglecaaP}[5]{
\StrBefore{#1}{,}[\pointA]
\StrBetween[1,2]{#1}{,}{,}[\pointB]
\StrBehind[2]{#1}{,}[\pointC]
\StrBefore{#2}{,}[\coteA]
\StrBetween[1,2]{#2}{,}{,}[\angleA]
\StrBehind[2]{#2}{,}[\angleB]
\lsegment{\pointA}{\coteA}{#5}{\pointB}{#3} 
\setcounter{tempangle}{#5}
\addtocounter{tempangle}{\angleA}
\lsegment{\pointA}{1}{\thetempangle}{Z1}{PointSymbol=none, PointName=none}
\setcounter{tempangle}{#5}
\addtocounter{tempangle}{180}
\addtocounter{tempangle}{-\angleB}
\lsegment{\pointB}{1}{\thetempangle}{Z2}{PointSymbol=none, PointName=none}
\pstInterLL[#4]{\pointA}{Z1}{\pointB}{Z2}{\pointC}
}

%Construction d'un cercle de centre #1 et de rayon #2 (en cm)
\newcommand{\Cercle}[2]{
\lsegment{#1}{#2}{0}{Z1}{PointSymbol=none, PointName=none}
\pstCircleOA{#1}{Z1}
}

%construction d'un parallélogramme #1 = liste des sommets, #2 = liste contenant les longueurs de 2 côtés consécutifs et leurs angles;  #3, #4 et #5 : paramètre d'affichage des sommets #6 inclinaison par rapport à l'horizontale 
% meme macro sans le tracé des segements
\newcommand{\Para}[6]{
\StrBefore{#1}{,}[\pointA]
\StrBetween[1,2]{#1}{,}{,}[\pointB]
\StrBetween[2,3]{#1}{,}{,}[\pointC]
\StrBehind[3]{#1}{,}[\pointD]
\StrBefore{#2}{,}[\longueur]
\StrBetween[1,2]{#2}{,}{,}[\largeur]
\StrBehind[2]{#2}{,}[\angle]
\tsegment{\pointA}{\longueur}{#6}{\pointB}{#3} 
\setcounter{tempangle}{#6}
\addtocounter{tempangle}{\angle}
\tsegment{\pointA}{\largeur}{\thetempangle}{\pointD}{#5}
\pstMiddleAB[PointName=none,PointSymbol=none]{\pointB}{\pointD}{Z1}
\pstSymO[#4]{Z1}{\pointA}[\pointC]
\pstLineAB{\pointB}{\pointC}
\pstLineAB{\pointC}{\pointD}
}
\newcommand{\ParaP}[6]{
\StrBefore{#1}{,}[\pointA]
\StrBetween[1,2]{#1}{,}{,}[\pointB]
\StrBetween[2,3]{#1}{,}{,}[\pointC]
\StrBehind[3]{#1}{,}[\pointD]
\StrBefore{#2}{,}[\longueur]
\StrBetween[1,2]{#2}{,}{,}[\largeur]
\StrBehind[2]{#2}{,}[\angle]
\lsegment{\pointA}{\longueur}{#6}{\pointB}{#3} 
\setcounter{tempangle}{#6}
\addtocounter{tempangle}{\angle}
\lsegment{\pointA}{\largeur}{\thetempangle}{\pointD}{#5}
\pstMiddleAB[PointName=none,PointSymbol=none]{\pointB}{\pointD}{Z1}
\pstSymO[#4]{Z1}{\pointA}[\pointC]
}


%construction d'un cerf-volant #1 = liste des sommets, #2 = liste contenant les longueurs de 2 côtés consécutifs et leurs angles;  #3, #4 et #5 : paramètre d'affichage des sommets #6 inclinaison par rapport à l'horizontale 
% meme macro sans le tracé des segements
\newcommand{\CerfVolant}[6]{
\StrBefore{#1}{,}[\pointA]
\StrBetween[1,2]{#1}{,}{,}[\pointB]
\StrBetween[2,3]{#1}{,}{,}[\pointC]
\StrBehind[3]{#1}{,}[\pointD]
\StrBefore{#2}{,}[\longueur]
\StrBetween[1,2]{#2}{,}{,}[\largeur]
\StrBehind[2]{#2}{,}[\angle]
\tsegment{\pointA}{\longueur}{#6}{\pointB}{#3} 
\setcounter{tempangle}{#6}
\addtocounter{tempangle}{\angle}
\tsegment{\pointA}{\largeur}{\thetempangle}{\pointD}{#5}
\pstOrtSym[#4]{\pointB}{\pointD}{\pointA}[\pointC]
\pstLineAB{\pointB}{\pointC}
\pstLineAB{\pointC}{\pointD}
}

%construction d'un quadrilatère quelconque #1 = liste des sommets, #2 = liste contenant les longueurs des 4 côtés et l'angle entre 2 cotés consécutifs  #3, #4 et #5 : paramètre d'affichage des sommets #6 inclinaison par rapport à l'horizontale 
% meme macro sans le tracé des segements
\newcommand{\Quadri}[6]{
\StrBefore{#1}{,}[\pointA]
\StrBetween[1,2]{#1}{,}{,}[\pointB]
\StrBetween[2,3]{#1}{,}{,}[\pointC]
\StrBehind[3]{#1}{,}[\pointD]
\StrBefore{#2}{,}[\coteA]
\StrBetween[1,2]{#2}{,}{,}[\coteB]
\StrBetween[2,3]{#2}{,}{,}[\coteC]
\StrBetween[3,4]{#2}{,}{,}[\coteD]
\StrBehind[4]{#2}{,}[\angle]
\tsegment{\pointA}{\coteA}{#6}{\pointB}{#3} 
\setcounter{tempangle}{#6}
\addtocounter{tempangle}{\angle}
\tsegment{\pointA}{\coteD}{\thetempangle}{\pointD}{#5}
\lsegment{\pointB}{\coteB}{0}{Z1}{PointSymbol=none, PointName=none}
\lsegment{\pointD}{\coteC}{0}{Z2}{PointSymbol=none, PointName=none}
\pstInterCC[PointNameA=none,PointSymbolA=none,#4]{\pointB}{Z1}{\pointD}{Z2}{Z3}{\pointC} 
\pstLineAB{\pointB}{\pointC}
\pstLineAB{\pointC}{\pointD}
}


% Définition des colonnes centrées ou à droite pour tabularx
\newcolumntype{Y}{>{\centering\arraybackslash}X}
\newcolumntype{Z}{>{\flushright\arraybackslash}X}

%Les pointillés à remplir par les élèves
\newcommand{\po}[1]{\makebox[#1 cm]{\dotfill}}
\newcommand{\lpo}[1][3]{%
\multido{}{#1}{\makebox[\linewidth]{\dotfill}
}}

%Liste des pictogrammes utilisés sur la fiche d'exercice ou d'activités
\newcommand{\bombe}{\faBomb}
\newcommand{\livre}{\faBook}
\newcommand{\calculatrice}{\faCalculator}
\newcommand{\oral}{\faCommentO}
\newcommand{\surfeuille}{\faEdit}
\newcommand{\ordinateur}{\faLaptop}
\newcommand{\ordi}{\faDesktop}
\newcommand{\ciseaux}{\faScissors}
\newcommand{\danger}{\faExclamationTriangle}
\newcommand{\out}{\faSignOut}
\newcommand{\cadeau}{\faGift}
\newcommand{\flash}{\faBolt}
\newcommand{\lumiere}{\faLightbulb}
\newcommand{\compas}{\dsmathematical}
\newcommand{\calcullitteral}{\faTimesCircleO}
\newcommand{\raisonnement}{\faCogs}
\newcommand{\recherche}{\faSearch}
\newcommand{\rappel}{\faHistory}
\newcommand{\video}{\faFilm}
\newcommand{\capacite}{\faPuzzlePiece}
\newcommand{\aide}{\faLifeRing}
\newcommand{\loin}{\faExternalLink}
\newcommand{\groupe}{\faUsers}
\newcommand{\bac}{\faGraduationCap}
\newcommand{\histoire}{\faUniversity}
\newcommand{\coeur}{\faSave}
\newcommand{\python}{\faPython}
\newcommand{\os}{\faMicrochip}
\newcommand{\rd}{\faCubes}
\newcommand{\data}{\faColumns}
\newcommand{\web}{\faCode}
\newcommand{\prog}{\faFile}
\newcommand{\algo}{\faCogs}
\newcommand{\important}{\faExclamationCircle}
\newcommand{\maths}{\faTimesCircle}
% Traitement des données en tables
\newcommand{\tables}{\faColumns}
% Types construits
\newcommand{\construits}{\faCubes}
% Type et valeurs de base
\newcommand{\debase}{{\footnotesize \faCube}}
% Systèmes d'exploitation
\newcommand{\linux}{\faLinux}
\newcommand{\sd}{\faProjectDiagram}
\newcommand{\bd}{\faDatabase}

%Les ensembles de nombres
\renewcommand{\N}{\mathbb{N}}
\newcommand{\D}{\mathbb{D}}
\newcommand{\Z}{\mathbb{Z}}
\newcommand{\Q}{\mathbb{Q}}
\newcommand{\R}{\mathbb{R}}
\newcommand{\C}{\mathbb{C}}

%Ecriture des vecteurs
\newcommand{\vect}[1]{\vbox{\halign{##\cr 
  \tiny\rightarrowfill\cr\noalign{\nointerlineskip\vskip1pt} 
  $#1\mskip2mu$\cr}}}


%Compteur activités/exos et question et mise en forme titre et questions
\newcounter{numact}
\setcounter{numact}{1}
\newcounter{numseance}
\setcounter{numseance}{1}
\newcounter{numexo}
\setcounter{numexo}{0}
\newcounter{numprojet}
\setcounter{numprojet}{0}
\newcounter{numquestion}
\newcommand{\espace}[1]{\rule[-1ex]{0pt}{#1 cm}}
\newcommand{\Quest}[3]{
\addtocounter{numquestion}{1}
\begin{tabularx}{\textwidth}{X|m{1cm}|}
\cline{2-2}
\textbf{\sffamily{\alph{numquestion})}} #1 & \dots / #2 \\
\hline 
\multicolumn{2}{|l|}{\espace{#3}} \\
\hline
\end{tabularx}
}
\newcommand{\QuestR}[3]{
\addtocounter{numquestion}{1}
\begin{tabularx}{\textwidth}{X|m{1cm}|}
\cline{2-2}
\textbf{\sffamily{\alph{numquestion})}} #1 & \dots / #2 \\
\hline 
\multicolumn{2}{|l|}{\cor{#3}} \\
\hline
\end{tabularx}
}
\newcommand{\Pre}{{\sc nsi} 1\textsuperscript{e}}
\newcommand{\Term}{{\sc nsi} Terminale}
\newcommand{\Sec}{2\textsuperscript{e}}
\newcommand{\Exo}[2]{ \addtocounter{numexo}{1} \ding{113} \textbf{\sffamily{Exercice \thenumexo}} : \textit{#1} \hfill #2  \setcounter{numquestion}{0}}
\newcommand{\Projet}[1]{ \addtocounter{numprojet}{1} \ding{118} \textbf{\sffamily{Projet \thenumprojet}} : \textit{#1}}
\newcommand{\ExoD}[2]{ \addtocounter{numexo}{1} \ding{113} \textbf{\sffamily{Exercice \thenumexo}}  \textit{(#1 pts)} \hfill #2  \setcounter{numquestion}{0}}
\newcommand{\ExoB}[2]{ \addtocounter{numexo}{1} \ding{113} \textbf{\sffamily{Exercice \thenumexo}}  \textit{(Bonus de +#1 pts maximum)} \hfill #2  \setcounter{numquestion}{0}}
\newcommand{\Act}[2]{ \ding{113} \textbf{\sffamily{Activité \thenumact}} : \textit{#1} \hfill #2  \addtocounter{numact}{1} \setcounter{numquestion}{0}}
\newcommand{\Seance}{ \rule{1.5cm}{0.5pt}\raisebox{-3pt}{\framebox[4cm]{\textbf{\sffamily{Séance \thenumseance}}}}\hrulefill  \\
  \addtocounter{numseance}{1}}
\newcommand{\Acti}[2]{ {\footnotesize \ding{117}} \textbf{\sffamily{Activité \thenumact}} : \textit{#1} \hfill #2  \addtocounter{numact}{1} \setcounter{numquestion}{0}}
\newcommand{\titre}[1]{\begin{Large}\textbf{\ding{118}}\end{Large} \begin{large}\textbf{ #1}\end{large} \vspace{0.2cm}}
\newcommand{\QListe}[1][0]{
\ifthenelse{#1=0}
{\begin{enumerate}[partopsep=0pt,topsep=0pt,parsep=0pt,itemsep=0pt,label=\textbf{\sffamily{\arabic*.}},series=question]}
{\begin{enumerate}[resume*=question]}}
\newcommand{\SQListe}[1][0]{
\ifthenelse{#1=0}
{\begin{enumerate}[partopsep=0pt,topsep=0pt,parsep=0pt,itemsep=0pt,label=\textbf{\sffamily{\alph*)}},series=squestion]}
{\begin{enumerate}[resume*=squestion]}}
\newcommand{\SQListeL}[1][0]{
\ifthenelse{#1=0}
{\begin{enumerate*}[partopsep=0pt,topsep=0pt,parsep=0pt,itemsep=0pt,label=\textbf{\sffamily{\alph*)}},series=squestion]}
{\begin{enumerate*}[resume*=squestion]}}
\newcommand{\FinListe}{\end{enumerate}}
\newcommand{\FinListeL}{\end{enumerate*}}

%Mise en forme de la correction
\newcommand{\cor}[1]{\par \textcolor{OliveGreen}{#1}}
\newcommand{\br}[1]{\cor{\textbf{#1}}}
\newcommand{\tcor}[1]{\begin{tcolorbox}[width=0.92\textwidth,colback={white},colbacktitle=white,coltitle=OliveGreen,colframe=green!75!black,boxrule=0.2mm]   
\cor{#1}
\end{tcolorbox}
}
\newcommand{\rc}[1]{\textcolor{OliveGreen}{#1}}
\newcommand{\pmc}[1]{\textcolor{blue}{\tt #1}}
\newcommand{\tmc}[1]{\textcolor{RawSienna}{\tt #1}}


%Référence aux exercices par leur numéro
\newcommand{\refexo}[1]{
\refstepcounter{numexo}
\addtocounter{numexo}{-1}
\label{#1}}

%Séparation entre deux activités
\newcommand{\separateur}{\begin{center}
\rule{1.5cm}{0.5pt}\raisebox{-3pt}{\ding{117}}\rule{1.5cm}{0.5pt}  \vspace{0.2cm}
\end{center}}

%Entête et pied de page
\newcommand{\snt}[1]{\lhead{\textbf{SNT -- La photographie numérique} \rhead{\textit{Lycée Nord}}}}
\newcommand{\Activites}[2]{\lhead{\textbf{{\sc #1}}}
\rhead{Activités -- \textbf{#2}}
\cfoot{}}
\newcommand{\Exos}[2]{\lhead{\textbf{Fiche d'exercices: {\sc #1}}}
\rhead{Niveau: \textbf{#2}}
\cfoot{}}
\newcommand{\Devoir}[2]{\lhead{\textbf{Devoir de mathématiques : {\sc #1}}}
\rhead{\textbf{#2}} \setlength{\fboxsep}{8pt}
\begin{center}
%Titre de la fiche
\fbox{\parbox[b][1cm][t]{0.3\textwidth}{Nom : \hfill \po{3} \par \vfill Prénom : \hfill \po{3}} } \hfill 
\fbox{\parbox[b][1cm][t]{0.6\textwidth}{Note : \po{1} / 20} }
\end{center} \cfoot{}}

%Devoir programmation en NSI (pas à rendre sur papier)
\newcommand{\PNSI}[2]{\lhead{\textbf{Devoir de {\sc nsi} : \textsf{ #1}}}
\rhead{\textbf{#2}} \setlength{\fboxsep}{8pt}
\begin{tcolorbox}[title=\textcolor{black}{\danger\; A lire attentivement},colbacktitle=lightgray]
{\begin{enumerate}
\item Rendre tous vous programmes en les envoyant par mail à l'adresse {\tt fnativel2@ac-reunion.fr}, en précisant bien dans le sujet vos noms et prénoms
\item Un programme qui fonctionne mal ou pas du tout peut rapporter des points
\item Les bonnes pratiques de programmation (clarté et lisiblité du code) rapportent des points
\end{enumerate}
}
\end{tcolorbox}
 \cfoot{}}


%Devoir de NSI
\newcommand{\DNSI}[2]{\lhead{\textbf{Devoir de {\sc nsi} : \textsf{ #1}}}
\rhead{\textbf{#2}} \setlength{\fboxsep}{8pt}
\begin{center}
%Titre de la fiche
\fbox{\parbox[b][1cm][t]{0.3\textwidth}{Nom : \hfill \po{3} \par \vfill Prénom : \hfill \po{3}} } \hfill 
\fbox{\parbox[b][1cm][t]{0.6\textwidth}{Note : \po{1} / 10} }
\end{center} \cfoot{}}

\newcommand{\DevoirNSI}[2]{\lhead{\textbf{Devoir de {\sc nsi} : {\sc #1}}}
\rhead{\textbf{#2}} \setlength{\fboxsep}{8pt}
\cfoot{}}

%La définition de la commande QCM pour auto-multiple-choice
%En premier argument le sujet du qcm, deuxième argument : la classe, 3e : la durée prévue et #4 : présence ou non de questions avec plusieurs bonnes réponses
\newcommand{\QCM}[4]{
{\large \textbf{\ding{52} QCM : #1}} -- Durée : \textbf{#3 min} \hfill {\large Note : \dots/10} 
\hrule \vspace{0.1cm}\namefield{}
Nom :  \textbf{\textbf{\nom{}}} \qquad \qquad Prénom :  \textbf{\prenom{}}  \hfill Classe: \textbf{#2}
\vspace{0.2cm}
\hrule  
\begin{itemize}[itemsep=0pt]
\item[-] \textit{Une bonne réponse vaut un point, une absence de réponse n'enlève pas de point. }
\item[\danger] \textit{Une mauvaise réponse enlève un point.}
\ifthenelse{#4=1}{\item[-] \textit{Les questions marquées du symbole \multiSymbole{} peuvent avoir plusieurs bonnes réponses possibles.}}{}
\end{itemize}
}
\newcommand{\DevoirC}[2]{
\renewcommand{\footrulewidth}{0.5pt}
\lhead{\textbf{Devoir de mathématiques : {\sc #1}}}
\rhead{\textbf{#2}} \setlength{\fboxsep}{8pt}
\fbox{\parbox[b][0.4cm][t]{0.955\textwidth}{Nom : \po{5} \hfill Prénom : \po{5} \hfill Classe: \textbf{1}\textsuperscript{$\dots$}} } 
\rfoot{\thepage} \cfoot{} \lfoot{Lycée Nord}}
\newcommand{\DevoirInfo}[2]{\lhead{\textbf{Evaluation : {\sc #1}}}
\rhead{\textbf{#2}} \setlength{\fboxsep}{8pt}
 \cfoot{}}
\newcommand{\DM}[2]{\lhead{\textbf{Devoir maison à rendre le #1}} \rhead{\textbf{#2}}}

%Macros permettant l'affichage des touches de la calculatrice
%Touches classiques : #1 = 0 fond blanc pour les nombres et #1= 1gris pour les opérations et entrer, second paramètre=contenu
%Si #2=1 touche arrondi avec fond gris
\newcommand{\TCalc}[2]{
\setlength{\fboxsep}{0.1pt}
\ifthenelse{#1=0}
{\psframebox[fillstyle=solid, fillcolor=white]{\parbox[c][0.25cm][c]{0.6cm}{\centering #2}}}
{\ifthenelse{#1=1}
{\psframebox[fillstyle=solid, fillcolor=lightgray]{\parbox[c][0.25cm][c]{0.6cm}{\centering #2}}}
{\psframebox[framearc=.5,fillstyle=solid, fillcolor=white]{\parbox[c][0.25cm][c]{0.6cm}{\centering #2}}}
}}
\newcommand{\Talpha}{\psdblframebox[fillstyle=solid, fillcolor=white]{\hspace{-0.05cm}\parbox[c][0.25cm][c]{0.65cm}{\centering \scriptsize{alpha}}} \;}
\newcommand{\Tsec}{\psdblframebox[fillstyle=solid, fillcolor=white]{\parbox[c][0.25cm][c]{0.6cm}{\centering \scriptsize 2nde}} \;}
\newcommand{\Tfx}{\psdblframebox[fillstyle=solid, fillcolor=white]{\parbox[c][0.25cm][c]{0.6cm}{\centering \scriptsize $f(x)$}} \;}
\newcommand{\Tvar}{\psframebox[framearc=.5,fillstyle=solid, fillcolor=white]{\hspace{-0.22cm} \parbox[c][0.25cm][c]{0.82cm}{$\scriptscriptstyle{X,T,\theta,n}$}}}
\newcommand{\Tgraphe}{\psdblframebox[fillstyle=solid, fillcolor=white]{\hspace{-0.08cm}\parbox[c][0.25cm][c]{0.68cm}{\centering \tiny{graphe}}} \;}
\newcommand{\Tfen}{\psdblframebox[fillstyle=solid, fillcolor=white]{\hspace{-0.08cm}\parbox[c][0.25cm][c]{0.68cm}{\centering \tiny{fenêtre}}} \;}
\newcommand{\Ttrace}{\psdblframebox[fillstyle=solid, fillcolor=white]{\parbox[c][0.25cm][c]{0.6cm}{\centering \scriptsize{trace}}} \;}

% Macroi pour l'affichage  d'un entier n dans  une base b
\newcommand{\base}[2]{ \overline{#1}^{#2}}
% Intervalle d'entiers
\newcommand{\intN}[2]{\llbracket #1; #2 \rrbracket}}

% Numéro et titre de chapitre
\setcounter{numchap}{3}
\newcommand{\Ctitle}{\cnum Modèle entité-association}

\makess{Rappel}
\begin{frame}{\Ctitle}{\stitle}
	\begin{block}{Limitation du modèle à une seule table}
		\begin{itemize}
			\item<1-> Pour le moment, nous avons manipulé des bases de données contenant une seule et unique table (dotée d'une \textcolor{blue}{clé primaire}) Ce modèle n'est pas pertinent et conduit à dupliquer l'information. Par exemple pour une base de données de livres, on stockerait sur chaque enregistrement les informations du livre, de l'auteur et de l'éditeur.
			\item<2-> Pour de multiples raisons (espace occupé, efficacité pour les recherches ou les modifications, \dots) une base de données est constituée d'un ensemble de tables liées entre elles.
			\item<3-> Le modèle \textcolor{blue}{entité-association} permet de concevoir des bases de données de façon efficace.
		\end{itemize}
	\end{block}
\end{frame}


\makess{Modèle entité-association}
\begin{frame}{\Ctitle}{\stitle}
	\begin{alertblock}{Définitions}
		\begin{itemize}
			\item<1-> Une \textcolor{blue}{entité} est une modélisation d'un objet concret ou abstrait à propos duquel on souhaite conserver des informations. Une entité doit pouvoir être identifiée de façon unique via un \textcolor{blue}{identifiant d'identité}.
				\onslide<2->\textcolor{gray}{Par exemple un livre (identifié par son {\sc isbn}), une facture (identifié par son code), un client (identifié par son email), un anniversaire (identifié par une personne et une date), une transaction commerciale (identifié par un code) \dots} \\
			\item<3-> Une entité possède un ou plusieurs \textcolor{blue}{attributs}.
				\onslide<4->\textcolor{gray}{Par exemple, l'entité \textit{film} peut avoir les attributs date, titre, année, \dots}
			\item<5-> Une \textcolor{blue}{instance} d'une entité est un objet en particulier.
				\onslide<6-> \textcolor{gray}{Par exemple, \textit{Forrest Gump} est une instance de l'entité \textit{Film}.}
			\item<7->  Une \textcolor{blue}{association} est un lien entre plusieurs entités. Le degré d'une association est le nombre d'entités intervenant dans l'association.
				\onslide<8->\textcolor{gray}{Par exemple, l'association \textit{écrit} de degré 2, relie l'entité \textit{auteur}  à l'entité \textit{livre}}
		\end{itemize}
	\end{alertblock}
\end{frame}

\makess{Modèle entité-association}
\begin{frame}{\Ctitle}{\stitle}
	\begin{alertblock}{Définitions}
		\begin{itemize}
			\item<2-> Pour les associations de degré 2 (binaire), on précise de chaque côté d'une association le nombre d'entités concernées. C'est la \textcolor{blue}{cardinalité} de l'association qui se résume à trois types principaux :
				\begin{itemize}
					\item<3-> \textcolor{blue}{1--1} association directe et exclusive entre deux entités (\textit{one to one}).
						\onslide<4->\textcolor{gray}{Par exemple, un \textit{lycée} a un \textit{proviseur}, un \textit{pays} a une seule \textit{capitale}.}
					\item<5-> \textcolor{blue}{1--*} (aussi noté \textcolor{blue}{1--1..N})  association d'une instance de la première entité à un ensemble d'instances de la seconde (\textit{one to many}).
						\onslide<6->\textcolor{gray}{Par exemple, un \textit{propriétaire} peut avoir plusieurs \textit{voitures}, un \textit{client} peut avoir plusieurs \textit{numéro de téléphone}.}
					\item<7-> \textcolor{blue}{*--*} (aussi noté \textcolor{blue}{1..N--1..N}) association d'un ensemble d'instances à un autre ensemble d'instance.
						\onslide<8->\textcolor{gray}{Par exemple, un \textit{livre} peut avoir plusieurs \textit{auteurs} et un \textit{auteur} peut écrire plusieurs \textit{livres}.}
				\end{itemize}
			\item<9-> Les associations de types \textcolor{blue}{*--*} peuvent être séparées entre deux associations de type \textcolor{blue}{1--*} à l'aide d'une nouvelle entité. \\
				\onslide<10->\textcolor{gray}{\small Par exemple, en créant une entité \textit{attribution}, un \textit{livre}  a plusieurs \textit{attributions} (car il a été écrit par plusieurs \textit{auteurs}) et un auteur à plusieurs \textit{attributions} (car il a écrit plusieurs livres)}
		\end{itemize}
	\end{alertblock}
\end{frame}


\begin{frame}{\Ctitle}{\stitle}
	\begin{exampleblock}{Exemples}
		\begin{itemize}
			\item<1-> Un exemple d'association \textit{one to one} :
				\begin{center}
					\renewcommand{\arraystretch}{1.1}
					\begin{tabular}{|p{2cm}|}
						\hline
						\multicolumn{1}{|c|}{\cellcolor{lightgray}{\small \textbf{Pays}}} \\
						\hline
						\cellcolor{white}\underline{\textbf {\footnotesize nom}}          \\
						\hline
						\cellcolor{white}{\footnotesize region}                           \\
						\hline
						\cellcolor{white}{\footnotesize population}                       \\
						\hline
						\cellcolor{white}{\footnotesize surface}                          \\
						\hline
					\end{tabular} \rnode{D}{$1$} \quad \quad \rnode{A}{$1$}
					\ncline[offset=-0.2cm,nodesep=-0.3cm,linewidth=1pt]{D}{A}
					\begin{tabular}{|p{2cm}|}
						\hline
						\multicolumn{1}{|c|}{\cellcolor{lightgray}{\small \textbf{Capitale}}} \\
						\hline
						\cellcolor{white}\underline{\textbf {\footnotesize nomcap}}              \\
						\hline
						\cellcolor{white}{\footnotesize longitude}                            \\
						\hline
						\cellcolor{white}{\footnotesize latitude}                             \\
						\hline
					\end{tabular}
				\end{center}
			\item<2-> Un exemple d'association \textit{one to many} :
				\begin{center}
					\renewcommand{\arraystretch}{1.1}
					\begin{tabular}{|p{2cm}|}
						\hline
						\multicolumn{1}{|c|}{\cellcolor{lightgray}{\small \textbf{Client}}} \\
						\hline
						\cellcolor{white}\underline{\textbf {\footnotesize num}}            \\
						\hline
						\cellcolor{white}{\footnotesize nom}                                \\
						\hline
						\cellcolor{white}{\small prenom}                                    \\
						\hline
						\cellcolor{white}{\footnotesize email}                              \\
						\hline
					\end{tabular} \rnode{D}{$1$} \quad \quad \rnode{A}{\phantom{1}$*$}
					\ncline[offset=-0.2cm,nodesepA=-0.3cm,nodesepB=-0.5cm,linewidth=1pt]{D}{A}
					\begin{tabular}{|p{2cm}|}
						\hline
						\multicolumn{1}{|c|}{\cellcolor{lightgray}{\small \textbf{Commande}}} \\
						\hline
						\cellcolor{white}\underline{\textbf {\footnotesize id}}               \\
						\hline
						\cellcolor{white}{\footnotesize prix}                                 \\
						\hline
						\cellcolor{white}{\footnotesize articles}                             \\
						\hline
						\cellcolor{white}{\footnotesize dates}                                \\
						\hline
					\end{tabular}
				\end{center}
		\end{itemize}
	\end{exampleblock}
\end{frame}


\begin{frame}{\Ctitle}{\stitle}
	\begin{exampleblock}{Exemples}
		\begin{itemize}
			\item<2-> Un exemple d'association \textit{many to many}:
				\begin{center}
					\renewcommand{\arraystretch}{1.1}
					\begin{tabular}{|p{2cm}|}
						\hline
						\multicolumn{1}{|c|}{\cellcolor{lightgray}{\small \textbf{Etudiant}}} \\
						\hline
						\cellcolor{white}\underline{\textbf {\footnotesize ine}}              \\
						\hline
						\cellcolor{white}{\footnotesize nom}                                  \\
						\hline
						\cellcolor{white}{\footnotesize prenom}                               \\
						\hline
						\cellcolor{white}{\footnotesize adresse}                              \\
						\hline
					\end{tabular} \rnode{D}{$*$} \quad \quad \rnode{A}{$*$}
					\ncline[offset=-0.22cm,nodesep=-0.3cm,linewidth=1pt]{D}{A}
					\begin{tabular}{|p{2cm}|}
						\hline
						\multicolumn{1}{|c|}{\cellcolor{lightgray}{\small \textbf{Cours}}} \\
						\hline
						\cellcolor{white}\underline{\textbf {\footnotesize id}}            \\
						\hline
						\cellcolor{white}{\footnotesize matière}                           \\
						\hline
						\cellcolor{white}{\footnotesize durée}                             \\
						\hline
						\cellcolor{white}{\footnotesize description}                       \\
						\hline
					\end{tabular}
				\end{center}
			\item<3-> Sa transformation en deux associations \textit{one to many} à l'aide d'une table de liaison
				\begin{center}
					\renewcommand{\arraystretch}{1.1}
					\begin{tabular}{|p{2cm}|}
						\hline
						\multicolumn{1}{|c|}{\cellcolor{lightgray}{\small \textbf{Etudiant}}} \\
						\hline
						\cellcolor{white}\underline{\textbf {\footnotesize ine}}              \\
						\hline
						\cellcolor{white}{\footnotesize nom}                                  \\
						\hline
						\cellcolor{white}{\footnotesize prenom}                               \\
						\hline
						\cellcolor{white}{\footnotesize adresse}                              \\
						\hline
					\end{tabular} \rnode{D}{$1$} \quad \quad \rnode{E}{\phantom{1}$*$}
					\ncline[offset=-0.22cm,nodesepA=-0.3cm,nodesepB=-0.45cm,linewidth=1pt]{D}{E}
					\begin{tabular}{|p{2cm}|}
						\hline
						\multicolumn{1}{|c|}{\cellcolor{lightgray}{\small \textbf{Inscription}}} \\
						\hline
						\cellcolor{white}\underline{\textbf {\footnotesize Etudiant}}            \\
						\hline
						\cellcolor{white}\underline{\textbf {\footnotesize Cours}}               \\
						\hline
					\end{tabular}
					\rnode{F}{\phantom{1}$*$} \quad \quad \rnode{A}{$1$}
					\ncline[offset=-0.22cm,nodesepA=-0.45cm,nodesepB=-0.3cm,linewidth=1pt]{F}{A}
					\begin{tabular}{|p{2cm}|}
						\hline
						\multicolumn{1}{|c|}{\cellcolor{lightgray}{\small \textbf{Cours}}} \\
						\hline
						\cellcolor{white}\underline{\textbf {\footnotesize id}}            \\
						\hline
						\cellcolor{white}{\footnotesize matière}                           \\
						\hline
						\cellcolor{white}{\footnotesize durée}                             \\
						\hline
						\cellcolor{white}{\footnotesize description}                       \\
						\hline
					\end{tabular}
				\end{center}
		\end{itemize}
	\end{exampleblock}
\end{frame}

\makess{Du modèle EA au modèle relationnel}
\begin{frame}{\Ctitle}{\stitle}
	\begin{block}{Méthode}
		Pour passer du modèle entité association au modèle relationnel :
		\begin{itemize}
			\item<1-> Une entité devient une relation (c'est-à-dire une table)
			\item<2-> L'identifiant d'identité devient la clé primaire de cette table
			\item<3-> On transforme les associations suivant les cas de figure
		\end{itemize}
	\end{block}
\end{frame}

\begin{frame}{\Ctitle}{\stitle}
	\begin{block}{Cas des associations \textit{one to one} : fusion}
		\onslide<1->{Deux entités associées en \textit{one to one} peuvent fusionner dans la même relation. \\}
		\onslide<2->\textcolor{gray}{Par exemple, les entités \textit{pays} et \textit{capitale} peuvent fusionner dans une seule table \textit{pays} en ajoutant dans cette table les attributs des capitales.}
		\onslide<3->
		\begin{center}
			\renewcommand{\arraystretch}{1.1}
			\begin{tabular}{|p{2cm}|}
				\hline
				\multicolumn{1}{|c|}{\cellcolor{lightgray}{\small \textbf{Pays}}} \\
				\hline
				\cellcolor{white}\underline{\textbf {\footnotesize nom}}          \\
				\hline
				\cellcolor{white}{\footnotesize region}                           \\
				\hline
				\cellcolor{white}{\footnotesize population}                       \\
				\hline
				\cellcolor{white}{\footnotesize surface}                          \\
				\hline
			\end{tabular} \rnode{D}{$1$} \quad \quad \rnode{A}{$1$}
			\ncline[offset=-0.2cm,nodesep=-0.3cm,linewidth=1pt]{D}{A}
			\begin{tabular}{|p{2cm}|}
				\hline
				\multicolumn{1}{|c|}{\cellcolor{lightgray}{\small \textbf{Capitale}}} \\
				\hline
				\cellcolor{white}\underline{\textbf {\footnotesize nomcap}}              \\
				\hline
				\cellcolor{white}{\footnotesize longitude}                            \\
				\hline
				\cellcolor{white}{\footnotesize latitude}                             \\
				\hline
			\end{tabular} \rnode{X}{} \quad \quad \rnode{Y}{}
			\ncline[doubleline=true,doublesep=2pt,doublecolor=blue,linecolor=blue,linewidth=1pt,arrowsize=10pt,arrowinset=0.2,arrowlength=1.2]{->}{X}{Y}
			\begin{tabular}{|p{2cm}>{\footnotesize \sc}r|}
				\hline
				\multicolumn{2}{|c|}{\cellcolor{lightgray}{\small \textbf{Pays}}}                   \\
				\hline
				\cellcolor{white}\underline{\textbf {\footnotesize nom}} & \cellcolor{white}{text}  \\
				\hline
				\cellcolor{white}{\footnotesize region}                  & \cellcolor{white}{text}  \\
				\hline
				\cellcolor{white}{\footnotesize population}              & \cellcolor{white}{int}   \\
				\hline
				\cellcolor{white}{\footnotesize surface}                 & \cellcolor{white}{float} \\
				\hline
				\cellcolor{white}{\footnotesize nomcap}                & \cellcolor{white}{text}  \\
				\hline
				\cellcolor{white}{\footnotesize longitude}               & \cellcolor{white}{float} \\
				\hline
				\cellcolor{white}{\footnotesize latitude}                & \cellcolor{white}{float} \\
				\hline
			\end{tabular}
		\end{center}
		\onslide<4->{
			On obtient alors le schéma relationnel suivant : \smallskip
			\begin{mdframed}
				\textbf{Pays} (\underline{nom}, region, population, surface, capitale, longitude, latitude)
			\end{mdframed}}
	\end{block}
\end{frame}


\begin{frame}{\Ctitle}{\stitle}
	\begin{block}{Cas des associations \textit{one to one} : clé étrangère}
		on peut aussi choisir de garder les deux entités séparées et donc dans deux relations différentes, on introduit alors le concept de \textcolor{blue}{clé étrangère} c'est-à-dire la clé primaire d'une autre table qui indique dans l'une des tables la référence vers l'autre\\
		\onslide<2->{\begin{center}
				\renewcommand{\arraystretch}{1.1}
				\begin{tabular}{|l|}
					\hline
					\multicolumn{1}{|c|}{\cellcolor{lightgray}{\small \textbf{Pays}}} \\
					\hline
					\cellcolor{white}\underline{\textbf {\footnotesize nom}}          \\
					\hline
					\cellcolor{white}{\footnotesize region}                           \\
					\hline
					\cellcolor{white}{\footnotesize population}                       \\
					\hline
					\cellcolor{white}{\footnotesize surface}                          \\
					\hline
				\end{tabular} \rnode{D}{$1$} \  \rnode{A}{$1$}
				\ncline[offset=-0.2cm,nodesep=-0.3cm,linewidth=1pt]{D}{A}
				\begin{tabular}{|l|}
					\hline
					\multicolumn{1}{|c|}{\cellcolor{lightgray}{\small \textbf{Capitale}}} \\
					\hline
					\cellcolor{white}\underline{\textbf {\footnotesize nom}}              \\
					\hline
					\cellcolor{white}{\footnotesize longitude}                            \\
					\hline
					\cellcolor{white}{\footnotesize latitude}                             \\
					\hline
				\end{tabular} \rnode{X}{} \quad \quad \rnode{Y}{}
				\ncline[doubleline=true,doublesep=2pt,doublecolor=blue,linecolor=blue,linewidth=1pt,arrowsize=10pt,arrowinset=0.2,arrowlength=1.2]{->}{X}{Y}
				\begin{tabular}{|l>{\footnotesize \sc}r|}
					\hline
					\multicolumn{2}{|c|}{\cellcolor{lightgray}{\small \textbf{Pays}}}                                      \\
					\hline
					\cellcolor{white}\underline{\textbf {\footnotesize nom}}        & \cellcolor{white}{text}              \\
					\hline
					\cellcolor{white}{\footnotesize region}                         & \cellcolor{white}{text}              \\
					\hline
					\cellcolor{white}{\footnotesize population}                     & \cellcolor{white}{int}               \\
					\hline
					\cellcolor{white}{\footnotesize surface}                        & \cellcolor{white}{float}             \\
					\hline
					\cellcolor{white}{\footnotesize \textcolor{Sepia}{\# capitale}} & \cellcolor{white}{\rnode{DCE}{text}} \\
					\hline
				\end{tabular} \quad \quad
				\begin{tabular}{|l>{\footnotesize \sc}r|}
					\hline
					\multicolumn{2}{|c|}{\cellcolor{lightgray}{\small \textbf{Capitale}}}                            \\
					\hline
					\cellcolor{white}\underline{\textbf {\footnotesize \rnode{ACE}{nom}}} & \cellcolor{white}{text}  \\
					\hline
					\cellcolor{white}{\footnotesize longitude}                            & \cellcolor{white}{float} \\
					\hline
					\cellcolor{white}{\footnotesize latitude}                             & \cellcolor{white}{float} \\
					\hline
				\end{tabular}
				\ncangle[nodesepA=0.1cm,angleA=0,angleB=180]{*->}{DCE}{ACE}
			\end{center}}
		\onslide<3->{
			On obtient alors le schéma relationnel suivant : \smallskip
			\begin{mdframed}
				\textbf{Pays} (\underline{nom}, region, population, surface, \#capitale) \\
				\textbf{Capitale} (\underline{nom}, longitude, latitude)
			\end{mdframed}}
		\onslide<4->{\textcolor{BrickRed}{\small \danger\;} \textcolor{BrickRed}{Intégrité référentielle} : un pays doit avoir une capitale !}
	\end{block}
\end{frame}

\begin{frame}{\Ctitle}{\stitle}
	\begin{block}{Cas des associations \textit{one to many}}
		On utilise là aussi la \textcolor{blue}{clé étrangère} de façon à ce qu'un élément du côté \textit{"many"} de l'association soit associé à un unique élément du côté \textit{"one"}. \\
		\onslide<2->{\begin{center}
				\renewcommand{\arraystretch}{1.1}
				\begin{tabular}{|l|}
					\hline
					\multicolumn{1}{|c|}{\cellcolor{lightgray}{\small \textbf{Client}}} \\
					\hline
					\cellcolor{white}\underline{\textbf {\footnotesize num}}            \\
					\hline
					\cellcolor{white}{\footnotesize nom}                                \\
					\hline
					\cellcolor{white}{\small prenom}                                    \\
					\hline
					\cellcolor{white}{\footnotesize email}                              \\
					\hline
				\end{tabular} \rnode{D}{$1$} \  \rnode{A}{\phantom{1}$*$}
				\ncline[offset=-0.2cm,nodesepA=-0.3cm,nodesepB=-0.5cm,linewidth=1pt]{D}{A}
				\begin{tabular}{|l|}
					\hline
					\multicolumn{1}{|c|}{\cellcolor{lightgray}{\small \textbf{Commande}}} \\
					\hline
					\cellcolor{white}\underline{\textbf {\footnotesize id}}               \\
					\hline
					\cellcolor{white}{\footnotesize prix}                                 \\
					\hline
					\cellcolor{white}{\footnotesize articles}                             \\
					\hline
					\cellcolor{white}{\footnotesize dates}                                \\
					\hline
				\end{tabular}
				\rnode{X}{} \quad \quad \rnode{Y}{}
				\ncline[doubleline=true,doublesep=2pt,doublecolor=blue,linecolor=blue,linewidth=1pt,arrowsize=10pt,arrowinset=0.2,arrowlength=1.2]{->}{X}{Y}
				\begin{tabular}{|l>{\footnotesize \sc}r|}
					\hline
					\multicolumn{2}{|c|}{\cellcolor{lightgray}{\small \textbf{Client}}}                             \\
					\hline
					\cellcolor{white}\underline{\textbf {\footnotesize \rnode{ACE}{num}}} & \cellcolor{white}{int}  \\
					\hline
					\cellcolor{white}{\footnotesize nom}                                  & \cellcolor{white}{text} \\
					\hline
					\cellcolor{white}{\footnotesize prenom}                               & \cellcolor{white}{text} \\
					\hline
					\cellcolor{white}{\footnotesize email}                                & \cellcolor{white}{text} \\
					\hline
				\end{tabular} \quad \quad
				\begin{tabular}{|l>{\footnotesize \sc}r|}
					\hline
					\multicolumn{2}{|c|}{\cellcolor{lightgray}{\small \textbf{Commande}}}                                     \\
					\hline
					\cellcolor{white}\underline{\textbf {\footnotesize id}}                        & \cellcolor{white}{int}   \\
					\hline
					\cellcolor{white}{\footnotesize \rnode{DCE}{\textcolor{Sepia}{\# id\_client}}} & \cellcolor{white}{int}   \\
					\hline
					\cellcolor{white}{\footnotesize prix}                                          & \cellcolor{white}{float} \\
					\hline
					\cellcolor{white}{\footnotesize articles}                                      & \cellcolor{white}{text}  \\
					\hline
					\cellcolor{white}{\footnotesize date}                                          & \cellcolor{white}{float} \\
					\hline
				\end{tabular}
				\ncangle[nodesepA=0.1cm,angleA=180,angleB=0,nodesepB=1.5cm]{*->}{DCE}{ACE}
			\end{center}}
		\onslide<4->{
			On obtient alors le schéma relationnel suivant : \smallskip
			\begin{mdframed}
				\textbf{Client} (\underline{num}, nom, prenom, email) \\
				\textbf{Commande} (\underline{id}, \# id\_client, prix, articles, date)
			\end{mdframed}}
	\end{block}
\end{frame}

\begin{frame}{\Ctitle}{\stitle}
	\begin{block}{Cas des associations \textit{many to many}}
		On crée trois tables : une pour chacune des entités et la table de liaison, celle-ci a pour clé primaire l'union des clés primaires des deux entités et est en liaison avec celles-ci en utilisant des clés étrangères.
		\onslide<2->{\begin{center}
				\renewcommand{\arraystretch}{1.1}
				\begin{tabular}{|p{2cm}|}
					\hline
					\multicolumn{1}{|c|}{\cellcolor{lightgray}{\small \textbf{Etudiant}}} \\
					\hline
					\cellcolor{white}\underline{\textbf {\footnotesize ine}}              \\
					\hline
					\cellcolor{white}{\footnotesize nom}                                  \\
					\hline
					\cellcolor{white}{\footnotesize prenom}                               \\
					\hline
					\cellcolor{white}{\footnotesize adresse}                              \\
					\hline
				\end{tabular} \rnode{D}{$1$} \quad  \rnode{E}{\phantom{1}$*$}
				\ncline[offset=-0.22cm,nodesepA=-0.3cm,nodesepB=-0.45cm,linewidth=1pt]{D}{E}
				\begin{tabular}{|p{2cm}|}
					\hline
					\multicolumn{1}{|c|}{\cellcolor{lightgray}{\small \textbf{Inscription}}} \\
					\hline
					\cellcolor{white}\underline{\textbf {\footnotesize Etudiant}}            \\
					\hline
					\cellcolor{white}\underline{\textbf {\footnotesize Cours}}               \\
					\hline
				\end{tabular}
				\rnode{F}{\phantom{1}$*$} \quad  \rnode{A}{$1$}
				\ncline[offset=-0.22cm,nodesepA=-0.45cm,nodesepB=-0.3cm,linewidth=1pt]{F}{A}
				\begin{tabular}{|p{2cm}|}
					\hline
					\multicolumn{1}{|c|}{\cellcolor{lightgray}{\small \textbf{Cours}}} \\
					\hline
					\cellcolor{white}\underline{\textbf {\footnotesize id}}            \\
					\hline
					\cellcolor{white}{\footnotesize matière}                           \\
					\hline
					\cellcolor{white}{\footnotesize durée}                             \\
					\hline
					\cellcolor{white}{\footnotesize description}                       \\
					\hline
				\end{tabular}
			\end{center} \vspace{0.4cm}}
		\onslide<3->{\begin{center}
				\begin{tabular}{|l>{\footnotesize \sc}r|}
					\hline
					\multicolumn{2}{|c|}{\cellcolor{lightgray}{\small \textbf{Etudiant}}}                          \\
					\hline
					\cellcolor{white}\underline{\textbf {\footnotesize \rnode{AE}{ine}}} & \cellcolor{white}{int}  \\
					\hline
					\cellcolor{white}{\footnotesize nom}                                 & \cellcolor{white}{text} \\
					\hline
					\cellcolor{white}{\footnotesize prenom}                              & \cellcolor{white}{text} \\
					\hline
					\cellcolor{white}{\footnotesize adresse}                             & \cellcolor{white}{text} \\
					\hline
				\end{tabular} \quad \quad \quad
				\begin{tabular}{|l>{\footnotesize \sc}r|}
					\hline
					\multicolumn{2}{|c|}{\cellcolor{lightgray}{\small \textbf{Inscription}}}                                                 \\
					\hline
					\cellcolor{white}\underline{\textbf {\footnotesize \textcolor{Sepia}{\rnode{DE}{\# Etudiant}}}} & \cellcolor{white}{int} \\
					\hline
					\cellcolor{white}\underline{\textbf {\footnotesize \textcolor{Sepia}{\rnode{DC}{\# Cours}}}}    & \cellcolor{white}{int} \\
					\hline
				\end{tabular} \quad \quad \quad
				\begin{tabular}{|l>{\footnotesize \sc}r|}
					\hline
					\multicolumn{2}{|c|}{\cellcolor{lightgray}{\small \textbf{Cours}}}                            \\
					\hline
					\cellcolor{white}\underline{\textbf {\footnotesize \rnode{AC}{id}}} & \cellcolor{white}{int}  \\
					\hline
					\cellcolor{white}{\footnotesize matière}                            & \cellcolor{white}{text} \\
					\hline
					\cellcolor{white}{\footnotesize durée}                              & \cellcolor{white}{int}  \\
					\hline
					\cellcolor{white}{\footnotesize description}                        & \cellcolor{white}{text} \\
					\hline
				\end{tabular}
				\ncangle[nodesepA=0.1cm,angleA=180,angleB=0,nodesepB=1.8cm]{*->}{DE}{AE}
				\ncangle[angleA=0,angleB=180,nodesepA=1.3cm,nodesepB=0.1cm]{*->}{DC}{AC}
			\end{center}}
	\end{block}
\end{frame}


\begin{frame}{\Ctitle}{\stitle}
	\begin{block}{Schéma relationnel}
		\begin{center}
			\begin{tabular}{|l>{\footnotesize \sc}r|}
				\hline
				\multicolumn{2}{|c|}{\cellcolor{lightgray}{\small \textbf{Etudiant}}}                          \\
				\hline
				\cellcolor{white}\underline{\textbf {\footnotesize \rnode{AE}{ine}}} & \cellcolor{white}{int}  \\
				\hline
				\cellcolor{white}{\footnotesize nom}                                 & \cellcolor{white}{text} \\
				\hline
				\cellcolor{white}{\footnotesize prenom}                              & \cellcolor{white}{text} \\
				\hline
				\cellcolor{white}{\footnotesize adresse}                             & \cellcolor{white}{text} \\
				\hline
			\end{tabular} \quad \quad \quad
			\begin{tabular}{|l>{\footnotesize \sc}r|}
				\hline
				\multicolumn{2}{|c|}{\cellcolor{lightgray}{\small \textbf{Inscription}}}                                                 \\
				\hline
				\cellcolor{white}\underline{\textbf {\footnotesize \textcolor{Sepia}{\rnode{DE}{\# Etudiant}}}} & \cellcolor{white}{int} \\
				\hline
				\cellcolor{white}\underline{\textbf {\footnotesize \textcolor{Sepia}{\rnode{DC}{\# Cours}}}}    & \cellcolor{white}{int} \\
				\hline
			\end{tabular} \quad \quad \quad
			\begin{tabular}{|l>{\footnotesize \sc}r|}
				\hline
				\multicolumn{2}{|c|}{\cellcolor{lightgray}{\small \textbf{Cours}}}                            \\
				\hline
				\cellcolor{white}\underline{\textbf {\footnotesize \rnode{AC}{id}}} & \cellcolor{white}{int}  \\
				\hline
				\cellcolor{white}{\footnotesize matière}                            & \cellcolor{white}{text} \\
				\hline
				\cellcolor{white}{\footnotesize durée}                              & \cellcolor{white}{int}  \\
				\hline
				\cellcolor{white}{\footnotesize description}                        & \cellcolor{white}{text} \\
				\hline
			\end{tabular}
			\ncangle[nodesepA=0.1cm,angleA=180,angleB=0,nodesepB=1.8cm]{*->}{DE}{AE}
			\ncangle[angleA=0,angleB=180,nodesepA=1.3cm,nodesepB=0.1cm]{*->}{DC}{AC}
		\end{center}
		\onslide<2->{
			On obtient alors le schéma relationnel suivant : \smallskip
			\begin{mdframed}
				\textbf{Etudiant} (\underline{ine}, nom, prenom, adresse) \\
				\textbf{Cours} (\underline{id}, matière, durée, description) \\
				\textbf{Inscription} (\underline{\# Etudiant}, \underline{\# cours})
			\end{mdframed}}
	\end{block}
\end{frame}

\begin{frame}{\Ctitle}{\stitle}
	\begin{exampleblock}{Exemple}
		\onslide<2->{On souhaite créer une base de données permettant de gérer les notes obtenus par des élèves dans des matières.
		\begin{itemize}
			\item[-]<3-> Les élèves ont les attributs suivants : nom, prénom, date de naissance, et identifiant unique.
			\item[-]<4-> Les matières ont les attributs suivants : nom (unique), horaire, coefficient
			\item[-]<5-> Chaque élève peut avoir plusieurs notes par matière.
		\end{itemize}
		\begin{itemize}
			\item<6-> Expliquer pourquoi un schéma relationnel d'une seule table notes n'est pas satisfaisant.
			\item<8-> Proposer un schéma relationnel constitué de 3 tables issu du modèle entité-association.
				}
		\end{itemize}
	\end{exampleblock}
\end{frame}


\makess{Requêtes sur plusieurs tables : opérations ensemblistes}
\begin{frame}{\Ctitle}{\stitle}
	\begin{block}{Union de deux tables}
		Lorsque deux tables $T_1$ et $T_2$ ont \textcolor{BrickRed}{le même schéma relationnel} (c'est-à-dire les même colonnes), $T_1 \cup T_2$ contient les enregistrement de $T_1$ ou $T_2$ (sans duplication).
		La syntaxe {\sc sql} correspondante est :
		\onslide<2->{\mintinline{sql}{SELECT * FROM T1 UNION SELECT * FROM T2;}}
	\end{block}
	\begin{exampleblock}{Exemple}
		\onslide<3->{\begin{center}
				\begin{tabular}{|>{\small}c|>{\small}c|>{\small}c|}
					\multicolumn{3}{c}{\textcolor{Sepia}{Table $T_1$}} \\
					\hline
					\rowcolor{lightgray} Id & Nom     & Prénom         \\
					\hline
					$7$                     & Payet   & Jean           \\
					\hline
					$28$                    & Hoarau  & Paul           \\
					\hline
					$42$                    & Untel   & Sam            \\
					\hline
					$57$                    & Casimir & Tom            \\
					\hline
				\end{tabular}
				\begin{tabular}{|>{\small}c|>{\small}c|>{\small}c|}
					\multicolumn{3}{c}{\textcolor{Sepia}{Table $T_2$}} \\
					\hline
					\rowcolor{lightgray} Id & Nom     & Prénom         \\
					\hline
					$12$                    & Martin  & Pierre         \\
					\hline
					$42$                    & Untel   & Sam            \\
					\hline
					$45$                    & Grondin & Eric           \\
					\hline
				\end{tabular} \quad
				\begin{tabular}{|>{\small}c|>{\small}c|>{\small}c|}
					\multicolumn{3}{c}{\textcolor{Sepia}{Table $T_1 \cup T_2$}} \\
					\hline
					\rowcolor{lightgray} Id & Nom     & Prénom                  \\
					\hline
					$7$                     & Payet   & Jean                    \\
					\hline
					$28$                    & Hoarau  & Paul                    \\
					\hline
					$42$                    & Untel   & Sam                     \\
					\hline
					$57$                    & Casimir & Tom                     \\
					\hline
					$12$                    & Martin  & Pierre                  \\
					\hline
					$45$                    & Grondin & Eric                    \\
					\hline
				\end{tabular}
			\end{center}}
	\end{exampleblock}
\end{frame}


\begin{frame}{\Ctitle}{\stitle}
	\begin{block}{Intersection de deux tables}
		Lorsque deux tables $T_1$ et $T_2$ ont \textcolor{BrickRed}{le même schéma relationnel} (c'est-à-dire les même colonnes), $T_1 \cap T_2$ contient les enregistrement apparaissant dans $T_1$ et dans $T_2$.
		\onslide<2->{La syntaxe {\sc sql} correspondante est : \\
			\mintinline{sql}{SELECT * FROM T1 INTERSECT SELECT * FROM T2 ;}}
	\end{block}
	\begin{exampleblock}{Exemple}
		\onslide<3->{\begin{center}
				\begin{tabular}{|>{\small}c|>{\small}c|>{\small}c|}
					\multicolumn{3}{c}{\textcolor{Sepia}{Table $T_1$}} \\
					\hline
					\rowcolor{lightgray} Id & Nom     & Prénom         \\
					\hline
					$7$                     & Payet   & Jean           \\
					\hline
					$28$                    & Hoarau  & Paul           \\
					\hline
					$42$                    & Untel   & Sam            \\
					\hline
					$57$                    & Casimir & Tom            \\
					\hline
				\end{tabular} \quad
				\begin{tabular}{|>{\small}c|>{\small}c|>{\small}c|}
					\multicolumn{3}{c}{\textcolor{Sepia}{Table $T_2$}} \\
					\hline
					\rowcolor{lightgray} Id & Nom     & Prénom         \\
					\hline
					$12$                    & Martin  & Pierre         \\
					\hline
					$42$                    & Untel   & Sam            \\
					\hline
					$45$                    & Grondin & Eric           \\
					\hline
				\end{tabular}
				\quad
				\begin{tabular}{|>{\small}c|>{\small}c|>{\small}c|}
					\multicolumn{3}{c}{\textcolor{Sepia}{Table $T_1 \cap T_2$}} \\
					\hline
					\rowcolor{lightgray} Id & Nom   & Prénom                    \\
					\hline
					$42$                    & Untel & Sam                       \\
					\hline
				\end{tabular}
			\end{center}}
	\end{exampleblock}
\end{frame}


\begin{frame}{\Ctitle}{\stitle}
	\begin{block}{Différence de deux tables}
		Lorsque deux tables $T_1$ et $T_2$ ont \textcolor{BrickRed}{le même schéma relationnel} (c'est-à-dire les même colonnes), $T_1 - T_2$ contient les enregistrement apparaissant dans $T_1$ et pas dans $T_2$.
		\onslide<2->{La syntaxe {\sc sql} correspondante est : \\
			\mintinline{sql}{SELECT * FROM T1 EXCEPT SELECT * FROM T2 ;}}
	\end{block}
	\begin{exampleblock}{Exemple}
		\onslide<3->{\begin{center}
				\begin{tabular}{|>{\small}c|>{\small}c|>{\small}c|}
					\multicolumn{3}{c}{\textcolor{Sepia}{Table $T_1$}} \\
					\hline
					\rowcolor{lightgray} Id & Nom     & Prénom         \\
					\hline
					$7$                     & Payet   & Jean           \\
					\hline
					$28$                    & Hoarau  & Paul           \\
					\hline
					$42$                    & Untel   & Sam            \\
					\hline
					$57$                    & Casimir & Tom            \\
					\hline
				\end{tabular}
				\begin{tabular}{|>{\small}c|>{\small}c|>{\small}c|}
					\multicolumn{3}{c}{\textcolor{Sepia}{Table $T_2$}} \\
					\hline
					\rowcolor{lightgray} Id & Nom     & Prénom         \\
					\hline
					$12$                    & Martin  & Pierre         \\
					\hline
					$42$                    & Untel   & Sam            \\
					\hline
					$45$                    & Grondin & Eric           \\
					\hline
				\end{tabular}
				\quad
				\begin{tabular}{|>{\small}c|>{\small}c|>{\small}c|}
					\multicolumn{3}{c}{\textcolor{Sepia}{Table $T_1 - T_2$}} \\
					\hline
					\rowcolor{lightgray} Id & Nom     & Prénom               \\
					\hline
					$7$                     & Payet   & Jean                 \\
					\hline
					$28$                    & Hoarau  & Paul                 \\
					\hline
					$57$                    & Casimir & Tom                  \\
					\hline
				\end{tabular}
			\end{center}}
	\end{exampleblock}
\end{frame}

\begin{frame}{\Ctitle}{\stitle}
	\begin{block}{Produit cartésien de deux tables}
		On peut réaliser le \textcolor{Sepia}{produit cartésien} de deux tables, c'est-à-dire l'ensemble des enregistrements formé d'un enregistrement de la première table et d'un enregistrement de la seconde. \\
		\onslide<2->{La syntaxe {\sc sql} correspondante est : \\ \mintinline{sql}{SELECT * FROM T1, T2 ;}}
	\end{block}
	\begin{exampleblock}{Exemple}
		\onslide<3->{
			\begin{tabular}{|>{\footnotesize}c|>{\footnotesize}c|>{\footnotesize}c|}
				\multicolumn{3}{c}{\textcolor{Sepia}{Table $T_1$}} \\
				\hline
				\rowcolor{lightgray} Id & Nom   & Prénom           \\
				\hline
				$7$                     & Payet & Jean             \\
				\hline
				$42$                    & Untel & Sam              \\
				\hline
			\end{tabular}
			\begin{tabular}{|>{\footnotesize}c|>{\footnotesize}c|}
				\multicolumn{2}{c}{\textcolor{Sepia}{Table $T_2$}} \\
				\hline
				\rowcolor{lightgray} Matière & Note                \\
				\hline
				Info                         & $15 $               \\
				\hline
				Maths                        & $9  $               \\
				\hline
				Physique                     & $10 $               \\
				\hline
			\end{tabular}
			\begin{tabular}{|>{\footnotesize}c|>{\footnotesize}c|>{\footnotesize}c|>{\footnotesize}c|>{\footnotesize}c|}
				\multicolumn{3}{c}{\textcolor{Sepia}{Table $T_1 \times T_2$}} \\
				\hline
				\rowcolor{lightgray} Id & Nom   & Prenom & Matière  & Note    \\
				\hline
				$7$                     & Payet & Jean   & Info     & $15 $   \\
				\hline
				$42$                    & Untel & Sam    & Info     & $15 $   \\
				\hline
				$7$                     & Payet & Jean   & Maths    & $9  $   \\
				\hline
				$42$                    & Untel & Sam    & Maths    & $9  $   \\
				\hline
				$7$                     & Payet & Jean   & Physique & $10 $   \\
				\hline
				$42$                    & Untel & Sam    & Physique & $10 $   \\
				\hline
			\end{tabular}}
	\end{exampleblock}
\end{frame}

\makess{Requêtes sur plusieurs tables : jointures}
\begin{frame}{\Ctitle}{\stitle}
	\begin{block}{Définition}
		\begin{itemize}
			\item<1-> La jointure de deux tables $T_1$ et $T_2$ sur les colonnes $A$ et $B$ revient à combiner les enregistrements de $T_1$ et $T_2$ lorsque les colonnes $A$ et $B$ coïncident.
			\item<2-> Cette jointure se note $T_1 \Join_{A=B} T_2$
			\item<3-> Cette notion est \textcolor{blue}{fortement liée} à celle  de clé étrangère, on effectuera souvent les jointures avec $A$ une clé primaire et $B$ une clé étrangère correspondante.
			\item<4-> La syntaxe {\sc sql} correspondante est : \\ \mintinline{sql}{SELECT * FROM T1 JOIN T2 on T1.A = T2.B}
			\item<5-> On peut joindre plus de deux tables.
		\end{itemize}
	\end{block}
\end{frame}


\begin{frame}{\Ctitle}{\stitle}
	\begin{exampleblock}{Exemple}
		\begin{center}
			\onslide<2->{\begin{tabular}{|>{\footnotesize}c|>{\footnotesize}c|>{\footnotesize}c|}
					\multicolumn{2}{c}{\textcolor{Sepia}{Table Auteurs}} \\
					\hline
					\rowcolor{lightgray} Id   & Prénom & Nom             \\
					\hline
					\textcolor{blue}{1}       & Isaac  & Asimov          \\
					\hline
					\textcolor{BrickRed}{4}   & Franck & Herbert         \\
					\hline
					\textcolor{OliveGreen}{7} & Jules  & Verne           \\
					\hline
				\end{tabular}} \quad \quad
			\onslide<3->{\begin{tabular}{|>{\footnotesize}c|>{\footnotesize}c|>{\footnotesize}c|}
					\multicolumn{2}{c}{\textcolor{Sepia}{Table Livres}}                      \\
					\hline
					\rowcolor{lightgray} Num & Auteur                    & Titre             \\
					\hline
					1                        & \textcolor{BrickRed}{4}   & Dune              \\
					\hline
					2                        & \textcolor{blue}{1}       & Les robots        \\
					\hline
					3                        & \textcolor{OliveGreen}{7} & L'île mystérieuse \\
					\hline
					4                        & \textcolor{blue}{1}       & Fondation         \\
					\hline
				\end{tabular}} \end{center}
		\onslide<4->{La jointure de \textcolor{Sepia}{Auteurs} et \textcolor{Sepia}{Livres} sur les colonnes Id et Auteur donne :}
		\onslide<5->{\begin{center}
			\begin{tabular}{|>{\columncolor{fluo} \footnotesize}c|>{\footnotesize}c|>{\footnotesize}c|>{\footnotesize}c|>{\columncolor{fluo}\footnotesize}c|>{\footnotesize}c|}
				\hline
				\rowcolor{lightgray} Id   & Prénom & Nom     & Num & Auteur                    & Titre             \\
				\hline
				\textcolor{blue}{1}       & Isaac  & Asimov  & 2   & \textcolor{blue}{1}       & Les robots        \\
				\hline
				\textcolor{blue}{1}       & Isaac  & Asimov  & 4   & \textcolor{blue}{1}       & Fondation         \\
				\hline
				\textcolor{BrickRed}{4}   & Franck & Herbert & 1   & \textcolor{BrickRed}{4}   & Dune              \\
				\hline
				\textcolor{OliveGreen}{7} & Jules  & Verne   & 3   & \textcolor{OliveGreen}{7} & L'île mystérieuse \\
				\hline
			\end{tabular}}\medskip \\
			\onslide<6->\mintinline{sql}{SELECT * FROM Auteurs JOIN Livres on Auteurs.ID = Livres.Auteur}
			\onslide<7->(On préfixe le nom des attributs par celui de la table afin d'éviter toute ambiguïté.)
		\end{center}
	\end{exampleblock}
\end{frame}


\makess{Requêtes imbriqués}
\begin{frame}{\Ctitle}{\stitle}
	\begin{block}{Requête dans le résultat d'une requête}
		\begin{itemize}
			\item<1-> Le résultat d'une requête peut-être utilisé afin d'effectuer une autre requête.
			\item<2-> C'est le principe des \textcolor{blue}{requêtes imbriquées}.
			\item<3-> C'est souvent dans un \mintinline{sql}{WHERE} ou un \mintinline{sql}{HAVING}
		\end{itemize}
	\end{block}
	\begin{exampleblock}{Exemple}
		\onslide<4->{Dans la relation \textit{Objet} (\underline{Référence} : {\sc int}, description : {\sc text}, prix : {\sc float}), comment retrouver le (ou les) objets ayant le prix le plus élevé ?}
		\begin{itemize}
			\item<5-> \mintinline{sql}{SELECT Référence, MAX(prix) FROM objet;} \\
				\onslide<5-> Cette solution n'est pas satisfaisante car elle renverra une seule référence même si plusieurs objets ont le prix maximal.
			\item<6-> \mintinline{sql}{SELECT Référence, prix FROM objet ORDER BY prix DESC LIMIT 1;}
				\onslide<6-> De même pour cette solution !
			\item<7-> \mintinline{sql}{SELECT Référence, prix FROM Objet } \\
				\mintinline{sql}{WHERE note = (SELECT MAX(prix) FROM objet);}
		\end{itemize}
	\end{exampleblock}
\end{frame}

\end{document}
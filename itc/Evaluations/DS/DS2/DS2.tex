\PassOptionsToPackage{dvipsnames,table}{xcolor}
\documentclass[11pt,a4paper]{article}

\usepackage{DS}

\begin{document}
\input{\detokenize{/home/fenarius/Travail/Cours/cpge-info/latex/Macros.tex}}
\ModeConcours


\DS{PC}{2}{Décembre 2023}
\newcommand{\SPATH}{/home/fenarius/Travail/Cours/cpge-info/docs/itc/Evaluations/DS/DS2}

\alertbox{\danger}{Remarques et consignes importantes}{
	\begin{itemize}
		\item[\textbullet] On pourra toujours librement utiliser une fonction demandée à une question précédente même si cette question n'a pas été traitée.
		\item[\textbullet] Veillez à présenter vos idées et vos réponses partielles même si vous ne trouvez pas la solution complète à une question.
		\item[\textbullet] La clarté et la lisibilité de la rédaction et des programmes sont des éléments de notation.
	\end{itemize}
}

\begin{Exercise}[title={Randonnée},origin={\bac \; d'après {\sc ccmp 2021 - pc, pc psi} (Partie 1)}]\\
	Lors de la préparation d'une randonnée, une accompagnatrice doit prendre en compte les exigences des participants. Elle dispose d'informations rassemblées dans deux tables d'une base de données :
	\begin{itemize}
		\item La table {\tt Rando} décrit les randonnées possibles : la clef primaire {\ttb rid}, son nom, le niveau de difficulté du parcours (entier entre 1 et 5), le dénivelé en mètres, la durée moyenne en minutes :
		      \begin{center}
			      \begin{tabular}{|l|l|l|l|l|}
				      \hline
				      \rowcolor{lightgray} \multicolumn{1}{|c|}{\ttb{\underline{rid}}} & \multicolumn{1}{|c|}{\textbf{\tt rnom}} & \multicolumn{1}{|c|}{\textbf{\tt diff}} & \multicolumn{1}{|c|}{\tt deniv} & \multicolumn{1}{|c|}{\textbf{\tt duree}} \\
				      \hline
				      1                                                                & La belle des champs                     & 1                                       & 20                              & 30                                       \\
				      \hline
				      2                                                                & Lac de Castellagne                      & 4                                       & 650                             & 150                                      \\
				      \hline
				      3                                                                & Le tour du mont                         & 2                                       & 200                             & 120                                      \\
				      \hline
				      4                                                                & Les crêtes de la mort                   & 5                                       & 1200                            & 360                                      \\
				      \hline
				      \dots                                                            & \dots                                   & \dots                                   & \dots                           & \dots                                    \\
				      \hline
			      \end{tabular}
		      \end{center}
		\item La table {\tt Participant} décrit les randonneurs : la clef promaire {\ttb pid}, le nom du randonneur, son année de naissance, le niveau de difficulté maximum de ses randonnées.
		      \begin{center}
			      \begin{tabular}{|l|l|l|l|}
				      \hline
				      \rowcolor{lightgray} \multicolumn{1}{|c|}{\ttb{\underline{pid}}} & \multicolumn{1}{|c|}{\textbf{\tt pnom}} & \multicolumn{1}{|c|}{\textbf{\tt ne}} & \multicolumn{1}{|c|}{\tt diff\_max} \\
				      \hline
				      1                                                                & Calvin                                  & 2014                                  & 2                                   \\
				      \hline
				      2                                                                & Hobbes                                  & 2015                                  & 2                                   \\
				      \hline
				      3                                                                & Susie                                   & 2014                                  & 2                                   \\
				      \hline
				      4                                                                & Rosalyn                                 & 2001                                  & 4                                   \\
				      \hline
				      \dots                                                            & \dots                                   & \dots                                 & \dots                               \\
				      \hline
			      \end{tabular}
		      \end{center}
	\end{itemize}
	Donner une requête {\sc sql} sur cette base pour :
	\Question{Compter le nombre de participants nés entre 1999 et 2003 inclus.}
	\Question{Calculer la durée moyenne des randonnées pour chaque niveau de difficulté.}
	\Question{Extraire le nom des participants pour lesquels la randonnée n°42 est trop difficile}
	\Question{Extraire les clés primaires des randonnées qui ont un ou des homonymes (nom identique et clé primaire distincte), sans redondance.}

\medskip 
\leftskip -\QuestionIndent
L'accompagnatrice a activé le suivi d'une randonnée par géolocalisation satellitaire et souhaite obtenir quelques propriétés de cette randonnée une fois celle-ci effectuée. Elle a exporté les données au format texte {\sc csv} (\textit{comma-separated-values} -- valeurs séparées par des virgules) dans un fichier nommé {\tt suivi\_rando.csv} : la première ligne annonce le format, les suivantes donnent les positions dans l'ordre chronologique.

Voici le début de ce fichier pour une randonnée partant de Valmorel, en Savoie, un bel après-midi d'été :

\begin{center}
	\begin{tabular}{>{\tt}l}
lat(°),long(°),height(m),time(s) \\
45.461516,6.44461,1315.221,1597496966 \\
45.461448,6.444426,1315.702,1597496970 \\
45.461383,6.444239,1316.182,1597496973  \\
45.461641,6.444035,1316.663,1597496979 \\
45.461534,6.443879,1317.144,1597496984 \\
45.461595,6.4437,1317.634,1597496989 \\
45.461562,6.443521,1318.105,1597496994 \\
... \\
	\end{tabular}
\end{center}

Le module {\tt math} de Python fournit les fonctions {\tt asin}, {\tt sin}, {\tt cos}, {\tt sqrt} et {\tt radians}. Cette dernière convertit des degrés en radians, unité des fonctions trigonométriques. La documentation donne aussi des éléments de manipulation de fichiers textuels :
\begin{itemize}
	\item {\tt fichier = open(NOM\_FICHIER, MODE)} ouvre le fichier, en lecture si {\tt MODE} est {\tt "r"}, en écriture si {\tt "w"}.
	\item {\tt ligne = fichier.readline()} récupère la ligne suivante de {\tt fichier} ouvert en lecture avec {\tt open}.
	\item {\tt lignes = fichier.readlines()} donne la liste des lignes suivantes.
	\item {\tt fichier.close()} ferme {\tt fichier}, ouvert avec {\tt open}, après son utilisation.
	\item {\tt ligne.split(SEP)} découpe la chaîne de caractères {\tt ligne} selon le séparateur {\tt SEP} : si ligne vaut {\tt "42,43,44"} alors {\tt ligne.split(",")} renvoie la liste {\tt ["42", "43", "44"]}.
\end{itemize}
On souhaite exploiter le fichier de suivi d'une randonnée -- supposé préalablement placé dans le répertoire de travail -- pour obtenir une liste {\tt coords} des listes de 4 flottants (latitude, longitude, altitude, temps) représentant les points de passage collectés lors de la randonnée.

À partir du canevas fourni en annexe, et en ajoutant les {\tt import} nécessaire :

\leftskip 0pt

\Question{Implémenter la fonction {\tt importe\_rando} qui réalise cette importation en renvoyant la liste souhaitée, par exemple en utilisant certaines des fonctions ci-dessus, ou une autre approche de votre choix.}

\leftskip -\QuestionIndent
On suppose maintenant l'importation effectuée dans {\tt coords}, avec au moins deux points d'instants distincts.

\leftskip 0pt
\Question{Implémenter la fonction {\tt plus\_haut} qui renvoie le liste (latitude, longitude) du point le plus haut de la randonnée.}

\Question{Implémenter la fonction {\tt deniveles} qui calcule les dénivelés cumulés positif et négatif en mètres de la randonnée, sous forme d'une liste de deux flottants. Le dénivelé positif est la somme des variations d'altitudes positives sur le chemin et inversement pour le dénivelé négatif.}

\medskip
\leftskip -\QuestionIndent
On souhaite évaluer de manière approchée la distance parcourue lors d'une randonnée. On suppose la Terre parfaitement sphérique de raron $R_T = 6371$ km au niveau de la mer. On utilise la formule de haversine pour calculer la distance $d$ du grand cercle sur une sphère de rayon $r$ entre deux points de coordonnées (latitude, longitude) $(\varphi_1, \lambda_1)$ et $(\varphi_2, \lambda_2)$

$$ d = 2\, r \arcsin \left(\sqrt{\sin^2\left( \frac{\varphi_2-\varphi_1}{2}\right) + \cos({\varphi_1})\cos({\varphi_2})\sin^2\left( \frac{\lambda_2-\lambda_1}{2}\right) }\right) $$

On prendra en compte l'altitude moyenne de l'arc, que l'on complètera pour la variation d'altitude par la formule de Pythagore, en assimilant la portion de ce cercle à un segment de droite perpendiculaire à la verticale.

\leftskip 0pt
\Question{Implémenter la fonction {\tt distance} qui calcule avec cette approche la distance en mètres entre deux points de passage. On décomposera obligatoirement les formules pour en améliorer la lisibilité.}

\Question{Implémenter la fonction {\tt distance\_totale} qui calcule la distance en mètres parcourue au cours d'une randonnée.}

\pagebreak

\begin{center}{\textbf{\large Annexe pour l'exercice 1 : canevas de codes Python}}\end{center}
\inputpartPython{canevas.py}{}{}{1}{27}

\end{Exercise}




\begin{Exercise}[title={Programmes divers et saut de taille maximale},origin={\bac \; d'après {\sc CAPES 2023} (Partie 1)}]

\medskip
\textbf{Notes de programmation }: Vous disposez pour répondre aux questions de cet exercice des fonctions Python de manipulation de listes  suivantes :
\begin{itemize}
	\item[\textbullet] On peut créer une liste de taille $n$ remplie avec la valeur $x$ avec {\tt li = [x] * n}
	\item[\textbullet] On peut obtenir la taille d'une liste {\tt li} avec {\tt len(li)}.
	\item[\textbullet] Si {\tt li} est une liste de $n$ éléments, on peut accéder au k-ème élément (pour $0 \leq k <$ {\tt len(li)}) avec { li[k]}. On peut définir sa valeur avec {li[k] = x}.
	\item[\textbullet] On peut concaténer deux listes {\tt li1} et {\tt li2} en utilisant l'opération {\tt li1 + li2}. On utilisera aussi cette opération dans des expressions mathématiques.
	\item[\textbullet] {\tt li[a:b]} désigne la liste des éléments d'indice compris entre $a$ et $b-1$ dans {\tt li}. On utilisera aussi cette opération dans des expressions mathématiques.
\end{itemize}

Les autres fonctions sur les listes ({\tt sort}, {\tt index}, {\tt max}, etc.) sont interdites à moins de les réécrire explicitement. L'opérateur {\tt in} d'appartenance à une liste est interdit, mais on peut utiliser ce mot-clé dans les autres contextes (par exemple dans une boucle {\tt for}).

\medskip
\textbf{Complexité} : Par \textit{complexité} d'un algorithme, on entend le nombre d'opérations élémentaires nécessaires à l'exécution de cet algorithme dans le pire cas. Lorsque cette complexité dépend d'un ou plusieurs paramètres $k_0, \dots, k_{r-1}$, on dit que la complexité est $O\left( f(k_0,\dots,k_{r-1}) \right)$ s'il existe une constante $C>0$ telle que, pour toutes les valeurs $k_0, \dots, k_{r-1}$ suffisamment grandes, ce nombre d'opérations élémentaires est majoré par $C \times f(k_0, \dots, k_{r-1})$.

\medskip
\ExePart[name = Programmes divers]

\Question{Ecrire une fonction {\tt fibonacci}} qui prend en argument un entier {\tt n} supérieur ou égal à 2 et renvoie la liste des {\tt n} premiers termes de la suite de Fibonacci $(F_n)_{n \in \N}$ définie par $F_0=0$, $F_1 = 1$ et $\forall n\geq 2, F_n = F_{n-1} + F_{n-2}$ (chaque terme est la somme des deux précédents).

\Question{Ecrire une fonction {\tt indice\_min} qui prend en argument une liste d'entiers {\tt li} et renvoie l'indice d'un de ses minimums.}

\Question{Que renverra {\tt indice\_min([1, 0, 2, 0])} avec votre programme ?}



\Question{Ecrire une fonction {\tt lettre\_majoritaire} qui prend en argument une chaîne de caractères non vide et renvoie le caractère qui apparait le plus fréquemment. Ainsi, {\tt lettre\_majoritaire('abcdedde')} devrait renvoyer {\tt 'd'}}.\\
\textit{Note : l'utilisation efficace d'un dictionnaire sera valorisée. On pourra alors utiliser l'opérateur {\tt in}}

\medskip
\ExePart[name = Saut de valeur maximale]
Dans une liste de flottants {\tt li}, on appelle \textit{saut} un couple $(i,j)$ avec $0 \leq i \leq j <$ {\tt len(li)  } et la \textit{valeur} d'un saut est la valeur {\tt li[j]-li[i]}. On va ici programmer plusieurs manières de trouver un saut de valeur maximale dans une liste. Par exemple, dans la liste {\tt [2.0, 0.2, 3.0, 5.3, 2.0]}, un tel saut est {\tt (1, 3)} (car {\tt 0.2} et {\tt 5.3} sont aux indices 1 et 3 respectivement).

\Question{Ecrire une fonction {\tt valeur} qui prend en argument une liste et un saut et renvoie la valeur de ce saut. Par exemple  {\tt valeur([2.0, 0.2, 3.0, 5.3, 2.0],(0,2))} renvoie {\tt 1.0} (car {\tt li[2]-li[0] = 1.0}).} 

\Question{Donner un exemple de liste avec exactement deux sauts de valeur maximale et préciser ces sauts.}

\Question{À l'aide d'un contre-exemple, montrer qu'on ne peut pas se contenter de chercher le minimum et le maximum d'une liste pour trouver un saut de valeur maximale.}

\Question{Écrire une fonction {\tt saut\_max\_naif} qui renvoie un saut de valeur maximale en testant tous les couples $(i,j)$ tels que $0 \leq i \leq j <$ {\tt len(li)}.\label{naif}}


\leftskip -\QuestionIndent
On décrit ici un algorithme utilisant le paradigme de la programmation dynamique pour résoudre ce problème : pour chaque $k$ entre 1 et {\tt len(li)}, on va calculer $m_k$ l'indice du minimum de {\tt li[0:k]}, et le couple $(i_k, j_k)$ un saut de valeur maximale dans {\tt li[0:k]}. Ainsi, on aura $m_1 = i_1 = j_1 = 0$ car {\tt li[0:1]} ne comporte qu'un seul élément.

\leftskip 0pt

\Question{ Pour $k <${\tt len(li)}, expliquer comment on peut calculer efficacement $m_{k+1}$ à partir de $m_k$ et des valeurs dans {\tt li}.}

\Question{Justifier que la relation suivante est correcte.\label{relation}}
$$ (i_{k+1}, j_{k+1}) = \left\{ 
	\begin{array}{l} 
		(i_k,j_k) \text{ si {\tt li[$k$]-li[$m_k$] $<$ li[$j_k$]-li[$i_k$]}} \\
		(m_k,k) \text{ sinon }
	\end{array}\right. $$

\Question{Ecrire une fonction {\tt saut\_max\_dynamique} qui prend en argument une liste {\tt li} et renvoie un saut de valeur maximale en utilisant la relation de la question \ref{relation}.}

\Question{Déterminer la complexité de votre programme dans le pire cas, puis comparer cette complexité avec celle du programme donnée en question \ref{naif}}

\end{Exercise}



\end{document}
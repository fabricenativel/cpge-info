%debug info

\setuppapersize[A4]
\setuplayout[
    width=middle,
    height=middle,
    topspace=1.5cm,
    backspace=2cm,
    leftmargin=0.9cm,
]
\setuplanguage[fr]

\setupheadertexts[Lycée Claude Fauriel][Informatique PC(*)]
\setupfootertexts[\pagenumber \, / \, \totalnumberofpages]
\setuppagenumbering[location=]
\setupwhitespace[medium]

\usecolors[x11]

\setuphead[section][style={\tfb \bf}]
\setuphead[subsection][style={\tfa \bf}]

\defineenumeration[question][text={Q.},alternative=inleft]

\usemodule[vim]
\definevimtyping[PYTHON][syntax=python]

\usemodule[tikz]
\usetikzlibrary[shapes]

\starttext

\midaligned{{\tfc \bf Corrigé}}
\section{Machines à café connectées}
\subsection{Requêtes sur la base de données}

\startquestion
{\tt SELECT COUNT(*) FROM clients;}
\stopquestion

\startquestion
{\tt SELECT id_client, AVG(note) FROM retours GROUP BY id_client;}
\stopquestion
\startquestion
{\tt SELECT id_client, AVG(note) FROM retours GROUP BY id_client HAVING AVG(note) <= 2;}
\stopquestion
\startquestion
{\tt SELECT nom, email, AVG(note) FROM clients JOIN retours ON id=id_client GROUP BY id_client HAVING AVG(note) <= 2;}
\stopquestion

\subsection{Interpolation des données par la méthode des~$k$ plus proches voisins}


\startquestion
\startPYTHON
def dist(p, q):
    return ((p[0] - q[0]) ** 2 + (p[1] - q[1]) ** 2) ** 0.5
\stopPYTHON
\stopquestion
\startquestion
\startPYTHON
def plus_proche(p, l):
    dmin = +float('inf')
    pp = None
    for x in l:
        d = dist(x, p)
        if d < dmin:
            dmin = d
            pp = x
    return pp
\stopPYTHON
\stopquestion


\startquestion
on écrit {\tt list(note.keys())} ou
\startPYTHON
l = []
for k in note:
    l.append(note[k])
\stopPYTHON
\stopquestion

\startquestion
  Écrire une fonction
  {\tt note_prédite_1NN(g: float, r: float, note: dict) -> int} fai\-sant
  intervenir les fonctions précédentes et mettant en œuvre la méthode
  des~$k$ plus proches voisins avec $k=1$.
\startPYTHON
def note_prédite_1NN(g, r, note):
    l = list(note.keys())
    pp = plus_proche((g, r), l)
    return note[pp]
\stopPYTHON
\stopquestion

\startquestion
Pour cette question le plus facile est d'extraire k fois le plus proche.
\startPYTHON
def plus_proches(p, l, k):
    assert(len(l) >= k)

    res = []
    for i in range(k):
        pp = plus_proche(p, l)
        l.remove(pp)
        res.append(pp)
    return res
\stopPYTHON
La complexité est alors en $O(nk)$. Une autre solution était de trier la liste de points par distance croissante à {\tt p}. La complexité était alors en $O(n \log n + k)$.
\stopquestion

\startquestion
\startPYTHON
def note_prédite_kNN(g, r, note, k):
    l = list(note.keys())
    pps = plus_proches((g,r), l, k)
    # On va maintenant compter les valeurs pour savoir la plus frequente
    compte = [0 for i in range(21)]
    for x in pps:
        compte[note[x]] += 1
    return argmin(compte)
\stopPYTHON
où {\tt argmin} est une fonction facile à écrire qui retourne l'indice de la valeur maximale d'une liste.
\stopquestion

\stoptext

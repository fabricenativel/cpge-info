% Auteur : Vincent Picard <vincent.picard1@ac-lyon.fr>
% Compiler avec la commande "context adn.tex"
% installation du module t-vim nécessaire

\setuppapersize[A4]
\setuplayout[
	width=middle,
	height=middle,
	topspace=1.5cm,
	backspace=2cm
]
\setuplanguage[fr]

\setupbodyfont[pagella, 12pt]

\setupheadertexts[Lycée Claude Fauriel][Informatique PC/PC*]
\setupfootertexts[\pagenumber \, / \, \totalnumberofpages]
\setuppagenumbering[location=]
\setupwhitespace[big]

\usecolors[x11]

\setuphead[section][
	style={\ss \bf \switchtobodyfont[14pt]},
	color=maroon,
]

% Definition de l'environnement question

\startuniqueMPgraphic{cadretitre}
path p; % cadre
path q; % cadre titre
z1 = (2mm, OverlayHeight - 7mm);
z2 = (0, OverlayHeight - 7mm);
z3 = (0, 0);
z4 = (OverlayWidth, 0);
z5 = (OverlayWidth, OverlayHeight - 7mm);
z6 = (2.9cm, OverlayHeight - 7mm);
p := z1 -- z2 -- z3 -- z4 -- z5 -- z6; 
%q := z1 shifted (5mm * up) -- z6 shifted(5mm * up) -- z6 shifted(5mm * down) -- z1 shifted (5mm * down) -- cycle;
%p := p randomized 0.1cm;
%q := q smoothed 0.4cm;
draw p withpen pencircle scaled 2pt withcolor \MPcolor{maroon};
%draw q withpen pencircle scaled 2pt withcolor \MPcolor{maroon};
setbounds currentpicture to boundingbox OverlayBox;
\stopuniqueMPgraphic
\defineoverlay[CadreTitre][\useMPgraphic{cadretitre}]

\defineframedtext[encadretitre]
	[frame=off,
	background=CadreTitre,
	width=fit,
	]
\defineenumeration[question]
	[text={Question},
	headcolor=maroon,
	before=\startencadretitre,
	after=\stopencadretitre,
	]

% Definition de l'environnement indications
\startuniqueMPgraphic{cadrepointille}
path p;
p := (0, 0) -- (OverlayWidth, 0) -- (OverlayWidth, OverlayHeight) -- (0, OverlayHeight) -- cycle;
p := p smoothed 0.5cm;
draw p withpen pencircle scaled 1pt dashed evenly;
setbounds currentpicture to boundingbox OverlayBox;
\stopuniqueMPgraphic
\defineoverlay[CadrePointille][\useMPgraphic{cadrepointille}]

\defineframedtext[encadrepointille][frame=off, background=CadrePointille, width=fit]

\defineenumeration[indication]
	[before=\startencadrepointille,
	 after=\stopencadrepointille,
	 text={Indications},
	 headstyle={\cap \bf},
	 number=no,
	 alternative=top,
	 ]

\usemodule[vim]
\defineframedtext[CadreCode]
	[
	frame=on,
	corner=0,
	rulethickness=1.5pt,
	framecolor=sienna,
	background=color,
	backgroundcolor=ivory,
	width=fit,
	]

\definevimtyping[PYTHON][syntax=python]

\starttext

\startmidaligned
{\tfb \bf \ss \color[maroon]{Jeu du puissance 4}}
\stopmidaligned

\blank[1cm]

{\bf \color[blue]{Thèmes} :} théorie des jeux, jeux d'accessibilité, algorithme min-max

\blank[1cm]

Le {\em puissance 4} est un jeu à deux joueurs, le joueur rouge (1) et le joueur jaune (2), se jouant sur une grille de 6 lignes et 7 colonnes. Chacun à son tour, les joueurs laissent tomber un jeton, en haut d'une des 7 colonnes, le jeton vient alors s'empiler en haut des jetons déjà présents dans la colonne. Le joueur qui parvient à créer un alignement de 4 jetons de sa couleur gagne la partie.

\startplacefigure[title={Jeu du puissance 4}, location=here]
\startmidaligned
{\externalfigure[puissance4.png][width=8cm]}
\stopmidaligned
\stopplacefigure


\section{Programmation du jeu}

On modélise la grille de jeu par une matrice d'entiers {\tt numpy} de 6 lignes et 7 colonnes. La ligne d'indice 0 sera la ligne située tout en haut de la grille de puissance 4; ainsi la ligne d'indice 5 est la ligne du bas.

Les entiers de la matrice représentent le contenu des cases selon le code suivant :
\startCadreCode
\startPYTHON
VIDE = 0
A = 1 // rouge
B = 2 // jaune
\stopPYTHON
\stopCadreCode

Un {\em coup} pour un joueur est un entier entre 0 et 6 représentant le numéro de colonne choisi.

\startquestion
\'Ecrire une fonction {\tt est_libre(grille, j)} retournant un booléen indiquant s'il est possible de jouer dans la colonne {\tt j} avec la {\tt grille} donnée.
\stopquestion

\startquestion
\'Ecrire une fonction {\tt coups_possibles(grille)} retournant la liste des coups qu'il est possible de jouer avec la {\tt grille} donnée.
\stopquestion


\startquestion
\'Ecrire une fonction {\tt jouer(grille, joueur, j)} qui modifie la {\tt grille} donnée en entrée, en faisant jouer le joueur {\tt joueur} en colonne {\tt j}. À l'aide d'une assertion, on pourra annoter en pré-condition que le coup proposé est jouable.
\stopquestion

On donne la fonction suivante :
\startCadreCode
\startPYTHON
def diagonale1(grille, joueur):
    for i in range(3):
        for j in range(4):
            if (grille[i][j] == joueur
                and grille[i+1][j+1] == joueur
                and grille[i+2][j+2] == joueur
                and grille[i+3][j+3] == joueur
                ):
                return True
    return False
\stopPYTHON
\stopCadreCode

\startquestion
Quel est le rôle de la fonction {\tt diagonale1} ? \'Ecrire de même les fonctions :
\startitemize[a, packed, nowhite]
\item {\tt ligne}
\item {\tt colonne}
\item {\tt diagonale2}
\stopitemize
\stopquestion

\startquestion
En déduire une fonction {\tt gagne(grille, joueur)} qui retourne {\tt True} si et seulement si dans la position {\tt grille} le joueur {\tt joueur} a gagné la partie.
\stopquestion

On pourra tester le jeu en humain contre humain en exécutant la dernière cellule du fichier fourni. Attention, le code de l'interface du jeu a été programmé pour vous, il n'est pas nécessaire de le modifier.

\section{Intelligence artificielle : algorithme min-max}

On souhaite maintenant avoir la possibilité de jouer au puissance 4 contre un {\em bot}. On programmera ce {\em bot} à l'aide de l'algorithme min-max vu en cours. On rappelle que celui-ci repose sur l'utilisation d'une fonction {\em heuristique} qui permet d'estimer les chances de gain de chaque joueur. On considérera qu'une heuristique positive signifie que le joueur A a l'avantage. L'heuristique vaudra $+\infty$ (resp. $-\infty$) lorsque la position est gagnée pour le joueur A (resp. le joueur B).

On donne la fonction heuristique basique suivante :

\startCadreCode
\startPYTHON
def h1(grille):
    if gagne(grille, A):
        return +float('inf')
    elif gagne(grille, B):
        return -float('inf')
    else:
        return 0
\stopPYTHON
\stopCadreCode

\startquestion
Compléter la fonction suivante qui retourne une évaluation de la {\tt grille}, dans la situation où c'est au tour du joueur {\tt joueur}. Cette évaluation reposera sur l'algorithme {\em min-max} utilisé avec la profondeur de recherche {\tt profondeur} et la fonction heuristique {\tt heuristique}.
%\startCadreCode
\startPYTHON
def eval(grille, joueur, profondeur, heuristique):
    coups = coups_possibles(grille)
    if profondeur==0 or est_final(grille):
        return (...)
    if joueur == A: # A veut maximiser
        max_eval = -float('inf')
        for c in coups:
            ng = grille.copy()
            jouer(ng, joueur, c)
            e = eval(ng, B, profondeur-1, heuristique)
            if e > max_eval:
                (...)
        return max_eval
    if joueur == B: # B veut minimiser
        (...)
\stopPYTHON
%\stopCadreCode
\stopquestion

\startquestion 
\'Ecrire une fonction
\blank
\startmidaligned
{\tt minmax(grille, joueur, profondeur, heuristique)}
\stopmidaligned
\blank
basée sur le code de la fonction {\tt eval} mais qui retourne cette fois-ci un couple {\tt (eval, best)} où {\tt eval} est l'évaluation de la position et {\tt best} est le meilleur coup à jouer pour le {\tt joueur}.
\stopquestion

On pourra tester le jeu contre la machine en assignant la valeur {\tt True} à la variable {\tt joueurIA}.
\blank
{\bf Pour aller plus loin...}
%\blank
Pour améliorer votre IA, vous pouvez définir une meilleure fonction heuristique. Par exemple, on pourra regarder chaque alignement de 4 cases et accorder 1 point si le joueur a placé 2 jetons et que les autres cases sont libres, 3 points s'il a placé 3 jetons.
\stoptext


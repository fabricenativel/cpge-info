\PassOptionsToPackage{dvipsnames,table}{xcolor}
\documentclass[11pt,a4paper]{article}

\usepackage{Act}
\begin{document}
\input{\detokenize{/home/fenarius/Travail/Cours/cpge-info/latex/Macros.tex}}
\ModeExercice
\setboolean{corrige}{false}
\ICX{Interro de cours}{1}{PCSI}{Récursivité}{10}\setcounter{Exercise}{0}

\begin{Exercise}[title={Minimum et récursivité}]
\Question{Donner la définition d'une fonction récursive}
\reponse{1}{1}
\Question{Ecrire une fonction \textit{itérative} {\tt minimum} qui prend en argument une liste {\tt lst} \textit{non vide} et renvoie le plus petit élément de cette liste. On vérifiera la précondition sur la liste à l'aide d'une instruction {\tt assert}.}
\reponse{5}{2}
\Question{Ecrire une fonction {\tt min2} qui renvoie le minimum des deux entiers données en argument, par exemple {\tt min2(5, -7)} renvoie {\tt -7}.}
\reponse{3}{1}
\Question{Exprimer le minimum d'une liste {\tt lst} contenant au moins deux éléments en fonction du minimum d'une liste plus petite (on pourra utiliser {\tt min2}).}
\reponse{0}{1}
\Question{En déduire une version récursive de la fonction minimum qui utilise {\tt min2}.}
\reponse{5}{2}
\end{Exercise}

\begin{Exercise}[title={Liste triée}]
\Question{Ecrire une fonction \textit{récursive} {\tt est\_croissante} qui prend en argument une liste et renvoie {\tt true} si et seulement si les élements de cette liste sont rangés dans l'ordre croissant}
\reponse{5}{3}
\end{Exercise}


\end{document}
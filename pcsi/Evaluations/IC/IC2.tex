\PassOptionsToPackage{dvipsnames,table}{xcolor}
\documentclass[11pt,a4paper]{article}

\usepackage{Act}

\begin{document}
\input{\detokenize{/home/fenarius/Travail/Cours/cpge-info/latex/Macros.tex}}
\ModeExercice
\setboolean{corrige}{false}
\ICX{Interro de cours}{2}{PCSI}{Représentation des entiers}{10}\setcounter{Exercise}{0}


\setcounter{Exercise}{0}
\begin{Exercise}[title={Conversions}]\\
	Commpléter le tableau de conversion suivant : \points{7}
	\renewcommand{\arraystretch}{2.2}
	\begin{center}
		\begin{tabularx}{0.8\textwidth}{|Y|Y|Y|}
			\hline
			Décimal                 & Binaire                            & Hexadécimal            \\
			\hline
			$\base{98}{10}$        & \comp{$\base{110\,0010}{2}$}       & \comp{$\base{62}{16}$}  \\
			\hline
            $\base{205}{10}$        & \comp{$\base{1100\,1101}{2}$}       & \comp{$\base{CD}{16}$}  \\
			\hline
            \comp{$\base{195}{10}$}  & $\base{1100\,0011}{2}$          & \comp{$\base{C3}{16}$} \\
			\hline
			\comp{$\base{327}{10}$}  & $\base{1\,0100\,0111}{2}$          & \comp{$\base{147}{16}$} \\
			\hline
			\comp{$\base{1068}{10}$} & \comp{$\base{100\,0010\,1100}{2}$}  & $\base{42C}{16}$       \\
			\hline
			\comp{$\base{912}{10}$}  & $\base{11\,1001\,0000}{2}$         & \comp{$\base{390}{16}$} \\
			\hline
			$\base{2654}{10}$       & \comp{$\base{1010\,0101\,1110}{2}$} & \comp{$\base{A5E}{16}$} \\
			\hline
		\end{tabularx}
	\end{center}
\end{Exercise}

\begin{Exercise}[title={Complément à deux}]\\
	Dans cet exercice, on suppose que les nombres entiers sont représentés en complément à deux sur 10 bits.
    \Question{Rappeler les trois étapes de la méthode vu en cours et qui permet d'obtenir la représentation en complément à deux d'un nombre entier négatif.}
    \newline \reponse{4}{1}
	\tcor{On calcule la représentation binaire de la valeur absolue du nombre sur sur le nombre de bits indiqué, on inverse tous les bits et on ajoute 1.}
	\Question{Donner la représentation de $\base{-421}{10}$}
	\newline \reponse{4}{1}
	\tcor{On calcule celle de $421$ \textit{sur 10 bits}, on inverse tous les bits, on ajoute 1, on obtient : $\base{10\,0101\,1011}{2}$}
	\Question{Donner la représentation de $\base{-59}{10}$}
	\newline \reponse{4}{1}
	\tcor{De la même façon, on obtient : $\base{11\,1100\,0101}{2}$}
\end{Exercise}
	



\end{document}
